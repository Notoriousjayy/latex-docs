
% =====================================================================
% Study Plan — Computational Geometry in C (User Story Template)
% =====================================================================
\documentclass[11pt,a4paper]{article}

% --- Encoding & layout ---
\usepackage[T1]{fontenc}
\usepackage[utf8]{inputenc}
\usepackage{lmodern}
\usepackage{microtype}
\usepackage[a4paper,margin=0.9in]{geometry}
\usepackage{parskip}

% --- Colors, links, boxes ---
\usepackage{xcolor}
\usepackage{amssymb}
\definecolor{CGBlue}{HTML}{0A84FF}
\definecolor{CGTeal}{HTML}{0FB5A9}
\definecolor{CGGray}{gray}{0.25}
\definecolor{PillBg}{HTML}{F2F4F7}
\definecolor{PillBorder}{HTML}{D0D5DD}

\usepackage{hyperref}
\hypersetup{colorlinks=true, linkcolor=CGBlue, urlcolor=CGBlue, citecolor=CGBlue}

\usepackage{enumitem}
\setlist{leftmargin=1.25em}

\usepackage{array}
\usepackage{booktabs}

\usepackage{amsmath,amssymb}
\usepackage[most]{tcolorbox}
\tcbuselibrary{skins, breakable}
\usepackage{tikz}

% --- Column preamble for meta table ---
% --- Key/Value rows without tabularx ---
\newcommand{\MetaRow}[2]{%
  \noindent\begin{minipage}[t]{0.24\linewidth}\raggedleft\bfseries #1\end{minipage}%
  \hspace{0.8em}%
  \begin{minipage}[t]{0.72\linewidth}\raggedright #2\end{minipage}\par\vspace{0.3em}}

% BDD rows reuse MetaRow for consistent layout

% --- Definition of the Story Card box with left gradient stripe ---
\tcbset{colback=white, colframe=black!15, boxrule=0.4pt, enhanced, breakable,
  sharp corners, drop fuzzy shadow, left=6mm,
  overlay={%
    \begin{tcbclipinterior}
      \shade[left color=CGTeal,right color=CGBlue]
        (frame.south west) rectangle ([xshift=3mm]frame.north west);
    \end{tcbclipinterior}}}

\newtcolorbox{StoryCard}[2][]{title={#2}, fonttitle=\bfseries, coltitle=black, #1}
\newtcolorbox{TaskBox}[1][]{title={Tasks}, fonttitle=\bfseries, #1}
% --- Inline label chips ---
\newtcbox{\Badge}{on line, enhanced, colback=CGBlue!10, colframe=CGBlue!60,
  boxrule=0.3pt, arc=2pt, boxsep=0.5ex, left=0.6ex, right=0.6ex, top=0.2ex, bottom=0.2ex,
  tcbox raise base}
\newtcbox{\Tag}{on line, enhanced, colback=black!5, colframe=black!30,
  boxrule=0.3pt, arc=2pt, boxsep=0.4ex, left=0.5ex, right=0.5ex, top=0.15ex, bottom=0.15ex,
  tcbox raise base}
% --- Simple checkbox ---
\newcommand{\CBox}{\(\square\)}
% --- BDD helper rows ---
\newcommand{\BDDRow}[2]{\MetaRow{#1}{#2}}
% --- Acceptance Criteria block (noop wrapper) ---
% --- Acceptance Criteria block ---
\newenvironment{BDD}{\par\medskip}{\par\medskip}

% --- Title helpers ---
\newcommand{\StoryTitle}[2]{\textbf{#1} --- #2}

% --- Footer lines (DoR/DoD) ---
\newcommand{\DoR}{\textit{Definition of Ready:} Persona clear; AC drafted; Dependencies known; Estimate set.}
\newcommand{\DoD}{\textit{Definition of Done:} All ACs pass; Tests green; Security/a11y checks; Docs updated; Deployed/flagged.}

\title{\textbf{Study Plan — Computational Geometry in C}\\\large User Story Template \& Examples}
\author{}
\date{}

\begin{document}
\maketitle
\tableofcontents
\bigskip

\section*{How to use this template}
Each chapter becomes a \emph{story card}. Fill the meta fields, keep the story in \emph{As a <persona>, I want <capability> so that <benefit>} form, and list verifiable \emph{Given/When/Then} acceptance criteria. Keep tasks small (10--40 minutes). Tags under \textbf{Non-Functional} are hints; add or remove as needed.

\clearpage

% ----------------------- Chapter 1 -----------------------------------
\begin{StoryCard}{\StoryTitle{CGC-1}{Triangulation Fundamentals}}
\MetaRow{Epic / Feature}{Core Geometry Primitives}
\MetaRow{Business Value}{Establish robust predicates and triangulation to unlock downstream algorithms (decomposition, shortest paths, meshing).}
\MetaRow{Priority / Estimate}{\Badge{Priority: Must}\ \ \Badge{SP: 3}}
\MetaRow{Persona}{Developer learning computational geometry in C}
\MetaRow{Dependencies}{C toolchain, unit test harness, simple SVG/PPM plotter}
\MetaRow{Assumptions / Risks}{Floating-point robustness; near-collinear points}
\MetaRow{Story}{\emph{As a developer, I want to triangulate simple polygons so that I can decompose shapes for further processing.}}

\medskip
\noindent\textbf{Non-Functional}\quad
\Tag{Performance}\ \Tag{Security}\ \Tag{Reliability}\ \Tag{Accessibility}\ \Tag{Privacy}\ \Tag{i18n}

\medskip
\noindent\textbf{Acceptance Criteria (BDD)}
\begin{BDD}
\BDDRow{Scenario}{Happy path}
\BDDRow{Given}{a valid simple polygon in CCW order}
\BDDRow{When}{I run \texttt{triangulate\_monotone()} on y-monotone input}
\BDDRow{Given}{adversarial inputs with collinear triples}
\BDDRow{When}{I run the intersection predicates}
\BDDRow{Then}{orientation and segment-intersection return correct results across fuzz tests}
\end{BDD}

\medskip
{\footnotesize \DoR\quad\textbullet\quad \DoD}
\end{StoryCard}

\begin{TaskBox}
\begin{itemize}
  \item \CBox\ Initialize repo; set up \texttt{clang}, \texttt{make}, and unit tests.
  \item \CBox\ Implement \texttt{orient()}, \texttt{on\_segment()}, \texttt{segments\_intersect()}.
  \item \CBox\ Implement \texttt{triangulate\_monotone()} and validator.
  \item \CBox\ Add fuzz tests with random polygons; export SVG for visual checks.
\end{itemize}
\end{TaskBox}

\clearpage

% ----------------------- Chapter 2 -----------------------------------
\begin{StoryCard}{\StoryTitle{CGC-2}{Polygon Partitioning}}
\MetaRow{Epic / Feature}{Decomposition}
\MetaRow{Business Value}{Partitioning enables linear-time triangulation per part and simpler downstream logic.}
\MetaRow{Priority / Estimate}{\Badge{Priority: Must}\ \ \Badge{SP: 3}}
\MetaRow{Persona}{Developer extending polygon ops}
\MetaRow{Dependencies}{CGC-1 predicates; sweep-line event queue}
\MetaRow{Assumptions / Risks}{Handling holes; event ordering ties}
\MetaRow{Story}{\emph{As a developer, I want to partition polygons into y-monotone pieces so that triangulation becomes straightforward.}}

\medskip
\noindent\textbf{Non-Functional}\quad
\Tag{Performance}\ \Tag{Reliability}\ \Tag{Testability}

\medskip
\noindent\textbf{Acceptance Criteria (BDD)}
\begin{BDD}
\BDDRow{Scenario}{Happy path}
\BDDRow{Given}{a simple polygon (possibly with holes)}
\BDDRow{When}{I run \texttt{partition\_to\_monotone()}}
\BDDRow{Then}{the union of parts equals the original polygon and all parts are y-monotone}
\end{BDD}

\medskip
{\footnotesize \DoR\quad\textbullet\quad \DoD}
\end{StoryCard}

\begin{TaskBox}
\begin{itemize}
  \item \CBox\ Implement vertical trapezoidalization via sweep-line.
  \item \CBox\ Emit monotone parts; triangulate each and stitch.
  \item \CBox\ Build regression tests with random polygons and known fixtures.
\end{itemize}
\end{TaskBox}

\clearpage

% ----------------------- Chapter 3 -----------------------------------
\begin{StoryCard}{\StoryTitle{CGC-3}{Convex Hulls (2D)}}
\MetaRow{Epic / Feature}{Extremal Geometry}
\MetaRow{Business Value}{Hulls support collision, diameter/width, and fast search.}
\MetaRow{Priority / Estimate}{\Badge{Priority: Must}\ \ \Badge{SP: 3}}
\MetaRow{Persona}{Algorithm engineer}
\MetaRow{Dependencies}{CGC-1 predicates}
\MetaRow{Assumptions / Risks}{Many duplicate or collinear points}
\MetaRow{Story}{\emph{As an engineer, I want a robust 2D convex hull so that extremal queries are reliable.}}

\medskip
\noindent\textbf{Non-Functional}\quad
\Tag{Performance}\ \Tag{Reliability}\ \Tag{Determinism}

\medskip
\noindent\textbf{Acceptance Criteria (BDD)}
\begin{BDD}
\BDDRow{Scenario}{Happy path}
\BDDRow{Given}{$n$ points in the plane}
\BDDRow{When}{I run \texttt{convex\_hull()} (monotone chain)}
\BDDRow{Then}{output is CCW hull with no collinear duplicates on edges and $O(n\log n)$ time observed}
\end{BDD}

\medskip
{\footnotesize \DoR\quad\textbullet\quad \DoD}
\end{StoryCard}

\begin{TaskBox}
\begin{itemize}
  \item \CBox\ Implement Graham scan and monotone chain; compare results.
  \item \CBox\ Add rotating-calipers: diameter, width, and support function.
  \item \CBox\ Benchmarks on random \& clustered inputs.
\end{itemize}
\end{TaskBox}

\clearpage

% ----------------------- Chapter 4 -----------------------------------
\begin{StoryCard}{\StoryTitle{CGC-4}{Convex Hulls (3D)}}
\MetaRow{Epic / Feature}{3D Structures}
\MetaRow{Business Value}{Enables mesh generation and 3D collision.}
\MetaRow{Priority / Estimate}{\Badge{Priority: Should}\ \ \Badge{SP: 5}}
\MetaRow{Persona}{3D developer}
\MetaRow{Dependencies}{CGC-3 (2D hull), half-edge/face structure}
\MetaRow{Assumptions / Risks}{General-position assumption; numerical tolerance}
\MetaRow{Story}{\emph{As a 3D developer, I want a 3D convex hull so that I can generate manifold meshes.}}

\medskip
\noindent\textbf{Non-Functional}\quad
\Tag{Reliability}\ \Tag{Testability}

\medskip
\noindent\textbf{Acceptance Criteria (BDD)}
\begin{BDD}
\BDDRow{Scenario}{Happy path}
\BDDRow{Given}{a set of 3D points}
\BDDRow{When}{I run \texttt{convex\_hull\_3d()} (incremental)}
\BDDRow{Then}{faces, edges, and vertices form a closed 2-manifold and satisfy Euler's formula}
\end{BDD}

\medskip
{\footnotesize \DoR\quad\textbullet\quad \DoD}
\end{StoryCard}

\begin{TaskBox}
\begin{itemize}
  \item \CBox\ Implement conflict graph and visibility checks.
  \item \CBox\ Export PLY/OBJ; add small viewer.
  \item \CBox\ Randomized incremental insertion; regression tests.
\end{itemize}
\end{TaskBox}

\clearpage

% ----------------------- Chapter 5 -----------------------------------
\begin{StoryCard}{\StoryTitle{CGC-5}{Voronoi \& Delaunay (2 weeks)}} 
\MetaRow{Epic / Feature}{Proximity Structures}
\MetaRow{Business Value}{Nearest-neighbor, meshing, interpolation, and path planning.}
\MetaRow{Priority / Estimate}{\Badge{Priority: Must}\ \ \Badge{SP: 8}}
\MetaRow{Persona}{Geometry practitioner}
\MetaRow{Dependencies}{CGC-3 (2D hull); robust circumcircle predicate}
\MetaRow{Assumptions / Risks}{Degenerate cocircular sets; numeric stability}
\MetaRow{Story}{\emph{As a practitioner, I want Delaunay triangulations and Voronoi diagrams so that I can solve proximity problems robustly.}}

\medskip
\noindent\textbf{Non-Functional}\quad
\Tag{Performance}\ \Tag{Reliability}\ \Tag{Visualization}

\medskip
\noindent\textbf{Acceptance Criteria (BDD)}
\begin{BDD}
\BDDRow{Scenario}{Happy path}
\BDDRow{Given}{a set of planar points}
\BDDRow{When}{I run \texttt{delaunay()} and derive Voronoi cells}
\BDDRow{Then}{duality holds; no triangle has an interior point inside its circumcircle; cells partition the plane}
\end{BDD}

\medskip
{\footnotesize \DoR\quad\textbullet\quad \DoD}
\end{StoryCard}

\begin{TaskBox}
\begin{itemize}
  \item \CBox\ Implement Bowyer--Watson; add edge-flip validator.
  \item \CBox\ Compute circumcenters; export Voronoi edges to SVG.
  \item \CBox\ Fuzz tests including grids and cocircular inputs.
\end{itemize}
\end{TaskBox}

\clearpage

% ----------------------- Chapter 6 -----------------------------------
\begin{StoryCard}{\StoryTitle{CGC-6}{Arrangements of Lines/Segments (2 weeks)}}
\MetaRow{Epic / Feature}{Combinatorial Plane Subdivision}
\MetaRow{Business Value}{Enables overlay, point location, and complex planar reasoning.}
\MetaRow{Priority / Estimate}{\Badge{Priority: Should}\ \ \Badge{SP: 8}}
\MetaRow{Persona}{Algorithms engineer}
\MetaRow{Dependencies}{DCEL structure; robust intersection}
\MetaRow{Assumptions / Risks}{Handling coincident and overlapping segments}
\MetaRow{Story}{\emph{As an engineer, I want an arrangement data structure so that I can query faces and overlay datasets.}}

\medskip
\noindent\textbf{Non-Functional}\quad
\Tag{Reliability}\ \Tag{Visualization}\ \Tag{Testability}

\medskip
\noindent\textbf{Acceptance Criteria (BDD)}
\begin{BDD}
\BDDRow{Scenario}{Happy path}
\BDDRow{Given}{a set of lines/segments}
\BDDRow{When}{I insert them incrementally}
\BDDRow{Then}{the DCEL remains consistent and point location answers face queries correctly}
\end{BDD}

\medskip
{\footnotesize \DoR\quad\textbullet\quad \DoD}
\end{StoryCard}

\begin{TaskBox}
\begin{itemize}
  \item \CBox\ Implement DCEL with split/merge operations.
  \item \CBox\ Incremental overlay; face-walk verification.
  \item \CBox\ Build point-location using trapezoidal map or DAG.
\end{itemize}
\end{TaskBox}

\clearpage

% ----------------------- Chapter 7 -----------------------------------
\begin{StoryCard}{\StoryTitle{CGC-7}{Search \& Intersection}} 
\MetaRow{Epic / Feature}{Query Primitives}
\MetaRow{Business Value}{Core for collision, CAD, GIS, and planning.}
\MetaRow{Priority / Estimate}{\Badge{Priority: Must}\ \ \Badge{SP: 5}}
\MetaRow{Persona}{Library author}
\MetaRow{Dependencies}{CGC-1 predicates; CGC-6 point location}
\MetaRow{Assumptions / Risks}{Degeneracies; performance on large inputs}
\MetaRow{Story}{\emph{As a library author, I want robust intersection and point-in-* tests so that downstream code is correct.}}

\medskip
\noindent\textbf{Non-Functional}\quad
\Tag{Performance}\ \Tag{Reliability}\ \Tag{Testability}

\medskip
\noindent\textbf{Acceptance Criteria (BDD)}
\begin{BDD}
\BDDRow{Scenario}{Happy path}
\BDDRow{Given}{convex polygons $A,B$}
\BDDRow{When}{I call \texttt{intersect\_convex(A,B)}}
\BDDRow{Then}{returns the exact intersection polygon or empty set and passes property tests}
\end{BDD}

\medskip
{\footnotesize \DoR\quad\textbullet\quad \DoD}
\end{StoryCard}

\begin{TaskBox}
\begin{itemize}
  \item \CBox\ Implement point-in-polygon (winding and ray-cast).
  \item \CBox\ Segment--segment and segment--triangle intersection.
  \item \CBox\ Convex polygon intersection via separating axis/rotating calipers.
\end{itemize}
\end{TaskBox}

\clearpage

% ----------------------- Chapter 8 -----------------------------------
\begin{StoryCard}{\StoryTitle{CGC-8}{Motion Planning Basics}}
\MetaRow{Epic / Feature}{Paths \& Clearance}
\MetaRow{Business Value}{Pathfinding for robots and games.}
\MetaRow{Priority / Estimate}{\Badge{Priority: Could}\ \ \Badge{SP: 5}}
\MetaRow{Persona}{Robotics/game developer}
\MetaRow{Dependencies}{CGC-7 intersections; visibility graph}
\MetaRow{Assumptions / Risks}{Narrow passages; numeric robustness}
\MetaRow{Story}{\emph{As a developer, I want shortest paths in polygonal environments so that agents can navigate safely.}}

\medskip
\noindent\textbf{Non-Functional}\quad
\Tag{Performance}\ \Tag{Reliability}\ \Tag{Visualization}

\medskip
\noindent\textbf{Acceptance Criteria (BDD)}
\begin{BDD}
\BDDRow{Scenario}{Happy path}
\BDDRow{Given}{start/goal inside a simple polygon}
\BDDRow{When}{I run the visibility-graph planner}
\BDDRow{Then}{it returns a collision-free shortest polyline path with vertex waypoints}
\end{BDD}

\medskip
{\footnotesize \DoR\quad\textbullet\quad \DoD}
\end{StoryCard}

\begin{TaskBox}
\begin{itemize}
  \item \CBox\ Build visibility graph from reflex vertices + endpoints.
  \item \CBox\ Shortest path via Dijkstra on the visibility graph.
  \item \CBox\ Optional: PRM with segment-collision queries.
\end{itemize}
\end{TaskBox}

\clearpage

% ----------------------- Chapter 9 -----------------------------------
\begin{StoryCard}{\StoryTitle{CGC-9}{Sources, Libraries \& Capstone}} 
\MetaRow{Epic / Feature}{Synthesis}
\MetaRow{Business Value}{Tie implementations to literature and production libraries.}
\MetaRow{Priority / Estimate}{\Badge{Priority: Should}\ \ \Badge{SP: 3}}
\MetaRow{Persona}{Research-minded engineer}
\MetaRow{Dependencies}{All previous chapters}
\MetaRow{Assumptions / Risks}{Overfitting benchmarks; scope creep}
\MetaRow{Story}{\emph{As an engineer, I want a small geometry toolkit and reference notes so that I can apply methods correctly.}}

\medskip
\noindent\textbf{Non-Functional}\quad
\Tag{Documentation}\ \Tag{Reproducibility}\ \Tag{Maintainability}

\medskip
\noindent\textbf{Acceptance Criteria (BDD)}
\begin{BDD}
\BDDRow{Scenario}{Happy path}
\BDDRow{Given}{the chapter modules}
\BDDRow{When}{I run the \texttt{cg} CLI on sample datasets}
\BDDRow{Then}{I obtain correct outputs with documented trade-offs and performance numbers}
\end{BDD}

\medskip
{\footnotesize \DoR\quad\textbullet\quad \DoD}
\end{StoryCard}

\begin{TaskBox}
\begin{itemize}
  \item \CBox\ Package modules into a small \texttt{cg} CLI with subcommands.
  \item \CBox\ Add benchmarks and a gallery of SVG outputs.
  \item \CBox\ Write a one-page ``When to use'' guide per algorithm.
\end{itemize}
\end{TaskBox}

\end{document}
