\documentclass[11pt]{article}

%% Build instructions:
%%   pdflatex -interaction=nonstopmode -halt-on-error Application_Security_Program_GHAS.tex
%%   (run twice to resolve the table of contents)

\usepackage[margin=1in]{geometry}
\usepackage[T1]{fontenc}
\usepackage[utf8]{inputenc}
\usepackage{lmodern}
\usepackage{microtype}

\usepackage{textcomp}
\usepackage{longtable}
\usepackage{booktabs}
\usepackage{array}
\usepackage{enumitem}

\usepackage{hyperref}
\usepackage{bookmark}

\usepackage{fancyhdr}
\usepackage{lastpage}

\setlength{\parindent}{0pt}
\setlength{\parskip}{0.6em}
\setlist[itemize]{noitemsep, topsep=0.25em}

\hypersetup{
  colorlinks=true,
  linkcolor=blue,
  urlcolor=blue,
  citecolor=blue,
  pdftitle={Application Security Program},
  pdfauthor={Application Security Program}
}

\setlength{\headheight}{14pt}
\pagestyle{fancy}
\fancyhf{}
\lhead{Application Security Program}
\rhead{v1.0}
\cfoot{\thepage\ of \pageref{LastPage}}

%% longtable spacing
\setlength{\LTpre}{0pt}
\setlength{\LTpost}{0pt}

\begin{document}
\begin{titlepage}
\centering
\vspace*{1.5cm}
{\LARGE\bfseries Application Security Program\par}
\vspace{0.4cm}
{\Large Powered by GitHub Advanced Security (GHAS)\par}
\vspace{0.9cm}
{\large\itshape Comprehensive Framework for Enterprise Security\par}
\vspace{0.25em}
{\large\itshape Integrating Security into the Developer Workflow\par}
\vspace{0.25em}
\vfill
{\large Version 1.0\par}
{\large December 17, 2025\par}
\end{titlepage}

\pagenumbering{roman}
\tableofcontents
\clearpage

\section*{Document Metadata}
\addcontentsline{toc}{section}{Document Metadata}

\begin{tabular}{@{}p{0.25\textwidth}p{0.70\textwidth}@{}}
\toprule
\textbf{Document Title} & Application Security Program \\
\textbf{Version} & 1.0 \\
\textbf{Effective Date} & December 17, 2025 \\
\textbf{Primary Tooling} & GitHub Advanced Security (GHAS) \\
\textbf{Purpose} & Comprehensive framework for integrating application security controls into the developer workflow. \\
\bottomrule
\end{tabular}

\clearpage
\pagenumbering{arabic}
\section{Executive Summary}

This Application Security Program establishes a comprehensive framework for protecting software applications throughout their lifecycle by leveraging GitHub Advanced Security (GHAS) as the foundational security tooling platform. The program integrates security directly into the developer workflow, enabling shift-left security practices while maintaining development velocity.

\subsection{Program Mission}

To systematically reduce application security risk across the enterprise by embedding GHAS-powered security into every phase of the software development lifecycle, fostering a security-conscious culture, and establishing measurable controls that protect organizational assets from application-layer threats.

\subsection{Strategic Objectives}

\begin{itemize}[leftmargin=*]
\item Shift-Left Security: Find and fix vulnerabilities early using CodeQL SAST, Secret Scanning, and Dependency Review
\item Developer Enablement: Provide security feedback in pull requests where developers already work
\item Supply Chain Protection: Secure third-party dependencies with Dependabot and SCA capabilities
\item Automated Governance: Enforce security policies through CI/CD pipeline integration
\item Measurable Progress: Track security posture with Security Overview dashboards
\end{itemize}
\subsection{GHAS Value Proposition}

GitHub Advanced Security provides native security capabilities integrated directly into GitHub's developer platform, eliminating the friction of external security tools and enabling security-at-speed for modern development teams.

\section{GitHub Advanced Security Components}

GHAS delivers three core security capabilities that form the foundation of this program, available as two purchasable products for private repositories: GitHub Code Security and GitHub Secret Protection.

\subsection{Code Scanning (CodeQL SAST)}

CodeQL is a powerful semantic code analysis engine that queries code as data to find security vulnerabilities and code quality issues.

\begin{itemize}[leftmargin=*]
\item Analyzes source code for vulnerabilities without execution
\item Supports 10+ languages including JavaScript, Python, Java, C/C++, Go, Ruby
\item Runs automatically on commits, pull requests, and scheduled scans
\item Findings displayed inline in PRs with severity, CWE, and remediation guidance
\item Customizable query suites for organization-specific threat models
\end{itemize}
\subsection{Secret Scanning}

Detects hardcoded credentials, API keys, tokens, and other secrets in code and git history before they can be exploited.

\begin{itemize}[leftmargin=*]
\item Partner patterns detect secrets from major cloud providers and SaaS vendors
\item Custom patterns enable organization-specific secret detection
\item Push Protection blocks commits containing secrets before they enter the repository
\item Historical scanning detects secrets already in git history
\item Alerts integrate with incident response workflows
\end{itemize}
\subsection{Dependency Review \& Dependabot (SCA)}

Software Composition Analysis capabilities secure the software supply chain by identifying and remediating vulnerable dependencies.

\begin{itemize}[leftmargin=*]
\item Dependency Review: Shows vulnerability impact of dependency changes at PR time
\item Dependabot Alerts: Notifies about known vulnerabilities (CVEs) in dependencies
\item Dependabot Security Updates: Creates automated PRs to remediate vulnerable dependencies
\item Dependabot Version Updates: Keeps dependencies current to reduce exposure window
\item Supports all major package ecosystems: npm, pip, Maven, NuGet, etc.
\end{itemize}
\subsection{Security Overview Dashboard}

Provides organization-wide visibility into security posture with aggregated metrics, filtering by repository/severity/ecosystem, and trend analysis for leadership reporting and audit evidence.

\section{Governance Framework}

\subsection{Organizational Structure}

The Application Security Program operates under a tiered governance model with clear accountability and decision-making authority.

\begin{longtable}{p{0.20\textwidth}p{0.46\textwidth}p{0.30\textwidth}}
\caption{Organizational Roles and Responsibilities}\label{tab:org_roles}\\
\toprule
\textbf{Role} & \textbf{Responsibilities} & \textbf{GHAS Focus} \\
\midrule
\endfirsthead
\toprule
\textbf{Role} & \textbf{Responsibilities} & \textbf{GHAS Focus} \\
\midrule
\endhead
\bottomrule
\endfoot
Executive Sponsor & Program authority, budget, risk acceptance & Security Overview review, policy approval \\
AppSec Program Lead & Strategy, operations, team leadership & GHAS configuration, rollout, metrics \\
Security Champions & Team liaison, first-line triage & Alert triage, developer coaching \\
Development Teams & Secure coding, remediation & Fix alerts in PRs, merge Dependabot PRs \\
\end{longtable}

\subsection{Application Risk Classification}

Applications are classified into risk tiers to enable appropriate security investment based on business criticality and data sensitivity.

\begin{longtable}{p{0.20\textwidth}p{0.44\textwidth}p{0.32\textwidth}}
\caption{Application Risk Classification Tiers}\label{tab:risk_tiers}\\
\toprule
\textbf{Tier} & \textbf{Criteria} & \textbf{GHAS Requirements} \\
\midrule
\endfirsthead
\toprule
\textbf{Tier} & \textbf{Criteria} & \textbf{GHAS Requirements} \\
\midrule
\endhead
\bottomrule
\endfoot
Tier 1 (Critical) & Customer-facing, PII/PCI data, regulatory scope, high revenue impact & All GHAS features enabled, blocking gates, mandatory PR checks \\
Tier 2 (High) & Internal sensitive data, moderate business impact, partner integrations & All GHAS features, advisory gates, critical findings block \\
Tier 3 (Medium) & Internal tools, limited data access, low business impact & CodeQL default, Dependabot alerts, informational gates \\
Tier 4 (Low) & Test/dev environments, no sensitive data, minimal impact & Basic scanning, no blocking gates, periodic review \\
\end{longtable}

\subsection{Core Security Policies}

\begin{itemize}[leftmargin=*]
\item GHAS Enablement Policy: All repositories must have appropriate GHAS features enabled based on tier
\item Branch Protection Policy: Protected branches require passing GHAS checks before merge
\item Vulnerability Remediation Policy: SLAs based on severity (Critical: 7 days, High: 30 days, Medium: 90 days)
\item Secret Exposure Policy: Push Protection enabled; exposed secrets require immediate rotation
\item Dependency Management Policy: Dependabot PRs must be reviewed within 14 days
\end{itemize}
\section{CI/CD Pipeline Integration (16-Gate Model)}

GHAS integrates security checks throughout the CI/CD pipeline, serving as the enforcement mechanism for multiple quality and security gates. This section maps GHAS capabilities to a comprehensive 16-gate pipeline model.

\subsection{GHAS Gate Mapping}

\begin{longtable}{p{0.08\textwidth}p{0.24\textwidth}p{0.28\textwidth}p{0.32\textwidth}}
\caption{GHAS Mapping to 16-Gate CI/CD Model}\label{tab:gate_mapping}\\
\toprule
\textbf{Gate} & \textbf{Gate Name} & \textbf{GHAS Feature} & \textbf{Blocking Behavior} \\
\midrule
\endfirsthead
\toprule
\textbf{Gate} & \textbf{Gate Name} & \textbf{GHAS Feature} & \textbf{Blocking Behavior} \\
\midrule
\endhead
\bottomrule
\endfoot
1 & Source Code Version Control & Secret Scanning (historical) & GHAS enabled on repo creation \\
2 & Branching Strategy & Branch Protection + GHAS & PR required with passing checks \\
3 & Static Analysis (SAST) & Code Scanning (CodeQL) & Blocks on high/critical \\
4 & Code Coverage (80\%+) & External (combined check) & Both coverage + CodeQL pass \\
5 & Vulnerability Scan & Code + Secret Scanning & Blocks on secrets/vulns \\
6 & Dependency Scan (SCA) & Dependency Review & Blocks vulnerable deps \\
7 & Artifact Version Control & Dependabot + Security Overview & Clean status for promotion \\
8-11 & Infra \& Testing Gates & GHAS status informs risk & Promotion requires clean status \\
12 & Full Automation & GHAS as required check & Pipeline fails if GHAS fails \\
13-15 & Rollback \& Release Gates & Security Overview status & Green status for prod deploy \\
16 & Feature Toggle Governance & Per-feature GHAS status & Block toggle if alerts open \\
\end{longtable}

\subsection{Developer Workflow Integration}

GHAS seamlessly integrates into existing GitHub-based development workflows:

\begin{itemize}[leftmargin=*]
\item Developer Commit: Code pushed to feature branch triggers CodeQL workflow
\item Pull Request Created: CodeQL, Dependency Review, and Secret Scanning run automatically
\item Inline Feedback: Findings appear as annotations directly in the PR diff view
\item Status Checks: Branch protection prevents merge until GHAS checks pass
\item Merge to Main: Clean merge triggers scheduled scans and updates Security Overview
\item Continuous Monitoring: Scheduled scans detect new vulnerabilities in existing code
\end{itemize}
\section{Implementation Roadmap}

The GHAS-powered security program is implemented in three phases over 18 months, progressing from foundation to full maturity.

\subsection{Phase 1: Foundation (Months 1-6)}

Objective: Establish governance, enable GHAS for critical applications, achieve quick wins.

\subsubsection{Governance Activities}

\begin{itemize}[leftmargin=*]
\item Secure executive sponsorship and budget approval for GHAS licenses
\item Establish AppSec governance committee with monthly cadence
\item Define and publish GHAS enablement policy and vulnerability SLAs
\item Complete application portfolio inventory and tier classification
\end{itemize}
\subsubsection{Technical Implementation}

\begin{itemize}[leftmargin=*]
\item Enable GitHub Code Security and Secret Protection for organization
\item Configure CodeQL default setup for all Tier 1 repositories
\item Enable Push Protection for Secret Scanning organization-wide
\item Configure Dependabot alerts and security updates for Tier 1-2
\item Implement branch protection rules requiring GHAS status checks
\end{itemize}
\subsubsection{Phase 1 Success Criteria}

\begin{itemize}[leftmargin=*]
\item 100\% of Tier 1 applications under CodeQL scanning
\item Push Protection enabled for all repositories
\item Security Champion identified for each development team
\item Baseline vulnerability metrics established
\item All policies approved and published
\end{itemize}
\subsection{Phase 2: Core Build (Months 7-12)}

Objective: Expand coverage, implement blocking gates, mature developer experience.

\subsubsection{Expansion Activities}

\begin{itemize}[leftmargin=*]
\item Extend CodeQL coverage to all Tier 2 applications
\item Configure advanced CodeQL queries for custom security requirements
\item Implement custom secret patterns for organization-specific tokens
\item Enable Dependency Review action with blocking for vulnerable dependencies
\item Configure required status checks for protected branches
\end{itemize}
\subsubsection{Developer Enablement}

\begin{itemize}[leftmargin=*]
\item Launch security awareness training program (100\% developer completion)
\item Create internal documentation hub for GHAS usage and triage guidance
\item Establish Security Champion advanced training curriculum
\item Implement feedback mechanism for false positive reporting
\end{itemize}
\subsubsection{Phase 2 Success Criteria}

\begin{itemize}[leftmargin=*]
\item 100\% of Tier 1-2 applications under full GHAS coverage
\item Blocking gates enforced for all Tier 1 applications
\item SLA compliance \textbackslash{}(\textbackslash{}geq\textbackslash{})85\% for vulnerability remediation
\item Developer security training completion 100\%
\item Mean time to remediate Critical findings \textless{}10 days
\end{itemize}
\subsection{Phase 3: Optimization (Months 13-18)}

Objective: Achieve full coverage, optimize processes, establish continuous improvement.

\subsubsection{Comprehensive Coverage}

\begin{itemize}[leftmargin=*]
\item Extend GHAS to all Tier 3-4 applications
\item Implement organization-wide security policies via GitHub Enterprise
\item Configure automated SBOM generation using Dependency Graph
\item Integrate GHAS data with enterprise SIEM/SOAR platforms
\end{itemize}
\subsubsection{Process Optimization}

\begin{itemize}[leftmargin=*]
\item Tune CodeQL queries to reduce false positive rate below 15\%
\item Implement automated triage rules for common finding patterns
\item Establish automated compliance evidence collection from Security Overview
\item Develop custom dashboards for leadership and audit reporting
\end{itemize}
\subsubsection{Phase 3 Success Criteria}

\begin{itemize}[leftmargin=*]
\item 100\% application portfolio coverage achieved
\item SLA compliance \textbackslash{}(\textbackslash{}geq\textbackslash{})95\% across all tiers
\item False positive rate \textless{}15\%
\item Developer satisfaction with security tooling \textbackslash{}(\textbackslash{}geq\textbackslash{})4.0/5.0
\item Automated compliance reporting operational
\end{itemize}
\section{Metrics and Reporting}

\subsection{Key Performance Indicators}

\begin{longtable}{p{0.34\textwidth}p{0.34\textwidth}p{0.26\textwidth}}
\caption{Key Performance Indicators}\label{tab:kpis}\\
\toprule
\textbf{Metric} & \textbf{Data Source} & \textbf{Target} \\
\midrule
\endfirsthead
\toprule
\textbf{Metric} & \textbf{Data Source} & \textbf{Target} \\
\midrule
\endhead
\bottomrule
\endfoot
Open Critical/High Findings & Security Overview dashboard & Decreasing trend \\
Mean Time to Remediate (MTTR) & Code scanning alerts timeline & Critical: \textless{}7 days \\
SLA Compliance Rate & Calculated from alert ages & \textgreater{}95\% \\
Secret Scanning Coverage & Organization settings & 100\% \\
Dependabot PR Merge Rate & Dependabot activity & \textgreater{}90\% within 14 days \\
Push Protection Bypass Rate & Secret scanning audit log & \textless{}5\% \\
False Positive Rate & Dismissed as false positive & \textless{}15\% \\
\end{longtable}

\subsection{Reporting Cadence}

\begin{itemize}[leftmargin=*]
\item Weekly: Security Champion sync - review new alerts, triage support
\item Monthly: Executive dashboard - Security Overview metrics, trend analysis
\item Quarterly: Steering committee - program progress, roadmap updates, resource needs
\item Annually: Maturity assessment - capability evaluation against OWASP SAMM
\end{itemize}
\section{Appendices}

\subsection{Vulnerability Remediation SLAs}

\begin{longtable}{p{0.18\textwidth}p{0.20\textwidth}p{0.20\textwidth}p{0.22\textwidth}}
\caption{Vulnerability Remediation SLAs by Tier}\label{tab:slas}\\
\toprule
\textbf{Severity} & \textbf{Tier 1 SLA} & \textbf{Tier 2 SLA} & \textbf{Tier 3-4 SLA} \\
\midrule
\endfirsthead
\toprule
\textbf{Severity} & \textbf{Tier 1 SLA} & \textbf{Tier 2 SLA} & \textbf{Tier 3-4 SLA} \\
\midrule
\endhead
\bottomrule
\endfoot
Critical & 7 days & 14 days & 30 days \\
High & 30 days & 45 days & 90 days \\
Medium & 90 days & 120 days & 180 days \\
Low & 180 days & 180 days & Best effort \\
\end{longtable}

\subsection{GHAS Configuration Checklist}

\begin{itemize}[leftmargin=*]
\item Enable GHAS at organization level
\item Configure default CodeQL setup for all new repositories
\item Enable Secret Scanning with Push Protection
\item Configure custom secret patterns for internal tokens
\item Enable Dependabot alerts and security updates
\item Configure Dependency Review GitHub Action
\item Set up branch protection rules with required status checks
\item Configure Security Overview access for security team
\item Set up alert notification routing to appropriate teams
\item Configure audit log streaming for compliance
\end{itemize}
\subsection{Reference Standards}

\begin{itemize}[leftmargin=*]
\item OWASP ASVS - Application Security Verification Standard
\item OWASP SAMM - Software Assurance Maturity Model
\item NIST SSDF - Secure Software Development Framework (SP 800-218)
\item CIS Controls - Center for Internet Security Controls
\item BSIMM - Building Security In Maturity Model
\end{itemize}
\section{Conclusion}

This Application Security Program, powered by GitHub Advanced Security, provides a comprehensive framework for systematically reducing application security risk across the enterprise. By embedding security directly into the developer workflow through CodeQL SAST, Secret Scanning, and Dependency Review, organizations can achieve shift-left security without sacrificing development velocity.

\subsection{Program Success Factors}

\begin{itemize}[leftmargin=*]
\item Executive Commitment: Sustained sponsorship, adequate licensing, and organizational authority
\item Developer Experience: Security integrated into existing workflows, not bolted on as friction
\item Risk-Based Approach: Prioritize coverage and gates based on application criticality
\item Measurable Progress: Leverage Security Overview for continuous visibility and improvement
\item Cultural Change: Security Champions and training build organization-wide capability
\end{itemize}
Application security is not a destination but a journey.

The goal is not perfection but continuous improvement---systematically reducing risk while enabling the business to innovate.

\end{document}
