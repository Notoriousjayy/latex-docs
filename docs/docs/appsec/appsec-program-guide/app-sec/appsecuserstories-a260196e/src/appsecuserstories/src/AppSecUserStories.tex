\documentclass[11pt,a4paper]{article}

% --- Page + typography ---
\usepackage[a4paper,margin=1in]{geometry}
\usepackage{lmodern}            % Latin Modern fonts
\usepackage[T1]{fontenc}
\usepackage[utf8]{inputenc}
\usepackage{microtype}          % better kerning/justification
\usepackage{parskip}            % space between paragraphs, no indents

% --- Structure + lists ---
\usepackage{enumitem}
\setlist{itemsep=2pt, topsep=4pt, leftmargin=1.2em}
\usepackage{titlesec}
\titlespacing*{\section}{0pt}{6pt plus 2pt}{4pt}
\titlespacing*{\subsection}{0pt}{5pt}{3pt}

% --- Color + links ---
\usepackage[dvipsnames]{xcolor}
\usepackage{hyperref}
\hypersetup{
  colorlinks=true,
  linkcolor=black,
  urlcolor=MidnightBlue,
  citecolor=black,
  pdfauthor={Jordan Suber},
  pdftitle={User Stories by Chapter: Application Security Program Guide}
}
\urlstyle{same}

% --- Math + symbols ---
\usepackage{amsmath,amssymb} % provides \square and math symbols

% --- Layout helpers for story cards ---
\usepackage[skins,breakable]{tcolorbox}
\tcbset{
  colback=gray!2,
  colframe=gray!50,
  arc=2pt,
  boxrule=0.4pt,
  left=8pt,right=8pt,top=8pt,bottom=8pt,
  enhanced jigsaw
}
\usepackage{tabularx}
\usepackage{array}
\usepackage{ragged2e}

% --- Readability helpers for cards ---
\newcolumntype{L}[1]{>{\raggedleft\arraybackslash\bfseries}p{#1}}
\newcolumntype{Y}{>{\RaggedRight\arraybackslash}X}
\newtcbox{\pill}{on line, arc=3pt, boxsep=0.8pt, left=4pt,right=4pt,top=1pt,bottom=1pt,
  colframe=gray!50, colback=gray!15, boxrule=0.3pt}
\newcommand{\badge}[1]{\pill{\footnotesize #1}}

% --- Shortcuts/labels ---
\newcommand{\cb}{\(\square\)}
\newcommand{\DoR}{\textbf{Definition of Ready:} Persona clear; AC drafted; Dependencies known; Estimate set.}
\newcommand{\DoD}{\textbf{Definition of Done:} All ACs pass; Tests green; Security/a11y checks; Docs updated; Deployed/flagged.}
\newcommand{\Priority}[1]{\textbf{Priority:} #1}

% --- Story Card (9-arg signature) ---
% 1: ID   2: Title   3: Epic/Feature   4: Business Value
% 5: Priority   6: Estimate(SP)   7: Persona   8: Dependencies   9: Assumptions/Risks
\newcommand{\StoryCard}[9]{%
  \newpage
  \begin{tcolorbox}[
    enhanced, breakable,
    colback=gray!2, colframe=gray!50, arc=2pt, boxrule=0.4pt,
    left=8pt,right=8pt,top=8pt,bottom=8pt,
    fonttitle=\bfseries\large,
    title={\textbf{#1}\ \textemdash\ #2},
    colbacktitle=gray!6, coltitle=black,
    borderline west={2pt}{0pt}{MidnightBlue}
  ]
  \small
  \begin{tabularx}{\textwidth}{@{}L{3.2cm}Y@{}}
    Epic / Feature          & #3 \\
    Business Value          & #4 \\
    Priority / Estimate     & \badge{Priority: #5}\ \badge{SP: #6} \\
    Persona                 & #7 \\
    Dependencies            & #8 \\
    Assumptions / Risks     & #9 \\
  \end{tabularx}

  \medskip
  \textbf{Story}\quad
  \emph{As a #7, I want to #2 so that #4.}

  \medskip
  \textbf{Non-Functional}\quad
  \badge{Performance}\ \badge{Security}\ \badge{Reliability}\ \badge{Accessibility}\ \badge{Privacy}\ \badge{i18n}

  \medskip
  \textbf{Acceptance Criteria (BDD)}
  \begin{description}[leftmargin=2.4cm, labelwidth=2.3cm, style=nextline, itemsep=2pt, topsep=2pt]
    \item[\textbf{Scenario}] Happy path
    \item[\textbf{Given}] the target repositories, environments, and program context are available
    \item[\textbf{When}] the \emph{Hands-on Objectives} for this chapter are executed
    \item[\textbf{Then}] the stated \emph{Outcomes/Deliverables} for this chapter are produced, reviewed, and published
  \end{description}

  \vspace{0.2\baselineskip}
  {\footnotesize\color{gray!60}\DoR\ \textbullet\ \DoD}
  \end{tcolorbox}
}

% --- Tasks box ---
\newenvironment{TasksBox}[1][Tasks]{%
  \begin{tcolorbox}[
    enhanced,breakable,
    colback=gray!1, colframe=gray!35,
    colbacktitle=gray!6, coltitle=black,
    title={#1}, fonttitle=\bfseries,
    borderline west={1.8pt}{0pt}{MidnightBlue},
    arc=2pt, boxrule=0.4pt,
    left=10pt,right=10pt,top=6pt,bottom=6pt,
    before skip=6pt, after skip=10pt
  ]
  \small
  \begin{itemize}[
    label=\cb,
    leftmargin=*,
    labelsep=0.6em,
    itemsep=4pt,
    topsep=2pt, parsep=0pt
  ]
}{%
  \end{itemize}
  \end{tcolorbox}
}

% --- Title ---
\title{\textbf{User Stories by Chapter:\\ Application Security Program Guide}}
\author{Compiled for Jordan Suber}
\date{}

\begin{document}
\maketitle
\tableofcontents
\newpage

\section*{How to Use This Template}
Each card maps one chapter’s \emph{Learning Goals} to a concise story, binds the chapter’s \emph{Hands-on Objectives} to concrete \emph{Tasks}, and verifies \emph{Outcomes} via BDD-style Acceptance Criteria. Import these cards into your backlog, tag by risk tier, and iterate.

\subsection*{Required Data on Every Story}
\begin{itemize}[itemsep=2pt,topsep=2pt]
  \item \textbf{ID} (e.g., APPSEC-1), \textbf{Title} (actionable verb), \textbf{Epic/Feature}, \textbf{Business Value} (outcome/why)
  \item \textbf{Priority} (Must/Should/Could), \textbf{Estimate} (SP), \textbf{Persona}, \textbf{Dependencies}, \textbf{Assumptions/Risks}
  \item \textbf{Acceptance Criteria} (Gherkin-ish BDD), \textbf{Tasks} (checklist), \textbf{NFR} (Security, Privacy, Reliability, etc.)
\end{itemize}

\subsection*{Writing Effective User Stories (Quick Guide)}
\textbf{Template:} As a \emph{[persona]}, I want to \emph{[do X]} so that \emph{[value/why]}.\\
\textbf{INVEST:} Independent, Negotiable, Valuable, Estimable, Small, Testable.\\
\textbf{Good:} “As an AppSec lead, I want a \emph{tiered SSDLC policy} so that \emph{teams ship securely with minimal friction}.”\\
\textbf{Anti-patterns:} Vague “Research X”; multi-team mega-stories; outputs without value (``create doc’’) unless tied to decision/change.

\newpage
\section{Stories by Chapter}

\StoryCard{APPSEC-1}{Publish an AppSec Program Charter}{Program Foundations}
{align engineering, product, and risk on scope, value, and success criteria}
{Must}{3}
{AppSec lead}
{Org strategy, security policy, product roadmap}
{Scope creep risk; time-box charter v1 and plan iterative updates}
\begin{TasksBox}
  \item Draft a one-page charter: mission, scope, definitions, interfaces, success metrics.
  \item Create a stakeholder map and RACI for threat modeling, testing, vuln mgmt, IR.
  \item Review with Eng/Product/Risk; capture decisions and open questions.
  \item Publish in the handbook repo; version as living document.
\end{TasksBox}

\StoryCard{APPSEC-2}{Create a Control Dictionary \& Traceability Matrix}{Security Foundations}
{give engineers clear, shared definitions and connect policies to app controls}
{Must}{5}
{Security architect}
{Enterprise policies/standards}
{Terminology mismatch; include concrete code/config examples}
\begin{TasksBox}
  \item Compile key concepts (authn, authz, logging, crypto, secrets, input validation).
  \item Map each enterprise policy to concrete application controls and test evidence.
  \item Add links to code samples, lints, and CI checks for each control.
  \item Publish as \texttt{/docs/control-dictionary.md} and keep PR-able.
\end{TasksBox}

\StoryCard{APPSEC-3}{Build an Application Inventory \& Tiering}{Program Scope}
{focus effort on highest-risk apps; enable tiered controls and SLAs}
{Must}{5}
{Product security engineer}
{CMDB/source of truth; service catalog}
{Owner gaps; require ownership to promote to higher envs}
\begin{TasksBox}
  \item Inventory apps/services/APIs with owners, data classes, exposure, tech stack.
  \item Define tiering model (e.g., P0–P3) with criteria and examples.
  \item Record lifecycle (active/sunset), compliance drivers, and repo links.
  \item Export registry to CSV/JSON; integrate with CI labels per repo.
\end{TasksBox}

\StoryCard{APPSEC-4}{Stand Up an App Risk Register}{Risk Management}
{turn threats into tracked items tied to owners, dates, and treatments}
{Must}{3}
{Risk manager}
{Inventory completed, risk rubric}
{Over-long registers stall; keep to top risks per app}
\begin{TasksBox}
  \item Define likelihood/impact rubric and treatment options.
  \item Run a 60–90 min risk workshop for two critical apps.
  \item Create entries with owner, due date, and linkage to epics/stories.
  \item Establish intake workflow (new risk \(\rightarrow\) triage \(\rightarrow\) acceptance).
\end{TasksBox}

\StoryCard{APPSEC-5}{Publish Secure Reference Architectures}{Secure Design Patterns}
{give teams golden paths that bake in zero-trust and least privilege}
{Should}{5}
{Security architect}
{Architecture council, platform patterns}
{Architecture drift; add linters/policies to reinforce}
\begin{TasksBox}
  \item Diagram monolith, microservices, async/event-driven, and serverless patterns.
  \item Annotate controls per tier (authn, mTLS, input validation, logging, backups).
  \item Provide IaC/app templates implementing the patterns.
  \item Add “choose-by-facts” table and decision records (ADRs).
\end{TasksBox}

\StoryCard{APPSEC-6}{Adopt a Tiered SSDLC Policy}{SSDLC Alignment}
{embed right-sized checks by risk tier to shift left without friction}
{Must}{5}
{AppSec lead}
{Engineering buy-in, CI access}
{Over-gating; start minimal and ratchet}
\begin{TasksBox}
  \item Define controls per SDLC phase and per tier (ASVS/SSDF-aligned).
  \item Wire required checks in CI (lint, SAST, SCA) with pass/fail thresholds.
  \item Add DoD/DoR updates to team templates referencing security checks.
  \item Document exceptions/waivers with expiry and approval path.
\end{TasksBox}

\StoryCard{APPSEC-7}{Launch the AppSec Champions Program}{Operating Model \& Teams}
{scale AppSec via embedded advocates and faster issue resolution}
{Should}{3}
{AppSec lead}
{Managers’ support, time allocation}
{Attrition/adoption risk; include incentives and community time}
\begin{TasksBox}
  \item Define selection rubric, responsibilities, and incentives.
  \item Create monthly office hours and a champions Slack channel.
  \item Provide starter kit (checklists, threat modeling kit, PR review guide).
  \item Track participation and outcomes (bugs prevented, PRs reviewed).
\end{TasksBox}

\StoryCard{APPSEC-8}{Standardize Threat Modeling}{Threat Modeling}
{catch design flaws early and convert threats into actionable requirements}
{Must}{5}
{Security champion}
{DFD notation, templates}
{Analysis paralysis; time-box sessions and prioritize}
\begin{TasksBox}
  \item Choose method (STRIDE/LINDDUN/misuse cases) and templates.
  \item Run two sessions on different architectures; capture DFDs and threats.
  \item Translate top threats into NFRs and tests.
  \item Add a reusable threats/mitigations catalogue to the wiki.
\end{TasksBox}

\StoryCard{APPSEC-9}{Publish Secure Coding Standards}{Secure Coding}
{reduce recurring vulnerabilities and speed reviews with clear checklists}
{Must}{3}
{Tech lead}
{Language stacks agreed}
{One-size-fits-none risk; tailor per language}
\begin{TasksBox}
  \item Write per-language standards (input validation, encoding, secrets, crypto).
  \item Add PR checklists and reviewer heuristics.
  \item Provide pre-commit hooks and code templates.
  \item Run a 45-min training; record and link in the repo.
\end{TasksBox}

\StoryCard{APPSEC-10}{Operationalize SAST/SCA/DAST/IAST}{Security Testing}
{improve signal-to-noise and make security checks part of normal CI}
{Must}{5}
{Automation engineer}
{Scanner licenses, CI capacity}
{Finding overload; enforce “new high/critical = fail”}
\begin{TasksBox}
  \item Integrate SAST \& SCA in CI; upload SARIF for code scanning.
  \item Stand up targeted DAST/IAST for a high-risk app.
  \item Establish severity thresholds, suppressions with expiry, and routing.
  \item Publish weekly trend reports and backlog hygiene metrics.
\end{TasksBox}

\StoryCard{APPSEC-11}{Generate SBOMs \& Sign Artifacts}{Supply Chain Security}
{improve provenance and compliance while enabling safe updates}
{Must}{5}
{Release engineer}
{SBOM tool, signer}
{Tooling gaps; start with top languages/images}
\begin{TasksBox}
  \item Produce SBOM (CycloneDX/SPDX) during builds; attach to artifacts.
  \item Sign artifacts/images and verify in promotion gates.
  \item Document third-party source allowlist and review cadence.
  \item Add attestation checks to release workflow.
\end{TasksBox}

\StoryCard{APPSEC-12}{Enforce API Security Standards}{API Security}
{protect data and consumers via consistent auth, validation, and quotas}
{Must}{5}
{API owner}
{OpenAPI/AsyncAPI specs}
{Shadow APIs; tie standard to inventory}
\begin{TasksBox}
  \item Write API security standard (authn/z, schema validation, rate limiting).
  \item Add contract tests and security tests to CI.
  \item Gate breaking changes and insecure defaults in PRs.
  \item Add discovery checks for undocumented endpoints.
\end{TasksBox}

\StoryCard{APPSEC-13}{Publish Cloud AppSec Baseline}{Cloud-Native App Security}
{set secure defaults for identity, secrets, network, and logging}
{Should}{3}
{Cloud security engineer}
{Cloud org access}
{Drift risk; add config conformance packs}
\begin{TasksBox}
  \item Define shared-responsibility for app teams; list must-have controls.
  \item Provide bootstrap templates for logging/telemetry and secrets.
  \item Add guardrails and conformance checks.
  \item Document carve-outs and exception review.
\end{TasksBox}

\StoryCard{APPSEC-14}{Harden Containers \& Kubernetes}{Container/K8s Security}
{reduce runtime risk with minimal images and admission policies}
{Must}{5}
{Platform engineer}
{Registry, admission controller}
{Breakages; start in warn mode, then enforce}
\begin{TasksBox}
  \item Create minimal, scanned base images; publish usage guidance.
  \item Enforce image provenance and vulnerability thresholds at admission.
  \item Apply Pod Security standards, RBAC, and NetworkPolicies.
  \item Add runtime policies for sensitive syscalls and egress.
\end{TasksBox}

\StoryCard{APPSEC-15}{Centralize Secrets \& Workload Identity}{Secrets \& IAM}
{eliminate hardcoded secrets and reduce blast radius via least privilege}
{Must}{3}
{Service owner}
{Secrets manager, IAM}
{Migration risk; migrate one app first}
\begin{TasksBox}
  \item Move secrets to a managed store with rotation.
  \item Adopt workload identity (mTLS/JWT/OIDC) for services.
  \item Review and minimize IAM policies per service.
  \item Add secrets scanning in CI and pre-commit.
\end{TasksBox}

\StoryCard{APPSEC-16}{Unify Vulnerability Intake \& SLAs}{Vulnerability Management}
{prioritize by exploitability and asset criticality to reduce MTTR}
{Must}{5}
{Vuln management owner}
{Scanner feeds, ticketing}
{Duplicate noise; dedupe by CWE/package/asset}
\begin{TasksBox}
  \item Define prioritization (CVSS/EPSS + criticality + exposure).
  \item Create unified intake and dedup logic across code/deps/containers/infra.
  \item Set SLAs per tier and auto-create tickets with owners and due dates.
  \item Build dashboard (age buckets, MTTR, reopen rate).
\end{TasksBox}

\StoryCard{APPSEC-17}{Integrate AppSec into Incident Response}{App IR}
{speed containment and comms for app-specific incidents}
{Should}{3}
{IR lead}
{On-call schedule, playbooks}
{Confusion in roles; publish contact matrix}
\begin{TasksBox}
  \item Write app-centric playbooks (auth bypass, data exfil, supply-chain).
  \item Define evidence capture and comms templates (legal/regulatory triggers).
  \item Run a tabletop; record actions and owners.
  \item Add lessons learned template and review cadence.
\end{TasksBox}

\StoryCard{APPSEC-18}{Set AI/ML Security Guardrails}{AI/ML Security}
{prevent model abuse and data leakage with standards and tests}
{Could}{5}
{ML product owner}
{Model inventory, logs}
{Novel threats; start with one model/feature}
\begin{TasksBox}
  \item Threat-model one ML feature (prompt injection, data poisoning, model theft).
  \item Add adversarial test cases and output filters.
  \item Log model interactions for abuse patterns.
  \item Document red-team scenarios and escalation paths.
\end{TasksBox}

\StoryCard{APPSEC-19}{Automate Evidence \& ChatOps}{Automation \& Orchestration}
{reduce toil and raise adoption with bots, policies-as-code, and summaries}
{Should}{3}
{Automation engineer}
{Bot account, APIs}
{Alert fatigue; keep messages concise with links}
\begin{TasksBox}
  \item Auto-comment PRs with scanner summaries and fix hints.
  \item Scaffold “new service” with secure defaults via a bot command.
  \item Export evidence (SBOM, test reports, approvals) automatically.
  \item Maintain an automation backlog with value stream mapping.
\end{TasksBox}

\StoryCard{APPSEC-20}{Ship Metrics Dashboard \& Maturity Plan}{Metrics \& Maturity}
{prove risk reduction and align roadmap with measurable outcomes}
{Must}{3}
{Program manager}
{Data sources, dashboard tool}
{Metric cargo-cult; define glossary and collection method}
\begin{TasksBox}
  \item Choose north-star KPIs (risk reduced, MTTR, escape rate) and definitions.
  \item Build a dashboard with trends and targets; segment by tier/team.
  \item Run baseline maturity assessment (e.g., SAMM) and publish a 12-month plan.
  \item Review quarterly and adjust priorities based on results.
\end{TasksBox}

\section*{Capstone \& Milestones (Reference)}
\textbf{Foundation:} Charter, control dictionary, inventory/tiering, risk register.\\
\textbf{Build-in Security:} Reference architectures, SSDLC, champions, secure coding, testing.\\
\textbf{Platform Guardrails:} SBOM/signing, API/cloud/K8s baselines, secrets/IAM.\\
\textbf{Operate \& Improve:} Vuln SLAs, App IR, AI/ML guardrails, automation, metrics+maturity.

\end{document}

