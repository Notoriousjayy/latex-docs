
\documentclass[11pt,letterpaper]{article}

% ============================================================================
% PACKAGES
% ============================================================================
\usepackage[utf8]{inputenc}
\usepackage[T1]{fontenc}
\usepackage[margin=1in]{geometry}
\usepackage{titlesec}
\usepackage{titletoc}
\usepackage{enumitem}
\usepackage{booktabs}
\usepackage{tabularx}
\usepackage{longtable}
\usepackage{array}
\usepackage{xcolor}
\usepackage{hyperref}
\usepackage{fancyhdr}
\usepackage{graphicx}
\usepackage{tcolorbox}
\usepackage{parskip}
\usepackage{etoolbox}

% ============================================================================
% COLOR DEFINITIONS
% ============================================================================
\definecolor{primaryblue}{RGB}{24, 78, 119}
\definecolor{accentblue}{RGB}{44, 130, 201}
\definecolor{lightgray}{RGB}{245, 245, 245}
\definecolor{darkgray}{RGB}{64, 64, 64}
\definecolor{epicgreen}{RGB}{34, 139, 34}
\definecolor{foundationpurple}{RGB}{102, 51, 153}

% ============================================================================
% HYPERREF SETUP
% ============================================================================
\hypersetup{
    colorlinks=true,
    linkcolor=primaryblue,
    urlcolor=accentblue,
    pdftitle={AppSec Architecture Documentation Package},
    pdfauthor={Application Security Team},
    pdfsubject={Views-and-Beyond Style Diagram Backlog}
}

% ============================================================================
% HEADER/FOOTER
% ============================================================================
\pagestyle{fancy}
\fancyhf{}
\fancyhead[L]{\small\textcolor{darkgray}{AppSec Architecture Documentation}}
\fancyhead[R]{\small\textcolor{darkgray}{Views-and-Beyond Backlog}}
\fancyfoot[C]{\thepage}
\renewcommand{\headrulewidth}{0.4pt}
\renewcommand{\footrulewidth}{0pt}

% ============================================================================
% SECTION FORMATTING
% ============================================================================
\titleformat{\section}
    {\Large\bfseries\color{primaryblue}}
    {\thesection}{1em}{}[\titlerule]

\titleformat{\subsection}
    {\large\bfseries\color{accentblue}}
    {\thesubsection}{1em}{}

\titleformat{\subsubsection}
    {\normalsize\bfseries\color{darkgray}}
    {\thesubsubsection}{1em}{}

% ============================================================================
% TCOLORBOX STYLES
% ============================================================================
\tcbset{
    epictitle/.style={
        colback=lightgray,
        colframe=primaryblue,
        fonttitle=\bfseries\large,
        title=#1,
        boxrule=1pt,
        arc=3pt,
        left=10pt,
        right=10pt,
        top=8pt,
        bottom=8pt
    }
}

\newtcolorbox{epicbox}[1][]{
    colback=lightgray,
    colframe=primaryblue,
    fonttitle=\bfseries\large,
    title=#1,
    boxrule=1pt,
    arc=3pt,
    left=10pt,
    right=10pt,
    top=8pt,
    bottom=8pt
}

\newtcolorbox{storycard}{
    colback=white,
    colframe=accentblue,
    boxrule=0.5pt,
    arc=2pt,
    left=8pt,
    right=8pt,
    top=6pt,
    bottom=6pt,
    before skip=10pt,
    after skip=10pt
}

\newtcolorbox{infobox}[1][]{
    colback=lightgray,
    colframe=darkgray,
    fonttitle=\bfseries,
    title=#1,
    boxrule=0.5pt,
    arc=2pt,
    left=8pt,
    right=8pt,
    top=6pt,
    bottom=6pt
}

% ============================================================================
% CUSTOM COMMANDS
% ============================================================================
\newcommand{\storyfield}[2]{\textbf{#1:} #2}
\newcommand{\epiclabel}[1]{\textcolor{epicgreen}{\textbf{#1}}}

% ============================================================================
% DOCUMENT
% ============================================================================
\begin{document}

% ============================================================================
% TITLE PAGE
% ============================================================================
\begin{titlepage}
    \centering
    \vspace*{2cm}
    
    {\Huge\bfseries\color{primaryblue} AppSec Architecture\\[0.3cm] Documentation Package\par}
    
    \vspace{1cm}
    
    {\Large\color{accentblue} Views-and-Beyond Style Diagram Backlog\par}
    
    \vspace{2cm}
    
    \begin{tcolorbox}[
        colback=lightgray,
        colframe=primaryblue,
        width=0.85\textwidth,
        arc=5pt,
        boxrule=1.5pt
    ]
    \centering
    \large
    \textbf{Scope:} Intake, Threat Modeling, Vulnerability Management,\\
    Exceptions, CI/CD Gates, and Reporting
    \end{tcolorbox}
    
    \vspace{2cm}
    
    \begin{tabular}{ll}
        \textbf{Document Type:} & Architecture Documentation Backlog \\[0.3cm]
        \textbf{Methodology:} & Views and Beyond (V\&B) \\[0.3cm]
        \textbf{Format:} & Epics and User Stories \\[0.3cm]
        \textbf{Status:} & Ready for Execution \\
    \end{tabular}
    
    \vfill
    
    {\large Application Security Program\par}
    
    \vspace{0.5cm}
    
    {\small\textcolor{darkgray}{Version 1.0}\par}
    
\end{titlepage}

% ============================================================================
% TABLE OF CONTENTS
% ============================================================================
\tableofcontents
\newpage

% ============================================================================
% SECTION: INTRODUCTION
% ============================================================================
\section{Introduction and Conventions}

This document presents a comprehensive diagram backlog for Application Security (AppSec) architecture documentation, organized using the Views-and-Beyond approach. The backlog is structured as architecture documentation packages (epics) containing diagram user stories (deliverables).

The execution order is designed to deliver incremental business value:

\begin{center}
\textbf{Foundation} $\rightarrow$ \textbf{Current-State} $\rightarrow$ \textbf{Target-State} $\rightarrow$ \textbf{Automation/Integration} $\rightarrow$ \textbf{Metrics \& Governance}
\end{center}

% ----------------------------------------------------------------------------
\subsection{View Types (Views \& Beyond)}

\begin{infobox}[Architectural View Types]
\begin{description}[leftmargin=2cm, style=nextline]
    \item[Context/Scope (C4 L1)] Boundaries, actors, external dependencies
    \item[Process View (BPMN)] End-to-end workflows, swimlanes, decision points
    \item[Information/Evidence View (DFD)] Inputs/outputs, evidence artifacts, record systems
    \item[Component-and-Connector (C4 L2--L3)] Tools/services and signal movement between them
    \item[Allocation View (Deployment/Responsibility)] Where things run, who owns what (RACI)
    \item[Beyond Views] Glossary, assumptions, policies, SLAs, decision rules, traceability, roadmap
\end{description}
\end{infobox}

% ----------------------------------------------------------------------------
\subsection{Standard Diagram Story Card Fields}

Each diagram deliverable follows a consistent story card format with the following fields:

\begin{storycard}
\begin{itemize}[nosep]
    \item \textbf{Deliverable} --- The artifact to be produced
    \item \textbf{Primary Stakeholders \& Concerns} --- Who needs this and why
    \item \textbf{Inputs} --- Required source materials
    \item \textbf{Notation} --- Diagram type and modeling language
    \item \textbf{Acceptance Criteria} --- Definition of done
    \item \textbf{Dependencies} --- Prerequisite deliverables
\end{itemize}
\end{storycard}

\newpage

% ============================================================================
% EPIC 0: FOUNDATION
% ============================================================================
\section{EPIC 0 --- Foundation}

\begin{epicbox}[AppSec Architecture Documentation Package (Foundation)]
This foundational epic establishes the baseline artifacts required for all subsequent documentation. It defines stakeholders, boundaries, services, and shared terminology.

\textbf{Dependencies:} None (this is the base for everything else)
\end{epicbox}

% ----------------------------------------------------------------------------
\subsection{0.1 Stakeholders \& Concerns Map}

\begin{storycard}
\storyfield{Deliverable}{Stakeholder--Concern matrix covering Exec/BOD, Engineering leaders, Development teams, SRE, GRC/Audit, and AppSec}

\storyfield{Notation}{Table + short narrative}

\storyfield{Acceptance Criteria}{Every later diagram links back to at least one concern}
\end{storycard}

% ----------------------------------------------------------------------------
\subsection{0.2 AppSec System Context and Boundaries}

\begin{storycard}
\storyfield{Deliverable}{System Context diagram showing AppSec as a service and its interfaces with Engineering, CI/CD, Ticketing, CMDB, IAM, and GRC}

\storyfield{Notation}{C4 Level 1}

\storyfield{Acceptance Criteria}{Named systems of record for: findings, exceptions, risk acceptance, reporting}
\end{storycard}

% ----------------------------------------------------------------------------
\subsection{0.3 AppSec Service Catalog}

\begin{storycard}
\storyfield{Deliverable}{Service catalog with entry criteria, outputs, SLAs, and escalation paths}

\storyfield{Notation}{Structured catalog page + lightweight service blueprint}

\storyfield{Acceptance Criteria}{Intake routes map 1:1 to services (no orphan request types)}
\end{storycard}

% ----------------------------------------------------------------------------
\subsection{0.4 Shared Glossary and Taxonomy}

\begin{storycard}
\storyfield{Deliverable}{Standard definitions for: ``finding,'' ``vulnerability,'' ``risk,'' ``exception/waiver,'' ``false positive,'' ``SLA,'' ``severity,'' ``gate''}

\storyfield{Acceptance Criteria}{Used consistently across all BPMN labels and decision tables}
\end{storycard}

\newpage

% ============================================================================
% EPIC 1: INTAKE
% ============================================================================
\section{EPIC 1 --- AppSec Intake}

\begin{epicbox}[Request $\rightarrow$ Triage $\rightarrow$ Routing]
This epic documents the intake process from initial request through triage, categorization, and routing to appropriate queues.

\textbf{Dependencies:} EPIC 0.3 Service Catalog, EPIC 0.4 Glossary
\end{epicbox}

% ----------------------------------------------------------------------------
\subsection{1.1 Current-State Intake Workflow}

\begin{storycard}
\storyfield{Deliverable}{``As-Is'' intake BPMN: request submission $\rightarrow$ triage $\rightarrow$ categorization $\rightarrow$ routing $\rightarrow$ queue/assignment $\rightarrow$ closure}

\storyfield{Primary Stakeholders}{Developers (friction), AppSec (load), Engineering managers (predictability)}

\storyfield{Inputs}{Existing intake channels (email/forms/tickets), categories, current SLAs}

\storyfield{Notation}{BPMN swimlanes (Dev / AppSec / Eng Manager / GRC as needed)}

\storyfield{Acceptance Criteria}{}
\begin{itemize}[nosep, leftmargin=1.5em]
    \item Single start event and explicit end states (Completed, Rejected, Needs Info, Routed)
    \item Triage decision points use named criteria (severity, due date, compliance driver)
\end{itemize}
\end{storycard}

% ----------------------------------------------------------------------------
\subsection{1.2 Target-State Intake Workflow + SLA Model}

\begin{storycard}
\storyfield{Deliverable}{``To-Be'' BPMN with standardized intake form fields, auto-routing rules, SLAs by request type}

\storyfield{Acceptance Criteria}{Every routing decision has an explicit rule and owner}
\end{storycard}

% ----------------------------------------------------------------------------
\subsection{1.3 Intake Decision Table (Routing Rules)}

\begin{storycard}
\storyfield{Deliverable}{Decision table that maps request type + risk + due date $\rightarrow$ queue/owner/SLA}

\storyfield{Notation}{Decision table (DMN-lite is acceptable)}

\storyfield{Acceptance Criteria}{No ``tribal knowledge'' steps remain; all routing logic is documented}
\end{storycard}

% ----------------------------------------------------------------------------
\subsection{1.4 Intake RACI + Escalation Path}

\begin{storycard}
\storyfield{Deliverable}{RACI chart + escalation swimlane overlay}

\storyfield{Acceptance Criteria}{For each step: exactly one \textbf{Accountable} role}
\end{storycard}

\newpage

% ============================================================================
% EPIC 2: THREAT MODELING
% ============================================================================
\section{EPIC 2 --- Threat Modeling}

\begin{epicbox}[Design Intake $\rightarrow$ Model $\rightarrow$ Findings $\rightarrow$ Tracking]
This epic covers the threat modeling lifecycle from initial design engagement through model creation, finding identification, and remediation tracking.

\textbf{Dependencies:} EPIC 1 Intake (threat modeling typically begins as an intake request)
\end{epicbox}

% ----------------------------------------------------------------------------
\subsection{2.1 Threat Modeling Service Blueprint}

\begin{storycard}
\storyfield{Deliverable}{Service blueprint showing frontstage developer experience + backstage AppSec work + support systems}

\storyfield{Primary Stakeholders}{Development leads, AppSec, Architects}

\storyfield{Acceptance Criteria}{Includes entry criteria, artifacts required, and defined outputs (model, mitigations, backlog items)}
\end{storycard}

% ----------------------------------------------------------------------------
\subsection{2.2 Current-State Threat Modeling BPMN}

\begin{storycard}
\storyfield{Deliverable}{As-Is BPMN: kickoff $\rightarrow$ context gathering $\rightarrow$ trust boundaries/data flows $\rightarrow$ threat enumeration $\rightarrow$ mitigations $\rightarrow$ sign-off $\rightarrow$ tracking}

\storyfield{Acceptance Criteria}{Artifacts are explicit outputs (DFD, threat list, mitigations)}
\end{storycard}

% ----------------------------------------------------------------------------
\subsection{2.3 Target-State Threat Modeling BPMN (Shift-Left)}

\begin{storycard}
\storyfield{Deliverable}{To-Be BPMN integrating threat modeling into SDLC stages (PRD/design review, architecture review, pre-implementation)}

\storyfield{Acceptance Criteria}{Shows when threat modeling is mandatory vs.\ optional}
\end{storycard}

% ----------------------------------------------------------------------------
\subsection{2.4 Threat Model Artifacts View}

\begin{storycard}
\storyfield{Deliverable}{Standard artifact set diagram: system context, DFD, trust boundaries, abuse cases, mitigations, residual risk}

\storyfield{Notation}{DFD + labeled trust boundaries + checklist}

\storyfield{Acceptance Criteria}{Each artifact mapped to where it's stored and how it's versioned}
\end{storycard}

% ----------------------------------------------------------------------------
\subsection{2.5 Findings Traceability (Threats $\rightarrow$ Requirements $\rightarrow$ Tickets)}

\begin{storycard}
\storyfield{Deliverable}{Traceability diagram tying threat scenarios to security requirements and tracked work items}

\storyfield{Acceptance Criteria}{One canonical system of record for mitigations and closure evidence}
\end{storycard}

\newpage

% ============================================================================
% EPIC 3: VULNERABILITY MANAGEMENT
% ============================================================================
\section{EPIC 3 --- Vulnerability Management}

\begin{epicbox}[Discover $\rightarrow$ Triage $\rightarrow$ Remediate $\rightarrow$ Verify $\rightarrow$ Close]
This epic documents the complete vulnerability management lifecycle from discovery through verified closure.

\textbf{Dependencies:} EPIC 0 Foundation
\end{epicbox}

% ----------------------------------------------------------------------------
\subsection{3.1 Vulnerability Management Value Stream Map}

\begin{storycard}
\storyfield{Deliverable}{Value stream map with lead time and wait states (discovery $\rightarrow$ SLA start $\rightarrow$ fix $\rightarrow$ verify $\rightarrow$ close)}

\storyfield{Primary Stakeholders}{Engineering leadership, AppSec, GRC}

\storyfield{Acceptance Criteria}{Identifies top 3 bottlenecks with data sources for measurement}
\end{storycard}

% ----------------------------------------------------------------------------
\subsection{3.2 Current-State Vulnerability Management BPMN}

\begin{storycard}
\storyfield{Deliverable}{As-Is BPMN including: deduplication, false positive handling, severity assignment, ticket creation, ownership assignment, remediation, verification, closure}

\storyfield{Acceptance Criteria}{Explicitly models re-open conditions and ``won't fix'' outcomes}
\end{storycard}

% ----------------------------------------------------------------------------
\subsection{3.3 Target-State Vulnerability Management BPMN (Automation-First)}

\begin{storycard}
\storyfield{Deliverable}{To-Be BPMN with automation steps (auto-ticketing, auto-dedupe, SLAs, exception triggers)}

\storyfield{Acceptance Criteria}{Automated vs.\ manual steps clearly annotated}
\end{storycard}

% ----------------------------------------------------------------------------
\subsection{3.4 Severity and Prioritization Decision Model}

\begin{storycard}
\storyfield{Deliverable}{Decision table: severity inputs (CVSS, exploitability, asset criticality, exposure) $\rightarrow$ priority and SLA}

\storyfield{Acceptance Criteria}{Approved by Engineering + GRC (or documented dissent + rationale)}
\end{storycard}

% ----------------------------------------------------------------------------
\subsection{3.5 Evidence \& Audit Trail View (Vulnerability Closure)}

\begin{storycard}
\storyfield{Deliverable}{Evidence flow diagram for closure: scan result $\rightarrow$ ticket $\rightarrow$ fix PR $\rightarrow$ deployment $\rightarrow$ rescan $\rightarrow$ closure record}

\storyfield{Acceptance Criteria}{For each closure state, required evidence is listed and retrievable}
\end{storycard}

\newpage

% ============================================================================
% EPIC 4: EXCEPTIONS AND RISK ACCEPTANCE
% ============================================================================
\section{EPIC 4 --- Exceptions and Risk Acceptance}

\begin{epicbox}[Waivers, Compensating Controls, Expiry]
This epic covers the exception lifecycle including risk acceptance governance, compensating controls, and evidence retention.

\textbf{Dependencies:} EPIC 3 Prioritization Model (exceptions depend on severity/impact framing)
\end{epicbox}

% ----------------------------------------------------------------------------
\subsection{4.1 Exception Lifecycle BPMN}

\begin{storycard}
\storyfield{Deliverable}{BPMN: request $\rightarrow$ justification $\rightarrow$ compensating controls $\rightarrow$ approval $\rightarrow$ expiry/review $\rightarrow$ revoke/renew}

\storyfield{Primary Stakeholders}{GRC, Product owners, AppSec, Engineering leadership}

\storyfield{Acceptance Criteria}{Every exception has: owner, scope, expiry date, review cadence, rollback plan}
\end{storycard}

% ----------------------------------------------------------------------------
\subsection{4.2 Risk Acceptance Governance View}

\begin{storycard}
\storyfield{Deliverable}{Governance diagram: who can accept what risk, thresholds, escalation rules, required approvers}

\storyfield{Notation}{RACI + decision table}

\storyfield{Acceptance Criteria}{Clear separation between ``AppSec recommends'' vs.\ ``business accepts''}
\end{storycard}

% ----------------------------------------------------------------------------
\subsection{4.3 Compensating Controls Catalog (Linked to Exceptions)}

\begin{storycard}
\storyfield{Deliverable}{Catalog mapping common exceptions to compensating controls and monitoring requirements}

\storyfield{Acceptance Criteria}{Each compensating control maps to measurable signals or checks}
\end{storycard}

% ----------------------------------------------------------------------------
\subsection{4.4 Exception Evidence \& Reporting View}

\begin{storycard}
\storyfield{Deliverable}{Data-flow diagram showing exception records, approvals, and evidence retention}

\storyfield{Acceptance Criteria}{Audit can answer: ``What exceptions exist right now and why?''}
\end{storycard}

\newpage

% ============================================================================
% EPIC 5: CI/CD SECURITY GATES
% ============================================================================
\section{EPIC 5 --- CI/CD Security Gates}

\begin{epicbox}[Checks, Policies, Break-Glass, Enforcement]
This epic documents the CI/CD security gate architecture including policy enforcement, toolchain integration, and emergency override procedures.

\textbf{Dependencies:} EPIC 4 Exceptions (override policy and exception policy must align)
\end{epicbox}

% ----------------------------------------------------------------------------
\subsection{5.1 Secure CI/CD Gate Model (Policy-to-Pipeline)}

\begin{storycard}
\storyfield{Deliverable}{Gate architecture diagram mapping required checks to pipeline stages (SAST/SCA/Secrets/IaC/Container as applicable)}

\storyfield{Notation}{Pipeline flow diagram + control mapping}

\storyfield{Acceptance Criteria}{Each gate has: purpose, pass/fail criteria, owner, override policy}
\end{storycard}

% ----------------------------------------------------------------------------
\subsection{5.2 Current-State CI/CD Gate BPMN}

\begin{storycard}
\storyfield{Deliverable}{BPMN for ``code change $\rightarrow$ build/test $\rightarrow$ security checks $\rightarrow$ decision $\rightarrow$ deploy''}

\storyfield{Acceptance Criteria}{Shows all outcomes: block, warn, create ticket, require approval}
\end{storycard}

% ----------------------------------------------------------------------------
\subsection{5.3 Target-State CI/CD Gate BPMN (Risk-Based Enforcement)}

\begin{storycard}
\storyfield{Deliverable}{To-Be BPMN implementing: severity thresholds, repo criticality, branch protections, staged enforcement rollout}

\storyfield{Acceptance Criteria}{Includes a rollout plan path (monitor-only $\rightarrow$ warn $\rightarrow$ enforce)}
\end{storycard}

% ----------------------------------------------------------------------------
\subsection{5.4 Toolchain Component-and-Connector View}

\begin{storycard}
\storyfield{Deliverable}{C4 L2/L3 showing connectors among: SCM, CI, scanners, artifact repo, ticketing, reporting, IAM}

\storyfield{Acceptance Criteria}{Every finding source has a defined ingestion path and deduplication strategy}
\end{storycard}

% ----------------------------------------------------------------------------
\subsection{5.5 Break-Glass / Override Workflow}

\begin{storycard}
\storyfield{Deliverable}{BPMN for emergency override including approvals, logging, expiry, and post-incident review}

\storyfield{Acceptance Criteria}{Override events automatically generate a review item and are reportable}
\end{storycard}

\newpage

% ============================================================================
% EPIC 6: REPORTING AND METRICS
% ============================================================================
\section{EPIC 6 --- Reporting, Metrics, and Executive Visibility}

\begin{epicbox}[Outcomes, Risk, Performance]
This epic establishes the metrics framework, reporting cadence, and executive dashboards for AppSec program visibility.

\textbf{Dependencies:} EPIC 3 Evidence Model, EPIC 4 Exception Records, EPIC 5 Toolchain Flows
\end{epicbox}

% ----------------------------------------------------------------------------
\subsection{6.1 AppSec Metrics Tree (KPI/OKR Alignment)}

\begin{storycard}
\storyfield{Deliverable}{KPI tree connecting operational metrics $\rightarrow$ risk outcomes $\rightarrow$ business outcomes}

\storyfield{Primary Stakeholders}{Executives/BOD, Engineering leadership, GRC}

\storyfield{Acceptance Criteria}{Every metric has an owner, source, and decision it supports}
\end{storycard}

% ----------------------------------------------------------------------------
\subsection{6.2 Scorecard + Cadence Map}

\begin{storycard}
\storyfield{Deliverable}{Scorecard (weekly operational / monthly leadership / quarterly exec) + meeting/decision cadence diagram}

\storyfield{Acceptance Criteria}{Cadence includes: vulnerability backlog review, exception review, gate policy changes, major escalations}
\end{storycard}

% ----------------------------------------------------------------------------
\subsection{6.3 Risk Posture Dashboard Model}

\begin{storycard}
\storyfield{Deliverable}{Dashboard wireframe + data lineage diagram (what data feeds what chart)}

\storyfield{Acceptance Criteria}{Defines ``single source of truth'' per metric and refresh frequency}
\end{storycard}

% ----------------------------------------------------------------------------
\subsection{6.4 Evidence Traceability End-to-End}

\begin{storycard}
\storyfield{Deliverable}{End-to-end traceability map from controls $\rightarrow$ checks $\rightarrow$ artifacts $\rightarrow$ evidence store $\rightarrow$ reports}

\storyfield{Acceptance Criteria}{Supports audit questions without manual reconstruction}
\end{storycard}

\newpage

% ============================================================================
% BEYOND VIEWS
% ============================================================================
\section{Cross-Epic ``Beyond Views'' Items}

\begin{epicbox}[Reusable Artifacts Across All Epics]
These artifacts are created once and referenced throughout all documentation packages. They ensure consistency, navigability, and maintainability of the architecture documentation.
\end{epicbox}

% ----------------------------------------------------------------------------
\subsection{BV-1: Diagram Index and Navigation}

\begin{storycard}
\storyfield{Deliverable}{One index page linking each diagram to its stakeholder concerns and related policies/SOPs}

\storyfield{Acceptance Criteria}{A new team member can find ``how intake works'' in under 2 minutes}
\end{storycard}

% ----------------------------------------------------------------------------
\subsection{BV-2: Standards, Policies, and Control Traceability}

\begin{storycard}
\storyfield{Deliverable}{Mapping: policy statements $\rightarrow$ process steps $\rightarrow$ automated checks $\rightarrow$ evidence artifacts}

\storyfield{Acceptance Criteria}{Each ``must'' statement has a verification method}
\end{storycard}

% ----------------------------------------------------------------------------
\subsection{BV-3: Change Log and Ownership}

\begin{storycard}
\storyfield{Deliverable}{Diagram ownership + update cadence + change log}

\storyfield{Acceptance Criteria}{Every diagram has an accountable owner and a review date}
\end{storycard}

\newpage

% ============================================================================
% EXECUTION ORDER
% ============================================================================
\section{Suggested Execution Order}

The following execution order is recommended to deliver the fastest business value:

\begin{center}
\begin{tabular}{clp{8cm}}
\toprule
\textbf{Phase} & \textbf{Epic} & \textbf{Rationale} \\
\midrule
1 & \textbf{EPIC 0} --- Foundation & Establishes shared vocabulary, boundaries, and stakeholder alignment \\
2 & \textbf{EPIC 1} --- Intake & Standardizes request handling and reduces friction \\
3 & \textbf{EPIC 3} --- Vulnerability Management & Core operational process with highest volume \\
4 & \textbf{EPIC 4} --- Exceptions & Governance for risk acceptance decisions \\
5 & \textbf{EPIC 5} --- CI/CD Gates & Automation and enforcement capabilities \\
6 & \textbf{EPIC 2} --- Threat Modeling & Best executed once intake is stable \\
7 & \textbf{EPIC 6} --- Reporting & Meaningful once data/evidence flows exist \\
\bottomrule
\end{tabular}
\end{center}

\vspace{1cm}

\begin{infobox}[Implementation Note]
If this backlog needs to be converted to ticketable work items, each deliverable can be formatted as a Jira-ready user story with:
\begin{itemize}[nosep]
    \item Story points
    \item Assigned owners (by role)
    \item Explicit Definition of Ready (DoR) and Definition of Done (DoD) checklists
\end{itemize}
The Views-and-Beyond packaging structure should be preserved as epic-level organization.
\end{infobox}

% ============================================================================
% APPENDIX: DEPENDENCY MAP
% ============================================================================
\newpage
\section*{Appendix A: Epic Dependency Map}
\addcontentsline{toc}{section}{Appendix A: Epic Dependency Map}

\begin{center}
\renewcommand{\arraystretch}{1.4}
\begin{tabular}{lll}
\toprule
\textbf{Epic} & \textbf{Depends On} & \textbf{Enables} \\
\midrule
EPIC 0 (Foundation) & --- & All subsequent epics \\
EPIC 1 (Intake) & EPIC 0.3, 0.4 & EPIC 2 \\
EPIC 2 (Threat Modeling) & EPIC 1 & EPIC 6 \\
EPIC 3 (Vulnerability Mgmt) & EPIC 0 & EPIC 4, EPIC 6 \\
EPIC 4 (Exceptions) & EPIC 3.4 & EPIC 5 \\
EPIC 5 (CI/CD Gates) & EPIC 4 & EPIC 6 \\
EPIC 6 (Reporting) & EPIC 3, 4, 5 & --- \\
\bottomrule
\end{tabular}
\end{center}

\vspace{2cm}

% ============================================================================
% APPENDIX: NOTATION REFERENCE
% ============================================================================
\section*{Appendix B: Notation Quick Reference}
\addcontentsline{toc}{section}{Appendix B: Notation Quick Reference}

\begin{center}
\renewcommand{\arraystretch}{1.4}
\begin{tabular}{p{4cm}p{9cm}}
\toprule
\textbf{Notation} & \textbf{Usage} \\
\midrule
C4 Level 1 & System context diagrams showing boundaries and external actors \\
C4 Level 2/L3 & Container and component diagrams for toolchain architecture \\
BPMN & Process workflows with swimlanes and decision gateways \\
DFD & Data flow diagrams for information and evidence views \\
DMN (lite) & Decision tables for routing and prioritization rules \\
RACI & Responsibility assignment matrices \\
Service Blueprint & Customer journey + frontstage/backstage mapping \\
Value Stream Map & Lead time analysis with wait states and bottlenecks \\
\bottomrule
\end{tabular}
\end{center}

\end{document}
