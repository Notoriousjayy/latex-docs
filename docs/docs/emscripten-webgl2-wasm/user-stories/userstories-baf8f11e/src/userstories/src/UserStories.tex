\documentclass[11pt,a4paper]{article}

% --- Page + typography ---
\usepackage[a4paper,margin=1in]{geometry}
\usepackage{lmodern}            % Latin Modern fonts
\usepackage[T1]{fontenc}
\usepackage[utf8]{inputenc}
\usepackage{microtype}          % better kerning/justification
\usepackage{parskip}            % space between paragraphs, no indents

% --- Structure + lists ---
\usepackage{enumitem}
\setlist{itemsep=2pt, topsep=4pt, leftmargin=1.2em}
\usepackage{titlesec}
\titlespacing*{\section}{0pt}{6pt plus 2pt}{4pt}
\titlespacing*{\subsection}{0pt}{5pt}{3pt}

% --- Color + links ---
\usepackage[dvipsnames]{xcolor}
\usepackage{hyperref}
\hypersetup{
  colorlinks=true,
  linkcolor=black,
  urlcolor=MidnightBlue,
  citecolor=black,
  pdfauthor={Jordan Suber},
  pdftitle={User Stories by Chapter: Application Security Program Guide}
}
\urlstyle{same}

% --- Math + symbols ---
\usepackage{amsmath,amssymb} % provides \square and math symbols

% --- Layout helpers for story cards ---
\usepackage[skins,breakable]{tcolorbox}
\tcbset{
  colback=gray!2,
  colframe=gray!50,
  arc=2pt,
  boxrule=0.4pt,
  left=8pt,right=8pt,top=8pt,bottom=8pt,
  enhanced jigsaw
}
\usepackage{tabularx}
\usepackage{array}
\usepackage{ragged2e}

% --- Readability helpers for cards ---
\newcolumntype{L}[1]{>{\raggedleft\arraybackslash\bfseries}p{#1}}
\newcolumntype{Y}{>{\RaggedRight\arraybackslash}X}
\newtcbox{\pill}{on line, arc=3pt, boxsep=0.8pt, left=4pt,right=4pt,top=1pt,bottom=1pt,
  colframe=gray!50, colback=gray!15, boxrule=0.3pt}
\newcommand{\badge}[1]{\pill{\footnotesize #1}}

% --- Shortcuts/labels ---
\newcommand{\cb}{\(\square\)}
\newcommand{\DoR}{\textbf{Definition of Ready:} Persona clear; AC drafted; Dependencies known; Estimate set.}
\newcommand{\DoD}{\textbf{Definition of Done:} All ACs pass; Tests green; Security/a11y checks; Docs updated; Deployed/flagged.}
\newcommand{\Priority}[1]{\textbf{Priority:} #1}

% --- Tasks box (robust, supports headings + nested lists) ---
\usepackage{environ}
\NewEnviron{TasksBox}[1][Tasks]{%
  \begin{tcolorbox}[
    enhanced,breakable,
    colback=gray!1, colframe=gray!35,
    colbacktitle=gray!6, coltitle=black,
    title={#1}, fonttitle=\bfseries,
    borderline west={1.8pt}{0pt}{MidnightBlue},
    arc=2pt, boxrule=0.4pt,
    left=10pt,right=10pt,top=6pt,bottom=6pt,
    before skip=6pt, after skip=10pt
  ]
  \small
  % Outer checklist with checkbox labels
  \begin{itemize}[
    label=\cb,
    leftmargin=*,
    labelsep=0.6em,
    itemsep=4pt,
    topsep=2pt, parsep=0pt
  ]
    % Allow free text/headings & nested lists anywhere in the body
    \item[]%
    {%
      % Ensure inner itemize lists have consistent spacing
      \setlist[itemize]{label=\cb,leftmargin=*,labelsep=0.6em,itemsep=4pt,topsep=2pt,parsep=0pt}%
      \BODY
    }%
  \end{itemize}
  \end{tcolorbox}
}

% --- Story Card (9-arg signature) ---
% 1: ID   2: Title   3: Epic/Feature   4: Business Value
% 5: Priority   6: Estimate(SP)   7: Persona   8: Dependencies   9: Assumptions/Risks
\newcommand{\StoryCard}[9]{%
  \newpage
  \begin{tcolorbox}[
    enhanced, breakable,
    colback=gray!2, colframe=gray!50, arc=2pt, boxrule=0.4pt,
    left=8pt,right=8pt,top=8pt,bottom=8pt,
    fonttitle=\bfseries\large,
    title={\textbf{#1}\ \textemdash\ #2},
    colbacktitle=gray!6, coltitle=black,
    borderline west={2pt}{0pt}{MidnightBlue}
  ]
  \small
  \begin{tabularx}{\textwidth}{@{}L{3.2cm}Y@{}}
    Epic / Feature          & #3 \\
    Business Value          & #4 \\
    Priority / Estimate     & \badge{Priority: #5}\ \badge{SP: #6} \\
    Persona                 & #7 \\
    Dependencies            & #8 \\
    Assumptions / Risks     & #9 \\
  \end{tabularx}

  \medskip
  \textbf{Story}\quad
  \emph{As a #7, I want to #2 so that #4.}

  \medskip
  \textbf{Non-Functional}\quad
  \badge{Performance}\ \badge{Security}\ \badge{Reliability}\ \badge{Accessibility}\ \badge{Privacy}\ \badge{i18n}

  \medskip
  \textbf{Acceptance Criteria (BDD)}
  \begin{description}[leftmargin=2.4cm, labelwidth=2.3cm, style=nextline, itemsep=2pt, topsep=2pt]
    \item[\textbf{Scenario}] Happy path
    \item[\textbf{Given}] the target repositories, environments, and program context are available
    \item[\textbf{When}] the \emph{Hands-on Objectives} for this chapter are executed
    \item[\textbf{Then}] the stated \emph{Outcomes/Deliverables} for this chapter are produced, reviewed, and published
  \end{description}

  \vspace{0.2\baselineskip}
  {\footnotesize\color{gray!60}\DoR\ \textbullet\ \DoD}
  \end{tcolorbox}
}

% --- Title ---
\title{\textbf{User Stories by Chapter:\\ Application Security Program Guide}}
\author{Compiled for Jordan Suber}
\date{}

\begin{document}
\maketitle
\tableofcontents
\newpage

\section*{How to Use This Template}
Each card maps one chapter’s \emph{Learning Goals} to a concise story, binds the chapter’s \emph{Hands-on Objectives} to concrete \emph{Tasks}, and verifies \emph{Outcomes} via BDD-style Acceptance Criteria. Import these cards into your backlog, tag by risk tier, and iterate.

\subsection*{Required Data on Every Story}
\begin{itemize}[itemsep=2pt,topsep=2pt]
  \item \textbf{ID} (e.g., APPSEC-1), \textbf{Title} (actionable verb), \textbf{Epic/Feature}, \textbf{Business Value} (outcome/why)
  \item \textbf{Priority} (Must/Should/Could), \textbf{Estimate} (SP), \textbf{Persona}, \textbf{Dependencies}, \textbf{Assumptions/Risks}
  \item \textbf{Acceptance Criteria} (Gherkin-ish BDD), \textbf{Tasks} (checklist), \textbf{NFR} (Security, Privacy, Reliability, etc.)
\end{itemize}

\subsection*{Writing Effective User Stories (Quick Guide)}
\textbf{Template:} As a \emph{[persona]}, I want to \emph{[do X]} so that \emph{[value/why]}.\\
\textbf{INVEST:} Independent, Negotiable, Valuable, Estimable, Small, Testable.\\
\textbf{Good:} “As an AppSec lead, I want a \emph{tiered SSDLC policy} so that \emph{teams ship securely with minimal friction}.”\\
\textbf{Anti-patterns:} Vague “Research X”; multi-team mega-stories; outputs without value (``create doc’’) unless tied to decision/change.

\newpage
\section{Stories by Chapter}



% ===== Reordered StoryCards (End-to-End Workflow) =====
\StoryCard{APPSEC-1}{Publish an AppSec Program Charter}{Program Foundations}
{align engineering, product, and risk on scope, value, and success criteria}
{Must}{3}
{AppSec lead}
{Org strategy, security policy, product roadmap}
{Scope creep risk; time-box charter v1 and plan iterative updates}
\begin{TasksBox}
  \item Draft a one-page charter: mission, scope, definitions, interfaces, success metrics.
  \item Create a stakeholder map and RACI for threat modeling, testing, vuln mgmt, IR.
  \item Review with Eng/Product/Risk; capture decisions and open questions.
  \item Publish in the handbook repo; version as living document.
\end{TasksBox}

\StoryCard{APPSEC-2}{Create a Control Dictionary \& Traceability Matrix}{Security Foundations}
{give engineers clear, shared definitions and connect policies to app controls}
{Must}{5}
{Security architect}
{Enterprise policies/standards}
{Terminology mismatch; include concrete code/config examples}
\begin{TasksBox}
  \item Compile key concepts (authn, authz, logging, crypto, secrets, input validation).
  \item Map each enterprise policy to concrete application controls and test evidence.
  \item Add links to code samples, lints, and CI checks for each control.
  \item Publish as \texttt{/docs/control-dictionary.md} and keep PR-able.
\end{TasksBox}

\StoryCard{APPSEC-3}{Build an Application Inventory \& Tiering}{Program Scope}
{focus effort on highest-risk apps; enable tiered controls and SLAs}
{Must}{5}
{Product security engineer}
{CMDB/source of truth; service catalog}
{Owner gaps; require ownership to promote to higher envs}
\begin{TasksBox}
  \item Inventory apps/services/APIs with owners, data classes, exposure, tech stack.
  \item Define tiering model (e.g., P0–P3) with criteria and examples.
  \item Record lifecycle (active/sunset), compliance drivers, and repo links.
  \item Export registry to CSV/JSON; integrate with CI labels per repo.
\end{TasksBox}

\StoryCard{APPSEC-4}{Stand Up an App Risk Register}{Risk Management}
{turn threats into tracked items tied to owners, dates, and treatments}
{Must}{3}
{Risk manager}
{Inventory completed, risk rubric}
{Over-long registers stall; keep to top risks per app}
\begin{TasksBox}
  \item Define likelihood/impact rubric and treatment options.
  \item Run a 60–90 min risk workshop for two critical apps.
  \item Create entries with owner, due date, and linkage to epics/stories.
  \item Establish intake workflow (new risk \(\rightarrow\) triage \(\rightarrow\) acceptance).
\end{TasksBox}

\StoryCard{APPSEC-5}{Publish Secure Reference Architectures}{Secure Design Patterns}
{give teams golden paths that bake in zero-trust and least privilege}
{Should}{5}
{Security architect}
{Architecture council, platform patterns}
{Architecture drift; add linters/policies to reinforce}
\begin{TasksBox}
  \item Diagram monolith, microservices, async/event-driven, and serverless patterns.
  \item Annotate controls per tier (authn, mTLS, input validation, logging, backups).
  \item Provide IaC/app templates implementing the patterns.
  \item Add “choose-by-facts” table and decision records (ADRs).
\end{TasksBox}

\StoryCard{APPSEC-6}{Adopt a Tiered SSDLC Policy}{SSDLC Alignment}
{embed right-sized checks by risk tier to shift left without friction}
{Must}{5}
{AppSec lead}
{Engineering buy-in, CI access}
{Over-gating; start minimal and ratchet}
\begin{TasksBox}
  \item Define controls per SDLC phase and per tier (ASVS/SSDF-aligned).
  \item Wire required checks in CI (lint, SAST, SCA) with pass/fail thresholds.
  \item Add DoD/DoR updates to team templates referencing security checks.
  \item Document exceptions/waivers with expiry and approval path.
\end{TasksBox}

\StoryCard{APPSEC-7}{Launch the AppSec Champions Program}{Operating Model \& Teams}
{scale AppSec via embedded advocates and faster issue resolution}
{Should}{3}
{AppSec lead}
{Managers’ support, time allocation}
{Attrition/adoption risk; include incentives and community time}
\begin{TasksBox}
  \item Define selection rubric, responsibilities, and incentives.
  \item Create monthly office hours and a champions Slack channel.
  \item Provide starter kit (checklists, threat modeling kit, PR review guide).
  \item Track participation and outcomes (bugs prevented, PRs reviewed).
\end{TasksBox}


% ===== Agile Application Security Stories =====
\StoryCard{APPSEC-34}{Define Security Definition of Ready (DoR) \\ \\ Definition of Done (DoD)}{Agile AppSec Foundations}{bake security into the team’s workflow gates so features ship with baseline controls}{Must}{3}{Scrum Master}{Agreed SSDLC policy; team working agreement}{Too heavy gates can slow delivery; right-size to risk tiers}
\begin{TasksBox}
\textbf{Tasks}
\begin{itemize}
  \item Document security DoR (threat model link, acceptance criteria, risk score)
  \item Document security DoD (tests green, SBOM present, secrets scan clean)
  \item Publish team board checklists and automate reminders
\end{itemize}

\textbf{Acceptance Criteria}
\begin{itemize}
  \item Security DoR/DoD approved and referenced in sprint templates
  \item PR template includes security checklist
  \item Pipeline enforces key DoD checks (fail on critical issues)
\end{itemize}
\end{TasksBox}

\StoryCard{APPSEC-35}{Create a Security Acceptance Criteria Library}{Agile AppSec Foundations}{accelerate secure delivery by reusing well-formed security ACs per story type}{Must}{5}{Product Owner}{Secure coding standards; ASVS mapping}{Generic ACs may not fit; allow tailoring per risk tier}
\begin{TasksBox}
\textbf{Tasks}
\begin{itemize}
  \item Curate AC snippets for auth, input validation, logging, PII handling
  \item Map ACs to ASVS controls and risk tiers
  \item Add AC snippets to backlog templates and story examples
\end{itemize}

\textbf{Acceptance Criteria}
\begin{itemize}
  \item AC library lives in repo/Wiki and is referenced by >80\% of new stories
  \item Each AC mapped to ASVS section and test evidence type
\end{itemize}
\end{TasksBox}

\StoryCard{APPSEC-36}{Stand Up a Security Backlog \\ \\ Risk Triage Kanban}{Agile AppSec Operations}{ensure visibility and flow for security work alongside product features}{Must}{5}{Product Security Engineer}{Control dictionary; risk register}{Security items may be starved; set WIP and capacity policies}
\begin{TasksBox}
\textbf{Tasks}
\begin{itemize}
  \item Create categories (hardening, testing, debt, education)
  \item Define SLA classes (expedite for criticals, standard, fixed-date)
  \item Integrate with bug tracker and CWE/CVSS tagging
\end{itemize}

\textbf{Acceptance Criteria}
\begin{itemize}
  \item Security backlog exists with WIP limits and classes of service
  \item Critical items auto-page, create expedite swimlane
\end{itemize}
\end{TasksBox}

\StoryCard{APPSEC-37}{Sprint 0 Security Enablement}{Agile AppSec Delivery}{set teams up for success with secure defaults before feature work begins}{Should}{8}{DevOps Engineer}{Reference architectures; templates available}{Rushing Sprint 0 leads to gaps; time-box essentials}
\begin{TasksBox}
\textbf{Tasks}
\begin{itemize}
  \item Provision repo templates with CI security jobs (SAST/SCA/secret scan)
  \item Generate baseline threat model and architecture diagram
  \item Seed env var policy, secret manager paths, logging/trace defaults
\end{itemize}

\textbf{Acceptance Criteria}
\begin{itemize}
  \item New repos inherit security CI and pass baseline checks
  \item Threat model ADR committed and linked in README
\end{itemize}
\end{TasksBox}

\StoryCard{APPSEC-38}{Security Champions Cadence \\ \\ Office Hours}{Agile AppSec Operations}{scale expertise via lightweight coaching and shared practices}{Should}{3}{AppSec Lead}{Champions program charter}{Low attendance risk; align with sprint rituals}
\begin{TasksBox}
\textbf{Tasks}
\begin{itemize}
  \item Hold bi-weekly office hours and monthly guild sessions
  \item Publish short playbooks and code examples
  \item Track engagement and topics to refine backlog
\end{itemize}

\textbf{Acceptance Criteria}
\begin{itemize}
  \item Attendance recorded; >70\% teams represented
  \item Two new playbooks published per quarter
\end{itemize}
\end{TasksBox}

\StoryCard{APPSEC-39}{Security Code Review Checklist \\ \\ Pairing}{Agile AppSec Delivery}{catch issues early by enriching PR reviews with targeted security checks}{Should}{5}{Senior Developer}{Secure coding standards; code owners defined}{Checklist fatigue; keep concise and role-based}
\begin{TasksBox}
\textbf{Tasks}
\begin{itemize}
  \item Create language/framework-specific checklists (input, authz, logging)
  \item Enable CODEOWNERS for sensitive paths (auth, crypto, infra)
  \item Pilot pairing/mobbing for risky changes
\end{itemize}

\textbf{Acceptance Criteria}
\begin{itemize}
  \item Checklist adopted in PR template; CODEOWNERS in repo
  \item Sampling shows >80\% PRs include security review notes
\end{itemize}
\end{TasksBox}

\StoryCard{APPSEC-40}{Security Test Harness in CI \\ (Unit, Integration, e2e)}{Agile AppSec Automation}{turn security ACs into repeatable tests that gate releases}{Must}{8}{DevOps Engineer}{CI runners; test data strategy}{Flaky tests disrupt delivery; quarantine policy required}
\begin{TasksBox}
\textbf{Tasks}
\begin{itemize}
  \item Translate ACs to tests (unit assertions, e2e negative cases)
  \item Add security smoke tests to PR/merge workflows
  \item Collect JUnit artifacts and trend failures
\end{itemize}

\textbf{Acceptance Criteria}
\begin{itemize}
  \item Security tests run on each PR and block on critical failures
  \item Dashboard shows pass rates per repo
\end{itemize}
\end{TasksBox}

\StoryCard{APPSEC-41}{Security SLOs/SLIs \\ Error Budgets}{Agile AppSec Metrics}{align risk tolerance with delivery by defining measurable targets}{Should}{5}{Product Owner}{Metrics dashboard pipeline}{Vanity metrics risk; tie SLIs to outcomes (vuln age, MTTR)}
\begin{TasksBox}
\textbf{Tasks}
\begin{itemize}
  \item Define SLIs (critical vuln age, secrets incidents, SBOM freshness)
  \item Set SLOs per tier; error budget burn alerts
  \item Review in sprint review/ops review
\end{itemize}

\textbf{Acceptance Criteria}
\begin{itemize}
  \item SLIs visible; SLOs approved by stakeholders
  \item Error budget policy documented and in use
\end{itemize}
\end{TasksBox}

\StoryCard{APPSEC-42}{Manage Security Debt \\ WIP Limits}{Agile AppSec Operations}{prevent accumulation of risk by reserving capacity for security work}{Should}{3}{Scrum Master}{Security backlog with classes of service}{Feature pressure can erode capacity; enforce WIP}
\begin{TasksBox}
\textbf{Tasks}
\begin{itemize}
  \item Reserve sprint capacity (e.g., 15–20\%) for security items
  \item Set WIP limits and visual policies on the board
  \item Track debt burndown
\end{itemize}

\textbf{Acceptance Criteria}
\begin{itemize}
  \item Capacity policy visible; burndown trends improving
  \item No sprint closes with critical debt untriaged
\end{itemize}
\end{TasksBox}

\StoryCard{APPSEC-43}{Lightweight Risk Exception \\ Time-Bound Waivers}{Agile AppSec Governance}{enable pragmatic shipping while controlling residual risk}{Could}{3}{Risk Manager}{Risk register; waiver workflow}{Waiver sprawl; enforce expirations and ownership}
\begin{TasksBox}
\textbf{Tasks}
\begin{itemize}
  \item Define exception template (owner, risk, compensating controls, expiry)
  \item Automate reminders and revoke on expiry
  \item Report exceptions in QBRs
\end{itemize}

\textbf{Acceptance Criteria}
\begin{itemize}
  \item All waivers have owners and expirations
  \item Expired waivers auto-alert and block releases if needed
\end{itemize}
\end{TasksBox}

\StoryCard{APPSEC-44}{Security Chaos/Game Days}{Agile AppSec Learning}{build muscle memory and validate controls under failure conditions}{Could}{5}{SRE Lead}{Staging environment; playbooks}{Customer impact risk; run in staging with guardrails}
\begin{TasksBox}
\textbf{Tasks}
\begin{itemize}
  \item Design adversarial scenarios (secret leak, token theft, SSRF attempts)
  \item Run drills with cross-functional teams
  \item Capture learnings and convert to backlog items
\end{itemize}

\textbf{Acceptance Criteria}
\begin{itemize}
  \item At least one drill per quarter with documented outcomes
  \item Follow-up stories created and prioritized
\end{itemize}
\end{TasksBox}

\StoryCard{APPSEC-45}{Release Readiness \\ Security Checklist}{Agile AppSec Delivery}{ensure releases meet baseline security before go-live}{Must}{3}{Release Manager}{DoD gates; metrics dashboard}{Last-minute crunch; automate checklist population}
\begin{TasksBox}
\textbf{Tasks}
\begin{itemize}
  \item Automate checklist (AC met, tests green, SBOM signed, secrets scan)
  \item Gate on unresolved criticals or expired waivers
  \item Publish release notes with security changes
\end{itemize}

\textbf{Acceptance Criteria}
\begin{itemize}
  \item Checklist artifact attached to each release
  \item No release proceeds with critical blockers
\end{itemize}
\end{TasksBox}

\StoryCard{APPSEC-46}{Continuous Education \\ Micro-Learning}{Agile AppSec Learning}{raise team capability with short, targeted security modules}{Could}{2}{Learning Lead}{Champions cadence; LMS}{Low engagement; keep modules <10 min tied to current work}
\begin{TasksBox}
\textbf{Tasks}
\begin{itemize}
  \item Publish bite-size modules (e.g., XSS in React, JWT pitfalls)
  \item Track completion and impact on defects
  \item Reward champions/teams who complete modules
\end{itemize}

\textbf{Acceptance Criteria}
\begin{itemize}
  \item Module catalog live; >60\% engineers complete at least one per quarter
  \item Correlation shows reduced related defects over time
\end{itemize}
\end{TasksBox}

\StoryCard{APPSEC-8}{Standardize Threat Modeling}{Threat Modeling}
{catch design flaws early and convert threats into actionable requirements}
{Must}{5}
{Security champion}
{DFD notation, templates}
{Analysis paralysis; time-box sessions and prioritize}
\begin{TasksBox}
  \item Choose method (STRIDE/LINDDUN/misuse cases) and templates.
  \item Run two sessions on different architectures; capture DFDs and threats.
  \item Translate top threats into NFRs and tests.
  \item Add a reusable threats/mitigations catalogue to the wiki.
\end{TasksBox}

\StoryCard{APPSEC-9}{Publish Secure Coding Standards}{Secure Coding}
{reduce recurring vulnerabilities and speed reviews with clear checklists}
{Must}{3}
{Tech lead}
{Language stacks agreed}
{One-size-fits-none risk; tailor per language}
\begin{TasksBox}
  \item Write per-language standards (input validation, encoding, secrets, crypto).
  \item Add PR checklists and reviewer heuristics.
  \item Provide pre-commit hooks and code templates.
  \item Run a 45-min training; record and link in the repo.
\end{TasksBox}

\StoryCard{APPSEC-12}{Enforce API Security Standards}{API Security}
{protect data and consumers via consistent auth, validation, and quotas}
{Must}{5}
{API owner}
{OpenAPI/AsyncAPI specs}
{Shadow APIs; tie standard to inventory}
\begin{TasksBox}
  \item Write API security standard (authn/z, schema validation, rate limiting).
  \item Add contract tests and security tests to CI.
  \item Gate breaking changes and insecure defaults in PRs.
  \item Add discovery checks for undocumented endpoints.
\end{TasksBox}

\StoryCard{APPSEC-10}{Operationalize SAST/SCA/DAST/IAST}{Security Testing}
{improve signal-to-noise and make security checks part of normal CI}
{Must}{5}
{Automation engineer}
{Scanner licenses, CI capacity}
{Finding overload; enforce “new high/critical = fail”}
\begin{TasksBox}
  \item Integrate SAST \& SCA in CI; upload SARIF for code scanning.
  \item Stand up targeted DAST/IAST for a high-risk app.
  \item Establish severity thresholds, suppressions with expiry, and routing.
  \item Publish weekly trend reports and backlog hygiene metrics.
\end{TasksBox}

\StoryCard{APPSEC-11}{Generate SBOMs \& Sign Artifacts}{Supply Chain Security}
{improve provenance and compliance while enabling safe updates}
{Must}{5}
{Release engineer}
{SBOM tool, signer}
{Tooling gaps; start with top languages/images}
\begin{TasksBox}
  \item Produce SBOM (CycloneDX/SPDX) during builds; attach to artifacts.
  \item Sign artifacts/images and verify in promotion gates.
  \item Document third-party source allowlist and review cadence.
  \item Add attestation checks to release workflow.
\end{TasksBox}


% ===== WAHH-inspired Manual Web App Testing Stories =====
\StoryCard{APPSEC-21}{Plan \\ Scope a Web App Penetration Test}{Web App Penetration Testing (WAHH)}{gain explicit scope, rules of engagement, and safe test windows to prevent production impact}{Must}{3}{Security tester}{Signed RoE; test accounts; staging/prod window}{Testing in prod may cause instability; throttle and monitor}
\begin{TasksBox}
  \item[\textbf{Tasks}]
    \begin{itemize}
      \item Define scope (domains, apps, APIs), out-of-scope targets, and credentials
      \item Document test data handling and PII safeguards
      \item Align comms, SLAs for critical findings, and retest windows
    \begin{itemize}
      \item[] % keep nested list structure valid
    \end{itemize}
    \end{itemize}

  \item[\textbf{Acceptance Criteria}]
    \begin{itemize}
      \item RoE doc approved by stakeholders
      \item Test accounts provisioned with role variants (user, admin, support)
      \item Monitoring/alerting teams notified of test window
    \end{itemize}
\end{TasksBox}

\StoryCard{APPSEC-22}{Reconnaissance \\ Application Mapping}{Web App Penetration Testing (WAHH)}{discover hidden attack surface to prioritize testing and coverage}{Must}{5}{Security tester}{Scope confirmed; wordlists; proxy + crawler}{Over-crawling may trigger rate limits; coordinate with SRE}
\begin{TasksBox}
  \item[\textbf{Tasks}]
    \begin{itemize}
      \item Map URLs, parameters, methods with an intercepting proxy
      \item Enumerate endpoints, SPA routes, and undocumented APIs
      \item Fingerprint frameworks, versions, and third-party components
    \end{itemize}

  \item[\textbf{Acceptance Criteria}]
    \begin{itemize}
      \item Site map exported with parameters and auth contexts
      \item List of potential high-risk surfaces identified (auth, upload, serialization)
    \end{itemize}
\end{TasksBox}

\StoryCard{APPSEC-23}{Test Authentication \\ Session Management}{Web App Penetration Testing (WAHH)}{prevent account takeover by finding flaws in login, MFA, and session controls}{Must}{8}{Security tester}{Accounts with/without MFA; password reset emails}{Lockouts during testing; ensure customer impact safeguards}
\begin{TasksBox}
  \item[\textbf{Tasks}]
    \begin{itemize}
      \item Probe MFA bypass, weak recovery flows, and magic-link abuse
      \item Assess session fixation/rotation, cookie flags, and idle timeouts
      \item Evaluate credential stuffing protections and lockout policies
    \end{itemize}

  \item[\textbf{Acceptance Criteria}]
    \begin{itemize}
      \item Documented results for MFA, recovery, and session rotation
      \item Remediation guidance aligned to OWASP ASVS controls
    \end{itemize}
\end{TasksBox}

\StoryCard{APPSEC-24}{Test Authorization \\ Access Control (IDOR/BOLA)}{Web App Penetration Testing (WAHH)}{stop horizontal/vertical privilege escalation via broken object-level auth}{Must}{8}{Security tester}{Multiple role accounts; seeded cross-tenant data}{Data exposure risk; use synthetic data}
\begin{TasksBox}
  \item[\textbf{Tasks}]
    \begin{itemize}
      \item Fuzz identifiers (IDs, GUIDs) and object references for IDOR/BOLA
      \item Probe multi-tenant boundaries; confirm server-side checks
      \item Check mass assignment and insecure direct mapping in APIs
    \end{itemize}

  \item[\textbf{Acceptance Criteria}]
    \begin{itemize}
      \item Evidence of any cross-tenant/object access or written 'no repro' with proof
      \item Mitigations mapped to enforcement in controllers/middleware
    \end{itemize}
\end{TasksBox}

\StoryCard{APPSEC-25}{Injection Testing (SQL/NoSQL/Command/LDAP)}{Web App Penetration Testing (WAHH)}{eliminate injection paths that lead to data breach or RCE}{Must}{13}{Security tester}{Safe test DB; command sandbox in staging}{Potential data corruption; use read-only techniques where possible}
\begin{TasksBox}
  \item[\textbf{Tasks}]
    \begin{itemize}
      \item Identify user-controlled inputs reaching interpreters
      \item Test with time-based, boolean, and error-based payloads
      \item Validate ORM parameterization and stored procedures
    \end{itemize}

  \item[\textbf{Acceptance Criteria}]
    \begin{itemize}
      \item List of vulnerable sinks with PoC payloads, impact, and severity
      \item Verification that parameterization/escaping prevents injection
    \end{itemize}
\end{TasksBox}

\StoryCard{APPSEC-26}{Cross-Site Scripting (Reflected/Stored/DOM)}{Web App Penetration Testing (WAHH)}{prevent account hijack and data theft via XSS in templates and SPA flows}{Must}{8}{Security tester}{CSP report URI; proxy instrumentation}{False negatives in SPA due to client-side routing; exhaustive param coverage needed}
\begin{TasksBox}
  \item[\textbf{Tasks}]
    \begin{itemize}
      \item Probe contexts (HTML, attribute, JS, URL, style) for escaping failures
      \item Verify CSP, output encoding, and template auto-escape settings
      \item DOM XSS checks in dynamic frameworks
    \end{itemize}

  \item[\textbf{Acceptance Criteria}]
    \begin{itemize}
      \item Any exploitable XSS documented with payload, context, and fix
      \item CSP evaluated; recommendations provided (nonce, strict-dynamic)
    \end{itemize}
\end{TasksBox}

\StoryCard{APPSEC-27}{CSRF \\ SameSite Protections}{Web App Penetration Testing (WAHH)}{block unauthorized state changes from cross-origin requests}{Should}{5}{Security tester}{Test harness for cross-origin forms/XHR/fetch}{CSRF tests may trigger state changes; only use reversible actions}
\begin{TasksBox}
  \item[\textbf{Tasks}]
    \begin{itemize}
      \item Validate anti-CSRF tokens, double-submit, and origin checks
      \item Verify cookie SameSite, secure flags, and CORS policies
      \item Test JSON/GraphQL mutations for CSRF gaps
    \end{itemize}

  \item[\textbf{Acceptance Criteria}]
    \begin{itemize}
      \item Critical state-changing routes confirmed protected or issues filed
      \item CORS and SameSite settings documented with recommendations
    \end{itemize}
\end{TasksBox}

\StoryCard{APPSEC-28}{File Upload \\ Path Traversal \\ RCE}{Web App Penetration Testing (WAHH)}{prevent arbitrary code execution and data exposure via unsafe file handling}{Must}{8}{Security tester}{Isolated storage; antivirus/sandbox rules}{Prod AV may quarantine test payloads; coordinate}
\begin{TasksBox}
  \item[\textbf{Tasks}]
    \begin{itemize}
      \item Test MIME/type/extension checks and content-sniffing bypasses
      \item Probe image/polyglot payloads and storage path traversal
      \item Validate media processing libraries for RCE vectors
    \end{itemize}

  \item[\textbf{Acceptance Criteria}]
    \begin{itemize}
      \item Uploads constrained by allowlist and verified server-side
      \item No traversal or remote execution demonstrated
    \end{itemize}
\end{TasksBox}

\StoryCard{APPSEC-29}{Deserialization \\ Cryptographic Failures}{Web App Penetration Testing (WAHH)}{mitigate code execution and privilege escalation through unsafe serialization and weak crypto}{Should}{8}{Security tester}{Known gadget chains in test env; key rotation docs}{Key leakage risk; use dummy keys in tests}
\begin{TasksBox}
  \item[\textbf{Tasks}]
    \begin{itemize}
      \item Identify serialization formats (Java, PHP, JWT, protobuf) and trust boundaries
      \item Attempt known gadget chains; check object injection paths
      \item Assess JWT alg confusion, weak signing, and key exposure
    \end{itemize}

  \item[\textbf{Acceptance Criteria}]
    \begin{itemize}
      \item Unsafe deserialization paths cataloged or remediated
      \item Crypto controls validated against ASVS (key mgmt, algs, rotation)
    \end{itemize}
\end{TasksBox}

\StoryCard{APPSEC-30}{SSRF/XXE \\ Server-Side Template Injection}{Web App Penetration Testing (WAHH)}{stop lateral movement to internal services and metadata endpoints}{Must}{8}{Security tester}{Egress controls; canary endpoints}{Risk of internal service impact; coordinate with platform team}
\begin{TasksBox}
  \item[\textbf{Tasks}]
    \begin{itemize}
      \item Probe URL fetchers and XML parsers for SSRF/XXE
      \item Validate denylists/allowlists, outbound proxy, and metadata protections
      \item Test template engines for SSTI to RCE chains
    \end{itemize}

  \item[\textbf{Acceptance Criteria}]
    \begin{itemize}
      \item No internal egress or metadata access possible without policy
      \item Template engines hardened or issues raised with PoCs
    \end{itemize}
\end{TasksBox}

\StoryCard{APPSEC-31}{Business Logic Abuse \\ Rate Limiting \\ Automation}{Web App Penetration Testing (WAHH)}{protect revenue and integrity by preventing workflow abuse and brute force}{Should}{5}{Security tester}{Analytics dashboards; throttling configs}{Blocking legitimate users during tests; throttle carefully}
\begin{TasksBox}
  \item[\textbf{Tasks}]
    \begin{itemize}
      \item Enumerate critical workflows (checkout, transfers, promotions)
      \item Test replay, race conditions, and coupon abuse
      \item Evaluate rate limiting, CAPTCHA, and bot defenses
    \end{itemize}

  \item[\textbf{Acceptance Criteria}]
    \begin{itemize}
      \item Abuse scenarios documented with loss estimates and fixes
      \item Effective rate limits in place for sensitive endpoints
    \end{itemize}
\end{TasksBox}

\StoryCard{APPSEC-32}{Clickjacking \\ Caching \\ Sensitive Data Exposure}{Web App Penetration Testing (WAHH)}{reduce data leakage and UI redress attacks}{Could}{3}{Security tester}{Response headers report; CDN config}{Cache poisoning risk; test in staging when possible}
\begin{TasksBox}
  \item[\textbf{Tasks}]
    \begin{itemize}
      \item Verify X-Frame-Options/Content-Security-Policy frame-ancestors
      \item Check cache-control on authenticated responses
      \item Scan for sensitive data in URLs, logs, and client storage
    \end{itemize}

  \item[\textbf{Acceptance Criteria}]
    \begin{itemize}
      \item Headers configured defensively (no-store where needed)
      \item No sensitive data found in caches or client-side storage
    \end{itemize}
\end{TasksBox}

\StoryCard{APPSEC-33}{Report \\ Triage \\ Retest Findings}{Web App Penetration Testing (WAHH)}{translate findings into engineering work, validate fixes, and build learning loops}{Must}{5}{Security tester}{Ticketing templates; CWE/CVRSS mapping}{Fix regressions possible; ensure retest scripts are reusable}
\begin{TasksBox}
  \item[\textbf{Tasks}]
    \begin{itemize}
      \item Create tickets with repro steps, impact, CWE, and severity
      \item Partner with owners on fixes and timelines
      \item Retest and close with evidence; update knowledge base
    \end{itemize}

  \item[\textbf{Acceptance Criteria}]
    \begin{itemize}
      \item All critical/high issues triaged within SLA and retested
      \item KB updated with playbooks and examples
    \end{itemize}
\end{TasksBox}

\StoryCard{APPSEC-13}{Publish Cloud AppSec Baseline}{Cloud-Native App Security}
{set secure defaults for identity, secrets, network, and logging}
{Should}{3}
{Cloud security engineer}
{Cloud org access}
{Drift risk; add config conformance packs}
\begin{TasksBox}
  \item Define shared-responsibility for app teams; list must-have controls.
  \item Provide bootstrap templates for logging/telemetry and secrets.
  \item Add guardrails and conformance checks.
  \item Document carve-outs and exception review.
\end{TasksBox}

\StoryCard{APPSEC-14}{Harden Containers \& Kubernetes}{Container/K8s Security}
{reduce runtime risk with minimal images and admission policies}
{Must}{5}
{Platform engineer}
{Registry, admission controller}
{Breakages; start in warn mode, then enforce}
\begin{TasksBox}
  \item Create minimal, scanned base images; publish usage guidance.
  \item Enforce image provenance and vulnerability thresholds at admission.
  \item Apply Pod Security standards, RBAC, and NetworkPolicies.
  \item Add runtime policies for sensitive syscalls and egress.
\end{TasksBox}

\StoryCard{APPSEC-15}{Centralize Secrets \& Workload Identity}{Secrets \& IAM}
{eliminate hardcoded secrets and reduce blast radius via least privilege}
{Must}{3}
{Service owner}
{Secrets manager, IAM}
{Migration risk; migrate one app first}
\begin{TasksBox}
  \item Move secrets to a managed store with rotation.
  \item Adopt workload identity (mTLS/JWT/OIDC) for services.
  \item Review and minimize IAM policies per service.
  \item Add secrets scanning in CI and pre-commit.
\end{TasksBox}


% ===== Policy as Code Stories (end-to-end) =====
\StoryCard{APPSEC-47}{Define Policy-as-Code Strategy \\ \\ Reference Architecture}{Policy as Code}{create consistent, testable guardrails across repos, pipelines, cloud, and clusters}{Must}{5}{Security Architect}{SSDLC policy; cloud/K8s baselines; CI access}{Too many frameworks increases toil; pick minimal viable set}
\begin{TasksBox}
\textbf{Tasks}
\begin{itemize}
  \item Select core frameworks and scopes: OPA/Rego (Conftest bundles), Gatekeeper/Kyverno (K8s), IaC checks (Terraform plans), pipeline policies
  \item Define target enforcement points: pre-commit, PR, CI, admission, deploy, runtime
  \item Write an ADR documenting choices, bundle layout, versioning, and promotion model (dev\(\rightarrow\)stg\(\rightarrow\)prod)
\end{itemize}

\textbf{Acceptance Criteria}
\begin{itemize}
  \item Reference architecture approved by Platform, AppSec, and SRE
  \item Hello-world policy proven in one repo and one cluster in \emph{audit} mode
  \item Docs published: “How policies run” + developer quickstart
\end{itemize}
\end{TasksBox}

\StoryCard{APPSEC-48}{Author Baseline Policy Library}{Policy as Code}{codify critical controls (secrets, SBOM, least privilege, network) with reusable rules}{Must}{8}{Policy Engineer}{Reference architecture; control dictionary}{Over-blocking risk; start with \emph{audit} severity and tune}
\begin{TasksBox}
\textbf{Tasks}
\begin{itemize}
  \item Write baseline policies: repo (branch protection, required checks), CI (required SAST/SCA), IaC (public buckets, open SGs, unencrypted volumes), K8s (PSa, runAsNonRoot, image provenance), Cloud (IAM wildcard deny)
  \item Provide passing/failing examples and unit tests (e.g., \texttt{rego} tests) for each rule
  \item Tag rules by tier (P0–P3) and map to ASVS/SSDF controls
\end{itemize}

\textbf{Acceptance Criteria}
\begin{itemize}
  \item Library stored as versioned bundles with tests passing in CI
  \item Each rule has rationale, remediation text, and references
\end{itemize}
\end{TasksBox}

\StoryCard{APPSEC-49}{Build Local Dev Tooling \\ \\ Pre-Commit Experience}{Policy as Code}{shift-left feedback via IDE/CLI so engineers fix before PR}{Should}{5}{Developer Experience Lead}{Baseline policy library}{Tool friction; ensure fast local runs}
\begin{TasksBox}
\textbf{Tasks}
\begin{itemize}
  \item Publish \texttt{make policy-test} + \texttt{pre-commit} hooks (conftest, yaml/json/plan inputs)
  \item Ship IDE tasks/snippets and a sample app showing policy passes/fails
  \item Document troubleshooting and rule suppression with expiry metadata
\end{itemize}

\textbf{Acceptance Criteria}
\begin{itemize}
  \item New repos enable pre-commit in <5 min and get local results <2s
  \item Suppressions require owner, ticket, expiry; flagged in CI on expiry
\end{itemize}
\end{TasksBox}

\StoryCard{APPSEC-50}{Integrate Policies into CI/CD \\ \\ Admission \& Deploy Gates}{Policy as Code}{prevent risky changes by gating merges and deploys with policy checks}{Must}{8}{Platform Engineer}{CI runners; admission controller; registry access}{Breaking builds en masse; roll out by cohort and audit-first}
\begin{TasksBox}
\textbf{Tasks}
\begin{itemize}
  \item Add conftest checks to PRs (IaC, manifests, pipeline config); publish SARIF annotations
  \item Install Gatekeeper/Kyverno; onboard namespaces in \emph{audit} then \emph{enforce}
  \item Enforce image provenance/SBOM signature at admission; block on criticals
\end{itemize}

\textbf{Acceptance Criteria}
\begin{itemize}
  \item CI fails for new critical violations; admission denies non-compliant pods/images
  \item Rollout plan tracked; <2\% false-positive rate post-tuning
\end{itemize}
\end{TasksBox}

\StoryCard{APPSEC-51}{Exceptions/Waivers as Code}{Policy as Code}{enable pragmatic delivery with time-bound, reviewable exceptions}{Must}{5}{Risk Manager}{Risk register; waiver workflow}{Shadow waivers; require owners and expirations}
\begin{TasksBox}
\textbf{Tasks}
\begin{itemize}
  \item Define waiver schema (owner, risk, justification, compensating controls, expiry)
  \item Store waivers near code (YAML/CRD); policies read waivers at evaluate-time
  \item Auto-alert before expiry; block builds on expired waivers
\end{itemize}

\textbf{Acceptance Criteria}
\begin{itemize}
  \item All policy suppressions reference a waiver ID and ticket
  \item Quarterly review report lists active/expired waivers by service
\end{itemize}
\end{TasksBox}

\StoryCard{APPSEC-52}{Policy Telemetry \\ \\ Dashboards \& Coverage}{Policy as Code}{observe adoption, denials, and drift to guide improvements}{Should}{5}{Program Manager}{Logging backend; metrics stack}{Noisy logs; sample and aggregate wisely}
\begin{TasksBox}
\textbf{Tasks}
\begin{itemize}
  \item Collect decision logs (OPA), admission denials, CI failures; tag by app/tier/team
  \item Build dashboard: pass/fail rates, top rules hit, time-to-fix, waiver counts
  \item Track coverage: \% repos with CI checks; \% namespaces enforcing; \% images verified
\end{itemize}

\textbf{Acceptance Criteria}
\begin{itemize}
  \item Monthly report shows improving coverage and reduced critical violations
  \item Error budget alerts for rising denial rates or stale waivers
\end{itemize}
\end{TasksBox}

\StoryCard{APPSEC-53}{Policy Bundles Registry \\ \\ Versioning \& Promotion}{Policy as Code}{safely evolve policies via semantic versions and environment promotion}{Should}{3}{Release Engineer}{OCI registry or artifact store}{Drift across envs; automate promotions}
\begin{TasksBox}
\textbf{Tasks}
\begin{itemize}
  \item Package policy bundles; publish to OCI registry with semver and changelogs
  \item Automate promotion (dev\(\rightarrow\)stg\(\rightarrow\)prod) after smoke-tests
  \item Define deprecation policy and migration guides for breaking changes
\end{itemize}

\textbf{Acceptance Criteria}
\begin{itemize}
  \item Envs reference immutable bundle digests
  \item Rollbacks possible by pinning previous versions
\end{itemize}
\end{TasksBox}


% ===== Security as Code (End-to-End Workflow) =====
\StoryCard{APPSEC-54}{Define Security Vision, Threats, and Controls}{Security as Code Foundations}
{align the team on risks and codify controls that will be enforced by pipelines}
{Must}{3}
{Platform engineer}
{Sample app repo; sandbox account}
{Over-scoping threat model; keep to top 5 risks}
\begin{TasksBox}
  \item Add \texttt{docs/security-vision.md}: goals, assumptions, non-goals.
  \item Create a 1-page STRIDE-lite model for the app \& cloud footprint.
  \item Publish a control catalog CSV with owner, evidence, and CI gate mapping.
  \item Link all of the above from the README; set a quarterly review.
\end{TasksBox}

\StoryCard{APPSEC-55}{Bootstrap IaC \& CI Foundations}{Security as Code Foundations}
{create a reproducible base that enables automated security checks}
{Must}{5}
{DevOps engineer}
{Artifact bucket/registry; CI runners}
{Leaked secrets risk; adopt OIDC and pre-commit scanners}
\begin{TasksBox}
  \item Provision minimal VPC, registry, and CI roles via IaC (encrypted by default).
  \item Pipeline builds container, runs linters and SCA, pushes image to registry.
  \item Enable pre-commit hooks (\texttt{tfsec}/\texttt{cfn-lint}, \texttt{hadolint}, secrets scan).
  \item Protect \texttt{main}: require passing checks; show badge in README.
\end{TasksBox}

\StoryCard{APPSEC-56}{Preventive \& Detective Controls as Code}{Security as Code Controls}
{block misconfigs before deploy and surface evidence automatically}
{Must}{8}
{Security champion}
{Working CI; IaC modules}
{False positives; add waivers with time-boxed expiry}
\begin{TasksBox}
  \item Write guard/OPA policies: no public buckets, encryption-at-rest, deny wildcard IAM.
  \item Enable Security Hub/GuardDuty/Config rules; encrypt logs with KMS.
  \item Add a \texttt{policy-check} job that fails on violations and posts rule summaries.
  \item Emit a control-coverage matrix artifact and link in job summary.
\end{TasksBox}

\StoryCard{APPSEC-57}{Centralize Telemetry \& Alerts}{Security as Code Observability}
{improve detection/triage via standard logs, metrics, and alarms as code}
{Must}{5}
{SRE / observability engineer}
{KMS keys; log shipping}
{Alert fatigue; tune severities and routes}
\begin{TasksBox}
  \item Enable org CloudTrail; VPC Flow Logs; cluster audit logs with retention.
  \item Emit app logs as structured JSON with correlation IDs.
  \item Create alarms for auth failures, 5xx spikes, throttling, and unusual egress.
  \item Build a dashboard JSON and link it from the README.
\end{TasksBox}

\StoryCard{APPSEC-58}{Automate Access (IAM, RBAC, IRSA)}{Security as Code Access Control}
{reduce standing privileges and make access auditable end-to-end}
{Must}{5}
{Cloud security engineer}
{EKS/ECS/OIDC configured}
{Privilege creep; schedule periodic reviews}
\begin{TasksBox}
  \item Adopt IRSA/OIDC for workloads; remove node-wide credentials.
  \item Generate least-priv IAM with Access Analyzer and validate in CI.
  \item Define Kubernetes RBAC via GitOps; separate dev/ops permissions.
  \item Add break-glass role with MFA and session recording.
\end{TasksBox}

\StoryCard{APPSEC-59}{Secrets Hygiene as Code}{Security as Code Secrets}
{prevent credential leaks and shrink blast radius}
{Must}{5}
{Dev lead}
{Pre-commit configured}
{Developer friction; provide quick-fix guidance}
\begin{TasksBox}
  \item Add secrets scanning in pre-commit and CI with org allowlist.
  \item Block merges on new high-sev matches; allow time-bound waivers.
  \item Publish rotation runbook; integrate auto-revocation for leaked keys.
\end{TasksBox}

\StoryCard{APPSEC-60}{Vault Integration \& Rotation}{Security as Code Secrets}
{eliminate static credentials and automate rotation evidence}
{Must}{5}
{Platform engineer}
{Secrets manager/Vault; CSI driver}
{Migration risk; start with one service}
\begin{TasksBox}
  \item Inject app config via CSI/env-from; remove plaintext secrets from repo.
  \item Configure rotation for DB/API keys; surface status in CI.
  \item Add policy test that fails if opaque K8s Secrets hold known sensitive patterns.
\end{TasksBox}

\StoryCard{APPSEC-61}{Container Hardening as Code}{Security as Code Supply Chain}
{standardize minimal, non-root images and enforce at deploy}
{Should}{5}
{Senior developer}
{Registry; base images}
{Breakages from base changes; canary rollout}
\begin{TasksBox}
  \item Provide hardened base images (non-root, pinned digests) and usage guide.
  \item Add \texttt{hadolint} \& \texttt{trivy image} with thresholds to CI.
  \item Enforce rootless, read-only FS via Helm/K8s manifests.
\end{TasksBox}

\StoryCard{APPSEC-62}{SBOM, Provenance \& Signing}{Security as Code Supply Chain}
{improve provenance and verify artifacts automatically}
{Must}{5}
{Release engineer}
{Cosign/Sigstore; CycloneDX/SPDX}
{Tooling variance; start with top services}
\begin{TasksBox}
  \item Generate SBOMs during build and publish as CI artifacts.
  \item Sign images and attest build provenance; verify at admission.
  \item Document KMS key rotation for signing; add failure runbook.
\end{TasksBox}

\StoryCard{APPSEC-63}{Security Unit \& Contract Tests}{Security as Code Testing}
{convert requirements to executable checks that block risky changes}
{Must}{5}
{QA engineer}
{AC library; test data}
{Flaky tests; add quarantine/nightly runs}
\begin{TasksBox}
  \item Add negative unit tests (authz, validation, encoding boundaries).
  \item Generate contract tests from OpenAPI (auth scopes, rate limits, schema).
  \item Publish JUnit; gate merges on critical failures.
\end{TasksBox}

\StoryCard{APPSEC-64}{API Security Tests in CI}{Security as Code Testing}
{prevent BOLA/IDOR and unsafe defaults with repeatable checks}
{Should}{5}
{Security tester}
{OpenAPI/GraphQL schema}
{Synthetic data required; avoid real PII}
\begin{TasksBox}
  \item Fuzz IDs with multi-identity accounts to detect IDOR/BOLA.
  \item Validate scopes/claims on sensitive endpoints; test CSRF/CORS.
  \item Fail pipeline on exploitable findings; auto-file tickets with repro.
\end{TasksBox}

\StoryCard{APPSEC-65}{Continuous Fuzzing as Code}{Security as Code Testing}
{discover edge-case bugs via coverage-guided fuzzing}
{Could}{5}
{DevOps engineer}
{Fuzz harnesses}
{Compute cost; run nightly for depth}
\begin{TasksBox}
  \item Add fuzzers for parsers/critical libs; short run on PRs.
  \item Extended fuzz nightly; publish minimized crashes as artifacts.
\end{TasksBox}

\StoryCard{APPSEC-66}{Release Readiness as Code}{Security as Code Release}
{ensure releases meet baseline security and ship evidence}
{Must}{3}
{Release manager}
{Previous SAC stories complete}
{Last-minute surprises; precompute checklist}
\begin{TasksBox}
  \item Generate release checklist (AC met, tests green, SBOM present, signatures valid, secrets scan clean).
  \item Block release on criticals or expired waivers; publish security notes.
\end{TasksBox}

\StoryCard{APPSEC-67}{Runtime Detection Rules as Code}{Security as Code Runtime}
{detect abuse/misuse with declarative runtime policies}
{Should}{5}
{SRE lead}
{Centralized logs/metrics}
{Noise risk; tune with incident feedback}
\begin{TasksBox}
  \item Deploy eBPF/Falco rules for exec in containers, sensitive file access, outbound spikes.
  \item Route alerts with enriched context (pod, image digest, commit SHA).
\end{TasksBox}

\StoryCard{APPSEC-68}{Compliance Mapping \& Validations}{Security as Code Compliance}
{prove control effectiveness continuously}
{Should}{5}
{Program manager}
{Control catalog}
{Stale mappings; auto-generate from source}
\begin{TasksBox}
  \item Map controls to CIS/SSDF in machine-readable form (CSV/OSCAL).
  \item Schedule validations (InSpec/Conftest) and export pass/fail to a lake.
  \item Generate monthly effectiveness report with trends.
\end{TasksBox}

\StoryCard{APPSEC-69}{Drift Detection \& Auto-Remediation}{Security as Code Operations}
{reduce exposure by catching and fixing drift quickly}
{Should}{5}
{Platform engineer}
{GitOps desired state}
{False remediation risk; start with suggest/fix PRs}
\begin{TasksBox}
  \item Enable drift detectors; post annotated diffs to PRs.
  \item Auto-open remediation PRs for low-risk drifts; page on critical drift.
\end{TasksBox}

\StoryCard{APPSEC-70}{Evidence Pipeline \& Dashboards}{Security as Code Metrics}
{make posture visible and self-serve to product teams}
{Must}{3}
{Data engineer}
{CI artifacts; logs; SBOMs}
{Data sprawl; define a minimal schema}
\begin{TasksBox}
  \item Ingest JUnit, SARIF, SBOMs, attestations into a lake with app/tier labels.
  \item Build dashboards: pass/fail rates, vuln age, waiver counts, coverage \%.
  \item Publish team scorecards and quarterly trend reports.
\end{TasksBox}


\StoryCard{APPSEC-16}{Unify Vulnerability Intake \& SLAs}{Vulnerability Management}
{prioritize by exploitability and asset criticality to reduce MTTR}
{Must}{5}
{Vuln management owner}
{Scanner feeds, ticketing}
{Duplicate noise; dedupe by CWE/package/asset}
\begin{TasksBox}
  \item Define prioritization (CVSS/EPSS + criticality + exposure).
  \item Create unified intake and dedup logic across code/deps/containers/infra.
  \item Set SLAs per tier and auto-create tickets with owners and due dates.
  \item Build dashboard (age buckets, MTTR, reopen rate).
\end{TasksBox}

\StoryCard{APPSEC-17}{Integrate AppSec into Incident Response}{App IR}
{speed containment and comms for app-specific incidents}
{Should}{3}
{IR lead}
{On-call schedule, playbooks}
{Confusion in roles; publish contact matrix}
\begin{TasksBox}
  \item Write app-centric playbooks (auth bypass, data exfil, supply-chain).
  \item Define evidence capture and comms templates (legal/regulatory triggers).
  \item Run a tabletop; record actions and owners.
  \item Add lessons learned template and review cadence.
\end{TasksBox}

\StoryCard{APPSEC-18}{Set AI/ML Security Guardrails}{AI/ML Security}
{prevent model abuse and data leakage with standards and tests}
{Could}{5}
{ML product owner}
{Model inventory, logs}
{Novel threats; start with one model/feature}
\begin{TasksBox}
  \item Threat-model one ML feature (prompt injection, data poisoning, model theft).
  \item Add adversarial test cases and output filters.
  \item Log model interactions for abuse patterns.
  \item Document red-team scenarios and escalation paths.
\end{TasksBox}

\StoryCard{APPSEC-19}{Automate Evidence \& ChatOps}{Automation \& Orchestration}
{reduce toil and raise adoption with bots, policies-as-code, and summaries}
{Should}{3}
{Automation engineer}
{Bot account, APIs}
{Alert fatigue; keep messages concise with links}
\begin{TasksBox}
  \item Auto-comment PRs with scanner summaries and fix hints.
  \item Scaffold “new service” with secure defaults via a bot command.
  \item Export evidence (SBOM, test reports, approvals) automatically.
  \item Maintain an automation backlog with value stream mapping.
\end{TasksBox}

\StoryCard{APPSEC-20}{Ship Metrics Dashboard \& Maturity Plan}{Metrics \& Maturity}
{prove risk reduction and align roadmap with measurable outcomes}
{Must}{3}
{Program manager}
{Data sources, dashboard tool}
{Metric cargo-cult; define glossary and collection method}
\begin{TasksBox}
  \item Choose north-star KPIs (risk reduced, MTTR, escape rate) and definitions.
  \item Build a dashboard with trends and targets; segment by tier/team.
  \item Run baseline maturity assessment (e.g., SAMM) and publish a 12-month plan.
  \item Review quarterly and adjust priorities based on results.
\end{TasksBox}

\section*{Capstone \& Milestones (Reference)}
\textbf{Foundation:} Charter, control dictionary, inventory/tiering, risk register.\\
\textbf{Build-in Security:} Reference architectures, SSDLC, champions, secure coding, testing.\\
\textbf{Platform Guardrails:} SBOM/signing, API/cloud/K8s baselines, secrets/IAM, \textbf{policy as code}, \textbf{security as code}.\\
\textbf{Operate \& Improve:} Vuln SLAs, App IR, AI/ML guardrails, automation, metrics+maturity.

\end{document}
