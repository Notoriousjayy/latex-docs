
\documentclass[11pt,a4paper]{article}

\usepackage[T1]{fontenc}
\usepackage[utf8]{inputenc}
\usepackage{lmodern}
\usepackage{inconsolata}
\usepackage{upquote}
\usepackage{microtype}
\usepackage[margin=1in]{geometry}
\usepackage{parskip}
\usepackage[hyphens]{url}
\usepackage{hyperref}
\usepackage{bookmark}
\usepackage{enumitem}
\usepackage{ragged2e}
\usepackage{array}
\usepackage{tabularx}
\usepackage[most]{tcolorbox}
\usepackage{xcolor}
\usepackage{amssymb,amsmath}

\definecolor{Primary}{HTML}{0E7490}
\definecolor{Accent}{HTML}{0EA5E9}
\definecolor{Soft}{HTML}{F1F5F9}
\definecolor{Ink}{HTML}{0F172A}
\definecolor{Meta}{HTML}{475569}

\hypersetup{colorlinks=true, linkcolor=Primary, urlcolor=Primary}

\tcbset{colback=Soft, colframe=Meta!60, colbacktitle=Soft!60, coltitle=black, arc=2pt, boxrule=0.4pt}

\newtcbox{\pill}{on line, arc=3pt, boxsep=0.8pt, left=4pt,right=4pt,top=1pt,bottom=1pt,
  colframe=Meta!60, colback=Soft!40, boxrule=0.3pt}
\newcommand{\badge}[1]{\pill{\footnotesize #1}}
\newcommand{\checkbox}{\(\square\)}
\newcommand{\checkedbox}{\(\blacksquare\)}
\newcommand{\DoR}{\textbf{Definition of Ready:} Persona clear; AC drafted; Dependencies known; Estimate set.}
\newcommand{\DoD}{\textbf{Definition of Done:} All ACs pass; Tests green; Security/a11y checks; Docs updated; Deployed/flagged.}
\let\cb\checkbox

% Story card box
\newtcolorbox{storyframe}[1]{enhanced,breakable,
  left=8pt,right=8pt,top=8pt,bottom=8pt,
  borderline west={2pt}{0pt}{Primary},
  title={\textbf{#1}}, fonttitle=\bfseries\large}

% Tasks box
\newenvironment{TasksBox}[1][Tasks]{%
  \begin{tcolorbox}[enhanced,breakable,
    colback=Soft!10, colframe=Meta!60,
    colbacktitle=Soft!60, coltitle=black,
    title={#1}, fonttitle=\bfseries,
    borderline west={1.8pt}{0pt}{Primary},
    left=10pt,right=10pt,top=6pt,bottom=6pt,
    before skip=6pt, after skip=10pt]
  \small
  \begin{itemize}[label=\cb, leftmargin=*, labelsep=0.6em, itemsep=4pt, topsep=2pt, parsep=0pt]
}{%
  \end{itemize}
  \end{tcolorbox}
}

% StoryCard macro (no custom coltypes; use explicit p{..} and X)
\newcommand{\StoryCard}[9]{%
  \begin{storyframe}{#1 — #2}
  \small
  \begin{tabularx}{\textwidth}{@{}>{\RaggedLeft\arraybackslash\bfseries}p{3.2cm}>{\RaggedRight\arraybackslash}X@{}}
    Epic / Feature          & #3 \\
    Business Value          & #4 \\
    Priority / Estimate     & \badge{Priority: #5}\ \badge{SP: #6} \\
    Persona                 & #7 \\
    Dependencies            & #8 \\
    Assumptions / Risks     & #9 \\
  \end{tabularx}

  \medskip
  \textbf{Story}\quad
  \emph{As a #7, I want to #2 so that #4.}

  \medskip
  \textbf{Non-Functional}\quad
  \badge{Performance}\ \badge{Security}\ \badge{Reliability}\ \badge{Accessibility}\ \badge{Privacy}\ \badge{i18n}

  \medskip
  \textbf{Acceptance Criteria (BDD)}
  \begin{description}[leftmargin=2.4cm,labelsep=1em]
    \item[\textbf{Scenario}] Happy path
    \item[\textbf{Given}] the target repositories, environments, and context are available
    \item[\textbf{When}] the Hands-on Objectives for this chapter are executed
    \item[\textbf{Then}] the stated Outcomes/Deliverables for this chapter are produced, reviewed, and published
  \end{description}

  {\footnotesize\color{Meta}\DoR\ \textbullet\ \DoD}
  \end{storyframe}
}

\title{\textbf{Study Plan — NGINX Cookbook, 3rd Edition}\\
\large User Story Template \& Story Card Definition}
\author{}
\date{\today}

\begin{document}
\maketitle
\tableofcontents
\clearpage

\section{How to Use This Template}
\begin{enumerate}
  \item Start with the \textbf{one-sentence story} (persona, goal, value).
  \item Add \textbf{acceptance criteria} in Gherkin (Given/When/Then).
  \item Capture \textbf{non-functional requirements} (performance, security, accessibility, \ldots).
  \item Confirm \textbf{Definition of Ready} (DoR) before sprint; confirm \textbf{Definition of Done} (DoD) before acceptance.
  \item Keep stories \textbf{INVEST}: Independent, Negotiable, Valuable, Estimable, Small, Testable.
\end{enumerate}

\clearpage
\section{Study Plan — Story Cards by Chapter}

% --- Cards ---
\StoryCard{NG-1}{Getting Started}{Foundations}
{Establish a clean base install and config layout to reduce future integration risk.}
{Must}{3}{Platform engineer}
{Package repo access; VM/container; sudo}{Port conflicts or SELinux/AppArmor blocks}

\begin{TasksBox}
\item Provision a VM/container and open port 80.
\item Install NGINX; enable via systemd.
\item Create a minimal site; modularize with \texttt{include}.
\item Add access/error logs with rotation.
\item Validate with \texttt{nginx -t} and \texttt{curl -I}; commit baseline configs and README.
\end{TasksBox}

\clearpage
\StoryCard{NG-2}{High Performance Load Balancing}{Traffic Distribution}
{Improve latency and resiliency by balancing across multiple backends with health checks.}
{Must}{5}{Traffic engineer}
{Two backends; load generator; ability to stop a node}
{Stateful apps may require stickiness; uneven load under spikes}

\begin{TasksBox}
\item Define an \texttt{upstream} with two services; expose via a \texttt{server} block.
\item Enable passive checks and tune timeouts; optionally enable slow start.
\item Benchmark policies (round robin, least conn, IP hash) with \texttt{wrk}.
\item Simulate node failure and verify failover and recovery.
\item Commit configs and a short decision record.
\end{TasksBox}

\clearpage
\StoryCard{NG-3}{Traffic Management and Shaping}{Smart Routing}
{Protect service quality and enable safe releases with canaries and rate limits.}
{Must}{5}{Release engineer}
{Header/cookie routing; GeoIP DB if used}
{Aggressive limits can block legitimate users; canary must be measurable}

\begin{TasksBox}
\item Implement header/cookie based split routing to a canary upstream.
\item Configure \texttt{limit\_conn} and \texttt{limit\_req} with a shared zone.
\item Add \texttt{real\_ip} to preserve client IP through proxies.
\item Document playbook for throttling and unthrottling.
\end{TasksBox}

\clearpage
\StoryCard{NG-4}{Massively Scalable Content Caching}{Edge Caching}
{Reduce latency and backend load by serving cached content with safe staleness.}
{Must}{5}{Edge engineer}
{Writable cache path; disk space; cache key design}
{Improper keys can leak personalized data; bypass rules required for authenticated users}

\begin{TasksBox}
\item Define cache zone and keys; enable \texttt{proxy\_cache\_lock}.
\item Add \texttt{proxy\_cache\_use\_stale} for safe error codes.
\item Implement bypass for authenticated users; purge admin endpoints if available.
\item Log cache status and add a Grafana panel.
\end{TasksBox}

\clearpage
\StoryCard{NG-5}{Programmability and Automation}{Automation}
{Lower toil and errors with declarative rollouts and simple scripting.}
{Should}{3}{Platform engineer}
{Config templates; VCS; CI runner}
{Templating mistakes can break routes; require \texttt{nginx -t} gates}

\begin{TasksBox}
\item Create config templates with variables for upstreams and routes.
\item Write an Ansible playbook to upload and reload on change.
\item Add CI job that runs \texttt{nginx -t} before deploy.
\item Add rollback procedure to the runbook.
\end{TasksBox}

\clearpage
\StoryCard{NG-6}{Authentication and Authorization}{Perimeter Security}
{Protect sensitive endpoints and enable single sign-on.}
{Must}{5}{Security engineer}
{Auth service/IdP; test JWTs}
{Clock skew breaks signatures; misconfigured bypass routes leak data}

\begin{TasksBox}
\item Protect \texttt{/admin} with \texttt{auth\_request}.
\item Validate JWTs on \texttt{/api} (alg, iss, aud).
\item Add clock skew tolerance and JWKS rotation if applicable.
\item Document error mappings and login flow.
\end{TasksBox}

\clearpage
\StoryCard{NG-7}{Security Controls and TLS}{Defense in Depth}
{Reduce attack surface and enforce transport security end to end.}
{Must}{5}{Security engineer}
{Certificates, CA chain, WAF trial if used}
{HSTS pins HTTPS and must be planned; false positives in WAF policies}

\begin{TasksBox}
\item Configure TLS (modern ciphers and ALPN).
\item Redirect HTTP to HTTPS; enable HSTS with safe max-age.
\item Implement CORS allowlist for specific origins/methods.
\item Optionally enable a WAF and tune one policy.
\end{TasksBox}

\clearpage
\StoryCard{NG-8}{HTTP/2, HTTP/3, and gRPC}{Modern Protocols}
{Improve performance and compatibility with multiplexed protocols and streaming RPC.}
{Should}{3}{Platform engineer}
{Browser or curl with HTTP/3 support; demo gRPC server}
{Middleboxes may block UDP for HTTP/3; ensure fallbacks}

\begin{TasksBox}
\item Enable HTTP/2 on TLS listeners; verify ALPN.
\item Enable HTTP/3 with QUIC and confirm UDP reachability.
\item Proxy a demo gRPC service and verify with a client.
\end{TasksBox}

\clearpage
\StoryCard{NG-9}{Media and File Streaming}{Content Delivery}
{Support large media delivery with seeking and bandwidth control.}
{Could}{3}{Media delivery engineer}
{MP4 asset; test player}
{Incorrect range settings cause buffering; respect licensing}

\begin{TasksBox}
\item Host an MP4 and validate range requests.
\item Generate a simple HLS variant and serve from NGINX.
\item Apply per-location bandwidth limits for test users.
\end{TasksBox}

\clearpage
\StoryCard{NG-10}{Cloud Deployments}{Cloud Ops}
{Provision repeatable, hardened NGINX in cloud and integrate with platform services.}
{Should}{5}{Cloud engineer}
{Cloud creds; Terraform; remote state}
{Ingress/egress must permit health checks; image hardening required}

\begin{TasksBox}
\item Create a hardened image with NGINX and baseline configs.
\item Write Terraform for networking, load balancer, and autoscaling.
\item Use user-data to render configs and run \texttt{nginx -t} before start.
\item Run a smoke test and record outputs.
\end{TasksBox}

\clearpage
\StoryCard{NG-11}{Containers and Microservices}{Service Gateway}
{Provide consistent ingress and routing in container platforms.}
{Must}{5}{SRE}
{Docker or Kubernetes cluster; manifests}
{Mis-specified probes cause restarts; config drift between images}

\begin{TasksBox}
\item Build a minimal NGINX image with mounted configs.
\item Compose a local API gateway to three services.
\item Deploy a Kubernetes ingress and verify routes and probes.
\end{TasksBox}

\clearpage
\StoryCard{NG-12}{High Availability and State Sync}{Resilience}
{Keep service available through node failures and maintenance.}
{Must}{5}{SRE}
{Two nodes; shared/synchronized state if required}
{Split brain risk if failover is not coordinated}

\begin{TasksBox}
\item Configure a two-node pair with a virtual IP or external LB.
\item Sync required state or use stateless routing.
\item Run a failover exercise and record timings.
\end{TasksBox}

\clearpage
\StoryCard{NG-13}{Monitoring and Telemetry}{Observability}
{Shorten MTTR with focused metrics and dashboards.}
{Must}{3}{Observability engineer}
{Prometheus + Grafana; access to logs}
{Excessive logging impacts performance; buffer appropriately}

\begin{TasksBox}
\item Enable \texttt{stub\_status} and test.
\item Export metrics to Prometheus and create basic alerts.
\item Add a Grafana dashboard for key KPIs.
\end{TasksBox}

\clearpage
\StoryCard{NG-14}{Debugging and Troubleshooting}{Diagnostics}
{Faster root cause and safer changes via structured logs and trace context.}
{Must}{3}{On-call engineer}
{Central log store or local capture}
{Debug logging must be scoped; avoid verbose defaults}

\begin{TasksBox}
\item Add a JSON access log with request ID and upstream timing.
\item Enable scoped debug logs for one location.
\item Document a playbook to gather evidence and rollback.
\end{TasksBox}

\clearpage
\StoryCard{NG-15}{Performance Tuning}{Throughput and Latency}
{Improve user experience and reduce cost by tuning kernel and server parameters.}
{Should}{5}{Performance engineer}
{Load generator; test plan; baseline configs}
{Overtuning can cause regressions under different workloads}

\begin{TasksBox}
\item Establish baseline metrics with a fixed test.
\item Tune NGINX keepalive, buffers, and workers.
\item Apply relevant \texttt{sysctl} parameters and retest.
\item Produce a before/after report with diffs.
\end{TasksBox}

\end{document}
