\documentclass[11pt]{article}

% ---------- Safe, common packages ----------
\usepackage[margin=1in]{geometry}
\usepackage[T1]{fontenc}
\usepackage[utf8]{inputenc}
\usepackage{lmodern}
\usepackage{microtype}
\usepackage{hyperref}
\usepackage{graphicx}
\usepackage{booktabs}
\usepackage{array}
\usepackage{tabularx}
\usepackage{enumitem}
\usepackage{amsmath}
\usepackage{setspace}

\hypersetup{
  colorlinks=true,
  linkcolor=black,
  urlcolor=blue
}

% ---------- Simple helpers ----------
\setlist[itemize]{itemsep=2pt, topsep=4pt}
\setlist[enumerate]{itemsep=4pt, topsep=6pt}
\newcommand{\Section}[1]{\vspace{0.6em}\noindent\textbf{\Large #1}\par\vspace{0.25em}}
\newcommand{\Subsection}[1]{\vspace{0.4em}\noindent\textbf{\large #1}\par\vspace{0.2em}}
\newcolumntype{L}{>{\raggedright\arraybackslash}X}

\begin{document}

\begin{center}
  {\LARGE \textbf{Zucchini Soup}}\\[4pt]
  \small by Stephanie \quad\textbullet\quad First posted: Aug 4, 2020 \quad\textbullet\quad Updated: Jun 22, 2025
\end{center}

\Section{Overview}
A healthy, hearty way to use an abundance of garden zucchini. Onion and garlic build flavor, zucchini and potatoes bring body, and a quick blend yields a naturally creamy soup without a roux. Add a splash of cream and a handful of cheese if you like it extra cozy. Freezer friendly and weeknight simple.

\Section{At a Glance}
\begin{tabularx}{\textwidth}{@{} l L @{}}
\toprule
\textbf{Method} & Stovetop; blended (immersion or countertop) \\
\textbf{Texture} & Silky without a roux; potatoes thicken naturally \\
\textbf{Make Ahead} & Refrigerates up to 3 days; freezes up to 3 months \\
\textbf{Optional} & Finish with cream/half-and-half and melted cheese \\
\bottomrule
\end{tabularx}

\Section{Ingredients}
\Subsection{Base}
\begin{itemize}
  \item Butter
  \item Onion, diced
  \item Garlic, minced
  \item Zucchini, diced
  \item Salt and black pepper
  \item Chicken broth \emph{(or vegetable broth)}
  \item Soy sauce
  \item Potatoes, peeled and diced
\end{itemize}

\Subsection{Optional Finishers}
\begin{itemize}
  \item Half-and-half or heavy cream
  \item Shredded cheese (sharp cheddar, white cheddar, Parmesan, or Gouda)
\end{itemize}

\Section{Instructions}
\begin{enumerate}
  \item \textbf{Sauté aromatics.} In a soup pot over medium heat, melt butter. Add diced onion and minced garlic; cook, stirring, until softened and fragrant.
  \item \textbf{Add zucchini \& season.} Stir in diced zucchini, season with salt and pepper, and sauté about 5 minutes.
  \item \textbf{Simmer.} Add broth, a splash of soy sauce, and diced potatoes. Bring to a boil, then reduce to a gentle simmer. Cook about 20 minutes, until the vegetables are very tender.
  \item \textbf{Blend.} Use an immersion blender to purée right in the pot until smooth (or transfer carefully to a blender in batches). For a rustic texture, leave some pieces unblended.
  \item \textbf{Enrich (optional).} Stir in a splash of half-and-half or cream. If using cheese, add a handful at a time off heat, stirring until smooth.
  \item \textbf{Season \& serve.} Taste and adjust salt and pepper. Ladle into bowls and enjoy.
\end{enumerate}

\Section{Tips}
\begin{itemize}
  \item \textbf{Peeling zucchini:} Very large zucchini can have tougher, slightly bitter skins; peel if needed. Medium zucchini generally do not require peeling.
  \item \textbf{Thickening without a roux:} Potatoes are the built-in thickener here; blending them into the broth creates body and creaminess.
  \item \textbf{No blender?} A potato masher works in a pinch for a more textured soup.
  \item \textbf{Cheese choices:} Sharp orange cheddar, white cheddar, Parmesan, or Gouda all melt in beautifully—add gradually and stir until smooth.
\end{itemize}

\Section{Storage}
\begin{itemize}
  \item \textbf{Refrigerate:} Airtight container up to 3 days.
  \item \textbf{Freeze:} Up to 3 months; thaw overnight in the fridge, then rewarm gently.
  \item \textbf{Reheat:} Stovetop over low heat or in the microwave; avoid boiling if you have added dairy.
\end{itemize}

\Section{Tools}
A 4-quart Dutch oven or soup pot; immersion blender (or countertop blender); measuring spoons; ladle; cheese grater (if adding cheese).

\vfill
\begin{center}
  \footnotesize
  \textit{Note: For a lighter soup, skip dairy entirely—the blended potatoes give ample creaminess.}
\end{center}

\end{document}

