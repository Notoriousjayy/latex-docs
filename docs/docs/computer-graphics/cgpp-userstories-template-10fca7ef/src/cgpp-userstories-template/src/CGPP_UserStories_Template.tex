\documentclass[11pt,a4paper]{article}

% --- Encoding & fonts ---
\usepackage[T1]{fontenc}
\usepackage[utf8]{inputenc}
\usepackage{lmodern}
\usepackage{microtype}

% --- Page ---
\usepackage[a4paper,margin=1in]{geometry}
\usepackage{parskip}

% --- Math & symbols ---
\usepackage{amsmath,amssymb}
\usepackage{pifont}

% --- Colors & links ---
\usepackage{xcolor}
\usepackage{hyperref}
\hypersetup{
  colorlinks=true,
  linkcolor=blue!60!black,
  urlcolor=blue!60!black,
  citecolor=blue!60!black,
  pdftitle={Hughes/Computer Graphics — User Story Template Study Plan},
  pdfauthor={},
  pdfsubject={User Story Template},
  pdfcreator={LaTeX}
}

% --- Tables & lists ---
\usepackage{array,tabularx}
\newcolumntype{L}[1]{>{\raggedright\arraybackslash}p{#1}}
\newcolumntype{Y}{>{\raggedright\arraybackslash}X}
\usepackage{enumitem}
\setlist{itemsep=2pt, topsep=4pt, leftmargin=1.2em}

% --- tcolorbox ---
\usepackage[most]{tcolorbox}
% --- Unicode + math fixes (added by ChatGPT on 2025-10-20) ---
\DeclareUnicodeCharacter{2194}{\ensuremath{\leftrightarrow}} % ↔
\DeclareUnicodeCharacter{2192}{\ensuremath{\rightarrow}}      % →
\DeclareUnicodeCharacter{2190}{\ensuremath{\leftarrow}}       % ←
\DeclareUnicodeCharacter{21D2}{\ensuremath{\Rightarrow}}      % ⇒
\DeclareUnicodeCharacter{21D4}{\ensuremath{\Leftrightarrow}}  % ⇔
% --- end fixes ---

\tcbuselibrary{skins,breakable,listings,documentation}

% --- Color palette ---
\definecolor{CardFrame}{RGB}{30,64,175}     % Indigo-600
\definecolor{CardBack}{RGB}{248,250,252}    % Slate-50
\definecolor{PillBack}{RGB}{243,244,246}    % Gray-100
\definecolor{PillBorder}{RGB}{209,213,219}  % Gray-300
\definecolor{Muted}{RGB}{55,65,81}          % Gray-700

% --- Inline "pill" tag ---
\newtcbox{\pill}{on line, sharp corners, colback=PillBack, colframe=PillBorder,
  boxrule=0.3pt, left=2pt, right=2pt, top=1pt, bottom=1pt}

% --- Checkbox macro ---
\newcommand{\citem}{\item[\ding{113}]}

% --- Given/When/Then ---
\newcommand{\Given}{\textbf{Given} }
\newcommand{\When}{\textbf{When} }
\newcommand{\Then}{\textbf{Then} }

% --- Safe guards to avoid redefinition errors ---
\makeatletter
\@ifundefined{StoryCard}{%
  \newtcolorbox{StoryCard}[1]{%
    enhanced, breakable, colback=CardBack, colframe=CardFrame,
    title={#1}, fonttitle=\large\bfseries,
    boxrule=0.6pt, arc=2pt, outer arc=2pt, left=8pt, right=8pt, top=6pt, bottom=6pt}
}{}
\@ifundefined{TasksBox}{%
  \newtcolorbox{TasksBox}{%
    enhanced, breakable, colback=white, colframe=CardFrame, title={Tasks},
    boxrule=0.6pt, arc=2pt, outer arc=2pt, left=8pt, right=8pt, top=6pt, bottom=6pt}
}{}
\makeatother

% --- Two-column key/value block ---
\newcommand{\kv}[2]{\noindent\begin{tabularx}{\linewidth}{L{0.24\linewidth}Y}
\textbf{#1} & #2
\end{tabularx}\par}

% --- DoR/DoD ribbons ---
\newcommand{\DoR}{\textit{Definition of Ready:} Persona clear; AC drafted; Dependencies known; Estimate set.}
\newcommand{\DoD}{\textit{Definition of Done:} All ACs pass; Tests green; Security/a11y checks; Docs updated; Deployed/flagged.}

\begin{document}

\begin{center}
{\LARGE \textbf{Study Plan — Hughes/Computer Graphics (CGPP 3/e)}}\\[4pt]
{\large \textbf{User Story Template \& Cards}}\\[2pt]
Version \today
\end{center}

\section*{How to Use This Template}
This document turns each chapter of \emph{Computer Graphics: Principles and Practice (3/e)} into an actionable user story. Every card has the same sections: Business Value, Priority/Estimate, Persona, Dependencies, Assumptions/Risks, the Story statement, Non-Functional tags, Acceptance Criteria (BDD), and a checklist of Tasks.

\subsection*{Writing Effective User Stories}
A good story is \textbf{concise, testable, and valuable}. Use the format:
\begin{quote}
\textbf{As a} \emph{<persona>}, \textbf{I want} \emph{<capability>}, \textbf{so that} \emph{<benefit>}.
\end{quote}
Examples:
\begin{itemize}
  \item \textbf{Weak:} ``Learn cameras.''
  \item \textbf{Better:} ``As a graphics engineer, I want to implement perspective and orthographic camera models so that screen-space depth and composition are predictable across scenes.''
\end{itemize}
Acceptance Criteria should follow BDD:
\begin{quote}
\Given preconditions, \When the hands-on objective is executed, \Then measurable outcomes are observed.
\end{quote}

\subsection*{Non-Functional Tags (suggested)}
\pill{Performance} \pill{Security} \pill{Reliability} \pill{Accessibility} \pill{Privacy} \pill{i18n}

\subsection*{Card Anatomy (Story Card Definition)}
The following sections are expected in every card. Adapt language to your organization.
\begin{itemize}[leftmargin=1.2em]
  \item \textbf{Epic / Feature} — the broader capability this chapter contributes to.
  \item \textbf{Business Value} — the reason the capability matters (why now).
  \item \textbf{Priority / Estimate} — e.g., Priority: Must/Should/Could; SP: 1--5.
  \item \textbf{Persona} — primary actor (e.g., Student, Graphics Engineer).
  \item \textbf{Dependencies} — pre-reads, libraries, or tools required.
  \item \textbf{Assumptions / Risks} — environment or scope caveats.
  \item \textbf{Story} — one sentence using the template above.
  \item \textbf{Non-Functional} — select relevant quality attributes as pills.
  \item \textbf{Acceptance Criteria (BDD)} — Happy-path Given/When/Then.
  \item \textbf{Tasks} — a short checklist that delivers the outcome.
\end{itemize}

\clearpage
\begin{StoryCard}{CH-01 --- Introduction}
\kv{Epic / Feature}{Chapter Mastery}
\kv{Business Value}{Build demonstrable skill aligned to this chapter; produce a small artifact and tests.}
\kv{Priority / Estimate}{\textbf{Priority:} Must \quad \pill{SP: 3}}
\kv{Persona}{Graphics engineer / student working through CGPP 3/e}
\kv{Dependencies}{Toolchain (C++17 + modern OpenGL or WebGPU), build scripts, test harness.}
\kv{Assumptions / Risks}{Numeric stability and platform differences may affect outputs; allow time for debugging.}
\textbf{Story} \quad As a learner of computer graphics, I want to complete \emph{Introduction} so that I can apply its concepts in a working demo with tests.
\par\vspace{4pt}\textbf{Non-Functional} \quad \pill{Performance} \pill{Reliability} \pill{Accessibility}
\par\vspace{4pt}\textbf{Acceptance Criteria (BDD)}
\begin{itemize}
\item \textbf{Scenario} Happy path
\item \Given the chapter pre-reads and starter code are available
\item \When the hands-on objectives are implemented and tests run in CI
\item \Then artifacts (demo + notes) and a summary of outcomes are committed to the repo
\end{itemize}
\DoR\quad\textbullet\quad\DoD
\end{StoryCard}
\begin{TasksBox}
\begin{itemize}[label=$\square$,leftmargin=1.2em]
\item Sketch the full rasterization pipeline and label each stage.
\item Pick your stack (OpenGL 3.3+ / WebGPU) and create a skeleton app.
\item Write a glossary of 25 core terms (frame, clip space, NDC, BRDF, etc.).
\item Set up project structure with per-chapter folders and a common math lib stub.
\item Commit a README listing success metrics for this study plan.
\end{itemize}
\end{TasksBox}
\clearpage

\begin{StoryCard}{CH-02 --- Intro to 2D Graphics (port from WPF)}
\kv{Epic / Feature}{Chapter Mastery}
\kv{Business Value}{Build demonstrable skill aligned to this chapter; produce a small artifact and tests.}
\kv{Priority / Estimate}{\textbf{Priority:} Must \quad \pill{SP: 3}}
\kv{Persona}{Graphics engineer / student working through CGPP 3/e}
\kv{Dependencies}{Toolchain (C++17 + modern OpenGL or WebGPU), build scripts, test harness.}
\kv{Assumptions / Risks}{Numeric stability and platform differences may affect outputs; allow time for debugging.}
\textbf{Story} \quad As a learner of computer graphics, I want to complete \emph{Intro to 2D Graphics (port from WPF)} so that I can apply its concepts in a working demo with tests.
\par\vspace{4pt}\textbf{Non-Functional} \quad \pill{Performance} \pill{Reliability} \pill{Accessibility}
\par\vspace{4pt}\textbf{Acceptance Criteria (BDD)}
\begin{itemize}
\item \textbf{Scenario} Happy path
\item \Given the chapter pre-reads and starter code are available
\item \When the hands-on objectives are implemented and tests run in CI
\item \Then artifacts (demo + notes) and a summary of outcomes are committed to the repo
\end{itemize}
\DoR\quad\textbullet\quad\DoD
\end{StoryCard}
\begin{TasksBox}
\begin{itemize}[label=$\square$,leftmargin=1.2em]
\item Implement a minimal retained-mode 2D scene graph (groups, transforms).
\item Render lines, paths, fills, alpha; support z-order and DPI scaling.
\item Create 3 demo scenes testing blending and overdraw cases.
\item Add hot-reload for assets/shaders; show a frame-time HUD.
\item Document differences from WPF concepts (dependency properties, layout).
\end{itemize}
\end{TasksBox}
\clearpage

\begin{StoryCard}{CH-03 --- An Ancient Renderer Made Modern}
\kv{Epic / Feature}{Chapter Mastery}
\kv{Business Value}{Build demonstrable skill aligned to this chapter; produce a small artifact and tests.}
\kv{Priority / Estimate}{\textbf{Priority:} Must \quad \pill{SP: 3}}
\kv{Persona}{Graphics engineer / student working through CGPP 3/e}
\kv{Dependencies}{Toolchain (C++17 + modern OpenGL or WebGPU), build scripts, test harness.}
\kv{Assumptions / Risks}{Numeric stability and platform differences may affect outputs; allow time for debugging.}
\textbf{Story} \quad As a learner of computer graphics, I want to complete \emph{An Ancient Renderer Made Modern} so that I can apply its concepts in a working demo with tests.
\par\vspace{4pt}\textbf{Non-Functional} \quad \pill{Performance} \pill{Reliability} \pill{Accessibility}
\par\vspace{4pt}\textbf{Acceptance Criteria (BDD)}
\begin{itemize}
\item \textbf{Scenario} Happy path
\item \Given the chapter pre-reads and starter code are available
\item \When the hands-on objectives are implemented and tests run in CI
\item \Then artifacts (demo + notes) and a summary of outcomes are committed to the repo
\end{itemize}
\DoR\quad\textbullet\quad\DoD
\end{StoryCard}
\begin{TasksBox}
\begin{itemize}[label=$\square$,leftmargin=1.2em]
\item Implement polygon scanline fill with winding / even-odd modes.
\item Show painter’s algorithm vs depth ordering on stacked quads.
\item Add front/back-face culling toggles and visualize overdraw heatmap.
\item Export reference images for 5 tricky concave cases.
\item Write property-based tests for edge walking and fill parity.
\end{itemize}
\end{TasksBox}
\clearpage

\begin{StoryCard}{CH-04 --- A 2D Graphics Test Bed}
\kv{Epic / Feature}{Chapter Mastery}
\kv{Business Value}{Build demonstrable skill aligned to this chapter; produce a small artifact and tests.}
\kv{Priority / Estimate}{\textbf{Priority:} Must \quad \pill{SP: 3}}
\kv{Persona}{Graphics engineer / student working through CGPP 3/e}
\kv{Dependencies}{Toolchain (C++17 + modern OpenGL or WebGPU), build scripts, test harness.}
\kv{Assumptions / Risks}{Numeric stability and platform differences may affect outputs; allow time for debugging.}
\textbf{Story} \quad As a learner of computer graphics, I want to complete \emph{A 2D Graphics Test Bed} so that I can apply its concepts in a working demo with tests.
\par\vspace{4pt}\textbf{Non-Functional} \quad \pill{Performance} \pill{Reliability} \pill{Accessibility}
\par\vspace{4pt}\textbf{Acceptance Criteria (BDD)}
\begin{itemize}
\item \textbf{Scenario} Happy path
\item \Given the chapter pre-reads and starter code are available
\item \When the hands-on objectives are implemented and tests run in CI
\item \Then artifacts (demo + notes) and a summary of outcomes are committed to the repo
\end{itemize}
\DoR\quad\textbullet\quad\DoD
\end{StoryCard}
\begin{TasksBox}
\begin{itemize}[label=$\square$,leftmargin=1.2em]
\item Create a test harness with scene switcher and screenshot capture.
\item Add perf counters (FPS, frame time p95) and input capture/logging.
\item Build regression tests that compare images with SSIM/PSNR thresholds.
\item Package three ‘golden’ scenes that stress blending and alpha.
\item Wire a CI job that runs image-diff tests on PRs.
\end{itemize}
\end{TasksBox}
\clearpage

\begin{StoryCard}{CH-05 --- Intro to Human Visual Perception}
\kv{Epic / Feature}{Chapter Mastery}
\kv{Business Value}{Build demonstrable skill aligned to this chapter; produce a small artifact and tests.}
\kv{Priority / Estimate}{\textbf{Priority:} Must \quad \pill{SP: 3}}
\kv{Persona}{Graphics engineer / student working through CGPP 3/e}
\kv{Dependencies}{Toolchain (C++17 + modern OpenGL or WebGPU), build scripts, test harness.}
\kv{Assumptions / Risks}{Numeric stability and platform differences may affect outputs; allow time for debugging.}
\textbf{Story} \quad As a learner of computer graphics, I want to complete \emph{Intro to Human Visual Perception} so that I can apply its concepts in a working demo with tests.
\par\vspace{4pt}\textbf{Non-Functional} \quad \pill{Performance} \pill{Reliability} \pill{Accessibility}
\par\vspace{4pt}\textbf{Acceptance Criteria (BDD)}
\begin{itemize}
\item \textbf{Scenario} Happy path
\item \Given the chapter pre-reads and starter code are available
\item \When the hands-on objectives are implemented and tests run in CI
\item \Then artifacts (demo + notes) and a summary of outcomes are committed to the repo
\end{itemize}
\DoR\quad\textbullet\quad\DoD
\end{StoryCard}
\begin{TasksBox}
\begin{itemize}[label=$\square$,leftmargin=1.2em]
\item Implement interactive demos for Mach bands and contrast sensitivity.
\item Add a tone-mapping panel (exposure, white balance, filmic).
\item Measure perceived banding before/after dithering at 8-bit output.
\item Compare perceived sharpness across MIP levels and anisotropic filtering.
\item Write a short memo mapping perception to sampling choices.
\end{itemize}
\end{TasksBox}
\clearpage

\begin{StoryCard}{CH-06 --- Fixed-Function 3D \& Hierarchical Modeling}
\kv{Epic / Feature}{Chapter Mastery}
\kv{Business Value}{Build demonstrable skill aligned to this chapter; produce a small artifact and tests.}
\kv{Priority / Estimate}{\textbf{Priority:} Must \quad \pill{SP: 3}}
\kv{Persona}{Graphics engineer / student working through CGPP 3/e}
\kv{Dependencies}{Toolchain (C++17 + modern OpenGL or WebGPU), build scripts, test harness.}
\kv{Assumptions / Risks}{Numeric stability and platform differences may affect outputs; allow time for debugging.}
\textbf{Story} \quad As a learner of computer graphics, I want to complete \emph{Fixed-Function 3D \& Hierarchical Modeling} so that I can apply its concepts in a working demo with tests.
\par\vspace{4pt}\textbf{Non-Functional} \quad \pill{Performance} \pill{Reliability} \pill{Accessibility}
\par\vspace{4pt}\textbf{Acceptance Criteria (BDD)}
\begin{itemize}
\item \textbf{Scenario} Happy path
\item \Given the chapter pre-reads and starter code are available
\item \When the hands-on objectives are implemented and tests run in CI
\item \Then artifacts (demo + notes) and a summary of outcomes are committed to the repo
\end{itemize}
\DoR\quad\textbullet\quad\DoD
\end{StoryCard}
\begin{TasksBox}
\begin{itemize}[label=$\square$,leftmargin=1.2em]
\item Build a node hierarchy with parent/child transforms and local/global frames.
\item Create a solar system demo (orbit + spin) with per-node pivots.
\item Render with a basic Phong-like approximation as a placeholder.
\item Toggle wireframe and bounding boxes per node.
\item Document how modern shaders replace fixed-function equivalents.
\end{itemize}
\end{TasksBox}
\clearpage

\begin{StoryCard}{CH-07 --- Essential Math \& Geometry of 2- and 3-Space}
\kv{Epic / Feature}{Chapter Mastery}
\kv{Business Value}{Build demonstrable skill aligned to this chapter; produce a small artifact and tests.}
\kv{Priority / Estimate}{\textbf{Priority:} Must \quad \pill{SP: 3}}
\kv{Persona}{Graphics engineer / student working through CGPP 3/e}
\kv{Dependencies}{Toolchain (C++17 + modern OpenGL or WebGPU), build scripts, test harness.}
\kv{Assumptions / Risks}{Numeric stability and platform differences may affect outputs; allow time for debugging.}
\textbf{Story} \quad As a learner of computer graphics, I want to complete \emph{Essential Math \& Geometry of 2- and 3-Space} so that I can apply its concepts in a working demo with tests.
\par\vspace{4pt}\textbf{Non-Functional} \quad \pill{Performance} \pill{Reliability} \pill{Accessibility}
\par\vspace{4pt}\textbf{Acceptance Criteria (BDD)}
\begin{itemize}
\item \textbf{Scenario} Happy path
\item \Given the chapter pre-reads and starter code are available
\item \When the hands-on objectives are implemented and tests run in CI
\item \Then artifacts (demo + notes) and a summary of outcomes are committed to the repo
\end{itemize}
\DoR\quad\textbullet\quad\DoD
\end{StoryCard}
\begin{TasksBox}
\begin{itemize}[label=$\square$,leftmargin=1.2em]
\item Implement vec2/3/4 and mat3/4 with affine helpers and unit tests.
\item Write robust segment/triangle intersection predicates.
\item Implement barycentric coordinates and point-in-triangle test.
\item Add coordinate-space converters (object/world/view/clip/NDC).
\item Benchmark numeric stability with randomized stress tests.
\end{itemize}
\end{TasksBox}
\clearpage

\begin{StoryCard}{CH-08 --- Describing Shape in 2D \& 3D}
\kv{Epic / Feature}{Chapter Mastery}
\kv{Business Value}{Build demonstrable skill aligned to this chapter; produce a small artifact and tests.}
\kv{Priority / Estimate}{\textbf{Priority:} Must \quad \pill{SP: 3}}
\kv{Persona}{Graphics engineer / student working through CGPP 3/e}
\kv{Dependencies}{Toolchain (C++17 + modern OpenGL or WebGPU), build scripts, test harness.}
\kv{Assumptions / Risks}{Numeric stability and platform differences may affect outputs; allow time for debugging.}
\textbf{Story} \quad As a learner of computer graphics, I want to complete \emph{Describing Shape in 2D \& 3D} so that I can apply its concepts in a working demo with tests.
\par\vspace{4pt}\textbf{Non-Functional} \quad \pill{Performance} \pill{Reliability} \pill{Accessibility}
\par\vspace{4pt}\textbf{Acceptance Criteria (BDD)}
\begin{itemize}
\item \textbf{Scenario} Happy path
\item \Given the chapter pre-reads and starter code are available
\item \When the hands-on objectives are implemented and tests run in CI
\item \Then artifacts (demo + notes) and a summary of outcomes are committed to the repo
\end{itemize}
\DoR\quad\textbullet\quad\DoD
\end{StoryCard}
\begin{TasksBox}
\begin{itemize}[label=$\square$,leftmargin=1.2em]
\item Load OBJ/PLY; compute per-face and per-vertex normals.
\item Add manifold checks and simple weld/merge operations.
\item Implement wireframe/solid toggles and normal visualization.
\item Export a repaired mesh after hole-filling and deduping vertices.
\item Write a short note on manifoldness and rendering implications.
\end{itemize}
\end{TasksBox}
\clearpage

\begin{StoryCard}{CH-09 --- Functions on Meshes}
\kv{Epic / Feature}{Chapter Mastery}
\kv{Business Value}{Build demonstrable skill aligned to this chapter; produce a small artifact and tests.}
\kv{Priority / Estimate}{\textbf{Priority:} Must \quad \pill{SP: 3}}
\kv{Persona}{Graphics engineer / student working through CGPP 3/e}
\kv{Dependencies}{Toolchain (C++17 + modern OpenGL or WebGPU), build scripts, test harness.}
\kv{Assumptions / Risks}{Numeric stability and platform differences may affect outputs; allow time for debugging.}
\textbf{Story} \quad As a learner of computer graphics, I want to complete \emph{Functions on Meshes} so that I can apply its concepts in a working demo with tests.
\par\vspace{4pt}\textbf{Non-Functional} \quad \pill{Performance} \pill{Reliability} \pill{Accessibility}
\par\vspace{4pt}\textbf{Acceptance Criteria (BDD)}
\begin{itemize}
\item \textbf{Scenario} Happy path
\item \Given the chapter pre-reads and starter code are available
\item \When the hands-on objectives are implemented and tests run in CI
\item \Then artifacts (demo + notes) and a summary of outcomes are committed to the repo
\end{itemize}
\DoR\quad\textbullet\quad\DoD
\end{StoryCard}
\begin{TasksBox}
\begin{itemize}[label=$\square$,leftmargin=1.2em]
\item Implement barycentric interpolation for scalar fields on triangles.
\item Visualize a temperature field with color ramps (linear vs diverging).
\item Add UV lookup and bilinear filtering on a textured mesh.
\item Compare per-vertex vs per-pixel interpolation artifacts.
\item Create tests that compare analytic vs sampled values at random points.
\end{itemize}
\end{TasksBox}
\clearpage

\begin{StoryCard}{CH-10 --- 2D Transformations}
\kv{Epic / Feature}{Chapter Mastery}
\kv{Business Value}{Build demonstrable skill aligned to this chapter; produce a small artifact and tests.}
\kv{Priority / Estimate}{\textbf{Priority:} Must \quad \pill{SP: 3}}
\kv{Persona}{Graphics engineer / student working through CGPP 3/e}
\kv{Dependencies}{Toolchain (C++17 + modern OpenGL or WebGPU), build scripts, test harness.}
\kv{Assumptions / Risks}{Numeric stability and platform differences may affect outputs; allow time for debugging.}
\textbf{Story} \quad As a learner of computer graphics, I want to complete \emph{2D Transformations} so that I can apply its concepts in a working demo with tests.
\par\vspace{4pt}\textbf{Non-Functional} \quad \pill{Performance} \pill{Reliability} \pill{Accessibility}
\par\vspace{4pt}\textbf{Acceptance Criteria (BDD)}
\begin{itemize}
\item \textbf{Scenario} Happy path
\item \Given the chapter pre-reads and starter code are available
\item \When the hands-on objectives are implemented and tests run in CI
\item \Then artifacts (demo + notes) and a summary of outcomes are committed to the repo
\end{itemize}
\DoR\quad\textbullet\quad\DoD
\end{StoryCard}
\begin{TasksBox}
\begin{itemize}[label=$\square$,leftmargin=1.2em]
\item Implement 3x3 homogeneous matrices for 2D TRS.
\item Build interactive 2D camera (pan/zoom) and transform gizmos.
\item Demonstrate window-to-viewport mappings with aspect correction.
\item Create a scripted animation that composes multiple transforms.
\item Write unit tests for composition associativity and inversion.
\end{itemize}
\end{TasksBox}
\clearpage

\begin{StoryCard}{CH-11 --- 3D Transformations}
\kv{Epic / Feature}{Chapter Mastery}
\kv{Business Value}{Build demonstrable skill aligned to this chapter; produce a small artifact and tests.}
\kv{Priority / Estimate}{\textbf{Priority:} Must \quad \pill{SP: 3}}
\kv{Persona}{Graphics engineer / student working through CGPP 3/e}
\kv{Dependencies}{Toolchain (C++17 + modern OpenGL or WebGPU), build scripts, test harness.}
\kv{Assumptions / Risks}{Numeric stability and platform differences may affect outputs; allow time for debugging.}
\textbf{Story} \quad As a learner of computer graphics, I want to complete \emph{3D Transformations} so that I can apply its concepts in a working demo with tests.
\par\vspace{4pt}\textbf{Non-Functional} \quad \pill{Performance} \pill{Reliability} \pill{Accessibility}
\par\vspace{4pt}\textbf{Acceptance Criteria (BDD)}
\begin{itemize}
\item \textbf{Scenario} Happy path
\item \Given the chapter pre-reads and starter code are available
\item \When the hands-on objectives are implemented and tests run in CI
\item \Then artifacts (demo + notes) and a summary of outcomes are committed to the repo
\end{itemize}
\DoR\quad\textbullet\quad\DoD
\end{StoryCard}
\begin{TasksBox}
\begin{itemize}[label=$\square$,leftmargin=1.2em]
\item Implement axis-angle, Euler, quaternion conversions and SLERP.
\item Build an arcball camera with constraints and damping.
\item Visualize gimbal lock with labeled axes and keyframes.
\item Add right-/left-handed toggles and verify against test scenes.
\item Document numerical pitfalls of quaternion normalization.
\end{itemize}
\end{TasksBox}
\clearpage

\begin{StoryCard}{CH-12 --- 2D \& 3D Transformation Library}
\kv{Epic / Feature}{Chapter Mastery}
\kv{Business Value}{Build demonstrable skill aligned to this chapter; produce a small artifact and tests.}
\kv{Priority / Estimate}{\textbf{Priority:} Must \quad \pill{SP: 3}}
\kv{Persona}{Graphics engineer / student working through CGPP 3/e}
\kv{Dependencies}{Toolchain (C++17 + modern OpenGL or WebGPU), build scripts, test harness.}
\kv{Assumptions / Risks}{Numeric stability and platform differences may affect outputs; allow time for debugging.}
\textbf{Story} \quad As a learner of computer graphics, I want to complete \emph{2D \& 3D Transformation Library} so that I can apply its concepts in a working demo with tests.
\par\vspace{4pt}\textbf{Non-Functional} \quad \pill{Performance} \pill{Reliability} \pill{Accessibility}
\par\vspace{4pt}\textbf{Acceptance Criteria (BDD)}
\begin{itemize}
\item \textbf{Scenario} Happy path
\item \Given the chapter pre-reads and starter code are available
\item \When the hands-on objectives are implemented and tests run in CI
\item \Then artifacts (demo + notes) and a summary of outcomes are committed to the repo
\end{itemize}
\DoR\quad\textbullet\quad\DoD
\end{StoryCard}
\begin{TasksBox}
\begin{itemize}[label=$\square$,leftmargin=1.2em]
\item Ship look-at, orthographic, and perspective constructors.
\item Add covector/normal transformation (inverse-transpose) helper.
\item Cache and invalidate matrices inside a Transform component.
\item Benchmark transform throughput on CPU and GPU instancing.
\item Write API docs with examples and edge cases.
\end{itemize}
\end{TasksBox}
\clearpage

\begin{StoryCard}{CH-13 --- Camera Specifications \& Transformations}
\kv{Epic / Feature}{Chapter Mastery}
\kv{Business Value}{Build demonstrable skill aligned to this chapter; produce a small artifact and tests.}
\kv{Priority / Estimate}{\textbf{Priority:} Must \quad \pill{SP: 3}}
\kv{Persona}{Graphics engineer / student working through CGPP 3/e}
\kv{Dependencies}{Toolchain (C++17 + modern OpenGL or WebGPU), build scripts, test harness.}
\kv{Assumptions / Risks}{Numeric stability and platform differences may affect outputs; allow time for debugging.}
\textbf{Story} \quad As a learner of computer graphics, I want to complete \emph{Camera Specifications \& Transformations} so that I can apply its concepts in a working demo with tests.
\par\vspace{4pt}\textbf{Non-Functional} \quad \pill{Performance} \pill{Reliability} \pill{Accessibility}
\par\vspace{4pt}\textbf{Acceptance Criteria (BDD)}
\begin{itemize}
\item \textbf{Scenario} Happy path
\item \Given the chapter pre-reads and starter code are available
\item \When the hands-on objectives are implemented and tests run in CI
\item \Then artifacts (demo + notes) and a summary of outcomes are committed to the repo
\end{itemize}
\DoR\quad\textbullet\quad\DoD
\end{StoryCard}
\begin{TasksBox}
\begin{itemize}[label=$\square$,leftmargin=1.2em]
\item Implement perspective/orthographic cameras mapped to clip/NDC.
\item Add dolly/orbit/FPS rigs and near/far UI with reversed-Z option.
\item Plot z-buffer precision across depth for different n/f settings.
\item Capture mismatches between math and raster outputs and fix them.
\item Create camera presets and a compare view.
\end{itemize}
\end{TasksBox}
\clearpage

\begin{StoryCard}{CH-14 --- Standard Approximations \& Representations}
\kv{Epic / Feature}{Chapter Mastery}
\kv{Business Value}{Build demonstrable skill aligned to this chapter; produce a small artifact and tests.}
\kv{Priority / Estimate}{\textbf{Priority:} Must \quad \pill{SP: 3}}
\kv{Persona}{Graphics engineer / student working through CGPP 3/e}
\kv{Dependencies}{Toolchain (C++17 + modern OpenGL or WebGPU), build scripts, test harness.}
\kv{Assumptions / Risks}{Numeric stability and platform differences may affect outputs; allow time for debugging.}
\textbf{Story} \quad As a learner of computer graphics, I want to complete \emph{Standard Approximations \& Representations} so that I can apply its concepts in a working demo with tests.
\par\vspace{4pt}\textbf{Non-Functional} \quad \pill{Performance} \pill{Reliability} \pill{Accessibility}
\par\vspace{4pt}\textbf{Acceptance Criteria (BDD)}
\begin{itemize}
\item \textbf{Scenario} Happy path
\item \Given the chapter pre-reads and starter code are available
\item \When the hands-on objectives are implemented and tests run in CI
\item \Then artifacts (demo + notes) and a summary of outcomes are committed to the repo
\end{itemize}
\DoR\quad\textbullet\quad\DoD
\end{StoryCard}
\begin{TasksBox}
\begin{itemize}[label=$\square$,leftmargin=1.2em]
\item Model a scene graph with materials and lights as first-class nodes.
\item Demonstrate image-based vs geometric vs volumetric approximations.
\item Build a blending showcase (premultiplied vs straight alpha).
\item Profile overdraw and propose mitigation steps per scene.
\item Create a decision table for representation choices by use-case.
\end{itemize}
\end{TasksBox}
\clearpage

\begin{StoryCard}{CH-15 --- Ray Casting \& Rasterization}
\kv{Epic / Feature}{Chapter Mastery}
\kv{Business Value}{Build demonstrable skill aligned to this chapter; produce a small artifact and tests.}
\kv{Priority / Estimate}{\textbf{Priority:} Must \quad \pill{SP: 3}}
\kv{Persona}{Graphics engineer / student working through CGPP 3/e}
\kv{Dependencies}{Toolchain (C++17 + modern OpenGL or WebGPU), build scripts, test harness.}
\kv{Assumptions / Risks}{Numeric stability and platform differences may affect outputs; allow time for debugging.}
\textbf{Story} \quad As a learner of computer graphics, I want to complete \emph{Ray Casting \& Rasterization} so that I can apply its concepts in a working demo with tests.
\par\vspace{4pt}\textbf{Non-Functional} \quad \pill{Performance} \pill{Reliability} \pill{Accessibility}
\par\vspace{4pt}\textbf{Acceptance Criteria (BDD)}
\begin{itemize}
\item \textbf{Scenario} Happy path
\item \Given the chapter pre-reads and starter code are available
\item \When the hands-on objectives are implemented and tests run in CI
\item \Then artifacts (demo + notes) and a summary of outcomes are committed to the repo
\end{itemize}
\DoR\quad\textbullet\quad\DoD
\end{StoryCard}
\begin{TasksBox}
\begin{itemize}[label=$\square$,leftmargin=1.2em]
\item Implement a CPU ray caster for spheres/triangles.
\item Implement a minimal triangle rasterizer with barycentric interpolation.
\item Render the same scene both ways and compare correctness/perf.
\item Visualize aliasing differences and sampling strategies.
\item Write a post-mortem on trade-offs and when to choose each.
\end{itemize}
\end{TasksBox}
\clearpage

\begin{StoryCard}{CH-16 --- Survey of Real-Time 3D Platforms}
\kv{Epic / Feature}{Chapter Mastery}
\kv{Business Value}{Build demonstrable skill aligned to this chapter; produce a small artifact and tests.}
\kv{Priority / Estimate}{\textbf{Priority:} Must \quad \pill{SP: 3}}
\kv{Persona}{Graphics engineer / student working through CGPP 3/e}
\kv{Dependencies}{Toolchain (C++17 + modern OpenGL or WebGPU), build scripts, test harness.}
\kv{Assumptions / Risks}{Numeric stability and platform differences may affect outputs; allow time for debugging.}
\textbf{Story} \quad As a learner of computer graphics, I want to complete \emph{Survey of Real-Time 3D Platforms} so that I can apply its concepts in a working demo with tests.
\par\vspace{4pt}\textbf{Non-Functional} \quad \pill{Performance} \pill{Reliability} \pill{Accessibility}
\par\vspace{4pt}\textbf{Acceptance Criteria (BDD)}
\begin{itemize}
\item \textbf{Scenario} Happy path
\item \Given the chapter pre-reads and starter code are available
\item \When the hands-on objectives are implemented and tests run in CI
\item \Then artifacts (demo + notes) and a summary of outcomes are committed to the repo
\end{itemize}
\DoR\quad\textbullet\quad\DoD
\end{StoryCard}
\begin{TasksBox}
\begin{itemize}[label=$\square$,leftmargin=1.2em]
\item Create a matrix of features across OpenGL, Vulkan, D3D, WebGPU.
\item Identify the minimal cross-platform subset for this study repo.
\item Build a portability checklist for shaders, textures, and buffers.
\item Run a smoke test on at least two platforms (desktop + web).
\item Document driver quirks encountered and workarounds.
\end{itemize}
\end{TasksBox}
\clearpage

\begin{StoryCard}{CH-17 --- Image Representation \& Manipulation}
\kv{Epic / Feature}{Chapter Mastery}
\kv{Business Value}{Build demonstrable skill aligned to this chapter; produce a small artifact and tests.}
\kv{Priority / Estimate}{\textbf{Priority:} Must \quad \pill{SP: 3}}
\kv{Persona}{Graphics engineer / student working through CGPP 3/e}
\kv{Dependencies}{Toolchain (C++17 + modern OpenGL or WebGPU), build scripts, test harness.}
\kv{Assumptions / Risks}{Numeric stability and platform differences may affect outputs; allow time for debugging.}
\textbf{Story} \quad As a learner of computer graphics, I want to complete \emph{Image Representation \& Manipulation} so that I can apply its concepts in a working demo with tests.
\par\vspace{4pt}\textbf{Non-Functional} \quad \pill{Performance} \pill{Reliability} \pill{Accessibility}
\par\vspace{4pt}\textbf{Acceptance Criteria (BDD)}
\begin{itemize}
\item \textbf{Scenario} Happy path
\item \Given the chapter pre-reads and starter code are available
\item \When the hands-on objectives are implemented and tests run in CI
\item \Then artifacts (demo + notes) and a summary of outcomes are committed to the repo
\end{itemize}
\DoR\quad\textbullet\quad\DoD
\end{StoryCard}
\begin{TasksBox}
\begin{itemize}[label=$\square$,leftmargin=1.2em]
\item Implement image I/O, mipmap generation, and premultiplied alpha.
\item Build a compositor supporting over/atop/in operators.
\item Create a CLI tool for format conversion and alpha premultiplication.
\item Add sRGB↔linear conversions and verify with unit tests.
\item Create regression images for edge cases (alpha fringes).
\end{itemize}
\end{TasksBox}
\clearpage

\begin{StoryCard}{CH-18 --- Images \& Signal Processing}
\kv{Epic / Feature}{Chapter Mastery}
\kv{Business Value}{Build demonstrable skill aligned to this chapter; produce a small artifact and tests.}
\kv{Priority / Estimate}{\textbf{Priority:} Must \quad \pill{SP: 3}}
\kv{Persona}{Graphics engineer / student working through CGPP 3/e}
\kv{Dependencies}{Toolchain (C++17 + modern OpenGL or WebGPU), build scripts, test harness.}
\kv{Assumptions / Risks}{Numeric stability and platform differences may affect outputs; allow time for debugging.}
\textbf{Story} \quad As a learner of computer graphics, I want to complete \emph{Images \& Signal Processing} so that I can apply its concepts in a working demo with tests.
\par\vspace{4pt}\textbf{Non-Functional} \quad \pill{Performance} \pill{Reliability} \pill{Accessibility}
\par\vspace{4pt}\textbf{Acceptance Criteria (BDD)}
\begin{itemize}
\item \textbf{Scenario} Happy path
\item \Given the chapter pre-reads and starter code are available
\item \When the hands-on objectives are implemented and tests run in CI
\item \Then artifacts (demo + notes) and a summary of outcomes are committed to the repo
\end{itemize}
\DoR\quad\textbullet\quad\DoD
\end{StoryCard}
\begin{TasksBox}
\begin{itemize}[label=$\square$,leftmargin=1.2em]
\item Implement convolution filters and separable kernels.
\item Demonstrate Nyquist sampling and aliasing with interactive sliders.
\item Compare reconstruction filters (box, triangle, Lanczos).
\item Add a frequency-domain view using FFT on test images.
\item Write guidelines for choosing filters in the pipeline.
\end{itemize}
\end{TasksBox}
\clearpage

\begin{StoryCard}{CH-19 --- Enlarging \& Shrinking Images}
\kv{Epic / Feature}{Chapter Mastery}
\kv{Business Value}{Build demonstrable skill aligned to this chapter; produce a small artifact and tests.}
\kv{Priority / Estimate}{\textbf{Priority:} Must \quad \pill{SP: 3}}
\kv{Persona}{Graphics engineer / student working through CGPP 3/e}
\kv{Dependencies}{Toolchain (C++17 + modern OpenGL or WebGPU), build scripts, test harness.}
\kv{Assumptions / Risks}{Numeric stability and platform differences may affect outputs; allow time for debugging.}
\textbf{Story} \quad As a learner of computer graphics, I want to complete \emph{Enlarging \& Shrinking Images} so that I can apply its concepts in a working demo with tests.
\par\vspace{4pt}\textbf{Non-Functional} \quad \pill{Performance} \pill{Reliability} \pill{Accessibility}
\par\vspace{4pt}\textbf{Acceptance Criteria (BDD)}
\begin{itemize}
\item \textbf{Scenario} Happy path
\item \Given the chapter pre-reads and starter code are available
\item \When the hands-on objectives are implemented and tests run in CI
\item \Then artifacts (demo + notes) and a summary of outcomes are committed to the repo
\end{itemize}
\DoR\quad\textbullet\quad\DoD
\end{StoryCard}
\begin{TasksBox}
\begin{itemize}[label=$\square$,leftmargin=1.2em]
\item Implement nearest, bilinear, bicubic, and Lanczos resampling.
\item Measure PSNR/SSIM on resized images vs high-res ground truth.
\item Show ringing vs smoothing trade-offs and mitigation.
\item Add a scaler demo UI with real-time comparisons.
\item Document quality vs performance recommendations.
\end{itemize}
\end{TasksBox}
\clearpage

\begin{StoryCard}{CH-20 --- Textures \& Texture Mapping}
\kv{Epic / Feature}{Chapter Mastery}
\kv{Business Value}{Build demonstrable skill aligned to this chapter; produce a small artifact and tests.}
\kv{Priority / Estimate}{\textbf{Priority:} Must \quad \pill{SP: 3}}
\kv{Persona}{Graphics engineer / student working through CGPP 3/e}
\kv{Dependencies}{Toolchain (C++17 + modern OpenGL or WebGPU), build scripts, test harness.}
\kv{Assumptions / Risks}{Numeric stability and platform differences may affect outputs; allow time for debugging.}
\textbf{Story} \quad As a learner of computer graphics, I want to complete \emph{Textures \& Texture Mapping} so that I can apply its concepts in a working demo with tests.
\par\vspace{4pt}\textbf{Non-Functional} \quad \pill{Performance} \pill{Reliability} \pill{Accessibility}
\par\vspace{4pt}\textbf{Acceptance Criteria (BDD)}
\begin{itemize}
\item \textbf{Scenario} Happy path
\item \Given the chapter pre-reads and starter code are available
\item \When the hands-on objectives are implemented and tests run in CI
\item \Then artifacts (demo + notes) and a summary of outcomes are committed to the repo
\end{itemize}
\DoR\quad\textbullet\quad\DoD
\end{StoryCard}
\begin{TasksBox}
\begin{itemize}[label=$\square$,leftmargin=1.2em]
\item Compute tangents/bitangents and enable normal mapping.
\item Demonstrate mipmapping, anisotropic filtering, and LOD bias.
\item Compare atlas vs array textures and PBR texture sets.
\item Implement texture address modes and gamma-correct sampling.
\item Create a material preview scene with UI controls.
\end{itemize}
\end{TasksBox}
\clearpage

\begin{StoryCard}{CH-21 --- Interaction Techniques}
\kv{Epic / Feature}{Chapter Mastery}
\kv{Business Value}{Build demonstrable skill aligned to this chapter; produce a small artifact and tests.}
\kv{Priority / Estimate}{\textbf{Priority:} Must \quad \pill{SP: 3}}
\kv{Persona}{Graphics engineer / student working through CGPP 3/e}
\kv{Dependencies}{Toolchain (C++17 + modern OpenGL or WebGPU), build scripts, test harness.}
\kv{Assumptions / Risks}{Numeric stability and platform differences may affect outputs; allow time for debugging.}
\textbf{Story} \quad As a learner of computer graphics, I want to complete \emph{Interaction Techniques} so that I can apply its concepts in a working demo with tests.
\par\vspace{4pt}\textbf{Non-Functional} \quad \pill{Performance} \pill{Reliability} \pill{Accessibility}
\par\vspace{4pt}\textbf{Acceptance Criteria (BDD)}
\begin{itemize}
\item \textbf{Scenario} Happy path
\item \Given the chapter pre-reads and starter code are available
\item \When the hands-on objectives are implemented and tests run in CI
\item \Then artifacts (demo + notes) and a summary of outcomes are committed to the repo
\end{itemize}
\DoR\quad\textbullet\quad\DoD
\end{StoryCard}
\begin{TasksBox}
\begin{itemize}[label=$\square$,leftmargin=1.2em]
\item Implement picking and 3D gizmos (translate/rotate/scale).
\item Add Unicam/multitouch gestures with constraints.
\item Build a selection and manipulation system for scene nodes.
\item Record usability notes from 3 tasks (position a light, orbit camera...).
\item Log input events and edge cases; propose ergonomic defaults.
\end{itemize}
\end{TasksBox}
\clearpage

\begin{StoryCard}{CH-22 --- Splines \& Subdivision Curves}
\kv{Epic / Feature}{Chapter Mastery}
\kv{Business Value}{Build demonstrable skill aligned to this chapter; produce a small artifact and tests.}
\kv{Priority / Estimate}{\textbf{Priority:} Must \quad \pill{SP: 3}}
\kv{Persona}{Graphics engineer / student working through CGPP 3/e}
\kv{Dependencies}{Toolchain (C++17 + modern OpenGL or WebGPU), build scripts, test harness.}
\kv{Assumptions / Risks}{Numeric stability and platform differences may affect outputs; allow time for debugging.}
\textbf{Story} \quad As a learner of computer graphics, I want to complete \emph{Splines \& Subdivision Curves} so that I can apply its concepts in a working demo with tests.
\par\vspace{4pt}\textbf{Non-Functional} \quad \pill{Performance} \pill{Reliability} \pill{Accessibility}
\par\vspace{4pt}\textbf{Acceptance Criteria (BDD)}
\begin{itemize}
\item \textbf{Scenario} Happy path
\item \Given the chapter pre-reads and starter code are available
\item \When the hands-on objectives are implemented and tests run in CI
\item \Then artifacts (demo + notes) and a summary of outcomes are committed to the repo
\end{itemize}
\DoR\quad\textbullet\quad\DoD
\end{StoryCard}
\begin{TasksBox}
\begin{itemize}[label=$\square$,leftmargin=1.2em]
\item Implement Hermite, Catmull–Rom, and cubic B-splines.
\item Reparameterize by arc length and compare uniform vs chordal.
\item Build a path editor and animation along a curve.
\item Show C\textasciicircum{}k continuity impacts on motion smoothness.
\item Add unit tests for endpoint conditions and continuity.
\end{itemize}
\end{TasksBox}
\clearpage

\begin{StoryCard}{CH-23 --- Splines \& Subdivision Surfaces}
\kv{Epic / Feature}{Chapter Mastery}
\kv{Business Value}{Build demonstrable skill aligned to this chapter; produce a small artifact and tests.}
\kv{Priority / Estimate}{\textbf{Priority:} Must \quad \pill{SP: 3}}
\kv{Persona}{Graphics engineer / student working through CGPP 3/e}
\kv{Dependencies}{Toolchain (C++17 + modern OpenGL or WebGPU), build scripts, test harness.}
\kv{Assumptions / Risks}{Numeric stability and platform differences may affect outputs; allow time for debugging.}
\textbf{Story} \quad As a learner of computer graphics, I want to complete \emph{Splines \& Subdivision Surfaces} so that I can apply its concepts in a working demo with tests.
\par\vspace{4pt}\textbf{Non-Functional} \quad \pill{Performance} \pill{Reliability} \pill{Accessibility}
\par\vspace{4pt}\textbf{Acceptance Criteria (BDD)}
\begin{itemize}
\item \textbf{Scenario} Happy path
\item \Given the chapter pre-reads and starter code are available
\item \When the hands-on objectives are implemented and tests run in CI
\item \Then artifacts (demo + notes) and a summary of outcomes are committed to the repo
\end{itemize}
\DoR\quad\textbullet\quad\DoD
\end{StoryCard}
\begin{TasksBox}
\begin{itemize}[label=$\square$,leftmargin=1.2em]
\item Implement Bézier patches and Catmull–Clark basics.
\item Compute limit positions and normals; visualize control mesh vs limit.
\item Evaluate patches on GPU and compare tessellation levels.
\item Export tessellated meshes for downstream rendering.
\item Write a note on performance/quality trade-offs.
\end{itemize}
\end{TasksBox}
\clearpage

\begin{StoryCard}{CH-24 --- Implicit Representations of Shape}
\kv{Epic / Feature}{Chapter Mastery}
\kv{Business Value}{Build demonstrable skill aligned to this chapter; produce a small artifact and tests.}
\kv{Priority / Estimate}{\textbf{Priority:} Must \quad \pill{SP: 3}}
\kv{Persona}{Graphics engineer / student working through CGPP 3/e}
\kv{Dependencies}{Toolchain (C++17 + modern OpenGL or WebGPU), build scripts, test harness.}
\kv{Assumptions / Risks}{Numeric stability and platform differences may affect outputs; allow time for debugging.}
\textbf{Story} \quad As a learner of computer graphics, I want to complete \emph{Implicit Representations of Shape} so that I can apply its concepts in a working demo with tests.
\par\vspace{4pt}\textbf{Non-Functional} \quad \pill{Performance} \pill{Reliability} \pill{Accessibility}
\par\vspace{4pt}\textbf{Acceptance Criteria (BDD)}
\begin{itemize}
\item \textbf{Scenario} Happy path
\item \Given the chapter pre-reads and starter code are available
\item \When the hands-on objectives are implemented and tests run in CI
\item \Then artifacts (demo + notes) and a summary of outcomes are committed to the repo
\end{itemize}
\DoR\quad\textbullet\quad\DoD
\end{StoryCard}
\begin{TasksBox}
\begin{itemize}[label=$\square$,leftmargin=1.2em]
\item Implement signed distance fields and basic CSG (union, inter, diff).
\item Polygonize with marching cubes and validate normals.
\item Ray trace sphere/torus implicits analytically.
\item Build an SDF modeling sandbox; export OBJ.
\item Collect numerical issues and fixes (epsilon, band limits).
\end{itemize}
\end{TasksBox}
\clearpage

\begin{StoryCard}{CH-25 --- Meshes}
\kv{Epic / Feature}{Chapter Mastery}
\kv{Business Value}{Build demonstrable skill aligned to this chapter; produce a small artifact and tests.}
\kv{Priority / Estimate}{\textbf{Priority:} Must \quad \pill{SP: 3}}
\kv{Persona}{Graphics engineer / student working through CGPP 3/e}
\kv{Dependencies}{Toolchain (C++17 + modern OpenGL or WebGPU), build scripts, test harness.}
\kv{Assumptions / Risks}{Numeric stability and platform differences may affect outputs; allow time for debugging.}
\textbf{Story} \quad As a learner of computer graphics, I want to complete \emph{Meshes} so that I can apply its concepts in a working demo with tests.
\par\vspace{4pt}\textbf{Non-Functional} \quad \pill{Performance} \pill{Reliability} \pill{Accessibility}
\par\vspace{4pt}\textbf{Acceptance Criteria (BDD)}
\begin{itemize}
\item \textbf{Scenario} Happy path
\item \Given the chapter pre-reads and starter code are available
\item \When the hands-on objectives are implemented and tests run in CI
\item \Then artifacts (demo + notes) and a summary of outcomes are committed to the repo
\end{itemize}
\DoR\quad\textbullet\quad\DoD
\end{StoryCard}
\begin{TasksBox}
\begin{itemize}[label=$\square$,leftmargin=1.2em]
\item Build a half-edge mesh library with adjacency queries.
\item Implement LOD generation and vertex cache optimization.
\item Add mesh repair: weld near-duplicates, fill small holes.
\item Bench traversal and cache effectiveness.
\item Ship a mesh viewer with toggles for topology overlays.
\end{itemize}
\end{TasksBox}
\clearpage

\begin{StoryCard}{CH-26 --- Light}
\kv{Epic / Feature}{Chapter Mastery}
\kv{Business Value}{Build demonstrable skill aligned to this chapter; produce a small artifact and tests.}
\kv{Priority / Estimate}{\textbf{Priority:} Must \quad \pill{SP: 3}}
\kv{Persona}{Graphics engineer / student working through CGPP 3/e}
\kv{Dependencies}{Toolchain (C++17 + modern OpenGL or WebGPU), build scripts, test harness.}
\kv{Assumptions / Risks}{Numeric stability and platform differences may affect outputs; allow time for debugging.}
\textbf{Story} \quad As a learner of computer graphics, I want to complete \emph{Light} so that I can apply its concepts in a working demo with tests.
\par\vspace{4pt}\textbf{Non-Functional} \quad \pill{Performance} \pill{Reliability} \pill{Accessibility}
\par\vspace{4pt}\textbf{Acceptance Criteria (BDD)}
\begin{itemize}
\item \textbf{Scenario} Happy path
\item \Given the chapter pre-reads and starter code are available
\item \When the hands-on objectives are implemented and tests run in CI
\item \Then artifacts (demo + notes) and a summary of outcomes are committed to the repo
\end{itemize}
\DoR\quad\textbullet\quad\DoD
\end{StoryCard}
\begin{TasksBox}
\begin{itemize}[label=$\square$,leftmargin=1.2em]
\item Summarize radiometry vs photometry and common units.
\item Implement basic light types and sampling (point, area, env).
\item Demonstrate Fresnel effects in a simple shader.
\item Build a light sampling demo with PDFs and plots.
\item Draft a unit cheat-sheet used across the repo.
\end{itemize}
\end{TasksBox}
\clearpage

\begin{StoryCard}{CH-27 --- Materials \& Scattering}
\kv{Epic / Feature}{Chapter Mastery}
\kv{Business Value}{Build demonstrable skill aligned to this chapter; produce a small artifact and tests.}
\kv{Priority / Estimate}{\textbf{Priority:} Must \quad \pill{SP: 3}}
\kv{Persona}{Graphics engineer / student working through CGPP 3/e}
\kv{Dependencies}{Toolchain (C++17 + modern OpenGL or WebGPU), build scripts, test harness.}
\kv{Assumptions / Risks}{Numeric stability and platform differences may affect outputs; allow time for debugging.}
\textbf{Story} \quad As a learner of computer graphics, I want to complete \emph{Materials \& Scattering} so that I can apply its concepts in a working demo with tests.
\par\vspace{4pt}\textbf{Non-Functional} \quad \pill{Performance} \pill{Reliability} \pill{Accessibility}
\par\vspace{4pt}\textbf{Acceptance Criteria (BDD)}
\begin{itemize}
\item \textbf{Scenario} Happy path
\item \Given the chapter pre-reads and starter code are available
\item \When the hands-on objectives are implemented and tests run in CI
\item \Then artifacts (demo + notes) and a summary of outcomes are committed to the repo
\end{itemize}
\DoR\quad\textbullet\quad\DoD
\end{StoryCard}
\begin{TasksBox}
\begin{itemize}[label=$\square$,leftmargin=1.2em]
\item Implement Lambert, Blinn–Phong, Cook–Torrance (GGX).
\item Add importance sampling for GGX with visible normal sampling.
\item Compare Disney principled parameters to microfacet terms.
\item Visualize BRDF lobes and energy conservation numerically.
\item Create a material inspector UI and presets.
\end{itemize}
\end{TasksBox}
\clearpage

\begin{StoryCard}{CH-28 --- Color}
\kv{Epic / Feature}{Chapter Mastery}
\kv{Business Value}{Build demonstrable skill aligned to this chapter; produce a small artifact and tests.}
\kv{Priority / Estimate}{\textbf{Priority:} Must \quad \pill{SP: 3}}
\kv{Persona}{Graphics engineer / student working through CGPP 3/e}
\kv{Dependencies}{Toolchain (C++17 + modern OpenGL or WebGPU), build scripts, test harness.}
\kv{Assumptions / Risks}{Numeric stability and platform differences may affect outputs; allow time for debugging.}
\textbf{Story} \quad As a learner of computer graphics, I want to complete \emph{Color} so that I can apply its concepts in a working demo with tests.
\par\vspace{4pt}\textbf{Non-Functional} \quad \pill{Performance} \pill{Reliability} \pill{Accessibility}
\par\vspace{4pt}\textbf{Acceptance Criteria (BDD)}
\begin{itemize}
\item \textbf{Scenario} Happy path
\item \Given the chapter pre-reads and starter code are available
\item \When the hands-on objectives are implemented and tests run in CI
\item \Then artifacts (demo + notes) and a summary of outcomes are committed to the repo
\end{itemize}
\DoR\quad\textbullet\quad\DoD
\end{StoryCard}
\begin{TasksBox}
\begin{itemize}[label=$\square$,leftmargin=1.2em]
\item Implement CIE XYZ/Lab conversions and white-point adaptation.
\item Build a color-management path (linear↔sRGB, display transform).
\item Add filmic tone mapping with exposure/white-balance controls.
\item Validate gradients for banding and add dithering.
\item Write a color pipeline decision memo.
\end{itemize}
\end{TasksBox}
\clearpage

\begin{StoryCard}{CH-29 --- Light Transport}
\kv{Epic / Feature}{Chapter Mastery}
\kv{Business Value}{Build demonstrable skill aligned to this chapter; produce a small artifact and tests.}
\kv{Priority / Estimate}{\textbf{Priority:} Must \quad \pill{SP: 3}}
\kv{Persona}{Graphics engineer / student working through CGPP 3/e}
\kv{Dependencies}{Toolchain (C++17 + modern OpenGL or WebGPU), build scripts, test harness.}
\kv{Assumptions / Risks}{Numeric stability and platform differences may affect outputs; allow time for debugging.}
\textbf{Story} \quad As a learner of computer graphics, I want to complete \emph{Light Transport} so that I can apply its concepts in a working demo with tests.
\par\vspace{4pt}\textbf{Non-Functional} \quad \pill{Performance} \pill{Reliability} \pill{Accessibility}
\par\vspace{4pt}\textbf{Acceptance Criteria (BDD)}
\begin{itemize}
\item \textbf{Scenario} Happy path
\item \Given the chapter pre-reads and starter code are available
\item \When the hands-on objectives are implemented and tests run in CI
\item \Then artifacts (demo + notes) and a summary of outcomes are committed to the repo
\end{itemize}
\DoR\quad\textbullet\quad\DoD
\end{StoryCard}
\begin{TasksBox}
\begin{itemize}[label=$\square$,leftmargin=1.2em]
\item Derive the rendering equation in notes; enumerate path types.
\item Implement path enumeration stubs and debug visualizations.
\item Map rendering choices to transport assumptions.
\item Create sanity scenes to exercise different transport effects.
\item Document insights that guide solver choices.
\end{itemize}
\end{TasksBox}
\clearpage

\begin{StoryCard}{CH-30 --- Probability \& Monte Carlo Integration}
\kv{Epic / Feature}{Chapter Mastery}
\kv{Business Value}{Build demonstrable skill aligned to this chapter; produce a small artifact and tests.}
\kv{Priority / Estimate}{\textbf{Priority:} Must \quad \pill{SP: 3}}
\kv{Persona}{Graphics engineer / student working through CGPP 3/e}
\kv{Dependencies}{Toolchain (C++17 + modern OpenGL or WebGPU), build scripts, test harness.}
\kv{Assumptions / Risks}{Numeric stability and platform differences may affect outputs; allow time for debugging.}
\textbf{Story} \quad As a learner of computer graphics, I want to complete \emph{Probability \& Monte Carlo Integration} so that I can apply its concepts in a working demo with tests.
\par\vspace{4pt}\textbf{Non-Functional} \quad \pill{Performance} \pill{Reliability} \pill{Accessibility}
\par\vspace{4pt}\textbf{Acceptance Criteria (BDD)}
\begin{itemize}
\item \textbf{Scenario} Happy path
\item \Given the chapter pre-reads and starter code are available
\item \When the hands-on objectives are implemented and tests run in CI
\item \Then artifacts (demo + notes) and a summary of outcomes are committed to the repo
\end{itemize}
\DoR\quad\textbullet\quad\DoD
\end{StoryCard}
\begin{TasksBox}
\begin{itemize}[label=$\square$,leftmargin=1.2em]
\item Implement RNG, stratified sampling, and alias tables.
\item Estimate integrals with different estimators and compare variance.
\item Implement MIS on simple integrands; visualize variance reduction.
\item Build plots of convergence vs samples per pixel.
\item Write rules-of-thumb for sampler selection.
\end{itemize}
\end{TasksBox}
\clearpage

\begin{StoryCard}{CH-31 --- Solving the Rendering Equation (Theory)}
\kv{Epic / Feature}{Chapter Mastery}
\kv{Business Value}{Build demonstrable skill aligned to this chapter; produce a small artifact and tests.}
\kv{Priority / Estimate}{\textbf{Priority:} Must \quad \pill{SP: 3}}
\kv{Persona}{Graphics engineer / student working through CGPP 3/e}
\kv{Dependencies}{Toolchain (C++17 + modern OpenGL or WebGPU), build scripts, test harness.}
\kv{Assumptions / Risks}{Numeric stability and platform differences may affect outputs; allow time for debugging.}
\textbf{Story} \quad As a learner of computer graphics, I want to complete \emph{Solving the Rendering Equation (Theory)} so that I can apply its concepts in a working demo with tests.
\par\vspace{4pt}\textbf{Non-Functional} \quad \pill{Performance} \pill{Reliability} \pill{Accessibility}
\par\vspace{4pt}\textbf{Acceptance Criteria (BDD)}
\begin{itemize}
\item \textbf{Scenario} Happy path
\item \Given the chapter pre-reads and starter code are available
\item \When the hands-on objectives are implemented and tests run in CI
\item \Then artifacts (demo + notes) and a summary of outcomes are committed to the repo
\end{itemize}
\DoR\quad\textbullet\quad\DoD
\end{StoryCard}
\begin{TasksBox}
\begin{itemize}[label=$\square$,leftmargin=1.2em]
\item Implement a radiosity toy on a Cornell box.
\item Compare with a simple path tracer result.
\item Discuss finite elements vs stochastic estimators trade-offs.
\item Add toggles to inspect form factors and basis choices.
\item Summarize applicability per scene class.
\end{itemize}
\end{TasksBox}
\clearpage

\begin{StoryCard}{CH-32 --- Rendering in Practice}
\kv{Epic / Feature}{Chapter Mastery}
\kv{Business Value}{Build demonstrable skill aligned to this chapter; produce a small artifact and tests.}
\kv{Priority / Estimate}{\textbf{Priority:} Must \quad \pill{SP: 3}}
\kv{Persona}{Graphics engineer / student working through CGPP 3/e}
\kv{Dependencies}{Toolchain (C++17 + modern OpenGL or WebGPU), build scripts, test harness.}
\kv{Assumptions / Risks}{Numeric stability and platform differences may affect outputs; allow time for debugging.}
\textbf{Story} \quad As a learner of computer graphics, I want to complete \emph{Rendering in Practice} so that I can apply its concepts in a working demo with tests.
\par\vspace{4pt}\textbf{Non-Functional} \quad \pill{Performance} \pill{Reliability} \pill{Accessibility}
\par\vspace{4pt}\textbf{Acceptance Criteria (BDD)}
\begin{itemize}
\item \textbf{Scenario} Happy path
\item \Given the chapter pre-reads and starter code are available
\item \When the hands-on objectives are implemented and tests run in CI
\item \Then artifacts (demo + notes) and a summary of outcomes are committed to the repo
\end{itemize}
\DoR\quad\textbullet\quad\DoD
\end{StoryCard}
\begin{TasksBox}
\begin{itemize}[label=$\square$,leftmargin=1.2em]
\item Implement a minimal path tracer with next-event estimation.
\item Add photon mapping or bidirectional variant for comparison.
\item Integrate false-color and heatmap debugging views.
\item Build a scene loader and CLI renderer entry point.
\item Write a renderer troubleshooting checklist.
\end{itemize}
\end{TasksBox}
\clearpage

\begin{StoryCard}{CH-33 --- Shaders}
\kv{Epic / Feature}{Chapter Mastery}
\kv{Business Value}{Build demonstrable skill aligned to this chapter; produce a small artifact and tests.}
\kv{Priority / Estimate}{\textbf{Priority:} Must \quad \pill{SP: 3}}
\kv{Persona}{Graphics engineer / student working through CGPP 3/e}
\kv{Dependencies}{Toolchain (C++17 + modern OpenGL or WebGPU), build scripts, test harness.}
\kv{Assumptions / Risks}{Numeric stability and platform differences may affect outputs; allow time for debugging.}
\textbf{Story} \quad As a learner of computer graphics, I want to complete \emph{Shaders} so that I can apply its concepts in a working demo with tests.
\par\vspace{4pt}\textbf{Non-Functional} \quad \pill{Performance} \pill{Reliability} \pill{Accessibility}
\par\vspace{4pt}\textbf{Acceptance Criteria (BDD)}
\begin{itemize}
\item \textbf{Scenario} Happy path
\item \Given the chapter pre-reads and starter code are available
\item \When the hands-on objectives are implemented and tests run in CI
\item \Then artifacts (demo + notes) and a summary of outcomes are committed to the repo
\end{itemize}
\DoR\quad\textbullet\quad\DoD
\end{StoryCard}
\begin{TasksBox}
\begin{itemize}[label=$\square$,leftmargin=1.2em]
\item Create a shader playground with live GLSL/WGSL editing.
\item Implement Phong, environment mapping, and toon shading.
\item Add a compute shader for an image-processing task.
\item Set up shader compilation diagnostics and includes.
\item Document style guidelines for shader code.
\end{itemize}
\end{TasksBox}
\clearpage

\begin{StoryCard}{CH-34 --- Expressive (NPR) Rendering}
\kv{Epic / Feature}{Chapter Mastery}
\kv{Business Value}{Build demonstrable skill aligned to this chapter; produce a small artifact and tests.}
\kv{Priority / Estimate}{\textbf{Priority:} Must \quad \pill{SP: 3}}
\kv{Persona}{Graphics engineer / student working through CGPP 3/e}
\kv{Dependencies}{Toolchain (C++17 + modern OpenGL or WebGPU), build scripts, test harness.}
\kv{Assumptions / Risks}{Numeric stability and platform differences may affect outputs; allow time for debugging.}
\textbf{Story} \quad As a learner of computer graphics, I want to complete \emph{Expressive (NPR) Rendering} so that I can apply its concepts in a working demo with tests.
\par\vspace{4pt}\textbf{Non-Functional} \quad \pill{Performance} \pill{Reliability} \pill{Accessibility}
\par\vspace{4pt}\textbf{Acceptance Criteria (BDD)}
\begin{itemize}
\item \textbf{Scenario} Happy path
\item \Given the chapter pre-reads and starter code are available
\item \When the hands-on objectives are implemented and tests run in CI
\item \Then artifacts (demo + notes) and a summary of outcomes are committed to the repo
\end{itemize}
\DoR\quad\textbullet\quad\DoD
\end{StoryCard}
\begin{TasksBox}
\begin{itemize}[label=$\square$,leftmargin=1.2em]
\item Implement edge/feature extraction and quantization shading.
\item Design a stroke/mark pipeline with parameter controls.
\item Add salience-guided stylization toggles.
\item Build before/after gallery scenes.
\item Summarize NPR use-cases for real-time apps.
\end{itemize}
\end{TasksBox}
\clearpage

\begin{StoryCard}{CH-35 --- Motion}
\kv{Epic / Feature}{Chapter Mastery}
\kv{Business Value}{Build demonstrable skill aligned to this chapter; produce a small artifact and tests.}
\kv{Priority / Estimate}{\textbf{Priority:} Must \quad \pill{SP: 3}}
\kv{Persona}{Graphics engineer / student working through CGPP 3/e}
\kv{Dependencies}{Toolchain (C++17 + modern OpenGL or WebGPU), build scripts, test harness.}
\kv{Assumptions / Risks}{Numeric stability and platform differences may affect outputs; allow time for debugging.}
\textbf{Story} \quad As a learner of computer graphics, I want to complete \emph{Motion} so that I can apply its concepts in a working demo with tests.
\par\vspace{4pt}\textbf{Non-Functional} \quad \pill{Performance} \pill{Reliability} \pill{Accessibility}
\par\vspace{4pt}\textbf{Acceptance Criteria (BDD)}
\begin{itemize}
\item \textbf{Scenario} Happy path
\item \Given the chapter pre-reads and starter code are available
\item \When the hands-on objectives are implemented and tests run in CI
\item \Then artifacts (demo + notes) and a summary of outcomes are committed to the repo
\end{itemize}
\DoR\quad\textbullet\quad\DoD
\end{StoryCard}
\begin{TasksBox}
\begin{itemize}[label=$\square$,leftmargin=1.2em]
\item Implement skeletal animation and pose interpolation.
\item Add camera path editing with splines.
\item Demonstrate motion blur considerations.
\item Stress-test stability with long sequences.
\item Document interpolation pitfalls and fixes.
\end{itemize}
\end{TasksBox}
\clearpage

\begin{StoryCard}{CH-36 --- Visibility Determination}
\kv{Epic / Feature}{Chapter Mastery}
\kv{Business Value}{Build demonstrable skill aligned to this chapter; produce a small artifact and tests.}
\kv{Priority / Estimate}{\textbf{Priority:} Must \quad \pill{SP: 3}}
\kv{Persona}{Graphics engineer / student working through CGPP 3/e}
\kv{Dependencies}{Toolchain (C++17 + modern OpenGL or WebGPU), build scripts, test harness.}
\kv{Assumptions / Risks}{Numeric stability and platform differences may affect outputs; allow time for debugging.}
\textbf{Story} \quad As a learner of computer graphics, I want to complete \emph{Visibility Determination} so that I can apply its concepts in a working demo with tests.
\par\vspace{4pt}\textbf{Non-Functional} \quad \pill{Performance} \pill{Reliability} \pill{Accessibility}
\par\vspace{4pt}\textbf{Acceptance Criteria (BDD)}
\begin{itemize}
\item \textbf{Scenario} Happy path
\item \Given the chapter pre-reads and starter code are available
\item \When the hands-on objectives are implemented and tests run in CI
\item \Then artifacts (demo + notes) and a summary of outcomes are committed to the repo
\end{itemize}
\DoR\quad\textbullet\quad\DoD
\end{StoryCard}
\begin{TasksBox}
\begin{itemize}[label=$\square$,leftmargin=1.2em]
\item Implement depth buffering and backface/frustum culling.
\item Prototype occlusion queries and compare methods.
\item Benchmark visibility methods on different scenes.
\item Create failure-case gallery (precision, temporal).
\item Write guidance for algorithm selection.
\end{itemize}
\end{TasksBox}
\clearpage

\begin{StoryCard}{CH-37 --- Spatial Data Structures}
\kv{Epic / Feature}{Chapter Mastery}
\kv{Business Value}{Build demonstrable skill aligned to this chapter; produce a small artifact and tests.}
\kv{Priority / Estimate}{\textbf{Priority:} Must \quad \pill{SP: 3}}
\kv{Persona}{Graphics engineer / student working through CGPP 3/e}
\kv{Dependencies}{Toolchain (C++17 + modern OpenGL or WebGPU), build scripts, test harness.}
\kv{Assumptions / Risks}{Numeric stability and platform differences may affect outputs; allow time for debugging.}
\textbf{Story} \quad As a learner of computer graphics, I want to complete \emph{Spatial Data Structures} so that I can apply its concepts in a working demo with tests.
\par\vspace{4pt}\textbf{Non-Functional} \quad \pill{Performance} \pill{Reliability} \pill{Accessibility}
\par\vspace{4pt}\textbf{Acceptance Criteria (BDD)}
\begin{itemize}
\item \textbf{Scenario} Happy path
\item \Given the chapter pre-reads and starter code are available
\item \When the hands-on objectives are implemented and tests run in CI
\item \Then artifacts (demo + notes) and a summary of outcomes are committed to the repo
\end{itemize}
\DoR\quad\textbullet\quad\DoD
\end{StoryCard}
\begin{TasksBox}
\begin{itemize}[label=$\square$,leftmargin=1.2em]
\item Implement BVH and KD-tree builders (basic SAH).
\item Benchmark ray/frustum traversal performance.
\item Visualize tree depth and node usage statistics.
\item Swap structures at runtime and compare.
\item Draft a tuning guide for builders.
\end{itemize}
\end{TasksBox}
\clearpage

\begin{StoryCard}{CH-38 --- Modern Graphics Hardware}
\kv{Epic / Feature}{Chapter Mastery}
\kv{Business Value}{Build demonstrable skill aligned to this chapter; produce a small artifact and tests.}
\kv{Priority / Estimate}{\textbf{Priority:} Must \quad \pill{SP: 3}}
\kv{Persona}{Graphics engineer / student working through CGPP 3/e}
\kv{Dependencies}{Toolchain (C++17 + modern OpenGL or WebGPU), build scripts, test harness.}
\kv{Assumptions / Risks}{Numeric stability and platform differences may affect outputs; allow time for debugging.}
\textbf{Story} \quad As a learner of computer graphics, I want to complete \emph{Modern Graphics Hardware} so that I can apply its concepts in a working demo with tests.
\par\vspace{4pt}\textbf{Non-Functional} \quad \pill{Performance} \pill{Reliability} \pill{Accessibility}
\par\vspace{4pt}\textbf{Acceptance Criteria (BDD)}
\begin{itemize}
\item \textbf{Scenario} Happy path
\item \Given the chapter pre-reads and starter code are available
\item \When the hands-on objectives are implemented and tests run in CI
\item \Then artifacts (demo + notes) and a summary of outcomes are committed to the repo
\end{itemize}
\DoR\quad\textbullet\quad\DoD
\end{StoryCard}
\begin{TasksBox}
\begin{itemize}[label=$\square$,leftmargin=1.2em]
\item Summarize GPU execution/memory models and latency hiding.
\item Build microbenchmarks for vertex transform and texturing.
\item Profile a shader and identify stalls vs occupancy.
\item Experiment with buffer update strategies (streaming, persist).
\item Write a post-mortem on performance lessons.
\end{itemize}
\end{TasksBox}
\clearpage

\end{document}