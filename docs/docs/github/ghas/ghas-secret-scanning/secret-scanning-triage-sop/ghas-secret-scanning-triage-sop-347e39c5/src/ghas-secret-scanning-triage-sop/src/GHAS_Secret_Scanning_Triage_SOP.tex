
% !TEX TS-program = pdflatex
\documentclass[11pt,a4paper]{article}

% -------------------- Packages --------------------
\usepackage[T1]{fontenc}
\usepackage[utf8]{inputenc}
\usepackage{lmodern}
\usepackage{microtype}
\usepackage[a4paper,margin=1in]{geometry}
\usepackage{parskip}
\usepackage[hyphens]{url}
\usepackage{hyperref}
\usepackage{bookmark}
\usepackage{enumitem}
\usepackage[dvipsnames]{xcolor}
\usepackage{booktabs}
\usepackage{tabularx}
\usepackage{array}
\usepackage[most]{tcolorbox}
\usepackage{fancyhdr}
\usepackage{titlesec}
\usepackage{pifont}

% -------------------- Hyperref --------------------
\hypersetup{
  colorlinks=true,
  linkcolor=MidnightBlue,
  urlcolor=MidnightBlue,
  citecolor=MidnightBlue
}

% -------------------- Header / Footer -------------
\pagestyle{fancy}
\fancyhf{}
\lhead{GHAS Secret Scanning Triage SOP}
\rhead{\thepage}
\renewcommand{\headrulewidth}{0.4pt}
\renewcommand{\footrulewidth}{0pt}

% -------------------- Styles ----------------------
\definecolor{brand}{RGB}{38,70,83}
\definecolor{accent}{RGB}{69,123,157}
\definecolor{softbg}{RGB}{245,248,250}
\definecolor{ok}{RGB}{31,135,68}
\definecolor{warn}{RGB}{203,132,0}
\definecolor{crit}{RGB}{179,35,35}

\newcommand{\cmark}{\ding{51}}
\newcommand{\xmark}{\ding{55}}

\titleformat{\section}
  {\large\bfseries\color{brand}}{\thesection}{0.8em}{}
\titleformat{\subsection}
  {\normalsize\bfseries\color{brand}}{\thesubsection}{0.6em}{}

% -------------------- Boxes -----------------------
\tcbset{
  colback=softbg,
  colframe=accent,
  arc=1.5mm,
  boxsep=1mm,
  left=2mm,right=2mm,top=2mm,bottom=2mm
}

\newtcolorbox{callout}[1][]{
  enhanced,
  colback=softbg,
  colframe=accent,
  title=#1
}

\newtcolorbox{criticalbox}[1][]{
  enhanced,
  colback=softbg,
  colframe=crit,
  title=#1
}

\newtcolorbox{warnbox}[1][]{
  enhanced,
  colback=softbg,
  colframe=warn,
  title=#1
}

% -------------------- Title -----------------------
\title{\textbf{Secret Scanning Triage SOP}\\[2pt]\large GitHub Advanced Security (GHAS)}
\author{Security Engineering}
\date{\today}

% -------------------- Document --------------------
\begin{document}
\maketitle
\begin{callout}[Document Control]
\textbf{Version:} 1.0 \hfill \textbf{Last Updated:} \today\\
\textbf{Owner:} Security Engineering (GHAS Admins) \hfill \textbf{Applies To:} All private/org repositories
\end{callout}

\tableofcontents
\bigskip

\section{Purpose \& Scope}
This Standard Operating Procedure (SOP) defines how the organization triages, remediates, and documents \textbf{GitHub Advanced Security (GHAS) Secret Scanning} alerts. It covers alerts raised by push protection (on push), on pull requests (PR checks), and retro/background scans across the commit history. Goals:
\begin{itemize}[nosep,leftmargin=1.2em]
  \item Prevent credential leaks from reaching default branches and releases.
  \item Respond rapidly based on impact-driven \textbf{SLAs}.
  \item Ensure consistent evidence, auditability, and reporting.
\end{itemize}

\section{Roles \& Access}
\begin{tabularx}{\linewidth}{>{\bfseries}l X}
\toprule
Developers & Remove secrets from code/PRs, rotate/revoke credentials, add remediation notes. \\
Repo Admins / Code Owners & Review/approve remediation PRs, enforce repo policy, coordinate fixes. \\
Security (GHAS Admins) & Define policy, tune patterns, manage org/repo settings, audit and reporting. \\
Access & Only users granted \emph{read security alerts} can view repo alerts. Org-level security managers can view/manage alerts across repositories. \\
\bottomrule
\end{tabularx}

\section{Where Alerts Come From}
\begin{itemize}[leftmargin=1.2em]
  \item \textbf{Push Protection (on push):} Blocks commits that contain secrets; committer must remove the secret or request an override with justification.
  \item \textbf{PR Checks:} Secrets surfaced during pull request validation to shift left.
  \item \textbf{Retro/Background Scans:} Scans find secrets in history or inactive branches.
\end{itemize}

\section{Severity Levels \& SLAs}
\begin{tabularx}{\linewidth}{p{6em} p{20em} p{8em}}
\toprule
\textbf{Severity} & \textbf{Description} & \textbf{SLA} \\
\midrule
\textcolor{crit}{\textbf{P0}} & Active production credential (cloud root/admin, DB admin, payment keys). & Contain within \textbf{1 hour}; rotate immediately. \\
\textcolor{warn}{\textbf{P1}} & Privileged non-production credential or elevated risk. & \textbf{24 hours}. \\
\textcolor{ok}{\textbf{P2}} & Low-impact, expired, or test credential. & \textbf{3 business days}. \\
\bottomrule
\end{tabularx}
\clearpage

\section{End-to-End Triage Workflow}
\subsection*{1) Acknowledge \& Assign}
Auto-assign to \textbf{Committer + Code Owners}; Security added as watcher. If push protection blocked the push, the committer must remediate or submit an exception with justification (recorded by GitHub).

\subsection*{2) Verify the Finding}
Confirm the item is truly a credential. Prefer provider/partner patterns; treat custom patterns carefully to avoid noise.

\subsection*{3) Contain}
\begin{itemize}[leftmargin=1.2em]
  \item Remove the secret from the PR/branch immediately.
  \item \textbf{Rotate/revoke} the exposed credential with its provider (treat as compromised).
\end{itemize}

\subsection*{4) Eradicate}
If the secret reached history, open a task to purge or rewrite history (e.g., \texttt{git filter-repo}) and \textbf{invalidate} any still-valid token.

\subsection*{5) Recover \& Harden}
Tune detection to prevent recurrence:
\begin{itemize}[leftmargin=1.2em]
  \item Add or refine \textbf{custom patterns} (regex) for proprietary secrets.
  \item Test new patterns in the GitHub UI to reduce false positives before publishing org-wide.
\end{itemize}

\subsection*{6) Document \& Disposition}
In the alert record, add remediation notes and set status:
\begin{itemize}[leftmargin=1.2em]
  \item \textbf{Resolved}: Secret removed, rotated, and (if needed) history cleaned.
  \item \textbf{Dismissed}: Select a reason (Section~\ref{sec:dismissal}) and add a brief comment; dismissed alerts remain visible for audit.
\end{itemize}

\section{Notification Routing}
\begin{callout}[Principles]
Notify the fewest people who can act, escalate only when needed, and avoid alert fatigue by tuning notification preferences.
\end{callout}

\begin{tabularx}{\linewidth}{>{\bfseries}l X}
\toprule
Push Protection Block & Committer (required), Code Owners, Security channel (informational). If an override is requested, page Security. \\
PR Alert & PR Author and Code Owners (action), Security (watch). \\
Background/Retro Finding & Repo Admin (owns remediation), Security (tracker). \\
Channels & GitHub notifications (per-user), plus the incident channel (e.g., Slack/Email) for P0/P1. \\
\bottomrule
\end{tabularx}

\section{Dismissal Reasons Policy}\label{sec:dismissal}
When dismissing an alert, you \textbf{must} choose a reason and add a short comment. Accepted reasons:
\begin{itemize}[leftmargin=1.2em]
  \item \textbf{False positive}: Matches the pattern but not a credential (include 1-line proof).
  \item \textbf{Test/intentional fixture}: Dummy/test key per test policy (link to policy).
  \item \textbf{Expired/Revoked}: Verified invalid; include ticket/ID of revocation.
  \item \textbf{Duplicate}: Same secret already tracked under a different alert (link it).
  \item \textbf{Compensating control}: Allowed only with Security approval and documented control.
\end{itemize}

\begin{criticalbox}[Never dismiss an active/valid credential]
Dismissal must not be used to avoid remediation of a valid secret. Treat all valid credentials as compromised until rotated or revoked.
\end{criticalbox}

\section{Exceptions (Push Protection Allow-List)}
If a committer overrides a push-protection block, they must provide a justification; identity and rationale are recorded by GitHub. Security reviews all exceptions daily and may revoke allow-list entries at any time.

\section{Reporting \& Audit}
\begin{itemize}[leftmargin=1.2em]
  \item \textbf{Weekly}: Security reviews opened/closed counts by repo and \emph{dismissal reasons}; verify push-protection exceptions.
  \item \textbf{Monthly}: Pattern tuning---add org-level patterns, retire noisy ones; test regex in UI before publishing.
  \item \textbf{Quarterly}: Retro scan review for inactive repositories and long-lived branches.
\end{itemize}

\section{Quick-Action Checklist (On-Call)}
\begin{enumerate}[leftmargin=1.2em]
  \item Identify severity (\textbf{P0}/\textbf{P1}/\textbf{P2}).
  \item Contain: remove secret and rotate/revoke immediately.
  \item Hunt for other occurrences (full repo/history search).
  \item Document remediation notes; link revocation evidence.
  \item Disposition: \textbf{Resolve} or \textbf{Dismiss} with reason.
  \item Tune patterns if applicable; test before rolling out org-wide.
\end{enumerate}

\section*{Appendix A --- Triage Notes Form}
\begin{callout}
\textbf{Alert ID:} \_\_\_\_\_\_ \quad \textbf{Repository:} \_\_\_\_\_\_ \quad \textbf{Branch/PR:} \_\_\_\_\_\_\\
\textbf{Committer:} \_\_\_\_\_\_ \quad \textbf{Assignee:} \_\_\_\_\_\_ \quad \textbf{Severity (P0/P1/P2):} \_\_\_\\[2mm]
\textbf{Containment:} Secret removed? (\cmark/\xmark) \quad Rotation/Revocation ticket: \_\_\_\_\_\\[2mm]
\textbf{Eradication:} History cleaned? (\cmark/\xmark) \quad Method: \_\_\_\_\_\\[2mm]
\textbf{Disposition:} Resolved / Dismissed (Reason: \_\_\_\_\_)\\[2mm]
\textbf{Notes/Evidence:} \hrulefill\\[2mm]
\textbf{Follow-ups:} Pattern tuning / Training / Control update \_\_\_\_\_
\end{callout}

\section*{Appendix B --- Exception Request (Push Protection Override)}
\begin{warnbox}
\textbf{Requester:} \_\_\_\_\_ \quad \textbf{Repo/Branch:} \_\_\_\_\_ \quad \textbf{Date:} \_\_\_\_\_\\[2mm]
\textbf{Justification:} \hrulefill\\[3mm]
\textbf{Risk Assessment (Security):} \hrulefill\\[3mm]
\textbf{Decision:} Approved / Denied \quad \textbf{Reviewer:} \_\_\_\_\_ \quad \textbf{Expiry:} \_\_\_\_\_
\end{warnbox}

\section*{Appendix C --- Glossary}
\textbf{GHAS}: GitHub Advanced Security. \quad
\textbf{PR}: Pull Request. \quad
\textbf{SLA}: Service Level Agreement.

\vfill
\begin{center}
{\small \textcopyright\ \the\year\ Security Engineering. All rights reserved.}
\end{center}

\end{document}
