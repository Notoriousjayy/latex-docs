
% !TEX TS-program = pdflatex
% Compile with: pdflatex -shell-escape GHAS_Secret_Scanning_Cheat_Sheet.tex
\documentclass[11pt,a4paper]{article}

% ---------- Packages ----------
\usepackage[a4paper,margin=0.85in]{geometry}
\usepackage[T1]{fontenc}
\usepackage[utf8]{inputenc}
\usepackage{lmodern}
\usepackage{microtype}
\usepackage{parskip}
\usepackage[dvipsnames]{xcolor}
\usepackage{hyperref}
\hypersetup{
  colorlinks=true,
  linkcolor=NavyBlue,
  urlcolor=MidnightBlue,
  citecolor=MidnightBlue
}
\usepackage{enumitem}
\setlist{itemsep=3pt, topsep=4pt, leftmargin=1.2em}
\usepackage{titlesec}
\titlespacing*{\section}{0pt}{8pt plus 2pt}{4pt}
\titlespacing*{\subsection}{0pt}{6pt}{3pt}
\usepackage{booktabs}
\usepackage{array}
\usepackage[most]{tcolorbox}
\tcbuselibrary{skins,breakable}

% Code blocks with syntax highlighting (requires -shell-escape)
\usepackage[newfloat]{minted}
\setminted{
  fontsize=\small,
  breaklines=true,
  autogobble,
  tabsize=2
}
\renewcommand\theFancyVerbLine{\sffamily\scriptsize\color{gray}\arabic{FancyVerbLine}}

% ---------- Styling ----------
\definecolor{CardStripe}{HTML}{2B6CB0} % blue
\definecolor{CardFrame}{HTML}{1A365D}  % darker blue
\definecolor{CardBg}{HTML}{F7FAFC}     % light gray-blue
\definecolor{Accent}{HTML}{22543D}     % green accent
\definecolor{NoteBg}{HTML}{FFFDF5}     % note background

\newtcolorbox{callout}[1][]{
  enhanced,
  breakable,
  colback=CardBg,
  colframe=CardFrame,
  left=8pt,right=8pt,top=8pt,bottom=8pt,
  boxrule=0.8pt,
  borderline west={4pt}{0pt}{CardStripe},
  sharp corners,
  title style={font=\bfseries\sffamily},
  fonttitle=\bfseries\sffamily,
  #1
}

\newtcolorbox{examtip}[1][]{
  enhanced, breakable,
  colback=NoteBg, colframe=Goldenrod,
  left=8pt,right=8pt,top=8pt,bottom=8pt,
  boxrule=0.8pt, title=\textsf{Exam Tip},
  title style={font=\bfseries\sffamily},
  #1
}

\newcommand{\kpi}[1]{\textbf{\textsf{#1}}}

% ---------- Document ----------
\begin{document}

\begin{center}
{\LARGE\bfseries GHAS Secret Scanning — Exam / Cheat Sheet}\\[2pt]
{\small \textit{GitHub Advanced Security (GHAS) focus: features, setup, workflow, and exam‑ready facts.}}
\end{center}

\begin{callout}[title=Core Idea]
\textbf{Secret scanning} detects sensitive values (API keys, tokens, credentials) across your repository content and history. With \textbf{push protection}, supported secrets can be blocked \emph{before} they land in the repo.
\end{callout}

\section{When Scans Run}
\begin{itemize}
  \item \kpi{On push}: recommended; push protection can block secrets pre‑merge.
  \item \kpi{On pull request}: checks run and raise alerts before merge to default branch.
  \item \kpi{Historical \& ongoing}: newly added commits are scanned; full history can be scanned by GitHub.
\end{itemize}

\section{Scope \& Licensing}
\begin{itemize}
  \item \kpi{Public repositories}: baseline provider coverage available.
  \item \kpi{Private/Internal repositories}: enable via \kpi{GitHub Advanced Security (GHAS)} to unlock full features (custom patterns, push protection controls, and additional options).
\end{itemize}

\section{Enablement \& Where to Click}
\subsection*{Organization level}
\begin{enumerate}
  \item \texttt{Settings} $\rightarrow$ \texttt{Security} $\rightarrow$ \texttt{Code security and analysis}.
  \item Enable: \textbf{Secret scanning}, \textbf{non‑provider patterns}, \textbf{validity checks}, \textbf{push protection}.
\end{enumerate}

\subsection*{Repository level}
\begin{itemize}
  \item Inherits org defaults; can be overridden by repo administrators/owners as permitted.
\end{itemize}

\section{Push Protection (Developer Experience)}
\begin{itemize}
  \item Blocks pushes containing supported secret formats; developer sees a helpful message and remediation guidance.
  \item Distinct from \textit{branch protection}; it prevents the push itself rather than failing a PR check post‑push.
\end{itemize}

\section{Patterns \& Coverage}
\begin{itemize}
  \item \kpi{Provider/partner patterns}: well‑known token formats (cloud, SaaS, etc.).
  \item \kpi{Non‑provider patterns}: generic patterns for likely secrets (e.g., private keys, DB strings).
  \item \kpi{Custom patterns}: define and test your regex at org or repo scope; use dry‑run/testing tools to reduce noise. (Capacity limits apply; prefer org‑level for reuse.)
\end{itemize}
\clearpage

\section{Repo Configuration: \texttt{.github/secret\_scanning.yml}}
Place a configuration file to tune allowlists and optional repo‑local custom patterns.

\begin{minted}[fontsize=\footnotesize]{yaml}
# .github/secret_scanning.yml
version: 1

# Allowlist paths or literal values to suppress known benign matches.
allowlist:
  paths:
    - "docs/examples/**"
    - "test/fixtures/**"
  secrets:
    - "DUMMY-KEY-DO-NOT-USE"

# Optional: repo-scoped custom patterns (consider org-level for reuse)
custom_patterns:
  - name: "Acme Internal Token"
    regex: "ACME-[A-Za-z0-9]{32}"
    # (optional) specify "max_issuers", "keywords", etc., if supported in your UI.
\end{minted}

\section{Alert Triage Workflow}
\begin{enumerate}
  \item \kpi{Assess} the alert: secret type, location (commit/PR/branch), exposure blast radius.
  \item \kpi{Remediate}: revoke/rotate at the provider; remove/replace from code; consider history rewrite if necessary.
  \item \kpi{Record} outcome: add notes, link to the rotation ticket, and resolve/dismiss appropriately.
\end{enumerate}

\subsection*{Dismissal Reasons (standard policy menu)}
\begin{itemize}
  \item False positive / test fixture
  \item Secret revoked / rotated
  \item Accepted risk (non‑prod/short‑lived) — with justification
  \item Not a secret (format match only)
\end{itemize}

\section{Visibility \& Permissions}
\begin{itemize}
  \item Minimum: users need permission to \textit{read security alerts} on the repo.
  \item Organization security admins can view org‑wide alerts depending on role settings.
\end{itemize}

\section{Where Alerts Live}
\begin{itemize}
  \item Repository: \texttt{Security} $\rightarrow$ \texttt{Secret scanning}.
  \item Related features live nearby (e.g., dependency graph, code scanning) but are distinct.
\end{itemize}

\section{Exam Quick Hits}
\begin{itemize}
  \item Config file name and path: \texttt{.github/secret\_scanning.yml}.
  \item \textbf{Push protection} blocks pushes (pre‑merge), not only PR checks.
  \item Differentiate \textit{provider} vs \textit{non‑provider} patterns; know that many providers support validity checks.
  \item Favor \textit{org‑level custom patterns} for consistency and reuse.
  \item Document your dismissal reasons and route alerts (e.g., to AppSec/Platform) for audits.
\end{itemize}

\section{Sample Regex (Reasoning Practice)}
\begin{minted}[fontsize=\footnotesize]{text}
^sk-[A-Za-z0-9]{48}$
\end{minted}
This stands in for a typical provider token shape. For the exam, recognize patterns and consider: is there a prefix, exact length, and allowed alphabet? Add keywords/anchors where appropriate to reduce false positives.

\begin{examtip}
\textbf{Compile tip:} This document uses \texttt{minted} for syntax highlighting. Compile with \texttt{-shell-escape} (e.g., \texttt{pdflatex -shell-escape GHAS\_Secret\_Scanning\_Cheat\_Sheet.tex}). If your toolchain can't use \texttt{minted}, switch to \texttt{verbatim} or \texttt{listings}.
\end{examtip}

\vspace{4pt}
\noindent\rule{\linewidth}{0.4pt}\\[-2pt]
{\footnotesize\textit{Prepared for quick recall during reviews and exams. Tailor the allowlists, patterns, and routing to your organization’s policies.}}

\end{document}
