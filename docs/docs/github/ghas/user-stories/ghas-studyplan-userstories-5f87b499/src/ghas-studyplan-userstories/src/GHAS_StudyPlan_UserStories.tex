
% !TEX TS-program = pdflatex
\documentclass[11pt,a4paper]{article}

% -------------------- Packages --------------------
\usepackage[T1]{fontenc}
\usepackage[utf8]{inputenc}
\usepackage{lmodern}
\usepackage{microtype}
\usepackage[margin=1in]{geometry}
\usepackage{parskip}
\usepackage{enumitem}
\usepackage[dvipsnames]{xcolor}
\usepackage{tabularx}
\usepackage{array}
\usepackage{ragged2e}
\usepackage{booktabs}
\usepackage{hyperref}
\usepackage{bookmark}
\usepackage{amssymb,amsmath}
\usepackage[most]{tcolorbox}

% -------------------- Colors --------------------
\definecolor{Ink}{RGB}{25,28,35}
\definecolor{Steel}{RGB}{96,110,123}
\definecolor{Teal}{RGB}{3,147,152}
\definecolor{Mist}{RGB}{246,248,250}
\definecolor{Slate}{RGB}{70,79,88}
\definecolor{Accent}{RGB}{0,114,178}

% -------------------- General formatting --------------------
\hypersetup{colorlinks=true, linkcolor=Accent, urlcolor=Accent}
\setlist{itemsep=4pt, topsep=4pt, leftmargin=1.3em}

% Label column for metadata tables
\newcolumntype{L}{>{\raggedleft\arraybackslash\bfseries}p{0.28\linewidth}}
\newcolumntype{V}{>{\RaggedRight\arraybackslash}X}

% Small "badge" chips (non-functional tags etc.)
\newtcbox{\ghasbadge}{on line, arc=3pt, boxsep=0.6pt, left=4pt,right=4pt,top=1pt,bottom=1pt,
  colframe=Steel!60, colback=Steel!12, boxrule=0.3pt}

% Header pill
\newtcbox{\ghasheader}{on line, arc=2pt, boxsep=0.6pt, left=6pt,right=6pt,top=2pt,bottom=2pt,
  colframe=Teal!70!black, colback=Teal!6, boxrule=0.6pt}

% Checkboxes
\newcommand{\cb}{\(\square\)}

% -------------------- Card + Tasks styles --------------------
\tcbset{
  ghascard/.style={
    breakable, enhanced,
    colframe=Steel!40, colback=Mist, coltitle=Ink,
    borderline west={2.4pt}{0pt}{Teal!80!black},
    arc=2pt, boxrule=0.5pt,
    left=10pt,right=10pt,top=8pt,bottom=8pt,
    before skip=8pt, after skip=12pt
  },
  ghastasks/.style={
    breakable, enhanced,
    colframe=Steel!40, colback=white,
    borderline west={2.4pt}{0pt}{Teal!80!black},
    arc=2pt, boxrule=0.5pt,
    left=10pt,right=10pt,top=6pt,bottom=6pt,
    before skip=4pt, after skip=10pt
  }
}

% Card environment
\newenvironment{GHASCard}[2]{%
  \begin{tcolorbox}[ghascard,title={\ghasheader{#1 --- #2}}]
  \small
}{%
  \end{tcolorbox}
}

% Tasks environment
\newenvironment{GHASTasks}{%
  \begin{tcolorbox}[ghastasks]
  \small
  \begin{itemize}[label=\cb, leftmargin=*, labelsep=0.6em]
}{%
  \end{itemize}
  \end{tcolorbox}
}

% Helpful shorthands
\newcommand{\DoR}{\textbf{Definition of Ready:} Persona clear; AC drafted; Dependencies known; Estimate set.}
\newcommand{\DoD}{\textbf{Definition of Done:} All ACs pass; tests green; security/a11y checks; docs updated; deployed/flagged.}

% -------------------- Title --------------------
\title{\textbf{Study Plan \& User Stories — GitHub Advanced Security (GHAS)}}
\author{}
\date{\today}

\begin{document}
\maketitle
\tableofcontents
\bigskip

\section{How to Use This Document}
This document is a polished, standalone template for GHAS study planning using user stories. Each backlog item is rendered as a \emph{story card} followed by a concrete \emph{tasks} checklist. Duplicate a card for each item you want to track. Fields are intentionally concise and testable.

\subsection*{Writing Effective User Stories}
Use this formula:
\begin{quote}
\textit{As a \textbf{[persona]}, I want to \textbf{[do/achieve]}, so that \textbf{[business outcome]}.}
\end{quote}
\textbf{Good} stories describe \emph{one} valuable behavior, include acceptance criteria (BDD style), and tie to observable outcomes. Avoid implementation detail in the story---put it in tasks. Keep estimates small (1--5 SP).

\textbf{Examples}
\begin{itemize}
  \item \textbf{Good:} \emph{As an org admin, I want to enforce security checks via rulesets so that all PRs are gated on CodeQL and secret scanning.}
  \item \textbf{Good:} \emph{As a security engineer, I want to author a custom CodeQL query pack so that we detect org-specific sinks.}
  \item \textbf{Anti-pattern:} \emph{Set up all of GHAS this quarter.} (too broad, no persona, no outcome)
\end{itemize}

\subsection*{Non-Functional Tags}
Use badges to call out cross-cutting concerns: \ghasbadge{Performance}\ \ghasbadge{Security}\ \ghasbadge{Reliability}\ \ghasbadge{Accessibility}\ \ghasbadge{Privacy}\ \ghasbadge{i18n}.

\subsection*{Prerequisites Checklist}
\begin{itemize}
  \item Admin access to a GitHub Enterprise/Team organization with GHAS licenses.
  \item Sample repositories (at least one compiled language project).
  \item Ability to create org \& repo \emph{rulesets}, enable security features, and view \emph{Security overview}.
\end{itemize}

% -----------------------------------------------------------------------------
\section{Study Roadmap (8 Weeks)}
Each week is one primary story card (with BDD acceptance criteria) and a task checklist. Adjust estimates and personas to fit your context.

% -------------------- Week 1 --------------------
\begin{GHASCard}{GHAS-1}{Foundations \& Governance}
\begin{tabularx}{\linewidth}{L V}
Epic / Feature & Program Foundations / Org Governance \tabularnewline
Business Value & Establish shared understanding of GHAS, fast feedback, and ``keep main green'' to reduce risk. \tabularnewline
Priority / Estimate & Priority: Must \quad SP: 3 \tabularnewline
Persona & Org admin / platform engineer \tabularnewline
Dependencies & Test organization and 3 seed repositories \tabularnewline
Assumptions / Risks & Time to enable features varies by repo; risk of noisy alerts initially \tabularnewline
\end{tabularx}

\medskip
\textbf{Story}\quad \emph{As an org admin, I want to enable GHAS foundations and configure repository \textbf{rulesets} so that PRs are gated on security checks and the org baseline is measurable.}

\medskip
\textbf{Non-Functional}\quad
\ghasbadge{Security}\ \ghasbadge{Reliability}\ \ghasbadge{Privacy}

\medskip
\textbf{Acceptance Criteria (BDD)}
\begin{description}[leftmargin=2.7cm, labelwidth=2.6cm, style=nextline]
  \item[\textbf{Scenario}] Happy path
  \item[\textbf{Given}] a sandbox org with 3 repos and permissions to manage security settings
  \item[\textbf{When}] rulesets and GHAS features are enabled per policy
  \item[\textbf{Then}] PRs require CodeQL and secret scanning checks; Security overview shows baseline metrics
\end{description}

\medskip
\DoR\quad\textbullet\quad\DoD
\end{GHASCard}

\begin{GHASTasks}
\item Enable on 3 repos: Dependency graph, Dependabot alerts/updates, secret scanning, code scanning (default setup).
\item Create org rulesets enforcing: required checks (CodeQL, secret scanning), linear history, signed commits.
\item Configure branch protections on \texttt{main} \& \texttt{release/*}; block force-push and direct commits.
\item Capture baseline in Security overview: open alerts by type, age $>$ 30 days.
\item Document governance in the platform handbook.
\end{GHASTasks}
\clearpage

% -------------------- Week 2 --------------------
\begin{GHASCard}{GHAS-2}{Code Scanning with CodeQL (Essentials)}
\begin{tabularx}{\linewidth}{L V}
Epic / Feature & Code Scanning \tabularnewline
Business Value & Detect high-impact vulnerabilities early; create PR-gated signal. \tabularnewline
Priority / Estimate & Priority: Must \quad SP: 5 \tabularnewline
Persona & Security engineer / repo maintainer \tabularnewline
Dependencies & GHAS-1 completed; languages identified \tabularnewline
Assumptions / Risks & False positives must be triaged; build steps for compiled languages may require caching \tabularnewline
\end{tabularx}

\medskip
\textbf{Story}\quad \emph{As a security engineer, I want to configure CodeQL default setup and PR checks so that critical issues are caught before merge.}

\medskip
\textbf{Non-Functional}\quad
\ghasbadge{Security}\ \ghasbadge{Reliability}

\medskip
\textbf{Acceptance Criteria (BDD)}
\begin{description}[leftmargin=2.7cm, labelwidth=2.6cm, style=nextline]
  \item[\textbf{Scenario}] Happy path
  \item[\textbf{Given}] repositories with CodeQL enabled
  \item[\textbf{When}] a PR introduces a vulnerable pattern
  \item[\textbf{Then}] the PR check fails, an alert is created, and triage notes are recorded
\end{description}

\medskip
\DoR\quad\textbullet\quad\DoD
\end{GHASCard}

\begin{GHASTasks}
\item Turn on \emph{Default setup} for 3 repos; verify first analysis completes.
\item Add schedules (nightly) and enable PR-only analysis for long builds.
\item Define triage workflow: labels, assignees, SLAs; close or suppress top 10 alerts with justifications.
\item Export SARIF from one run and archive in the security evidence folder.
\end{GHASTasks}
\clearpage

% -------------------- Week 3 --------------------
\begin{GHASCard}{GHAS-3}{CodeQL Deep Dive: CLI, Databases, Custom Queries}
\begin{tabularx}{\linewidth}{L V}
Epic / Feature & CodeQL Query Authoring \tabularnewline
Business Value & Detect org-specific anti-patterns and reduce MTTR with precise alerts. \tabularnewline
Priority / Estimate & Priority: Should \quad SP: 8 \tabularnewline
Persona & Security engineer \tabularnewline
Dependencies & GHAS-2; local dev environment for CodeQL CLI \tabularnewline
Assumptions / Risks & Large projects may require extraction tuning; query quality must be validated \tabularnewline
\end{tabularx}

\medskip
\textbf{Story}\quad \emph{As a security engineer, I want to author and ship a custom CodeQL query pack so that our repos detect org-specific vulnerabilities.}

\medskip
\textbf{Non-Functional}\quad
\ghasbadge{Security}\ \ghasbadge{Reliability}\ \ghasbadge{Performance}

\medskip
\textbf{Acceptance Criteria (BDD)}
\begin{description}[leftmargin=2.7cm, labelwidth=2.6cm, style=nextline]
  \item[\textbf{Scenario}] Query pack in CI
  \item[\textbf{Given}] a CodeQL database for a compiled-language repo
  \item[\textbf{When}] a custom query identifies a tainted flow to a dangerous sink
  \item[\textbf{Then}] CI fails with a clear alert and remediation guidance
\end{description}

\medskip
\DoR\quad\textbullet\quad\DoD
\end{GHASCard}

\begin{GHASTasks}
\item Install CodeQL CLI; generate a local database for one compiled repo.
\item Write one custom QL query; validate with unit tests and \texttt{codeql test}.
\item Package queries into a query pack; reference it from the CodeQL workflow.
\item Demonstrate SARIF upload from CLI; document process in handbook.
\end{GHASTasks}
\clearpage

% -------------------- Week 4 --------------------
\begin{GHASCard}{GHAS-4}{Secret Scanning \& Push Protection}
\begin{tabularx}{\linewidth}{L V}
Epic / Feature & Secret Scanning \tabularnewline
Business Value & Prevent leaked credentials from entering history; speed incident response. \tabularnewline
Priority / Estimate & Priority: Must \quad SP: 5 \tabularnewline
Persona & Platform engineer / repo maintainer \tabularnewline
Dependencies & GHAS-1 \tabularnewline
Assumptions / Risks & Exclusions required for test data; bypass governance must be defined \tabularnewline
\end{tabularx}

\medskip
\textbf{Story}\quad \emph{As a platform engineer, I want to enable secret scanning with push protection so that high-confidence secrets are blocked before commit.}

\medskip
\textbf{Non-Functional}\quad
\ghasbadge{Security}\ \ghasbadge{Reliability}\ \ghasbadge{Privacy}

\medskip
\textbf{Acceptance Criteria (BDD)}
\begin{description}[leftmargin=2.7cm, labelwidth=2.6cm, style=nextline]
  \item[\textbf{Scenario}] Blocked push
  \item[\textbf{Given}] push protection enabled on 3 repos
  \item[\textbf{When}] a developer attempts to push a simulated token
  \item[\textbf{Then}] the push is blocked; bypass requires justification and is auditable
\end{description}

\medskip
\DoR\quad\textbullet\quad\DoD
\end{GHASCard}

\begin{GHASTasks}
\item Enable secret scanning and push protection on 3 repos.
\item Add \texttt{secret\_scanning.yml} to exclude noisy paths (e.g., test fixtures).
\item Simulate a blocked push with a dummy token; capture the developer UX and audit event.
\item Define delegated bypass roles and documentation.
\end{GHASTasks}
\clearpage

% -------------------- Week 5 --------------------
\begin{GHASCard}{GHAS-5}{Supply Chain: Dependabot, Advisories, PVR}
\begin{tabularx}{\linewidth}{L V}
Epic / Feature & Supply Chain Security \tabularnewline
Business Value & Reduce exposure from vulnerable dependencies; handle inbound reports securely. \tabularnewline
Priority / Estimate & Priority: Should \quad SP: 5 \tabularnewline
Persona & Security engineer / maintainer \tabularnewline
Dependencies & GHAS-1 \tabularnewline
Assumptions / Risks & Update noise; coordination required for coordinated disclosure \tabularnewline
\end{tabularx}

\medskip
\textbf{Story}\quad \emph{As a maintainer, I want Dependabot updates and Private Vulnerability Reporting so that we remediate CVEs quickly and accept reports responsibly.}

\medskip
\textbf{Non-Functional}\quad
\ghasbadge{Security}\ \ghasbadge{Reliability}

\medskip
\textbf{Acceptance Criteria (BDD)}
\begin{description}[leftmargin=2.7cm, labelwidth=2.6cm, style=nextline]
  \item[\textbf{Scenario}] Weekly updates
  \item[\textbf{Given}] Dependabot alerts \& updates enabled on study repos
  \item[\textbf{When}] critical advisories exist
  \item[\textbf{Then}] grouped PRs are raised and merged within SLA; PVR workflow is validated end-to-end
\end{description}

\medskip
\DoR\quad\textbullet\quad\DoD
\end{GHASCard}

\begin{GHASTasks}
\item Configure \texttt{dependabot.yml}: weekly schedule, grouped minor bumps, auto-merge for safe updates.
\item Enable Private Vulnerability Reporting; publish one test advisory and triage to closure.
\item Build a remediation dashboard: open alerts, aging, MTTR.
\end{GHASTasks}
\clearpage

% -------------------- Week 6 --------------------
\begin{GHASCard}{GHAS-6}{Org Reporting \& Workflow Hardening}
\begin{tabularx}{\linewidth}{L V}
Epic / Feature & Security Overview \& Actions Hardening \tabularnewline
Business Value & Drive remediation through metrics; protect CI from supply-chain risks. \tabularnewline
Priority / Estimate & Priority: Should \quad SP: 5 \tabularnewline
Persona & Security program owner / platform engineer \tabularnewline
Dependencies & GHAS-1..5 \tabularnewline
Assumptions / Risks & Fork PRs need safe permissions; action pinning reduces risk but needs maintenance \tabularnewline
\end{tabularx}

\medskip
\textbf{Story}\quad \emph{As a program owner, I want org-level dashboards and hardened workflows so that leaders see progress and CI remains trustworthy.}

\medskip
\textbf{Non-Functional}\quad
\ghasbadge{Security}\ \ghasbadge{Reliability}\ \ghasbadge{Performance}

\medskip
\textbf{Acceptance Criteria (BDD)}
\begin{description}[leftmargin=2.7cm, labelwidth=2.6cm, style=nextline]
  \item[\textbf{Scenario}] Dashboard-driven remediation
  \item[\textbf{Given}] Security overview with feature adoption metrics
  \item[\textbf{When}] teams review weekly
  \item[\textbf{Then}] MTTR for High/Critical $<$ 7 days; adoption $>$ 90\% on target repos
\end{description}

\medskip
\DoR\quad\textbullet\quad\DoD
\end{GHASCard}

\begin{GHASTasks}
\item Build an adoption scorecard: feature enablement \%, alert MTTR, backlog trend.
\item Harden Actions: least-privilege tokens, OIDC to cloud, pin actions by SHA, required checks on protected branches.
\item Create an incident runbook: secret exfiltration, vulnerability disclosure, CodeQL regression.
\end{GHASTasks}
\clearpage

% -------------------- Week 7 --------------------
\begin{GHASCard}{GHAS-7}{Capstone: End-to-End Implementation}
\begin{tabularx}{\linewidth}{L V}
Epic / Feature & Capstone \tabularnewline
Business Value & Prove value on a production-like repo; socialize rollout approach. \tabularnewline
Priority / Estimate & Priority: Must \quad SP: 8 \tabularnewline
Persona & Security engineer / repo owner \tabularnewline
Dependencies & GHAS-1..6 \tabularnewline
Assumptions / Risks & Coordination with repo owners; change management for required checks \tabularnewline
\end{tabularx}

\medskip
\textbf{Story}\quad \emph{As a repo owner, I want an end-to-end GHAS setup so that our main branch stays clean and secure.}

\medskip
\textbf{Non-Functional}\quad
\ghasbadge{Security}\ \ghasbadge{Reliability}\ \ghasbadge{Privacy}

\medskip
\textbf{Acceptance Criteria (BDD)}
\begin{description}[leftmargin=2.7cm, labelwidth=2.6cm, style=nextline]
  \item[\textbf{Scenario}] E2E success
  \item[\textbf{Given}] a target repo
  \item[\textbf{When}] rulesets, CodeQL (with custom pack), secret scanning w/ push protection, Dependabot, and PVR are configured
  \item[\textbf{Then}] PRs are gated; main has zero critical alerts; dashboard reflects improvements
\end{description}

\medskip
\DoR\quad\textbullet\quad\DoD
\end{GHASCard}

\begin{GHASTasks}
\item Apply all security features and rulesets to the capstone repo.
\item Integrate the custom CodeQL query pack; verify failing PR then fix and re-run.
\item Demo results and metrics to stakeholders; capture lessons learned.
\end{GHASTasks}
\clearpage

% -------------------- Week 8 --------------------
\begin{GHASCard}{GHAS-8}{Rollout Plan \& (Optional) Certification}
\begin{tabularx}{\linewidth}{L V}
Epic / Feature & Program Rollout \tabularnewline
Business Value & Scale GHAS across the org; validate skills via certification. \tabularnewline
Priority / Estimate & Priority: Should \quad SP: 3 \tabularnewline
Persona & Program owner \tabularnewline
Dependencies & GHAS-7 \tabularnewline
Assumptions / Risks & Team readiness varies; certification optional \tabularnewline
\end{tabularx}

\medskip
\textbf{Story}\quad \emph{As a program owner, I want a 90-day rollout and training plan so that GHAS adoption is consistent and measurable.}

\medskip
\textbf{Non-Functional}\quad
\ghasbadge{Security}\ \ghasbadge{Reliability}

\medskip
\textbf{Acceptance Criteria (BDD)}
\begin{description}[leftmargin=2.7cm, labelwidth=2.6cm, style=nextline]
  \item[\textbf{Scenario}] Rollout approved
  \item[\textbf{Given}] a pilot completed and metrics available
  \item[\textbf{When}] the 90-day rollout plan is reviewed
  \item[\textbf{Then}] leadership signs off; training \& enablement assets are published
\end{description}

\medskip
\DoR\quad\textbullet\quad\DoD
\end{GHASCard}

\begin{GHASTasks}
\item Create a 90-day rollout plan: scope, milestones, enablement sessions, metrics.
\item Prepare a GHAS playbook: setup steps, ruleset recipes, CodeQL pack usage, secret scanning patterns, PVR guide.
\item (Optional) Schedule the GitHub Advanced Security certification after a passing practice exam.
\end{GHASTasks}

% -----------------------------------------------------------------------------
\section{Appendix: Quick Reference}
\textbf{Story Template}\quad \emph{As a [persona], I want to [goal], so that [business outcome].}

\textbf{Acceptance Criteria}\quad Use Given/When/Then with observable outcomes. Cover happy and negative paths. Include data boundaries and permissions.

\textbf{Definitions}\quad \DoR\ \ \DoD

\end{document}
