%========================================================
% Collaborating with Your Community on GitHub
% Templates, Forms, and Practical Workflows
%========================================================
\documentclass[11pt]{article}

% ---------- Encoding & layout ----------
\usepackage[T1]{fontenc}
\usepackage[utf8]{inputenc}
\usepackage{lmodern}         % Better Latin fonts
\usepackage{geometry}
\geometry{margin=1in}
\usepackage{microtype}
\usepackage{setspace}
\setstretch{1.1}

% ---------- Colors, links, boxes ----------
\usepackage{xcolor}
\definecolor{ink}{HTML}{111827}      % gray-900
\definecolor{soft}{HTML}{F9FAFB}     % gray-50
\definecolor{accent}{HTML}{2563EB}   % blue-600
\definecolor{ok}{HTML}{059669}       % emerald-600
\definecolor{warn}{HTML}{D97706}     % amber-600
\definecolor{bad}{HTML}{DC2626}      % red-600

\usepackage{hyperref}
\hypersetup{colorlinks=true, linkcolor=accent, urlcolor=accent, citecolor=accent}

\usepackage{tcolorbox}
\tcbuselibrary{skins,breakable} % ensure 'enhanced' is available
\tcbset{
  enhanced,
  breakable,
  boxrule=0.5pt,
  colframe=ink,
  colback=soft,
  arc=2pt      % rounded corners (do not combine with 'sharp corners')
}

\usepackage{enumitem}
\setlist[itemize]{topsep=3pt,itemsep=2pt,parsep=0pt}
\setlist[enumerate]{topsep=3pt,itemsep=2pt,parsep=0pt}

% ---------- Code (minted) ----------
% Compile with: pdflatex -shell-escape file.tex  (or xelatex -shell-escape)
\usepackage[cache=false]{minted}
\setminted{
  fontsize=\footnotesize,
  breaklines=true,
  breaksymbolleft={}, breaksymbolright={},
  tabsize=2,
  autogobble=true
}

% ---------- Title ----------
\title{\textbf{Collaborating with Your Community on GitHub}\\
\large Templates, Forms, and Practical Workflows}
\author{}
\date{}

\begin{document}
\maketitle
\vspace{-0.75em}
\begin{tcolorbox}
\textbf{How to use this guide.}
Drop the shown files into your repository (or a central \texttt{.github} repository).
All examples are ASCII-only and ready to copy.
When building this PDF, compile with \texttt{-shell-escape} so the \texttt{minted} code blocks render.
\end{tcolorbox}

\tableofcontents
\clearpage

\section{Purpose}
A lightweight, battle-tested starter to make your repository friendly for contributors:
issue forms and templates, pull request templates, discussion forms, centralization patterns, and common pitfalls.
Everything here assumes the files live on the \emph{default branch} so GitHub can pick them up.

\section{Ready-to-use Repository Layout}
\begin{tcolorbox}
\textbf{Tip.} Directory and some file names are case-sensitive (e.g., \texttt{ISSUE\_TEMPLATE}, \texttt{discussion\_template}).
\end{tcolorbox}

\begin{minted}{text}
.github/
  ISSUE_TEMPLATE/
    bug_report.yml
    feature_request.md
    config.yml
  PULL_REQUEST_TEMPLATE.md
  discussion_template/
    ideas.yml
\end{minted}
\clearpage

\section{Issue \emph{Form} (YAML)}
Place YAML-based forms in \texttt{.github/ISSUE\_TEMPLATE/}. Forms render as a guided UI.
\begin{tcolorbox}
\textbf{Heads-up.} YAML indentation matters. Use spaces (no tabs). Keep quotes simple ASCII.
\end{tcolorbox}

\subsection{Bug Report Form}
\noindent\textbf{File:} \texttt{.github/ISSUE\_TEMPLATE/bug\_report.yml}
\begin{minted}{yaml}
name: "Bug report"
description: "Report a reproducible problem"
title: "bug: <concise summary>"
labels: ["bug"]
assignees: []
body:
  - type: markdown
    attributes:
      value: |
        Thanks for filing a bug! Please fill out all required fields.
  - type: input
    id: version
    attributes:
      label: Affected version
      description: "Choose from releases if possible."
      placeholder: "v1.2.3"
    validations:
      required: true
  - type: textarea
    id: repro
    attributes:
      label: Steps to reproduce
      description: "What did you do? What happened? What did you expect?"
      placeholder: |
        1) ...
        2) ...
        3) ...
    validations:
      required: true
  - type: textarea
    id: logs
    attributes:
      label: Logs / stacktrace
      description: "Paste relevant logs. Use a gist for long traces."
      render: shell
\end{minted}
\clearpage

\section{Issue \emph{Template} (Markdown)}
Markdown templates live alongside forms and appear under the ``New issue'' flow.

\subsection{Feature Request Template (Markdown)}
\noindent\textbf{File:} \texttt{.github/ISSUE\_TEMPLATE/feature\_request.md}
\begin{minted}{markdown}
---
name: "Feature request"
about: "Propose an enhancement"
title: "feat: <short title>"
labels: ["enhancement"]
assignees: []
---

## Problem
<!-- What problem are you trying to solve? -->

## Proposal
<!-- What would you like to see happen? -->

## Alternatives
<!-- What did you consider and why is this better? -->

## Additional context
<!-- Links, screenshots, references -->
\end{minted}

\section{Issue Template Config}
\noindent\textbf{File:} \texttt{.github/ISSUE\_TEMPLATE/config.yml}
\begin{tcolorbox}
\textbf{Use cases.} Disable ``blank issues,'' and add dedicated buttons that route users to Discussions,
a support portal, or private security reporting.
\end{tcolorbox}

\begin{minted}{yaml}
blank_issues_enabled: false
contact_links:
  - name: Ask a question (Discussions)
    url: https://github.com/<owner>/<repo>/discussions
    about: "Use Discussions for Q&A and ideas."
  - name: Security bug bounty
    url: https://security.example.com/bounty
    about: "Report security issues privately."
\end{minted}
\clearpage

\section{Pull Request Template}
You may keep a single template or multiple templates via a folder.

\subsection{Single Template (Markdown)}
\noindent\textbf{File:} \texttt{.github/PULL\_REQUEST\_TEMPLATE.md}
\begin{minted}{markdown}
<!--
Briefly explain the change and link issues (e.g., Closes #123).
The comments in this header will not appear after submission.
-->

## Summary
- ...

## Checklist
- [ ] Tests added/updated
- [ ] Docs updated
- [ ] Linked issue(s)

## Screenshots / Logs
\end{minted}

\subsection{Multiple PR Templates}
Create \texttt{.github/PULL\_REQUEST\_TEMPLATE/} and add several files (for example,
\texttt{bugfix.md}, \texttt{feature.md}). Share links like:
\begin{minted}{text}
https://github.com/<owner>/<repo>/compare/main...my-branch?expand=1&template=feature.md
\end{minted}
\clearpage

\section{Discussion Category Form}
Discussion forms help funnel ideas and questions with required fields.

\noindent\textbf{Folder:} \texttt{.github/discussion\_template/}\quad
\textbf{File:} \texttt{ideas.yml}
\begin{minted}{yaml}
title: "Idea: <concise summary>"
labels: ["discussion", "idea"]
body:
  - type: markdown
    attributes:
      value: |
        Use this form to propose an idea. Required fields help us triage quickly.
  - type: textarea
    id: motivation
    attributes:
      label: Motivation
      description: "What problem would this idea solve?"
    validations:
      required: true
  - type: textarea
    id: details
    attributes:
      label: Details
      description: "Sketch the solution, trade-offs, and impact."
\end{minted}

\section{Centralization \& Overrides}
\begin{itemize}
  \item \textbf{Organization Defaults.} Create an org- or user-level repository named \texttt{.github} and place the same folders/files there to apply defaults across all repositories.
  \item \textbf{Per-Repo Overrides.} A repository overrides the org defaults by adding its own files of the same path/name on its default branch.
\end{itemize}

\section{Troubleshooting \& Gotchas}
\begin{enumerate}
  \item \textbf{Default Branch Only.} Templates/config must live on the repository's default branch.
  \item \textbf{Case Sensitivity.} \texttt{ISSUE\_TEMPLATE} and \texttt{discussion\_template} paths are case-sensitive.
  \item \textbf{YAML Indentation.} Two spaces per level; no tabs. Validate YAML before committing.
  \item \textbf{Non-ASCII Characters.} Avoid emojis or special symbols in YAML or Markdown front-matter to prevent tooling issues.
  \item \textbf{Private Security Reporting.} If private vulnerability reporting is enabled, GitHub surfaces a security disclosure button.\footnote{Availability depends on your GitHub plan/instance.}
  \item \textbf{Compiling this PDF.} Ensure \texttt{pdflatex -shell-escape} (or \texttt{xelatex -shell-escape}) so \texttt{minted} can call Pygments. If you cannot enable shell-escape, replace \texttt{minted} with \texttt{verbatim}.
\end{enumerate}

\section{Quick Start Checklists}
\subsection*{For a New Repository}
\begin{itemize}
  \item Create \texttt{.github/ISSUE\_TEMPLATE/} with one YAML form and one Markdown template.
  \item Add \texttt{config.yml} to disable blank issues and add contact links.
  \item Add \texttt{PULL\_REQUEST\_TEMPLATE.md}.
  \item (Optional) Add \texttt{discussion\_template/} for ideas or Q\&A.
  \item Push to the \emph{default branch}, then verify via the ``New issue'' \& ``New pull request'' flows.
\end{itemize}

\subsection*{Hardening the Triage Flow}
\begin{itemize}
  \item Require required fields in forms (e.g., reproduction steps, version).
  \item Pre-apply labels/assignees to route bugs and features automatically.
  \item Add contact links for Discussions and private security intake.
\end{itemize}
\clearpage

\section{Appendix: Copy-ready Snippets}

\subsection*{Minimal Feature Request (Markdown)}
\begin{minted}{markdown}
---
name: "Feature request"
about: "Propose an enhancement"
title: "feat: <short title>"
labels: ["enhancement"]
---

## Problem
## Proposal
## Alternatives
## Additional context
\end{minted}

\subsection*{Minimal PR Template}
\begin{minted}{markdown}
## Summary
- ...

## Checklist
- [ ] Tests
- [ ] Docs
- [ ] Linked issues
\end{minted}

\subsection*{Minimal Issue Form (YAML)}
\begin{minted}{yaml}
name: "Bug report"
title: "bug: <summary>"
labels: ["bug"]
body:
  - type: input
    id: version
    attributes: { label: Affected version, placeholder: "v1.2.3" }
    validations: { required: true }
  - type: textarea
    id: repro
    attributes: { label: Steps to reproduce }
    validations: { required: true }
\end{minted}

\vfill
\begin{center}\small
\textit{This document is designed to be copied into your workflow repos and adapted to your project's voice.}
\end{center}

\end{document}
