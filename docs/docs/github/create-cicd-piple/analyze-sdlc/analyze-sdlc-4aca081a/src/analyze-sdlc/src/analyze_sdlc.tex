
%=============================================
% Analyze the SDLC — DevOps, Agile, Waterfall
%=============================================
\documentclass[11pt]{article}

% ---------- Encoding & layout ----------
\usepackage[T1]{fontenc}
\usepackage[utf8]{inputenc}
\usepackage{lmodern}
\usepackage{geometry}
\geometry{margin=1in}
\usepackage{microtype}
\usepackage{setspace}
\setstretch{1.1}

% ---------- Colors, links, boxes ----------
\usepackage{xcolor}
\definecolor{ink}{HTML}{111827}      % gray-900
\definecolor{soft}{HTML}{F9FAFB}     % gray-50
\definecolor{accent}{HTML}{2563EB}   % blue-600
\definecolor{ok}{HTML}{059669}       % emerald-600
\definecolor{warn}{HTML}{D97706}     % amber-600
\definecolor{bad}{HTML}{DC2626}      % red-600

\usepackage[colorlinks=true,linkcolor=accent,citecolor=accent,urlcolor=accent]{hyperref}
\usepackage{titlesec}
\titleformat{\section}{\large\bfseries\color{ink}}{\thesection}{0.6em}{}
\titleformat{\subsection}{\normalsize\bfseries\color{ink}}{\thesubsection}{0.6em}{}
\titleformat{\paragraph}[runin]{\bfseries\color{ink}}{}{}{}[.]

\usepackage{enumitem}
\setlist[itemize]{topsep=3pt,itemsep=3pt,parsep=0pt}
\setlist[enumerate]{topsep=3pt,itemsep=3pt,parsep=0pt}

\usepackage{tabularx}
\usepackage{array}
\usepackage{booktabs}
\usepackage{pifont}

\newcommand{\cmark}{\textcolor{ok}{\ding{51}}}
\newcommand{\xmark}{\textcolor{bad}{\ding{55}}}

% ---------- tcolorbox presets ----------
\usepackage[skins,breakable]{tcolorbox}
\tcbset{enhanced, boxrule=0.6pt, colframe=ink, colback=soft, sharp corners, arc=2pt}

\newtcolorbox{keybox}[1][]{breakable, title={#1}, colback=soft, colframe=accent}
\newtcolorbox{note}{breakable, colback=soft}

% ---------- Convenience ----------
\newcommand{\term}[1]{\textbf{#1}}
\newcommand{\gh}{GitHub}

% ---------- Document ----------
\begin{document}

\begin{center}
{\LARGE\bfseries Analyze the SDLC}\\[4pt]
{\large DevOps, Agile vs.\ Waterfall, and Where \gh{} Fits}\\[6pt]
\textit{Concise, practical study guide}
\end{center}
\vspace{0.5em}
\hrule
\vspace{1em}

\section*{Learning Objectives}
\begin{itemize}
  \item Understand what DevOps is (origin and the CALMS model).
  \item Contrast Waterfall and Agile, with trade-offs and use-cases.
  \item Recognize common Agile frameworks (Kanban, Scrum, XP) and practices.
  \item Map the DevOps lifecycle phases and where \gh{} capabilities fit.
\end{itemize}

\section{DevOps at a Glance (CALMS)}
\begin{keybox}[CALMS pillars]
\begin{itemize}
  \item \term{Culture} — cross-functional collaboration, shared responsibility for outcomes.
  \item \term{Automation} — remove toil; make builds, tests, and releases repeatable.
  \item \term{Lean} — eliminate waste, optimize flow of value, focus on small batch sizes.
  \item \term{Measurement} — observe with meaningful metrics; improve constraints.
  \item \term{Sharing} — open knowledge flow across teams; blameless learning.
\end{itemize}
\end{keybox}

\paragraph{Typical supporting metrics (examples)} Deployment Frequency, Lead Time for Changes, Change Failure Rate, and Mean Time to Restore. These help validate improvements across CALMS dimensions.

\section{Waterfall SDLC (Classic Model)}
\subsection{Phases}
\begin{enumerate}
  \item \term{Requirements} — clarify scope, risks, constraints, acceptance criteria.
  \item \term{Design} — architecture, interfaces, data models, UX flows.
  \item \term{Implementation} — coding and integration of components.
  \item \term{Verification} — testing against requirements/specification.
  \item \term{Deployment \& Maintenance} — release, operate, and support.
\end{enumerate}

\subsection{Strengths \& Limitations}
\begin{tabularx}{\linewidth}{@{}>{\raggedright\arraybackslash}p{0.48\linewidth} >{\raggedright\arraybackslash}X@{}}
\toprule
\textbf{Strengths} & \textbf{Limitations} \\
\midrule
Clear stage gates and artifacts; easier regulatory traceability. & Struggles with frequent change; feedback comes late. \\
Predictable budgets and timelines for stable scope. & Risk of building the ``wrong'' thing before validation. \\
Well-suited for low-uncertainty domains. & Long cycles discourage continuous learning and adaptation. \\
\bottomrule
\end{tabularx}

\section{Agile Fundamentals}
\subsection{Values (Manifesto, summarized)}
\begin{itemize}
  \item \term{Individuals and interactions} over processes and tools.
  \item \term{Working software} over comprehensive documentation.
  \item \term{Customer collaboration} over contract negotiation.
  \item \term{Responding to change} over following a plan.
\end{itemize}

\subsection{Common Frameworks and Practices}
\begin{itemize}
  \item \term{Kanban} — visualize work, limit work-in-progress (WIP), manage flow.
  \item \term{Scrum} — timeboxed sprints, roles (PO/SM/Team), ceremonies, backlog.
  \item \term{Extreme Programming (XP)} — TDD, pair programming, continuous integration, refactoring.
\end{itemize}

\subsection{Agile vs.\ Waterfall (Key Contrasts)}
\begin{tabularx}{\linewidth}{@{}>{\raggedright\arraybackslash}p{0.42\linewidth} >{\raggedright\arraybackslash}X@{}}
\toprule
\textbf{Agile Principle} & \textbf{Why It Matters} \\
\midrule
Individuals \& interactions & Optimize collaboration and flow of value. \\
Working software & Ship usable value sooner for earlier feedback. \\
Customer collaboration & Continuous feedback tightens product--market fit. \\
Responding to change & Embrace inevitable change without heavy ceremony. \\
\bottomrule
\end{tabularx}
\clearpage

\section{DevOps Lifecycle (Where \gh{} Fits)}
\subsection{End-to-End View}
\begin{tabularx}{\linewidth}{@{}>{\raggedright\arraybackslash}p{0.33\linewidth} >{\raggedright\arraybackslash}X@{}}
\toprule
\textbf{Lifecycle Phase} & \textbf{Typical \gh{} Capabilities} \\
\midrule
\term{Continuous Planning} & Issues, labels, milestones; \textbf{Projects} for boards; Discussions; Wikis; organization-level planning across repositories. \\
\term{Code \& Review} & Pull Requests, code review workflows, branch protections, required reviews, status checks, \textbf{Codespaces} for dev environments, \textbf{Copilot} for assistance. \\
\term{Build \& Test (CI)} & \textbf{Actions} workflows for build/test, matrix builds, caching; packages, reusable workflows; marketplace actions. \\
\term{Security \& Quality} & Secret scanning, dependency scanning/updates, code scanning (SAST), branch protections, required checks. \\
\term{Continuous Delivery/Deployment} & Actions for deployment to environments; environment protection rules; Pages for static sites; registries via \textbf{Packages}. \\
\term{Operate \& Learn} & Release notes; Discussions; issues for incidents/postmortems; CI signals surfaced in PRs; artifacts and logs for traceability. \\
\bottomrule
\end{tabularx}

\subsection{Minimal Practical Checklist}
\begin{itemize}
  \item \cmark{} Track work with Issues, milestones, and a \gh{} Projects board.
  \item \cmark{} Enforce branch protections and required status checks.
  \item \cmark{} Build and test with Actions on pull requests and default branch.
  \item \cmark{} Enable secret scanning, code scanning, and dependency updates.
  \item \cmark{} Use environments and required approvals for production deploys.
  \item \cmark{} Publish release notes and capture post-incident learnings.
\end{itemize}

\section*{Glossary (Quick Reference)}
\begin{description}[leftmargin=1.5em]
  \item[CALMS] Culture, Automation, Lean, Measurement, Sharing.
  \item[CI/CD] Continuous Integration / Continuous Delivery (or Deployment).
  \item[WIP] Work in Progress, often limited in Kanban to improve flow.
  \item[Sprint] Timeboxed iteration in Scrum (e.g., 1--2 weeks).
\end{description}

\begin{note}
\textbf{Tip.} Start small: choose one value stream, instrument its flow (lead time, deployment frequency, change failure rate, and time to restore), and iterate on constraints. Use \gh{} branch protections and Actions to create crisp, automated feedback early.
\end{note}

\vfill
\noindent\rule{\linewidth}{0.4pt}\\
{\footnotesize This handout was generated from concise study notes on SDLC, Agile vs.\ Waterfall, and DevOps lifecycle mapping to \gh{}.}

\end{document}
