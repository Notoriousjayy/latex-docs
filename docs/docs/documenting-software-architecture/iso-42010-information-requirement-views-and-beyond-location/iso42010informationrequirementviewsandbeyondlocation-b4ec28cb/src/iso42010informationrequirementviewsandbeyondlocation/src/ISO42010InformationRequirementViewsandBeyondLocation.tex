\documentclass[11pt]{article}

\usepackage[margin=1in]{geometry}
\usepackage{tabularx}
\usepackage{array}
\usepackage{booktabs}

\setlength{\parindent}{0pt}

\begin{document}

\title{ISO 42010 Information Requirements and Views \& Beyond}
\author{}
\date{}
\maketitle

\section*{Mapping ISO 42010 to Views and Beyond}

\begin{tabularx}{\textwidth}{@{}%
  >{\raggedright\arraybackslash}p{0.47\textwidth}%
  >{\raggedright\arraybackslash}p{0.47\textwidth}@{}}
\toprule
\textbf{ISO 42010 Information Requirement} &
\textbf{Views and Beyond Location} \\
\midrule

\textit{Identification and overview information}, as appropriate to
stakeholder, project, and organization needs. For example: summary, context,
glossary, references, and change history. &
Several items in this category amount to good bookkeeping. Context is
addressed in the context diagrams; the other items are prescribed in the
standard organizations of Chapter~10. \\
\addlinespace[0.6em]

\textit{Stakeholders and concerns}. Identify architecturally relevant
stakeholders. At a minimum consider customers, users, operators, acquirers,
suppliers, developers, and maintainers. Identify their architecture-related
concerns. At a minimum consider system purposes, suitability of architecture to
meet purposes, feasibility of construction, potential risks throughout life
cycle, maintainability, deployability, and evolvability. &
The documentation roadmap called for in Section~10.2 captures information
about stakeholders and their concerns---specifically, how they will use the
documentation package. For ISO~42010 compliance, make sure the stakeholders
and concerns include those named in the left-hand column. \\
\addlinespace[0.6em]

\textit{Viewpoints}. For each viewpoint, the following must be specified:
\begin{itemize}
  \item The viewpoint name
  \item The subset of identified architecture-related concerns (from above)
        framed by this viewpoint
  \item The identification of each type of architecture model used by this
        viewpoint
  \item For each type of model: the languages, notations, rules, constraints,
        modeling techniques, analytical methods, or operations to be used in
        creating and interpreting the view
  \item Rationale for selection of the viewpoint
  \item Any additional information, such as completeness and correctness
        checks, evaluation criteria, heuristics, or guidelines
\end{itemize}
&
We define several commonly used module, C\&C, and allocation styles. Each
style guide defines the concepts---elements, relations, and
properties---that should be used in documenting a system in accordance with
the style. It contains information about useful notations and modeling
techniques for that style. Each style guide also contains a section noting
what it is for, which should help users in deciding what concerns will be
addressed by the style.

All of this information in a style guide constitutes an implicit viewpoint
definition, but the standard requires including an explicit set in your
document, either directly or by reference. You can easily accommodate this
requirement by adding a section for viewpoint definitions to the
``documentation beyond views'' template in Section~10.2. There, you can
reproduce or refer to the specific style guide information as needed. \\
\addlinespace[0.6em]

\textit{Views}. Each view must include:
\begin{itemize}
  \item A view identifier
  \item Overview and configuration information as required by project or
        organization
  \item One or more architecture models covering the whole system from the
        viewpoint
\end{itemize}
A record of all inconsistencies among views, preferably accompanied by an
analysis of consistency among all views.

Rationale for the key architectural decisions made, preferably accompanied by
evidence of alternatives considered and rationale for the choices made. &
Chapter~10 discusses the information that should be documented for a view.

In Chapter~6, we discuss techniques for documenting relations among views,
which is then recorded in the ``documentation beyond views'' part of the
package, as detailed in Chapter~10.

Reserved spots for rationale are provided in each view, in the documentation
beyond views, and in interface documentation. \\
\bottomrule
\end{tabularx}

\end{document}

