%===============================================================================
% Cloud Governance & Control Framework Architecture
% Views & Beyond Documentation Package (Template + Starter Content)
%===============================================================================
\documentclass[11pt]{report}

%--------------------------- Packages -----------------------------------------
\usepackage[margin=1in]{geometry}
\usepackage[T1]{fontenc}
\usepackage{lmodern}
\usepackage{microtype}
\usepackage{graphicx}
\usepackage{booktabs}
\usepackage{tabularx}
\usepackage{longtable}
\usepackage{array}
\usepackage{enumitem}
\usepackage{xcolor}
\usepackage{hyperref}
\usepackage{titlesec}
\usepackage{fancyhdr}
\usepackage{minted}  % requires -shell-escape
\usepackage{tcolorbox}

\hypersetup{
  colorlinks=true,
  linkcolor=blue,
  urlcolor=blue,
  citecolor=blue
}

%--------------------------- Header/Footer ------------------------------------
\pagestyle{fancy}
\fancyhf{}
\fancyhead[L]{Cloud Governance \& Control Framework}
\fancyhead[R]{Architecture Documentation Package}
\fancyfoot[C]{\thepage}

%--------------------------- Simple Styles ------------------------------------
\setlist[itemize]{noitemsep, topsep=4pt}
\setlist[enumerate]{noitemsep, topsep=4pt}

\newcommand{\DocTitle}{Cloud Governance \& Control Framework Architecture}
\newcommand{\DocSubtitle}{Views \& Beyond Documentation Package}
\newcommand{\DocVersion}{v0.1}
\newcommand{\DocStatus}{Draft}
\newcommand{\DocOwner}{Cloud Platform \& Governance Team}
\newcommand{\DocDate}{\today}

% A lightweight “view packet” box
\newtcolorbox{viewbox}{
  colback=gray!3,
  colframe=gray!50,
  arc=2mm,
  boxrule=0.5pt,
  left=3mm,
  right=3mm,
  top=2mm,
  bottom=2mm
}

%===============================================================================
\begin{document}

%===============================================================================
% Title Page
%===============================================================================
\begin{titlepage}
  \centering
  \vspace*{1.5cm}
  {\LARGE \textbf{\DocTitle}\par}
  \vspace{0.5cm}
  {\Large \DocSubtitle\par}
  \vspace{1.0cm}
  {\large Version: \DocVersion \quad \textbar \quad Status: \DocStatus\par}
  \vspace{0.5cm}
  {\large Owner: \DocOwner\par}
  \vspace{0.5cm}
  {\large Date: \DocDate\par}
  \vfill
  \begin{minipage}{0.9\textwidth}
    \small
    \textbf{Purpose:} Provide a stakeholder-ready architecture description for the Cloud Governance \&
    Control Framework, organized using Views \& Beyond. This package includes the Context, Logical,
    Process, and Deployment views, plus “Beyond Views” information such as control catalog, evidence
    model, rationale, risks, and documentation roadmap.
  \end{minipage}
  \vfill
\end{titlepage}

\tableofcontents
\clearpage

%===============================================================================
\chapter*{Document Control}
\addcontentsline{toc}{chapter}{Document Control}

\section*{Revision History}
\addcontentsline{toc}{section}{Revision History}

\begin{tabularx}{\textwidth}{@{}l l l X@{}}
\toprule
\textbf{Version} & \textbf{Date} & \textbf{Author} & \textbf{Change Summary} \\
\midrule
0.1 & \DocDate & \DocOwner & Initial Views \& Beyond package template and starter content. \\
\bottomrule
\end{tabularx}

\section*{Distribution}
\addcontentsline{toc}{section}{Distribution}
\begin{itemize}
  \item Cloud Platform Engineering
  \item Security / IAM
  \item FinOps
  \item Compliance / Audit
  \item Application/Product Teams
\end{itemize}

\clearpage

%===============================================================================
\chapter{Architecture Overview}
\section{System Scope \& Boundary}
\begin{viewbox}
\textbf{In scope:} Policy definition, guardrails/control plane, automated enforcement, monitoring,
evidence collection, reporting, and optimization loops for cloud usage across one or more providers.\\
\textbf{Out of scope:} Application business logic and domain-specific controls not related to cloud
platform governance (unless explicitly integrated as control signals/evidence).
\end{viewbox}

\section{Business Goals \& Outcomes}
\begin{itemize}
  \item Enable secure, compliant, and cost-effective cloud adoption without constraining delivery velocity.
  \item Provide standardized guardrails (preventive controls) and continuous monitoring (detective controls).
  \item Automate audit readiness through evidence-as-a-product (continuous control evidence).
  \item Establish a continuous improvement loop for policies, controls, and platform capabilities.
\end{itemize}

\section{Stakeholders \& Concerns (Driver Matrix)}
\subsection{Stakeholder Catalog}
\begin{tabularx}{\textwidth}{@{}l l X@{}}
\toprule
\textbf{Stakeholder} & \textbf{Role} & \textbf{Primary Concerns} \\
\midrule
Cloud Governance Team / CCoE & Policy \& guardrails & Consistency, exceptions, adoption, org-wide standards \\
Security / IAM & Security baseline & Least privilege, identity assurance, monitoring, incident response \\
FinOps & Cost governance & Allocation, budgets, anomaly detection, optimization \\
Platform Engineering & Enablement & Landing zones, IaC modules, drift prevention, operability \\
Compliance / Audit & Assurance & Evidence quality, control mapping, reporting cadence \\
Product Teams & Consumers & Self-service, speed, clarity of boundaries, minimal friction \\
\bottomrule
\end{tabularx}

\subsection{Stakeholder--View Mapping}
\begin{tabularx}{\textwidth}{@{}l X X@{}}
\toprule
\textbf{Stakeholder} & \textbf{Views Needed} & \textbf{Decisions Enabled} \\
\midrule
Security / IAM & Context, Logical, Process, Beyond (Controls/Evidence) & Baselines, access patterns, detections \\
FinOps & Context, Process, Beyond (Cost Controls/Metrics) & Budgeting, chargeback, optimization priorities \\
Audit & Beyond (Control Catalog/Evidence), Deployment & Audit readiness, evidence sufficiency \\
Platform Eng & Logical, Deployment, Process & Reference implementations, operations model \\
Product Teams & Context, Process (key scenarios) & How to build within guardrails; escalation paths \\
\bottomrule
\end{tabularx}

\section{Architectural Drivers}
\subsection{Key Quality Attributes}
\begin{itemize}
  \item \textbf{Security:} enforce least privilege and secure-by-default configurations.
  \item \textbf{Compliance:} continuous compliance posture and auditable evidence trails.
  \item \textbf{Agility:} self-service enablement with minimal governance friction.
  \item \textbf{Scalability:} handle org-wide resource growth and multi-account/subscription expansion.
  \item \textbf{Reliability/Operability:} monitoring, alerting, runbooks, and clear ownership.
\end{itemize}

\subsection{Constraints \& Assumptions}
\begin{itemize}
  \item Governance must be \textbf{automation-first} (policy-as-code, IaC, continuous monitoring).
  \item Multi-cloud support may be required (or planned); provider-specific controls must map to common abstractions.
  \item All controls must have an \textbf{evidence model} (what is collected, where stored, retention, and access).
\end{itemize}

\section{Document Roadmap (How to Use This Package)}
\begin{itemize}
  \item Start with \textbf{Context View} to understand boundary, actors, and external dependencies.
  \item Use \textbf{Logical View} to understand the core governance subsystems and responsibilities.
  \item Use \textbf{Process View} to see runtime behavior and key governance scenarios.
  \item Use \textbf{Deployment View} to understand how the system maps to cloud/provider/tooling infrastructure.
  \item Use \textbf{Beyond Views} for controls, evidence, rationale, risks, and mappings.
\end{itemize}

\clearpage

%===============================================================================
\chapter{Context View}
\section{View Packet Summary}
\begin{viewbox}
\textbf{View type:} Context (System Boundary)\\
\textbf{Primary stakeholders:} Product teams, Security/IAM, FinOps, Compliance/Audit, Platform Engineering\\
\textbf{Primary concerns:} Who interacts with governance, what is inside vs. outside, trust boundaries,
major integrations, and data/evidence flows.
\end{viewbox}

\section{Context Diagram (Primary Presentation)}
\textbf{Insert your context diagram here.} Recommended: C4 Context-style or a clean boundary diagram.

\begin{figure}[htbp]
  \centering
  % Replace with your exported diagram image:
  \fbox{\parbox{0.9\textwidth}{\centering Placeholder: Context Diagram (PNG/SVG)}}
  \caption{Context View: System boundary and external actors/systems.}
\end{figure}

\section{External Actors \& Systems}
\begin{tabularx}{\textwidth}{@{}l l X@{}}
\toprule
\textbf{Actor/System} & \textbf{Type} & \textbf{Interaction} \\
\midrule
Product Teams & Human/org & Consume self-service catalog, deploy via IaC pipelines within guardrails \\
Security/IAM & Org & Defines security baseline; consumes alerts, posture reports \\
FinOps & Org & Defines budgets/allocations; consumes cost anomaly signals and KPI dashboards \\
Audit/Compliance & Org & Requests/consumes evidence, control mappings, audit reports \\
Cloud Provider(s) & External system & Policy enforcement primitives, telemetry, resource APIs \\
CI/CD Platform & External system & Policy gates, drift checks, release approvals, evidence emission \\
SIEM/Logging & External system & Centralized security telemetry ingestion and alerting \\
CMDB/Asset Inventory & External system & Asset ownership, service mapping, lifecycle tracking \\
\bottomrule
\end{tabularx}

\section{Trust Boundaries \& Data Flows}
\begin{itemize}
  \item Governance plane must be logically separated from workload planes (accounts/subscriptions/projects).
  \item Evidence and logs are sensitive assets; access must be least-privilege and audited.
  \item Key flows:
    \begin{itemize}
      \item \textbf{Policy intent} $\rightarrow$ \textbf{Policy-as-code} $\rightarrow$ \textbf{Enforcement}
      \item \textbf{Telemetry} $\rightarrow$ \textbf{Detection} $\rightarrow$ \textbf{Tickets/Notifications}
      \item \textbf{Control results} $\rightarrow$ \textbf{Evidence store} $\rightarrow$ \textbf{Audit reporting}
    \end{itemize}
\end{itemize}

\section{Context View: Rationale \& Notes}
\begin{itemize}
  \item This view is intentionally stable; details belong in Logical/Deployment views.
  \item Provider-specific services are abstracted as “Cloud Provider APIs/Telemetry” to support multi-cloud evolution.
\end{itemize}

\clearpage

%===============================================================================
\chapter{Logical View}
\section{View Packet Summary}
\begin{viewbox}
\textbf{View type:} Logical / Module decomposition\\
\textbf{Primary stakeholders:} Platform Engineering, Security/IAM, FinOps, Architecture\\
\textbf{Primary concerns:} Responsibilities, module boundaries, interfaces, data stores, and integration points.
\end{viewbox}

\section{Logical Decomposition (Primary Presentation)}
\textbf{Insert your logical/module diagram here.} Recommended: “Policy/Control Plane/Execution/Evidence”
layered module diagram.

\begin{figure}[htbp]
  \centering
  \fbox{\parbox{0.9\textwidth}{\centering Placeholder: Logical Decomposition Diagram (PNG/SVG)}}
  \caption{Logical View: Core modules and their primary responsibilities.}
\end{figure}

\section{Module Catalog (Element Catalog)}
\begin{longtable}{@{}p{0.22\textwidth} p{0.18\textwidth} p{0.54\textwidth}@{}}
\toprule
\textbf{Module} & \textbf{Type} & \textbf{Responsibilities} \\
\midrule
Policy Library & Data/Knowledge & Policies, standards, control objectives, exception rules, versioning \\
Policy-as-Code Engine & Service & Encodes policies into enforceable rules; integrates with provider/IaC/CI gates \\
Landing Zone Management & Service & Org/account/subscription baseline, network baseline, identity baseline hooks \\
Self-Service Catalog & Service & Approved patterns, IaC modules, golden paths, request workflows \\
Continuous Monitoring & Service & Posture checks, drift detection, vuln signals, compliance checks \\
Cost Governance & Service & Budgets, allocation rules, anomaly detection, optimization recommendations \\
Evidence Store & Data/Service & Control results, logs, attestations, snapshots, retention, access control \\
Reporting \& Dashboards & Service & Security posture, compliance status, FinOps KPIs, operational SLOs \\
Exception/Waiver Workflow & Process/App & Approval, expiry, compensating controls, audit traceability \\
\bottomrule
\end{longtable}

\section{Interfaces \& Contracts}
\subsection{Key Interfaces}
\begin{itemize}
  \item \textbf{Policy API:} publish/version policy sets; retrieve effective policies by scope (org/account/project).
  \item \textbf{Control Results API:} standard schema for checks (pass/fail, severity, resource identifiers, metadata).
  \item \textbf{Evidence API:} query evidence by control, timeframe, resource, or audit request.
  \item \textbf{Notification API:} route signals to SIEM/ticketing/on-call channels with enrichment.
\end{itemize}

\subsection{Canonical Data Objects (Starter)}
\begin{itemize}
  \item \texttt{Policy}: id, scope, statement, parameters, owner, version, effective\_date
  \item \texttt{Control}: id, objective, test\_procedure, automation\_level, frequency, evidence\_type
  \item \texttt{ControlResult}: control\_id, resource\_id, status, severity, timestamp, evidence\_ref
  \item \texttt{Exception}: policy/control reference, justification, approver, expiry, compensating\_controls
\end{itemize}

\section{Logical View: Variability}
\begin{itemize}
  \item \textbf{Single-cloud vs. multi-cloud:} provider adapters behind common policy/control abstractions.
  \item \textbf{Tooling choices:} multiple posture tools may exist; standardize output into ControlResult schema.
  \item \textbf{Org maturity:} begin with preventive guardrails + minimal evidence; evolve to continuous controls.
\end{itemize}

\section{Logical View: Rationale}
\begin{itemize}
  \item Separating \textbf{policy intent} from \textbf{enforcement} enables policy iteration without re-architecting execution.
  \item Explicit \textbf{evidence store} turns audit readiness into a product, not an ad hoc activity.
\end{itemize}

\clearpage

%===============================================================================
\chapter{Process View}
\section{View Packet Summary}
\begin{viewbox}
\textbf{View type:} Process / Runtime behavior\\
\textbf{Primary stakeholders:} Platform Engineering, Security Operations, FinOps, Product Teams\\
\textbf{Primary concerns:} Control lifecycle, event flows, concurrency, escalation paths, and operational procedures.
\end{viewbox}

\section{Key Runtime Scenarios (Primary Presentation)}
\subsection{Scenario A: Provisioning a New Workload (Golden Path)}
\begin{enumerate}
  \item Product team requests/initiates a workload via self-service catalog.
  \item IaC pipeline applies approved modules; preventive policies validate configuration.
  \item Landing zone baselines attach monitoring/logging and identity constraints.
  \item Evidence artifacts emitted (deployment attestation, policy evaluation results).
\end{enumerate}

\subsection{Scenario B: Policy Violation Detection \& Remediation}
\begin{enumerate}
  \item Continuous monitoring detects drift/misconfiguration (control check fails).
  \item Alert enriched with ownership, severity, and remediation guidance.
  \item Ticket/notification created; remediation runbook executed (manual or automated).
  \item Evidence updated with resolution and timestamps; metrics captured for trend analysis.
\end{enumerate}

\subsection{Scenario C: Cost Anomaly \& Optimization Loop}
\begin{enumerate}
  \item Cost governance detects anomaly (budget threshold, spike, or idle resources).
  \item FinOps triage assigns action: rightsizing, shutdown, reservation/commitment planning.
  \item Changes implemented via IaC; evidence and KPI dashboards updated.
\end{enumerate}

\section{Process Diagrams}
\textbf{Insert sequence/activity diagrams here} (e.g., BPMN-like flow for “Define $\rightarrow$ Implement $\rightarrow$ Monitor $\rightarrow$ Optimize”).

\begin{figure}[htbp]
  \centering
  \fbox{\parbox{0.9\textwidth}{\centering Placeholder: Governance Lifecycle Flow Diagram (PNG/SVG)}}
  \caption{Process View: Continuous governance lifecycle and operational workflows.}
\end{figure}

\section{Process View: Concurrency \& Scaling Considerations}
\begin{itemize}
  \item Controls execute continuously across many accounts/projects; prioritize by risk (severity tiers).
  \item Evidence ingestion must be idempotent and support high-frequency signals (posture checks, log events).
  \item Notifications require rate limiting and deduplication to avoid alert fatigue.
\end{itemize}

\section{Process View: Operational Policies}
\begin{itemize}
  \item \textbf{SLOs:} time-to-detect, time-to-remediate for high severity violations.
  \item \textbf{Change management:} policy changes are versioned and rolled out with staged enforcement.
  \item \textbf{Exception handling:} every exception has an owner, expiry, and compensating controls.
\end{itemize}

\clearpage

%===============================================================================
\chapter{Deployment View}
\section{View Packet Summary}
\begin{viewbox}
\textbf{View type:} Deployment / Allocation\\
\textbf{Primary stakeholders:} Platform Engineering, Security, Operations\\
\textbf{Primary concerns:} Where modules run, network boundaries, identity boundaries, integrations,
resilience, and operational ownership.
\end{viewbox}

\section{Deployment Diagram (Primary Presentation)}
\textbf{Insert your deployment diagram here.} Recommended: governance plane services + provider environments +
tooling integrations (CI/CD, SIEM, CMDB, ticketing).

\begin{figure}[htbp]
  \centering
  \fbox{\parbox{0.9\textwidth}{\centering Placeholder: Deployment Diagram (PNG/SVG)}}
  \caption{Deployment View: Mapping of logical modules to runtime infrastructure and integrations.}
\end{figure}

\section{Node/Environment Catalog}
\begin{tabularx}{\textwidth}{@{}l l X@{}}
\toprule
\textbf{Node/Env} & \textbf{Type} & \textbf{Notes} \\
\midrule
Governance Control Plane & Runtime env & Hosts policy-as-code engine, orchestration, evidence ingestion \\
Cloud Provider Workload Planes & Runtime env & Accounts/subscriptions/projects with guardrails applied \\
CI/CD Platform & External system & Runs IaC pipelines, policy gates, attestations \\
Logging/SIEM & External system & Central collection, detection rules, alert routing \\
Evidence Store & Data platform & Immutable storage, retention, access controls, audit queries \\
Dashboards/Reporting & Service & Role-based access; exports for audit packages \\
\bottomrule
\end{tabularx}

\section{Deployment View: Availability \& Resilience}
\begin{itemize}
  \item Evidence store is a critical asset; enforce backups, retention controls, and least-privilege access.
  \item Control plane services should be horizontally scalable and support retry/backoff for provider APIs.
  \item Separate dev/test/prod governance environments to validate policy changes before enforcement.
\end{itemize}

\section{Deployment View: Security Considerations}
\begin{itemize}
  \item Strong identity boundary for governance administrators (PIM/PAM, MFA, approvals).
  \item Encrypt evidence/log data at rest and in transit; audit all access to evidence.
  \item Ensure separation-of-duties for exception approval vs. implementation.
\end{itemize}

\clearpage

%===============================================================================
\chapter{Beyond Views}
\section{Beyond Views Overview}
\begin{viewbox}
\textbf{Purpose:} Provide the “glue” information that makes the views usable: rationale, mappings,
control catalog, evidence model, risks, glossary, and documentation practices.
\end{viewbox}

%--------------------------- Controls -----------------------------------------
\section{Control Catalog (Template + Starter)}
\subsection{Control Model}
Each control should have:
\begin{itemize}
  \item \textbf{Control ID} (stable identifier)
  \item \textbf{Control Objective} (what it ensures)
  \item \textbf{Control Type} (preventive / detective / corrective)
  \item \textbf{Implementation Mechanism} (policy-as-code, IaC gate, monitoring check, process)
  \item \textbf{Frequency} (continuous, daily, per-deploy, quarterly)
  \item \textbf{Evidence Artifact(s)} (what proves it)
  \item \textbf{Owner} and \textbf{Escalation Path}
\end{itemize}

\subsection{Starter Control Catalog}
\begin{longtable}{@{}p{0.12\textwidth} p{0.22\textwidth} p{0.12\textwidth} p{0.20\textwidth} p{0.26\textwidth}@{}}
\toprule
\textbf{ID} & \textbf{Objective} & \textbf{Type} & \textbf{Mechanism} & \textbf{Evidence} \\
\midrule
IAM-01 & Enforce least privilege via RBAC & Preventive & Policy baseline + role templates & Role assignment snapshots; approval logs \\
NET-01 & Restrict public exposure by default & Preventive & Policy-as-code + IaC module & Policy eval results; deployment attestations \\
LOG-01 & Centralize logs for all workloads & Preventive/Detective & Landing zone baseline & Log ingestion status; coverage report \\
CFG-01 & Detect config drift from baselines & Detective & Continuous posture checks & ControlResult stream; drift reports \\
COST-01 & Budget thresholds per cost center & Detective & Budget alerts/anomaly detection & Budget config; alert records; KPI trends \\
EXC-01 & Time-bound exceptions with approvals & Corrective & Exception workflow & Exception record; expiry; compensating controls \\
\bottomrule
\end{longtable}

%--------------------------- Evidence -----------------------------------------
\section{Evidence Model (Evidence-as-a-Product)}
\subsection{Evidence Types}
\begin{itemize}
  \item \textbf{Automated evaluations:} policy results, posture scans, CI/CD gate outputs
  \item \textbf{Telemetry-derived:} logs, alerts, detections, incident records
  \item \textbf{Attestations:} approvals, exception waivers, quarterly access reviews
  \item \textbf{State snapshots:} inventory exports, configuration baselines, role assignment snapshots
\end{itemize}

\subsection{Evidence Mapping Template}
\begin{tabularx}{\textwidth}{@{}l l l X@{}}
\toprule
\textbf{Control ID} & \textbf{Evidence Artifact} & \textbf{Retention} & \textbf{Access/Owner} \\
\midrule
IAM-01 & RBAC assignment export (daily) & 1--7 years & Security/IAM (read-only for Audit) \\
LOG-01 & Log coverage report (weekly) & 1 year & Platform Ops \\
CFG-01 & Drift findings (continuous) & 1 year & Platform Ops + Security \\
\bottomrule
\end{tabularx}

%--------------------------- Rationale & ADRs ---------------------------------
\section{Rationale \& Key Design Decisions}
\subsection{Architecture Decisions (ADR Index)}
Maintain an ADR log for materially significant governance decisions:
\begin{itemize}
  \item ADR-001: Policy-as-code approach and enforcement tiers (warn/block)
  \item ADR-002: Evidence store technology choice and retention strategy
  \item ADR-003: Exception workflow requirements (expiry, compensating controls)
  \item ADR-004: Canonical schema for ControlResult and evidence referencing
\end{itemize}

\subsection{Design Principles}
\begin{itemize}
  \item \textbf{Automation-first:} policies and controls must be machine-enforceable where feasible.
  \item \textbf{Least privilege by default:} identity is the primary control plane.
  \item \textbf{Evidence is a product:} audit readiness is continuous, not event-driven.
  \item \textbf{Golden paths:} enablement reduces risk more effectively than “after-the-fact policing.”
\end{itemize}

%--------------------------- Mappings -----------------------------------------
\section{Mappings}
\subsection{Mapping Between Views}
\begin{tabularx}{\textwidth}{@{}l X@{}}
\toprule
\textbf{Mapping} & \textbf{Purpose} \\
\midrule
Context $\leftrightarrow$ Logical & Trace external actors to the modules they interact with \\
Logical $\leftrightarrow$ Process & Trace runtime scenarios to responsible modules \\
Logical $\leftrightarrow$ Deployment & Trace modules to runtime nodes/environments and trust boundaries \\
Controls $\leftrightarrow$ Evidence & Prove each control has objective, mechanism, and auditable artifacts \\
\bottomrule
\end{tabularx}

\subsection{Standards/Framework Mapping (Template)}
\begin{tabularx}{\textwidth}{@{}l l X@{}}
\toprule
\textbf{Control ID} & \textbf{Framework Ref} & \textbf{Notes} \\
\midrule
IAM-01 & NIST / ISO / SOC2 (fill) & Map objective to requirement language; reference internal policy \\
LOG-01 & NIST / ISO / SOC2 (fill) & Evidence demonstrates continuous logging coverage \\
\bottomrule
\end{tabularx}

%--------------------------- Risk & Exceptions --------------------------------
\section{Risks, Issues, \& Exceptions}
\subsection{Risk Register (Template)}
\begin{tabularx}{\textwidth}{@{}l l X X@{}}
\toprule
\textbf{ID} & \textbf{Severity} & \textbf{Risk} & \textbf{Mitigation} \\
\midrule
R-01 & High & Policy enforcement too strict blocks delivery & Start with warn/tiered enforcement; golden paths; exception workflow \\
R-02 & High & Evidence store access becomes a data exposure risk & Least privilege; auditing; encryption; segregation of duties \\
R-03 & Medium & Tool sprawl produces inconsistent control results & Canonical ControlResult schema; normalization layer \\
\bottomrule
\end{tabularx}

\subsection{Exception Policy (Template)}
\begin{itemize}
  \item All exceptions require: justification, owner, approver, expiry date, compensating controls.
  \item Exceptions are reviewed periodically; expired exceptions auto-escalate.
\end{itemize}

%--------------------------- Glossary -----------------------------------------
\section{Glossary}
\begin{tabularx}{\textwidth}{@{}l X@{}}
\toprule
\textbf{Term} & \textbf{Definition} \\
\midrule
Guardrail & Preventive constraint (policy baseline) that blocks or restricts unsafe configurations \\
Control & A mechanism that enforces or verifies a policy objective (prevent/detect/correct) \\
Evidence & Artifact demonstrating control operation (logs, snapshots, attestations, control results) \\
Landing Zone & Standardized cloud environment baseline (identity, network, logging, policies) \\
Policy-as-Code & Declarative encoding of policy intent into machine-evaluable rules \\
Golden Path & Approved deployment pattern that is secure and compliant by default \\
\bottomrule
\end{tabularx}

%--------------------------- Packaging/Publishing -----------------------------
\section{Documentation Packaging, Release, and Governance}
\subsection{Documentation Release Cadence}
\begin{itemize}
  \item Release this package versioned alongside policy sets (e.g., quarterly or per major baseline change).
  \item Each release includes: updated views, updated control catalog, updated evidence mapping, updated ADR index.
\end{itemize}

\subsection{Confluence Page Tree Mapping (Suggested)}
\begin{itemize}
  \item \textbf{Space Home:} Cloud Governance \& Control Framework
  \item \textbf{1. Overview \& Strategy}
    \begin{itemize}
      \item Architecture Overview (this chapter)
      \item Stakeholders \& Concerns
      \item Principles \& ADR Index
    \end{itemize}
  \item \textbf{2. Architecture Views}
    \begin{itemize}
      \item Context View
      \item Logical View
      \item Process View
      \item Deployment View
    \end{itemize}
  \item \textbf{3. Controls \& Evidence}
    \begin{itemize}
      \item Control Catalog
      \item Evidence Model \& Retention
      \item Standards Mapping
      \item Exception Policy
    \end{itemize}
  \item \textbf{4. Operations}
    \begin{itemize}
      \item Monitoring \& Alerting
      \item Runbooks \& Playbooks
      \item SLOs \& Metrics
    \end{itemize}
  \item \textbf{5. Risks \& Roadmap}
    \begin{itemize}
      \item Risk Register
      \item Backlog / Improvement Roadmap
    \end{itemize}
\end{itemize}

\clearpage

%===============================================================================
\appendix
\chapter{Appendix A: Diagram Source (Optional)}
If you maintain diagram sources (PlantUML/Mermaid) alongside exported images, include them here
for reproducibility.

\section{PlantUML: High-Level Reference (From Earlier Draft)}
\begin{minted}[fontsize=\small,breaklines]{text}
@startuml
title Cloud Governance & Control Framework - Reference Architecture (High Level)
skinparam componentStyle rectangle

package "Strategy & Policy (Intent)" {
  [Governance Policies\n(Security, Identity, Cost, Ops, Compliance)]
  [Standards & Control Mapping\n(NIST/ISO/SOC2/etc.)]
  [Risk Appetite & Exceptions\n(waivers, approvals, expiry)]
}

package "Guardrails & Control Plane (Prevent/Detect)" {
  [Landing Zones & Org Structure]
  [IAM/RBAC Guardrails\n(least privilege, PIM)]
  [Policy-as-Code\n(tagging, encryption, regions)]
  [Config/Posture Monitoring\n(drift, misconfig)]
  [Central Logging/SIEM Feed]
  [Cost Controls\n(budgets, alerts, allocation)]
}

package "Execution (Build/Run)" {
  [Self-Service Catalog]
  [IaC Pipelines\n(modules, reviews, gates)]
  [Change/Release Mgmt]
  [Incident & Vulnerability Mgmt]
  [Asset/CMDB Inventory]
}

package "Evidence & Reporting (Prove/Improve)" {
  [Dashboards\n(Security/Compliance/FinOps/Ops)]
  [Automated Evidence Store]
  [Audit Reporting]
  [Optimization Backlog\n(remediation + improvements)]
}

[Governance Policies\n(Security, Identity, Cost, Ops, Compliance)] --> [Policy-as-Code\n(tagging, encryption, regions)]
[Standards & Control Mapping\n(NIST/ISO/SOC2/etc.)] --> [Automated Evidence Store]
[Landing Zones & Org Structure] --> [Self-Service Catalog]
[IaC Pipelines\n(modules, reviews, gates)] --> [Config/Posture Monitoring\n(drift, misconfig)]
[Central Logging/SIEM Feed] --> [Dashboards\n(Security/Compliance/FinOps/Ops)]
[Cost Controls\n(budgets, alerts, allocation)] --> [Dashboards\n(Security/Compliance/FinOps/Ops)]
[Automated Evidence Store] --> [Audit Reporting]
[Dashboards\n(Security/Compliance/FinOps/Ops)] --> [Optimization Backlog\n(remediation + improvements)]
@enduml
\end{minted}

\chapter{Appendix B: View Packet Checklist (Template)}
For each view, ensure the packet includes:
\begin{itemize}
  \item View packet summary (stakeholders, concerns, scope)
  \item Primary presentation (diagram)
  \item Element catalog (tables of elements and responsibilities)
  \item Interfaces/assumptions/constraints relevant to the view
  \item Variability (what can change, extension points)
  \item Rationale (why structured this way)
  \item Cross-references to other views and to controls/evidence
\end{itemize}

\end{document}

