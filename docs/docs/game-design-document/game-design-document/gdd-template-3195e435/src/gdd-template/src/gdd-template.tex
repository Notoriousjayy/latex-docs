\documentclass[12pt]{article}

\usepackage[margin=1in]{geometry}
\usepackage[T1]{fontenc}
\usepackage[utf8]{inputenc}
\usepackage{lmodern}
\usepackage{microtype}
\usepackage{setspace}
\usepackage{enumitem}
\usepackage{hyperref}

% Global typography and spacing for readability
\setstretch{1.2}
\setlength{\parindent}{0pt}
\setlength{\parskip}{0.7em}

% Compact but readable lists
\setlist{itemsep=0.35em, topsep=0.25em}

% Link styling
\hypersetup{
  colorlinks=true,
  linkcolor=blue,
  urlcolor=blue,
  citecolor=blue
}

\title{Game Design Document (GDD) Template}
\author{<Your Name / Studio>}
\date{\today}

% Helper for placeholders the team should fill in
\newcommand{\placeholder}[1]{\textbf{\textit{[#1]}}}

\begin{document}
\raggedright

\maketitle
\thispagestyle{empty}

\begin{center}
    \vspace{1em}
    \textbf{How to Use This Template}
\end{center}

Use this template to outline the key design elements of your game. Fill in each section with
details specific to your project. Not all sections will be relevant for every game, so feel free
to add, remove, or adjust sections as needed.

The goal is to create a clear, concise document that:
\begin{itemize}
  \item communicates the vision to new team members quickly,
  \item guides design and implementation decisions, and
  \item stays useful throughout development (not just at the pitch stage).
\end{itemize}

\vfill
\begin{center}
    \textit{Version:} \placeholder{Version Number} \\
    \textit{Last Updated:} \placeholder{Date}
\end{center}

\setcounter{tocdepth}{2}
\newpage
\tableofcontents
\newpage

%======================================================
\section{Game Overview}
%======================================================

\textbf{Purpose.} Give a high-level snapshot so anyone can understand what the game is, who it is for,
and what makes it special after reading just this section.

\subsection{High-Level Summary}

\begin{description}[leftmargin=3.2cm, labelsep=0.8cm, style=nextline]
    \item[Title]
        \placeholder{Game Title} \\
        Name of your game (working or final).
    \item[Genre]
        \placeholder{Core genre(s)} \\
        For example: action-platformer, narrative adventure, puzzle game, tactics RPG, etc.
    \item[Platform(s)]
        \placeholder{Target platforms} \\
        For example: PC, console, mobile, web, or specific hardware targets.
    \item[Target Audience]
        \placeholder{Who you are building this for} \\
        Age range, experience level, and any niche communities.
    \item[Elevator Pitch]
        \placeholder{One or two sentences that capture the core hook} \\
        For example: \textit{"A fast-paced, physics-driven arena game where players control
        customizable drones in competitive matches."}
\end{description}

\subsection{Vision and Goals}

\begin{itemize}
    \item \textbf{Design Goals.} \placeholder{Key design goals for the player experience.}
    \item \textbf{Differentiators.} \placeholder{What makes this game stand out from similar titles?}
    \item \textbf{Success Criteria.} \placeholder{How you will know the game has succeeded (qualitative and quantitative).}
\end{itemize}

%======================================================
\section{Gameplay Mechanics}
%======================================================

\textbf{Purpose.} Describe what the player actually does, how it feels, and how the game keeps
them engaged over time.

\subsection{Core Gameplay Loop}

\subsubsection*{Description}

\placeholder{Describe the primary loop.} For example: \\
\textit{Explore} $\rightarrow$ \textit{Collect} $\rightarrow$ \textit{Upgrade} $\rightarrow$ \textit{Repeat}.

Explain what the player does repeatedly during normal play and why that loop is satisfying.

\subsubsection*{Player Verbs}

List the main things the player can do and when they can do them.

\begin{itemize}
    \item \placeholder{Movement actions (walk, run, jump, dash, etc.)}
    \item \placeholder{Combat / interaction actions (attack, block, shoot, talk, interact, etc.)}
    \item \placeholder{Puzzle / system interactions (inspect, combine, operate, etc.)}
\end{itemize}

\subsection{Player Actions and Controls}

\subsubsection*{Control Scheme}

\begin{description}[leftmargin=3.2cm, labelsep=0.8cm, style=nextline]
    \item[Input Devices]
        \placeholder{Keyboard and mouse, controller, touch, etc.}
    \item[Default Mapping]
        \placeholder{Describe the default key / button layout for each supported device.}
    \item[Accessibility]
        \placeholder{Remapping, accessibility settings, input assist features, etc.}
\end{description}

\subsection{Progression System}

\begin{itemize}
    \item \textbf{Player Progression.} \placeholder{Levels, experience, skill trees, stat upgrades, etc.}
    \item \textbf{Content Unlocks.} \placeholder{New abilities, areas, characters, or modes unlocked over time.}
    \item \textbf{Difficulty Curve.} \placeholder{How difficulty ramps up or adapts.}
\end{itemize}

\subsection{Game Modes}

\begin{itemize}
    \item \textbf{Single-Player.} \placeholder{Campaign, story mode, challenges, etc.}
    \item \textbf{Multiplayer / Co-op.} \placeholder{Local / online modes, competitive or cooperative focus.}
    \item \textbf{Additional Modes.} \placeholder{Endless, time attack, sandbox, practice, etc.}
\end{itemize}

%======================================================
\section{Story and World-Building}
%======================================================

\textbf{Purpose.} Explain the narrative, setting, and characters well enough that writers, artists,
and designers make consistent choices.

\subsection{Narrative Summary}

\subsubsection*{Premise}

\placeholder{One or two paragraphs that summarize the core story.} Focus on the central conflict
and what the player is trying to achieve.

\subsubsection*{Themes}

\begin{itemize}
    \item \placeholder{Core themes (e.g., redemption, survival, friendship, betrayal).}
    \item \placeholder{How those themes show up in gameplay and story events.}
\end{itemize}

\subsection{Setting and World}

\begin{itemize}
    \item \textbf{Time Period.} \placeholder{When the game takes place (past, present, future, etc.).}
    \item \textbf{Location.} \placeholder{Primary regions, nations, or realms.}
    \item \textbf{Atmosphere.} \placeholder{Overall vibe: dark and gritty, cute and whimsical, etc.}
\end{itemize}

\subsection{Characters}

List the main characters and any important secondary characters.

\begin{description}[leftmargin=3.2cm, labelsep=0.8cm, style=nextline]
    \item[Protagonist(s)]
        \placeholder{Name, role, motivation, and core traits.}
    \item[Antagonists]
        \placeholder{Primary enemy or opposing force, plus notable villains.}
    \item[NPCs]
        \placeholder{Important non-player characters and their purpose.}
\end{description}

\subsection{Dialogue and Tone}

\begin{itemize}
    \item \textbf{Tone.} \placeholder{Humorous, serious, dark, lighthearted, etc.}
    \item \textbf{Dialogue Style.} \placeholder{Short and punchy, verbose, stylized, realistic, etc.}
    \item \textbf{Delivery.} \placeholder{Text only, voice acting, dialogue choices, cutscenes, in-world events, etc.}
\end{itemize}

%======================================================
\section{Art and Visuals}
%======================================================

\textbf{Purpose.} Provide a clear visual target so all art feels like one coherent game, not a mix
of unrelated styles.

\subsection{Art Direction Overview}

\begin{itemize}
    \item \textbf{Overall Style.} \placeholder{2D / 3D, pixel art, low-poly, painterly, realistic, etc.}
    \item \textbf{Color Palette.} \placeholder{General palette and how color supports mood and gameplay.}
    \item \textbf{Visual References.} \placeholder{Other games, films, or art styles that are relevant.}
\end{itemize}

\subsection{Characters and Creatures}

\begin{itemize}
    \item \textbf{Silhouette.} \placeholder{Distinctive shapes that make characters readable at a glance.}
    \item \textbf{Costuming.} \placeholder{Clothing, armor, accessories, and how they reflect role and personality.}
    \item \textbf{Animation Considerations.} \placeholder{How design supports movement, combat, and expression.}
\end{itemize}

\subsection{Environment Art}

\begin{itemize}
    \item \textbf{Regions / Biomes.} \placeholder{List key regions (forest, city, desert, etc.) and their visual identity.}
    \item \textbf{Landmarks.} \placeholder{Important structures or locations that help orientation and storytelling.}
    \item \textbf{Environmental Storytelling.} \placeholder{How the world hints at history, conflict, or ongoing events.}
\end{itemize}

\subsection{User Interface (UI)}

\begin{itemize}
    \item \textbf{Style.} \placeholder{Minimalist, skeuomorphic, retro, high-tech, etc.}
    \item \textbf{Layout.} \placeholder{Key screens: main menu, HUD, inventory, settings, etc.}
    \item \textbf{Readability.} \placeholder{Legibility at different resolutions, colorblind-safe choices, contrast, etc.}
\end{itemize}

\subsection{Effects and Feedback}

\begin{itemize}
    \item \textbf{VFX.} \placeholder{Particles, explosions, spell effects, hit flashes, etc.}
    \item \textbf{Feedback.} \placeholder{How visual effects reinforce hits, damage, healing, status changes, etc.}
\end{itemize}

%======================================================
\section{Audio and Sound}
%======================================================

\textbf{Purpose.} Define how music, sound effects, and voice support the game’s mood and feedback.

\subsection{Music}

\begin{itemize}
    \item \textbf{Style.} \placeholder{Epic orchestral, lo-fi, retro chiptune, ambient, etc.}
    \item \textbf{Mood.} \placeholder{How the music supports gameplay, pacing, and emotion.}
    \item \textbf{References.} \placeholder{Inspiration from other games, films, or artists.}
\end{itemize}

\subsection{Sound Effects}

\begin{itemize}
    \item \textbf{Key Categories.} \placeholder{Movement, combat, environment, UI, system alerts.}
    \item \textbf{Style.} \placeholder{Realistic, arcade-like, comedic, minimalistic, etc.}
    \item \textbf{Gameplay Feedback.} \placeholder{Distinct audio cues for damage, warnings, rewards, etc.}
\end{itemize}

\subsection{Voice-Over}

\begin{itemize}
    \item \textbf{Scope.} \placeholder{Fully voiced, partial VO, or none.}
    \item \textbf{Language Support.} \placeholder{Supported languages and localization plans.}
    \item \textbf{Direction.} \placeholder{Desired performance style and constraints.}
\end{itemize}

%======================================================
\section{Level and Environment Design}
%======================================================

\textbf{Purpose.} Show how players move through the game world and how each space supports the
gameplay and story.

\subsection{Structure}

\begin{itemize}
    \item \textbf{Overall Flow.} \placeholder{Linear, hub-and-spoke, open world, mission-based, etc.}
    \item \textbf{Pacing.} \placeholder{How intensity, challenge, and downtime are arranged.}
    \item \textbf{Replayability.} \placeholder{Shortcuts, secrets, multiple paths, optional content, etc.}
\end{itemize}

\subsection{Key Locations / Levels}

\begin{itemize}
    \item \placeholder{List each major level or area and provide a short description.}
    \item \placeholder{Note the primary gameplay focus of each (exploration, combat, puzzle, narrative, etc.).}
\end{itemize}

\subsection{Encounter and Challenge Design}

\begin{itemize}
    \item \placeholder{How enemies, traps, puzzles, or challenges are distributed.}
    \item \placeholder{How difficulty escalates and how players learn new mechanics.}
\end{itemize}

%======================================================
\section{Technical Specifications}
%======================================================

\textbf{Purpose.} Capture the technical constraints and targets so engineering and design stay aligned.

\subsection{Target Platforms and Performance}

\begin{itemize}
    \item \textbf{Platform Targets.} \placeholder{PC, console, mobile, browser, etc.}
    \item \textbf{Frame Rate Goal.} \placeholder{Target FPS and resolution.}
    \item \textbf{Performance Constraints.} \placeholder{Memory, loading times, file size, etc.}
\end{itemize}

\subsection{Engine and Tools}

\begin{itemize}
    \item \textbf{Engine.} \placeholder{Game engine or custom framework.}
    \item \textbf{Toolchain.} \placeholder{Level editors, scripting tools, asset pipelines, etc.}
\end{itemize}

\subsection{Networking (if applicable)}

\begin{itemize}
    \item \textbf{Model.} \placeholder{Peer-to-peer, client-server, dedicated server, etc.}
    \item \textbf{Features.} \placeholder{Matchmaking, leaderboards, lobbies, cross-play, etc.}
    \item \textbf{Constraints.} \placeholder{Latency tolerance, bandwidth considerations, tick rate, etc.}
\end{itemize}

\subsection{Save and Persistence}

\begin{itemize}
    \item \textbf{Save System.} \placeholder{Manual saves, autosave, checkpoints, cloud saves.}
    \item \textbf{Data Stored.} \placeholder{Player progress, inventory, options, statistics, etc.}
\end{itemize}

%======================================================
\section{Monetization and Business Model (if applicable)}
%======================================================

\textbf{Purpose.} Make the money model explicit so it supports the design instead of undermining it.

\subsection{Model Overview}

\begin{itemize}
    \item \textbf{Business Model.} \placeholder{Premium, free-to-play, subscription, DLC-based, etc.}
    \item \textbf{Purchasable Content.} \placeholder{Cosmetics, expansions, boosts, battle passes, etc.}
\end{itemize}

\subsection{Player-Friendly Principles}

\begin{itemize}
    \item \placeholder{How you avoid pay-to-win, exploitative systems, or aggressive friction.}
    \item \placeholder{How purchases feel optional, rewarding, and fair.}
\end{itemize}

%======================================================
\section{Project Schedule and Milestones}
%======================================================

\textbf{Purpose.} Provide a realistic roadmap that production, stakeholders, and team members can
refer to at a glance.

\subsection{High-Level Phases}

\begin{itemize}
    \item \placeholder{Pre-production (concept, prototyping, planning).}
    \item \placeholder{Production (core development, content creation).}
    \item \placeholder{Alpha / Beta (feature complete, polish, bug fixing).}
    \item \placeholder{Launch and Post-launch (release, patches, updates, live operations).}
\end{itemize}

\subsection{Key Milestones}

\begin{itemize}
    \item \placeholder{Prototype complete.}
    \item \placeholder{Vertical slice or first playable.}
    \item \placeholder{Content complete.}
    \item \placeholder{Release candidates, launch date.}
\end{itemize}

\subsection{Risks and Dependencies}

\begin{itemize}
    \item \placeholder{Technical risks (engine limitations, performance issues, platform requirements).}
    \item \placeholder{Design risks (unproven mechanics, scope creep).}
    \item \placeholder{External dependencies (publishers, vendors, certification, approvals).}
\end{itemize}

\subsection{Resourcing (Optional)}

\begin{itemize}
    \item \placeholder{Team roles and approximate headcount.}
    \item \placeholder{External contractors, vendors, or partners.}
\end{itemize}

\subsection{Timeline Summary}

\begin{itemize}
    \item \placeholder{Estimated schedule by date or by phase duration.}
    \item \placeholder{Key internal and external deadlines.}
    \item \placeholder{Buffers or risk allowances for delays.}
\end{itemize}

%======================================================
\section{Appendices}
%======================================================

\textbf{Purpose.} Collect supporting material without cluttering the main sections.

\subsection{Additional Materials}

\begin{itemize}
    \item \placeholder{Links or references to concept art boards.}
    \item \placeholder{Level layout sketches and diagrams.}
    \item \placeholder{Flowcharts for mechanics, combat systems, or menus.}
    \item \placeholder{Storyboards for cutscenes or key narrative moments.}
\end{itemize}

\subsection{References and Inspiration}

\begin{itemize}
    \item \placeholder{List of games, films, books, or articles that inspired the design.}
    \item \placeholder{Any external resources or research that informed this document.}
\end{itemize}

\end{document}
