
% =====================================================================
% Study Plan — Handbook of Calculus (User Stories) — ALT (no tabularx)
% =====================================================================
\documentclass[11pt,a4paper]{article}

% Encoding & layout
\usepackage[T1]{fontenc}
\usepackage[utf8]{inputenc}
\usepackage{lmodern}
\usepackage{microtype}
\usepackage[a4paper,margin=1in]{geometry}
\usepackage{parskip}

% Colors & links
\usepackage{xcolor}
\definecolor{CardBorder}{RGB}{190,190,190}
\definecolor{AccentA}{RGB}{14,116,144}
\definecolor{AccentB}{RGB}{56,189,248}
\definecolor{PillBack}{gray}{0.95}
\definecolor{PillFrame}{gray}{0.80}
\definecolor{Muted}{gray}{0.35}

\usepackage{hyperref}
\hypersetup{colorlinks=true, linkcolor=AccentA, urlcolor=AccentA, citecolor=AccentA}

% Tables / lists
\usepackage{array}
\usepackage{booktabs}
\usepackage{enumitem}

% Math / symbols
\usepackage{amsmath,amssymb}

% Story-card visuals
\usepackage[most]{tcolorbox}
\tcbset{
  colback=white,
  colframe=CardBorder,
  boxrule=0.35pt,
  arc=2mm,
  enhanced,
  breakable,
}

% Left accent helper (dual stripe)
\newtcolorbox{storycard}[2][]{
  title={\textbf{#2}},
  fonttitle=\bfseries,
  borderline west={2pt}{0pt}{AccentA},
  borderline west={4pt}{-2pt}{AccentB!70},
  boxed title style={colback=white, colframe=white, left=0pt},
  #1
}

% Metadata/Bdd tables (plain tabular with fixed widths)
\newenvironment{MetaTable}{%
  \begin{tabular}{@{}>{\bfseries}p{0.27\linewidth}@{\hspace{0.75em}}p{0.70\linewidth}@{}}%
}{%
  \end{tabular}%
}
\newcommand{\MetaRow}[2]{#1 & #2\\}

% "pill" chips
\newtcbox{\pill}{on line, arc=6pt, colback=PillBack, colframe=PillFrame, boxrule=0.3pt, left=5pt, right=5pt, top=2pt, bottom=2pt}

% Checklist
\newlist{checklist}{itemize}{1}
\setlist[checklist]{label=$\square$, left=1.6em, itemsep=0.25em, topsep=0.3em}

% BDD block reuses MetaTable
\newenvironment{BDD}{\smallskip\begin{MetaTable}}{\end{MetaTable}\smallskip}

% Tiny footnote line (DoR/DoD)
\newcommand{\footline}[1]{\par\smallskip{\scriptsize\color{Muted}#1}\par}

\begin{document}

\begin{center}
{\LARGE \textbf{Study Plan --- Handbook of Calculus}}\\[0.25em]
{\large User Stories for Each Chapter}
\end{center}

\bigskip
This deck includes one story card per chapter. Each card ends with \texttt{\textbackslash clearpage} to ensure print separation.

% ---------------------------- P --------------------------------------
\begin{storycard}{P --- Preparation for Calculus}
\begin{MetaTable}
  \MetaRow{Epic / Feature}{Precalculus Fluency}
  \MetaRow{Business Value}{Reduce rework during calculus by restoring algebra/trig/log skills and modeling literacy.}
  \MetaRow{Priority / Estimate}{Priority: Must \quad SP: 2}
  \MetaRow{Persona}{learner returning to math or entering Calculus I}
  \MetaRow{Dependencies}{None}
  \MetaRow{Assumptions / Risks}{Risk: hidden skill gaps; plan daily 20-minute drills and spaced repetition.}
\end{MetaTable}

\medskip
\textbf{Story}\quad
\emph{As a calculus learner, I want to refresh function operations, trig, and exponential/log rules so that I can focus on calculus concepts instead of algebraic manipulation.}

\medskip
\textbf{Skills Emphasis}\quad
\pill{Functions} \pill{Graphs} \pill{Trig} \pill{Exp/Log}

\medskip
\textbf{Acceptance Criteria (BDD)}
\begin{BDD}
  \MetaRow{Scenario}{Fluency check}
  \MetaRow{Given}{a mixed set of 25 precalculus items (domains, inverses, identities, transformations)}
  \MetaRow{When}{I solve without a key and justify steps}
  \MetaRow{Then}{I score at least 80 percent and record topics for targeted review.}
\end{BDD}

\footline{Definition of Ready: Drill bank prepared; schedule set. \quad Definition of Done: Diagnostic 80+; error log created; review plan scheduled.}

\medskip\hrule\medskip

\textbf{Tasks}
\begin{checklist}
  \item Complete diagnostic (25 questions) across functions, trig, and logs.
  \item Build a formula sheet: trig identities, log laws, function transformations.
  \item Graph five function families with key features (intercepts, asymptotes).
  \item Mini-project: model a cooling or population scenario with exp/log.
  \item Reflection: list three algebra pitfalls and fixes.
\end{checklist}
\end{storycard}
\clearpage

% ---------------------------- 1 --------------------------------------
\begin{storycard}{Ch. 1 --- Limits and Their Properties}
\begin{MetaTable}
  \MetaRow{Epic / Feature}{Foundations}
  \MetaRow{Business Value}{Ground the derivative and integral in precise limiting behavior.}
  \MetaRow{Priority / Estimate}{Priority: Must \quad SP: 3}
  \MetaRow{Persona}{calculus student}
  \MetaRow{Dependencies}{P Preparation}
  \MetaRow{Assumptions / Risks}{Confusing value with limit; overusing numeric tables without justification.}
\end{MetaTable}

\medskip
\textbf{Story}\quad
\emph{As a learner, I want to compute and reason about limits and continuity so that I can validate derivative rules and model transitions reliably.}

\medskip
\textbf{Skills Emphasis}\quad
\pill{Analytic} \pill{Graphical} \pill{Continuity} \pill{IVT}

\medskip
\textbf{Acceptance Criteria (BDD)}
\begin{BDD}
  \MetaRow{Scenario}{Happy path}
  \MetaRow{Given}{piecewise and rational functions}
  \MetaRow{When}{I compute one-sided and two-sided limits, detect asymptotes, and apply continuity tests}
  \MetaRow{Then}{My conclusions match graphs/tables and I can cite theorems (Limit Laws, Squeeze, IVT) correctly.}
\end{BDD}

\footline{DoR: Persona clear; AC drafted; dependencies known. \quad DoD: All ACs pass; proof sketch for Squeeze or IVT written; summary logged.}

\medskip\hrule\medskip

\textbf{Tasks}
\begin{checklist}
  \item Build a concept map for limits (notation, properties, pitfalls).
  \item Evaluate 15 mixed limits (algebraic, trigonometric, infinity).
  \item Prove continuity of polynomials and rational functions on domains.
  \item Investigate a removable vs. jump discontinuity with a custom example.
  \item Reflection: three conditions where substitution fails.
\end{checklist}
\end{storycard}
\clearpage

% ---------------------------- 2 --------------------------------------
\begin{storycard}{Ch. 2 --- Differentiation}
\begin{MetaTable}
  \MetaRow{Epic / Feature}{Core Techniques}
  \MetaRow{Business Value}{Compute derivatives efficiently for modeling instantaneous change.}
  \MetaRow{Priority / Estimate}{Priority: Must \quad SP: 3}
  \MetaRow{Persona}{learner preparing for STEM applications}
  \MetaRow{Dependencies}{Ch. 1}
  \MetaRow{Assumptions / Risks}{Over-memorizing rules without linearization intuition.}
\end{MetaTable}

\medskip
\textbf{Story}\quad
\emph{As a learner, I want to differentiate using rules, implicit methods, and differentials so that I can analyze local behavior quickly and accurately.}

\medskip
\textbf{Skills Emphasis}\quad
\pill{Rules} \pill{Implicit} \pill{Related Rates} \pill{Linearization}

\medskip
\textbf{Acceptance Criteria (BDD)}
\begin{BDD}
  \MetaRow{Scenario}{Technique selection}
  \MetaRow{Given}{compositions and products/quotients}
  \MetaRow{When}{I select and apply the appropriate rules and simplify}
  \MetaRow{Then}{Results match a CAS on random checks and I can interpret units.}
\end{BDD}

\footline{DoR: Problems sourced; checklist ready. \quad DoD: ACs pass; linearization mini-lab submitted; error analysis done.}

\medskip\hrule\medskip

\textbf{Tasks}
\begin{checklist}
  \item Drill set (20): power, product, quotient, chain in mixed order.
  \item Implicit differentiation including circles, exponentials, and logs.
  \item Related rates with units (ladder, tank, or shadow problem).
  \item Linear approximation of a nontrivial value; compute relative error.
  \item Reflection: create a rule-selection flowchart.
\end{checklist}
\end{storycard}
\clearpage

% ---------------------------- 3 --------------------------------------
\begin{storycard}{Ch. 3 --- Applications of Differentiation}
\begin{MetaTable}
  \MetaRow{Epic / Feature}{Analysis and Design}
  \MetaRow{Business Value}{Optimize systems and understand graph behavior for decision-making.}
  \MetaRow{Priority / Estimate}{Priority: Must \quad SP: 3}
  \MetaRow{Persona}{applied learner}
  \MetaRow{Dependencies}{Ch. 2}
  \MetaRow{Assumptions / Risks}{Confusing local and global extrema; Newton method divergence if not guarded.}
\end{MetaTable}

\medskip
\textbf{Story}\quad
\emph{As a learner, I want to analyze first and second derivatives to optimize and sketch curves so that I can solve design and root-finding problems.}

\medskip
\textbf{Skills Emphasis}\quad
\pill{Extrema} \pill{Concavity} \pill{Optimization} \pill{Newton}

\medskip
\textbf{Acceptance Criteria (BDD)}
\begin{BDD}
  \MetaRow{Scenario}{Optimization}
  \MetaRow{Given}{a constrained word problem with meaningful units}
  \MetaRow{When}{I model, differentiate, and apply derivative tests}
  \MetaRow{Then}{I obtain a defendable optimum and interpret sensitivity.}
\end{BDD}

\footline{DoR: Realistic contexts selected. \quad DoD: ACs pass; sketching gallery created; Newton method guardrails documented.}

\medskip\hrule\medskip

\textbf{Tasks}
\begin{checklist}
  \item Classification practice using first/second derivative tests (12 items).
  \item Model and solve two optimization problems (geometry and economics).
  \item Limits at infinity and asymptotes set (8 items).
  \item Implement Newton method on two functions; document failure cases.
  \item Reflection: checklists for model assumptions and units.
\end{checklist}
\end{storycard}
\clearpage

% ---------------------------- 4 --------------------------------------
\begin{storycard}{Ch. 4 --- Integration}
\begin{MetaTable}
  \MetaRow{Epic / Feature}{Accumulation}
  \MetaRow{Business Value}{Connect areas and accumulations to definite integrals and the FTC.}
  \MetaRow{Priority / Estimate}{Priority: Must \quad SP: 3}
  \MetaRow{Persona}{calculus student}
  \MetaRow{Dependencies}{Ch. 1--3}
  \MetaRow{Assumptions / Risks}{Confusing antiderivative with definite integral; unit interpretation neglected.}
\end{MetaTable}

\medskip
\textbf{Story}\quad
\emph{As a learner, I want to interpret the definite integral as an accumulation and compute it using substitution so that I can solve area and total-change problems.}

\medskip
\textbf{Skills Emphasis}\quad
\pill{Riemann} \pill{FTC} \pill{Substitution} \pill{Units}

\medskip
\textbf{Acceptance Criteria (BDD)}
\begin{BDD}
  \MetaRow{Scenario}{Area and accumulation}
  \MetaRow{Given}{a rate function with units}
  \MetaRow{When}{I set up and evaluate the definite integral and explain units}
  \MetaRow{Then}{My answer matches a numeric check and the interpretation is correct.}
\end{BDD}

\footline{DoR: Contexts chosen; notation sheet prepared. \quad DoD: ACs pass; summary card created; error log updated.}

\medskip\hrule\medskip
\textbf{Tasks}
\begin{checklist}
  \item Derive FTC parts I and II from area accumulation arguments (sketch).
  \item Evaluate 15 integrals (basic antiderivatives and substitution).
  \item Area between a curve and axis; signed vs total area comparison.
  \item Create a units table linking rate, integral, and accumulation.
  \item Reflection: three cases where substitution is not suitable.
\end{checklist}
\end{storycard}
\clearpage

% ---------------------------- 5 --------------------------------------
\begin{storycard}{Ch. 5 --- Logarithmic, Exponential, and Other Transcendentals}
\begin{MetaTable}
  \MetaRow{Epic / Feature}{Transcendentals}
  \MetaRow{Business Value}{Model growth/decay and handle inverse trig/hyperbolic functions; resolve indeterminate forms.}
  \MetaRow{Priority / Estimate}{Priority: Must \quad SP: 3}
  \MetaRow{Persona}{STEM learner}
  \MetaRow{Dependencies}{Ch. 2, 4}
  \MetaRow{Assumptions / Risks}{Misusing L'Hopital; domain issues for logs and inverse trig.}
\end{MetaTable}

\medskip
\textbf{Story}\quad
\emph{As a learner, I want to differentiate and integrate logarithmic, exponential, inverse trig, and hyperbolic functions so that I can solve real growth/decay and geometry problems.}

\medskip
\textbf{Skills Emphasis}\quad
\pill{Exp/Log} \pill{InverseTr} \pill{Hyperbolic} \pill{LHopital}

\medskip
\textbf{Acceptance Criteria (BDD)}
\begin{BDD}
  \MetaRow{Scenario}{Model fit}
  \MetaRow{Given}{data compatible with exponential or logarithmic behavior}
  \MetaRow{When}{I fit a model, compute derivatives/integrals, and analyze}
  \MetaRow{Then}{Predictions match baseline within tolerance and domain constraints are respected.}
\end{BDD}

\footline{DoR: Data chosen; domain notes prepared. \quad DoD: ACs pass; L'Hopital applicability checklist written.}

\medskip\hrule\medskip
\textbf{Tasks}
\begin{checklist}
  \item Log differentiation exercises, including products and powers.
  \item Integrate forms with inverse trig; note geometry interpretations.
  \item Apply L'Hopital to limits with proof of conditions.
  \item Growth/decay mini-project with parameter interpretation.
  \item Reflection: domain/range pitfalls encountered.
\end{checklist}
\end{storycard}
\clearpage

% ---------------------------- 6 --------------------------------------
\begin{storycard}{Ch. 6 --- Differential Equations (Intro)}
\begin{MetaTable}
  \MetaRow{Epic / Feature}{Modeling Change}
  \MetaRow{Business Value}{Translate real processes into solvable first-order ODEs.}
  \MetaRow{Priority / Estimate}{Priority: Should \quad SP: 3}
  \MetaRow{Persona}{applied learner}
  \MetaRow{Dependencies}{Ch. 4, 5}
  \MetaRow{Assumptions / Risks}{Overfitting models; ignoring equilibrium analysis.}
\end{MetaTable}

\medskip
\textbf{Story}\quad
\emph{As a learner, I want to read slope fields and solve separable and linear ODEs so that I can model growth/decay and approach-to-equilibrium systems.}

\medskip
\textbf{Skills Emphasis}\quad
\pill{SlopeFields} \pill{Separable} \pill{Linear} \pill{Logistic}

\medskip
\textbf{Acceptance Criteria (BDD)}
\begin{BDD}
  \MetaRow{Scenario}{Model validation}
  \MetaRow{Given}{data and a plausible first-order model}
  \MetaRow{When}{I solve the ODE and compare to data}
  \MetaRow{Then}{Residuals are small; equilibria and stability are interpreted.}
\end{BDD}

\footline{DoR: Context picked; numeric/analytic comparison plan. \quad DoD: ACs pass; slope-field sketch; equilibrium table submitted.}

\medskip\hrule\medskip
\textbf{Tasks}
\begin{checklist}
  \item Sketch slope fields; identify isoclines and equilibria.
  \item Solve three separable ODEs and two linear ODEs.
  \item Fit a logistic model and interpret parameters.
  \item Euler method approximation for one problem; compare error.
  \item Reflection: modeling assumptions check.
\end{checklist}
\end{storycard}
\clearpage

% ---------------------------- 7 --------------------------------------
\begin{storycard}{Ch. 7 --- Applications of Integration}
\begin{MetaTable}
  \MetaRow{Epic / Feature}{Geometry and Physics}
  \MetaRow{Business Value}{Solve area, volume, work, and centroid problems via integrals.}
  \MetaRow{Priority / Estimate}{Priority: Must \quad SP: 3}
  \MetaRow{Persona}{applied learner}
  \MetaRow{Dependencies}{Ch. 4}
  \MetaRow{Assumptions / Risks}{Setup errors with bounds or radii; units confusion.}
\end{MetaTable}

\medskip
\textbf{Story}\quad
\emph{As a learner, I want to formulate geometry and physics problems as definite integrals so that I can compute areas, volumes, and work reliably.}

\medskip
\textbf{Skills Emphasis}\quad
\pill{Areas} \pill{Volumes} \pill{ArcLength} \pill{Work/Centroids}

\medskip
\textbf{Acceptance Criteria (BDD)}
\begin{BDD}
  \MetaRow{Scenario}{Volume calculation}
  \MetaRow{Given}{a region described by curves}
  \MetaRow{When}{I compute volume by disks/washers and by shells}
  \MetaRow{Then}{Both methods agree and units are correct.}
\end{BDD}

\footline{DoR: Diagrams prepared. \quad DoD: ACs pass; summary of method choice rules written.}

\medskip\hrule\medskip
\textbf{Tasks}
\begin{checklist}
  \item Areas between curves (5 problems).
  \item Volumes by disks/washers and shells (6 problems).
  \item Arc length and surface area (3 problems).
  \item Work and centroids with densities (2 problems).
  \item Reflection: method selection decision tree.
\end{checklist}
\end{storycard}
\clearpage

% ---------------------------- 8 --------------------------------------
\begin{storycard}{Ch. 8 --- Integration Techniques and Improper Integrals}
\begin{MetaTable}
  \MetaRow{Epic / Feature}{Toolbox Expansion}
  \MetaRow{Business Value}{Handle complex integrals and convergence questions.}
  \MetaRow{Priority / Estimate}{Priority: Must \quad SP: 3}
  \MetaRow{Persona}{calculus student}
  \MetaRow{Dependencies}{Ch. 4, 5}
  \MetaRow{Assumptions / Risks}{Choosing wrong technique; algebra slips increase.}
\end{MetaTable}

\medskip
\textbf{Story}\quad
\emph{As a learner, I want a reliable process for choosing integration techniques and judging convergence so that I can solve challenging integrals.}

\medskip
\textbf{Skills Emphasis}\quad
\pill{ByParts} \pill{TrigInt} \pill{PartialFrac} \pill{Improper}

\medskip
\textbf{Acceptance Criteria (BDD)}
\begin{BDD}
  \MetaRow{Scenario}{Technique selection}
  \MetaRow{Given}{a mixed set of 15 integrals}
  \MetaRow{When}{I categorize and solve them with appropriate methods}
  \MetaRow{Then}{My solutions check by differentiation and convergence is justified.}
\end{BDD}

\footline{DoR: Mixed set prepared. \quad DoD: ACs pass; personal technique guide created.}

\medskip\hrule\medskip
\textbf{Tasks}
\begin{checklist}
  \item Create a technique-selection flowchart.
  \item Solve by parts (5), trig integrals/substitutions (5), partial fractions (5).
  \item Evaluate three improper integrals with tests.
  \item Compare exact vs numeric approximations and discuss error.
  \item Reflection: log frequent algebra errors.
\end{checklist}
\end{storycard}
\clearpage

% ---------------------------- 9 --------------------------------------
\begin{storycard}{Ch. 9 --- Infinite Series}
\begin{MetaTable}
  \MetaRow{Epic / Feature}{Approximation}
  \MetaRow{Business Value}{Decide convergence and approximate functions via power series.}
  \MetaRow{Priority / Estimate}{Priority: Must \quad SP: 3}
  \MetaRow{Persona}{STEM learner}
  \MetaRow{Dependencies}{Ch. 4, 5, 8}
  \MetaRow{Assumptions / Risks}{Confusing necessary vs sufficient conditions; ignoring remainder bounds.}
\end{MetaTable}

\medskip
\textbf{Story}\quad
\emph{As a learner, I want to test series for convergence and construct Taylor approximations so that I can approximate difficult functions and analyze error.}

\medskip
\textbf{Skills Emphasis}\quad
\pill{ConvergenceTests} \pill{PowerSeries} \pill{Taylor} \pill{ErrorBounds}

\medskip
\textbf{Acceptance Criteria (BDD)}
\begin{BDD}
  \MetaRow{Scenario}{Power series construction}
  \MetaRow{Given}{a differentiable function near a point}
  \MetaRow{When}{I build its Taylor series, determine radius/interval of convergence, and estimate remainder}
  \MetaRow{Then}{Approximations meet a stated error tolerance on a test interval.}
\end{BDD}

\footline{DoR: Test catalogue ready. \quad DoD: ACs pass; summary sheet of tests with conditions written.}

\medskip\hrule\medskip
\textbf{Tasks}
\begin{checklist}
  \item Apply comparison, ratio/root, alternating, and integral tests (12 items).
  \item Build Taylor polynomials of degree 2, 4, 6 for target functions.
  \item Determine radius/interval of convergence for five power series.
  \item Use remainder estimates to choose degree for tolerance.
  \item Reflection: create a tests decision table with cues.
\end{checklist}
\end{storycard}
\clearpage

% ---------------------------- 10 -------------------------------------
\begin{storycard}{Ch. 10 --- Conics, Parametric Equations, and Polar Coordinates}
\begin{MetaTable}
  \MetaRow{Epic / Feature}{Beyond y=f(x)}
  \MetaRow{Business Value}{Model plane curves and compute areas/lengths in alternate coordinates.}
  \MetaRow{Priority / Estimate}{Priority: Should \quad SP: 3}
  \MetaRow{Persona}{STEM learner}
  \MetaRow{Dependencies}{Ch. 4}
  \MetaRow{Assumptions / Risks}{Mismatched parameter ranges; orientation issues in polar.}
\end{MetaTable}

\medskip
\textbf{Story}\quad
\emph{As a learner, I want to analyze parametric and polar curves and conics so that I can compute tangents, areas, and lengths beyond Cartesian form.}

\medskip
\textbf{Skills Emphasis}\quad
\pill{Parametric} \pill{PolarArea} \pill{ArcLength} \pill{Conics}

\medskip
\textbf{Acceptance Criteria (BDD)}
\begin{BDD}
  \MetaRow{Scenario}{Polar area}
  \MetaRow{Given}{a polar curve with specified theta bounds}
  \MetaRow{When}{I compute enclosed area and verify with a numeric plot}
  \MetaRow{Then}{Results match within tolerance and orientation is explained.}
\end{BDD}

\footline{DoR: Plotting tool ready. \quad DoD: ACs pass; gallery of curves with annotations produced.}

\medskip\hrule\medskip
\textbf{Tasks}
\begin{checklist}
  \item Parametric derivatives and tangents; eliminate parameter where possible.
  \item Polar area for petals and loops (3 problems).
  \item Arc length for one parametric and one polar curve.
  \item Conic classification and focus-directrix properties.
  \item Reflection: parameter range and orientation pitfalls.
\end{checklist}
\end{storycard}
\clearpage

% ---------------------------- 11 -------------------------------------
\begin{storycard}{Ch. 11 --- Vectors and the Geometry of Space}
\begin{MetaTable}
  \MetaRow{Epic / Feature}{3D Foundations}
  \MetaRow{Business Value}{Operate in R3 with vectors, lines, planes, and coordinate systems.}
  \MetaRow{Priority / Estimate}{Priority: Must \quad SP: 3}
  \MetaRow{Persona}{multivariable learner}
  \MetaRow{Dependencies}{Ch. 1--3 (conceptual), algebra}
  \MetaRow{Assumptions / Risks}{Cross product direction errors; sign conventions.}
\end{MetaTable}

\medskip
\textbf{Story}\quad
\emph{As a learner, I want to compute vector operations and equations of lines and planes so that I can model geometry and motion in space.}

\medskip
\textbf{Skills Emphasis}\quad
\pill{Dot/Cross} \pill{Lines/Planes} \pill{Surfaces} \pill{Coordinates}

\medskip
\textbf{Acceptance Criteria (BDD)}
\begin{BDD}
  \MetaRow{Scenario}{Line-plane problems}
  \MetaRow{Given}{two points and a normal vector}
  \MetaRow{When}{I compute the line through points and plane through one point}
  \MetaRow{Then}{Intersections and distances are computed with correct units.}
\end{BDD}

\footline{DoR: Practice set selected. \quad DoD: ACs pass; coordinate conversion table created.}

\medskip\hrule\medskip
\textbf{Tasks}
\begin{checklist}
  \item Compute dot and cross products with geometric interpretation.
  \item Write parametric equations for lines; plane equations from points/normals.
  \item Classify quadric surfaces from equations.
  \item Convert between rectangular, cylindrical, spherical coordinates.
  \item Reflection: mnemonic for right-hand rule and orientation.
\end{checklist}
\end{storycard}
\clearpage

% ---------------------------- 12 -------------------------------------
\begin{storycard}{Ch. 12 --- Vector-Valued Functions}
\begin{MetaTable}
  \MetaRow{Epic / Feature}{Space Curves}
  \MetaRow{Business Value}{Model motion and curvature in R3.}
  \MetaRow{Priority / Estimate}{Priority: Should \quad SP: 3}
  \MetaRow{Persona}{STEM learner}
  \MetaRow{Dependencies}{Ch. 11}
  \MetaRow{Assumptions / Risks}{Arc-length parameterization mistakes.}
\end{MetaTable}

\medskip
\textbf{Story}\quad
\emph{As a learner, I want to differentiate/integrate vector functions and compute curvature so that I can analyze motion and turning behavior.}

\medskip
\textbf{Skills Emphasis}\quad
\pill{Velocity/Accel} \pill{T,N,B} \pill{Curvature} \pill{ArcLength}

\medskip
\textbf{Acceptance Criteria (BDD)}
\begin{BDD}
  \MetaRow{Scenario}{Curvature analysis}
  \MetaRow{Given}{a space curve r(t)}
  \MetaRow{When}{I compute T, N, curvature, and speed profiles}
  \MetaRow{Then}{Results are consistent with plots and units are correct.}
\end{BDD}

\footline{DoR: Curve library chosen. \quad DoD: ACs pass; annotated plot pack produced.}

\medskip\hrule\medskip
\textbf{Tasks}
\begin{checklist}
  \item Compute derivatives/integrals of vector functions (5 problems).
  \item Find velocity/acceleration; tangential/normal components.
  \item Curvature and osculating circle at selected points.
  \item Arc length and reparameterization by arc length.
  \item Reflection: unit vectors and interpretation checklist.
\end{checklist}
\end{storycard}
\clearpage

% ---------------------------- 13 -------------------------------------
\begin{storycard}{Ch. 13 --- Functions of Several Variables}
\begin{MetaTable}
  \MetaRow{Epic / Feature}{Multivariable Foundations}
  \MetaRow{Business Value}{Generalize limits, derivatives, and optimization in higher dimensions.}
  \MetaRow{Priority / Estimate}{Priority: Must \quad SP: 3}
  \MetaRow{Persona}{multivariable learner}
  \MetaRow{Dependencies}{Ch. 11}
  \MetaRow{Assumptions / Risks}{Incorrect limit conclusions from single-path checks.}
\end{MetaTable}

\medskip
\textbf{Story}\quad
\emph{As a learner, I want to compute partial and directional derivatives and use gradients so that I can linearize and optimize multivariable functions.}

\medskip
\textbf{Skills Emphasis}\quad
\pill{Limits/Cont} \pill{Partials} \pill{ChainRule} \pill{Grad/Extrema}

\medskip
\textbf{Acceptance Criteria (BDD)}
\begin{BDD}
  \MetaRow{Scenario}{Critical point classification}
  \MetaRow{Given}{a twice-differentiable function}
  \MetaRow{When}{I find critical points and analyze the Hessian}
  \MetaRow{Then}{Local minima/maxima/saddles are correctly classified, and constraints are handled via Lagrange multipliers.}
\end{BDD}

\footline{DoR: Examples chosen. \quad DoD: ACs pass; linear approximation examples documented.}

\medskip\hrule\medskip
\textbf{Tasks}
\begin{checklist}
  \item Compute multivariable limits with counterexamples.
  \item Partial derivatives and gradient fields for 6 functions.
  \item Multivariable chain rule exercises.
  \item Hessian-based classification; Lagrange multipliers on two problems.
  \item Reflection: checklist for limit proofs and pitfalls.
\end{checklist}
\end{storycard}
\clearpage

% ---------------------------- 14 -------------------------------------
\begin{storycard}{Ch. 14 --- Multiple Integration}
\begin{MetaTable}
  \MetaRow{Epic / Feature}{Volume and Mass}
  \MetaRow{Business Value}{Compute mass/volume/area and probabilities via double/triple integrals.}
  \MetaRow{Priority / Estimate}{Priority: Must \quad SP: 3}
  \MetaRow{Persona}{multivariable learner}
  \MetaRow{Dependencies}{Ch. 13}
  \MetaRow{Assumptions / Risks}{Incorrect region bounds; missed Jacobian in coordinate changes.}
\end{MetaTable}

\medskip
\textbf{Story}\quad
\emph{As a learner, I want to set up and evaluate double and triple integrals with coordinate changes so that I can compute geometric and physical quantities.}

\medskip
\textbf{Skills Emphasis}\quad
\pill{Iterated} \pill{Polar/Cyl/Sph} \pill{Jacobian} \pill{Apps}

\medskip
\textbf{Acceptance Criteria (BDD)}
\begin{BDD}
  \MetaRow{Scenario}{Region conversion}
  \MetaRow{Given}{a planar region suitable for polar coordinates}
  \MetaRow{When}{I convert the integral and include the Jacobian}
  \MetaRow{Then}{Results match the Cartesian computation and setup is justified.}
\end{BDD}

\footline{DoR: Region diagrams prepared. \quad DoD: ACs pass; method comparison table created.}

\medskip\hrule\medskip
\textbf{Tasks}
\begin{checklist}
  \item Set up and evaluate 4 double integrals over non-rectangular regions.
  \item Change order of integration on two examples.
  \item Convert to polar, cylindrical, spherical for appropriate problems.
  \item Compute center of mass or moment of inertia for a lamina/solid.
  \item Reflection: Jacobian mnemonic and common mistakes.
\end{checklist}
\end{storycard}
\clearpage

% ---------------------------- 15 -------------------------------------
\begin{storycard}{Ch. 15 --- Vector Analysis}
\begin{MetaTable}
  \MetaRow{Epic / Feature}{Field Integrals}
  \MetaRow{Business Value}{Use line/surface integrals and fundamental theorems (Green, Divergence, Stokes).}
  \MetaRow{Priority / Estimate}{Priority: Must \quad SP: 3}
  \MetaRow{Persona}{multivariable learner}
  \MetaRow{Dependencies}{Ch. 11, 13, 14}
  \MetaRow{Assumptions / Risks}{Orientation errors; confusing conservative fields conditions.}
\end{MetaTable}

\medskip
\textbf{Story}\quad
\emph{As a learner, I want to compute line and surface integrals and apply Green, Divergence, and Stokes so that I can relate local derivatives to global flux/circulation.}

\medskip
\textbf{Skills Emphasis}\quad
\pill{LineInt} \pill{SurfaceInt} \pill{Green} \pill{Stokes/Div}

\medskip
\textbf{Acceptance Criteria (BDD)}
\begin{BDD}
  \MetaRow{Scenario}{Theorem verification}
  \MetaRow{Given}{a vector field and suitable region/surface}
  \MetaRow{When}{I compute both sides of the relevant theorem}
  \MetaRow{Then}{The equality holds and orientation is explained.}
\end{BDD}

\footline{DoR: Fields and regions selected. \quad DoD: ACs pass; conservative field tests summarized.}

\medskip\hrule\medskip
\textbf{Tasks}
\begin{checklist}
  \item Determine whether a field is conservative; find potentials where possible.
  \item Evaluate line integrals directly and via Fundamental Theorem for line integrals.
  \item Apply Green's theorem on a planar region for circulation/flux.
  \item Compute surface integral and verify with Stokes or Divergence theorem.
  \item Reflection: orientation rules and right-hand conventions.
\end{checklist}
\end{storycard}
\clearpage

% ---------------------------- 16 -------------------------------------
\begin{storycard}{Ch. 16 --- Additional Topics in Differential Equations}
\begin{MetaTable}
  \MetaRow{Epic / Feature}{ODE Extensions}
  \MetaRow{Business Value}{Extend models with exact equations and second-order linear ODEs.}
  \MetaRow{Priority / Estimate}{Priority: Should \quad SP: 3}
  \MetaRow{Persona}{applied learner}
  \MetaRow{Dependencies}{Ch. 6}
  \MetaRow{Assumptions / Risks}{Characteristic equation sign errors; initial condition handling.}
\end{MetaTable}

\medskip
\textbf{Story}\quad
\emph{As a learner, I want to solve exact first-order and second-order linear ODEs so that I can model oscillations and forced systems.}

\medskip
\textbf{Skills Emphasis}\quad
\pill{Exact} \pill{2nd Order} \pill{Hom/Nonhom} \pill{SeriesIntro}

\medskip
\textbf{Acceptance Criteria (BDD)}
\begin{BDD}
  \MetaRow{Scenario}{Oscillator model}
  \MetaRow{Given}{parameters for mass-spring-damper systems}
  \MetaRow{When}{I solve the ODE for homogeneous and forced cases and apply initial conditions}
  \MetaRow{Then}{Solutions match expected qualitative behavior (over/critically/under-damped).}
\end{BDD}

\footline{DoR: Parameter sets chosen. \quad DoD: ACs pass; solution plots with phase interpretation provided.}

\medskip\hrule\medskip
\textbf{Tasks}
\begin{checklist}
  \item Identify exact equations and integrate factors where needed.
  \item Solve homogeneous constant-coefficient ODEs; classify damping.
  \item Solve nonhomogeneous with method of undetermined coefficients.
  \item Sketch phase portraits for representative parameter regimes.
  \item Reflection: checklist for initial condition application.
\end{checklist}
\end{storycard}
\clearpage

\end{document}
