\documentclass[11pt]{article}

% ---------- Encoding & Fonts ----------
\usepackage[T1]{fontenc}
\usepackage[utf8]{inputenc}
\usepackage{lmodern}
\IfFileExists{microtype.sty}{\usepackage{microtype}}{}

% ---------- Layout ----------
\usepackage[margin=1in]{geometry}
\usepackage{setspace}
\setstretch{1.08}

% ---------- Links ----------
\PassOptionsToPackage{hyphens}{url}
\usepackage[hidelinks]{hyperref}
\usepackage{url}

% ---------- Lists ----------
\usepackage{enumitem}
\setlist[itemize]{topsep=4pt,itemsep=2pt,parsep=0pt}
\setlist[enumerate]{topsep=4pt,itemsep=2pt,parsep=0pt}

% ---------- Title ----------
\title{\textbf{How to Invest in Certificates of Deposit (CDs)}}
\author{}
\date{\today}

\begin{document}
\maketitle

\begin{abstract}
To invest in a Certificate of Deposit (CD), you deposit a lump sum at a bank or credit union for a fixed period at a predetermined interest rate. Funds generally must remain until ``maturity,'' or early withdrawal penalties may apply \textit{[1, 2, 3, 4]}. This guide explains CD types, where to buy, how to open and manage an account, an advanced ladder strategy, and key pros and cons.
\end{abstract}

\section*{How to Invest in CDs}

\subsection*{1. Choose your CD type}
\begin{itemize}
  \item \textbf{Traditional:} Fixed rate for a set term; simple and predictable.
  \item \textbf{High-yield:} Often at online banks; typically higher rates.
  \item \textbf{No-penalty (``liquid''):} Allows early withdrawal without a fee; rates are usually a bit lower.
  \item \textbf{Jumbo:} Requires a large minimum (often \$100{,}000) for potentially higher rates.
  \item \textbf{Brokered:} Purchased via a brokerage; can offer higher rates and be sold on a secondary market before maturity; terms can be more complex.
  \item \textbf{IRA CDs:} Held inside an Individual Retirement Account for tax advantages \textit{[5, 6, 7, 8, 9]}.
\end{itemize}

\subsection*{2. Decide where to buy}
\begin{itemize}
  \item \textbf{Banks and credit unions:} Open directly (often online) and shop for the best rates across institutions.
  \item \textbf{Brokerage firms:} Platforms such as Fidelity or Charles Schwab list brokered CDs from many issuing banks so you can compare in one place \textit{[1, 10, 11, 12, 13]}.
\end{itemize}

\subsection*{3. Open and fund the account}
\begin{itemize}
  \item \textbf{Complete the application:} Usually online with identification and contact information.
  \item \textbf{Deposit funds:} Make a single lump-sum deposit. Minimums vary (e.g., \$500 for many traditional CDs; \$100{,}000 for jumbo CDs).
  \item \textbf{Choose interest disbursement:} Take interest periodically, compound it back into the CD, or receive it at maturity \textit{[3, 7, 14, 15, 16]}.
\end{itemize}

\subsection*{4. Manage your CD}
\begin{itemize}
  \item \textbf{Hold to maturity:} Principal and accrued interest are returned at the maturity date.
  \item \textbf{Renew or reinvest:} CDs may auto-renew. You can also cash out or select a new term \textit{[14, 17, 18, 19, 20]}.
\end{itemize}

\section*{Advanced Strategy: Building a CD Ladder}
A CD ladder balances higher long-term rates with short-term liquidity:
\begin{enumerate}
  \item Divide the total into equal parts (e.g., \$20{,}000 into four \$5{,}000 pieces).
  \item Buy multiple CDs with staggered maturities (e.g., 1-, 2-, 3-, and 4-year terms).
  \item As each CD matures, either cash out or roll into a new longest-term rung (e.g., another 4-year CD) to maintain the ladder \textit{[22, 23, 24, 25, 26]}.
\end{enumerate}

\section*{Pros and Cons of CDs}

\subsection*{Pros}
\begin{itemize}
  \item \textbf{Low risk:} Typically insured (FDIC for banks; NCUA for credit unions) up to applicable limits per depositor and ownership category.
  \item \textbf{Guaranteed returns:} Fixed rate for the term; earnings are predictable.
  \item \textbf{Often higher than savings:} Locking funds can yield higher rates versus standard savings accounts \textit{[27, 28, 29, 30, 31]}.
\end{itemize}

\subsection*{Cons}
\begin{itemize}
  \item \textbf{Early withdrawal penalties:} Taking funds before maturity usually triggers penalties that can reduce interest and potentially principal.
  \item \textbf{Lower returns than risk assets:} Generally less than typical long-run returns of stocks or certain bonds.
  \item \textbf{Interest rate risk:} If market rates rise after purchase, your rate is locked, creating opportunity cost \textit{[4, 32, 33, 34, 35]}.
\end{itemize}

\section*{Additional Notes (Helpful Context)}
\begin{itemize}
  \item \textbf{Insurance mechanics:} Bank CDs are insured by the FDIC; credit-union share certificates by the NCUA, generally up to \$250{,}000 per depositor, per insured institution, per ownership category.\footnote{See FDIC/NCUA guidance and your institution's disclosures.}
  \item \textbf{Auto-renew grace period:} Many institutions offer a short grace window (often about one week) after maturity to change or close the CD without penalty.
  \item \textbf{Tax treatment:} Outside an IRA, interest is typically taxed as ordinary income in the year credited. IRA CDs follow IRA tax rules.\footnote{See IRS Publication 550 and your tax advisor.}
  \item \textbf{Other variants:} Bump-up/step-up CDs (rate increase feature), add-on CDs (allow extra deposits), and callable CDs (issuer can redeem early; common with brokered CDs).
\end{itemize}

\section*{One-Minute Checklist}
\begin{itemize}
  \item Confirm FDIC/NCUA insurance coverage and your aggregate balances by ownership category.
  \item Compare APY (not just rate), compounding, minimums, and early withdrawal penalties.
  \item For brokered CDs, check call features and understand secondary-market price risk if selling early.
  \item Calendar the maturity date and grace-period window.
  \item Consider IRA CDs if tax deferral or Roth treatment is desired.
\end{itemize}

\section*{References}
% Your original sources [1]–[35]
\begin{enumerate}[label={[\arabic*]}]
  \item \url{https://www.fidelity.com/learning-center/smart-money/how-does-a-cd-work}
  \item \url{https://www.bankofamerica.com/deposits/bank-cds/cd-accounts/}
  \item \url{https://www.ebsco.com/research-starters/business-and-management/certificate-deposit-cd}
  \item \url{https://www.bankrate.com/banking/cds/the-pros-and-cons-of-cd-investing/}
  \item \url{https://www.usbank.com/bank-accounts/savings-accounts/certificate-of-deposit.html}
  \item \url{https://www.nerdwallet.com/article/banking/cd-certificate-of-deposit}
  \item \url{https://www.investopedia.com/certificate-of-deposits-4689733}
  \item \url{https://www.nerdwallet.com/article/banking/brokered-cds}
  \item \url{https://www.businessinsider.com/personal-finance/banking/what-are-brokered-cds}
  \item \url{https://www.nerdwallet.com/article/banking/how-to-open-a-cd}
  \item \url{https://www.investopedia.com/how-to-open-a-cd-5225191}
  \item \url{https://finance.yahoo.com/news/brokered-cds-161416234.html}
  \item \url{https://www.fidelity.com/learning-center/trading-investing/brokered-cd}
  \item \url{https://www.bankrate.com/banking/cds/how-to-open-a-cd/}
  \item \url{https://www.investopedia.com/how-to-open-a-cd-5225191}
  \item \url{https://www.nerdwallet.com/article/banking/how-to-open-a-cd}
  \item \url{https://www.wellsfargo.com/savings-cds/certificate-of-deposit/}
  \item \url{https://www.investopedia.com/what-can-cds-be-used-for-5235790}
  \item \url{https://www.pnc.com/insights/personal-finance/save/what-is-a-cd-ladder.html}
  \item \url{https://www.fidelity.com/learning-center/trading-investing/brokered-cd}
  \item \url{https://www.pnc.com/insights/personal-finance/save/what-is-a-cd-ladder.html}
  \item \url{https://www.bankrate.com/banking/cds/how-to-invest-in-cds/}
  \item \url{https://www.nerdwallet.com/article/banking/how-to-invest-in-cds}
  \item \url{https://www.lafcu.com/cdladdering}
  \item \url{https://www.cbsnews.com/news/easy-ways-to-earn-big-returns-after-your-cd-matures/}
  \item \url{https://www.investopedia.com/unsure-about-committing-to-cd-5-ways-to-make-it-work-7629433}
  \item \url{https://www.nerdwallet.com/article/banking/how-much-should-i-put-into-cds}
  \item \url{https://www.schwab.com/fixed-income/certificates-deposit}
  \item \url{https://www.synchrony.com/blog/bank/what-is-a-cd-account}
  \item \url{https://finance.yahoo.com/news/pros-cons-cd-investing-215643113.html}
  \item \url{https://www.jsb.bank/resources/comparing-a-certificate-of-deposit-to-other-investments-pros-and-cons}
  \item \url{https://www.fidelity.com/learning-center/smart-money/how-does-a-cd-work}
  \item \url{https://www.gainbridge.io/post/certificates-of-deposit}
  \item \url{https://www.fidelity.com/fixed-income-bonds/cds}
  \item \url{https://www.raisin.com/en-us/investing/how-to-invest-in-cds}
\end{enumerate}

\vspace{1ex}
\noindent\textit{Note:} This document is for educational purposes and does not constitute financial, tax, or investment advice.

\end{document}
