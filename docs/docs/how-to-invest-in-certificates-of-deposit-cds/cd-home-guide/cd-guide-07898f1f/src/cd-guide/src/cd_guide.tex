
\documentclass[11pt]{article}
\usepackage[T1]{fontenc}
\usepackage[utf8]{inputenc}
\usepackage[a4paper,margin=1in]{geometry}
\usepackage{lmodern}
\usepackage{microtype}
\usepackage{hyperref}
\usepackage{booktabs}
\usepackage{array}
\usepackage{amsmath, amssymb}
\hypersetup{colorlinks=true,linkcolor=black,urlcolor=blue}

\title{Build a 4-Year CD Ladder to Reach \$100{,}000}
\author{Practical Playbook}
\date{\today}

\begin{document}
\maketitle

\section*{Why a CD Ladder}
A certificate of deposit (CD) ladder splits your savings across multiple CD terms so that a portion of your money matures regularly. This approach aims to (1) lock in higher rates offered by longer terms while (2) preserving near-term access as each rung matures on a schedule. CDs from FDIC-insured banks or NCUA-insured credit unions are insured up to \$250{,}000 per depositor, per institution, per ownership category.\footnote{Verify coverage at your bank/credit union; brokered CDs settle at custodians but are still subject to issuer insurance limits.}

\section*{Target}
Save \$100{,}000 in 48 months. If you start from zero and contribute monthly, the required contribution depends on your average yield. As a planning shortcut, you can estimate the needed monthly deposit $C$ for an annual percentage yield (APY) $r$ with:
\begin{equation*}
C \approx \frac{FV \cdot \frac{r}{12}}{(1+\frac{r}{12})^{48}-1},
\end{equation*}
and for $r=0$ simply use \$100{,}000/48.

\section*{Choose a Ladder Structure}
There are two practical options. Pick one and stick to its roll-over rules.
\subsection*{Option A --- Quarterly 4-Rung Ladder (Liquidity First)}
Create rungs at 3, 6, 9, and 12 months. Every 3 months, one CD matures. Roll each maturity into a new 12-month CD. After the first year, you will always have a CD maturing every quarter while most of your balance earns a 12-month rate.
\begin{itemize}
  \item Best for: frequent access, nimble rate resets.
  \item How to start from zero: batch contributions for the first quarter, then open the 3-, 6-, 9-, and 12-month rungs in quick succession (or open smaller CDs as cash allows). From then on, always roll the maturing rung into 12 months.
\end{itemize}

\subsection*{Option B --- Goal-Date Ladder (Yield Maximized for Month 48)}
Stage rungs so they all mature no later than your goal date (month 48). For example, create 12-, 24-, 36-, and 48-month rungs and direct new money to the longest remaining term. In the final year, stop rolling and let all rungs mature into cash \emph{by} month 48.
\begin{itemize}
  \item Best for: a hard goal date with minimal reinvestment risk right before the goal.
\end{itemize}

\section*{Funding Plan (from Zero)}
\begin{enumerate}
  \item \textbf{Pick your ladder option} and the banks you will use (compare APY, minimums, early withdrawal penalties, and partial withdrawal rules).
  \item \textbf{Automate a monthly transfer} from checking to a high-yield savings account (HYSA). Use this as “staging” so you can open/roll CDs in tidy chunks (monthly or quarterly) instead of dozens of micro-CDs.
  \item \textbf{Open initial rungs} as soon as you’ve batched enough cash: either the 3/6/9/12-month set (Option A) or the 12/24/36/48-month set (Option B). Split funds evenly at first.
  \item \textbf{Roll and extend:} when a CD matures, in Option A roll it into a new 12-month CD; in Option B, roll into the longest term that still matures by month 48. Redirect fresh monthly contributions to the longest-earning rung.
  \item \textbf{Glide path into cash:} in the last 12 months, stop extending maturities beyond your goal date. Let the ladder collapse into cash right on schedule.
\end{enumerate}

\section*{How Much to Contribute Each Month}
Table~\ref{tab:contrib} shows the approximate monthly deposit needed to reach \$100{,}000 in 48 months at several average APYs, assuming monthly compounding on your aggregated balance (a planning approximation for a rolling ladder).

\begin{table}[h]
\centering
\begin{tabular}{>{\raggedright}p{3cm} >{\raggedleft}p{5cm} >{\raggedleft}p{5cm} >{\raggedleft\arraybackslash}p{4cm}}
\toprule
APY (annual) & Required Monthly Contribution & Total Contributed (48 mo) & Est.\ Interest Earned \\
\midrule
0.00\% & \$2{,}083.33 & \$100{,}000.00 & \$0.00 \\
2.00\% & \$2{,}038.36 & \$97{,}841.28 & \$2{,}158.72 \\
3.00\% & \$2{,}015.51 & \$96{,}744.48 & \$3{,}255.52 \\
4.00\% & \$1{,}993.63 & \$95{,}694.24 & \$4{,}305.76 \\
5.00\% & \$1{,}972.70 & \$94{,}689.60 & \$5{,}310.40 \\
\bottomrule
\end{tabular}
\caption{Monthly contribution estimates for a 48-month horizon.}
\label{tab:contrib}
\end{table}

\paragraph{Tip.} If your actual blended APY differs, recompute using the formula above or a spreadsheet. Overshoot by 5--10\% to cover rate slippage or months you might miss.

\section*{Example Roll Schedule (Option A)}
\textbf{Month 1:} Open 3-, 6-, 9-, and 12-month CDs, splitting the cash equally.\\
\textbf{Month 3:} The 3-month CD matures. Reinvest \emph{principal + interest} into a 12-month CD.\\
\textbf{Months 6/9/12:} Repeat. After month 12, you hold four 12-month CDs that mature every 3 months (months 15, 18, 21, 24...).\\
\textbf{Months 13--48:} Keep monthly contributions flowing to your HYSA and add them to the \emph{next} maturity when it rolls into a new 12-month CD.\\
\textbf{Final Year:} Stop extending past the goal date; let maturities fall to cash.

\section*{Bank Selection Checklist}
\begin{itemize}
  \item \textbf{APY and term menu:} look for competitive 6--48 month options; verify APY compounding basis.
  \item \textbf{Minimums and penalties:} know early withdrawal penalties (commonly 3--12 months of interest). Consider \emph{no-penalty CDs} for your shortest rung.
  \item \textbf{Partial withdrawals:} some banks allow partial early withdrawals---useful for tight cash needs.
  \item \textbf{Account limits:} confirm FDIC/NCUA coverage (\$250{,}000 per depositor, per institution, per ownership category). Spread across issuers if you will exceed limits.
  \item \textbf{Funding logistics:} ACH speed, mobile check limits, grace periods at maturity, auto-roll options, and beneficiary (POD/TOD) settings.
  \item \textbf{Brokered vs bank CDs:} brokered CDs add convenience for multiple issuers but watch call features and secondary-market price risk if you sell before maturity.
\end{itemize}

\section*{Tax Notes (U.S.)}
CD interest is typically taxed as ordinary income in the year it is credited (Form 1099-INT), even if you let it compound. In tax-advantaged accounts (IRA CDs, HSA CDs), taxes are deferred or avoided depending on account rules. Consult a professional for your situation.

\section*{Risk and Safeguards}
\begin{itemize}
  \item \textbf{Rate risk:} your ladder naturally re-prices as rungs mature. Short rungs provide upward-rate capture; long rungs protect if rates fall.
  \item \textbf{Liquidity:} your next maturity is your emergency valve. Keep 1--3 months of expenses in a liquid HYSA separate from the ladder.
  \item \textbf{Behavioral risk:} automate transfers, and set calendar reminders for maturity windows and grace periods.
\end{itemize}

\section*{Putting It All Together}
\begin{enumerate}
  \item Automate $\geq\$2{,}000$/month (see Table~\ref{tab:contrib}) to your HYSA.
  \item Open the initial rungs (Option A or B) and roll per the rules above.
  \item In months 37--48, stop extending past the goal date; let CDs mature to cash so the full \$100k is available on time.
\end{enumerate}

\bigskip
\noindent\textit{Planning heuristic:} If you can contribute \$2{,}000/month and your blended ladder APY averages near 4\%, you should reach \$100{,}000 in roughly four years (Table~\ref{tab:contrib}). If rates drop, increase the contribution by 5--10\%.

\end{document}
