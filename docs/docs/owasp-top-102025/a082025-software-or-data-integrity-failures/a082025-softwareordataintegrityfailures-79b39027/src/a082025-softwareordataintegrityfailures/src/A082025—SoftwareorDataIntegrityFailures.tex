% !TeX program = pdflatex
% If you prefer minted (Pygments) highlighting, see the optional minted setup
% near the bottom of this preamble.

\documentclass[11pt]{article}

% --------------------------
% Packages
% --------------------------
\usepackage[T1]{fontenc}
\usepackage[utf8]{inputenc}
\usepackage{lmodern}
\usepackage{microtype}
\usepackage{geometry}
\geometry{margin=1in}

\usepackage{booktabs}
\usepackage{longtable}
\usepackage{array}
\usepackage{xcolor}
\usepackage{enumitem}
\usepackage{hyperref}
\usepackage{cleveref}

\usepackage{tcolorbox}
\tcbuselibrary{breakable,skins}

\hypersetup{
  colorlinks=true,
  linkcolor=blue,
  urlcolor=blue,
  citecolor=blue,
  pdftitle={A08:2025 — Software or Data Integrity Failures},
  pdfauthor={},
  pdfsubject={Application Security},
  pdfkeywords={OWASP, Software Integrity, Data Integrity, CI/CD, Deserialization, Signing, Trust Boundaries, CWE, CVE}
}

% --------------------------
% Styling
% --------------------------
\setlist[itemize]{topsep=4pt, itemsep=2pt, leftmargin=1.25em}
\setlist[enumerate]{topsep=4pt, itemsep=2pt, leftmargin=1.5em}

\newtcolorbox{infobox}[1][]{
  breakable,
  colback=gray!5,
  colframe=gray!35,
  arc=2mm,
  left=3mm,right=3mm,top=2mm,bottom=2mm,
  title=#1,
  fonttitle=\bfseries
}

\newtcolorbox{controlbox}[1][]{
  breakable,
  colback=green!3,
  colframe=green!25,
  arc=2mm,
  left=3mm,right=3mm,top=2mm,bottom=2mm,
  title=#1,
  fonttitle=\bfseries
}

\newtcolorbox{scenario}[1][]{
  breakable,
  colback=blue!2,
  colframe=blue!20,
  arc=2mm,
  left=3mm,right=3mm,top=2mm,bottom=2mm,
  title=#1,
  fonttitle=\bfseries
}

% Optional: minted setup (requires -shell-escape)
% \usepackage{minted}
% \setminted{
%   fontsize=\small,
%   breaklines=true,
%   bgcolor=gray!10,
%   frame=lines
% }

% --------------------------
% Document
% --------------------------
\title{\textbf{A08:2025 — Software or Data Integrity Failures}}
\date{\today}

\begin{document}
\maketitle

\begin{infobox}[Document Summary]
This document consolidates the provided content for \textit{A08:2025 — Software or Data Integrity Failures} into a structured, print-ready reference, including background context, scoring metrics, description of integrity and trust-boundary failures (untrusted code/data treated as trusted), prevention guidance (signing, trusted repositories, review and segregation in CI/CD, and deserialization integrity protections), attack scenarios, references, and the mapped CWE list.
\end{infobox}

\tableofcontents
\newpage

\section{Background}
Software or Data Integrity Failures continues at \#8, with a slight, clarifying name change from ``Software and Data Integrity Failures.'' This category focuses on failures to maintain trust boundaries and verify the integrity of software, code, and data artifacts at a lower level than Software Supply Chain Failures.

This category emphasizes making assumptions related to software updates and critical data without verifying integrity.

Notable Common Weakness Enumerations (CWEs) include:
\begin{itemize}
  \item CWE-829: Inclusion of Functionality from Untrusted Control Sphere
  \item CWE-915: Improperly Controlled Modification of Dynamically-Determined Object Attributes
  \item CWE-502: Deserialization of Untrusted Data
\end{itemize}

\section{Score Table}
\begin{table}[h!]
\centering
\renewcommand{\arraystretch}{1.2}
\begin{tabular}{@{}>{\bfseries}p{4.9cm}p{3.1cm}@{}}
\toprule
Metric & Value \\
\midrule
CWEs Mapped & 14 \\
Max Incidence Rate & 8.98\% \\
Avg Incidence Rate & 2.75\% \\
Max Coverage & 78.52\% \\
Avg Coverage & 45.49\% \\
Avg Weighted Exploit & 7.11 \\
Avg Weighted Impact & 4.79 \\
Total Occurrences & 501{,}327 \\
Total CVEs & 3{,}331 \\
\bottomrule
\end{tabular}
\caption{Provided scoring summary for Software or Data Integrity Failures.}
\end{table}

\section{Description}
Software and data integrity failures relate to code and infrastructure that do not protect against invalid or untrusted code or data being treated as trusted and valid.

Examples include:
\begin{itemize}
  \item Applications that rely upon plugins, libraries, or modules from untrusted sources, repositories, and content delivery networks (CDNs).
  \item Insecure CI/CD pipelines that do not consume and provide software integrity checks, introducing the potential for unauthorized access, insecure or malicious code, or system compromise.
  \item CI/CD processes that pull code or artifacts from untrusted locations and/or do not verify them before use (e.g., by checking signatures or similar mechanisms).
  \item Auto-update functionality where updates are downloaded and applied without sufficient integrity verification; attackers could potentially upload malicious updates to be distributed widely.
  \item Insecure deserialization, where objects or state are serialized into attacker-visible and modifiable structures and then used without appropriate integrity and trust validation.
\end{itemize}

\section{How to Prevent}

\begin{controlbox}[Integrity and Trust Controls]
\begin{enumerate}
  \item Use digital signatures or similar mechanisms to verify software or data is from the expected source and has not been altered.
  \item Ensure libraries and dependencies (e.g., npm, Maven) are obtained only from trusted repositories. For higher risk profiles, consider hosting an internal known-good repository that is vetted.
  \item Ensure there is a review process for code and configuration changes to reduce the likelihood that malicious code or configuration is introduced into the software pipeline.
  \item Ensure the CI/CD pipeline has proper segregation, configuration, and access control to preserve integrity of the code flowing through build and deploy processes.
  \item Ensure unsigned or unencrypted serialized data is not received from untrusted clients and used without integrity checks or digital signatures to detect tampering or replay.
\end{enumerate}
\end{controlbox}

\section{Example Attack Scenarios}

\begin{scenario}[Scenario \#1: Inclusion of Web Functionality from an Untrusted Source]
A company uses an external service provider to provide support functionality. For convenience, it sets a DNS mapping for \texttt{myCompany.SupportProvider.com} to \texttt{support.myCompany.com}. This causes all cookies set on the \texttt{myCompany.com} domain (including authentication cookies) to be sent to the support provider.

Anyone with access to the support provider’s infrastructure can steal cookies for users who visit \texttt{support.myCompany.com} and perform session hijacking.
\end{scenario}

\begin{scenario}[Scenario \#2: Updates Without Signing]
Many home routers, set-top boxes, device firmware, and similar systems do not verify updates via signed firmware. Unsigned firmware is a growing target for attackers. This is a major concern because there may be no remediation path other than fixing a future version and waiting for previous versions to age out.
\end{scenario}

\begin{scenario}[Scenario \#3: Package from an Untrusted Source]
A developer cannot find an updated version of a package in the usual trusted package manager, so they download it from a random website. The package is not signed, leaving no mechanism to verify integrity. The package contains malicious code.
\end{scenario}

\begin{scenario}[Scenario \#4: Insecure Deserialization]
A React application calls a set of Spring Boot microservices. To enforce immutability, the system serializes user state and passes it back and forth with each request. An attacker notices the \texttt{rO0} Java object signature (in base64) and uses a Java Deserialization Scanner to gain remote code execution on the application server.
\end{scenario}

\section{References}
\begin{itemize}
  \item OWASP Cheat Sheet: Software Supply Chain Security
  \item OWASP Cheat Sheet: Infrastructure as Code
  \item OWASP Cheat Sheet: Deserialization
  \item SAFECode Software Integrity Controls
  \item A ``Worst Nightmare'' Cyberattack: The Untold Story Of The SolarWinds Hack
  \item CodeCov Bash Uploader Compromise
  \item \textit{Securing DevOps} by Julien Vehent
  \item \textit{Insecure Deserialization} by Tenendo
\end{itemize}

\newpage
\section{List of Mapped CWEs}
\renewcommand{\arraystretch}{1.1}
\begin{longtable}{@{}p{2.4cm}p{12.0cm}@{}}
\toprule
\textbf{CWE} & \textbf{Title} \\
\midrule
\endfirsthead
\toprule
\textbf{CWE} & \textbf{Title} \\
\midrule
\endhead
\bottomrule
\endfoot

CWE-345 & Insufficient Verification of Data Authenticity \\
CWE-353 & Missing Support for Integrity Check \\
CWE-426 & Untrusted Search Path \\
CWE-427 & Uncontrolled Search Path Element \\
CWE-494 & Download of Code Without Integrity Check \\
CWE-502 & Deserialization of Untrusted Data \\
CWE-506 & Embedded Malicious Code \\
CWE-509 & Replicating Malicious Code (Virus or Worm) \\
CWE-565 & Reliance on Cookies without Validation and Integrity Checking \\
CWE-784 & Reliance on Cookies without Validation and Integrity Checking in a Security Decision \\
CWE-829 & Inclusion of Functionality from Untrusted Control Sphere \\
CWE-830 & Inclusion of Web Functionality from an Untrusted Source \\
CWE-915 & Improperly Controlled Modification of Dynamically-Determined Object Attributes \\
CWE-926 & Improper Export of Android Application Components \\

\end{longtable}

\end{document}

