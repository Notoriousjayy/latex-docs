
\documentclass[11pt,a4paper]{article}
\usepackage[a4paper,margin=1in]{geometry}
\usepackage{lmodern}
\usepackage[T1]{fontenc}
\usepackage[utf8]{inputenc}
\usepackage{microtype}
\usepackage{parskip}
\usepackage{enumitem}
\setlist{itemsep=2pt, topsep=4pt, leftmargin=1.2em}
\usepackage{titlesec}
\titlespacing*{\section}{0pt}{8pt plus 2pt}{4pt}
\titlespacing*{\subsection}{0pt}{6pt}{3pt}
\usepackage[dvipsnames]{xcolor}
\usepackage{hyperref}
\hypersetup{colorlinks=true,linkcolor=MidnightBlue,urlcolor=MidnightBlue,citecolor=MidnightBlue}
\usepackage[most]{tcolorbox}
\usepackage{tabularx}
\usepackage{array}
\usepackage{amssymb}
\usepackage{pifont}
\usepackage{graphicx}
\usepackage{ifthen}
\usepackage{fancyvrb}
\definecolor{CardFrame}{RGB}{33,37,41}
\definecolor{CardBack}{RGB}{248,249,250}
\definecolor{PillBack}{RGB}{235,238,240}
\definecolor{PillBorder}{RGB}{205,208,210}
\definecolor{GoodGreen}{RGB}{46,125,50}
\newcommand{\Pill}[1]{\tcbox[colback=PillBack,colframe=PillBorder,boxrule=0.4pt,arc=3pt,left=3pt,right=3pt,top=1.5pt,bottom=1.5pt,boxsep=0pt]{\small #1}}
\newcommand{\TaskItem}[1]{\item $\square$~#1}
\newcommand{\Yes}{\textcolor{GoodGreen}{\ding{51}}}
\newcommand{\No}{\textcolor{red}{\ding{55}}}
\newcommand{\StoryID}{}\newcommand{\setStoryID}[1]{\renewcommand{\StoryID}{#1}}
\newcommand{\StoryTitle}{}\newcommand{\setStoryTitle}[1]{\renewcommand{\StoryTitle}{#1}}
\newcommand{\Epic}{}\newcommand{\setEpic}[1]{\renewcommand{\Epic}{#1}}
\newcommand{\BusinessValue}{}\newcommand{\setBusinessValue}[1]{\renewcommand{\BusinessValue}{#1}}
\newcommand{\PriorityLevel}{}\newcommand{\setPriority}[1]{\renewcommand{\PriorityLevel}{#1}}
\newcommand{\StoryPoints}{}\newcommand{\setSP}[1]{\renewcommand{\StoryPoints}{#1}}
\newcommand{\Persona}{}\newcommand{\setPersona}[1]{\renewcommand{\Persona}{#1}}
\newcommand{\Dependencies}{}\newcommand{\setDependencies}[1]{\renewcommand{\Dependencies}{#1}}
\newcommand{\Assumptions}{}\newcommand{\setAssumptions}[1]{\renewcommand{\Assumptions}{#1}}
\newcommand{\UserStory}{}\newcommand{\setUserStory}[1]{\renewcommand{\UserStory}{#1}}
\newcommand{\NonFunctional}{}\newcommand{\setNonFunctional}[1]{\renewcommand{\NonFunctional}{#1}}
\newcommand{\ScenarioName}{}\newcommand{\setScenario}[1]{\renewcommand{\ScenarioName}{#1}}
\newcommand{\GivenText}{}\newcommand{\setGiven}[1]{\renewcommand{\GivenText}{#1}}
\newcommand{\WhenText}{}\newcommand{\setWhen}[1]{\renewcommand{\WhenText}{#1}}
\newcommand{\ThenText}{}\newcommand{\setThen}[1]{\renewcommand{\ThenText}{#1}}
\newcommand{\DefinitionOfReady}{}\newcommand{\setDoR}[1]{\renewcommand{\DefinitionOfReady}{#1}}
\newcommand{\DefinitionOfDone}{}\newcommand{\setDoD}[1]{\renewcommand{\DefinitionOfDone}{#1}}
\makeatletter
\newcommand{\RenderPills}[1]{%
  \begingroup
  % Iterate over a comma-separated list without using \item
  \@for\x:=#1\do{%
    \expandafter\Pill\expandafter{\x}\hspace{3pt}%
  }%
  \endgroup
}
\makeatother
\tcbset{cardstyle/.style={colframe=CardFrame,colback=CardBack,boxrule=0.6pt,arc=2mm,left=6pt,right=6pt,top=6pt,bottom=6pt},header/.style={fontupper=\bfseries\large}}
\newenvironment{StoryCardBox}[1]{\begin{tcolorbox}[cardstyle,title={#1},fonttitle=\bfseries\large]}{\end{tcolorbox}}
\newcommand{\RenderStoryCard}{%
\begin{StoryCardBox}{\StoryID{} --- \StoryTitle}
\renewcommand{\arraystretch}{1.2}
\begin{tabularx}{\linewidth}{>{\raggedright\arraybackslash}p{0.24\linewidth} X}
\textbf{Epic / Feature} & \Epic\\
\textbf{Business Value} & \BusinessValue\\
\textbf{Priority / Estimate} & \textbf{Priority:}~\PriorityLevel\quad \textbf{SP:}~\StoryPoints\\
\textbf{Persona} & \Persona\\
\textbf{Dependencies} & \Dependencies\\
\textbf{Assumptions / Risks} & \Assumptions
\end{tabularx}
\vspace{6pt}
\textbf{Story}\quad \emph{\UserStory}
\vspace{4pt}
\textbf{Non-Functional}\quad \RenderPills{\NonFunctional}
\vspace{6pt}
\textbf{Acceptance Criteria (BDD)}
\begin{description}[leftmargin=2.2em,style=nextline]
  \item[Scenario] \ScenarioName
  \item[Given] \GivenText
  \item[When] \WhenText
  \item[Then] \ThenText
\end{description}
{\footnotesize\textit{Definition of Ready:} \DefinitionOfReady{}\quad\textbullet\quad
\textit{Definition of Done:} \DefinitionOfDone{}}
\end{StoryCardBox}}
\newenvironment{TasksBox}{\begin{tcolorbox}[cardstyle,title={Tasks},fonttitle=\bfseries]}{\end{tcolorbox}}
\begin{document}
\begin{center}
{\LARGE \textbf{Computer and Information Security Handbook (4e)}\\[2pt]
\large Full Study Plan as User Story Cards}\\[6pt]
\normalsize Each chapter is represented as a story card with BDD acceptance criteria, DoR/DoD, and a concrete task list.
\end{center}
\clearpage

\begin{center}
\includegraphics[width=0.28\textwidth]{advatar.jpeg}
\end{center}

\clearpage
\setStoryID{CISH-001}
\setStoryTitle{Information Security in the Modern Enterprise --- Learn \& Lab}
\setEpic{Part 1: Overview of System and Network Security}
\setBusinessValue{Build a working understanding of information security in the modern enterprise and its place in a modern security program; be able to explain core concepts, map them to the CIA triad, and identify common threats and controls.}
\setPriority{Must}
\setSP{3}
\setPersona{Security Engineer}
\setDependencies{Lab VM or container runtime, Git repo for notes, Markdown/PDF export tool}
\setAssumptions{Time-box chapter to one iteration; open issues captured for later}
\setUserStory{As a Security Engineer, I want to study and practice 'Information Security in the Modern Enterprise' so that I can apply its concepts to reduce risk and improve outcomes.}
\setNonFunctional{Security, Reliability, Performance}
\setScenario{Apply key controls for Information Security in the Modern Enterprise}
\setGiven{the lab environment and topic-specific tools for 'Information Security in the Modern Enterprise' are available}
\setWhen{I execute the hands-on objectives and lab: Hands-on: Build a small lab demonstrating key concepts in 'Information Security in the Modern Enterprise'. Capture screenshots/notes and one measurable result (e.g., a passing test, alert fired, or control verified).}
\setThen{the deliverables are produced (2–3 page brief on 'Information Security in the Modern Enterprise': risks, architecture, controls, and a checklist; plus a one-slide executive summary.); evidence (screenshots/logs/configs) is attached and reviewed}
\setDoR{Persona clear; AC drafted; Dependencies known; Estimate set.}
\setDoD{All ACs pass; Tests green; Security checks; Docs updated; Evidence attached.}
\RenderStoryCard
\begin{TasksBox}
\begin{itemize}
\TaskItem{Draft a one-page chapter plan: scope, objectives, interfaces, success metrics.}
\TaskItem{Set up tools, datasets, and accounts; document versions and configuration.}
\TaskItem{Complete objective: Define key terms and articulate why this topic matters to security outcomes.}
\TaskItem{Complete objective: Diagram the architecture/data flows and identify threat surfaces.}
\TaskItem{Execute lab: Hands-on: Build a small lab demonstrating key concepts in 'Information Security in the Modern Enterprise'. Capture screenshots/notes and one measurable result (e.g., a passing test, alert fired, or control verified).}
\end{itemize}
\end{TasksBox}
\clearpage
\setStoryID{CISH-002}
\setStoryTitle{Building a Secure Organization --- Learn \& Lab}
\setEpic{Part 1: Overview of System and Network Security}
\setBusinessValue{Build a working understanding of building a secure organization and its place in a modern security program; be able to explain core concepts, map them to the CIA triad, and identify common threats and controls.}
\setPriority{Must}
\setSP{3}
\setPersona{Security Program Manager}
\setDependencies{Lab VM or container runtime, Git repo for notes, Markdown/PDF export tool}
\setAssumptions{Time-box chapter to one iteration; open issues captured for later}
\setUserStory{As a Security Program Manager, I want to study and practice 'Building a Secure Organization' so that I can apply its concepts to reduce risk and improve outcomes.}
\setNonFunctional{Security, Reliability, Performance}
\setScenario{Apply key controls for Building a Secure Organization}
\setGiven{the lab environment and topic-specific tools for 'Building a Secure Organization' are available}
\setWhen{I execute the hands-on objectives and lab: Hands-on: Build a small lab demonstrating key concepts in 'Building a Secure Organization'. Capture screenshots/notes and one measurable result (e.g., a passing test, alert fired, or control verified).}
\setThen{the deliverables are produced (2–3 page brief on 'Building a Secure Organization': risks, architecture, controls, and a checklist; plus a one-slide executive summary.); evidence (screenshots/logs/configs) is attached and reviewed}
\setDoR{Persona clear; AC drafted; Dependencies known; Estimate set.}
\setDoD{All ACs pass; Tests green; Security checks; Docs updated; Evidence attached.}
\RenderStoryCard
\begin{TasksBox}
\begin{itemize}
\TaskItem{Draft a one-page chapter plan: scope, objectives, interfaces, success metrics.}
\TaskItem{Set up tools, datasets, and accounts; document versions and configuration.}
\TaskItem{Complete objective: Define key terms and articulate why this topic matters to security outcomes.}
\TaskItem{Complete objective: Diagram the architecture/data flows and identify threat surfaces.}
\TaskItem{Execute lab: Hands-on: Build a small lab demonstrating key concepts in 'Building a Secure Organization'. Capture screenshots/notes and one measurable result (e.g., a passing test, alert fired, or control verified).}
\end{itemize}
\end{TasksBox}
\clearpage
\setStoryID{CISH-003}
\setStoryTitle{A Cryptography Primer --- Learn \& Lab}
\setEpic{Part 1: Overview of System and Network Security}
\setBusinessValue{Build a working understanding of a cryptography primer and its place in a modern security program; be able to explain core concepts, map them to the CIA triad, and identify common threats and controls.}
\setPriority{Must}
\setSP{3}
\setPersona{Platform Engineer}
\setDependencies{Lab VM or container runtime, Git repo for notes, Markdown/PDF export tool, OpenSSL/mkcert, TLS scanner}
\setAssumptions{Time-box chapter to one iteration; open issues captured for later}
\setUserStory{As a Platform Engineer, I want to study and practice 'A Cryptography Primer' so that I can apply its concepts to reduce risk and improve outcomes.}
\setNonFunctional{Security, Reliability, Performance}
\setScenario{Apply key controls for A Cryptography Primer}
\setGiven{the lab environment and topic-specific tools for 'A Cryptography Primer' are available}
\setWhen{I execute the hands-on objectives and lab: Hands-on: Build a small lab demonstrating key concepts in 'A Cryptography Primer'. Capture screenshots/notes and one measurable result (e.g., a passing test, alert fired, or control verified).}
\setThen{the deliverables are produced (2–3 page brief on 'A Cryptography Primer': risks, architecture, controls, and a checklist; plus a one-slide executive summary.); evidence (screenshots/logs/configs) is attached and reviewed}
\setDoR{Persona clear; AC drafted; Dependencies known; Estimate set.}
\setDoD{All ACs pass; Tests green; Security checks; Docs updated; Evidence attached.}
\RenderStoryCard
\begin{TasksBox}
\begin{itemize}
\TaskItem{Draft a one-page chapter plan: scope, objectives, interfaces, success metrics.}
\TaskItem{Set up tools, datasets, and accounts; document versions and configuration.}
\TaskItem{Complete objective: Define key terms and articulate why this topic matters to security outcomes.}
\TaskItem{Complete objective: Diagram the architecture/data flows and identify threat surfaces.}
\TaskItem{Execute lab: Hands-on: Build a small lab demonstrating key concepts in 'A Cryptography Primer'. Capture screenshots/notes and one measurable result (e.g., a passing test, alert fired, or control verified).}
\end{itemize}
\end{TasksBox}
\clearpage
\setStoryID{CISH-004}
\setStoryTitle{Verifying User and Host Identity --- Learn \& Lab}
\setEpic{Part 1: Overview of System and Network Security}
\setBusinessValue{Build a working understanding of verifying user and host identity and its place in a modern security program; be able to explain core concepts, map them to the CIA triad, and identify common threats and controls.}
\setPriority{Must}
\setSP{3}
\setPersona{Security Engineer}
\setDependencies{Lab VM or container runtime, Git repo for notes, Markdown/PDF export tool, IDP / MFA-capable test app}
\setAssumptions{Time-box chapter to one iteration; open issues captured for later}
\setUserStory{As a Security Engineer, I want to study and practice 'Verifying User and Host Identity' so that I can apply its concepts to reduce risk and improve outcomes.}
\setNonFunctional{Security, Reliability, Performance, Privacy}
\setScenario{Apply key controls for Verifying User and Host Identity}
\setGiven{the lab environment and topic-specific tools for 'Verifying User and Host Identity' are available}
\setWhen{I execute the hands-on objectives and lab: Hands-on: Build a small lab demonstrating key concepts in 'Verifying User and Host Identity'. Capture screenshots/notes and one measurable result (e.g., a passing test, alert fired, or control verified).}
\setThen{the deliverables are produced (2–3 page brief on 'Verifying User and Host Identity': risks, architecture, controls, and a checklist; plus a one-slide executive summary.); evidence (screenshots/logs/configs) is attached and reviewed}
\setDoR{Persona clear; AC drafted; Dependencies known; Estimate set.}
\setDoD{All ACs pass; Tests green; Security checks; Docs updated; Evidence attached.}
\RenderStoryCard
\begin{TasksBox}
\begin{itemize}
\TaskItem{Draft a one-page chapter plan: scope, objectives, interfaces, success metrics.}
\TaskItem{Set up tools, datasets, and accounts; document versions and configuration.}
\TaskItem{Complete objective: Define key terms and articulate why this topic matters to security outcomes.}
\TaskItem{Complete objective: Diagram the architecture/data flows and identify threat surfaces.}
\TaskItem{Execute lab: Hands-on: Build a small lab demonstrating key concepts in 'Verifying User and Host Identity'. Capture screenshots/notes and one measurable result (e.g., a passing test, alert fired, or control verified).}
\end{itemize}
\end{TasksBox}
\clearpage
\setStoryID{CISH-005}
\setStoryTitle{Detecting System Intrusions --- Learn \& Lab}
\setEpic{Part 1: Overview of System and Network Security}
\setBusinessValue{Build a working understanding of detecting system intrusions and its place in a modern security program; be able to explain core concepts, map them to the CIA triad, and identify common threats and controls.}
\setPriority{Must}
\setSP{3}
\setPersona{Security Engineer}
\setDependencies{Lab VM or container runtime, Git repo for notes, Markdown/PDF export tool}
\setAssumptions{Time-box chapter to one iteration; open issues captured for later}
\setUserStory{As a Security Engineer, I want to study and practice 'Detecting System Intrusions' so that I can apply its concepts to reduce risk and improve outcomes.}
\setNonFunctional{Security, Reliability, Performance, Observability}
\setScenario{Apply key controls for Detecting System Intrusions}
\setGiven{the lab environment and topic-specific tools for 'Detecting System Intrusions' are available}
\setWhen{I execute the hands-on objectives and lab: Hands-on: Build a small lab demonstrating key concepts in 'Detecting System Intrusions'. Capture screenshots/notes and one measurable result (e.g., a passing test, alert fired, or control verified).}
\setThen{the deliverables are produced (2–3 page brief on 'Detecting System Intrusions': risks, architecture, controls, and a checklist; plus a one-slide executive summary.); evidence (screenshots/logs/configs) is attached and reviewed}
\setDoR{Persona clear; AC drafted; Dependencies known; Estimate set.}
\setDoD{All ACs pass; Tests green; Security checks; Docs updated; Evidence attached.}
\RenderStoryCard
\begin{TasksBox}
\begin{itemize}
\TaskItem{Draft a one-page chapter plan: scope, objectives, interfaces, success metrics.}
\TaskItem{Set up tools, datasets, and accounts; document versions and configuration.}
\TaskItem{Complete objective: Define key terms and articulate why this topic matters to security outcomes.}
\TaskItem{Complete objective: Diagram the architecture/data flows and identify threat surfaces.}
\TaskItem{Execute lab: Hands-on: Build a small lab demonstrating key concepts in 'Detecting System Intrusions'. Capture screenshots/notes and one measurable result (e.g., a passing test, alert fired, or control verified).}
\end{itemize}
\end{TasksBox}
\clearpage
\setStoryID{CISH-006}
\setStoryTitle{Intrusion Detection in Contemporary Environments --- Learn \& Lab}
\setEpic{Part 1: Overview of System and Network Security}
\setBusinessValue{Build a working understanding of intrusion detection in contemporary environments and its place in a modern security program; be able to explain core concepts, map them to the CIA triad, and identify common threats and controls.}
\setPriority{Must}
\setSP{3}
\setPersona{Security Engineer}
\setDependencies{Lab VM or container runtime, Git repo for notes, Markdown/PDF export tool}
\setAssumptions{Time-box chapter to one iteration; open issues captured for later}
\setUserStory{As a Security Engineer, I want to study and practice 'Intrusion Detection in Contemporary Environments' so that I can apply its concepts to reduce risk and improve outcomes.}
\setNonFunctional{Security, Reliability, Performance, Observability}
\setScenario{Apply key controls for Intrusion Detection in Contemporary Environments}
\setGiven{the lab environment and topic-specific tools for 'Intrusion Detection in Contemporary Environments' are available}
\setWhen{I execute the hands-on objectives and lab: Hands-on: Build a small lab demonstrating key concepts in 'Intrusion Detection in Contemporary Environments'. Capture screenshots/notes and one measurable result (e.g., a passing test, alert fired, or control verified).}
\setThen{the deliverables are produced (2–3 page brief on 'Intrusion Detection in Contemporary Environments': risks, architecture, controls, and a checklist; plus a one-slide executive summary.); evidence (screenshots/logs/configs) is attached and reviewed}
\setDoR{Persona clear; AC drafted; Dependencies known; Estimate set.}
\setDoD{All ACs pass; Tests green; Security checks; Docs updated; Evidence attached.}
\RenderStoryCard
\begin{TasksBox}
\begin{itemize}
\TaskItem{Draft a one-page chapter plan: scope, objectives, interfaces, success metrics.}
\TaskItem{Set up tools, datasets, and accounts; document versions and configuration.}
\TaskItem{Complete objective: Define key terms and articulate why this topic matters to security outcomes.}
\TaskItem{Complete objective: Diagram the architecture/data flows and identify threat surfaces.}
\TaskItem{Execute lab: Hands-on: Build a small lab demonstrating key concepts in 'Intrusion Detection in Contemporary Environments'. Capture screenshots/notes and one measurable result (e.g., a passing test, alert fired, or control verified).}
\end{itemize}
\end{TasksBox}
\clearpage
\setStoryID{CISH-007}
\setStoryTitle{Preventing System Intrusions --- Learn \& Lab}
\setEpic{Part 1: Overview of System and Network Security}
\setBusinessValue{Build a working understanding of preventing system intrusions and its place in a modern security program; be able to explain core concepts, map them to the CIA triad, and identify common threats and controls.}
\setPriority{Must}
\setSP{3}
\setPersona{Security Engineer}
\setDependencies{Lab VM or container runtime, Git repo for notes, Markdown/PDF export tool}
\setAssumptions{Time-box chapter to one iteration; open issues captured for later}
\setUserStory{As a Security Engineer, I want to study and practice 'Preventing System Intrusions' so that I can apply its concepts to reduce risk and improve outcomes.}
\setNonFunctional{Security, Reliability, Performance}
\setScenario{Apply key controls for Preventing System Intrusions}
\setGiven{the lab environment and topic-specific tools for 'Preventing System Intrusions' are available}
\setWhen{I execute the hands-on objectives and lab: Hands-on: Build a small lab demonstrating key concepts in 'Preventing System Intrusions'. Capture screenshots/notes and one measurable result (e.g., a passing test, alert fired, or control verified).}
\setThen{the deliverables are produced (2–3 page brief on 'Preventing System Intrusions': risks, architecture, controls, and a checklist; plus a one-slide executive summary.); evidence (screenshots/logs/configs) is attached and reviewed}
\setDoR{Persona clear; AC drafted; Dependencies known; Estimate set.}
\setDoD{All ACs pass; Tests green; Security checks; Docs updated; Evidence attached.}
\RenderStoryCard
\begin{TasksBox}
\begin{itemize}
\TaskItem{Draft a one-page chapter plan: scope, objectives, interfaces, success metrics.}
\TaskItem{Set up tools, datasets, and accounts; document versions and configuration.}
\TaskItem{Complete objective: Define key terms and articulate why this topic matters to security outcomes.}
\TaskItem{Complete objective: Diagram the architecture/data flows and identify threat surfaces.}
\TaskItem{Execute lab: Hands-on: Build a small lab demonstrating key concepts in 'Preventing System Intrusions'. Capture screenshots/notes and one measurable result (e.g., a passing test, alert fired, or control verified).}
\end{itemize}
\end{TasksBox}
\clearpage
\setStoryID{CISH-008}
\setStoryTitle{Guarding Against Network Intrusions --- Learn \& Lab}
\setEpic{Part 1: Overview of System and Network Security}
\setBusinessValue{Build a working understanding of guarding against network intrusions and its place in a modern security program; be able to explain core concepts, map them to the CIA triad, and identify common threats and controls.}
\setPriority{Must}
\setSP{3}
\setPersona{Network Security Engineer}
\setDependencies{Lab VM or container runtime, Git repo for notes, Markdown/PDF export tool, Packet capture tool (tcpdump/Wireshark), Firewall/router lab}
\setAssumptions{Time-box chapter to one iteration; open issues captured for later}
\setUserStory{As a Network Security Engineer, I want to study and practice 'Guarding Against Network Intrusions' so that I can apply its concepts to reduce risk and improve outcomes.}
\setNonFunctional{Security, Reliability, Performance}
\setScenario{Apply key controls for Guarding Against Network Intrusions}
\setGiven{the lab environment and topic-specific tools for 'Guarding Against Network Intrusions' are available}
\setWhen{I execute the hands-on objectives and lab: Hands-on: Build a small lab demonstrating key concepts in 'Guarding Against Network Intrusions'. Capture screenshots/notes and one measurable result (e.g., a passing test, alert fired, or control verified).}
\setThen{the deliverables are produced (2–3 page brief on 'Guarding Against Network Intrusions': risks, architecture, controls, and a checklist; plus a one-slide executive summary.); evidence (screenshots/logs/configs) is attached and reviewed}
\setDoR{Persona clear; AC drafted; Dependencies known; Estimate set.}
\setDoD{All ACs pass; Tests green; Security checks; Docs updated; Evidence attached.}
\RenderStoryCard
\begin{TasksBox}
\begin{itemize}
\TaskItem{Draft a one-page chapter plan: scope, objectives, interfaces, success metrics.}
\TaskItem{Set up tools, datasets, and accounts; document versions and configuration.}
\TaskItem{Complete objective: Define key terms and articulate why this topic matters to security outcomes.}
\TaskItem{Complete objective: Diagram the architecture/data flows and identify threat surfaces.}
\TaskItem{Execute lab: Hands-on: Build a small lab demonstrating key concepts in 'Guarding Against Network Intrusions'. Capture screenshots/notes and one measurable result (e.g., a passing test, alert fired, or control verified).}
\end{itemize}
\end{TasksBox}
\clearpage
\setStoryID{CISH-009}
\setStoryTitle{Fault Tolerance and Resilience in Cloud Computing Environments --- Learn \& Lab}
\setEpic{Part 1: Overview of System and Network Security}
\setBusinessValue{Build a working understanding of fault tolerance and resilience in cloud computing environments and its place in a modern security program; be able to explain core concepts, map them to the CIA triad, and identify common threats and controls.}
\setPriority{Must}
\setSP{3}
\setPersona{Cloud Security Engineer}
\setDependencies{Lab VM or container runtime, Git repo for notes, Markdown/PDF export tool, Cloud sandbox account, Terraform, Benchmark tool (e.g., CIS)}
\setAssumptions{Sandbox-only changes; no production accounts; Time-box chapter to one iteration; open issues captured for later}
\setUserStory{As a Cloud Security Engineer, I want to study and practice 'Fault Tolerance and Resilience in Cloud Computing Environments' so that I can apply its concepts to reduce risk and improve outcomes.}
\setNonFunctional{Security, Reliability, Performance, Compliance}
\setScenario{Apply key controls for Fault Tolerance and Resilience in Cloud Computing Environments}
\setGiven{the lab environment and topic-specific tools for 'Fault Tolerance and Resilience in Cloud Computing Environments' are available}
\setWhen{I execute the hands-on objectives and lab: Hands-on: Build a small lab demonstrating key concepts in 'Fault Tolerance and Resilience in Cloud Computing Environments'. Capture screenshots/notes and one measurable result (e.g., a passing test, alert fired, or control verified).}
\setThen{the deliverables are produced (2–3 page brief on 'Fault Tolerance and Resilience in Cloud Computing Environments': risks, architecture, controls, and a checklist; plus a one-slide executive summary.); evidence (screenshots/logs/configs) is attached and reviewed}
\setDoR{Persona clear; AC drafted; Dependencies known; Estimate set.}
\setDoD{All ACs pass; Tests green; Security checks; Docs updated; Evidence attached.}
\RenderStoryCard
\begin{TasksBox}
\begin{itemize}
\TaskItem{Draft a one-page chapter plan: scope, objectives, interfaces, success metrics.}
\TaskItem{Set up tools, datasets, and accounts; document versions and configuration.}
\TaskItem{Complete objective: Define key terms and articulate why this topic matters to security outcomes.}
\TaskItem{Complete objective: Diagram the architecture/data flows and identify threat surfaces.}
\TaskItem{Execute lab: Hands-on: Build a small lab demonstrating key concepts in 'Fault Tolerance and Resilience in Cloud Computing Environments'. Capture screenshots/notes and one measurable result (e.g., a passing test, alert fired, or control verified).}
\end{itemize}
\end{TasksBox}
\clearpage
\setStoryID{CISH-010}
\setStoryTitle{Securing Web Applications, Services and Servers --- Learn \& Lab}
\setEpic{Part 1: Overview of System and Network Security}
\setBusinessValue{Build a working understanding of securing web applications, services and servers and its place in a modern security program; be able to explain core concepts, map them to the CIA triad, and identify common threats and controls.}
\setPriority{Must}
\setSP{3}
\setPersona{Application Security Engineer}
\setDependencies{Lab VM or container runtime, Git repo for notes, Markdown/PDF export tool, Sample web app, SAST/DAST scanner}
\setAssumptions{Time-box chapter to one iteration; open issues captured for later}
\setUserStory{As a Application Security Engineer, I want to study and practice 'Securing Web Applications, Services and Servers' so that I can apply its concepts to reduce risk and improve outcomes.}
\setNonFunctional{Security, Reliability, Performance}
\setScenario{Apply key controls for Securing Web Applications, Services and Servers}
\setGiven{the lab environment and topic-specific tools for 'Securing Web Applications, Services and Servers' are available}
\setWhen{I execute the hands-on objectives and lab: Hands-on: Build a small lab demonstrating key concepts in 'Securing Web Applications, Services and Servers'. Capture screenshots/notes and one measurable result (e.g., a passing test, alert fired, or control verified).}
\setThen{the deliverables are produced (2–3 page brief on 'Securing Web Applications, Services and Servers': risks, architecture, controls, and a checklist; plus a one-slide executive summary.); evidence (screenshots/logs/configs) is attached and reviewed}
\setDoR{Persona clear; AC drafted; Dependencies known; Estimate set.}
\setDoD{All ACs pass; Tests green; Security checks; Docs updated; Evidence attached.}
\RenderStoryCard
\begin{TasksBox}
\begin{itemize}
\TaskItem{Draft a one-page chapter plan: scope, objectives, interfaces, success metrics.}
\TaskItem{Set up tools, datasets, and accounts; document versions and configuration.}
\TaskItem{Complete objective: Define key terms and articulate why this topic matters to security outcomes.}
\TaskItem{Complete objective: Diagram the architecture/data flows and identify threat surfaces.}
\TaskItem{Execute lab: Hands-on: Build a small lab demonstrating key concepts in 'Securing Web Applications, Services and Servers'. Capture screenshots/notes and one measurable result (e.g., a passing test, alert fired, or control verified).}
\end{itemize}
\end{TasksBox}
\clearpage
\setStoryID{CISH-011}
\setStoryTitle{UNIX and Linux Security --- Learn \& Lab}
\setEpic{Part 1: Overview of System and Network Security}
\setBusinessValue{Build a working understanding of unix and linux security and its place in a modern security program; be able to explain core concepts, map them to the CIA triad, and identify common threats and controls.}
\setPriority{Must}
\setSP{3}
\setPersona{Systems Engineer}
\setDependencies{Lab VM or container runtime, Git repo for notes, Markdown/PDF export tool}
\setAssumptions{Scope excludes legacy, unsupported OS versions where impractical; Time-box chapter to one iteration; open issues captured for later}
\setUserStory{As a Systems Engineer, I want to study and practice 'UNIX and Linux Security' so that I can apply its concepts to reduce risk and improve outcomes.}
\setNonFunctional{Security, Reliability, Performance}
\setScenario{Apply key controls for UNIX and Linux Security}
\setGiven{the lab environment and topic-specific tools for 'UNIX and Linux Security' are available}
\setWhen{I execute the hands-on objectives and lab: Hands-on: Build a small lab demonstrating key concepts in 'UNIX and Linux Security'. Capture screenshots/notes and one measurable result (e.g., a passing test, alert fired, or control verified).}
\setThen{the deliverables are produced (2–3 page brief on 'UNIX and Linux Security': risks, architecture, controls, and a checklist; plus a one-slide executive summary.); evidence (screenshots/logs/configs) is attached and reviewed}
\setDoR{Persona clear; AC drafted; Dependencies known; Estimate set.}
\setDoD{All ACs pass; Tests green; Security checks; Docs updated; Evidence attached.}
\RenderStoryCard
\begin{TasksBox}
\begin{itemize}
\TaskItem{Draft a one-page chapter plan: scope, objectives, interfaces, success metrics.}
\TaskItem{Set up tools, datasets, and accounts; document versions and configuration.}
\TaskItem{Complete objective: Define key terms and articulate why this topic matters to security outcomes.}
\TaskItem{Complete objective: Diagram the architecture/data flows and identify threat surfaces.}
\TaskItem{Execute lab: Hands-on: Build a small lab demonstrating key concepts in 'UNIX and Linux Security'. Capture screenshots/notes and one measurable result (e.g., a passing test, alert fired, or control verified).}
\end{itemize}
\end{TasksBox}
\clearpage
\setStoryID{CISH-012}
\setStoryTitle{Eliminating the Security Weakness of Linux and UNIX Operating Systems --- Learn \& Lab}
\setEpic{Part 1: Overview of System and Network Security}
\setBusinessValue{Build a working understanding of eliminating the security weakness of linux and unix operating systems and its place in a modern security program; be able to explain core concepts, map them to the CIA triad, and identify common threats and controls.}
\setPriority{Must}
\setSP{3}
\setPersona{Systems Engineer}
\setDependencies{Lab VM or container runtime, Git repo for notes, Markdown/PDF export tool}
\setAssumptions{Scope excludes legacy, unsupported OS versions where impractical; Time-box chapter to one iteration; open issues captured for later}
\setUserStory{As a Systems Engineer, I want to study and practice 'Eliminating the Security Weakness of Linux and UNIX Operating Systems' so that I can apply its concepts to reduce risk and improve outcomes.}
\setNonFunctional{Security, Reliability, Performance}
\setScenario{Apply key controls for Eliminating the Security Weakness of Linux and UNIX Operating Systems}
\setGiven{the lab environment and topic-specific tools for 'Eliminating the Security Weakness of Linux and UNIX Operating Systems' are available}
\setWhen{I execute the hands-on objectives and lab: Hands-on: Build a small lab demonstrating key concepts in 'Eliminating the Security Weakness of Linux and UNIX Operating Systems'. Capture screenshots/notes and one measurable result (e.g., a passing test, alert fired, or control verified).}
\setThen{the deliverables are produced (2–3 page brief on 'Eliminating the Security Weakness of Linux and UNIX Operating Systems': risks, architecture, controls, and a checklist; plus a one-slide executive summary.); evidence (screenshots/logs/configs) is attached and reviewed}
\setDoR{Persona clear; AC drafted; Dependencies known; Estimate set.}
\setDoD{All ACs pass; Tests green; Security checks; Docs updated; Evidence attached.}
\RenderStoryCard
\begin{TasksBox}
\begin{itemize}
\TaskItem{Draft a one-page chapter plan: scope, objectives, interfaces, success metrics.}
\TaskItem{Set up tools, datasets, and accounts; document versions and configuration.}
\TaskItem{Complete objective: Define key terms and articulate why this topic matters to security outcomes.}
\TaskItem{Complete objective: Diagram the architecture/data flows and identify threat surfaces.}
\TaskItem{Execute lab: Hands-on: Build a small lab demonstrating key concepts in 'Eliminating the Security Weakness of Linux and UNIX Operating Systems'. Capture screenshots/notes and one measurable result (e.g., a passing test, alert fired, or control verified).}
\end{itemize}
\end{TasksBox}
\clearpage
\setStoryID{CISH-013}
\setStoryTitle{Internet Security --- Learn \& Lab}
\setEpic{Part 1: Overview of System and Network Security}
\setBusinessValue{Build a working understanding of internet security and its place in a modern security program; be able to explain core concepts, map them to the CIA triad, and identify common threats and controls.}
\setPriority{Must}
\setSP{3}
\setPersona{Network Security Engineer}
\setDependencies{Lab VM or container runtime, Git repo for notes, Markdown/PDF export tool, Packet capture tool (tcpdump/Wireshark), Firewall/router lab}
\setAssumptions{Time-box chapter to one iteration; open issues captured for later}
\setUserStory{As a Network Security Engineer, I want to study and practice 'Internet Security' so that I can apply its concepts to reduce risk and improve outcomes.}
\setNonFunctional{Security, Reliability, Performance}
\setScenario{Apply key controls for Internet Security}
\setGiven{the lab environment and topic-specific tools for 'Internet Security' are available}
\setWhen{I execute the hands-on objectives and lab: Hands-on: Build a small lab demonstrating key concepts in 'Internet Security'. Capture screenshots/notes and one measurable result (e.g., a passing test, alert fired, or control verified).}
\setThen{the deliverables are produced (2–3 page brief on 'Internet Security': risks, architecture, controls, and a checklist; plus a one-slide executive summary.); evidence (screenshots/logs/configs) is attached and reviewed}
\setDoR{Persona clear; AC drafted; Dependencies known; Estimate set.}
\setDoD{All ACs pass; Tests green; Security checks; Docs updated; Evidence attached.}
\RenderStoryCard
\begin{TasksBox}
\begin{itemize}
\TaskItem{Draft a one-page chapter plan: scope, objectives, interfaces, success metrics.}
\TaskItem{Set up tools, datasets, and accounts; document versions and configuration.}
\TaskItem{Complete objective: Define key terms and articulate why this topic matters to security outcomes.}
\TaskItem{Complete objective: Diagram the architecture/data flows and identify threat surfaces.}
\TaskItem{Execute lab: Hands-on: Build a small lab demonstrating key concepts in 'Internet Security'. Capture screenshots/notes and one measurable result (e.g., a passing test, alert fired, or control verified).}
\end{itemize}
\end{TasksBox}
\clearpage
\setStoryID{CISH-014}
\setStoryTitle{The Botnet Problem --- Learn \& Lab}
\setEpic{Part 1: Overview of System and Network Security}
\setBusinessValue{Build a working understanding of the botnet problem and its place in a modern security program; be able to explain core concepts, map them to the CIA triad, and identify common threats and controls.}
\setPriority{Must}
\setSP{3}
\setPersona{Security Engineer}
\setDependencies{Lab VM or container runtime, Git repo for notes, Markdown/PDF export tool}
\setAssumptions{Time-box chapter to one iteration; open issues captured for later}
\setUserStory{As a Security Engineer, I want to study and practice 'The Botnet Problem' so that I can apply its concepts to reduce risk and improve outcomes.}
\setNonFunctional{Security, Reliability, Performance}
\setScenario{Apply key controls for The Botnet Problem}
\setGiven{the lab environment and topic-specific tools for 'The Botnet Problem' are available}
\setWhen{I execute the hands-on objectives and lab: Hands-on: Build a small lab demonstrating key concepts in 'The Botnet Problem'. Capture screenshots/notes and one measurable result (e.g., a passing test, alert fired, or control verified).}
\setThen{the deliverables are produced (2–3 page brief on 'The Botnet Problem': risks, architecture, controls, and a checklist; plus a one-slide executive summary.); evidence (screenshots/logs/configs) is attached and reviewed}
\setDoR{Persona clear; AC drafted; Dependencies known; Estimate set.}
\setDoD{All ACs pass; Tests green; Security checks; Docs updated; Evidence attached.}
\RenderStoryCard
\begin{TasksBox}
\begin{itemize}
\TaskItem{Draft a one-page chapter plan: scope, objectives, interfaces, success metrics.}
\TaskItem{Set up tools, datasets, and accounts; document versions and configuration.}
\TaskItem{Complete objective: Define key terms and articulate why this topic matters to security outcomes.}
\TaskItem{Complete objective: Diagram the architecture/data flows and identify threat surfaces.}
\TaskItem{Execute lab: Hands-on: Build a small lab demonstrating key concepts in 'The Botnet Problem'. Capture screenshots/notes and one measurable result (e.g., a passing test, alert fired, or control verified).}
\end{itemize}
\end{TasksBox}
\clearpage
\setStoryID{CISH-015}
\setStoryTitle{Intranet Security --- Learn \& Lab}
\setEpic{Part 1: Overview of System and Network Security}
\setBusinessValue{Build a working understanding of intranet security and its place in a modern security program; be able to explain core concepts, map them to the CIA triad, and identify common threats and controls.}
\setPriority{Must}
\setSP{3}
\setPersona{Network Security Engineer}
\setDependencies{Lab VM or container runtime, Git repo for notes, Markdown/PDF export tool}
\setAssumptions{Time-box chapter to one iteration; open issues captured for later}
\setUserStory{As a Network Security Engineer, I want to study and practice 'Intranet Security' so that I can apply its concepts to reduce risk and improve outcomes.}
\setNonFunctional{Security, Reliability, Performance}
\setScenario{Apply key controls for Intranet Security}
\setGiven{the lab environment and topic-specific tools for 'Intranet Security' are available}
\setWhen{I execute the hands-on objectives and lab: Hands-on: Build a small lab demonstrating key concepts in 'Intranet Security'. Capture screenshots/notes and one measurable result (e.g., a passing test, alert fired, or control verified).}
\setThen{the deliverables are produced (2–3 page brief on 'Intranet Security': risks, architecture, controls, and a checklist; plus a one-slide executive summary.); evidence (screenshots/logs/configs) is attached and reviewed}
\setDoR{Persona clear; AC drafted; Dependencies known; Estimate set.}
\setDoD{All ACs pass; Tests green; Security checks; Docs updated; Evidence attached.}
\RenderStoryCard
\begin{TasksBox}
\begin{itemize}
\TaskItem{Draft a one-page chapter plan: scope, objectives, interfaces, success metrics.}
\TaskItem{Set up tools, datasets, and accounts; document versions and configuration.}
\TaskItem{Complete objective: Define key terms and articulate why this topic matters to security outcomes.}
\TaskItem{Complete objective: Diagram the architecture/data flows and identify threat surfaces.}
\TaskItem{Execute lab: Hands-on: Build a small lab demonstrating key concepts in 'Intranet Security'. Capture screenshots/notes and one measurable result (e.g., a passing test, alert fired, or control verified).}
\end{itemize}
\end{TasksBox}
\clearpage
\setStoryID{CISH-016}
\setStoryTitle{Local Area Network Security --- Learn \& Lab}
\setEpic{Part 1: Overview of System and Network Security}
\setBusinessValue{Build a working understanding of local area network security and its place in a modern security program; be able to explain core concepts, map them to the CIA triad, and identify common threats and controls.}
\setPriority{Must}
\setSP{3}
\setPersona{Network Security Engineer}
\setDependencies{Lab VM or container runtime, Git repo for notes, Markdown/PDF export tool, Packet capture tool (tcpdump/Wireshark), Firewall/router lab}
\setAssumptions{Time-box chapter to one iteration; open issues captured for later}
\setUserStory{As a Network Security Engineer, I want to study and practice 'Local Area Network Security' so that I can apply its concepts to reduce risk and improve outcomes.}
\setNonFunctional{Security, Reliability, Performance}
\setScenario{Apply key controls for Local Area Network Security}
\setGiven{the lab environment and topic-specific tools for 'Local Area Network Security' are available}
\setWhen{I execute the hands-on objectives and lab: Hands-on: Build a small lab demonstrating key concepts in 'Local Area Network Security'. Capture screenshots/notes and one measurable result (e.g., a passing test, alert fired, or control verified).}
\setThen{the deliverables are produced (2–3 page brief on 'Local Area Network Security': risks, architecture, controls, and a checklist; plus a one-slide executive summary.); evidence (screenshots/logs/configs) is attached and reviewed}
\setDoR{Persona clear; AC drafted; Dependencies known; Estimate set.}
\setDoD{All ACs pass; Tests green; Security checks; Docs updated; Evidence attached.}
\RenderStoryCard
\begin{TasksBox}
\begin{itemize}
\TaskItem{Draft a one-page chapter plan: scope, objectives, interfaces, success metrics.}
\TaskItem{Set up tools, datasets, and accounts; document versions and configuration.}
\TaskItem{Complete objective: Define key terms and articulate why this topic matters to security outcomes.}
\TaskItem{Complete objective: Diagram the architecture/data flows and identify threat surfaces.}
\TaskItem{Execute lab: Hands-on: Build a small lab demonstrating key concepts in 'Local Area Network Security'. Capture screenshots/notes and one measurable result (e.g., a passing test, alert fired, or control verified).}
\end{itemize}
\end{TasksBox}
\clearpage
\setStoryID{CISH-017}
\setStoryTitle{Wireless Network Security --- Learn \& Lab}
\setEpic{Part 1: Overview of System and Network Security}
\setBusinessValue{Build a working understanding of wireless network security and its place in a modern security program; be able to explain core concepts, map them to the CIA triad, and identify common threats and controls.}
\setPriority{Must}
\setSP{3}
\setPersona{Network Security Engineer}
\setDependencies{Lab VM or container runtime, Git repo for notes, Markdown/PDF export tool, Packet capture tool (tcpdump/Wireshark), Firewall/router lab}
\setAssumptions{Time-box chapter to one iteration; open issues captured for later}
\setUserStory{As a Network Security Engineer, I want to study and practice 'Wireless Network Security' so that I can apply its concepts to reduce risk and improve outcomes.}
\setNonFunctional{Security, Reliability, Performance}
\setScenario{Apply key controls for Wireless Network Security}
\setGiven{the lab environment and topic-specific tools for 'Wireless Network Security' are available}
\setWhen{I execute the hands-on objectives and lab: Hands-on: Build a small lab demonstrating key concepts in 'Wireless Network Security'. Capture screenshots/notes and one measurable result (e.g., a passing test, alert fired, or control verified).}
\setThen{the deliverables are produced (2–3 page brief on 'Wireless Network Security': risks, architecture, controls, and a checklist; plus a one-slide executive summary.); evidence (screenshots/logs/configs) is attached and reviewed}
\setDoR{Persona clear; AC drafted; Dependencies known; Estimate set.}
\setDoD{All ACs pass; Tests green; Security checks; Docs updated; Evidence attached.}
\RenderStoryCard
\begin{TasksBox}
\begin{itemize}
\TaskItem{Draft a one-page chapter plan: scope, objectives, interfaces, success metrics.}
\TaskItem{Set up tools, datasets, and accounts; document versions and configuration.}
\TaskItem{Complete objective: Define key terms and articulate why this topic matters to security outcomes.}
\TaskItem{Complete objective: Diagram the architecture/data flows and identify threat surfaces.}
\TaskItem{Execute lab: Hands-on: Build a small lab demonstrating key concepts in 'Wireless Network Security'. Capture screenshots/notes and one measurable result (e.g., a passing test, alert fired, or control verified).}
\end{itemize}
\end{TasksBox}
\clearpage
\setStoryID{CISH-018}
\setStoryTitle{Wireless Sensor Network Security: The Internet of Things --- Learn \& Lab}
\setEpic{Part 1: Overview of System and Network Security}
\setBusinessValue{Build a working understanding of wireless sensor network security: the internet of things and its place in a modern security program; be able to explain core concepts, map them to the CIA triad, and identify common threats and controls.}
\setPriority{Must}
\setSP{3}
\setPersona{Network Security Engineer}
\setDependencies{Lab VM or container runtime, Git repo for notes, Markdown/PDF export tool, Packet capture tool (tcpdump/Wireshark), Firewall/router lab}
\setAssumptions{Time-box chapter to one iteration; open issues captured for later}
\setUserStory{As a Network Security Engineer, I want to study and practice 'Wireless Sensor Network Security: The Internet of Things' so that I can apply its concepts to reduce risk and improve outcomes.}
\setNonFunctional{Security, Reliability, Performance}
\setScenario{Apply key controls for Wireless Sensor Network Security: The Internet of Things}
\setGiven{the lab environment and topic-specific tools for 'Wireless Sensor Network Security: The Internet of Things' are available}
\setWhen{I execute the hands-on objectives and lab: Hands-on: Build a small lab demonstrating key concepts in 'Wireless Sensor Network Security: The Internet of Things'. Capture screenshots/notes and one measurable result (e.g., a passing test, alert fired, or control verified).}
\setThen{the deliverables are produced (2–3 page brief on 'Wireless Sensor Network Security: The Internet of Things': risks, architecture, controls, and a checklist; plus a one-slide executive summary.); evidence (screenshots/logs/configs) is attached and reviewed}
\setDoR{Persona clear; AC drafted; Dependencies known; Estimate set.}
\setDoD{All ACs pass; Tests green; Security checks; Docs updated; Evidence attached.}
\RenderStoryCard
\begin{TasksBox}
\begin{itemize}
\TaskItem{Draft a one-page chapter plan: scope, objectives, interfaces, success metrics.}
\TaskItem{Set up tools, datasets, and accounts; document versions and configuration.}
\TaskItem{Complete objective: Define key terms and articulate why this topic matters to security outcomes.}
\TaskItem{Complete objective: Diagram the architecture/data flows and identify threat surfaces.}
\TaskItem{Execute lab: Hands-on: Build a small lab demonstrating key concepts in 'Wireless Sensor Network Security: The Internet of Things'. Capture screenshots/notes and one measurable result (e.g., a passing test, alert fired, or control verified).}
\end{itemize}
\end{TasksBox}
\clearpage
\setStoryID{CISH-019}
\setStoryTitle{Security for the Internet of Things --- Learn \& Lab}
\setEpic{Part 1: Overview of System and Network Security}
\setBusinessValue{Build a working understanding of security for the internet of things and its place in a modern security program; be able to explain core concepts, map them to the CIA triad, and identify common threats and controls.}
\setPriority{Must}
\setSP{3}
\setPersona{Network Security Engineer}
\setDependencies{Lab VM or container runtime, Git repo for notes, Markdown/PDF export tool, Packet capture tool (tcpdump/Wireshark), Firewall/router lab}
\setAssumptions{Time-box chapter to one iteration; open issues captured for later}
\setUserStory{As a Network Security Engineer, I want to study and practice 'Security for the Internet of Things' so that I can apply its concepts to reduce risk and improve outcomes.}
\setNonFunctional{Security, Reliability, Performance}
\setScenario{Apply key controls for Security for the Internet of Things}
\setGiven{the lab environment and topic-specific tools for 'Security for the Internet of Things' are available}
\setWhen{I execute the hands-on objectives and lab: Hands-on: Build a small lab demonstrating key concepts in 'Security for the Internet of Things'. Capture screenshots/notes and one measurable result (e.g., a passing test, alert fired, or control verified).}
\setThen{the deliverables are produced (2–3 page brief on 'Security for the Internet of Things': risks, architecture, controls, and a checklist; plus a one-slide executive summary.); evidence (screenshots/logs/configs) is attached and reviewed}
\setDoR{Persona clear; AC drafted; Dependencies known; Estimate set.}
\setDoD{All ACs pass; Tests green; Security checks; Docs updated; Evidence attached.}
\RenderStoryCard
\begin{TasksBox}
\begin{itemize}
\TaskItem{Draft a one-page chapter plan: scope, objectives, interfaces, success metrics.}
\TaskItem{Set up tools, datasets, and accounts; document versions and configuration.}
\TaskItem{Complete objective: Define key terms and articulate why this topic matters to security outcomes.}
\TaskItem{Complete objective: Diagram the architecture/data flows and identify threat surfaces.}
\TaskItem{Execute lab: Hands-on: Build a small lab demonstrating key concepts in 'Security for the Internet of Things'. Capture screenshots/notes and one measurable result (e.g., a passing test, alert fired, or control verified).}
\end{itemize}
\end{TasksBox}
\clearpage
\setStoryID{CISH-020}
\setStoryTitle{Cellular Network Security --- Learn \& Lab}
\setEpic{Part 1: Overview of System and Network Security}
\setBusinessValue{Build a working understanding of cellular network security and its place in a modern security program; be able to explain core concepts, map them to the CIA triad, and identify common threats and controls.}
\setPriority{Must}
\setSP{3}
\setPersona{Network Security Engineer}
\setDependencies{Lab VM or container runtime, Git repo for notes, Markdown/PDF export tool, Packet capture tool (tcpdump/Wireshark), Firewall/router lab}
\setAssumptions{Time-box chapter to one iteration; open issues captured for later}
\setUserStory{As a Network Security Engineer, I want to study and practice 'Cellular Network Security' so that I can apply its concepts to reduce risk and improve outcomes.}
\setNonFunctional{Security, Reliability, Performance}
\setScenario{Apply key controls for Cellular Network Security}
\setGiven{the lab environment and topic-specific tools for 'Cellular Network Security' are available}
\setWhen{I execute the hands-on objectives and lab: Hands-on: Build a small lab demonstrating key concepts in 'Cellular Network Security'. Capture screenshots/notes and one measurable result (e.g., a passing test, alert fired, or control verified).}
\setThen{the deliverables are produced (2–3 page brief on 'Cellular Network Security': risks, architecture, controls, and a checklist; plus a one-slide executive summary.); evidence (screenshots/logs/configs) is attached and reviewed}
\setDoR{Persona clear; AC drafted; Dependencies known; Estimate set.}
\setDoD{All ACs pass; Tests green; Security checks; Docs updated; Evidence attached.}
\RenderStoryCard
\begin{TasksBox}
\begin{itemize}
\TaskItem{Draft a one-page chapter plan: scope, objectives, interfaces, success metrics.}
\TaskItem{Set up tools, datasets, and accounts; document versions and configuration.}
\TaskItem{Complete objective: Define key terms and articulate why this topic matters to security outcomes.}
\TaskItem{Complete objective: Diagram the architecture/data flows and identify threat surfaces.}
\TaskItem{Execute lab: Hands-on: Build a small lab demonstrating key concepts in 'Cellular Network Security'. Capture screenshots/notes and one measurable result (e.g., a passing test, alert fired, or control verified).}
\end{itemize}
\end{TasksBox}
\clearpage
\setStoryID{CISH-021}
\setStoryTitle{Radio Frequency Identification Security --- Learn \& Lab}
\setEpic{Part 1: Overview of System and Network Security}
\setBusinessValue{Build a working understanding of radio frequency identification security and its place in a modern security program; be able to explain core concepts, map them to the CIA triad, and identify common threats and controls.}
\setPriority{Must}
\setSP{3}
\setPersona{Security Engineer}
\setDependencies{Lab VM or container runtime, Git repo for notes, Markdown/PDF export tool}
\setAssumptions{Time-box chapter to one iteration; open issues captured for later}
\setUserStory{As a Security Engineer, I want to study and practice 'Radio Frequency Identification Security' so that I can apply its concepts to reduce risk and improve outcomes.}
\setNonFunctional{Security, Reliability, Performance}
\setScenario{Apply key controls for Radio Frequency Identification Security}
\setGiven{the lab environment and topic-specific tools for 'Radio Frequency Identification Security' are available}
\setWhen{I execute the hands-on objectives and lab: Hands-on: Build a small lab demonstrating key concepts in 'Radio Frequency Identification Security'. Capture screenshots/notes and one measurable result (e.g., a passing test, alert fired, or control verified).}
\setThen{the deliverables are produced (2–3 page brief on 'Radio Frequency Identification Security': risks, architecture, controls, and a checklist; plus a one-slide executive summary.); evidence (screenshots/logs/configs) is attached and reviewed}
\setDoR{Persona clear; AC drafted; Dependencies known; Estimate set.}
\setDoD{All ACs pass; Tests green; Security checks; Docs updated; Evidence attached.}
\RenderStoryCard
\begin{TasksBox}
\begin{itemize}
\TaskItem{Draft a one-page chapter plan: scope, objectives, interfaces, success metrics.}
\TaskItem{Set up tools, datasets, and accounts; document versions and configuration.}
\TaskItem{Complete objective: Define key terms and articulate why this topic matters to security outcomes.}
\TaskItem{Complete objective: Diagram the architecture/data flows and identify threat surfaces.}
\TaskItem{Execute lab: Hands-on: Build a small lab demonstrating key concepts in 'Radio Frequency Identification Security'. Capture screenshots/notes and one measurable result (e.g., a passing test, alert fired, or control verified).}
\end{itemize}
\end{TasksBox}
\clearpage
\setStoryID{CISH-022}
\setStoryTitle{Optical Network Security --- Learn \& Lab}
\setEpic{Part 1: Overview of System and Network Security}
\setBusinessValue{Build a working understanding of optical network security and its place in a modern security program; be able to explain core concepts, map them to the CIA triad, and identify common threats and controls.}
\setPriority{Must}
\setSP{3}
\setPersona{Network Security Engineer}
\setDependencies{Lab VM or container runtime, Git repo for notes, Markdown/PDF export tool, Packet capture tool (tcpdump/Wireshark), Firewall/router lab}
\setAssumptions{Time-box chapter to one iteration; open issues captured for later}
\setUserStory{As a Network Security Engineer, I want to study and practice 'Optical Network Security' so that I can apply its concepts to reduce risk and improve outcomes.}
\setNonFunctional{Security, Reliability, Performance}
\setScenario{Apply key controls for Optical Network Security}
\setGiven{the lab environment and topic-specific tools for 'Optical Network Security' are available}
\setWhen{I execute the hands-on objectives and lab: Hands-on: Build a small lab demonstrating key concepts in 'Optical Network Security'. Capture screenshots/notes and one measurable result (e.g., a passing test, alert fired, or control verified).}
\setThen{the deliverables are produced (2–3 page brief on 'Optical Network Security': risks, architecture, controls, and a checklist; plus a one-slide executive summary.); evidence (screenshots/logs/configs) is attached and reviewed}
\setDoR{Persona clear; AC drafted; Dependencies known; Estimate set.}
\setDoD{All ACs pass; Tests green; Security checks; Docs updated; Evidence attached.}
\RenderStoryCard
\begin{TasksBox}
\begin{itemize}
\TaskItem{Draft a one-page chapter plan: scope, objectives, interfaces, success metrics.}
\TaskItem{Set up tools, datasets, and accounts; document versions and configuration.}
\TaskItem{Complete objective: Define key terms and articulate why this topic matters to security outcomes.}
\TaskItem{Complete objective: Diagram the architecture/data flows and identify threat surfaces.}
\TaskItem{Execute lab: Hands-on: Build a small lab demonstrating key concepts in 'Optical Network Security'. Capture screenshots/notes and one measurable result (e.g., a passing test, alert fired, or control verified).}
\end{itemize}
\end{TasksBox}
\clearpage
\setStoryID{CISH-023}
\setStoryTitle{Optical Wireless Security --- Learn \& Lab}
\setEpic{Part 1: Overview of System and Network Security}
\setBusinessValue{Build a working understanding of optical wireless security and its place in a modern security program; be able to explain core concepts, map them to the CIA triad, and identify common threats and controls.}
\setPriority{Must}
\setSP{3}
\setPersona{Network Security Engineer}
\setDependencies{Lab VM or container runtime, Git repo for notes, Markdown/PDF export tool, Packet capture tool (tcpdump/Wireshark), Firewall/router lab}
\setAssumptions{Time-box chapter to one iteration; open issues captured for later}
\setUserStory{As a Network Security Engineer, I want to study and practice 'Optical Wireless Security' so that I can apply its concepts to reduce risk and improve outcomes.}
\setNonFunctional{Security, Reliability, Performance}
\setScenario{Apply key controls for Optical Wireless Security}
\setGiven{the lab environment and topic-specific tools for 'Optical Wireless Security' are available}
\setWhen{I execute the hands-on objectives and lab: Hands-on: Build a small lab demonstrating key concepts in 'Optical Wireless Security'. Capture screenshots/notes and one measurable result (e.g., a passing test, alert fired, or control verified).}
\setThen{the deliverables are produced (2–3 page brief on 'Optical Wireless Security': risks, architecture, controls, and a checklist; plus a one-slide executive summary.); evidence (screenshots/logs/configs) is attached and reviewed}
\setDoR{Persona clear; AC drafted; Dependencies known; Estimate set.}
\setDoD{All ACs pass; Tests green; Security checks; Docs updated; Evidence attached.}
\RenderStoryCard
\begin{TasksBox}
\begin{itemize}
\TaskItem{Draft a one-page chapter plan: scope, objectives, interfaces, success metrics.}
\TaskItem{Set up tools, datasets, and accounts; document versions and configuration.}
\TaskItem{Complete objective: Define key terms and articulate why this topic matters to security outcomes.}
\TaskItem{Complete objective: Diagram the architecture/data flows and identify threat surfaces.}
\TaskItem{Execute lab: Hands-on: Build a small lab demonstrating key concepts in 'Optical Wireless Security'. Capture screenshots/notes and one measurable result (e.g., a passing test, alert fired, or control verified).}
\end{itemize}
\end{TasksBox}
\clearpage
\setStoryID{CISH-024}
\setStoryTitle{Information Security Essentials for Information Technology Managers: Protecting Mission-Critical Systems --- Learn \& Lab}
\setEpic{Part 2: Managing Information Security}
\setBusinessValue{Build a working understanding of information security essentials for information technology managers: protecting mission-critical systems and its place in a modern security program; be able to explain core concepts, map them to the CIA triad, and identify common threats and controls.}
\setPriority{Must}
\setSP{3}
\setPersona{Security Engineer}
\setDependencies{Lab VM or container runtime, Git repo for notes, Markdown/PDF export tool}
\setAssumptions{Time-box chapter to one iteration; open issues captured for later}
\setUserStory{As a Security Engineer, I want to study and practice 'Information Security Essentials for Information Technology Managers: Protecting Mission-Critical Systems' so that I can apply its concepts to reduce risk and improve outcomes.}
\setNonFunctional{Security, Reliability, Performance}
\setScenario{Apply key controls for Information Security Essentials for Information Technology Managers: Protecting Mission-Critical Systems}
\setGiven{the lab environment and topic-specific tools for 'Information Security Essentials for Information Technology Managers: Protecting Mission-Critical Systems' are available}
\setWhen{I execute the hands-on objectives and lab: Hands-on: Build a small lab demonstrating key concepts in 'Information Security Essentials for Information Technology Managers: Protecting Mission-Critical Systems'. Capture screenshots/notes and one measurable result (e.g., a passing test, alert fired, or control verified).}
\setThen{the deliverables are produced (2–3 page brief on 'Information Security Essentials for Information Technology Managers: Protecting Mission-Critical Systems': risks, architecture, controls, and a checklist; plus a one-slide executive summary.); evidence (screenshots/logs/configs) is attached and reviewed}
\setDoR{Persona clear; AC drafted; Dependencies known; Estimate set.}
\setDoD{All ACs pass; Tests green; Security checks; Docs updated; Evidence attached.}
\RenderStoryCard
\begin{TasksBox}
\begin{itemize}
\TaskItem{Draft a one-page chapter plan: scope, objectives, interfaces, success metrics.}
\TaskItem{Set up tools, datasets, and accounts; document versions and configuration.}
\TaskItem{Complete objective: Define key terms and articulate why this topic matters to security outcomes.}
\TaskItem{Complete objective: Diagram the architecture/data flows and identify threat surfaces.}
\TaskItem{Execute lab: Hands-on: Build a small lab demonstrating key concepts in 'Information Security Essentials for Information Technology Managers: Protecting Mission-Critical Systems'. Capture screenshots/notes and one measurable result (e.g., a passing test, alert fired, or control verified).}
\end{itemize}
\end{TasksBox}
\clearpage
\setStoryID{CISH-025}
\setStoryTitle{Security Management Systems --- Learn \& Lab}
\setEpic{Part 2: Managing Information Security}
\setBusinessValue{Build a working understanding of security management systems and its place in a modern security program; be able to explain core concepts, map them to the CIA triad, and identify common threats and controls.}
\setPriority{Must}
\setSP{3}
\setPersona{Security Program Manager}
\setDependencies{Lab VM or container runtime, Git repo for notes, Markdown/PDF export tool}
\setAssumptions{Time-box chapter to one iteration; open issues captured for later}
\setUserStory{As a Security Program Manager, I want to study and practice 'Security Management Systems' so that I can apply its concepts to reduce risk and improve outcomes.}
\setNonFunctional{Security, Reliability, Performance}
\setScenario{Apply key controls for Security Management Systems}
\setGiven{the lab environment and topic-specific tools for 'Security Management Systems' are available}
\setWhen{I execute the hands-on objectives and lab: Hands-on: Build a small lab demonstrating key concepts in 'Security Management Systems'. Capture screenshots/notes and one measurable result (e.g., a passing test, alert fired, or control verified).}
\setThen{the deliverables are produced (2–3 page brief on 'Security Management Systems': risks, architecture, controls, and a checklist; plus a one-slide executive summary.); evidence (screenshots/logs/configs) is attached and reviewed}
\setDoR{Persona clear; AC drafted; Dependencies known; Estimate set.}
\setDoD{All ACs pass; Tests green; Security checks; Docs updated; Evidence attached.}
\RenderStoryCard
\begin{TasksBox}
\begin{itemize}
\TaskItem{Draft a one-page chapter plan: scope, objectives, interfaces, success metrics.}
\TaskItem{Set up tools, datasets, and accounts; document versions and configuration.}
\TaskItem{Complete objective: Define key terms and articulate why this topic matters to security outcomes.}
\TaskItem{Complete objective: Diagram the architecture/data flows and identify threat surfaces.}
\TaskItem{Execute lab: Hands-on: Build a small lab demonstrating key concepts in 'Security Management Systems'. Capture screenshots/notes and one measurable result (e.g., a passing test, alert fired, or control verified).}
\end{itemize}
\end{TasksBox}
\clearpage
\setStoryID{CISH-026}
\setStoryTitle{Policy-Driven System Management --- Learn \& Lab}
\setEpic{Part 2: Managing Information Security}
\setBusinessValue{Build a working understanding of policy-driven system management and its place in a modern security program; be able to explain core concepts, map them to the CIA triad, and identify common threats and controls.}
\setPriority{Must}
\setSP{3}
\setPersona{Security Program Manager}
\setDependencies{Lab VM or container runtime, Git repo for notes, Markdown/PDF export tool}
\setAssumptions{Time-box chapter to one iteration; open issues captured for later}
\setUserStory{As a Security Program Manager, I want to study and practice 'Policy-Driven System Management' so that I can apply its concepts to reduce risk and improve outcomes.}
\setNonFunctional{Security, Reliability, Performance, Compliance}
\setScenario{Apply key controls for Policy-Driven System Management}
\setGiven{the lab environment and topic-specific tools for 'Policy-Driven System Management' are available}
\setWhen{I execute the hands-on objectives and lab: Hands-on: Build a small lab demonstrating key concepts in 'Policy-Driven System Management'. Capture screenshots/notes and one measurable result (e.g., a passing test, alert fired, or control verified).}
\setThen{the deliverables are produced (2–3 page brief on 'Policy-Driven System Management': risks, architecture, controls, and a checklist; plus a one-slide executive summary.); evidence (screenshots/logs/configs) is attached and reviewed}
\setDoR{Persona clear; AC drafted; Dependencies known; Estimate set.}
\setDoD{All ACs pass; Tests green; Security checks; Docs updated; Evidence attached.}
\RenderStoryCard
\begin{TasksBox}
\begin{itemize}
\TaskItem{Draft a one-page chapter plan: scope, objectives, interfaces, success metrics.}
\TaskItem{Set up tools, datasets, and accounts; document versions and configuration.}
\TaskItem{Complete objective: Define key terms and articulate why this topic matters to security outcomes.}
\TaskItem{Complete objective: Diagram the architecture/data flows and identify threat surfaces.}
\TaskItem{Execute lab: Hands-on: Build a small lab demonstrating key concepts in 'Policy-Driven System Management'. Capture screenshots/notes and one measurable result (e.g., a passing test, alert fired, or control verified).}
\end{itemize}
\end{TasksBox}
\clearpage
\setStoryID{CISH-027}
\setStoryTitle{Information Technology Security Management --- Learn \& Lab}
\setEpic{Part 2: Managing Information Security}
\setBusinessValue{Build a working understanding of information technology security management and its place in a modern security program; be able to explain core concepts, map them to the CIA triad, and identify common threats and controls.}
\setPriority{Must}
\setSP{3}
\setPersona{Security Program Manager}
\setDependencies{Lab VM or container runtime, Git repo for notes, Markdown/PDF export tool}
\setAssumptions{Time-box chapter to one iteration; open issues captured for later}
\setUserStory{As a Security Program Manager, I want to study and practice 'Information Technology Security Management' so that I can apply its concepts to reduce risk and improve outcomes.}
\setNonFunctional{Security, Reliability, Performance}
\setScenario{Apply key controls for Information Technology Security Management}
\setGiven{the lab environment and topic-specific tools for 'Information Technology Security Management' are available}
\setWhen{I execute the hands-on objectives and lab: Hands-on: Build a small lab demonstrating key concepts in 'Information Technology Security Management'. Capture screenshots/notes and one measurable result (e.g., a passing test, alert fired, or control verified).}
\setThen{the deliverables are produced (2–3 page brief on 'Information Technology Security Management': risks, architecture, controls, and a checklist; plus a one-slide executive summary.); evidence (screenshots/logs/configs) is attached and reviewed}
\setDoR{Persona clear; AC drafted; Dependencies known; Estimate set.}
\setDoD{All ACs pass; Tests green; Security checks; Docs updated; Evidence attached.}
\RenderStoryCard
\begin{TasksBox}
\begin{itemize}
\TaskItem{Draft a one-page chapter plan: scope, objectives, interfaces, success metrics.}
\TaskItem{Set up tools, datasets, and accounts; document versions and configuration.}
\TaskItem{Complete objective: Define key terms and articulate why this topic matters to security outcomes.}
\TaskItem{Complete objective: Diagram the architecture/data flows and identify threat surfaces.}
\TaskItem{Execute lab: Hands-on: Build a small lab demonstrating key concepts in 'Information Technology Security Management'. Capture screenshots/notes and one measurable result (e.g., a passing test, alert fired, or control verified).}
\end{itemize}
\end{TasksBox}
\clearpage
\setStoryID{CISH-028}
\setStoryTitle{The Enemy (The Intruder’s Genesis) --- Learn \& Lab}
\setEpic{Part 2: Managing Information Security}
\setBusinessValue{Build a working understanding of the enemy (the intruder’s genesis) and its place in a modern security program; be able to explain core concepts, map them to the CIA triad, and identify common threats and controls.}
\setPriority{Must}
\setSP{3}
\setPersona{Security Engineer}
\setDependencies{Lab VM or container runtime, Git repo for notes, Markdown/PDF export tool}
\setAssumptions{Time-box chapter to one iteration; open issues captured for later}
\setUserStory{As a Security Engineer, I want to study and practice 'The Enemy (The Intruder’s Genesis)' so that I can apply its concepts to reduce risk and improve outcomes.}
\setNonFunctional{Security, Reliability, Performance}
\setScenario{Apply key controls for The Enemy (The Intruder’s Genesis)}
\setGiven{the lab environment and topic-specific tools for 'The Enemy (The Intruder’s Genesis)' are available}
\setWhen{I execute the hands-on objectives and lab: Hands-on: Build a small lab demonstrating key concepts in 'The Enemy (The Intruder’s Genesis)'. Capture screenshots/notes and one measurable result (e.g., a passing test, alert fired, or control verified).}
\setThen{the deliverables are produced (2–3 page brief on 'The Enemy (The Intruder’s Genesis)': risks, architecture, controls, and a checklist; plus a one-slide executive summary.); evidence (screenshots/logs/configs) is attached and reviewed}
\setDoR{Persona clear; AC drafted; Dependencies known; Estimate set.}
\setDoD{All ACs pass; Tests green; Security checks; Docs updated; Evidence attached.}
\RenderStoryCard
\begin{TasksBox}
\begin{itemize}
\TaskItem{Draft a one-page chapter plan: scope, objectives, interfaces, success metrics.}
\TaskItem{Set up tools, datasets, and accounts; document versions and configuration.}
\TaskItem{Complete objective: Define key terms and articulate why this topic matters to security outcomes.}
\TaskItem{Complete objective: Diagram the architecture/data flows and identify threat surfaces.}
\TaskItem{Execute lab: Hands-on: Build a small lab demonstrating key concepts in 'The Enemy (The Intruder’s Genesis)'. Capture screenshots/notes and one measurable result (e.g., a passing test, alert fired, or control verified).}
\end{itemize}
\end{TasksBox}
\clearpage
\setStoryID{CISH-029}
\setStoryTitle{Social Engineering Deceptions and Defenses --- Learn \& Lab}
\setEpic{Part 2: Managing Information Security}
\setBusinessValue{Build a working understanding of social engineering deceptions and defenses and its place in a modern security program; be able to explain core concepts, map them to the CIA triad, and identify common threats and controls.}
\setPriority{Must}
\setSP{3}
\setPersona{Security Architect}
\setDependencies{Lab VM or container runtime, Git repo for notes, Markdown/PDF export tool}
\setAssumptions{Time-box chapter to one iteration; open issues captured for later}
\setUserStory{As a Security Architect, I want to study and practice 'Social Engineering Deceptions and Defenses' so that I can apply its concepts to reduce risk and improve outcomes.}
\setNonFunctional{Security, Reliability, Performance}
\setScenario{Apply key controls for Social Engineering Deceptions and Defenses}
\setGiven{the lab environment and topic-specific tools for 'Social Engineering Deceptions and Defenses' are available}
\setWhen{I execute the hands-on objectives and lab: Hands-on: Build a small lab demonstrating key concepts in 'Social Engineering Deceptions and Defenses'. Capture screenshots/notes and one measurable result (e.g., a passing test, alert fired, or control verified).}
\setThen{the deliverables are produced (2–3 page brief on 'Social Engineering Deceptions and Defenses': risks, architecture, controls, and a checklist; plus a one-slide executive summary.); evidence (screenshots/logs/configs) is attached and reviewed}
\setDoR{Persona clear; AC drafted; Dependencies known; Estimate set.}
\setDoD{All ACs pass; Tests green; Security checks; Docs updated; Evidence attached.}
\RenderStoryCard
\begin{TasksBox}
\begin{itemize}
\TaskItem{Draft a one-page chapter plan: scope, objectives, interfaces, success metrics.}
\TaskItem{Set up tools, datasets, and accounts; document versions and configuration.}
\TaskItem{Complete objective: Define key terms and articulate why this topic matters to security outcomes.}
\TaskItem{Complete objective: Diagram the architecture/data flows and identify threat surfaces.}
\TaskItem{Execute lab: Hands-on: Build a small lab demonstrating key concepts in 'Social Engineering Deceptions and Defenses'. Capture screenshots/notes and one measurable result (e.g., a passing test, alert fired, or control verified).}
\end{itemize}
\end{TasksBox}
\clearpage
\setStoryID{CISH-030}
\setStoryTitle{Ethical Hacking --- Learn \& Lab}
\setEpic{Part 2: Managing Information Security}
\setBusinessValue{Build a working understanding of ethical hacking and its place in a modern security program; be able to explain core concepts, map them to the CIA triad, and identify common threats and controls.}
\setPriority{Must}
\setSP{3}
\setPersona{Security Engineer}
\setDependencies{Lab VM or container runtime, Git repo for notes, Markdown/PDF export tool}
\setAssumptions{Time-box chapter to one iteration; open issues captured for later}
\setUserStory{As a Security Engineer, I want to study and practice 'Ethical Hacking' so that I can apply its concepts to reduce risk and improve outcomes.}
\setNonFunctional{Security, Reliability, Performance}
\setScenario{Apply key controls for Ethical Hacking}
\setGiven{the lab environment and topic-specific tools for 'Ethical Hacking' are available}
\setWhen{I execute the hands-on objectives and lab: Hands-on: Build a small lab demonstrating key concepts in 'Ethical Hacking'. Capture screenshots/notes and one measurable result (e.g., a passing test, alert fired, or control verified).}
\setThen{the deliverables are produced (2–3 page brief on 'Ethical Hacking': risks, architecture, controls, and a checklist; plus a one-slide executive summary.); evidence (screenshots/logs/configs) is attached and reviewed}
\setDoR{Persona clear; AC drafted; Dependencies known; Estimate set.}
\setDoD{All ACs pass; Tests green; Security checks; Docs updated; Evidence attached.}
\RenderStoryCard
\begin{TasksBox}
\begin{itemize}
\TaskItem{Draft a one-page chapter plan: scope, objectives, interfaces, success metrics.}
\TaskItem{Set up tools, datasets, and accounts; document versions and configuration.}
\TaskItem{Complete objective: Define key terms and articulate why this topic matters to security outcomes.}
\TaskItem{Complete objective: Diagram the architecture/data flows and identify threat surfaces.}
\TaskItem{Execute lab: Hands-on: Build a small lab demonstrating key concepts in 'Ethical Hacking'. Capture screenshots/notes and one measurable result (e.g., a passing test, alert fired, or control verified).}
\end{itemize}
\end{TasksBox}
\clearpage
\setStoryID{CISH-031}
\setStoryTitle{What Is Vulnerability Assessment? --- Learn \& Lab}
\setEpic{Part 2: Managing Information Security}
\setBusinessValue{Build a working understanding of what is vulnerability assessment? and its place in a modern security program; be able to explain core concepts, map them to the CIA triad, and identify common threats and controls.}
\setPriority{Must}
\setSP{3}
\setPersona{Security Engineer}
\setDependencies{Lab VM or container runtime, Git repo for notes, Markdown/PDF export tool}
\setAssumptions{Time-box chapter to one iteration; open issues captured for later}
\setUserStory{As a Security Engineer, I want to study and practice 'What Is Vulnerability Assessment?' so that I can apply its concepts to reduce risk and improve outcomes.}
\setNonFunctional{Security, Reliability, Performance}
\setScenario{Apply key controls for What Is Vulnerability Assessment?}
\setGiven{the lab environment and topic-specific tools for 'What Is Vulnerability Assessment?' are available}
\setWhen{I execute the hands-on objectives and lab: Hands-on: Build a small lab demonstrating key concepts in 'What Is Vulnerability Assessment?'. Capture screenshots/notes and one measurable result (e.g., a passing test, alert fired, or control verified).}
\setThen{the deliverables are produced (2–3 page brief on 'What Is Vulnerability Assessment?': risks, architecture, controls, and a checklist; plus a one-slide executive summary.); evidence (screenshots/logs/configs) is attached and reviewed}
\setDoR{Persona clear; AC drafted; Dependencies known; Estimate set.}
\setDoD{All ACs pass; Tests green; Security checks; Docs updated; Evidence attached.}
\RenderStoryCard
\begin{TasksBox}
\begin{itemize}
\TaskItem{Draft a one-page chapter plan: scope, objectives, interfaces, success metrics.}
\TaskItem{Set up tools, datasets, and accounts; document versions and configuration.}
\TaskItem{Complete objective: Define key terms and articulate why this topic matters to security outcomes.}
\TaskItem{Complete objective: Diagram the architecture/data flows and identify threat surfaces.}
\TaskItem{Execute lab: Hands-on: Build a small lab demonstrating key concepts in 'What Is Vulnerability Assessment?'. Capture screenshots/notes and one measurable result (e.g., a passing test, alert fired, or control verified).}
\end{itemize}
\end{TasksBox}
\clearpage
\setStoryID{CISH-032}
\setStoryTitle{Security Metrics --- Learn \& Lab}
\setEpic{Part 2: Managing Information Security}
\setBusinessValue{Build a working understanding of security metrics and its place in a modern security program; be able to explain core concepts, map them to the CIA triad, and identify common threats and controls.}
\setPriority{Must}
\setSP{3}
\setPersona{Security Program Manager}
\setDependencies{Lab VM or container runtime, Git repo for notes, Markdown/PDF export tool}
\setAssumptions{Time-box chapter to one iteration; open issues captured for later}
\setUserStory{As a Security Program Manager, I want to study and practice 'Security Metrics' so that I can apply its concepts to reduce risk and improve outcomes.}
\setNonFunctional{Security, Reliability, Performance}
\setScenario{Apply key controls for Security Metrics}
\setGiven{the lab environment and topic-specific tools for 'Security Metrics' are available}
\setWhen{I execute the hands-on objectives and lab: Hands-on: Build a small lab demonstrating key concepts in 'Security Metrics'. Capture screenshots/notes and one measurable result (e.g., a passing test, alert fired, or control verified).}
\setThen{the deliverables are produced (2–3 page brief on 'Security Metrics': risks, architecture, controls, and a checklist; plus a one-slide executive summary.); evidence (screenshots/logs/configs) is attached and reviewed}
\setDoR{Persona clear; AC drafted; Dependencies known; Estimate set.}
\setDoD{All ACs pass; Tests green; Security checks; Docs updated; Evidence attached.}
\RenderStoryCard
\begin{TasksBox}
\begin{itemize}
\TaskItem{Draft a one-page chapter plan: scope, objectives, interfaces, success metrics.}
\TaskItem{Set up tools, datasets, and accounts; document versions and configuration.}
\TaskItem{Complete objective: Define key terms and articulate why this topic matters to security outcomes.}
\TaskItem{Complete objective: Diagram the architecture/data flows and identify threat surfaces.}
\TaskItem{Execute lab: Hands-on: Build a small lab demonstrating key concepts in 'Security Metrics'. Capture screenshots/notes and one measurable result (e.g., a passing test, alert fired, or control verified).}
\end{itemize}
\end{TasksBox}
\clearpage
\setStoryID{CISH-033}
\setStoryTitle{Security Education, Training, and Awareness --- Learn \& Lab}
\setEpic{Part 2: Managing Information Security}
\setBusinessValue{Build a working understanding of security education, training, and awareness and its place in a modern security program; be able to explain core concepts, map them to the CIA triad, and identify common threats and controls.}
\setPriority{Must}
\setSP{3}
\setPersona{Security Program Manager}
\setDependencies{Lab VM or container runtime, Git repo for notes, Markdown/PDF export tool}
\setAssumptions{Time-box chapter to one iteration; open issues captured for later}
\setUserStory{As a Security Program Manager, I want to study and practice 'Security Education, Training, and Awareness' so that I can apply its concepts to reduce risk and improve outcomes.}
\setNonFunctional{Security, Reliability, Performance}
\setScenario{Apply key controls for Security Education, Training, and Awareness}
\setGiven{the lab environment and topic-specific tools for 'Security Education, Training, and Awareness' are available}
\setWhen{I execute the hands-on objectives and lab: Hands-on: Build a small lab demonstrating key concepts in 'Security Education, Training, and Awareness'. Capture screenshots/notes and one measurable result (e.g., a passing test, alert fired, or control verified).}
\setThen{the deliverables are produced (2–3 page brief on 'Security Education, Training, and Awareness': risks, architecture, controls, and a checklist; plus a one-slide executive summary.); evidence (screenshots/logs/configs) is attached and reviewed}
\setDoR{Persona clear; AC drafted; Dependencies known; Estimate set.}
\setDoD{All ACs pass; Tests green; Security checks; Docs updated; Evidence attached.}
\RenderStoryCard
\begin{TasksBox}
\begin{itemize}
\TaskItem{Draft a one-page chapter plan: scope, objectives, interfaces, success metrics.}
\TaskItem{Set up tools, datasets, and accounts; document versions and configuration.}
\TaskItem{Complete objective: Define key terms and articulate why this topic matters to security outcomes.}
\TaskItem{Complete objective: Diagram the architecture/data flows and identify threat surfaces.}
\TaskItem{Execute lab: Hands-on: Build a small lab demonstrating key concepts in 'Security Education, Training, and Awareness'. Capture screenshots/notes and one measurable result (e.g., a passing test, alert fired, or control verified).}
\end{itemize}
\end{TasksBox}
\clearpage
\setStoryID{CISH-034}
\setStoryTitle{Risk Management --- Learn \& Lab}
\setEpic{Part 2: Managing Information Security}
\setBusinessValue{Build a working understanding of risk management and its place in a modern security program; be able to explain core concepts, map them to the CIA triad, and identify common threats and controls.}
\setPriority{Must}
\setSP{3}
\setPersona{Security Program Manager}
\setDependencies{Lab VM or container runtime, Git repo for notes, Markdown/PDF export tool}
\setAssumptions{Time-box chapter to one iteration; open issues captured for later}
\setUserStory{As a Security Program Manager, I want to study and practice 'Risk Management' so that I can apply its concepts to reduce risk and improve outcomes.}
\setNonFunctional{Security, Reliability, Performance}
\setScenario{Apply key controls for Risk Management}
\setGiven{the lab environment and topic-specific tools for 'Risk Management' are available}
\setWhen{I execute the hands-on objectives and lab: Hands-on: Build a small lab demonstrating key concepts in 'Risk Management'. Capture screenshots/notes and one measurable result (e.g., a passing test, alert fired, or control verified).}
\setThen{the deliverables are produced (2–3 page brief on 'Risk Management': risks, architecture, controls, and a checklist; plus a one-slide executive summary.); evidence (screenshots/logs/configs) is attached and reviewed}
\setDoR{Persona clear; AC drafted; Dependencies known; Estimate set.}
\setDoD{All ACs pass; Tests green; Security checks; Docs updated; Evidence attached.}
\RenderStoryCard
\begin{TasksBox}
\begin{itemize}
\TaskItem{Draft a one-page chapter plan: scope, objectives, interfaces, success metrics.}
\TaskItem{Set up tools, datasets, and accounts; document versions and configuration.}
\TaskItem{Complete objective: Define key terms and articulate why this topic matters to security outcomes.}
\TaskItem{Complete objective: Diagram the architecture/data flows and identify threat surfaces.}
\TaskItem{Execute lab: Hands-on: Build a small lab demonstrating key concepts in 'Risk Management'. Capture screenshots/notes and one measurable result (e.g., a passing test, alert fired, or control verified).}
\end{itemize}
\end{TasksBox}
\clearpage
\setStoryID{CISH-035}
\setStoryTitle{Insider Threats --- Learn \& Lab}
\setEpic{Part 2: Managing Information Security}
\setBusinessValue{Build a working understanding of insider threats and its place in a modern security program; be able to explain core concepts, map them to the CIA triad, and identify common threats and controls.}
\setPriority{Must}
\setSP{3}
\setPersona{Security Program Manager}
\setDependencies{Lab VM or container runtime, Git repo for notes, Markdown/PDF export tool}
\setAssumptions{Time-box chapter to one iteration; open issues captured for later}
\setUserStory{As a Security Program Manager, I want to study and practice 'Insider Threats' so that I can apply its concepts to reduce risk and improve outcomes.}
\setNonFunctional{Security, Reliability, Performance}
\setScenario{Apply key controls for Insider Threats}
\setGiven{the lab environment and topic-specific tools for 'Insider Threats' are available}
\setWhen{I execute the hands-on objectives and lab: Hands-on: Build a small lab demonstrating key concepts in 'Insider Threats'. Capture screenshots/notes and one measurable result (e.g., a passing test, alert fired, or control verified).}
\setThen{the deliverables are produced (2–3 page brief on 'Insider Threats': risks, architecture, controls, and a checklist; plus a one-slide executive summary.); evidence (screenshots/logs/configs) is attached and reviewed}
\setDoR{Persona clear; AC drafted; Dependencies known; Estimate set.}
\setDoD{All ACs pass; Tests green; Security checks; Docs updated; Evidence attached.}
\RenderStoryCard
\begin{TasksBox}
\begin{itemize}
\TaskItem{Draft a one-page chapter plan: scope, objectives, interfaces, success metrics.}
\TaskItem{Set up tools, datasets, and accounts; document versions and configuration.}
\TaskItem{Complete objective: Define key terms and articulate why this topic matters to security outcomes.}
\TaskItem{Complete objective: Diagram the architecture/data flows and identify threat surfaces.}
\TaskItem{Execute lab: Hands-on: Build a small lab demonstrating key concepts in 'Insider Threats'. Capture screenshots/notes and one measurable result (e.g., a passing test, alert fired, or control verified).}
\end{itemize}
\end{TasksBox}
\clearpage
\setStoryID{CISH-036}
\setStoryTitle{Disaster Recovery --- Learn \& Lab}
\setEpic{Part 3: Disaster Recovery Security}
\setBusinessValue{Build a working understanding of disaster recovery and its place in a modern security program; be able to explain core concepts, map them to the CIA triad, and identify common threats and controls.}
\setPriority{Must}
\setSP{3}
\setPersona{Security Engineer}
\setDependencies{Lab VM or container runtime, Git repo for notes, Markdown/PDF export tool}
\setAssumptions{Time-box chapter to one iteration; open issues captured for later}
\setUserStory{As a Security Engineer, I want to study and practice 'Disaster Recovery' so that I can apply its concepts to reduce risk and improve outcomes.}
\setNonFunctional{Security, Reliability, Performance}
\setScenario{Apply key controls for Disaster Recovery}
\setGiven{the lab environment and topic-specific tools for 'Disaster Recovery' are available}
\setWhen{I execute the hands-on objectives and lab: Hands-on: Build a small lab demonstrating key concepts in 'Disaster Recovery'. Capture screenshots/notes and one measurable result (e.g., a passing test, alert fired, or control verified).}
\setThen{the deliverables are produced (2–3 page brief on 'Disaster Recovery': risks, architecture, controls, and a checklist; plus a one-slide executive summary.); evidence (screenshots/logs/configs) is attached and reviewed}
\setDoR{Persona clear; AC drafted; Dependencies known; Estimate set.}
\setDoD{All ACs pass; Tests green; Security checks; Docs updated; Evidence attached.}
\RenderStoryCard
\begin{TasksBox}
\begin{itemize}
\TaskItem{Draft a one-page chapter plan: scope, objectives, interfaces, success metrics.}
\TaskItem{Set up tools, datasets, and accounts; document versions and configuration.}
\TaskItem{Complete objective: Define key terms and articulate why this topic matters to security outcomes.}
\TaskItem{Complete objective: Diagram the architecture/data flows and identify threat surfaces.}
\TaskItem{Execute lab: Hands-on: Build a small lab demonstrating key concepts in 'Disaster Recovery'. Capture screenshots/notes and one measurable result (e.g., a passing test, alert fired, or control verified).}
\end{itemize}
\end{TasksBox}
\clearpage
\setStoryID{CISH-037}
\setStoryTitle{Disaster Recovery Plans for Small and Medium Business (SMB) --- Learn \& Lab}
\setEpic{Part 3: Disaster Recovery Security}
\setBusinessValue{Build a working understanding of disaster recovery plans for small and medium business (smb) and its place in a modern security program; be able to explain core concepts, map them to the CIA triad, and identify common threats and controls.}
\setPriority{Must}
\setSP{3}
\setPersona{Network Security Engineer}
\setDependencies{Lab VM or container runtime, Git repo for notes, Markdown/PDF export tool, Packet capture tool (tcpdump/Wireshark), Firewall/router lab}
\setAssumptions{Time-box chapter to one iteration; open issues captured for later}
\setUserStory{As a Network Security Engineer, I want to study and practice 'Disaster Recovery Plans for Small and Medium Business (SMB)' so that I can apply its concepts to reduce risk and improve outcomes.}
\setNonFunctional{Security, Reliability, Performance}
\setScenario{Apply key controls for Disaster Recovery Plans for Small and Medium Business (SMB)}
\setGiven{the lab environment and topic-specific tools for 'Disaster Recovery Plans for Small and Medium Business (SMB)' are available}
\setWhen{I execute the hands-on objectives and lab: Hands-on: Build a small lab demonstrating key concepts in 'Disaster Recovery Plans for Small and Medium Business (SMB)'. Capture screenshots/notes and one measurable result (e.g., a passing test, alert fired, or control verified).}
\setThen{the deliverables are produced (2–3 page brief on 'Disaster Recovery Plans for Small and Medium Business (SMB)': risks, architecture, controls, and a checklist; plus a one-slide executive summary.); evidence (screenshots/logs/configs) is attached and reviewed}
\setDoR{Persona clear; AC drafted; Dependencies known; Estimate set.}
\setDoD{All ACs pass; Tests green; Security checks; Docs updated; Evidence attached.}
\RenderStoryCard
\begin{TasksBox}
\begin{itemize}
\TaskItem{Draft a one-page chapter plan: scope, objectives, interfaces, success metrics.}
\TaskItem{Set up tools, datasets, and accounts; document versions and configuration.}
\TaskItem{Complete objective: Define key terms and articulate why this topic matters to security outcomes.}
\TaskItem{Complete objective: Diagram the architecture/data flows and identify threat surfaces.}
\TaskItem{Execute lab: Hands-on: Build a small lab demonstrating key concepts in 'Disaster Recovery Plans for Small and Medium Business (SMB)'. Capture screenshots/notes and one measurable result (e.g., a passing test, alert fired, or control verified).}
\end{itemize}
\end{TasksBox}
\clearpage
\setStoryID{CISH-038}
\setStoryTitle{Security Certification And Standards Implementation --- Learn \& Lab}
\setEpic{Part 4: Security Standards And Policies}
\setBusinessValue{Build a working understanding of security certification and standards implementation and its place in a modern security program; be able to explain core concepts, map them to the CIA triad, and identify common threats and controls.}
\setPriority{Must}
\setSP{3}
\setPersona{Security Program Manager}
\setDependencies{Lab VM or container runtime, Git repo for notes, Markdown/PDF export tool}
\setAssumptions{Time-box chapter to one iteration; open issues captured for later}
\setUserStory{As a Security Program Manager, I want to study and practice 'Security Certification And Standards Implementation' so that I can apply its concepts to reduce risk and improve outcomes.}
\setNonFunctional{Security, Reliability, Performance, Compliance}
\setScenario{Apply key controls for Security Certification And Standards Implementation}
\setGiven{the lab environment and topic-specific tools for 'Security Certification And Standards Implementation' are available}
\setWhen{I execute the hands-on objectives and lab: Hands-on: Build a small lab demonstrating key concepts in 'Security Certification And Standards Implementation'. Capture screenshots/notes and one measurable result (e.g., a passing test, alert fired, or control verified).}
\setThen{the deliverables are produced (2–3 page brief on 'Security Certification And Standards Implementation': risks, architecture, controls, and a checklist; plus a one-slide executive summary.); evidence (screenshots/logs/configs) is attached and reviewed}
\setDoR{Persona clear; AC drafted; Dependencies known; Estimate set.}
\setDoD{All ACs pass; Tests green; Security checks; Docs updated; Evidence attached.}
\RenderStoryCard
\begin{TasksBox}
\begin{itemize}
\TaskItem{Draft a one-page chapter plan: scope, objectives, interfaces, success metrics.}
\TaskItem{Set up tools, datasets, and accounts; document versions and configuration.}
\TaskItem{Complete objective: Define key terms and articulate why this topic matters to security outcomes.}
\TaskItem{Complete objective: Diagram the architecture/data flows and identify threat surfaces.}
\TaskItem{Execute lab: Hands-on: Build a small lab demonstrating key concepts in 'Security Certification And Standards Implementation'. Capture screenshots/notes and one measurable result (e.g., a passing test, alert fired, or control verified).}
\end{itemize}
\end{TasksBox}
\clearpage
\setStoryID{CISH-039}
\setStoryTitle{Security Policies And Plans Development --- Learn \& Lab}
\setEpic{Part 4: Security Standards And Policies}
\setBusinessValue{Build a working understanding of security policies and plans development and its place in a modern security program; be able to explain core concepts, map them to the CIA triad, and identify common threats and controls.}
\setPriority{Must}
\setSP{3}
\setPersona{Network Security Engineer}
\setDependencies{Lab VM or container runtime, Git repo for notes, Markdown/PDF export tool, Packet capture tool (tcpdump/Wireshark), Firewall/router lab}
\setAssumptions{Time-box chapter to one iteration; open issues captured for later}
\setUserStory{As a Network Security Engineer, I want to study and practice 'Security Policies And Plans Development' so that I can apply its concepts to reduce risk and improve outcomes.}
\setNonFunctional{Security, Reliability, Performance}
\setScenario{Apply key controls for Security Policies And Plans Development}
\setGiven{the lab environment and topic-specific tools for 'Security Policies And Plans Development' are available}
\setWhen{I execute the hands-on objectives and lab: Hands-on: Build a small lab demonstrating key concepts in 'Security Policies And Plans Development'. Capture screenshots/notes and one measurable result (e.g., a passing test, alert fired, or control verified).}
\setThen{the deliverables are produced (2–3 page brief on 'Security Policies And Plans Development': risks, architecture, controls, and a checklist; plus a one-slide executive summary.); evidence (screenshots/logs/configs) is attached and reviewed}
\setDoR{Persona clear; AC drafted; Dependencies known; Estimate set.}
\setDoD{All ACs pass; Tests green; Security checks; Docs updated; Evidence attached.}
\RenderStoryCard
\begin{TasksBox}
\begin{itemize}
\TaskItem{Draft a one-page chapter plan: scope, objectives, interfaces, success metrics.}
\TaskItem{Set up tools, datasets, and accounts; document versions and configuration.}
\TaskItem{Complete objective: Define key terms and articulate why this topic matters to security outcomes.}
\TaskItem{Complete objective: Diagram the architecture/data flows and identify threat surfaces.}
\TaskItem{Execute lab: Hands-on: Build a small lab demonstrating key concepts in 'Security Policies And Plans Development'. Capture screenshots/notes and one measurable result (e.g., a passing test, alert fired, or control verified).}
\end{itemize}
\end{TasksBox}
\clearpage
\setStoryID{CISH-040}
\setStoryTitle{Cyber Forensics --- Learn \& Lab}
\setEpic{Part 5: Cyber, Network, and Systems Forensics Security and Assurance}
\setBusinessValue{Build a working understanding of cyber forensics and its place in a modern security program; be able to explain core concepts, map them to the CIA triad, and identify common threats and controls.}
\setPriority{Must}
\setSP{3}
\setPersona{Security Analyst}
\setDependencies{Lab VM or container runtime, Git repo for notes, Markdown/PDF export tool, Imaging tool (dd/FTK), Hash tool (sha256sum), Write-blocker (emulated ok)}
\setAssumptions{Time-box chapter to one iteration; open issues captured for later}
\setUserStory{As a Security Analyst, I want to study and practice 'Cyber Forensics' so that I can apply its concepts to reduce risk and improve outcomes.}
\setNonFunctional{Security, Reliability, Performance, Compliance, Observability}
\setScenario{Apply key controls for Cyber Forensics}
\setGiven{the lab environment and topic-specific tools for 'Cyber Forensics' are available}
\setWhen{I execute the hands-on objectives and lab: Hands-on: Build a small lab demonstrating key concepts in 'Cyber Forensics'. Capture screenshots/notes and one measurable result (e.g., a passing test, alert fired, or control verified).}
\setThen{the deliverables are produced (2–3 page brief on 'Cyber Forensics': risks, architecture, controls, and a checklist; plus a one-slide executive summary.); evidence (screenshots/logs/configs) is attached and reviewed}
\setDoR{Persona clear; AC drafted; Dependencies known; Estimate set.}
\setDoD{All ACs pass; Tests green; Security checks; Docs updated; Evidence attached.}
\RenderStoryCard
\begin{TasksBox}
\begin{itemize}
\TaskItem{Draft a one-page chapter plan: scope, objectives, interfaces, success metrics.}
\TaskItem{Set up tools, datasets, and accounts; document versions and configuration.}
\TaskItem{Complete objective: Define key terms and articulate why this topic matters to security outcomes.}
\TaskItem{Complete objective: Diagram the architecture/data flows and identify threat surfaces.}
\TaskItem{Execute lab: Hands-on: Build a small lab demonstrating key concepts in 'Cyber Forensics'. Capture screenshots/notes and one measurable result (e.g., a passing test, alert fired, or control verified).}
\end{itemize}
\end{TasksBox}
\clearpage
\setStoryID{CISH-041}
\setStoryTitle{Cyber Forensics And Incidence Response --- Learn \& Lab}
\setEpic{Part 5: Cyber, Network, and Systems Forensics Security and Assurance}
\setBusinessValue{Build a working understanding of cyber forensics and incidence response and its place in a modern security program; be able to explain core concepts, map them to the CIA triad, and identify common threats and controls.}
\setPriority{Must}
\setSP{3}
\setPersona{Security Analyst}
\setDependencies{Lab VM or container runtime, Git repo for notes, Markdown/PDF export tool, Imaging tool (dd/FTK), Hash tool (sha256sum), Write-blocker (emulated ok)}
\setAssumptions{Time-box chapter to one iteration; open issues captured for later}
\setUserStory{As a Security Analyst, I want to study and practice 'Cyber Forensics And Incidence Response' so that I can apply its concepts to reduce risk and improve outcomes.}
\setNonFunctional{Security, Reliability, Performance, Compliance, Observability}
\setScenario{Apply key controls for Cyber Forensics And Incidence Response}
\setGiven{the lab environment and topic-specific tools for 'Cyber Forensics And Incidence Response' are available}
\setWhen{I execute the hands-on objectives and lab: Hands-on: Build a small lab demonstrating key concepts in 'Cyber Forensics And Incidence Response'. Capture screenshots/notes and one measurable result (e.g., a passing test, alert fired, or control verified).}
\setThen{the deliverables are produced (2–3 page brief on 'Cyber Forensics And Incidence Response': risks, architecture, controls, and a checklist; plus a one-slide executive summary.); evidence (screenshots/logs/configs) is attached and reviewed}
\setDoR{Persona clear; AC drafted; Dependencies known; Estimate set.}
\setDoD{All ACs pass; Tests green; Security checks; Docs updated; Evidence attached.}
\RenderStoryCard
\begin{TasksBox}
\begin{itemize}
\TaskItem{Draft a one-page chapter plan: scope, objectives, interfaces, success metrics.}
\TaskItem{Set up tools, datasets, and accounts; document versions and configuration.}
\TaskItem{Complete objective: Define key terms and articulate why this topic matters to security outcomes.}
\TaskItem{Complete objective: Diagram the architecture/data flows and identify threat surfaces.}
\TaskItem{Execute lab: Hands-on: Build a small lab demonstrating key concepts in 'Cyber Forensics And Incidence Response'. Capture screenshots/notes and one measurable result (e.g., a passing test, alert fired, or control verified).}
\end{itemize}
\end{TasksBox}
\clearpage
\setStoryID{CISH-042}
\setStoryTitle{Securing e-Discovery --- Learn \& Lab}
\setEpic{Part 5: Cyber, Network, and Systems Forensics Security and Assurance}
\setBusinessValue{Build a working understanding of securing e-discovery and its place in a modern security program; be able to explain core concepts, map them to the CIA triad, and identify common threats and controls.}
\setPriority{Must}
\setSP{3}
\setPersona{Security Analyst}
\setDependencies{Lab VM or container runtime, Git repo for notes, Markdown/PDF export tool, Imaging tool (dd/FTK), Hash tool (sha256sum), Write-blocker (emulated ok)}
\setAssumptions{Time-box chapter to one iteration; open issues captured for later}
\setUserStory{As a Security Analyst, I want to study and practice 'Securing e-Discovery' so that I can apply its concepts to reduce risk and improve outcomes.}
\setNonFunctional{Security, Reliability, Performance}
\setScenario{Apply key controls for Securing e-Discovery}
\setGiven{the lab environment and topic-specific tools for 'Securing e-Discovery' are available}
\setWhen{I execute the hands-on objectives and lab: Hands-on: Build a small lab demonstrating key concepts in 'Securing e-Discovery'. Capture screenshots/notes and one measurable result (e.g., a passing test, alert fired, or control verified).}
\setThen{the deliverables are produced (2–3 page brief on 'Securing e-Discovery': risks, architecture, controls, and a checklist; plus a one-slide executive summary.); evidence (screenshots/logs/configs) is attached and reviewed}
\setDoR{Persona clear; AC drafted; Dependencies known; Estimate set.}
\setDoD{All ACs pass; Tests green; Security checks; Docs updated; Evidence attached.}
\RenderStoryCard
\begin{TasksBox}
\begin{itemize}
\TaskItem{Draft a one-page chapter plan: scope, objectives, interfaces, success metrics.}
\TaskItem{Set up tools, datasets, and accounts; document versions and configuration.}
\TaskItem{Complete objective: Define key terms and articulate why this topic matters to security outcomes.}
\TaskItem{Complete objective: Diagram the architecture/data flows and identify threat surfaces.}
\TaskItem{Execute lab: Hands-on: Build a small lab demonstrating key concepts in 'Securing e-Discovery'. Capture screenshots/notes and one measurable result (e.g., a passing test, alert fired, or control verified).}
\end{itemize}
\end{TasksBox}
\clearpage
\setStoryID{CISH-043}
\setStoryTitle{Network Forensics --- Learn \& Lab}
\setEpic{Part 5: Cyber, Network, and Systems Forensics Security and Assurance}
\setBusinessValue{Build a working understanding of network forensics and its place in a modern security program; be able to explain core concepts, map them to the CIA triad, and identify common threats and controls.}
\setPriority{Must}
\setSP{3}
\setPersona{Security Analyst}
\setDependencies{Lab VM or container runtime, Git repo for notes, Markdown/PDF export tool, Imaging tool (dd/FTK), Hash tool (sha256sum), Write-blocker (emulated ok), Packet capture tool (tcpdump/Wireshark), Firewall/router lab}
\setAssumptions{Time-box chapter to one iteration; open issues captured for later}
\setUserStory{As a Security Analyst, I want to study and practice 'Network Forensics' so that I can apply its concepts to reduce risk and improve outcomes.}
\setNonFunctional{Security, Reliability, Performance, Compliance, Observability}
\setScenario{Apply key controls for Network Forensics}
\setGiven{the lab environment and topic-specific tools for 'Network Forensics' are available}
\setWhen{I execute the hands-on objectives and lab: Hands-on: Build a small lab demonstrating key concepts in 'Network Forensics'. Capture screenshots/notes and one measurable result (e.g., a passing test, alert fired, or control verified).}
\setThen{the deliverables are produced (2–3 page brief on 'Network Forensics': risks, architecture, controls, and a checklist; plus a one-slide executive summary.); evidence (screenshots/logs/configs) is attached and reviewed}
\setDoR{Persona clear; AC drafted; Dependencies known; Estimate set.}
\setDoD{All ACs pass; Tests green; Security checks; Docs updated; Evidence attached.}
\RenderStoryCard
\begin{TasksBox}
\begin{itemize}
\TaskItem{Draft a one-page chapter plan: scope, objectives, interfaces, success metrics.}
\TaskItem{Set up tools, datasets, and accounts; document versions and configuration.}
\TaskItem{Complete objective: Define key terms and articulate why this topic matters to security outcomes.}
\TaskItem{Complete objective: Diagram the architecture/data flows and identify threat surfaces.}
\TaskItem{Execute lab: Hands-on: Build a small lab demonstrating key concepts in 'Network Forensics'. Capture screenshots/notes and one measurable result (e.g., a passing test, alert fired, or control verified).}
\end{itemize}
\end{TasksBox}

\setStoryID{CISH-044}
\setStoryTitle{Microsoft Office and Metadata Forensics: A Deeper Dive --- Learn \& Lab}
\setEpic{Part 5: Cyber, Network, and Systems Forensics Security and Assurance}
\setBusinessValue{Build a working understanding of microsoft office and metadata forensics: a deeper dive and its place in a modern security program; be able to explain core concepts, map them to the CIA triad, and identify common threats and controls.}
\setPriority{Must}
\setSP{3}
\setPersona{Security Analyst}
\setDependencies{Lab VM or container runtime, Git repo for notes, Markdown/PDF export tool, Imaging tool (dd/FTK), Hash tool (sha256sum), Write-blocker (emulated ok)}
\setAssumptions{Time-box chapter to one iteration; open issues captured for later}
\setUserStory{As a Security Analyst, I want to study and practice 'Microsoft Office and Metadata Forensics: A Deeper Dive' so that I can apply its concepts to reduce risk and improve outcomes.}
\setNonFunctional{Security, Reliability, Performance, Compliance, Observability}
\setScenario{Apply key controls for Microsoft Office and Metadata Forensics: A Deeper Dive}
\setGiven{the lab environment and topic-specific tools for 'Microsoft Office and Metadata Forensics: A Deeper Dive' are available}
\setWhen{I execute the hands-on objectives and lab: Hands-on: Build a small lab demonstrating key concepts in 'Microsoft Office and Metadata Forensics: A Deeper Dive'. Capture screenshots/notes and one measurable result (e.g., a passing test, alert fired, or control verified).}
\setThen{the deliverables are produced (2–3 page brief on 'Microsoft Office and Metadata Forensics: A Deeper Dive': risks, architecture, controls, and a checklist; plus a one-slide executive summary.); evidence (screenshots/logs/configs) is attached and reviewed}
\setDoR{Persona clear; AC drafted; Dependencies known; Estimate set.}
\setDoD{All ACs pass; Tests green; Security checks; Docs updated; Evidence attached.}
\RenderStoryCard
\begin{TasksBox}
\begin{itemize}
\TaskItem{Draft a one-page chapter plan: scope, objectives, interfaces, success metrics.}
\TaskItem{Set up tools, datasets, and accounts; document versions and configuration.}
\TaskItem{Complete objective: Define key terms and articulate why this topic matters to security outcomes.}
\TaskItem{Complete objective: Diagram the architecture/data flows and identify threat surfaces.}
\TaskItem{Execute lab: Hands-on: Build a small lab demonstrating key concepts in 'Microsoft Office and Metadata Forensics: A Deeper Dive'. Capture screenshots/notes and one measurable result (e.g., a passing test, alert fired, or control verified).}
\end{itemize}
\end{TasksBox}
\clearpage
\setStoryID{CISH-045}
\setStoryTitle{Hard Drive Imaging --- Learn \& Lab}
\setEpic{Part 5: Cyber, Network, and Systems Forensics Security and Assurance}
\setBusinessValue{Build a working understanding of hard drive imaging and its place in a modern security program; be able to explain core concepts, map them to the CIA triad, and identify common threats and controls.}
\setPriority{Must}
\setSP{3}
\setPersona{Security Analyst}
\setDependencies{Lab VM or container runtime, Git repo for notes, Markdown/PDF export tool, Imaging tool (dd/FTK), Hash tool (sha256sum), Write-blocker (emulated ok)}
\setAssumptions{Time-box chapter to one iteration; open issues captured for later}
\setUserStory{As a Security Analyst, I want to study and practice 'Hard Drive Imaging' so that I can apply its concepts to reduce risk and improve outcomes.}
\setNonFunctional{Security, Reliability, Performance}
\setScenario{Apply key controls for Hard Drive Imaging}
\setGiven{the lab environment and topic-specific tools for 'Hard Drive Imaging' are available}
\setWhen{I execute the hands-on objectives and lab: Hands-on: Build a small lab demonstrating key concepts in 'Hard Drive Imaging'. Capture screenshots/notes and one measurable result (e.g., a passing test, alert fired, or control verified).}
\setThen{the deliverables are produced (2–3 page brief on 'Hard Drive Imaging': risks, architecture, controls, and a checklist; plus a one-slide executive summary.); evidence (screenshots/logs/configs) is attached and reviewed}
\setDoR{Persona clear; AC drafted; Dependencies known; Estimate set.}
\setDoD{All ACs pass; Tests green; Security checks; Docs updated; Evidence attached.}
\RenderStoryCard
\begin{TasksBox}
\begin{itemize}
\TaskItem{Draft a one-page chapter plan: scope, objectives, interfaces, success metrics.}
\TaskItem{Set up tools, datasets, and accounts; document versions and configuration.}
\TaskItem{Complete objective: Define key terms and articulate why this topic matters to security outcomes.}
\TaskItem{Complete objective: Diagram the architecture/data flows and identify threat surfaces.}
\TaskItem{Execute lab: Hands-on: Build a small lab demonstrating key concepts in 'Hard Drive Imaging'. Capture screenshots/notes and one measurable result (e.g., a passing test, alert fired, or control verified).}
\end{itemize}
\end{TasksBox}
\clearpage
\setStoryID{CISH-046}
\setStoryTitle{Data Encryption --- Learn \& Lab}
\setEpic{Part 6: Encryption Technology}
\setBusinessValue{Build a working understanding of data encryption and its place in a modern security program; be able to explain core concepts, map them to the CIA triad, and identify common threats and controls.}
\setPriority{Must}
\setSP{3}
\setPersona{Platform Engineer}
\setDependencies{Lab VM or container runtime, Git repo for notes, Markdown/PDF export tool, OpenSSL/mkcert, TLS scanner}
\setAssumptions{Time-box chapter to one iteration; open issues captured for later}
\setUserStory{As a Platform Engineer, I want to study and practice 'Data Encryption' so that I can apply its concepts to reduce risk and improve outcomes.}
\setNonFunctional{Security, Reliability, Performance}
\setScenario{Apply key controls for Data Encryption}
\setGiven{the lab environment and topic-specific tools for 'Data Encryption' are available}
\setWhen{I execute the hands-on objectives and lab: Hands-on: Build a small lab demonstrating key concepts in 'Data Encryption'. Capture screenshots/notes and one measurable result (e.g., a passing test, alert fired, or control verified).}
\setThen{the deliverables are produced (2–3 page brief on 'Data Encryption': risks, architecture, controls, and a checklist; plus a one-slide executive summary.); evidence (screenshots/logs/configs) is attached and reviewed}
\setDoR{Persona clear; AC drafted; Dependencies known; Estimate set.}
\setDoD{All ACs pass; Tests green; Security checks; Docs updated; Evidence attached.}
\RenderStoryCard
\begin{TasksBox}
\begin{itemize}
\TaskItem{Draft a one-page chapter plan: scope, objectives, interfaces, success metrics.}
\TaskItem{Set up tools, datasets, and accounts; document versions and configuration.}
\TaskItem{Complete objective: Define key terms and articulate why this topic matters to security outcomes.}
\TaskItem{Complete objective: Diagram the architecture/data flows and identify threat surfaces.}
\TaskItem{Execute lab: Hands-on: Build a small lab demonstrating key concepts in 'Data Encryption'. Capture screenshots/notes and one measurable result (e.g., a passing test, alert fired, or control verified).}
\end{itemize}
\end{TasksBox}
\clearpage
\setStoryID{CISH-047}
\setStoryTitle{Satellite Encryption --- Learn \& Lab}
\setEpic{Part 6: Encryption Technology}
\setBusinessValue{Build a working understanding of satellite encryption and its place in a modern security program; be able to explain core concepts, map them to the CIA triad, and identify common threats and controls.}
\setPriority{Must}
\setSP{3}
\setPersona{Platform Engineer}
\setDependencies{Lab VM or container runtime, Git repo for notes, Markdown/PDF export tool, OpenSSL/mkcert, TLS scanner}
\setAssumptions{Time-box chapter to one iteration; open issues captured for later}
\setUserStory{As a Platform Engineer, I want to study and practice 'Satellite Encryption' so that I can apply its concepts to reduce risk and improve outcomes.}
\setNonFunctional{Security, Reliability, Performance}
\setScenario{Apply key controls for Satellite Encryption}
\setGiven{the lab environment and topic-specific tools for 'Satellite Encryption' are available}
\setWhen{I execute the hands-on objectives and lab: Hands-on: Build a small lab demonstrating key concepts in 'Satellite Encryption'. Capture screenshots/notes and one measurable result (e.g., a passing test, alert fired, or control verified).}
\setThen{the deliverables are produced (2–3 page brief on 'Satellite Encryption': risks, architecture, controls, and a checklist; plus a one-slide executive summary.); evidence (screenshots/logs/configs) is attached and reviewed}
\setDoR{Persona clear; AC drafted; Dependencies known; Estimate set.}
\setDoD{All ACs pass; Tests green; Security checks; Docs updated; Evidence attached.}
\RenderStoryCard
\begin{TasksBox}
\begin{itemize}
\TaskItem{Draft a one-page chapter plan: scope, objectives, interfaces, success metrics.}
\TaskItem{Set up tools, datasets, and accounts; document versions and configuration.}
\TaskItem{Complete objective: Define key terms and articulate why this topic matters to security outcomes.}
\TaskItem{Complete objective: Diagram the architecture/data flows and identify threat surfaces.}
\TaskItem{Execute lab: Hands-on: Build a small lab demonstrating key concepts in 'Satellite Encryption'. Capture screenshots/notes and one measurable result (e.g., a passing test, alert fired, or control verified).}
\end{itemize}
\end{TasksBox}
\clearpage
\setStoryID{CISH-048}
\setStoryTitle{Public Key Infrastructure --- Learn \& Lab}
\setEpic{Part 6: Encryption Technology}
\setBusinessValue{Build a working understanding of public key infrastructure and its place in a modern security program; be able to explain core concepts, map them to the CIA triad, and identify common threats and controls.}
\setPriority{Must}
\setSP{3}
\setPersona{Platform Engineer}
\setDependencies{Lab VM or container runtime, Git repo for notes, Markdown/PDF export tool, OpenSSL/mkcert, TLS scanner}
\setAssumptions{Time-box chapter to one iteration; open issues captured for later}
\setUserStory{As a Platform Engineer, I want to study and practice 'Public Key Infrastructure' so that I can apply its concepts to reduce risk and improve outcomes.}
\setNonFunctional{Security, Reliability, Performance}
\setScenario{Apply key controls for Public Key Infrastructure}
\setGiven{the lab environment and topic-specific tools for 'Public Key Infrastructure' are available}
\setWhen{I execute the hands-on objectives and lab: Hands-on: Build a small lab demonstrating key concepts in 'Public Key Infrastructure'. Capture screenshots/notes and one measurable result (e.g., a passing test, alert fired, or control verified).}
\setThen{the deliverables are produced (2–3 page brief on 'Public Key Infrastructure': risks, architecture, controls, and a checklist; plus a one-slide executive summary.); evidence (screenshots/logs/configs) is attached and reviewed}
\setDoR{Persona clear; AC drafted; Dependencies known; Estimate set.}
\setDoD{All ACs pass; Tests green; Security checks; Docs updated; Evidence attached.}
\RenderStoryCard
\begin{TasksBox}
\begin{itemize}
\TaskItem{Draft a one-page chapter plan: scope, objectives, interfaces, success metrics.}
\TaskItem{Set up tools, datasets, and accounts; document versions and configuration.}
\TaskItem{Complete objective: Define key terms and articulate why this topic matters to security outcomes.}
\TaskItem{Complete objective: Diagram the architecture/data flows and identify threat surfaces.}
\TaskItem{Execute lab: Hands-on: Build a small lab demonstrating key concepts in 'Public Key Infrastructure'. Capture screenshots/notes and one measurable result (e.g., a passing test, alert fired, or control verified).}
\end{itemize}
\end{TasksBox}
\clearpage
\setStoryID{CISH-049}
\setStoryTitle{Password-based Authenticated Key Establishment Protocols --- Learn \& Lab}
\setEpic{Part 6: Encryption Technology}
\setBusinessValue{Build a working understanding of password-based authenticated key establishment protocols and its place in a modern security program; be able to explain core concepts, map them to the CIA triad, and identify common threats and controls.}
\setPriority{Must}
\setSP{3}
\setPersona{Platform Engineer}
\setDependencies{Lab VM or container runtime, Git repo for notes, Markdown/PDF export tool, OpenSSL/mkcert, TLS scanner}
\setAssumptions{Time-box chapter to one iteration; open issues captured for later}
\setUserStory{As a Platform Engineer, I want to study and practice 'Password-based Authenticated Key Establishment Protocols' so that I can apply its concepts to reduce risk and improve outcomes.}
\setNonFunctional{Security, Reliability, Performance}
\setScenario{Apply key controls for Password-based Authenticated Key Establishment Protocols}
\setGiven{the lab environment and topic-specific tools for 'Password-based Authenticated Key Establishment Protocols' are available}
\setWhen{I execute the hands-on objectives and lab: Hands-on: Build a small lab demonstrating key concepts in 'Password-based Authenticated Key Establishment Protocols'. Capture screenshots/notes and one measurable result (e.g., a passing test, alert fired, or control verified).}
\setThen{the deliverables are produced (2–3 page brief on 'Password-based Authenticated Key Establishment Protocols': risks, architecture, controls, and a checklist; plus a one-slide executive summary.); evidence (screenshots/logs/configs) is attached and reviewed}
\setDoR{Persona clear; AC drafted; Dependencies known; Estimate set.}
\setDoD{All ACs pass; Tests green; Security checks; Docs updated; Evidence attached.}
\RenderStoryCard
\begin{TasksBox}
\begin{itemize}
\TaskItem{Draft a one-page chapter plan: scope, objectives, interfaces, success metrics.}
\TaskItem{Set up tools, datasets, and accounts; document versions and configuration.}
\TaskItem{Complete objective: Define key terms and articulate why this topic matters to security outcomes.}
\TaskItem{Complete objective: Diagram the architecture/data flows and identify threat surfaces.}
\TaskItem{Execute lab: Hands-on: Build a small lab demonstrating key concepts in 'Password-based Authenticated Key Establishment Protocols'. Capture screenshots/notes and one measurable result (e.g., a passing test, alert fired, or control verified).}
\end{itemize}
\end{TasksBox}
\clearpage
\setStoryID{CISH-050}
\setStoryTitle{Context-Aware Multifactor Authentication Survey --- Learn \& Lab}
\setEpic{Part 6: Encryption Technology}
\setBusinessValue{Build a working understanding of context-aware multifactor authentication survey and its place in a modern security program; be able to explain core concepts, map them to the CIA triad, and identify common threats and controls.}
\setPriority{Must}
\setSP{3}
\setPersona{Platform Engineer}
\setDependencies{Lab VM or container runtime, Git repo for notes, Markdown/PDF export tool, IDP / MFA-capable test app}
\setAssumptions{Time-box chapter to one iteration; open issues captured for later}
\setUserStory{As a Platform Engineer, I want to study and practice 'Context-Aware Multifactor Authentication Survey' so that I can apply its concepts to reduce risk and improve outcomes.}
\setNonFunctional{Security, Reliability, Performance}
\setScenario{Apply key controls for Context-Aware Multifactor Authentication Survey}
\setGiven{the lab environment and topic-specific tools for 'Context-Aware Multifactor Authentication Survey' are available}
\setWhen{I execute the hands-on objectives and lab: Hands-on: Build a small lab demonstrating key concepts in 'Context-Aware Multifactor Authentication Survey'. Capture screenshots/notes and one measurable result (e.g., a passing test, alert fired, or control verified).}
\setThen{the deliverables are produced (2–3 page brief on 'Context-Aware Multifactor Authentication Survey': risks, architecture, controls, and a checklist; plus a one-slide executive summary.); evidence (screenshots/logs/configs) is attached and reviewed}
\setDoR{Persona clear; AC drafted; Dependencies known; Estimate set.}
\setDoD{All ACs pass; Tests green; Security checks; Docs updated; Evidence attached.}
\RenderStoryCard
\begin{TasksBox}
\begin{itemize}
\TaskItem{Draft a one-page chapter plan: scope, objectives, interfaces, success metrics.}
\TaskItem{Set up tools, datasets, and accounts; document versions and configuration.}
\TaskItem{Complete objective: Define key terms and articulate why this topic matters to security outcomes.}
\TaskItem{Complete objective: Diagram the architecture/data flows and identify threat surfaces.}
\TaskItem{Execute lab: Hands-on: Build a small lab demonstrating key concepts in 'Context-Aware Multifactor Authentication Survey'. Capture screenshots/notes and one measurable result (e.g., a passing test, alert fired, or control verified).}
\end{itemize}
\end{TasksBox}
\clearpage
\setStoryID{CISH-051}
\setStoryTitle{Instant-Messaging Security --- Learn \& Lab}
\setEpic{Part 6: Encryption Technology}
\setBusinessValue{Build a working understanding of instant-messaging security and its place in a modern security program; be able to explain core concepts, map them to the CIA triad, and identify common threats and controls.}
\setPriority{Must}
\setSP{3}
\setPersona{Security Engineer}
\setDependencies{Lab VM or container runtime, Git repo for notes, Markdown/PDF export tool}
\setAssumptions{Time-box chapter to one iteration; open issues captured for later}
\setUserStory{As a Security Engineer, I want to study and practice 'Instant-Messaging Security' so that I can apply its concepts to reduce risk and improve outcomes.}
\setNonFunctional{Security, Reliability, Performance}
\setScenario{Apply key controls for Instant-Messaging Security}
\setGiven{the lab environment and topic-specific tools for 'Instant-Messaging Security' are available}
\setWhen{I execute the hands-on objectives and lab: Hands-on: Build a small lab demonstrating key concepts in 'Instant-Messaging Security'. Capture screenshots/notes and one measurable result (e.g., a passing test, alert fired, or control verified).}
\setThen{the deliverables are produced (2–3 page brief on 'Instant-Messaging Security': risks, architecture, controls, and a checklist; plus a one-slide executive summary.); evidence (screenshots/logs/configs) is attached and reviewed}
\setDoR{Persona clear; AC drafted; Dependencies known; Estimate set.}
\setDoD{All ACs pass; Tests green; Security checks; Docs updated; Evidence attached.}
\RenderStoryCard
\begin{TasksBox}
\begin{itemize}
\TaskItem{Draft a one-page chapter plan: scope, objectives, interfaces, success metrics.}
\TaskItem{Set up tools, datasets, and accounts; document versions and configuration.}
\TaskItem{Complete objective: Define key terms and articulate why this topic matters to security outcomes.}
\TaskItem{Complete objective: Diagram the architecture/data flows and identify threat surfaces.}
\TaskItem{Execute lab: Hands-on: Build a small lab demonstrating key concepts in 'Instant-Messaging Security'. Capture screenshots/notes and one measurable result (e.g., a passing test, alert fired, or control verified).}
\end{itemize}
\end{TasksBox}
\clearpage
\setStoryID{CISH-052}
\setStoryTitle{Online Privacy --- Learn \& Lab}
\setEpic{Part 7: Privacy and Access Management}
\setBusinessValue{Build a working understanding of online privacy and its place in a modern security program; be able to explain core concepts, map them to the CIA triad, and identify common threats and controls.}
\setPriority{Must}
\setSP{3}
\setPersona{Privacy Engineer}
\setDependencies{Lab VM or container runtime, Git repo for notes, Markdown/PDF export tool}
\setAssumptions{Time-box chapter to one iteration; open issues captured for later}
\setUserStory{As a Privacy Engineer, I want to study and practice 'Online Privacy' so that I can apply its concepts to reduce risk and improve outcomes.}
\setNonFunctional{Security, Reliability, Performance, Privacy}
\setScenario{Apply key controls for Online Privacy}
\setGiven{the lab environment and topic-specific tools for 'Online Privacy' are available}
\setWhen{I execute the hands-on objectives and lab: Hands-on: Build a small lab demonstrating key concepts in 'Online Privacy'. Capture screenshots/notes and one measurable result (e.g., a passing test, alert fired, or control verified).}
\setThen{the deliverables are produced (2–3 page brief on 'Online Privacy': risks, architecture, controls, and a checklist; plus a one-slide executive summary.); evidence (screenshots/logs/configs) is attached and reviewed}
\setDoR{Persona clear; AC drafted; Dependencies known; Estimate set.}
\setDoD{All ACs pass; Tests green; Security checks; Docs updated; Evidence attached.}
\RenderStoryCard
\begin{TasksBox}
\begin{itemize}
\TaskItem{Draft a one-page chapter plan: scope, objectives, interfaces, success metrics.}
\TaskItem{Set up tools, datasets, and accounts; document versions and configuration.}
\TaskItem{Complete objective: Define key terms and articulate why this topic matters to security outcomes.}
\TaskItem{Complete objective: Diagram the architecture/data flows and identify threat surfaces.}
\TaskItem{Execute lab: Hands-on: Build a small lab demonstrating key concepts in 'Online Privacy'. Capture screenshots/notes and one measurable result (e.g., a passing test, alert fired, or control verified).}
\end{itemize}
\end{TasksBox}
\clearpage
\setStoryID{CISH-053}
\setStoryTitle{Privacy-enhancing Technologies --- Learn \& Lab}
\setEpic{Part 7: Privacy and Access Management}
\setBusinessValue{Build a working understanding of privacy-enhancing technologies and its place in a modern security program; be able to explain core concepts, map them to the CIA triad, and identify common threats and controls.}
\setPriority{Must}
\setSP{3}
\setPersona{Privacy Engineer}
\setDependencies{Lab VM or container runtime, Git repo for notes, Markdown/PDF export tool}
\setAssumptions{Time-box chapter to one iteration; open issues captured for later}
\setUserStory{As a Privacy Engineer, I want to study and practice 'Privacy-enhancing Technologies' so that I can apply its concepts to reduce risk and improve outcomes.}
\setNonFunctional{Security, Reliability, Performance, Privacy}
\setScenario{Apply key controls for Privacy-enhancing Technologies}
\setGiven{the lab environment and topic-specific tools for 'Privacy-enhancing Technologies' are available}
\setWhen{I execute the hands-on objectives and lab: Hands-on: Build a small lab demonstrating key concepts in 'Privacy-enhancing Technologies'. Capture screenshots/notes and one measurable result (e.g., a passing test, alert fired, or control verified).}
\setThen{the deliverables are produced (2–3 page brief on 'Privacy-enhancing Technologies': risks, architecture, controls, and a checklist; plus a one-slide executive summary.); evidence (screenshots/logs/configs) is attached and reviewed}
\setDoR{Persona clear; AC drafted; Dependencies known; Estimate set.}
\setDoD{All ACs pass; Tests green; Security checks; Docs updated; Evidence attached.}
\RenderStoryCard
\begin{TasksBox}
\begin{itemize}
\TaskItem{Draft a one-page chapter plan: scope, objectives, interfaces, success metrics.}
\TaskItem{Set up tools, datasets, and accounts; document versions and configuration.}
\TaskItem{Complete objective: Define key terms and articulate why this topic matters to security outcomes.}
\TaskItem{Complete objective: Diagram the architecture/data flows and identify threat surfaces.}
\TaskItem{Execute lab: Hands-on: Build a small lab demonstrating key concepts in 'Privacy-enhancing Technologies'. Capture screenshots/notes and one measurable result (e.g., a passing test, alert fired, or control verified).}
\end{itemize}
\end{TasksBox}
\clearpage
\setStoryID{CISH-054}
\setStoryTitle{Personal Privacy Policies --- Learn \& Lab}
\setEpic{Part 7: Privacy and Access Management}
\setBusinessValue{Build a working understanding of personal privacy policies and its place in a modern security program; be able to explain core concepts, map them to the CIA triad, and identify common threats and controls.}
\setPriority{Must}
\setSP{3}
\setPersona{Privacy Engineer}
\setDependencies{Lab VM or container runtime, Git repo for notes, Markdown/PDF export tool}
\setAssumptions{Time-box chapter to one iteration; open issues captured for later}
\setUserStory{As a Privacy Engineer, I want to study and practice 'Personal Privacy Policies' so that I can apply its concepts to reduce risk and improve outcomes.}
\setNonFunctional{Security, Reliability, Performance, Privacy}
\setScenario{Apply key controls for Personal Privacy Policies}
\setGiven{the lab environment and topic-specific tools for 'Personal Privacy Policies' are available}
\setWhen{I execute the hands-on objectives and lab: Hands-on: Build a small lab demonstrating key concepts in 'Personal Privacy Policies'. Capture screenshots/notes and one measurable result (e.g., a passing test, alert fired, or control verified).}
\setThen{the deliverables are produced (2–3 page brief on 'Personal Privacy Policies': risks, architecture, controls, and a checklist; plus a one-slide executive summary.); evidence (screenshots/logs/configs) is attached and reviewed}
\setDoR{Persona clear; AC drafted; Dependencies known; Estimate set.}
\setDoD{All ACs pass; Tests green; Security checks; Docs updated; Evidence attached.}
\RenderStoryCard
\begin{TasksBox}
\begin{itemize}
\TaskItem{Draft a one-page chapter plan: scope, objectives, interfaces, success metrics.}
\TaskItem{Set up tools, datasets, and accounts; document versions and configuration.}
\TaskItem{Complete objective: Define key terms and articulate why this topic matters to security outcomes.}
\TaskItem{Complete objective: Diagram the architecture/data flows and identify threat surfaces.}
\TaskItem{Execute lab: Hands-on: Build a small lab demonstrating key concepts in 'Personal Privacy Policies'. Capture screenshots/notes and one measurable result (e.g., a passing test, alert fired, or control verified).}
\end{itemize}
\end{TasksBox}
\clearpage
\setStoryID{CISH-055}
\setStoryTitle{Detection Of Conflicts In Security Policies --- Learn \& Lab}
\setEpic{Part 7: Privacy and Access Management}
\setBusinessValue{Build a working understanding of detection of conflicts in security policies and its place in a modern security program; be able to explain core concepts, map them to the CIA triad, and identify common threats and controls.}
\setPriority{Must}
\setSP{3}
\setPersona{Security Engineer}
\setDependencies{Lab VM or container runtime, Git repo for notes, Markdown/PDF export tool}
\setAssumptions{Time-box chapter to one iteration; open issues captured for later}
\setUserStory{As a Security Engineer, I want to study and practice 'Detection Of Conflicts In Security Policies' so that I can apply its concepts to reduce risk and improve outcomes.}
\setNonFunctional{Security, Reliability, Performance, Observability}
\setScenario{Apply key controls for Detection Of Conflicts In Security Policies}
\setGiven{the lab environment and topic-specific tools for 'Detection Of Conflicts In Security Policies' are available}
\setWhen{I execute the hands-on objectives and lab: Hands-on: Build a small lab demonstrating key concepts in 'Detection Of Conflicts In Security Policies'. Capture screenshots/notes and one measurable result (e.g., a passing test, alert fired, or control verified).}
\setThen{the deliverables are produced (2–3 page brief on 'Detection Of Conflicts In Security Policies': risks, architecture, controls, and a checklist; plus a one-slide executive summary.); evidence (screenshots/logs/configs) is attached and reviewed}
\setDoR{Persona clear; AC drafted; Dependencies known; Estimate set.}
\setDoD{All ACs pass; Tests green; Security checks; Docs updated; Evidence attached.}
\RenderStoryCard
\begin{TasksBox}
\begin{itemize}
\TaskItem{Draft a one-page chapter plan: scope, objectives, interfaces, success metrics.}
\TaskItem{Set up tools, datasets, and accounts; document versions and configuration.}
\TaskItem{Complete objective: Define key terms and articulate why this topic matters to security outcomes.}
\TaskItem{Complete objective: Diagram the architecture/data flows and identify threat surfaces.}
\TaskItem{Execute lab: Hands-on: Build a small lab demonstrating key concepts in 'Detection Of Conflicts In Security Policies'. Capture screenshots/notes and one measurable result (e.g., a passing test, alert fired, or control verified).}
\end{itemize}
\end{TasksBox}
\clearpage
\setStoryID{CISH-056}
\setStoryTitle{Supporting User Privacy Preferences in Digital Interactions --- Learn \& Lab}
\setEpic{Part 7: Privacy and Access Management}
\setBusinessValue{Build a working understanding of supporting user privacy preferences in digital interactions and its place in a modern security program; be able to explain core concepts, map them to the CIA triad, and identify common threats and controls.}
\setPriority{Must}
\setSP{3}
\setPersona{Privacy Engineer}
\setDependencies{Lab VM or container runtime, Git repo for notes, Markdown/PDF export tool}
\setAssumptions{Time-box chapter to one iteration; open issues captured for later}
\setUserStory{As a Privacy Engineer, I want to study and practice 'Supporting User Privacy Preferences in Digital Interactions' so that I can apply its concepts to reduce risk and improve outcomes.}
\setNonFunctional{Security, Reliability, Performance, Privacy}
\setScenario{Apply key controls for Supporting User Privacy Preferences in Digital Interactions}
\setGiven{the lab environment and topic-specific tools for 'Supporting User Privacy Preferences in Digital Interactions' are available}
\setWhen{I execute the hands-on objectives and lab: Hands-on: Build a small lab demonstrating key concepts in 'Supporting User Privacy Preferences in Digital Interactions'. Capture screenshots/notes and one measurable result (e.g., a passing test, alert fired, or control verified).}
\setThen{the deliverables are produced (2–3 page brief on 'Supporting User Privacy Preferences in Digital Interactions': risks, architecture, controls, and a checklist; plus a one-slide executive summary.); evidence (screenshots/logs/configs) is attached and reviewed}
\setDoR{Persona clear; AC drafted; Dependencies known; Estimate set.}
\setDoD{All ACs pass; Tests green; Security checks; Docs updated; Evidence attached.}
\RenderStoryCard
\begin{TasksBox}
\begin{itemize}
\TaskItem{Draft a one-page chapter plan: scope, objectives, interfaces, success metrics.}
\TaskItem{Set up tools, datasets, and accounts; document versions and configuration.}
\TaskItem{Complete objective: Define key terms and articulate why this topic matters to security outcomes.}
\TaskItem{Complete objective: Diagram the architecture/data flows and identify threat surfaces.}
\TaskItem{Execute lab: Hands-on: Build a small lab demonstrating key concepts in 'Supporting User Privacy Preferences in Digital Interactions'. Capture screenshots/notes and one measurable result (e.g., a passing test, alert fired, or control verified).}
\end{itemize}
\end{TasksBox}


\setStoryID{CISH-057}
\setStoryTitle{Privacy and Security in Environmental Monitoring Systems: Issues and Solutions --- Learn \& Lab}
\setEpic{Part 7: Privacy and Access Management}
\setBusinessValue{Build a working understanding of privacy and security in environmental monitoring systems: issues and solutions and its place in a modern security program; be able to explain core concepts, map them to the CIA triad, and identify common threats and controls.}
\setPriority{Must}
\setSP{3}
\setPersona{Privacy Engineer}
\setDependencies{Lab VM or container runtime, Git repo for notes, Markdown/PDF export tool}
\setAssumptions{Time-box chapter to one iteration; open issues captured for later}
\setUserStory{As a Privacy Engineer, I want to study and practice 'Privacy and Security in Environmental Monitoring Systems: Issues and Solutions' so that I can apply its concepts to reduce risk and improve outcomes.}
\setNonFunctional{Security, Reliability, Performance, Privacy, Observability}
\setScenario{Apply key controls for Privacy and Security in Environmental Monitoring Systems: Issues and Solutions}
\setGiven{the lab environment and topic-specific tools for 'Privacy and Security in Environmental Monitoring Systems: Issues and Solutions' are available}
\setWhen{I execute the hands-on objectives and lab: Hands-on: Build a small lab demonstrating key concepts in 'Privacy and Security in Environmental Monitoring Systems: Issues and Solutions'. Capture screenshots/notes and one measurable result (e.g., a passing test, alert fired, or control verified).}
\setThen{the deliverables are produced (2–3 page brief on 'Privacy and Security in Environmental Monitoring Systems: Issues and Solutions': risks, architecture, controls, and a checklist; plus a one-slide executive summary.); evidence (screenshots/logs/configs) is attached and reviewed}
\setDoR{Persona clear; AC drafted; Dependencies known; Estimate set.}
\setDoD{All ACs pass; Tests green; Security checks; Docs updated; Evidence attached.}
\RenderStoryCard
\begin{TasksBox}
\begin{itemize}
\TaskItem{Draft a one-page chapter plan: scope, objectives, interfaces, success metrics.}
\TaskItem{Set up tools, datasets, and accounts; document versions and configuration.}
\TaskItem{Complete objective: Define key terms and articulate why this topic matters to security outcomes.}
\TaskItem{Complete objective: Diagram the architecture/data flows and identify threat surfaces.}
\TaskItem{Execute lab: Hands-on: Build a small lab demonstrating key concepts in 'Privacy and Security in Environmental Monitoring Systems: Issues and Solutions'. Capture screenshots/notes and one measurable result (e.g., a passing test, alert fired, or control verified).}
\end{itemize}
\end{TasksBox}
\clearpage
\setStoryID{CISH-058}
\setStoryTitle{Virtual Private Networks --- Learn \& Lab}
\setEpic{Part 7: Privacy and Access Management}
\setBusinessValue{Build a working understanding of virtual private networks and its place in a modern security program; be able to explain core concepts, map them to the CIA triad, and identify common threats and controls.}
\setPriority{Must}
\setSP{3}
\setPersona{Network Security Engineer}
\setDependencies{Lab VM or container runtime, Git repo for notes, Markdown/PDF export tool, Packet capture tool (tcpdump/Wireshark), Firewall/router lab}
\setAssumptions{Time-box chapter to one iteration; open issues captured for later}
\setUserStory{As a Network Security Engineer, I want to study and practice 'Virtual Private Networks' so that I can apply its concepts to reduce risk and improve outcomes.}
\setNonFunctional{Security, Reliability, Performance}
\setScenario{Apply key controls for Virtual Private Networks}
\setGiven{the lab environment and topic-specific tools for 'Virtual Private Networks' are available}
\setWhen{I execute the hands-on objectives and lab: Hands-on: Build a small lab demonstrating key concepts in 'Virtual Private Networks'. Capture screenshots/notes and one measurable result (e.g., a passing test, alert fired, or control verified).}
\setThen{the deliverables are produced (2–3 page brief on 'Virtual Private Networks': risks, architecture, controls, and a checklist; plus a one-slide executive summary.); evidence (screenshots/logs/configs) is attached and reviewed}
\setDoR{Persona clear; AC drafted; Dependencies known; Estimate set.}
\setDoD{All ACs pass; Tests green; Security checks; Docs updated; Evidence attached.}
\RenderStoryCard
\begin{TasksBox}
\begin{itemize}
\TaskItem{Draft a one-page chapter plan: scope, objectives, interfaces, success metrics.}
\TaskItem{Set up tools, datasets, and accounts; document versions and configuration.}
\TaskItem{Complete objective: Define key terms and articulate why this topic matters to security outcomes.}
\TaskItem{Complete objective: Diagram the architecture/data flows and identify threat surfaces.}
\TaskItem{Execute lab: Hands-on: Build a small lab demonstrating key concepts in 'Virtual Private Networks'. Capture screenshots/notes and one measurable result (e.g., a passing test, alert fired, or control verified).}
\end{itemize}
\end{TasksBox}
\clearpage
\setStoryID{CISH-059}
\setStoryTitle{Identity Theft --- Learn \& Lab}
\setEpic{Part 7: Privacy and Access Management}
\setBusinessValue{Build a working understanding of identity theft and its place in a modern security program; be able to explain core concepts, map them to the CIA triad, and identify common threats and controls.}
\setPriority{Must}
\setSP{3}
\setPersona{Privacy Engineer}
\setDependencies{Lab VM or container runtime, Git repo for notes, Markdown/PDF export tool, IDP / MFA-capable test app}
\setAssumptions{Time-box chapter to one iteration; open issues captured for later}
\setUserStory{As a Privacy Engineer, I want to study and practice 'Identity Theft' so that I can apply its concepts to reduce risk and improve outcomes.}
\setNonFunctional{Security, Reliability, Performance, Privacy}
\setScenario{Apply key controls for Identity Theft}
\setGiven{the lab environment and topic-specific tools for 'Identity Theft' are available}
\setWhen{I execute the hands-on objectives and lab: Hands-on: Build a small lab demonstrating key concepts in 'Identity Theft'. Capture screenshots/notes and one measurable result (e.g., a passing test, alert fired, or control verified).}
\setThen{the deliverables are produced (2–3 page brief on 'Identity Theft': risks, architecture, controls, and a checklist; plus a one-slide executive summary.); evidence (screenshots/logs/configs) is attached and reviewed}
\setDoR{Persona clear; AC drafted; Dependencies known; Estimate set.}
\setDoD{All ACs pass; Tests green; Security checks; Docs updated; Evidence attached.}
\RenderStoryCard
\begin{TasksBox}
\begin{itemize}
\TaskItem{Draft a one-page chapter plan: scope, objectives, interfaces, success metrics.}
\TaskItem{Set up tools, datasets, and accounts; document versions and configuration.}
\TaskItem{Complete objective: Define key terms and articulate why this topic matters to security outcomes.}
\TaskItem{Complete objective: Diagram the architecture/data flows and identify threat surfaces.}
\TaskItem{Execute lab: Hands-on: Build a small lab demonstrating key concepts in 'Identity Theft'. Capture screenshots/notes and one measurable result (e.g., a passing test, alert fired, or control verified).}
\end{itemize}
\end{TasksBox}
\clearpage
\setStoryID{CISH-060}
\setStoryTitle{VoIP Security --- Learn \& Lab}
\setEpic{Part 7: Privacy and Access Management}
\setBusinessValue{Build a working understanding of voip security and its place in a modern security program; be able to explain core concepts, map them to the CIA triad, and identify common threats and controls.}
\setPriority{Must}
\setSP{3}
\setPersona{Network Security Engineer}
\setDependencies{Lab VM or container runtime, Git repo for notes, Markdown/PDF export tool, Packet capture tool (tcpdump/Wireshark), Firewall/router lab}
\setAssumptions{Time-box chapter to one iteration; open issues captured for later}
\setUserStory{As a Network Security Engineer, I want to study and practice 'VoIP Security' so that I can apply its concepts to reduce risk and improve outcomes.}
\setNonFunctional{Security, Reliability, Performance}
\setScenario{Apply key controls for VoIP Security}
\setGiven{the lab environment and topic-specific tools for 'VoIP Security' are available}
\setWhen{I execute the hands-on objectives and lab: Hands-on: Build a small lab demonstrating key concepts in 'VoIP Security'. Capture screenshots/notes and one measurable result (e.g., a passing test, alert fired, or control verified).}
\setThen{the deliverables are produced (2–3 page brief on 'VoIP Security': risks, architecture, controls, and a checklist; plus a one-slide executive summary.); evidence (screenshots/logs/configs) is attached and reviewed}
\setDoR{Persona clear; AC drafted; Dependencies known; Estimate set.}
\setDoD{All ACs pass; Tests green; Security checks; Docs updated; Evidence attached.}
\RenderStoryCard
\begin{TasksBox}
\begin{itemize}
\TaskItem{Draft a one-page chapter plan: scope, objectives, interfaces, success metrics.}
\TaskItem{Set up tools, datasets, and accounts; document versions and configuration.}
\TaskItem{Complete objective: Define key terms and articulate why this topic matters to security outcomes.}
\TaskItem{Complete objective: Diagram the architecture/data flows and identify threat surfaces.}
\TaskItem{Execute lab: Hands-on: Build a small lab demonstrating key concepts in 'VoIP Security'. Capture screenshots/notes and one measurable result (e.g., a passing test, alert fired, or control verified).}
\end{itemize}
\end{TasksBox}
\clearpage
\setStoryID{CISH-061}
\setStoryTitle{SAN Security --- Learn \& Lab}
\setEpic{Part 8: Storage Security}
\setBusinessValue{Build a working understanding of san security and its place in a modern security program; be able to explain core concepts, map them to the CIA triad, and identify common threats and controls.}
\setPriority{Must}
\setSP{3}
\setPersona{Security Engineer}
\setDependencies{Lab VM or container runtime, Git repo for notes, Markdown/PDF export tool}
\setAssumptions{Time-box chapter to one iteration; open issues captured for later}
\setUserStory{As a Security Engineer, I want to study and practice 'SAN Security' so that I can apply its concepts to reduce risk and improve outcomes.}
\setNonFunctional{Security, Reliability, Performance}
\setScenario{Apply key controls for SAN Security}
\setGiven{the lab environment and topic-specific tools for 'SAN Security' are available}
\setWhen{I execute the hands-on objectives and lab: Hands-on: Build a small lab demonstrating key concepts in 'SAN Security'. Capture screenshots/notes and one measurable result (e.g., a passing test, alert fired, or control verified).}
\setThen{the deliverables are produced (2–3 page brief on 'SAN Security': risks, architecture, controls, and a checklist; plus a one-slide executive summary.); evidence (screenshots/logs/configs) is attached and reviewed}
\setDoR{Persona clear; AC drafted; Dependencies known; Estimate set.}
\setDoD{All ACs pass; Tests green; Security checks; Docs updated; Evidence attached.}
\RenderStoryCard
\begin{TasksBox}
\begin{itemize}
\TaskItem{Draft a one-page chapter plan: scope, objectives, interfaces, success metrics.}
\TaskItem{Set up tools, datasets, and accounts; document versions and configuration.}
\TaskItem{Complete objective: Define key terms and articulate why this topic matters to security outcomes.}
\TaskItem{Complete objective: Diagram the architecture/data flows and identify threat surfaces.}
\TaskItem{Execute lab: Hands-on: Build a small lab demonstrating key concepts in 'SAN Security'. Capture screenshots/notes and one measurable result (e.g., a passing test, alert fired, or control verified).}
\end{itemize}
\end{TasksBox}
\clearpage
\setStoryID{CISH-062}
\setStoryTitle{Storage Area Networking Devices Security --- Learn \& Lab}
\setEpic{Part 8: Storage Security}
\setBusinessValue{Build a working understanding of storage area networking devices security and its place in a modern security program; be able to explain core concepts, map them to the CIA triad, and identify common threats and controls.}
\setPriority{Must}
\setSP{3}
\setPersona{Network Security Engineer}
\setDependencies{Lab VM or container runtime, Git repo for notes, Markdown/PDF export tool, Packet capture tool (tcpdump/Wireshark), Firewall/router lab}
\setAssumptions{Time-box chapter to one iteration; open issues captured for later}
\setUserStory{As a Network Security Engineer, I want to study and practice 'Storage Area Networking Devices Security' so that I can apply its concepts to reduce risk and improve outcomes.}
\setNonFunctional{Security, Reliability, Performance}
\setScenario{Apply key controls for Storage Area Networking Devices Security}
\setGiven{the lab environment and topic-specific tools for 'Storage Area Networking Devices Security' are available}
\setWhen{I execute the hands-on objectives and lab: Hands-on: Build a small lab demonstrating key concepts in 'Storage Area Networking Devices Security'. Capture screenshots/notes and one measurable result (e.g., a passing test, alert fired, or control verified).}
\setThen{the deliverables are produced (2–3 page brief on 'Storage Area Networking Devices Security': risks, architecture, controls, and a checklist; plus a one-slide executive summary.); evidence (screenshots/logs/configs) is attached and reviewed}
\setDoR{Persona clear; AC drafted; Dependencies known; Estimate set.}
\setDoD{All ACs pass; Tests green; Security checks; Docs updated; Evidence attached.}
\RenderStoryCard
\begin{TasksBox}
\begin{itemize}
\TaskItem{Draft a one-page chapter plan: scope, objectives, interfaces, success metrics.}
\TaskItem{Set up tools, datasets, and accounts; document versions and configuration.}
\TaskItem{Complete objective: Define key terms and articulate why this topic matters to security outcomes.}
\TaskItem{Complete objective: Diagram the architecture/data flows and identify threat surfaces.}
\TaskItem{Execute lab: Hands-on: Build a small lab demonstrating key concepts in 'Storage Area Networking Devices Security'. Capture screenshots/notes and one measurable result (e.g., a passing test, alert fired, or control verified).}
\end{itemize}
\end{TasksBox}
\clearpage
\setStoryID{CISH-063}
\setStoryTitle{Securing Cloud Computing Systems --- Learn \& Lab}
\setEpic{Part 9: Cloud Security}
\setBusinessValue{Build a working understanding of securing cloud computing systems and its place in a modern security program; be able to explain core concepts, map them to the CIA triad, and identify common threats and controls.}
\setPriority{Must}
\setSP{3}
\setPersona{Cloud Security Engineer}
\setDependencies{Lab VM or container runtime, Git repo for notes, Markdown/PDF export tool, Cloud sandbox account, Terraform, Benchmark tool (e.g., CIS)}
\setAssumptions{Sandbox-only changes; no production accounts; Time-box chapter to one iteration; open issues captured for later}
\setUserStory{As a Cloud Security Engineer, I want to study and practice 'Securing Cloud Computing Systems' so that I can apply its concepts to reduce risk and improve outcomes.}
\setNonFunctional{Security, Reliability, Performance, Compliance}
\setScenario{Apply key controls for Securing Cloud Computing Systems}
\setGiven{the lab environment and topic-specific tools for 'Securing Cloud Computing Systems' are available}
\setWhen{I execute the hands-on objectives and lab: Hands-on: Build a small lab demonstrating key concepts in 'Securing Cloud Computing Systems'. Capture screenshots/notes and one measurable result (e.g., a passing test, alert fired, or control verified).}
\setThen{the deliverables are produced (2–3 page brief on 'Securing Cloud Computing Systems': risks, architecture, controls, and a checklist; plus a one-slide executive summary.); evidence (screenshots/logs/configs) is attached and reviewed}
\setDoR{Persona clear; AC drafted; Dependencies known; Estimate set.}
\setDoD{All ACs pass; Tests green; Security checks; Docs updated; Evidence attached.}
\RenderStoryCard
\begin{TasksBox}
\begin{itemize}
\TaskItem{Draft a one-page chapter plan: scope, objectives, interfaces, success metrics.}
\TaskItem{Set up tools, datasets, and accounts; document versions and configuration.}
\TaskItem{Complete objective: Define key terms and articulate why this topic matters to security outcomes.}
\TaskItem{Complete objective: Diagram the architecture/data flows and identify threat surfaces.}
\TaskItem{Execute lab: Hands-on: Build a small lab demonstrating key concepts in 'Securing Cloud Computing Systems'. Capture screenshots/notes and one measurable result (e.g., a passing test, alert fired, or control verified).}
\end{itemize}
\end{TasksBox}
\clearpage
\setStoryID{CISH-064}
\setStoryTitle{Cloud Security --- Learn \& Lab}
\setEpic{Part 9: Cloud Security}
\setBusinessValue{Build a working understanding of cloud security and its place in a modern security program; be able to explain core concepts, map them to the CIA triad, and identify common threats and controls.}
\setPriority{Must}
\setSP{3}
\setPersona{Cloud Security Engineer}
\setDependencies{Lab VM or container runtime, Git repo for notes, Markdown/PDF export tool, Cloud sandbox account, Terraform, Benchmark tool (e.g., CIS)}
\setAssumptions{Sandbox-only changes; no production accounts; Time-box chapter to one iteration; open issues captured for later}
\setUserStory{As a Cloud Security Engineer, I want to study and practice 'Cloud Security' so that I can apply its concepts to reduce risk and improve outcomes.}
\setNonFunctional{Security, Reliability, Performance, Compliance}
\setScenario{Apply key controls for Cloud Security}
\setGiven{the lab environment and topic-specific tools for 'Cloud Security' are available}
\setWhen{I execute the hands-on objectives and lab: Hands-on: Build a small lab demonstrating key concepts in 'Cloud Security'. Capture screenshots/notes and one measurable result (e.g., a passing test, alert fired, or control verified).}
\setThen{the deliverables are produced (2–3 page brief on 'Cloud Security': risks, architecture, controls, and a checklist; plus a one-slide executive summary.); evidence (screenshots/logs/configs) is attached and reviewed}
\setDoR{Persona clear; AC drafted; Dependencies known; Estimate set.}
\setDoD{All ACs pass; Tests green; Security checks; Docs updated; Evidence attached.}
\RenderStoryCard
\begin{TasksBox}
\begin{itemize}
\TaskItem{Draft a one-page chapter plan: scope, objectives, interfaces, success metrics.}
\TaskItem{Set up tools, datasets, and accounts; document versions and configuration.}
\TaskItem{Complete objective: Define key terms and articulate why this topic matters to security outcomes.}
\TaskItem{Complete objective: Diagram the architecture/data flows and identify threat surfaces.}
\TaskItem{Execute lab: Hands-on: Build a small lab demonstrating key concepts in 'Cloud Security'. Capture screenshots/notes and one measurable result (e.g., a passing test, alert fired, or control verified).}
\end{itemize}
\end{TasksBox}
\clearpage
\setStoryID{CISH-065}
\setStoryTitle{Private Cloud Security --- Learn \& Lab}
\setEpic{Part 9: Cloud Security}
\setBusinessValue{Build a working understanding of private cloud security and its place in a modern security program; be able to explain core concepts, map them to the CIA triad, and identify common threats and controls.}
\setPriority{Must}
\setSP{3}
\setPersona{Cloud Security Engineer}
\setDependencies{Lab VM or container runtime, Git repo for notes, Markdown/PDF export tool, Cloud sandbox account, Terraform, Benchmark tool (e.g., CIS)}
\setAssumptions{Sandbox-only changes; no production accounts; Time-box chapter to one iteration; open issues captured for later}
\setUserStory{As a Cloud Security Engineer, I want to study and practice 'Private Cloud Security' so that I can apply its concepts to reduce risk and improve outcomes.}
\setNonFunctional{Security, Reliability, Performance, Compliance}
\setScenario{Apply key controls for Private Cloud Security}
\setGiven{the lab environment and topic-specific tools for 'Private Cloud Security' are available}
\setWhen{I execute the hands-on objectives and lab: Hands-on: Build a small lab demonstrating key concepts in 'Private Cloud Security'. Capture screenshots/notes and one measurable result (e.g., a passing test, alert fired, or control verified).}
\setThen{the deliverables are produced (2–3 page brief on 'Private Cloud Security': risks, architecture, controls, and a checklist; plus a one-slide executive summary.); evidence (screenshots/logs/configs) is attached and reviewed}
\setDoR{Persona clear; AC drafted; Dependencies known; Estimate set.}
\setDoD{All ACs pass; Tests green; Security checks; Docs updated; Evidence attached.}
\RenderStoryCard
\begin{TasksBox}
\begin{itemize}
\TaskItem{Draft a one-page chapter plan: scope, objectives, interfaces, success metrics.}
\TaskItem{Set up tools, datasets, and accounts; document versions and configuration.}
\TaskItem{Complete objective: Define key terms and articulate why this topic matters to security outcomes.}
\TaskItem{Complete objective: Diagram the architecture/data flows and identify threat surfaces.}
\TaskItem{Execute lab: Hands-on: Build a small lab demonstrating key concepts in 'Private Cloud Security'. Capture screenshots/notes and one measurable result (e.g., a passing test, alert fired, or control verified).}
\end{itemize}
\end{TasksBox}
\clearpage
\setStoryID{CISH-066}
\setStoryTitle{Virtual Private Cloud Security --- Learn \& Lab}
\setEpic{Part 9: Cloud Security}
\setBusinessValue{Build a working understanding of virtual private cloud security and its place in a modern security program; be able to explain core concepts, map them to the CIA triad, and identify common threats and controls.}
\setPriority{Must}
\setSP{3}
\setPersona{Cloud Security Engineer}
\setDependencies{Lab VM or container runtime, Git repo for notes, Markdown/PDF export tool, Cloud sandbox account, Terraform, Benchmark tool (e.g., CIS)}
\setAssumptions{Sandbox-only changes; no production accounts; Time-box chapter to one iteration; open issues captured for later}
\setUserStory{As a Cloud Security Engineer, I want to study and practice 'Virtual Private Cloud Security' so that I can apply its concepts to reduce risk and improve outcomes.}
\setNonFunctional{Security, Reliability, Performance, Compliance}
\setScenario{Apply key controls for Virtual Private Cloud Security}
\setGiven{the lab environment and topic-specific tools for 'Virtual Private Cloud Security' are available}
\setWhen{I execute the hands-on objectives and lab: Hands-on: Build a small lab demonstrating key concepts in 'Virtual Private Cloud Security'. Capture screenshots/notes and one measurable result (e.g., a passing test, alert fired, or control verified).}
\setThen{the deliverables are produced (2–3 page brief on 'Virtual Private Cloud Security': risks, architecture, controls, and a checklist; plus a one-slide executive summary.); evidence (screenshots/logs/configs) is attached and reviewed}
\setDoR{Persona clear; AC drafted; Dependencies known; Estimate set.}
\setDoD{All ACs pass; Tests green; Security checks; Docs updated; Evidence attached.}
\RenderStoryCard
\begin{TasksBox}
\begin{itemize}
\TaskItem{Draft a one-page chapter plan: scope, objectives, interfaces, success metrics.}
\TaskItem{Set up tools, datasets, and accounts; document versions and configuration.}
\TaskItem{Complete objective: Define key terms and articulate why this topic matters to security outcomes.}
\TaskItem{Complete objective: Diagram the architecture/data flows and identify threat surfaces.}
\TaskItem{Execute lab: Hands-on: Build a small lab demonstrating key concepts in 'Virtual Private Cloud Security'. Capture screenshots/notes and one measurable result (e.g., a passing test, alert fired, or control verified).}
\end{itemize}
\end{TasksBox}
\clearpage
\setStoryID{CISH-067}
\setStoryTitle{Protecting Virtual Infrastructure --- Learn \& Lab}
\setEpic{Part 10: Virtual Security}
\setBusinessValue{Build a working understanding of protecting virtual infrastructure and its place in a modern security program; be able to explain core concepts, map them to the CIA triad, and identify common threats and controls.}
\setPriority{Must}
\setSP{3}
\setPersona{Security Architect}
\setDependencies{Lab VM or container runtime, Git repo for notes, Markdown/PDF export tool}
\setAssumptions{Time-box chapter to one iteration; open issues captured for later}
\setUserStory{As a Security Architect, I want to study and practice 'Protecting Virtual Infrastructure' so that I can apply its concepts to reduce risk and improve outcomes.}
\setNonFunctional{Security, Reliability, Performance}
\setScenario{Apply key controls for Protecting Virtual Infrastructure}
\setGiven{the lab environment and topic-specific tools for 'Protecting Virtual Infrastructure' are available}
\setWhen{I execute the hands-on objectives and lab: Hands-on: Build a small lab demonstrating key concepts in 'Protecting Virtual Infrastructure'. Capture screenshots/notes and one measurable result (e.g., a passing test, alert fired, or control verified).}
\setThen{the deliverables are produced (2–3 page brief on 'Protecting Virtual Infrastructure': risks, architecture, controls, and a checklist; plus a one-slide executive summary.); evidence (screenshots/logs/configs) is attached and reviewed}
\setDoR{Persona clear; AC drafted; Dependencies known; Estimate set.}
\setDoD{All ACs pass; Tests green; Security checks; Docs updated; Evidence attached.}
\RenderStoryCard
\begin{TasksBox}
\begin{itemize}
\TaskItem{Draft a one-page chapter plan: scope, objectives, interfaces, success metrics.}
\TaskItem{Set up tools, datasets, and accounts; document versions and configuration.}
\TaskItem{Complete objective: Define key terms and articulate why this topic matters to security outcomes.}
\TaskItem{Complete objective: Diagram the architecture/data flows and identify threat surfaces.}
\TaskItem{Execute lab: Hands-on: Build a small lab demonstrating key concepts in 'Protecting Virtual Infrastructure'. Capture screenshots/notes and one measurable result (e.g., a passing test, alert fired, or control verified).}
\end{itemize}
\end{TasksBox}
\clearpage
\setStoryID{CISH-068}
\setStoryTitle{SDN and NFV Security --- Learn \& Lab}
\setEpic{Part 10: Virtual Security}
\setBusinessValue{Build a working understanding of sdn and nfv security and its place in a modern security program; be able to explain core concepts, map them to the CIA triad, and identify common threats and controls.}
\setPriority{Must}
\setSP{3}
\setPersona{Cloud Security Engineer}
\setDependencies{Lab VM or container runtime, Git repo for notes, Markdown/PDF export tool}
\setAssumptions{Time-box chapter to one iteration; open issues captured for later}
\setUserStory{As a Cloud Security Engineer, I want to study and practice 'SDN and NFV Security' so that I can apply its concepts to reduce risk and improve outcomes.}
\setNonFunctional{Security, Reliability, Performance}
\setScenario{Apply key controls for SDN and NFV Security}
\setGiven{the lab environment and topic-specific tools for 'SDN and NFV Security' are available}
\setWhen{I execute the hands-on objectives and lab: Hands-on: Build a small lab demonstrating key concepts in 'SDN and NFV Security'. Capture screenshots/notes and one measurable result (e.g., a passing test, alert fired, or control verified).}
\setThen{the deliverables are produced (2–3 page brief on 'SDN and NFV Security': risks, architecture, controls, and a checklist; plus a one-slide executive summary.); evidence (screenshots/logs/configs) is attached and reviewed}
\setDoR{Persona clear; AC drafted; Dependencies known; Estimate set.}
\setDoD{All ACs pass; Tests green; Security checks; Docs updated; Evidence attached.}
\RenderStoryCard
\begin{TasksBox}
\begin{itemize}
\TaskItem{Draft a one-page chapter plan: scope, objectives, interfaces, success metrics.}
\TaskItem{Set up tools, datasets, and accounts; document versions and configuration.}
\TaskItem{Complete objective: Define key terms and articulate why this topic matters to security outcomes.}
\TaskItem{Complete objective: Diagram the architecture/data flows and identify threat surfaces.}
\TaskItem{Execute lab: Hands-on: Build a small lab demonstrating key concepts in 'SDN and NFV Security'. Capture screenshots/notes and one measurable result (e.g., a passing test, alert fired, or control verified).}
\end{itemize}
\end{TasksBox}
\clearpage
\setStoryID{CISH-069}
\setStoryTitle{Physical Security Essentials --- Learn \& Lab}
\setEpic{Part 11: Cyber Physical Security}
\setBusinessValue{Build a working understanding of physical security essentials and its place in a modern security program; be able to explain core concepts, map them to the CIA triad, and identify common threats and controls.}
\setPriority{Must}
\setSP{3}
\setPersona{Physical Security Specialist}
\setDependencies{Lab VM or container runtime, Git repo for notes, Markdown/PDF export tool}
\setAssumptions{Time-box chapter to one iteration; open issues captured for later}
\setUserStory{As a Physical Security Specialist, I want to study and practice 'Physical Security Essentials' so that I can apply its concepts to reduce risk and improve outcomes.}
\setNonFunctional{Security, Reliability, Performance}
\setScenario{Apply key controls for Physical Security Essentials}
\setGiven{the lab environment and topic-specific tools for 'Physical Security Essentials' are available}
\setWhen{I execute the hands-on objectives and lab: Hands-on: Build a small lab demonstrating key concepts in 'Physical Security Essentials'. Capture screenshots/notes and one measurable result (e.g., a passing test, alert fired, or control verified).}
\setThen{the deliverables are produced (2–3 page brief on 'Physical Security Essentials': risks, architecture, controls, and a checklist; plus a one-slide executive summary.); evidence (screenshots/logs/configs) is attached and reviewed}
\setDoR{Persona clear; AC drafted; Dependencies known; Estimate set.}
\setDoD{All ACs pass; Tests green; Security checks; Docs updated; Evidence attached.}
\RenderStoryCard
\begin{TasksBox}
\begin{itemize}
\TaskItem{Draft a one-page chapter plan: scope, objectives, interfaces, success metrics.}
\TaskItem{Set up tools, datasets, and accounts; document versions and configuration.}
\TaskItem{Complete objective: Define key terms and articulate why this topic matters to security outcomes.}
\TaskItem{Complete objective: Diagram the architecture/data flows and identify threat surfaces.}
\TaskItem{Execute lab: Hands-on: Build a small lab demonstrating key concepts in 'Physical Security Essentials'. Capture screenshots/notes and one measurable result (e.g., a passing test, alert fired, or control verified).}
\end{itemize}
\end{TasksBox}
\clearpage
\setStoryID{CISH-070}
\setStoryTitle{Biometrics --- Learn \& Lab}
\setEpic{Part 11: Cyber Physical Security}
\setBusinessValue{Build a working understanding of biometrics and its place in a modern security program; be able to explain core concepts, map them to the CIA triad, and identify common threats and controls.}
\setPriority{Must}
\setSP{3}
\setPersona{Privacy Engineer}
\setDependencies{Lab VM or container runtime, Git repo for notes, Markdown/PDF export tool, IDP / MFA-capable test app}
\setAssumptions{Time-box chapter to one iteration; open issues captured for later}
\setUserStory{As a Privacy Engineer, I want to study and practice 'Biometrics' so that I can apply its concepts to reduce risk and improve outcomes.}
\setNonFunctional{Security, Reliability, Performance}
\setScenario{Apply key controls for Biometrics}
\setGiven{the lab environment and topic-specific tools for 'Biometrics' are available}
\setWhen{I execute the hands-on objectives and lab: Hands-on: Build a small lab demonstrating key concepts in 'Biometrics'. Capture screenshots/notes and one measurable result (e.g., a passing test, alert fired, or control verified).}
\setThen{the deliverables are produced (2–3 page brief on 'Biometrics': risks, architecture, controls, and a checklist; plus a one-slide executive summary.); evidence (screenshots/logs/configs) is attached and reviewed}
\setDoR{Persona clear; AC drafted; Dependencies known; Estimate set.}
\setDoD{All ACs pass; Tests green; Security checks; Docs updated; Evidence attached.}
\RenderStoryCard
\begin{TasksBox}
\begin{itemize}
\TaskItem{Draft a one-page chapter plan: scope, objectives, interfaces, success metrics.}
\TaskItem{Set up tools, datasets, and accounts; document versions and configuration.}
\TaskItem{Complete objective: Define key terms and articulate why this topic matters to security outcomes.}
\TaskItem{Complete objective: Diagram the architecture/data flows and identify threat surfaces.}
\TaskItem{Execute lab: Hands-on: Build a small lab demonstrating key concepts in 'Biometrics'. Capture screenshots/notes and one measurable result (e.g., a passing test, alert fired, or control verified).}
\end{itemize}
\end{TasksBox}
\clearpage
\setStoryID{CISH-071}
\setStoryTitle{Online Identity and User Management Services --- Learn \& Lab}
\setEpic{Part 12: Practical Security}
\setBusinessValue{Build a working understanding of online identity and user management services and its place in a modern security program; be able to explain core concepts, map them to the CIA triad, and identify common threats and controls.}
\setPriority{Must}
\setSP{3}
\setPersona{Application Security Engineer}
\setDependencies{Lab VM or container runtime, Git repo for notes, Markdown/PDF export tool, IDP / MFA-capable test app}
\setAssumptions{Time-box chapter to one iteration; open issues captured for later}
\setUserStory{As a Application Security Engineer, I want to study and practice 'Online Identity and User Management Services' so that I can apply its concepts to reduce risk and improve outcomes.}
\setNonFunctional{Security, Reliability, Performance, Privacy}
\setScenario{Apply key controls for Online Identity and User Management Services}
\setGiven{the lab environment and topic-specific tools for 'Online Identity and User Management Services' are available}
\setWhen{I execute the hands-on objectives and lab: Hands-on: Build a small lab demonstrating key concepts in 'Online Identity and User Management Services'. Capture screenshots/notes and one measurable result (e.g., a passing test, alert fired, or control verified).}
\setThen{the deliverables are produced (2–3 page brief on 'Online Identity and User Management Services': risks, architecture, controls, and a checklist; plus a one-slide executive summary.); evidence (screenshots/logs/configs) is attached and reviewed}
\setDoR{Persona clear; AC drafted; Dependencies known; Estimate set.}
\setDoD{All ACs pass; Tests green; Security checks; Docs updated; Evidence attached.}
\RenderStoryCard
\begin{TasksBox}
\begin{itemize}
\TaskItem{Draft a one-page chapter plan: scope, objectives, interfaces, success metrics.}
\TaskItem{Set up tools, datasets, and accounts; document versions and configuration.}
\TaskItem{Complete objective: Define key terms and articulate why this topic matters to security outcomes.}
\TaskItem{Complete objective: Diagram the architecture/data flows and identify threat surfaces.}
\TaskItem{Execute lab: Hands-on: Build a small lab demonstrating key concepts in 'Online Identity and User Management Services'. Capture screenshots/notes and one measurable result (e.g., a passing test, alert fired, or control verified).}
\end{itemize}
\end{TasksBox}
\clearpage
\setStoryID{CISH-072}
\setStoryTitle{Intrusion Detection and Prevention Systems --- Learn \& Lab}
\setEpic{Part 12: Practical Security}
\setBusinessValue{Build a working understanding of intrusion detection and prevention systems and its place in a modern security program; be able to explain core concepts, map them to the CIA triad, and identify common threats and controls.}
\setPriority{Must}
\setSP{3}
\setPersona{Security Engineer}
\setDependencies{Lab VM or container runtime, Git repo for notes, Markdown/PDF export tool}
\setAssumptions{Time-box chapter to one iteration; open issues captured for later}
\setUserStory{As a Security Engineer, I want to study and practice 'Intrusion Detection and Prevention Systems' so that I can apply its concepts to reduce risk and improve outcomes.}
\setNonFunctional{Security, Reliability, Performance, Observability}
\setScenario{Apply key controls for Intrusion Detection and Prevention Systems}
\setGiven{the lab environment and topic-specific tools for 'Intrusion Detection and Prevention Systems' are available}
\setWhen{I execute the hands-on objectives and lab: Hands-on: Build a small lab demonstrating key concepts in 'Intrusion Detection and Prevention Systems'. Capture screenshots/notes and one measurable result (e.g., a passing test, alert fired, or control verified).}
\setThen{the deliverables are produced (2–3 page brief on 'Intrusion Detection and Prevention Systems': risks, architecture, controls, and a checklist; plus a one-slide executive summary.); evidence (screenshots/logs/configs) is attached and reviewed}
\setDoR{Persona clear; AC drafted; Dependencies known; Estimate set.}
\setDoD{All ACs pass; Tests green; Security checks; Docs updated; Evidence attached.}
\RenderStoryCard
\begin{TasksBox}
\begin{itemize}
\TaskItem{Draft a one-page chapter plan: scope, objectives, interfaces, success metrics.}
\TaskItem{Set up tools, datasets, and accounts; document versions and configuration.}
\TaskItem{Complete objective: Define key terms and articulate why this topic matters to security outcomes.}
\TaskItem{Complete objective: Diagram the architecture/data flows and identify threat surfaces.}
\TaskItem{Execute lab: Hands-on: Build a small lab demonstrating key concepts in 'Intrusion Detection and Prevention Systems'. Capture screenshots/notes and one measurable result (e.g., a passing test, alert fired, or control verified).}
\end{itemize}
\end{TasksBox}
\clearpage
\setStoryID{CISH-073}
\setStoryTitle{Transmission Control Protocol/Internet Protocol Packet Analysis --- Learn \& Lab}
\setEpic{Part 12: Practical Security}
\setBusinessValue{Build a working understanding of transmission control protocol/internet protocol packet analysis and its place in a modern security program; be able to explain core concepts, map them to the CIA triad, and identify common threats and controls.}
\setPriority{Must}
\setSP{3}
\setPersona{Network Security Engineer}
\setDependencies{Lab VM or container runtime, Git repo for notes, Markdown/PDF export tool, Packet capture tool (tcpdump/Wireshark), Firewall/router lab}
\setAssumptions{Time-box chapter to one iteration; open issues captured for later}
\setUserStory{As a Network Security Engineer, I want to study and practice 'Transmission Control Protocol/Internet Protocol Packet Analysis' so that I can apply its concepts to reduce risk and improve outcomes.}
\setNonFunctional{Security, Reliability, Performance}
\setScenario{Apply key controls for Transmission Control Protocol/Internet Protocol Packet Analysis}
\setGiven{the lab environment and topic-specific tools for 'Transmission Control Protocol/Internet Protocol Packet Analysis' are available}
\setWhen{I execute the hands-on objectives and lab: Hands-on: Build a small lab demonstrating key concepts in 'Transmission Control Protocol/Internet Protocol Packet Analysis'. Capture screenshots/notes and one measurable result (e.g., a passing test, alert fired, or control verified).}
\setThen{the deliverables are produced (2–3 page brief on 'Transmission Control Protocol/Internet Protocol Packet Analysis': risks, architecture, controls, and a checklist; plus a one-slide executive summary.); evidence (screenshots/logs/configs) is attached and reviewed}
\setDoR{Persona clear; AC drafted; Dependencies known; Estimate set.}
\setDoD{All ACs pass; Tests green; Security checks; Docs updated; Evidence attached.}
\RenderStoryCard
\begin{TasksBox}
\begin{itemize}
\TaskItem{Draft a one-page chapter plan: scope, objectives, interfaces, success metrics.}
\TaskItem{Set up tools, datasets, and accounts; document versions and configuration.}
\TaskItem{Complete objective: Define key terms and articulate why this topic matters to security outcomes.}
\TaskItem{Complete objective: Diagram the architecture/data flows and identify threat surfaces.}
\TaskItem{Execute lab: Hands-on: Build a small lab demonstrating key concepts in 'Transmission Control Protocol/Internet Protocol Packet Analysis'. Capture screenshots/notes and one measurable result (e.g., a passing test, alert fired, or control verified).}
\end{itemize}
\end{TasksBox}
\clearpage
\setStoryID{CISH-074}
\setStoryTitle{Firewalls --- Learn \& Lab}
\setEpic{Part 12: Practical Security}
\setBusinessValue{Build a working understanding of firewalls and its place in a modern security program; be able to explain core concepts, map them to the CIA triad, and identify common threats and controls.}
\setPriority{Must}
\setSP{3}
\setPersona{Security Engineer}
\setDependencies{Lab VM or container runtime, Git repo for notes, Markdown/PDF export tool}
\setAssumptions{Time-box chapter to one iteration; open issues captured for later}
\setUserStory{As a Security Engineer, I want to study and practice 'Firewalls' so that I can apply its concepts to reduce risk and improve outcomes.}
\setNonFunctional{Security, Reliability, Performance}
\setScenario{Apply key controls for Firewalls}
\setGiven{the lab environment and topic-specific tools for 'Firewalls' are available}
\setWhen{I execute the hands-on objectives and lab: Hands-on: Build a small lab demonstrating key concepts in 'Firewalls'. Capture screenshots/notes and one measurable result (e.g., a passing test, alert fired, or control verified).}
\setThen{the deliverables are produced (2–3 page brief on 'Firewalls': risks, architecture, controls, and a checklist; plus a one-slide executive summary.); evidence (screenshots/logs/configs) is attached and reviewed}
\setDoR{Persona clear; AC drafted; Dependencies known; Estimate set.}
\setDoD{All ACs pass; Tests green; Security checks; Docs updated; Evidence attached.}
\RenderStoryCard
\begin{TasksBox}
\begin{itemize}
\TaskItem{Draft a one-page chapter plan: scope, objectives, interfaces, success metrics.}
\TaskItem{Set up tools, datasets, and accounts; document versions and configuration.}
\TaskItem{Complete objective: Define key terms and articulate why this topic matters to security outcomes.}
\TaskItem{Complete objective: Diagram the architecture/data flows and identify threat surfaces.}
\TaskItem{Execute lab: Hands-on: Build a small lab demonstrating key concepts in 'Firewalls'. Capture screenshots/notes and one measurable result (e.g., a passing test, alert fired, or control verified).}
\end{itemize}
\end{TasksBox}
\clearpage
\setStoryID{CISH-075}
\setStoryTitle{Penetration Testing --- Learn \& Lab}
\setEpic{Part 12: Practical Security}
\setBusinessValue{Build a working understanding of penetration testing and its place in a modern security program; be able to explain core concepts, map them to the CIA triad, and identify common threats and controls.}
\setPriority{Must}
\setSP{3}
\setPersona{Application Security Engineer}
\setDependencies{Lab VM or container runtime, Git repo for notes, Markdown/PDF export tool}
\setAssumptions{Time-box chapter to one iteration; open issues captured for later}
\setUserStory{As a Application Security Engineer, I want to study and practice 'Penetration Testing' so that I can apply its concepts to reduce risk and improve outcomes.}
\setNonFunctional{Security, Reliability, Performance}
\setScenario{Apply key controls for Penetration Testing}
\setGiven{the lab environment and topic-specific tools for 'Penetration Testing' are available}
\setWhen{I execute the hands-on objectives and lab: Hands-on: Build a small lab demonstrating key concepts in 'Penetration Testing'. Capture screenshots/notes and one measurable result (e.g., a passing test, alert fired, or control verified).}
\setThen{the deliverables are produced (2–3 page brief on 'Penetration Testing': risks, architecture, controls, and a checklist; plus a one-slide executive summary.); evidence (screenshots/logs/configs) is attached and reviewed}
\setDoR{Persona clear; AC drafted; Dependencies known; Estimate set.}
\setDoD{All ACs pass; Tests green; Security checks; Docs updated; Evidence attached.}
\RenderStoryCard
\begin{TasksBox}
\begin{itemize}
\TaskItem{Draft a one-page chapter plan: scope, objectives, interfaces, success metrics.}
\TaskItem{Set up tools, datasets, and accounts; document versions and configuration.}
\TaskItem{Complete objective: Define key terms and articulate why this topic matters to security outcomes.}
\TaskItem{Complete objective: Diagram the architecture/data flows and identify threat surfaces.}
\TaskItem{Execute lab: Hands-on: Build a small lab demonstrating key concepts in 'Penetration Testing'. Capture screenshots/notes and one measurable result (e.g., a passing test, alert fired, or control verified).}
\end{itemize}
\end{TasksBox}
\clearpage
\setStoryID{CISH-076}
\setStoryTitle{System Security --- Learn \& Lab}
\setEpic{Part 12: Practical Security}
\setBusinessValue{Build a working understanding of system security and its place in a modern security program; be able to explain core concepts, map them to the CIA triad, and identify common threats and controls.}
\setPriority{Must}
\setSP{3}
\setPersona{Systems Engineer}
\setDependencies{Lab VM or container runtime, Git repo for notes, Markdown/PDF export tool}
\setAssumptions{Time-box chapter to one iteration; open issues captured for later}
\setUserStory{As a Systems Engineer, I want to study and practice 'System Security' so that I can apply its concepts to reduce risk and improve outcomes.}
\setNonFunctional{Security, Reliability, Performance}
\setScenario{Apply key controls for System Security}
\setGiven{the lab environment and topic-specific tools for 'System Security' are available}
\setWhen{I execute the hands-on objectives and lab: Hands-on: Build a small lab demonstrating key concepts in 'System Security'. Capture screenshots/notes and one measurable result (e.g., a passing test, alert fired, or control verified).}
\setThen{the deliverables are produced (2–3 page brief on 'System Security': risks, architecture, controls, and a checklist; plus a one-slide executive summary.); evidence (screenshots/logs/configs) is attached and reviewed}
\setDoR{Persona clear; AC drafted; Dependencies known; Estimate set.}
\setDoD{All ACs pass; Tests green; Security checks; Docs updated; Evidence attached.}
\RenderStoryCard
\begin{TasksBox}
\begin{itemize}
\TaskItem{Draft a one-page chapter plan: scope, objectives, interfaces, success metrics.}
\TaskItem{Set up tools, datasets, and accounts; document versions and configuration.}
\TaskItem{Complete objective: Define key terms and articulate why this topic matters to security outcomes.}
\TaskItem{Complete objective: Diagram the architecture/data flows and identify threat surfaces.}
\TaskItem{Execute lab: Hands-on: Build a small lab demonstrating key concepts in 'System Security'. Capture screenshots/notes and one measurable result (e.g., a passing test, alert fired, or control verified).}
\end{itemize}
\end{TasksBox}
\clearpage
\setStoryID{CISH-077}
\setStoryTitle{Access Controls --- Learn \& Lab}
\setEpic{Part 12: Practical Security}
\setBusinessValue{Build a working understanding of access controls and its place in a modern security program; be able to explain core concepts, map them to the CIA triad, and identify common threats and controls.}
\setPriority{Must}
\setSP{3}
\setPersona{Privacy Engineer}
\setDependencies{Lab VM or container runtime, Git repo for notes, Markdown/PDF export tool, IDP / MFA-capable test app}
\setAssumptions{Time-box chapter to one iteration; open issues captured for later}
\setUserStory{As a Privacy Engineer, I want to study and practice 'Access Controls' so that I can apply its concepts to reduce risk and improve outcomes.}
\setNonFunctional{Security, Reliability, Performance}
\setScenario{Apply key controls for Access Controls}
\setGiven{the lab environment and topic-specific tools for 'Access Controls' are available}
\setWhen{I execute the hands-on objectives and lab: Hands-on: Build a small lab demonstrating key concepts in 'Access Controls'. Capture screenshots/notes and one measurable result (e.g., a passing test, alert fired, or control verified).}
\setThen{the deliverables are produced (2–3 page brief on 'Access Controls': risks, architecture, controls, and a checklist; plus a one-slide executive summary.); evidence (screenshots/logs/configs) is attached and reviewed}
\setDoR{Persona clear; AC drafted; Dependencies known; Estimate set.}
\setDoD{All ACs pass; Tests green; Security checks; Docs updated; Evidence attached.}
\RenderStoryCard
\begin{TasksBox}
\begin{itemize}
\TaskItem{Draft a one-page chapter plan: scope, objectives, interfaces, success metrics.}
\TaskItem{Set up tools, datasets, and accounts; document versions and configuration.}
\TaskItem{Complete objective: Define key terms and articulate why this topic matters to security outcomes.}
\TaskItem{Complete objective: Diagram the architecture/data flows and identify threat surfaces.}
\TaskItem{Execute lab: Hands-on: Build a small lab demonstrating key concepts in 'Access Controls'. Capture screenshots/notes and one measurable result (e.g., a passing test, alert fired, or control verified).}
\end{itemize}
\end{TasksBox}
\clearpage
\setStoryID{CISH-078}
\setStoryTitle{Endpoint Security --- Learn \& Lab}
\setEpic{Part 12: Practical Security}
\setBusinessValue{Build a working understanding of endpoint security and its place in a modern security program; be able to explain core concepts, map them to the CIA triad, and identify common threats and controls.}
\setPriority{Must}
\setSP{3}
\setPersona{Systems Engineer}
\setDependencies{Lab VM or container runtime, Git repo for notes, Markdown/PDF export tool}
\setAssumptions{Time-box chapter to one iteration; open issues captured for later}
\setUserStory{As a Systems Engineer, I want to study and practice 'Endpoint Security' so that I can apply its concepts to reduce risk and improve outcomes.}
\setNonFunctional{Security, Reliability, Performance}
\setScenario{Apply key controls for Endpoint Security}
\setGiven{the lab environment and topic-specific tools for 'Endpoint Security' are available}
\setWhen{I execute the hands-on objectives and lab: Hands-on: Build a small lab demonstrating key concepts in 'Endpoint Security'. Capture screenshots/notes and one measurable result (e.g., a passing test, alert fired, or control verified).}
\setThen{the deliverables are produced (2–3 page brief on 'Endpoint Security': risks, architecture, controls, and a checklist; plus a one-slide executive summary.); evidence (screenshots/logs/configs) is attached and reviewed}
\setDoR{Persona clear; AC drafted; Dependencies known; Estimate set.}
\setDoD{All ACs pass; Tests green; Security checks; Docs updated; Evidence attached.}
\RenderStoryCard
\begin{TasksBox}
\begin{itemize}
\TaskItem{Draft a one-page chapter plan: scope, objectives, interfaces, success metrics.}
\TaskItem{Set up tools, datasets, and accounts; document versions and configuration.}
\TaskItem{Complete objective: Define key terms and articulate why this topic matters to security outcomes.}
\TaskItem{Complete objective: Diagram the architecture/data flows and identify threat surfaces.}
\TaskItem{Execute lab: Hands-on: Build a small lab demonstrating key concepts in 'Endpoint Security'. Capture screenshots/notes and one measurable result (e.g., a passing test, alert fired, or control verified).}
\end{itemize}
\end{TasksBox}
\clearpage
\setStoryID{CISH-079}
\setStoryTitle{Assessments and Audits --- Learn \& Lab}
\setEpic{Part 12: Practical Security}
\setBusinessValue{Build a working understanding of assessments and audits and its place in a modern security program; be able to explain core concepts, map them to the CIA triad, and identify common threats and controls.}
\setPriority{Must}
\setSP{3}
\setPersona{Application Security Engineer}
\setDependencies{Lab VM or container runtime, Git repo for notes, Markdown/PDF export tool}
\setAssumptions{Time-box chapter to one iteration; open issues captured for later}
\setUserStory{As a Application Security Engineer, I want to study and practice 'Assessments and Audits' so that I can apply its concepts to reduce risk and improve outcomes.}
\setNonFunctional{Security, Reliability, Performance, Observability}
\setScenario{Apply key controls for Assessments and Audits}
\setGiven{the lab environment and topic-specific tools for 'Assessments and Audits' are available}
\setWhen{I execute the hands-on objectives and lab: Hands-on: Build a small lab demonstrating key concepts in 'Assessments and Audits'. Capture screenshots/notes and one measurable result (e.g., a passing test, alert fired, or control verified).}
\setThen{the deliverables are produced (2–3 page brief on 'Assessments and Audits': risks, architecture, controls, and a checklist; plus a one-slide executive summary.); evidence (screenshots/logs/configs) is attached and reviewed}
\setDoR{Persona clear; AC drafted; Dependencies known; Estimate set.}
\setDoD{All ACs pass; Tests green; Security checks; Docs updated; Evidence attached.}
\RenderStoryCard
\begin{TasksBox}
\begin{itemize}
\TaskItem{Draft a one-page chapter plan: scope, objectives, interfaces, success metrics.}
\TaskItem{Set up tools, datasets, and accounts; document versions and configuration.}
\TaskItem{Complete objective: Define key terms and articulate why this topic matters to security outcomes.}
\TaskItem{Complete objective: Diagram the architecture/data flows and identify threat surfaces.}
\TaskItem{Execute lab: Hands-on: Build a small lab demonstrating key concepts in 'Assessments and Audits'. Capture screenshots/notes and one measurable result (e.g., a passing test, alert fired, or control verified).}
\end{itemize}
\end{TasksBox}
\clearpage
\setStoryID{CISH-080}
\setStoryTitle{Fundamentals of Cryptography --- Learn \& Lab}
\setEpic{Part 12: Practical Security}
\setBusinessValue{Build a working understanding of fundamentals of cryptography and its place in a modern security program; be able to explain core concepts, map them to the CIA triad, and identify common threats and controls.}
\setPriority{Must}
\setSP{3}
\setPersona{Platform Engineer}
\setDependencies{Lab VM or container runtime, Git repo for notes, Markdown/PDF export tool, OpenSSL/mkcert, TLS scanner}
\setAssumptions{Time-box chapter to one iteration; open issues captured for later}
\setUserStory{As a Platform Engineer, I want to study and practice 'Fundamentals of Cryptography' so that I can apply its concepts to reduce risk and improve outcomes.}
\setNonFunctional{Security, Reliability, Performance}
\setScenario{Apply key controls for Fundamentals of Cryptography}
\setGiven{the lab environment and topic-specific tools for 'Fundamentals of Cryptography' are available}
\setWhen{I execute the hands-on objectives and lab: Hands-on: Build a small lab demonstrating key concepts in 'Fundamentals of Cryptography'. Capture screenshots/notes and one measurable result (e.g., a passing test, alert fired, or control verified).}
\setThen{the deliverables are produced (2–3 page brief on 'Fundamentals of Cryptography': risks, architecture, controls, and a checklist; plus a one-slide executive summary.); evidence (screenshots/logs/configs) is attached and reviewed}
\setDoR{Persona clear; AC drafted; Dependencies known; Estimate set.}
\setDoD{All ACs pass; Tests green; Security checks; Docs updated; Evidence attached.}
\RenderStoryCard
\begin{TasksBox}
\begin{itemize}
\TaskItem{Draft a one-page chapter plan: scope, objectives, interfaces, success metrics.}
\TaskItem{Set up tools, datasets, and accounts; document versions and configuration.}
\TaskItem{Complete objective: Define key terms and articulate why this topic matters to security outcomes.}
\TaskItem{Complete objective: Diagram the architecture/data flows and identify threat surfaces.}
\TaskItem{Execute lab: Hands-on: Build a small lab demonstrating key concepts in 'Fundamentals of Cryptography'. Capture screenshots/notes and one measurable result (e.g., a passing test, alert fired, or control verified).}
\end{itemize}
\end{TasksBox}
\clearpage
\setStoryID{CISH-081}
\setStoryTitle{Securing the Infrastructure --- Learn \& Lab}
\setEpic{Part 13: Critical Infrastructure Security}
\setBusinessValue{Build a working understanding of securing the infrastructure and its place in a modern security program; be able to explain core concepts, map them to the CIA triad, and identify common threats and controls.}
\setPriority{Must}
\setSP{3}
\setPersona{Security Architect}
\setDependencies{Lab VM or container runtime, Git repo for notes, Markdown/PDF export tool}
\setAssumptions{Time-box chapter to one iteration; open issues captured for later}
\setUserStory{As a Security Architect, I want to study and practice 'Securing the Infrastructure' so that I can apply its concepts to reduce risk and improve outcomes.}
\setNonFunctional{Security, Reliability, Performance}
\setScenario{Apply key controls for Securing the Infrastructure}
\setGiven{the lab environment and topic-specific tools for 'Securing the Infrastructure' are available}
\setWhen{I execute the hands-on objectives and lab: Hands-on: Build a small lab demonstrating key concepts in 'Securing the Infrastructure'. Capture screenshots/notes and one measurable result (e.g., a passing test, alert fired, or control verified).}
\setThen{the deliverables are produced (2–3 page brief on 'Securing the Infrastructure': risks, architecture, controls, and a checklist; plus a one-slide executive summary.); evidence (screenshots/logs/configs) is attached and reviewed}
\setDoR{Persona clear; AC drafted; Dependencies known; Estimate set.}
\setDoD{All ACs pass; Tests green; Security checks; Docs updated; Evidence attached.}
\RenderStoryCard
\begin{TasksBox}
\begin{itemize}
\TaskItem{Draft a one-page chapter plan: scope, objectives, interfaces, success metrics.}
\TaskItem{Set up tools, datasets, and accounts; document versions and configuration.}
\TaskItem{Complete objective: Define key terms and articulate why this topic matters to security outcomes.}
\TaskItem{Complete objective: Diagram the architecture/data flows and identify threat surfaces.}
\TaskItem{Execute lab: Hands-on: Build a small lab demonstrating key concepts in 'Securing the Infrastructure'. Capture screenshots/notes and one measurable result (e.g., a passing test, alert fired, or control verified).}
\end{itemize}
\end{TasksBox}
\clearpage
\setStoryID{CISH-082}
\setStoryTitle{Threat Landscape and Good Practices for the Internet Infrastructure --- Learn \& Lab}
\setEpic{Part 13: Critical Infrastructure Security}
\setBusinessValue{Build a working understanding of threat landscape and good practices for the internet infrastructure and its place in a modern security program; be able to explain core concepts, map them to the CIA triad, and identify common threats and controls.}
\setPriority{Must}
\setSP{3}
\setPersona{Network Security Engineer}
\setDependencies{Lab VM or container runtime, Git repo for notes, Markdown/PDF export tool, Packet capture tool (tcpdump/Wireshark), Firewall/router lab}
\setAssumptions{Time-box chapter to one iteration; open issues captured for later}
\setUserStory{As a Network Security Engineer, I want to study and practice 'Threat Landscape and Good Practices for the Internet Infrastructure' so that I can apply its concepts to reduce risk and improve outcomes.}
\setNonFunctional{Security, Reliability, Performance}
\setScenario{Apply key controls for Threat Landscape and Good Practices for the Internet Infrastructure}
\setGiven{the lab environment and topic-specific tools for 'Threat Landscape and Good Practices for the Internet Infrastructure' are available}
\setWhen{I execute the hands-on objectives and lab: Hands-on: Build a small lab demonstrating key concepts in 'Threat Landscape and Good Practices for the Internet Infrastructure'. Capture screenshots/notes and one measurable result (e.g., a passing test, alert fired, or control verified).}
\setThen{the deliverables are produced (2–3 page brief on 'Threat Landscape and Good Practices for the Internet Infrastructure': risks, architecture, controls, and a checklist; plus a one-slide executive summary.); evidence (screenshots/logs/configs) is attached and reviewed}
\setDoR{Persona clear; AC drafted; Dependencies known; Estimate set.}
\setDoD{All ACs pass; Tests green; Security checks; Docs updated; Evidence attached.}
\RenderStoryCard
\begin{TasksBox}
\begin{itemize}
\TaskItem{Draft a one-page chapter plan: scope, objectives, interfaces, success metrics.}
\TaskItem{Set up tools, datasets, and accounts; document versions and configuration.}
\TaskItem{Complete objective: Define key terms and articulate why this topic matters to security outcomes.}
\TaskItem{Complete objective: Diagram the architecture/data flows and identify threat surfaces.}
\TaskItem{Execute lab: Hands-on: Build a small lab demonstrating key concepts in 'Threat Landscape and Good Practices for the Internet Infrastructure'. Capture screenshots/notes and one measurable result (e.g., a passing test, alert fired, or control verified).}
\end{itemize}
\end{TasksBox}
\clearpage
\setStoryID{CISH-083}
\setStoryTitle{Cyber Attacks Against the Grid Infrastructure --- Learn \& Lab}
\setEpic{Part 13: Critical Infrastructure Security}
\setBusinessValue{Build a working understanding of cyber attacks against the grid infrastructure and its place in a modern security program; be able to explain core concepts, map them to the CIA triad, and identify common threats and controls.}
\setPriority{Must}
\setSP{3}
\setPersona{Security Architect}
\setDependencies{Lab VM or container runtime, Git repo for notes, Markdown/PDF export tool}
\setAssumptions{Time-box chapter to one iteration; open issues captured for later}
\setUserStory{As a Security Architect, I want to study and practice 'Cyber Attacks Against the Grid Infrastructure' so that I can apply its concepts to reduce risk and improve outcomes.}
\setNonFunctional{Security, Reliability, Performance}
\setScenario{Apply key controls for Cyber Attacks Against the Grid Infrastructure}
\setGiven{the lab environment and topic-specific tools for 'Cyber Attacks Against the Grid Infrastructure' are available}
\setWhen{I execute the hands-on objectives and lab: Hands-on: Build a small lab demonstrating key concepts in 'Cyber Attacks Against the Grid Infrastructure'. Capture screenshots/notes and one measurable result (e.g., a passing test, alert fired, or control verified).}
\setThen{the deliverables are produced (2–3 page brief on 'Cyber Attacks Against the Grid Infrastructure': risks, architecture, controls, and a checklist; plus a one-slide executive summary.); evidence (screenshots/logs/configs) is attached and reviewed}
\setDoR{Persona clear; AC drafted; Dependencies known; Estimate set.}
\setDoD{All ACs pass; Tests green; Security checks; Docs updated; Evidence attached.}
\RenderStoryCard
\begin{TasksBox}
\begin{itemize}
\TaskItem{Draft a one-page chapter plan: scope, objectives, interfaces, success metrics.}
\TaskItem{Set up tools, datasets, and accounts; document versions and configuration.}
\TaskItem{Complete objective: Define key terms and articulate why this topic matters to security outcomes.}
\TaskItem{Complete objective: Diagram the architecture/data flows and identify threat surfaces.}
\TaskItem{Execute lab: Hands-on: Build a small lab demonstrating key concepts in 'Cyber Attacks Against the Grid Infrastructure'. Capture screenshots/notes and one measurable result (e.g., a passing test, alert fired, or control verified).}
\end{itemize}
\end{TasksBox}
\clearpage
\setStoryID{CISH-084}
\setStoryTitle{Threat Landscape and Good Practices For The Smart Grid Infrastructure --- Learn \& Lab}
\setEpic{Part 13: Critical Infrastructure Security}
\setBusinessValue{Build a working understanding of threat landscape and good practices for the smart grid infrastructure and its place in a modern security program; be able to explain core concepts, map them to the CIA triad, and identify common threats and controls.}
\setPriority{Must}
\setSP{3}
\setPersona{Network Security Engineer}
\setDependencies{Lab VM or container runtime, Git repo for notes, Markdown/PDF export tool, Packet capture tool (tcpdump/Wireshark), Firewall/router lab}
\setAssumptions{Time-box chapter to one iteration; open issues captured for later}
\setUserStory{As a Network Security Engineer, I want to study and practice 'Threat Landscape and Good Practices For The Smart Grid Infrastructure' so that I can apply its concepts to reduce risk and improve outcomes.}
\setNonFunctional{Security, Reliability, Performance}
\setScenario{Apply key controls for Threat Landscape and Good Practices For The Smart Grid Infrastructure}
\setGiven{the lab environment and topic-specific tools for 'Threat Landscape and Good Practices For The Smart Grid Infrastructure' are available}
\setWhen{I execute the hands-on objectives and lab: Hands-on: Build a small lab demonstrating key concepts in 'Threat Landscape and Good Practices For The Smart Grid Infrastructure'. Capture screenshots/notes and one measurable result (e.g., a passing test, alert fired, or control verified).}
\setThen{the deliverables are produced (2–3 page brief on 'Threat Landscape and Good Practices For The Smart Grid Infrastructure': risks, architecture, controls, and a checklist; plus a one-slide executive summary.); evidence (screenshots/logs/configs) is attached and reviewed}
\setDoR{Persona clear; AC drafted; Dependencies known; Estimate set.}
\setDoD{All ACs pass; Tests green; Security checks; Docs updated; Evidence attached.}
\RenderStoryCard
\begin{TasksBox}
\begin{itemize}
\TaskItem{Draft a one-page chapter plan: scope, objectives, interfaces, success metrics.}
\TaskItem{Set up tools, datasets, and accounts; document versions and configuration.}
\TaskItem{Complete objective: Define key terms and articulate why this topic matters to security outcomes.}
\TaskItem{Complete objective: Diagram the architecture/data flows and identify threat surfaces.}
\TaskItem{Execute lab: Hands-on: Build a small lab demonstrating key concepts in 'Threat Landscape and Good Practices For The Smart Grid Infrastructure'. Capture screenshots/notes and one measurable result (e.g., a passing test, alert fired, or control verified).}
\end{itemize}
\end{TasksBox}
\clearpage
\setStoryID{CISH-085}
\setStoryTitle{Energy Infrastructure Cyber Security --- Learn \& Lab}
\setEpic{Part 13: Critical Infrastructure Security}
\setBusinessValue{Build a working understanding of energy infrastructure cyber security and its place in a modern security program; be able to explain core concepts, map them to the CIA triad, and identify common threats and controls.}
\setPriority{Must}
\setSP{3}
\setPersona{Security Architect}
\setDependencies{Lab VM or container runtime, Git repo for notes, Markdown/PDF export tool}
\setAssumptions{Time-box chapter to one iteration; open issues captured for later}
\setUserStory{As a Security Architect, I want to study and practice 'Energy Infrastructure Cyber Security' so that I can apply its concepts to reduce risk and improve outcomes.}
\setNonFunctional{Security, Reliability, Performance}
\setScenario{Apply key controls for Energy Infrastructure Cyber Security}
\setGiven{the lab environment and topic-specific tools for 'Energy Infrastructure Cyber Security' are available}
\setWhen{I execute the hands-on objectives and lab: Hands-on: Build a small lab demonstrating key concepts in 'Energy Infrastructure Cyber Security'. Capture screenshots/notes and one measurable result (e.g., a passing test, alert fired, or control verified).}
\setThen{the deliverables are produced (2–3 page brief on 'Energy Infrastructure Cyber Security': risks, architecture, controls, and a checklist; plus a one-slide executive summary.); evidence (screenshots/logs/configs) is attached and reviewed}
\setDoR{Persona clear; AC drafted; Dependencies known; Estimate set.}
\setDoD{All ACs pass; Tests green; Security checks; Docs updated; Evidence attached.}
\RenderStoryCard
\begin{TasksBox}
\begin{itemize}
\TaskItem{Draft a one-page chapter plan: scope, objectives, interfaces, success metrics.}
\TaskItem{Set up tools, datasets, and accounts; document versions and configuration.}
\TaskItem{Complete objective: Define key terms and articulate why this topic matters to security outcomes.}
\TaskItem{Complete objective: Diagram the architecture/data flows and identify threat surfaces.}
\TaskItem{Execute lab: Hands-on: Build a small lab demonstrating key concepts in 'Energy Infrastructure Cyber Security'. Capture screenshots/notes and one measurable result (e.g., a passing test, alert fired, or control verified).}
\end{itemize}
\end{TasksBox}
\clearpage
\setStoryID{CISH-086}
\setStoryTitle{Homeland Security --- Learn \& Lab}
\setEpic{Part 13: Critical Infrastructure Security}
\setBusinessValue{Build a working understanding of homeland security and its place in a modern security program; be able to explain core concepts, map them to the CIA triad, and identify common threats and controls.}
\setPriority{Must}
\setSP{3}
\setPersona{Network Security Engineer}
\setDependencies{Lab VM or container runtime, Git repo for notes, Markdown/PDF export tool, Packet capture tool (tcpdump/Wireshark), Firewall/router lab}
\setAssumptions{Time-box chapter to one iteration; open issues captured for later}
\setUserStory{As a Network Security Engineer, I want to study and practice 'Homeland Security' so that I can apply its concepts to reduce risk and improve outcomes.}
\setNonFunctional{Security, Reliability, Performance, Compliance}
\setScenario{Apply key controls for Homeland Security}
\setGiven{the lab environment and topic-specific tools for 'Homeland Security' are available}
\setWhen{I execute the hands-on objectives and lab: Hands-on: Build a small lab demonstrating key concepts in 'Homeland Security'. Capture screenshots/notes and one measurable result (e.g., a passing test, alert fired, or control verified).}
\setThen{the deliverables are produced (2–3 page brief on 'Homeland Security': risks, architecture, controls, and a checklist; plus a one-slide executive summary.); evidence (screenshots/logs/configs) is attached and reviewed}
\setDoR{Persona clear; AC drafted; Dependencies known; Estimate set.}
\setDoD{All ACs pass; Tests green; Security checks; Docs updated; Evidence attached.}
\RenderStoryCard
\begin{TasksBox}
\begin{itemize}
\TaskItem{Draft a one-page chapter plan: scope, objectives, interfaces, success metrics.}
\TaskItem{Set up tools, datasets, and accounts; document versions and configuration.}
\TaskItem{Complete objective: Define key terms and articulate why this topic matters to security outcomes.}
\TaskItem{Complete objective: Diagram the architecture/data flows and identify threat surfaces.}
\TaskItem{Execute lab: Hands-on: Build a small lab demonstrating key concepts in 'Homeland Security'. Capture screenshots/notes and one measurable result (e.g., a passing test, alert fired, or control verified).}
\end{itemize}
\end{TasksBox}
\clearpage
\setStoryID{CISH-087}
\setStoryTitle{Cyber Warfare --- Learn \& Lab}
\setEpic{Part 13: Critical Infrastructure Security}
\setBusinessValue{Build a working understanding of cyber warfare and its place in a modern security program; be able to explain core concepts, map them to the CIA triad, and identify common threats and controls.}
\setPriority{Must}
\setSP{3}
\setPersona{Security Engineer}
\setDependencies{Lab VM or container runtime, Git repo for notes, Markdown/PDF export tool}
\setAssumptions{Time-box chapter to one iteration; open issues captured for later}
\setUserStory{As a Security Engineer, I want to study and practice 'Cyber Warfare' so that I can apply its concepts to reduce risk and improve outcomes.}
\setNonFunctional{Security, Reliability, Performance}
\setScenario{Apply key controls for Cyber Warfare}
\setGiven{the lab environment and topic-specific tools for 'Cyber Warfare' are available}
\setWhen{I execute the hands-on objectives and lab: Hands-on: Build a small lab demonstrating key concepts in 'Cyber Warfare'. Capture screenshots/notes and one measurable result (e.g., a passing test, alert fired, or control verified).}
\setThen{the deliverables are produced (2–3 page brief on 'Cyber Warfare': risks, architecture, controls, and a checklist; plus a one-slide executive summary.); evidence (screenshots/logs/configs) is attached and reviewed}
\setDoR{Persona clear; AC drafted; Dependencies known; Estimate set.}
\setDoD{All ACs pass; Tests green; Security checks; Docs updated; Evidence attached.}
\RenderStoryCard
\begin{TasksBox}
\begin{itemize}
\TaskItem{Draft a one-page chapter plan: scope, objectives, interfaces, success metrics.}
\TaskItem{Set up tools, datasets, and accounts; document versions and configuration.}
\TaskItem{Complete objective: Define key terms and articulate why this topic matters to security outcomes.}
\TaskItem{Complete objective: Diagram the architecture/data flows and identify threat surfaces.}
\TaskItem{Execute lab: Hands-on: Build a small lab demonstrating key concepts in 'Cyber Warfare'. Capture screenshots/notes and one measurable result (e.g., a passing test, alert fired, or control verified).}
\end{itemize}
\end{TasksBox}
\clearpage
\setStoryID{CISH-088}
\setStoryTitle{Cyber Attack Process --- Learn \& Lab}
\setEpic{Part 13: Critical Infrastructure Security}
\setBusinessValue{Build a working understanding of cyber attack process and its place in a modern security program; be able to explain core concepts, map them to the CIA triad, and identify common threats and controls.}
\setPriority{Must}
\setSP{3}
\setPersona{Security Engineer}
\setDependencies{Lab VM or container runtime, Git repo for notes, Markdown/PDF export tool}
\setAssumptions{Time-box chapter to one iteration; open issues captured for later}
\setUserStory{As a Security Engineer, I want to study and practice 'Cyber Attack Process' so that I can apply its concepts to reduce risk and improve outcomes.}
\setNonFunctional{Security, Reliability, Performance}
\setScenario{Apply key controls for Cyber Attack Process}
\setGiven{the lab environment and topic-specific tools for 'Cyber Attack Process' are available}
\setWhen{I execute the hands-on objectives and lab: Hands-on: Build a small lab demonstrating key concepts in 'Cyber Attack Process'. Capture screenshots/notes and one measurable result (e.g., a passing test, alert fired, or control verified).}
\setThen{the deliverables are produced (2–3 page brief on 'Cyber Attack Process': risks, architecture, controls, and a checklist; plus a one-slide executive summary.); evidence (screenshots/logs/configs) is attached and reviewed}
\setDoR{Persona clear; AC drafted; Dependencies known; Estimate set.}
\setDoD{All ACs pass; Tests green; Security checks; Docs updated; Evidence attached.}
\RenderStoryCard
\begin{TasksBox}
\begin{itemize}
\TaskItem{Draft a one-page chapter plan: scope, objectives, interfaces, success metrics.}
\TaskItem{Set up tools, datasets, and accounts; document versions and configuration.}
\TaskItem{Complete objective: Define key terms and articulate why this topic matters to security outcomes.}
\TaskItem{Complete objective: Diagram the architecture/data flows and identify threat surfaces.}
\TaskItem{Execute lab: Hands-on: Build a small lab demonstrating key concepts in 'Cyber Attack Process'. Capture screenshots/notes and one measurable result (e.g., a passing test, alert fired, or control verified).}
\end{itemize}
\end{TasksBox}
\clearpage
\setStoryID{CISH-089}
\setStoryTitle{Smart Cities: Cyber Security Concerns --- Learn \& Lab}
\setEpic{Part 14: Cyber Security for the Smart City And Smart Homes}
\setBusinessValue{Build a working understanding of smart cities: cyber security concerns and its place in a modern security program; be able to explain core concepts, map them to the CIA triad, and identify common threats and controls.}
\setPriority{Must}
\setSP{3}
\setPersona{Security Architect}
\setDependencies{Lab VM or container runtime, Git repo for notes, Markdown/PDF export tool}
\setAssumptions{Time-box chapter to one iteration; open issues captured for later}
\setUserStory{As a Security Architect, I want to study and practice 'Smart Cities: Cyber Security Concerns' so that I can apply its concepts to reduce risk and improve outcomes.}
\setNonFunctional{Security, Reliability, Performance}
\setScenario{Apply key controls for Smart Cities: Cyber Security Concerns}
\setGiven{the lab environment and topic-specific tools for 'Smart Cities: Cyber Security Concerns' are available}
\setWhen{I execute the hands-on objectives and lab: Hands-on: Build a small lab demonstrating key concepts in 'Smart Cities: Cyber Security Concerns'. Capture screenshots/notes and one measurable result (e.g., a passing test, alert fired, or control verified).}
\setThen{the deliverables are produced (2–3 page brief on 'Smart Cities: Cyber Security Concerns': risks, architecture, controls, and a checklist; plus a one-slide executive summary.); evidence (screenshots/logs/configs) is attached and reviewed}
\setDoR{Persona clear; AC drafted; Dependencies known; Estimate set.}
\setDoD{All ACs pass; Tests green; Security checks; Docs updated; Evidence attached.}
\RenderStoryCard
\begin{TasksBox}
\begin{itemize}
\TaskItem{Draft a one-page chapter plan: scope, objectives, interfaces, success metrics.}
\TaskItem{Set up tools, datasets, and accounts; document versions and configuration.}
\TaskItem{Complete objective: Define key terms and articulate why this topic matters to security outcomes.}
\TaskItem{Complete objective: Diagram the architecture/data flows and identify threat surfaces.}
\TaskItem{Execute lab: Hands-on: Build a small lab demonstrating key concepts in 'Smart Cities: Cyber Security Concerns'. Capture screenshots/notes and one measurable result (e.g., a passing test, alert fired, or control verified).}
\end{itemize}
\end{TasksBox}
\clearpage
\setStoryID{CISH-090}
\setStoryTitle{Community Preparedness Action Groups for Smart City Cyber Security --- Learn \& Lab}
\setEpic{Part 14: Cyber Security for the Smart City And Smart Homes}
\setBusinessValue{Build a working understanding of community preparedness action groups for smart city cyber security and its place in a modern security program; be able to explain core concepts, map them to the CIA triad, and identify common threats and controls.}
\setPriority{Must}
\setSP{3}
\setPersona{Security Architect}
\setDependencies{Lab VM or container runtime, Git repo for notes, Markdown/PDF export tool}
\setAssumptions{Time-box chapter to one iteration; open issues captured for later}
\setUserStory{As a Security Architect, I want to study and practice 'Community Preparedness Action Groups for Smart City Cyber Security' so that I can apply its concepts to reduce risk and improve outcomes.}
\setNonFunctional{Security, Reliability, Performance}
\setScenario{Apply key controls for Community Preparedness Action Groups for Smart City Cyber Security}
\setGiven{the lab environment and topic-specific tools for 'Community Preparedness Action Groups for Smart City Cyber Security' are available}
\setWhen{I execute the hands-on objectives and lab: Hands-on: Build a small lab demonstrating key concepts in 'Community Preparedness Action Groups for Smart City Cyber Security'. Capture screenshots/notes and one measurable result (e.g., a passing test, alert fired, or control verified).}
\setThen{the deliverables are produced (2–3 page brief on 'Community Preparedness Action Groups for Smart City Cyber Security': risks, architecture, controls, and a checklist; plus a one-slide executive summary.); evidence (screenshots/logs/configs) is attached and reviewed}
\setDoR{Persona clear; AC drafted; Dependencies known; Estimate set.}
\setDoD{All ACs pass; Tests green; Security checks; Docs updated; Evidence attached.}
\RenderStoryCard
\begin{TasksBox}
\begin{itemize}
\TaskItem{Draft a one-page chapter plan: scope, objectives, interfaces, success metrics.}
\TaskItem{Set up tools, datasets, and accounts; document versions and configuration.}
\TaskItem{Complete objective: Define key terms and articulate why this topic matters to security outcomes.}
\TaskItem{Complete objective: Diagram the architecture/data flows and identify threat surfaces.}
\TaskItem{Execute lab: Hands-on: Build a small lab demonstrating key concepts in 'Community Preparedness Action Groups for Smart City Cyber Security'. Capture screenshots/notes and one measurable result (e.g., a passing test, alert fired, or control verified).}
\end{itemize}
\end{TasksBox}
\clearpage
\setStoryID{CISH-091}
\setStoryTitle{Smart City Disaster Preparedness and Resilience --- Learn \& Lab}
\setEpic{Part 14: Cyber Security for the Smart City And Smart Homes}
\setBusinessValue{Build a working understanding of smart city disaster preparedness and resilience and its place in a modern security program; be able to explain core concepts, map them to the CIA triad, and identify common threats and controls.}
\setPriority{Must}
\setSP{3}
\setPersona{Security Architect}
\setDependencies{Lab VM or container runtime, Git repo for notes, Markdown/PDF export tool}
\setAssumptions{Time-box chapter to one iteration; open issues captured for later}
\setUserStory{As a Security Architect, I want to study and practice 'Smart City Disaster Preparedness and Resilience' so that I can apply its concepts to reduce risk and improve outcomes.}
\setNonFunctional{Security, Reliability, Performance}
\setScenario{Apply key controls for Smart City Disaster Preparedness and Resilience}
\setGiven{the lab environment and topic-specific tools for 'Smart City Disaster Preparedness and Resilience' are available}
\setWhen{I execute the hands-on objectives and lab: Hands-on: Build a small lab demonstrating key concepts in 'Smart City Disaster Preparedness and Resilience'. Capture screenshots/notes and one measurable result (e.g., a passing test, alert fired, or control verified).}
\setThen{the deliverables are produced (2–3 page brief on 'Smart City Disaster Preparedness and Resilience': risks, architecture, controls, and a checklist; plus a one-slide executive summary.); evidence (screenshots/logs/configs) is attached and reviewed}
\setDoR{Persona clear; AC drafted; Dependencies known; Estimate set.}
\setDoD{All ACs pass; Tests green; Security checks; Docs updated; Evidence attached.}
\RenderStoryCard
\begin{TasksBox}
\begin{itemize}
\TaskItem{Draft a one-page chapter plan: scope, objectives, interfaces, success metrics.}
\TaskItem{Set up tools, datasets, and accounts; document versions and configuration.}
\TaskItem{Complete objective: Define key terms and articulate why this topic matters to security outcomes.}
\TaskItem{Complete objective: Diagram the architecture/data flows and identify threat surfaces.}
\TaskItem{Execute lab: Hands-on: Build a small lab demonstrating key concepts in 'Smart City Disaster Preparedness and Resilience'. Capture screenshots/notes and one measurable result (e.g., a passing test, alert fired, or control verified).}
\end{itemize}
\end{TasksBox}
\clearpage
\setStoryID{CISH-092}
\setStoryTitle{Disaster Preparedness and Resiliency Policy Considerations for the Smart City --- Learn \& Lab}
\setEpic{Part 14: Cyber Security for the Smart City And Smart Homes}
\setBusinessValue{Build a working understanding of disaster preparedness and resiliency policy considerations for the smart city and its place in a modern security program; be able to explain core concepts, map them to the CIA triad, and identify common threats and controls.}
\setPriority{Must}
\setSP{3}
\setPersona{Security Program Manager}
\setDependencies{Lab VM or container runtime, Git repo for notes, Markdown/PDF export tool}
\setAssumptions{Time-box chapter to one iteration; open issues captured for later}
\setUserStory{As a Security Program Manager, I want to study and practice 'Disaster Preparedness and Resiliency Policy Considerations for the Smart City' so that I can apply its concepts to reduce risk and improve outcomes.}
\setNonFunctional{Security, Reliability, Performance, Compliance}
\setScenario{Apply key controls for Disaster Preparedness and Resiliency Policy Considerations for the Smart City}
\setGiven{the lab environment and topic-specific tools for 'Disaster Preparedness and Resiliency Policy Considerations for the Smart City' are available}
\setWhen{I execute the hands-on objectives and lab: Hands-on: Build a small lab demonstrating key concepts in 'Disaster Preparedness and Resiliency Policy Considerations for the Smart City'. Capture screenshots/notes and one measurable result (e.g., a passing test, alert fired, or control verified).}
\setThen{the deliverables are produced (2–3 page brief on 'Disaster Preparedness and Resiliency Policy Considerations for the Smart City': risks, architecture, controls, and a checklist; plus a one-slide executive summary.); evidence (screenshots/logs/configs) is attached and reviewed}
\setDoR{Persona clear; AC drafted; Dependencies known; Estimate set.}
\setDoD{All ACs pass; Tests green; Security checks; Docs updated; Evidence attached.}
\RenderStoryCard
\begin{TasksBox}
\begin{itemize}
\TaskItem{Draft a one-page chapter plan: scope, objectives, interfaces, success metrics.}
\TaskItem{Set up tools, datasets, and accounts; document versions and configuration.}
\TaskItem{Complete objective: Define key terms and articulate why this topic matters to security outcomes.}
\TaskItem{Complete objective: Diagram the architecture/data flows and identify threat surfaces.}
\TaskItem{Execute lab: Hands-on: Build a small lab demonstrating key concepts in 'Disaster Preparedness and Resiliency Policy Considerations for the Smart City'. Capture screenshots/notes and one measurable result (e.g., a passing test, alert fired, or control verified).}
\end{itemize}
\end{TasksBox}
\clearpage
\setStoryID{CISH-093}
\setStoryTitle{Cyber Security in Smart Homes --- Learn \& Lab}
\setEpic{Part 14: Cyber Security for the Smart City And Smart Homes}
\setBusinessValue{Build a working understanding of cyber security in smart homes and its place in a modern security program; be able to explain core concepts, map them to the CIA triad, and identify common threats and controls.}
\setPriority{Must}
\setSP{3}
\setPersona{Security Architect}
\setDependencies{Lab VM or container runtime, Git repo for notes, Markdown/PDF export tool}
\setAssumptions{Time-box chapter to one iteration; open issues captured for later}
\setUserStory{As a Security Architect, I want to study and practice 'Cyber Security in Smart Homes' so that I can apply its concepts to reduce risk and improve outcomes.}
\setNonFunctional{Security, Reliability, Performance}
\setScenario{Apply key controls for Cyber Security in Smart Homes}
\setGiven{the lab environment and topic-specific tools for 'Cyber Security in Smart Homes' are available}
\setWhen{I execute the hands-on objectives and lab: Hands-on: Build a small lab demonstrating key concepts in 'Cyber Security in Smart Homes'. Capture screenshots/notes and one measurable result (e.g., a passing test, alert fired, or control verified).}
\setThen{the deliverables are produced (2–3 page brief on 'Cyber Security in Smart Homes': risks, architecture, controls, and a checklist; plus a one-slide executive summary.); evidence (screenshots/logs/configs) is attached and reviewed}
\setDoR{Persona clear; AC drafted; Dependencies known; Estimate set.}
\setDoD{All ACs pass; Tests green; Security checks; Docs updated; Evidence attached.}
\RenderStoryCard
\begin{TasksBox}
\begin{itemize}
\TaskItem{Draft a one-page chapter plan: scope, objectives, interfaces, success metrics.}
\TaskItem{Set up tools, datasets, and accounts; document versions and configuration.}
\TaskItem{Complete objective: Define key terms and articulate why this topic matters to security outcomes.}
\TaskItem{Complete objective: Diagram the architecture/data flows and identify threat surfaces.}
\TaskItem{Execute lab: Hands-on: Build a small lab demonstrating key concepts in 'Cyber Security in Smart Homes'. Capture screenshots/notes and one measurable result (e.g., a passing test, alert fired, or control verified).}
\end{itemize}
\end{TasksBox}
\clearpage
\setStoryID{CISH-094}
\setStoryTitle{Threat Landscape and Good Practices for Smart Homes and Converged Media --- Learn \& Lab}
\setEpic{Part 14: Cyber Security for the Smart City And Smart Homes}
\setBusinessValue{Build a working understanding of threat landscape and good practices for smart homes and converged media and its place in a modern security program; be able to explain core concepts, map them to the CIA triad, and identify common threats and controls.}
\setPriority{Must}
\setSP{3}
\setPersona{Network Security Engineer}
\setDependencies{Lab VM or container runtime, Git repo for notes, Markdown/PDF export tool, Packet capture tool (tcpdump/Wireshark), Firewall/router lab}
\setAssumptions{Time-box chapter to one iteration; open issues captured for later}
\setUserStory{As a Network Security Engineer, I want to study and practice 'Threat Landscape and Good Practices for Smart Homes and Converged Media' so that I can apply its concepts to reduce risk and improve outcomes.}
\setNonFunctional{Security, Reliability, Performance}
\setScenario{Apply key controls for Threat Landscape and Good Practices for Smart Homes and Converged Media}
\setGiven{the lab environment and topic-specific tools for 'Threat Landscape and Good Practices for Smart Homes and Converged Media' are available}
\setWhen{I execute the hands-on objectives and lab: Hands-on: Build a small lab demonstrating key concepts in 'Threat Landscape and Good Practices for Smart Homes and Converged Media'. Capture screenshots/notes and one measurable result (e.g., a passing test, alert fired, or control verified).}
\setThen{the deliverables are produced (2–3 page brief on 'Threat Landscape and Good Practices for Smart Homes and Converged Media': risks, architecture, controls, and a checklist; plus a one-slide executive summary.); evidence (screenshots/logs/configs) is attached and reviewed}
\setDoR{Persona clear; AC drafted; Dependencies known; Estimate set.}
\setDoD{All ACs pass; Tests green; Security checks; Docs updated; Evidence attached.}
\RenderStoryCard
\begin{TasksBox}
\begin{itemize}
\TaskItem{Draft a one-page chapter plan: scope, objectives, interfaces, success metrics.}
\TaskItem{Set up tools, datasets, and accounts; document versions and configuration.}
\TaskItem{Complete objective: Define key terms and articulate why this topic matters to security outcomes.}
\TaskItem{Complete objective: Diagram the architecture/data flows and identify threat surfaces.}
\TaskItem{Execute lab: Hands-on: Build a small lab demonstrating key concepts in 'Threat Landscape and Good Practices for Smart Homes and Converged Media'. Capture screenshots/notes and one measurable result (e.g., a passing test, alert fired, or control verified).}
\end{itemize}
\end{TasksBox}
\clearpage
\setStoryID{CISH-095}
\setStoryTitle{Future Trends For Cyber Security for Smart- Cities And Homes --- Learn \& Lab}
\setEpic{Part 14: Cyber Security for the Smart City And Smart Homes}
\setBusinessValue{Build a working understanding of future trends for cyber security for smart- cities and homes and its place in a modern security program; be able to explain core concepts, map them to the CIA triad, and identify common threats and controls.}
\setPriority{Must}
\setSP{3}
\setPersona{Security Architect}
\setDependencies{Lab VM or container runtime, Git repo for notes, Markdown/PDF export tool}
\setAssumptions{Time-box chapter to one iteration; open issues captured for later}
\setUserStory{As a Security Architect, I want to study and practice 'Future Trends For Cyber Security for Smart- Cities And Homes' so that I can apply its concepts to reduce risk and improve outcomes.}
\setNonFunctional{Security, Reliability, Performance}
\setScenario{Apply key controls for Future Trends For Cyber Security for Smart- Cities And Homes}
\setGiven{the lab environment and topic-specific tools for 'Future Trends For Cyber Security for Smart- Cities And Homes' are available}
\setWhen{I execute the hands-on objectives and lab: Hands-on: Build a small lab demonstrating key concepts in 'Future Trends For Cyber Security for Smart- Cities And Homes'. Capture screenshots/notes and one measurable result (e.g., a passing test, alert fired, or control verified).}
\setThen{the deliverables are produced (2–3 page brief on 'Future Trends For Cyber Security for Smart- Cities And Homes': risks, architecture, controls, and a checklist; plus a one-slide executive summary.); evidence (screenshots/logs/configs) is attached and reviewed}
\setDoR{Persona clear; AC drafted; Dependencies known; Estimate set.}
\setDoD{All ACs pass; Tests green; Security checks; Docs updated; Evidence attached.}
\RenderStoryCard
\begin{TasksBox}
\begin{itemize}
\TaskItem{Draft a one-page chapter plan: scope, objectives, interfaces, success metrics.}
\TaskItem{Set up tools, datasets, and accounts; document versions and configuration.}
\TaskItem{Complete objective: Define key terms and articulate why this topic matters to security outcomes.}
\TaskItem{Complete objective: Diagram the architecture/data flows and identify threat surfaces.}
\TaskItem{Execute lab: Hands-on: Build a small lab demonstrating key concepts in 'Future Trends For Cyber Security for Smart- Cities And Homes'. Capture screenshots/notes and one measurable result (e.g., a passing test, alert fired, or control verified).}
\end{itemize}
\end{TasksBox}
\clearpage
\setStoryID{CISH-096}
\setStoryTitle{An Overview of Cyber Attacks and Defenses on Intelligent Connected Vehicles --- Learn \& Lab}
\setEpic{Part 15: Cyber Security Of Connected And Automated Vehicles}
\setBusinessValue{Build a working understanding of an overview of cyber attacks and defenses on intelligent connected vehicles and its place in a modern security program; be able to explain core concepts, map them to the CIA triad, and identify common threats and controls.}
\setPriority{Must}
\setSP{3}
\setPersona{Security Architect}
\setDependencies{Lab VM or container runtime, Git repo for notes, Markdown/PDF export tool}
\setAssumptions{Time-box chapter to one iteration; open issues captured for later}
\setUserStory{As a Security Architect, I want to study and practice 'An Overview of Cyber Attacks and Defenses on Intelligent Connected Vehicles' so that I can apply its concepts to reduce risk and improve outcomes.}
\setNonFunctional{Security, Reliability, Performance}
\setScenario{Apply key controls for An Overview of Cyber Attacks and Defenses on Intelligent Connected Vehicles}
\setGiven{the lab environment and topic-specific tools for 'An Overview of Cyber Attacks and Defenses on Intelligent Connected Vehicles' are available}
\setWhen{I execute the hands-on objectives and lab: Hands-on: Build a small lab demonstrating key concepts in 'An Overview of Cyber Attacks and Defenses on Intelligent Connected Vehicles'. Capture screenshots/notes and one measurable result (e.g., a passing test, alert fired, or control verified).}
\setThen{the deliverables are produced (2–3 page brief on 'An Overview of Cyber Attacks and Defenses on Intelligent Connected Vehicles': risks, architecture, controls, and a checklist; plus a one-slide executive summary.); evidence (screenshots/logs/configs) is attached and reviewed}
\setDoR{Persona clear; AC drafted; Dependencies known; Estimate set.}
\setDoD{All ACs pass; Tests green; Security checks; Docs updated; Evidence attached.}
\RenderStoryCard
\begin{TasksBox}
\begin{itemize}
\TaskItem{Draft a one-page chapter plan: scope, objectives, interfaces, success metrics.}
\TaskItem{Set up tools, datasets, and accounts; document versions and configuration.}
\TaskItem{Complete objective: Define key terms and articulate why this topic matters to security outcomes.}
\TaskItem{Complete objective: Diagram the architecture/data flows and identify threat surfaces.}
\TaskItem{Execute lab: Hands-on: Build a small lab demonstrating key concepts in 'An Overview of Cyber Attacks and Defenses on Intelligent Connected Vehicles'. Capture screenshots/notes and one measurable result (e.g., a passing test, alert fired, or control verified).}
\end{itemize}
\end{TasksBox}
\clearpage
\setStoryID{CISH-097}
\setStoryTitle{An Overview Of Cyber Security Issues In Vehicular Ad-hoc Networks (VANETs) --- Learn \& Lab}
\setEpic{Part 15: Cyber Security Of Connected And Automated Vehicles}
\setBusinessValue{Build a working understanding of an overview of cyber security issues in vehicular ad-hoc networks (vanets) and its place in a modern security program; be able to explain core concepts, map them to the CIA triad, and identify common threats and controls.}
\setPriority{Must}
\setSP{3}
\setPersona{Network Security Engineer}
\setDependencies{Lab VM or container runtime, Git repo for notes, Markdown/PDF export tool, Packet capture tool (tcpdump/Wireshark), Firewall/router lab}
\setAssumptions{Time-box chapter to one iteration; open issues captured for later}
\setUserStory{As a Network Security Engineer, I want to study and practice 'An Overview Of Cyber Security Issues In Vehicular Ad-hoc Networks (VANETs)' so that I can apply its concepts to reduce risk and improve outcomes.}
\setNonFunctional{Security, Reliability, Performance}
\setScenario{Apply key controls for An Overview Of Cyber Security Issues In Vehicular Ad-hoc Networks (VANETs)}
\setGiven{the lab environment and topic-specific tools for 'An Overview Of Cyber Security Issues In Vehicular Ad-hoc Networks (VANETs)' are available}
\setWhen{I execute the hands-on objectives and lab: Hands-on: Build a small lab demonstrating key concepts in 'An Overview Of Cyber Security Issues In Vehicular Ad-hoc Networks (VANETs)'. Capture screenshots/notes and one measurable result (e.g., a passing test, alert fired, or control verified).}
\setThen{the deliverables are produced (2–3 page brief on 'An Overview Of Cyber Security Issues In Vehicular Ad-hoc Networks (VANETs)': risks, architecture, controls, and a checklist; plus a one-slide executive summary.); evidence (screenshots/logs/configs) is attached and reviewed}
\setDoR{Persona clear; AC drafted; Dependencies known; Estimate set.}
\setDoD{All ACs pass; Tests green; Security checks; Docs updated; Evidence attached.}
\RenderStoryCard
\begin{TasksBox}
\begin{itemize}
\TaskItem{Draft a one-page chapter plan: scope, objectives, interfaces, success metrics.}
\TaskItem{Set up tools, datasets, and accounts; document versions and configuration.}
\TaskItem{Complete objective: Define key terms and articulate why this topic matters to security outcomes.}
\TaskItem{Complete objective: Diagram the architecture/data flows and identify threat surfaces.}
\TaskItem{Execute lab: Hands-on: Build a small lab demonstrating key concepts in 'An Overview Of Cyber Security Issues In Vehicular Ad-hoc Networks (VANETs)'. Capture screenshots/notes and one measurable result (e.g., a passing test, alert fired, or control verified).}
\end{itemize}
\end{TasksBox}
\clearpage
\setStoryID{CISH-098}
\setStoryTitle{An Overview: Various Cyber Attacks in VANET --- Learn \& Lab}
\setEpic{Part 15: Cyber Security Of Connected And Automated Vehicles}
\setBusinessValue{Build a working understanding of an overview: various cyber attacks in vanet and its place in a modern security program; be able to explain core concepts, map them to the CIA triad, and identify common threats and controls.}
\setPriority{Must}
\setSP{3}
\setPersona{Security Architect}
\setDependencies{Lab VM or container runtime, Git repo for notes, Markdown/PDF export tool}
\setAssumptions{Time-box chapter to one iteration; open issues captured for later}
\setUserStory{As a Security Architect, I want to study and practice 'An Overview: Various Cyber Attacks in VANET' so that I can apply its concepts to reduce risk and improve outcomes.}
\setNonFunctional{Security, Reliability, Performance}
\setScenario{Apply key controls for An Overview: Various Cyber Attacks in VANET}
\setGiven{the lab environment and topic-specific tools for 'An Overview: Various Cyber Attacks in VANET' are available}
\setWhen{I execute the hands-on objectives and lab: Hands-on: Build a small lab demonstrating key concepts in 'An Overview: Various Cyber Attacks in VANET'. Capture screenshots/notes and one measurable result (e.g., a passing test, alert fired, or control verified).}
\setThen{the deliverables are produced (2–3 page brief on 'An Overview: Various Cyber Attacks in VANET': risks, architecture, controls, and a checklist; plus a one-slide executive summary.); evidence (screenshots/logs/configs) is attached and reviewed}
\setDoR{Persona clear; AC drafted; Dependencies known; Estimate set.}
\setDoD{All ACs pass; Tests green; Security checks; Docs updated; Evidence attached.}
\RenderStoryCard
\begin{TasksBox}
\begin{itemize}
\TaskItem{Draft a one-page chapter plan: scope, objectives, interfaces, success metrics.}
\TaskItem{Set up tools, datasets, and accounts; document versions and configuration.}
\TaskItem{Complete objective: Define key terms and articulate why this topic matters to security outcomes.}
\TaskItem{Complete objective: Diagram the architecture/data flows and identify threat surfaces.}
\TaskItem{Execute lab: Hands-on: Build a small lab demonstrating key concepts in 'An Overview: Various Cyber Attacks in VANET'. Capture screenshots/notes and one measurable result (e.g., a passing test, alert fired, or control verified).}
\end{itemize}
\end{TasksBox}
\clearpage
\setStoryID{CISH-099}
\setStoryTitle{Security Through Diversity --- Learn \& Lab}
\setEpic{Part 16: Advanced Security}
\setBusinessValue{Build a working understanding of security through diversity and its place in a modern security program; be able to explain core concepts, map them to the CIA triad, and identify common threats and controls.}
\setPriority{Must}
\setSP{3}
\setPersona{Security Engineer}
\setDependencies{Lab VM or container runtime, Git repo for notes, Markdown/PDF export tool}
\setAssumptions{Time-box chapter to one iteration; open issues captured for later}
\setUserStory{As a Security Engineer, I want to study and practice 'Security Through Diversity' so that I can apply its concepts to reduce risk and improve outcomes.}
\setNonFunctional{Security, Reliability, Performance}
\setScenario{Apply key controls for Security Through Diversity}
\setGiven{the lab environment and topic-specific tools for 'Security Through Diversity' are available}
\setWhen{I execute the hands-on objectives and lab: Hands-on: Build a small lab demonstrating key concepts in 'Security Through Diversity'. Capture screenshots/notes and one measurable result (e.g., a passing test, alert fired, or control verified).}
\setThen{the deliverables are produced (2–3 page brief on 'Security Through Diversity': risks, architecture, controls, and a checklist; plus a one-slide executive summary.); evidence (screenshots/logs/configs) is attached and reviewed}
\setDoR{Persona clear; AC drafted; Dependencies known; Estimate set.}
\setDoD{All ACs pass; Tests green; Security checks; Docs updated; Evidence attached.}
\RenderStoryCard
\begin{TasksBox}
\begin{itemize}
\TaskItem{Draft a one-page chapter plan: scope, objectives, interfaces, success metrics.}
\TaskItem{Set up tools, datasets, and accounts; document versions and configuration.}
\TaskItem{Complete objective: Define key terms and articulate why this topic matters to security outcomes.}
\TaskItem{Complete objective: Diagram the architecture/data flows and identify threat surfaces.}
\TaskItem{Execute lab: Hands-on: Build a small lab demonstrating key concepts in 'Security Through Diversity'. Capture screenshots/notes and one measurable result (e.g., a passing test, alert fired, or control verified).}
\end{itemize}
\end{TasksBox}
\clearpage
\setStoryID{CISH-100}
\setStoryTitle{Online e-Reputation Management Services --- Learn \& Lab}
\setEpic{Part 16: Advanced Security}
\setBusinessValue{Build a working understanding of online e-reputation management services and its place in a modern security program; be able to explain core concepts, map them to the CIA triad, and identify common threats and controls.}
\setPriority{Must}
\setSP{3}
\setPersona{Application Security Engineer}
\setDependencies{Lab VM or container runtime, Git repo for notes, Markdown/PDF export tool}
\setAssumptions{Time-box chapter to one iteration; open issues captured for later}
\setUserStory{As a Application Security Engineer, I want to study and practice 'Online e-Reputation Management Services' so that I can apply its concepts to reduce risk and improve outcomes.}
\setNonFunctional{Security, Reliability, Performance}
\setScenario{Apply key controls for Online e-Reputation Management Services}
\setGiven{the lab environment and topic-specific tools for 'Online e-Reputation Management Services' are available}
\setWhen{I execute the hands-on objectives and lab: Hands-on: Build a small lab demonstrating key concepts in 'Online e-Reputation Management Services'. Capture screenshots/notes and one measurable result (e.g., a passing test, alert fired, or control verified).}
\setThen{the deliverables are produced (2–3 page brief on 'Online e-Reputation Management Services': risks, architecture, controls, and a checklist; plus a one-slide executive summary.); evidence (screenshots/logs/configs) is attached and reviewed}
\setDoR{Persona clear; AC drafted; Dependencies known; Estimate set.}
\setDoD{All ACs pass; Tests green; Security checks; Docs updated; Evidence attached.}
\RenderStoryCard
\begin{TasksBox}
\begin{itemize}
\TaskItem{Draft a one-page chapter plan: scope, objectives, interfaces, success metrics.}
\TaskItem{Set up tools, datasets, and accounts; document versions and configuration.}
\TaskItem{Complete objective: Define key terms and articulate why this topic matters to security outcomes.}
\TaskItem{Complete objective: Diagram the architecture/data flows and identify threat surfaces.}
\TaskItem{Execute lab: Hands-on: Build a small lab demonstrating key concepts in 'Online e-Reputation Management Services'. Capture screenshots/notes and one measurable result (e.g., a passing test, alert fired, or control verified).}
\end{itemize}
\end{TasksBox}
\clearpage
\setStoryID{CISH-101}
\setStoryTitle{Content Filtering --- Learn \& Lab}
\setEpic{Part 16: Advanced Security}
\setBusinessValue{Build a working understanding of content filtering and its place in a modern security program; be able to explain core concepts, map them to the CIA triad, and identify common threats and controls.}
\setPriority{Must}
\setSP{3}
\setPersona{Application Security Engineer}
\setDependencies{Lab VM or container runtime, Git repo for notes, Markdown/PDF export tool}
\setAssumptions{Time-box chapter to one iteration; open issues captured for later}
\setUserStory{As a Application Security Engineer, I want to study and practice 'Content Filtering' so that I can apply its concepts to reduce risk and improve outcomes.}
\setNonFunctional{Security, Reliability, Performance}
\setScenario{Apply key controls for Content Filtering}
\setGiven{the lab environment and topic-specific tools for 'Content Filtering' are available}
\setWhen{I execute the hands-on objectives and lab: Hands-on: Build a small lab demonstrating key concepts in 'Content Filtering'. Capture screenshots/notes and one measurable result (e.g., a passing test, alert fired, or control verified).}
\setThen{the deliverables are produced (2–3 page brief on 'Content Filtering': risks, architecture, controls, and a checklist; plus a one-slide executive summary.); evidence (screenshots/logs/configs) is attached and reviewed}
\setDoR{Persona clear; AC drafted; Dependencies known; Estimate set.}
\setDoD{All ACs pass; Tests green; Security checks; Docs updated; Evidence attached.}
\RenderStoryCard
\begin{TasksBox}
\begin{itemize}
\TaskItem{Draft a one-page chapter plan: scope, objectives, interfaces, success metrics.}
\TaskItem{Set up tools, datasets, and accounts; document versions and configuration.}
\TaskItem{Complete objective: Define key terms and articulate why this topic matters to security outcomes.}
\TaskItem{Complete objective: Diagram the architecture/data flows and identify threat surfaces.}
\TaskItem{Execute lab: Hands-on: Build a small lab demonstrating key concepts in 'Content Filtering'. Capture screenshots/notes and one measurable result (e.g., a passing test, alert fired, or control verified).}
\end{itemize}
\end{TasksBox}
\clearpage
\setStoryID{CISH-102}
\setStoryTitle{Data Loss Protection --- Learn \& Lab}
\setEpic{Part 16: Advanced Security}
\setBusinessValue{Build a working understanding of data loss protection and its place in a modern security program; be able to explain core concepts, map them to the CIA triad, and identify common threats and controls.}
\setPriority{Must}
\setSP{3}
\setPersona{Security Engineer}
\setDependencies{Lab VM or container runtime, Git repo for notes, Markdown/PDF export tool}
\setAssumptions{Time-box chapter to one iteration; open issues captured for later}
\setUserStory{As a Security Engineer, I want to study and practice 'Data Loss Protection' so that I can apply its concepts to reduce risk and improve outcomes.}
\setNonFunctional{Security, Reliability, Performance}
\setScenario{Apply key controls for Data Loss Protection}
\setGiven{the lab environment and topic-specific tools for 'Data Loss Protection' are available}
\setWhen{I execute the hands-on objectives and lab: Hands-on: Build a small lab demonstrating key concepts in 'Data Loss Protection'. Capture screenshots/notes and one measurable result (e.g., a passing test, alert fired, or control verified).}
\setThen{the deliverables are produced (2–3 page brief on 'Data Loss Protection': risks, architecture, controls, and a checklist; plus a one-slide executive summary.); evidence (screenshots/logs/configs) is attached and reviewed}
\setDoR{Persona clear; AC drafted; Dependencies known; Estimate set.}
\setDoD{All ACs pass; Tests green; Security checks; Docs updated; Evidence attached.}
\RenderStoryCard
\begin{TasksBox}
\begin{itemize}
\TaskItem{Draft a one-page chapter plan: scope, objectives, interfaces, success metrics.}
\TaskItem{Set up tools, datasets, and accounts; document versions and configuration.}
\TaskItem{Complete objective: Define key terms and articulate why this topic matters to security outcomes.}
\TaskItem{Complete objective: Diagram the architecture/data flows and identify threat surfaces.}
\TaskItem{Execute lab: Hands-on: Build a small lab demonstrating key concepts in 'Data Loss Protection'. Capture screenshots/notes and one measurable result (e.g., a passing test, alert fired, or control verified).}
\end{itemize}
\end{TasksBox}
\clearpage
\setStoryID{CISH-103}
\setStoryTitle{Satellite Cyber Attack Search and Destroy --- Learn \& Lab}
\setEpic{Part 16: Advanced Security}
\setBusinessValue{Build a working understanding of satellite cyber attack search and destroy and its place in a modern security program; be able to explain core concepts, map them to the CIA triad, and identify common threats and controls.}
\setPriority{Must}
\setSP{3}
\setPersona{Security Engineer}
\setDependencies{Lab VM or container runtime, Git repo for notes, Markdown/PDF export tool}
\setAssumptions{Time-box chapter to one iteration; open issues captured for later}
\setUserStory{As a Security Engineer, I want to study and practice 'Satellite Cyber Attack Search and Destroy' so that I can apply its concepts to reduce risk and improve outcomes.}
\setNonFunctional{Security, Reliability, Performance}
\setScenario{Apply key controls for Satellite Cyber Attack Search and Destroy}
\setGiven{the lab environment and topic-specific tools for 'Satellite Cyber Attack Search and Destroy' are available}
\setWhen{I execute the hands-on objectives and lab: Hands-on: Build a small lab demonstrating key concepts in 'Satellite Cyber Attack Search and Destroy'. Capture screenshots/notes and one measurable result (e.g., a passing test, alert fired, or control verified).}
\setThen{the deliverables are produced (2–3 page brief on 'Satellite Cyber Attack Search and Destroy': risks, architecture, controls, and a checklist; plus a one-slide executive summary.); evidence (screenshots/logs/configs) is attached and reviewed}
\setDoR{Persona clear; AC drafted; Dependencies known; Estimate set.}
\setDoD{All ACs pass; Tests green; Security checks; Docs updated; Evidence attached.}
\RenderStoryCard
\begin{TasksBox}
\begin{itemize}
\TaskItem{Draft a one-page chapter plan: scope, objectives, interfaces, success metrics.}
\TaskItem{Set up tools, datasets, and accounts; document versions and configuration.}
\TaskItem{Complete objective: Define key terms and articulate why this topic matters to security outcomes.}
\TaskItem{Complete objective: Diagram the architecture/data flows and identify threat surfaces.}
\TaskItem{Execute lab: Hands-on: Build a small lab demonstrating key concepts in 'Satellite Cyber Attack Search and Destroy'. Capture screenshots/notes and one measurable result (e.g., a passing test, alert fired, or control verified).}
\end{itemize}
\end{TasksBox}
\clearpage
\setStoryID{CISH-104}
\setStoryTitle{Verifiable Voting Systems --- Learn \& Lab}
\setEpic{Part 16: Advanced Security}
\setBusinessValue{Build a working understanding of verifiable voting systems and its place in a modern security program; be able to explain core concepts, map them to the CIA triad, and identify common threats and controls.}
\setPriority{Must}
\setSP{3}
\setPersona{Security Engineer}
\setDependencies{Lab VM or container runtime, Git repo for notes, Markdown/PDF export tool}
\setAssumptions{Time-box chapter to one iteration; open issues captured for later}
\setUserStory{As a Security Engineer, I want to study and practice 'Verifiable Voting Systems' so that I can apply its concepts to reduce risk and improve outcomes.}
\setNonFunctional{Security, Reliability, Performance}
\setScenario{Apply key controls for Verifiable Voting Systems}
\setGiven{the lab environment and topic-specific tools for 'Verifiable Voting Systems' are available}
\setWhen{I execute the hands-on objectives and lab: Hands-on: Build a small lab demonstrating key concepts in 'Verifiable Voting Systems'. Capture screenshots/notes and one measurable result (e.g., a passing test, alert fired, or control verified).}
\setThen{the deliverables are produced (2–3 page brief on 'Verifiable Voting Systems': risks, architecture, controls, and a checklist; plus a one-slide executive summary.); evidence (screenshots/logs/configs) is attached and reviewed}
\setDoR{Persona clear; AC drafted; Dependencies known; Estimate set.}
\setDoD{All ACs pass; Tests green; Security checks; Docs updated; Evidence attached.}
\RenderStoryCard
\begin{TasksBox}
\begin{itemize}
\TaskItem{Draft a one-page chapter plan: scope, objectives, interfaces, success metrics.}
\TaskItem{Set up tools, datasets, and accounts; document versions and configuration.}
\TaskItem{Complete objective: Define key terms and articulate why this topic matters to security outcomes.}
\TaskItem{Complete objective: Diagram the architecture/data flows and identify threat surfaces.}
\TaskItem{Execute lab: Hands-on: Build a small lab demonstrating key concepts in 'Verifiable Voting Systems'. Capture screenshots/notes and one measurable result (e.g., a passing test, alert fired, or control verified).}
\end{itemize}
\end{TasksBox}
\clearpage
\setStoryID{CISH-105}
\setStoryTitle{Advanced Data Encryption --- Learn \& Lab}
\setEpic{Part 16: Advanced Security}
\setBusinessValue{Build a working understanding of advanced data encryption and its place in a modern security program; be able to explain core concepts, map them to the CIA triad, and identify common threats and controls.}
\setPriority{Must}
\setSP{3}
\setPersona{Platform Engineer}
\setDependencies{Lab VM or container runtime, Git repo for notes, Markdown/PDF export tool, OpenSSL/mkcert, TLS scanner}
\setAssumptions{Time-box chapter to one iteration; open issues captured for later}
\setUserStory{As a Platform Engineer, I want to study and practice 'Advanced Data Encryption' so that I can apply its concepts to reduce risk and improve outcomes.}
\setNonFunctional{Security, Reliability, Performance}
\setScenario{Apply key controls for Advanced Data Encryption}
\setGiven{the lab environment and topic-specific tools for 'Advanced Data Encryption' are available}
\setWhen{I execute the hands-on objectives and lab: Hands-on: Build a small lab demonstrating key concepts in 'Advanced Data Encryption'. Capture screenshots/notes and one measurable result (e.g., a passing test, alert fired, or control verified).}
\setThen{the deliverables are produced (2–3 page brief on 'Advanced Data Encryption': risks, architecture, controls, and a checklist; plus a one-slide executive summary.); evidence (screenshots/logs/configs) is attached and reviewed}
\setDoR{Persona clear; AC drafted; Dependencies known; Estimate set.}
\setDoD{All ACs pass; Tests green; Security checks; Docs updated; Evidence attached.}
\RenderStoryCard
\begin{TasksBox}
\begin{itemize}
\TaskItem{Draft a one-page chapter plan: scope, objectives, interfaces, success metrics.}
\TaskItem{Set up tools, datasets, and accounts; document versions and configuration.}
\TaskItem{Complete objective: Define key terms and articulate why this topic matters to security outcomes.}
\TaskItem{Complete objective: Diagram the architecture/data flows and identify threat surfaces.}
\TaskItem{Execute lab: Hands-on: Build a small lab demonstrating key concepts in 'Advanced Data Encryption'. Capture screenshots/notes and one measurable result (e.g., a passing test, alert fired, or control verified).}
\end{itemize}
\end{TasksBox}
\clearpage
\setStoryID{CISH-106}
\setStoryTitle{Use Of Artificial Intelligence (AI) In Cyber Security --- Learn \& Lab}
\setEpic{Part 16: Advanced Security}
\setBusinessValue{Build a working understanding of use of artificial intelligence (ai) in cyber security and its place in a modern security program; be able to explain core concepts, map them to the CIA triad, and identify common threats and controls.}
\setPriority{Must}
\setSP{3}
\setPersona{ML Security Engineer}
\setDependencies{Lab VM or container runtime, Git repo for notes, Markdown/PDF export tool}
\setAssumptions{Time-box chapter to one iteration; open issues captured for later}
\setUserStory{As a ML Security Engineer, I want to study and practice 'Use Of Artificial Intelligence (AI) In Cyber Security' so that I can apply its concepts to reduce risk and improve outcomes.}
\setNonFunctional{Security, Reliability, Performance}
\setScenario{Apply key controls for Use Of Artificial Intelligence (AI) In Cyber Security}
\setGiven{the lab environment and topic-specific tools for 'Use Of Artificial Intelligence (AI) In Cyber Security' are available}
\setWhen{I execute the hands-on objectives and lab: Hands-on: Build a small lab demonstrating key concepts in 'Use Of Artificial Intelligence (AI) In Cyber Security'. Capture screenshots/notes and one measurable result (e.g., a passing test, alert fired, or control verified).}
\setThen{the deliverables are produced (2–3 page brief on 'Use Of Artificial Intelligence (AI) In Cyber Security': risks, architecture, controls, and a checklist; plus a one-slide executive summary.); evidence (screenshots/logs/configs) is attached and reviewed}
\setDoR{Persona clear; AC drafted; Dependencies known; Estimate set.}
\setDoD{All ACs pass; Tests green; Security checks; Docs updated; Evidence attached.}
\RenderStoryCard
\begin{TasksBox}
\begin{itemize}
\TaskItem{Draft a one-page chapter plan: scope, objectives, interfaces, success metrics.}
\TaskItem{Set up tools, datasets, and accounts; document versions and configuration.}
\TaskItem{Complete objective: Define key terms and articulate why this topic matters to security outcomes.}
\TaskItem{Complete objective: Diagram the architecture/data flows and identify threat surfaces.}
\TaskItem{Execute lab: Hands-on: Build a small lab demonstrating key concepts in 'Use Of Artificial Intelligence (AI) In Cyber Security'. Capture screenshots/notes and one measurable result (e.g., a passing test, alert fired, or control verified).}
\end{itemize}
\end{TasksBox}
\clearpage
\setStoryID{CISH-107}
\setStoryTitle{New Cyber Security Vulnerabilities And Trends Facing Aerospace And Defense Systems --- Learn \& Lab}
\setEpic{Part 17: Future Cyber Security Trends And Directions}
\setBusinessValue{Build a working understanding of new cyber security vulnerabilities and trends facing aerospace and defense systems and its place in a modern security program; be able to explain core concepts, map them to the CIA triad, and identify common threats and controls.}
\setPriority{Must}
\setSP{3}
\setPersona{Security Architect}
\setDependencies{Lab VM or container runtime, Git repo for notes, Markdown/PDF export tool}
\setAssumptions{Time-box chapter to one iteration; open issues captured for later}
\setUserStory{As a Security Architect, I want to study and practice 'New Cyber Security Vulnerabilities And Trends Facing Aerospace And Defense Systems' so that I can apply its concepts to reduce risk and improve outcomes.}
\setNonFunctional{Security, Reliability, Performance}
\setScenario{Apply key controls for New Cyber Security Vulnerabilities And Trends Facing Aerospace And Defense Systems}
\setGiven{the lab environment and topic-specific tools for 'New Cyber Security Vulnerabilities And Trends Facing Aerospace And Defense Systems' are available}
\setWhen{I execute the hands-on objectives and lab: Hands-on: Build a small lab demonstrating key concepts in 'New Cyber Security Vulnerabilities And Trends Facing Aerospace And Defense Systems'. Capture screenshots/notes and one measurable result (e.g., a passing test, alert fired, or control verified).}
\setThen{the deliverables are produced (2–3 page brief on 'New Cyber Security Vulnerabilities And Trends Facing Aerospace And Defense Systems': risks, architecture, controls, and a checklist; plus a one-slide executive summary.); evidence (screenshots/logs/configs) is attached and reviewed}
\setDoR{Persona clear; AC drafted; Dependencies known; Estimate set.}
\setDoD{All ACs pass; Tests green; Security checks; Docs updated; Evidence attached.}
\RenderStoryCard
\begin{TasksBox}
\begin{itemize}
\TaskItem{Draft a one-page chapter plan: scope, objectives, interfaces, success metrics.}
\TaskItem{Set up tools, datasets, and accounts; document versions and configuration.}
\TaskItem{Complete objective: Define key terms and articulate why this topic matters to security outcomes.}
\TaskItem{Complete objective: Diagram the architecture/data flows and identify threat surfaces.}
\TaskItem{Execute lab: Hands-on: Build a small lab demonstrating key concepts in 'New Cyber Security Vulnerabilities And Trends Facing Aerospace And Defense Systems'. Capture screenshots/notes and one measurable result (e.g., a passing test, alert fired, or control verified).}
\end{itemize}
\end{TasksBox}
\clearpage
\setStoryID{CISH-108}
\setStoryTitle{How Aerospace And Defense Companies Will Respond To Future Cyber Security Threats --- Learn \& Lab}
\setEpic{Part 17: Future Cyber Security Trends And Directions}
\setBusinessValue{Build a working understanding of how aerospace and defense companies will respond to future cyber security threats and its place in a modern security program; be able to explain core concepts, map them to the CIA triad, and identify common threats and controls.}
\setPriority{Must}
\setSP{3}
\setPersona{Security Architect}
\setDependencies{Lab VM or container runtime, Git repo for notes, Markdown/PDF export tool}
\setAssumptions{Time-box chapter to one iteration; open issues captured for later}
\setUserStory{As a Security Architect, I want to study and practice 'How Aerospace And Defense Companies Will Respond To Future Cyber Security Threats' so that I can apply its concepts to reduce risk and improve outcomes.}
\setNonFunctional{Security, Reliability, Performance}
\setScenario{Apply key controls for How Aerospace And Defense Companies Will Respond To Future Cyber Security Threats}
\setGiven{the lab environment and topic-specific tools for 'How Aerospace And Defense Companies Will Respond To Future Cyber Security Threats' are available}
\setWhen{I execute the hands-on objectives and lab: Hands-on: Build a small lab demonstrating key concepts in 'How Aerospace And Defense Companies Will Respond To Future Cyber Security Threats'. Capture screenshots/notes and one measurable result (e.g., a passing test, alert fired, or control verified).}
\setThen{the deliverables are produced (2–3 page brief on 'How Aerospace And Defense Companies Will Respond To Future Cyber Security Threats': risks, architecture, controls, and a checklist; plus a one-slide executive summary.); evidence (screenshots/logs/configs) is attached and reviewed}
\setDoR{Persona clear; AC drafted; Dependencies known; Estimate set.}
\setDoD{All ACs pass; Tests green; Security checks; Docs updated; Evidence attached.}
\RenderStoryCard
\begin{TasksBox}
\begin{itemize}
\TaskItem{Draft a one-page chapter plan: scope, objectives, interfaces, success metrics.}
\TaskItem{Set up tools, datasets, and accounts; document versions and configuration.}
\TaskItem{Complete objective: Define key terms and articulate why this topic matters to security outcomes.}
\TaskItem{Complete objective: Diagram the architecture/data flows and identify threat surfaces.}
\TaskItem{Execute lab: Hands-on: Build a small lab demonstrating key concepts in 'How Aerospace And Defense Companies Will Respond To Future Cyber Security Threats'. Capture screenshots/notes and one measurable result (e.g., a passing test, alert fired, or control verified).}
\end{itemize}
\end{TasksBox}
\clearpage
\setStoryID{CISH-109}
\setStoryTitle{Understanding the Future Trends of the Aviation Cyber Security Threat Landscape --- Learn \& Lab}
\setEpic{Part 17: Future Cyber Security Trends And Directions}
\setBusinessValue{Build a working understanding of understanding the future trends of the aviation cyber security threat landscape and its place in a modern security program; be able to explain core concepts, map them to the CIA triad, and identify common threats and controls.}
\setPriority{Must}
\setSP{3}
\setPersona{Network Security Engineer}
\setDependencies{Lab VM or container runtime, Git repo for notes, Markdown/PDF export tool, Packet capture tool (tcpdump/Wireshark), Firewall/router lab}
\setAssumptions{Time-box chapter to one iteration; open issues captured for later}
\setUserStory{As a Network Security Engineer, I want to study and practice 'Understanding the Future Trends of the Aviation Cyber Security Threat Landscape' so that I can apply its concepts to reduce risk and improve outcomes.}
\setNonFunctional{Security, Reliability, Performance}
\setScenario{Apply key controls for Understanding the Future Trends of the Aviation Cyber Security Threat Landscape}
\setGiven{the lab environment and topic-specific tools for 'Understanding the Future Trends of the Aviation Cyber Security Threat Landscape' are available}
\setWhen{I execute the hands-on objectives and lab: Hands-on: Build a small lab demonstrating key concepts in 'Understanding the Future Trends of the Aviation Cyber Security Threat Landscape'. Capture screenshots/notes and one measurable result (e.g., a passing test, alert fired, or control verified).}
\setThen{the deliverables are produced (2–3 page brief on 'Understanding the Future Trends of the Aviation Cyber Security Threat Landscape': risks, architecture, controls, and a checklist; plus a one-slide executive summary.); evidence (screenshots/logs/configs) is attached and reviewed}
\setDoR{Persona clear; AC drafted; Dependencies known; Estimate set.}
\setDoD{All ACs pass; Tests green; Security checks; Docs updated; Evidence attached.}
\RenderStoryCard
\begin{TasksBox}
\begin{itemize}
\TaskItem{Draft a one-page chapter plan: scope, objectives, interfaces, success metrics.}
\TaskItem{Set up tools, datasets, and accounts; document versions and configuration.}
\TaskItem{Complete objective: Define key terms and articulate why this topic matters to security outcomes.}
\TaskItem{Complete objective: Diagram the architecture/data flows and identify threat surfaces.}
\TaskItem{Execute lab: Hands-on: Build a small lab demonstrating key concepts in 'Understanding the Future Trends of the Aviation Cyber Security Threat Landscape'. Capture screenshots/notes and one measurable result (e.g., a passing test, alert fired, or control verified).}
\end{itemize}
\end{TasksBox}
\clearpage
\setStoryID{CISH-110}
\setStoryTitle{Fighting the Rising Trends Of Cyber Attacks on Aviation --- Learn \& Lab}
\setEpic{Part 17: Future Cyber Security Trends And Directions}
\setBusinessValue{Build a working understanding of fighting the rising trends of cyber attacks on aviation and its place in a modern security program; be able to explain core concepts, map them to the CIA triad, and identify common threats and controls.}
\setPriority{Must}
\setSP{3}
\setPersona{Security Architect}
\setDependencies{Lab VM or container runtime, Git repo for notes, Markdown/PDF export tool}
\setAssumptions{Time-box chapter to one iteration; open issues captured for later}
\setUserStory{As a Security Architect, I want to study and practice 'Fighting the Rising Trends Of Cyber Attacks on Aviation' so that I can apply its concepts to reduce risk and improve outcomes.}
\setNonFunctional{Security, Reliability, Performance}
\setScenario{Apply key controls for Fighting the Rising Trends Of Cyber Attacks on Aviation}
\setGiven{the lab environment and topic-specific tools for 'Fighting the Rising Trends Of Cyber Attacks on Aviation' are available}
\setWhen{I execute the hands-on objectives and lab: Hands-on: Build a small lab demonstrating key concepts in 'Fighting the Rising Trends Of Cyber Attacks on Aviation'. Capture screenshots/notes and one measurable result (e.g., a passing test, alert fired, or control verified).}
\setThen{the deliverables are produced (2–3 page brief on 'Fighting the Rising Trends Of Cyber Attacks on Aviation': risks, architecture, controls, and a checklist; plus a one-slide executive summary.); evidence (screenshots/logs/configs) is attached and reviewed}
\setDoR{Persona clear; AC drafted; Dependencies known; Estimate set.}
\setDoD{All ACs pass; Tests green; Security checks; Docs updated; Evidence attached.}
\RenderStoryCard
\begin{TasksBox}
\begin{itemize}
\TaskItem{Draft a one-page chapter plan: scope, objectives, interfaces, success metrics.}
\TaskItem{Set up tools, datasets, and accounts; document versions and configuration.}
\TaskItem{Complete objective: Define key terms and articulate why this topic matters to security outcomes.}
\TaskItem{Complete objective: Diagram the architecture/data flows and identify threat surfaces.}
\TaskItem{Execute lab: Hands-on: Build a small lab demonstrating key concepts in 'Fighting the Rising Trends Of Cyber Attacks on Aviation'. Capture screenshots/notes and one measurable result (e.g., a passing test, alert fired, or control verified).}
\end{itemize}
\end{TasksBox}
\clearpage
\setStoryID{CISH-111}
\setStoryTitle{Future Trends For Cyber Security Hardening of Aviation Systems --- Learn \& Lab}
\setEpic{Part 17: Future Cyber Security Trends And Directions}
\setBusinessValue{Build a working understanding of future trends for cyber security hardening of aviation systems and its place in a modern security program; be able to explain core concepts, map them to the CIA triad, and identify common threats and controls.}
\setPriority{Must}
\setSP{3}
\setPersona{Security Architect}
\setDependencies{Lab VM or container runtime, Git repo for notes, Markdown/PDF export tool}
\setAssumptions{Time-box chapter to one iteration; open issues captured for later}
\setUserStory{As a Security Architect, I want to study and practice 'Future Trends For Cyber Security Hardening of Aviation Systems' so that I can apply its concepts to reduce risk and improve outcomes.}
\setNonFunctional{Security, Reliability, Performance}
\setScenario{Apply key controls for Future Trends For Cyber Security Hardening of Aviation Systems}
\setGiven{the lab environment and topic-specific tools for 'Future Trends For Cyber Security Hardening of Aviation Systems' are available}
\setWhen{I execute the hands-on objectives and lab: Hands-on: Build a small lab demonstrating key concepts in 'Future Trends For Cyber Security Hardening of Aviation Systems'. Capture screenshots/notes and one measurable result (e.g., a passing test, alert fired, or control verified).}
\setThen{the deliverables are produced (2–3 page brief on 'Future Trends For Cyber Security Hardening of Aviation Systems': risks, architecture, controls, and a checklist; plus a one-slide executive summary.); evidence (screenshots/logs/configs) is attached and reviewed}
\setDoR{Persona clear; AC drafted; Dependencies known; Estimate set.}
\setDoD{All ACs pass; Tests green; Security checks; Docs updated; Evidence attached.}
\RenderStoryCard
\begin{TasksBox}
\begin{itemize}
\TaskItem{Draft a one-page chapter plan: scope, objectives, interfaces, success metrics.}
\TaskItem{Set up tools, datasets, and accounts; document versions and configuration.}
\TaskItem{Complete objective: Define key terms and articulate why this topic matters to security outcomes.}
\TaskItem{Complete objective: Diagram the architecture/data flows and identify threat surfaces.}
\TaskItem{Execute lab: Hands-on: Build a small lab demonstrating key concepts in 'Future Trends For Cyber Security Hardening of Aviation Systems'. Capture screenshots/notes and one measurable result (e.g., a passing test, alert fired, or control verified).}
\end{itemize}
\end{TasksBox}
\clearpage
\setStoryID{CISH-112}
\setStoryTitle{Future Trends For Cyber Security in the Gaming Industry --- Learn \& Lab}
\setEpic{Part 17: Future Cyber Security Trends And Directions}
\setBusinessValue{Build a working understanding of future trends for cyber security in the gaming industry and its place in a modern security program; be able to explain core concepts, map them to the CIA triad, and identify common threats and controls.}
\setPriority{Must}
\setSP{3}
\setPersona{Security Architect}
\setDependencies{Lab VM or container runtime, Git repo for notes, Markdown/PDF export tool}
\setAssumptions{Time-box chapter to one iteration; open issues captured for later}
\setUserStory{As a Security Architect, I want to study and practice 'Future Trends For Cyber Security in the Gaming Industry' so that I can apply its concepts to reduce risk and improve outcomes.}
\setNonFunctional{Security, Reliability, Performance}
\setScenario{Apply key controls for Future Trends For Cyber Security in the Gaming Industry}
\setGiven{the lab environment and topic-specific tools for 'Future Trends For Cyber Security in the Gaming Industry' are available}
\setWhen{I execute the hands-on objectives and lab: Hands-on: Build a small lab demonstrating key concepts in 'Future Trends For Cyber Security in the Gaming Industry'. Capture screenshots/notes and one measurable result (e.g., a passing test, alert fired, or control verified).}
\setThen{the deliverables are produced (2–3 page brief on 'Future Trends For Cyber Security in the Gaming Industry': risks, architecture, controls, and a checklist; plus a one-slide executive summary.); evidence (screenshots/logs/configs) is attached and reviewed}
\setDoR{Persona clear; AC drafted; Dependencies known; Estimate set.}
\setDoD{All ACs pass; Tests green; Security checks; Docs updated; Evidence attached.}
\RenderStoryCard
\begin{TasksBox}
\begin{itemize}
\TaskItem{Draft a one-page chapter plan: scope, objectives, interfaces, success metrics.}
\TaskItem{Set up tools, datasets, and accounts; document versions and configuration.}
\TaskItem{Complete objective: Define key terms and articulate why this topic matters to security outcomes.}
\TaskItem{Complete objective: Diagram the architecture/data flows and identify threat surfaces.}
\TaskItem{Execute lab: Hands-on: Build a small lab demonstrating key concepts in 'Future Trends For Cyber Security in the Gaming Industry'. Capture screenshots/notes and one measurable result (e.g., a passing test, alert fired, or control verified).}
\end{itemize}
\end{TasksBox}
\clearpage
\setStoryID{CISH-113}
\setStoryTitle{Future Trends For Cyber Attacks in the Health Care Industry --- Learn \& Lab}
\setEpic{Part 17: Future Cyber Security Trends And Directions}
\setBusinessValue{Build a working understanding of future trends for cyber attacks in the health care industry and its place in a modern security program; be able to explain core concepts, map them to the CIA triad, and identify common threats and controls.}
\setPriority{Must}
\setSP{3}
\setPersona{Security Architect}
\setDependencies{Lab VM or container runtime, Git repo for notes, Markdown/PDF export tool}
\setAssumptions{Time-box chapter to one iteration; open issues captured for later}
\setUserStory{As a Security Architect, I want to study and practice 'Future Trends For Cyber Attacks in the Health Care Industry' so that I can apply its concepts to reduce risk and improve outcomes.}
\setNonFunctional{Security, Reliability, Performance}
\setScenario{Apply key controls for Future Trends For Cyber Attacks in the Health Care Industry}
\setGiven{the lab environment and topic-specific tools for 'Future Trends For Cyber Attacks in the Health Care Industry' are available}
\setWhen{I execute the hands-on objectives and lab: Hands-on: Build a small lab demonstrating key concepts in 'Future Trends For Cyber Attacks in the Health Care Industry'. Capture screenshots/notes and one measurable result (e.g., a passing test, alert fired, or control verified).}
\setThen{the deliverables are produced (2–3 page brief on 'Future Trends For Cyber Attacks in the Health Care Industry': risks, architecture, controls, and a checklist; plus a one-slide executive summary.); evidence (screenshots/logs/configs) is attached and reviewed}
\setDoR{Persona clear; AC drafted; Dependencies known; Estimate set.}
\setDoD{All ACs pass; Tests green; Security checks; Docs updated; Evidence attached.}
\RenderStoryCard
\begin{TasksBox}
\begin{itemize}
\TaskItem{Draft a one-page chapter plan: scope, objectives, interfaces, success metrics.}
\TaskItem{Set up tools, datasets, and accounts; document versions and configuration.}
\TaskItem{Complete objective: Define key terms and articulate why this topic matters to security outcomes.}
\TaskItem{Complete objective: Diagram the architecture/data flows and identify threat surfaces.}
\TaskItem{Execute lab: Hands-on: Build a small lab demonstrating key concepts in 'Future Trends For Cyber Attacks in the Health Care Industry'. Capture screenshots/notes and one measurable result (e.g., a passing test, alert fired, or control verified).}
\end{itemize}
\end{TasksBox}
\clearpage
\setStoryID{CISH-114}
\setStoryTitle{Future Trends For Cyber Defense Of Offshore Drilling Rigs --- Learn \& Lab}
\setEpic{Part 17: Future Cyber Security Trends And Directions}
\setBusinessValue{Build a working understanding of future trends for cyber defense of offshore drilling rigs and its place in a modern security program; be able to explain core concepts, map them to the CIA triad, and identify common threats and controls.}
\setPriority{Must}
\setSP{3}
\setPersona{Security Architect}
\setDependencies{Lab VM or container runtime, Git repo for notes, Markdown/PDF export tool}
\setAssumptions{Time-box chapter to one iteration; open issues captured for later}
\setUserStory{As a Security Architect, I want to study and practice 'Future Trends For Cyber Defense Of Offshore Drilling Rigs' so that I can apply its concepts to reduce risk and improve outcomes.}
\setNonFunctional{Security, Reliability, Performance}
\setScenario{Apply key controls for Future Trends For Cyber Defense Of Offshore Drilling Rigs}
\setGiven{the lab environment and topic-specific tools for 'Future Trends For Cyber Defense Of Offshore Drilling Rigs' are available}
\setWhen{I execute the hands-on objectives and lab: Hands-on: Build a small lab demonstrating key concepts in 'Future Trends For Cyber Defense Of Offshore Drilling Rigs'. Capture screenshots/notes and one measurable result (e.g., a passing test, alert fired, or control verified).}
\setThen{the deliverables are produced (2–3 page brief on 'Future Trends For Cyber Defense Of Offshore Drilling Rigs': risks, architecture, controls, and a checklist; plus a one-slide executive summary.); evidence (screenshots/logs/configs) is attached and reviewed}
\setDoR{Persona clear; AC drafted; Dependencies known; Estimate set.}
\setDoD{All ACs pass; Tests green; Security checks; Docs updated; Evidence attached.}
\RenderStoryCard
\begin{TasksBox}
\begin{itemize}
\TaskItem{Draft a one-page chapter plan: scope, objectives, interfaces, success metrics.}
\TaskItem{Set up tools, datasets, and accounts; document versions and configuration.}
\TaskItem{Complete objective: Define key terms and articulate why this topic matters to security outcomes.}
\TaskItem{Complete objective: Diagram the architecture/data flows and identify threat surfaces.}
\TaskItem{Execute lab: Hands-on: Build a small lab demonstrating key concepts in 'Future Trends For Cyber Defense Of Offshore Drilling Rigs'. Capture screenshots/notes and one measurable result (e.g., a passing test, alert fired, or control verified).}
\end{itemize}
\end{TasksBox}
\clearpage
\setStoryID{CISH-115}
\setStoryTitle{Future Trends In Maritime Cyber Security --- Learn \& Lab}
\setEpic{Part 17: Future Cyber Security Trends And Directions}
\setBusinessValue{Build a working understanding of future trends in maritime cyber security and its place in a modern security program; be able to explain core concepts, map them to the CIA triad, and identify common threats and controls.}
\setPriority{Must}
\setSP{3}
\setPersona{Security Architect}
\setDependencies{Lab VM or container runtime, Git repo for notes, Markdown/PDF export tool}
\setAssumptions{Time-box chapter to one iteration; open issues captured for later}
\setUserStory{As a Security Architect, I want to study and practice 'Future Trends In Maritime Cyber Security' so that I can apply its concepts to reduce risk and improve outcomes.}
\setNonFunctional{Security, Reliability, Performance}
\setScenario{Apply key controls for Future Trends In Maritime Cyber Security}
\setGiven{the lab environment and topic-specific tools for 'Future Trends In Maritime Cyber Security' are available}
\setWhen{I execute the hands-on objectives and lab: Hands-on: Build a small lab demonstrating key concepts in 'Future Trends In Maritime Cyber Security'. Capture screenshots/notes and one measurable result (e.g., a passing test, alert fired, or control verified).}
\setThen{the deliverables are produced (2–3 page brief on 'Future Trends In Maritime Cyber Security': risks, architecture, controls, and a checklist; plus a one-slide executive summary.); evidence (screenshots/logs/configs) is attached and reviewed}
\setDoR{Persona clear; AC drafted; Dependencies known; Estimate set.}
\setDoD{All ACs pass; Tests green; Security checks; Docs updated; Evidence attached.}
\RenderStoryCard
\begin{TasksBox}
\begin{itemize}
\TaskItem{Draft a one-page chapter plan: scope, objectives, interfaces, success metrics.}
\TaskItem{Set up tools, datasets, and accounts; document versions and configuration.}
\TaskItem{Complete objective: Define key terms and articulate why this topic matters to security outcomes.}
\TaskItem{Complete objective: Diagram the architecture/data flows and identify threat surfaces.}
\TaskItem{Execute lab: Hands-on: Build a small lab demonstrating key concepts in 'Future Trends In Maritime Cyber Security'. Capture screenshots/notes and one measurable result (e.g., a passing test, alert fired, or control verified).}
\end{itemize}
\end{TasksBox}
\clearpage
\section*{Appendix: Attached Definitions/Templates (optional)}
\IfFileExists{StoryCardDefinition.tex}{\subsection*{StoryCardDefinition.tex}\VerbatimInput[fontsize=\small]{StoryCardDefinition.tex}}{\subsection*{StoryCardDefinition.tex not found}}
\IfFileExists{UserStoryTemplate.tex}{\subsection*{UserStoryTemplate.tex}\VerbatimInput[fontsize=\small]{UserStoryTemplate.tex}}{\subsection*{UserStoryTemplate.tex not found}}
\end{document}
