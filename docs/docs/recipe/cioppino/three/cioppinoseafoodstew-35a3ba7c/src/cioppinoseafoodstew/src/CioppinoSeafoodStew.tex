\documentclass[11pt]{article}

% ---------- Common, stable packages ----------
\usepackage[margin=1in]{geometry}
\usepackage[T1]{fontenc}
\usepackage[utf8]{inputenc}
\usepackage{lmodern}
\usepackage{microtype}
\usepackage{hyperref}
\usepackage{booktabs}
\usepackage{array}
\usepackage{tabularx}
\usepackage{enumitem}
\usepackage{amsmath}

\hypersetup{
  colorlinks=true,
  linkcolor=black,
  urlcolor=blue
}

% ---------- Helpers ----------
\setlist[itemize]{itemsep=2pt, topsep=4pt}
\setlist[enumerate]{itemsep=4pt, topsep=6pt}
\newcommand{\Section}[1]{\vspace{0.6em}\noindent\textbf{\Large #1}\par\vspace{0.25em}}
\newcommand{\Subsection}[1]{\vspace{0.4em}\noindent\textbf{\large #1}\par\vspace{0.2em}}
\newcolumntype{L}{>{\raggedright\arraybackslash}X}

\begin{document}

\begin{center}
  {\LARGE \textbf{Cioppino Recipe (Seafood Stew)}}\\[4pt]
  \small By Sara May \quad\textbullet\quad Updated: Dec 14, 2022 \quad\textbullet\quad Rating: 4.95/5 (40 votes)
\end{center}

\Section{Overview}
Cioppino (pronounced \emph{chuh-pee-no}) is a San Francisco--born fisherman's stew of clams, mussels, white fish, and shrimp in a tomato--fennel, wine-kissed broth. This version feeds a crowd and includes an optional parsley--olive gremolata for a bright finish. Serve with grilled sourdough for dunking.

\Section{At a Glance}
\begin{tabularx}{\textwidth}{@{} l L @{}}
\toprule
\textbf{Cuisine} & American / Mediterranean \\
\textbf{Course} & Entree or Side Dish; Seafood; Soup \\
\textbf{Yield} & Serves 10 \\
\textbf{Prep Time} & 50 minutes \\
\textbf{Cook Time} & 40 minutes \\
\textbf{Total Time} & 1 hour 30 minutes \\
\bottomrule
\end{tabularx}

\Section{Ingredients}

\Subsection{Parsley--Olive Gremolata (Optional)}
\begin{itemize}
  \item $\tfrac{1}{2}$ bunch parsley (about 2 oz), roughly chopped
  \item Zest of 1 orange
  \item $\tfrac{1}{2}$ cup mixed olives, pitted
  \item 1 garlic clove, peeled and roughly chopped
  \item $\tfrac{1}{2}$ teaspoon kosher salt
  \item $\tfrac{1}{2}$ teaspoon red pepper flakes (optional)
  \item 2 tablespoons extra-virgin olive oil
\end{itemize}

\Subsection{Cioppino}
\begin{itemize}
  \item 1 lb clams, scrubbed well
  \item 1 lb mussels, debearded and scrubbed well
  \item $\tfrac{1}{4}$ cup extra-virgin olive oil
  \item 1 large onion, $\tfrac{1}{2}$-inch dice (about 2 cups)
  \item 1 large fennel bulb, $\tfrac{1}{2}$-inch dice (about 2 cups)
  \item Kosher salt, to taste
  \item 2 garlic cloves, finely minced
  \item 1 (12 oz) jar roasted red peppers, drained and roughly chopped
  \item 2 teaspoons dried oregano
  \item 2 teaspoons dried thyme
  \item 1\,$\tfrac{1}{4}$ cups dry white wine
  \item 1 (28 oz) can whole tomatoes, crushed by hand with juices
  \item 1\,$\tfrac{1}{4}$ cups seafood stock\footnote{If needed, substitute bottled clam juice or low-sodium chicken stock.}
  \item 1 lb skinless firm white fish (halibut or cod), cut into 1-inch pieces
  \item 1 lb large shrimp, peeled and deveined
  \item Grilled sourdough bread, for serving (optional)
\end{itemize}

\Section{Equipment}
Large stockpot with steamer basket; large Dutch oven or 8-quart stockpot; wooden spoon; ladle.

\Section{Instructions}

\Subsection{Make the Parsley--Olive Gremolata}
\begin{enumerate}
  \item \textbf{Process.} In a food processor, pulse parsley, orange zest, olives, garlic, salt, and red pepper flakes until a chunky paste forms. Scrape down the bowl, add olive oil, and pulse to a uniformly chunky consistency.
  \item \textbf{Rest.} Let stand at room temperature while you prepare the cioppino so the flavors bloom.
\end{enumerate}

\Subsection{Make the Cioppino}
\begin{enumerate}
  \item \textbf{Steam shellfish.} Bring about 2 cups water to a boil in a stockpot fitted with a steamer basket. Add mussels and clams; cover and steam 5--8 minutes until just opened. Transfer to a bowl (discard any that do not open). \textit{Reserve the steaming liquid} (you will use about 1 cup now; save extra to adjust consistency if needed).
  \item \textbf{Sauté aromatics.} In a Dutch oven over medium heat, warm the olive oil until shimmering. Add onion, fennel, and a generous pinch of salt; sauté 8--10 minutes until translucent. Add garlic, roasted red peppers, oregano, and thyme; cook 3--4 minutes until very fragrant and most moisture has evaporated.
  \item \textbf{Deglaze.} Add white wine; scrape up any browned bits. Bring to a simmer and cook 5 minutes.
  \item \textbf{Build the base.} Add hand-crushed tomatoes (with juices), seafood stock, and about 1 cup reserved steaming liquid. Stir, bring to a boil, then reduce to a lively simmer. Cook uncovered 20 minutes. Taste and season with salt as needed.
  \item \textbf{Cook fish and shrimp.} Add the white fish and shrimp. Return to a simmer, cover, and cook 2--3 minutes, until fish is opaque and shrimp are pink and curled.
  \item \textbf{Finish.} Remove from heat; gently stir in the cooked mussels and clams. Taste and adjust salt. If desired, thin with a splash more reserved steaming liquid.
  \item \textbf{Serve.} Ladle into warm bowls. Top each serving with a spoonful of parsley--olive gremolata. Serve with grilled sourdough.
\end{enumerate}

\Section{How-To Guides}

\Subsection{Peel \& Devein Shrimp}
\begin{itemize}
  \item Have a bowl of cold water ready. With kitchen shears, cut along the back to the tail; peel off shell, legs, and tail.
  \item Use the tip of a paring knife to lift out the vein; swish shrimp in cold water. Repeat. Save shells for stock.
\end{itemize}

\Subsection{Debeard Mussels}
\begin{itemize}
  \item Scrub shells clean. If a fibrous ``beard'' is present, grasp it firmly and pull toward the hinge to remove (use a paper towel for grip if needed).
\end{itemize}

\Section{Wine Notes \& Variations}
\begin{itemize}
  \item \textbf{Wine choice:} Use a dry, unoaked white (pinot grigio, sauvignon blanc, or chardonnay). Avoid ``cooking wine.'' For a heartier profile, red wine may be used.
  \item \textbf{No alcohol:} Substitute low-sodium chicken stock for the wine.
  \item \textbf{Seafood swaps:} Add or substitute scallops, cockles, crab, or lobster. Avoid very oily fish (e.g., salmon, trout, tuna) in this style of stew.
\end{itemize}

\Section{Serving \& Keeping}
\begin{itemize}
  \item \textbf{Serve with:} Grilled or toasted sourdough; set small forks for shellfish and a bowl for shells.
  \item \textbf{Leftovers:} Refrigerate up to 3 days. Reheat gently; do not boil.
\end{itemize}

\Section{Nutrition (Approximate, per serving; 10 servings)}
\begin{tabular}{@{}ll@{}}
\toprule
Calories & 243.4 kcal \\
Carbohydrates & 11 g \\
Protein & 20.3 g \\
Fat & 11.1 g \\
\quad Saturated Fat & 1.6 g \\
Polyunsaturated Fat & 1.5 g \\
Monounsaturated Fat & 7.1 g \\
Trans Fat & 0.01 g \\
Cholesterol & 85.2 mg \\
Sodium & 1268.4 mg \\
Potassium & 721.7 mg \\
Fiber & 2.8 g \\
Sugar & 3.7 g \\
Vitamin A & 1008.1 IU \\
Vitamin C & 37.1 mg \\
Calcium & 128.4 mg \\
Iron & 3.5 mg \\
\bottomrule
\end{tabular}

\vfill
\begin{center}
  \footnotesize
  \textit{Tip: Reserve extra shellfish steaming liquid to fine-tune broth consistency just before serving.}
\end{center}

\end{document}

