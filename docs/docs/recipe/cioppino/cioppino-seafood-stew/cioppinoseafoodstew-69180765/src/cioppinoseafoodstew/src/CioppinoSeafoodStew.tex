\documentclass[11pt]{article}

% ---------- Common, stable packages ----------
\usepackage[margin=1in]{geometry}
\usepackage[T1]{fontenc}
\usepackage[utf8]{inputenc}
\usepackage{lmodern}
\usepackage{microtype}
\usepackage{hyperref}
\usepackage{booktabs}
\usepackage{array}
\usepackage{tabularx}
\usepackage{enumitem}
\usepackage{amsmath}
\usepackage{setspace}

\hypersetup{
  colorlinks=true,
  linkcolor=black,
  urlcolor=blue
}

% ---------- Helpers ----------
\setlist[itemize]{itemsep=2pt, topsep=4pt}
\setlist[enumerate]{itemsep=4pt, topsep=6pt}
\newcommand{\Section}[1]{\vspace{0.6em}\noindent\textbf{\Large #1}\par\vspace{0.25em}}
\newcommand{\Subsection}[1]{\vspace{0.4em}\noindent\textbf{\large #1}\par\vspace{0.2em}}
\newcolumntype{L}{>{\raggedright\arraybackslash}X}

\begin{document}

\begin{center}
  {\LARGE \textbf{Cioppino Seafood Stew}}\\[4pt]
  \small by Heidi Larsen \quad\textbullet\quad Updated: Jul 14, 2025
\end{center}

\Section{Overview}
A tomato-based fisherman’s stew loaded with fresh seafood in a savory wine broth. Inspired by Ina Garten’s \emph{Cook Like a Pro}, this cioppino is weeknight-easy yet special-occasion worthy. Serve with crusty sourdough for dunking.

\Section{At a Glance}
\begin{tabularx}{\textwidth}{@{} l L @{}}
\toprule
\textbf{Cuisine} & Italian--American (San Francisco origin) \\
\textbf{Yield} & Serves 6 \\
\textbf{Prep Time} & 30 minutes \\
\textbf{Cook Time} & 40 minutes \\
\textbf{Total Time} & 1 hour 10 minutes \\
\textbf{Serve With} & Sourdough, garlic bread, or a simple green salad \\
\bottomrule
\end{tabularx}

\Section{Ingredients}
\Subsection{Broth Base}
\begin{itemize}
  \item $\tfrac{1}{4}$ cup olive oil
  \item 2 cups fennel bulb (white only), cut into $\tfrac{1}{2}$-inch dice
  \item 1\,$\tfrac{1}{2}$ cups yellow onion (about 1 large), $\tfrac{1}{2}$-inch dice
  \item 3 garlic cloves, minced or pressed
  \item 1 teaspoon whole fennel seeds
  \item $\tfrac{1}{2}$ teaspoon red pepper flakes
  \item 1 (28-ounce) can crushed tomatoes
  \item 4 cups seafood stock\footnote{If unavailable, use clam juice or a mix of fish stock and water.}
  \item 1\,$\tfrac{1}{2}$ cups dry white wine (e.g., Pinot Grigio)
  \item Kosher salt and freshly ground black pepper (about 1 tablespoon salt \& 1 teaspoon pepper for the pot)
\end{itemize}

\Subsection{Seafood}
\begin{itemize}
  \item 1 lb cod fillets, skin removed, cut into 2-inch pieces
  \item 1 lb large shrimp, peeled and deveined
  \item 1 lb sea scallops, halved crosswise if large
  \item 1 dozen mussels, scrubbed and debearded
  \item 1 dozen littleneck clams (or other small clams), scrubbed
\end{itemize}

\Subsection{Finish \& Garnish}
\begin{itemize}
  \item 1 tablespoon anise-flavored liqueur (Pernod, Pastis, ouzo, or sambuca)
  \item 3 tablespoons fresh flat-leaf parsley, minced
  \item Sliced sourdough baguette, for serving
\end{itemize}

\Section{Instructions}
\begin{enumerate}
  \item \textbf{Sauté aromatics.} Heat olive oil in a heavy pot or Dutch oven over medium heat. Add fennel and onion; cook, stirring, about 10 minutes until tender.
  \item \textbf{Bloom spices.} Stir in garlic, fennel seeds, and red pepper flakes; cook 2 minutes until fragrant.
  \item \textbf{Build the broth.} Add crushed tomatoes, seafood stock, wine, 1 tablespoon kosher salt, and 1 teaspoon black pepper. Bring to a boil, then reduce heat and simmer uncovered 30 minutes.
  \item \textbf{Add seafood in order.} Add seafood gently in layers: first cod, then shrimp, then scallops, and finally mussels and clams. Do not stir. Bring just to a simmer, reduce heat, cover, and cook about 10 minutes, until seafood is opaque and shellfish open.
  \item \textbf{Finish.} Stir in the anise liqueur carefully (avoid breaking the fish). Cover and rest 3 minutes so flavors meld. Discard any unopened mussels or clams.
  \item \textbf{Serve.} Ladle into warm shallow bowls, sprinkle with parsley, and serve with crusty sourdough.
\end{enumerate}

\Section{Notes \& Tips}
\begin{itemize}
  \item \textbf{Shellfish prep:} Scrub shells; soak mussels 30 minutes in cold water with a few tablespoons flour to help purge grit, then rinse.
  \item \textbf{Shrimp:} For easier eating, peel and remove tails before cooking.
  \item \textbf{Make-ahead:} Prepare broth base up to 2 days ahead; refrigerate. Reheat, then add seafood just before serving.
  \item \textbf{Wine choice:} Use a dry white you would drink (e.g., Pinot Grigio, Sauvignon Blanc). Avoid “cooking wine.”
  \item \textbf{Liqueur swap:} Pernod is classic; Pastis, ouzo, or sambuca work too. A little goes a long way.
\end{itemize}

\Section{What to Serve}
Garlic bread; Caesar or chopped Italian salad; simple pasta (e.g., cacio e pepe); kale salad with Parmesan and pine nuts.

\Section{Nutrition (Approximate, per serving; 6 servings)}
\begin{tabular}{@{}ll@{}}
\toprule
Calories & 373 kcal \\
Carbohydrates & 11 g \\
Protein & 42 g \\
Fat & 12 g \\
\quad Saturated Fat & 1 g \\
Cholesterol & 241 mg \\
Sodium & 1467 mg \\
Potassium & 905 mg \\
Fiber & 1 g \\
Sugar & 2 g \\
Vitamin A & 285 IU \\
Vitamin C & 13.4 mg \\
Calcium & 209 mg \\
Iron & 3.2 mg \\
\bottomrule
\end{tabular}

\vfill
\begin{center}
  \footnotesize
  \textit{Credit: Adapted from Ina Garten, \emph{Cook Like a Pro}. For a French cousin, see bouillabaisse (fish-stock base with saffron).}
\end{center}

\end{document}
