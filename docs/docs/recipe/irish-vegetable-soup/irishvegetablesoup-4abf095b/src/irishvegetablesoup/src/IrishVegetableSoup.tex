\documentclass[11pt]{article}

% ---------- Common, stable packages ----------
\usepackage[margin=1in]{geometry}
\usepackage[T1]{fontenc}
\usepackage[utf8]{inputenc}
\usepackage{lmodern}
\usepackage{microtype}
\usepackage{hyperref}
\usepackage{booktabs}
\usepackage{array}
\usepackage{tabularx}
\usepackage{enumitem}
\usepackage{amsmath}

\hypersetup{
  colorlinks=true,
  linkcolor=black,
  urlcolor=blue
}

% ---------- Helpers ----------
\setlist[itemize]{itemsep=2pt, topsep=4pt}
\setlist[enumerate]{itemsep=4pt, topsep=6pt}
\newcommand{\Section}[1]{\vspace{0.6em}\noindent\textbf{\Large #1}\par\vspace{0.25em}}
\newcommand{\Subsection}[1]{\vspace{0.4em}\noindent\textbf{\large #1}\par\vspace{0.2em}}
\newcolumntype{L}{>{\raggedright\arraybackslash}X}

\begin{document}

\begin{center}
  {\LARGE \textbf{Irish Vegetable Soup}}\\[4pt]
  \small By Michelle Alston \quad\textbullet\quad Published: March 7, 2023 (Modified: March 7, 2023)
\end{center}

\Section{Overview}
A creamy, comforting Irish farmhouse-style vegetable soup made with simple, seasonal veg. Partially blending the pot yields a silky texture while keeping hearty chunks. Serve with warm Irish soda bread and butter.

\Section{At a Glance}
\begin{tabularx}{\textwidth}{@{} l L @{}}
\toprule
\textbf{Yield} & 6 servings \\
\textbf{Prep Time} & 10 minutes \\
\textbf{Cook Time} & 45 minutes \\
\textbf{Total Time} & 55 minutes \\
\textbf{Cuisine / Diet} & Irish \; / \; Vegetarian \\
\bottomrule
\end{tabularx}

\Section{Ingredients}

\Subsection{Base \& Vegetables}
\begin{itemize}
  \item 1 tablespoon olive oil
  \item 1 medium onion, chopped
  \item 1 stalk celery, finely chopped
  \item 2 medium carrots, peeled and roughly chopped
  \item 1 medium leek, outer leaves removed, cleaned, sliced
  \item 1 clove garlic, finely minced (or pressed)
  \item 2 medium parsnips, peeled and roughly chopped
  \item 3 medium floury potatoes, peeled and cut into large chunks \\
        \hspace*{1.5em}\emph{(e.g., Maris Piper)}
\end{itemize}

\Subsection{Stock \& Herbs}
\begin{itemize}
  \item 1 litre / 4 cups vegetable stock \emph{(made with 2 vegetable stock cubes)}%
  \footnote{Vegetable broth may be used instead.}
  \item 3--4 sprigs fresh thyme
  \item 2 dried bay leaves
\end{itemize}

\Subsection{Finish \& Seasoning}
\begin{itemize}
  \item 100 g / $\tfrac{3}{4}$ cup frozen garden peas
  \item $\tfrac{1}{4}$ teaspoon freshly ground black pepper
  \item $\tfrac{1}{2}$ teaspoon sea salt \emph{(plus more to taste)}
  \item 150 ml / $\tfrac{3}{4}$ cup double cream (heavy cream)
  \item Fresh parsley, chopped, for garnish (optional)
\end{itemize}

\Section{Instructions}
\begin{enumerate}
  \item \textbf{Sweat the aromatics.} Heat olive oil in a large stockpot or Dutch oven over medium heat. Add onion, celery, leek, and carrot; cook, stirring occasionally, until the onion is soft, about 20 minutes. Add garlic; cook 1 minute.
  \item \textbf{Simmer the roots.} Stir in parsnips, potatoes, and stock. Add bay leaves and thyme. Season with salt and pepper. Bring to a boil, then reduce to a low simmer. Cook about 15 minutes, or until potatoes are cooked through.
  \item \textbf{Blend partially.} Remove bay leaves and thyme sprigs. Ladle about 2 ladles of soup into a blender; blend until smooth, then return to the pot and stir well to combine.
  \item \textbf{Finish \& serve.} Add frozen peas; cook a few minutes until tender. Stir in the cream and cook another couple of minutes until piping hot. Adjust seasoning. Serve immediately with bread and parsley garnish if you like.
\end{enumerate}

\Section{Top Tips}
\begin{itemize}
  \item \textbf{Build flavor slowly:} A gentle 20-minute sweat of the aromatics deepens flavor and yields a softer onion.
  \item \textbf{Choose floury potatoes:} They naturally thicken the soup and blend silky-smooth.
  \item \textbf{Blend part, not all:} Blending a portion gives creaminess while keeping a rustic, chunky texture.
\end{itemize}

\Section{Variations}
\begin{itemize}
  \item \textbf{More veg:} Turnip/swede works well; add chopped kale near the end.
  \item \textbf{Barley:} Stir in cooked pearl barley to make it extra hearty.
  \item \textbf{Fully blended:} For a smooth soup, use an immersion blender to pur\'ee to your preferred consistency.
\end{itemize}

\Section{FAQs}
\textbf{Can I freeze this soup?} Due to the cream, freezing is not recommended as the texture may split when reheated. \\
\textbf{Can I reheat this?} Yes. Reheat gently over low heat until hot (avoid boiling). \\
\textbf{Can I make this dairy-free?} Yes. Omit the cream or substitute a splash of unsweetened plant cream; the partial blend still gives a creamy feel.

\Section{Storage}
\begin{itemize}
  \item \textbf{Refrigerate:} Cool completely; store airtight up to 2 days.
  \item \textbf{Reheat:} Warm gently on the stovetop; do not boil once cream is added.
\end{itemize}

\Section{Nutrition (Approximate, per serving; 6 servings)}
\begin{tabular}{@{}ll@{}}
\toprule
Calories & 242 kcal \\
Carbohydrates & 30.6 g \\
Protein & 3.9 g \\
Fat & 13.5 g \\
\quad Saturated Fat & 7.5 g \\
Cholesterol & 35 mg \\
Sodium & 668 mg \\
Potassium & 498 mg \\
Fiber & 4.3 g \\
Sugar & 5.4 g \\
Calcium & 55 mg \\
Iron & 1 mg \\
\bottomrule
\end{tabular}

\vfill
\begin{center}
  \footnotesize
  \textit{Note on measurements: For best accuracy, weigh ingredients. Cup/spoon volumes vary by brand.}
\end{center}

\end{document}

