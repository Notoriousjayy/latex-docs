\documentclass[11pt]{article}

% ---------- Common, stable packages ----------
\usepackage[margin=1in]{geometry}
\usepackage[T1]{fontenc}
\usepackage[utf8]{inputenc}
\usepackage{lmodern}
\usepackage{microtype}
\usepackage{hyperref}
\usepackage{booktabs}
\usepackage{array}
\usepackage{tabularx}
\usepackage{enumitem}
\usepackage{amsmath} % for \tfrac

\hypersetup{
  colorlinks=true,
  linkcolor=black,
  urlcolor=blue
}

% ---------- Helpers ----------
\setlist[itemize]{itemsep=2pt, topsep=4pt}
\setlist[enumerate]{itemsep=4pt, topsep=6pt}
\newcommand{\Section}[1]{\vspace{0.6em}\noindent\textbf{\Large #1}\par\vspace{0.25em}}
\newcommand{\Subsection}[1]{\vspace{0.4em}\noindent\textbf{\large #1}\par\vspace{0.2em}}
\newcolumntype{L}{>{\raggedright\arraybackslash}X}

\begin{document}

\begin{center}
  {\LARGE \textbf{Chicken Parmesan}}\\[4pt]
  \small Crisp, golden chicken cutlets topped (not drowned) with sauce, basil, and a trio of cheeses.
\end{center}

\Section{Overview}
This approach keeps the breading shatter-crisp by using only a \emph{little} sauce on top (none underneath) and baking hot to finish. Parmesan is mixed into the panko for extra flavor, and the topping blends fresh mozzarella, provolone, and more Parmesan.

\Section{At a Glance}
\begin{tabularx}{\textwidth}{@{} l L @{}}
\toprule
\textbf{Yield} & 4 servings \\
\textbf{Prep Time} & 15 minutes \\
\textbf{Cook Time} & 20 minutes \\
\textbf{Additional} & 10 minutes (resting the breaded cutlets) \\
\textbf{Total Time} & 45 minutes \\
\textbf{Method} & Pound cutlets; flour $\rightarrow$ egg $\rightarrow$ panko/Parmesan; shallow-fry; top; bake hot (450\,$^\circ$F / 230\,$^\circ$C) \\
\bottomrule
\end{tabularx}

\Section{Ingredients}

\Subsection{Chicken \& Breading}
\begin{itemize}
  \item 4 skinless, boneless chicken breast halves
  \item Kosher salt and freshly ground black pepper
  \item 2 tablespoons all-purpose flour (more as needed, for dusting)
  \item 2 large eggs, beaten
  \item 1 cup panko breadcrumbs (more as needed)
  \item $\tfrac{1}{2}$ cup grated Parmesan cheese (mixed into crumbs)
\end{itemize}

\Subsection{For Frying}
\begin{itemize}
  \item About $\tfrac{1}{2}$ cup olive oil (or as needed for a $\sim\,\tfrac{1}{2}$-inch layer)
\end{itemize}

\Subsection{To Finish}
\begin{itemize}
  \item $\tfrac{1}{2}$ cup prepared tomato sauce (plus extra warmed on the side, optional)
  \item $\tfrac{1}{4}$ cup fresh mozzarella, cut into small cubes
  \item $\tfrac{1}{4}$ cup chopped fresh basil
  \item $\tfrac{1}{2}$ cup grated provolone cheese
  \item $\tfrac{1}{4}$ cup grated Parmesan cheese (remaining, for topping)
  \item 2 teaspoons olive oil (for final drizzle)
\end{itemize}

\Section{Equipment}
Meat mallet; two sheets of heavy plastic or parchment; shallow bowls; fine strainer or sifter (optional, for flouring); large skillet; 9\,$\times$\,13-inch baking dish; instant-read thermometer; rack or paper towels.

\Section{Instructions}
\begin{enumerate}
  \item \textbf{Preheat.} Heat oven to 450\,$^\circ$F (230\,$^\circ$C). Set a rack in the upper-middle position.
  \item \textbf{Pound cutlets.} Place each breast between sheets of plastic or parchment. Pound to an even $\tfrac{1}{2}$-inch thickness.
  \item \textbf{Season \& flour.} Season both sides generously with salt and pepper. Using a strainer (optional), dust evenly with flour on both sides; shake off excess.
  \item \textbf{Bread.} Place beaten eggs in one shallow bowl. In another, mix panko with $\tfrac{1}{2}$ cup Parmesan. Dip each floured cutlet in egg, then press into the crumb mixture to coat well. Set on a tray and let rest 10--15 minutes (helps coating adhere).
  \item \textbf{Shallow-fry.} Heat about $\tfrac{1}{2}$ inch of olive oil in a large skillet over medium-high until shimmering. Fry cutlets until golden, about 2 minutes per side (they'll finish in the oven). Transfer to a rack or paper towels to drain briefly.
  \item \textbf{Top (lightly).} Arrange cutlets in a baking dish. Spoon \emph{2 tablespoons} sauce over each (no sauce underneath). Divide mozzarella, basil, and provolone evenly over tops. Sprinkle with the remaining $\tfrac{1}{4}$ cup Parmesan and drizzle each with about $\tfrac{1}{2}$ teaspoon olive oil.
  \item \textbf{Bake hot.} Bake 15--20 minutes until cheese is browned and bubbly and the chicken reaches 165\,$^\circ$F (74\,$^\circ$C) in the center. Rest 3--5 minutes.
  \item \textbf{Serve.} Garnish with extra basil if you like. Offer additional warmed sauce on the side to preserve the cutlets' crispness.
\end{enumerate}

\Section{Chef's Notes \& Tips}
\begin{itemize}
  \item \textbf{Even thickness} ensures even cooking---pound the thick end to match the thin.
  \item \textbf{Season the chicken,} not the dredges; it sticks where you want it.
  \item \textbf{Parmesan in the crumbs} adds flavor and extra crunch.
  \item \textbf{Rest the breaded cutlets} 10--15 minutes before frying for better adhesion.
  \item \textbf{Go easy on sauce:} keep it on top only; serve more at the table if desired.
  \item \textbf{Hot oven, fast finish:} 450\,$^\circ$F browns cheese and re-crisps crumbs before overcooking.
  \item \textbf{Herb swaps:} substitute pesto or a pinch of dried Italian herbs if basil isn't handy.
  \item \textbf{Sauce quality matters:} use a good prepared tomato sauce or your favorite homemade.
\end{itemize}

\Section{Serving Suggestions}
Serve with spaghetti or another pasta, garlic bread, and a simple green salad. A crisp Italian white or light red pairs nicely.

\vfill
\begin{center}
  \footnotesize
  \textit{Food safety:} cook chicken to an internal temperature of 165\,$^\circ$F (74\,$^\circ$C).\\
  \textit{Make-ahead:} breaded cutlets can be refrigerated (unfried) up to 4 hours; let sit 10 minutes at room temp before frying.
\end{center}

\end{document}
