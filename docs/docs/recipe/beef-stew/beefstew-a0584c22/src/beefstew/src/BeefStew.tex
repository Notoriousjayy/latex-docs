\documentclass[11pt]{article}

% --- Page & typography ---
\usepackage[a4paper,margin=1in]{geometry}
\usepackage[T1]{fontenc}
\usepackage[utf8]{inputenc}
\usepackage{lmodern}
\usepackage{microtype}

% --- Math & formatting helpers ---
\usepackage{amsmath}
\usepackage{enumitem}
\usepackage{booktabs}
\usepackage{array}
\usepackage{xcolor}
\usepackage[hidelinks]{hyperref}

\setlist[itemize]{itemsep=2pt, topsep=4pt, leftmargin=1.2em}
\setlist[enumerate]{itemsep=4pt, topsep=6pt, leftmargin=1.4em}

\newcommand{\meta}[2]{\textbf{#1}\,\, #2}
\newcommand{\qty}[1]{\ensuremath{#1}} % numbers/quantities in math (for nice fractions)

\begin{document}

\begin{center}
  {\LARGE \textbf{Old-Fashioned Beef Stew}}\\[0.75em]
  {\small A classic, cozy, stick-to-your-ribs stew for chilly weekends.}
\end{center}

\vspace{0.75em}

% --- Meta info table ---
\renewcommand{\arraystretch}{1.2}
\noindent\begin{tabular}{@{}p{0.23\linewidth}p{0.27\linewidth}p{0.22\linewidth}p{0.28\linewidth}@{}}
\meta{Total Time:}{2 hours 45 minutes} &
\meta{Prep Time:}{15 minutes} &
\meta{Cook Time:}{2 hours 30 minutes} &
\meta{Rating:}{5/5 \,(26{,}468 reviews)} \\
\end{tabular}

\vspace{0.5em}
\noindent\meta{Yield:}{4 servings}

\bigskip

\section*{Ingredients}

\begin{itemize}
  \item \qty{\tfrac{1}{4}} cup all-purpose flour
  \item \qty{\tfrac{1}{4}} teaspoon freshly ground black pepper
  \item 1 pound beef stewing meat, trimmed and cut into 1-inch cubes
  \item 5 teaspoons vegetable oil (divided)
  \item 2 tablespoons red wine vinegar
  \item 1 cup dry red wine
  \item \qty{3\tfrac{1}{2}} cups beef broth (homemade or low-sodium canned), plus more as needed
  \item 2 bay leaves
  \item 1 medium onion, peeled and chopped
  \item 5 medium carrots, peeled and cut into \qty{\tfrac{1}{4}}-inch rounds
  \item 2 large baking potatoes, peeled and cut into \qty{\tfrac{3}{4}}-inch cubes
  \item 2 teaspoons kosher salt (or to taste)
\end{itemize}

\bigskip

\section*{Instructions}

\paragraph{Step 1.}
Combine the flour and pepper in a medium bowl. Add the beef and toss to coat well. Heat 3 teaspoons of the oil in a large heavy pot or Dutch oven over medium-high heat. Add the beef a few pieces at a time (do not overcrowd). Cook, turning occasionally, until browned on all sides, about 5 minutes per batch; add more oil as needed between batches. Transfer browned beef to a plate.

\paragraph{Step 2.}
Pour the vinegar and wine into the pot. Cook over medium-high heat, scraping up the browned bits with a wooden spoon. Return the beef (and accumulated juices) to the pot; add the beef broth and bay leaves. Bring to a boil, then reduce to a gentle simmer.

\paragraph{Step 3.}
Cover and cook, skimming the broth from time to time, until the beef is tender, about \qty{1\tfrac{1}{2}} hours. Add the onions and carrots; simmer, covered, for 10 minutes. Add the potatoes and continue to simmer, covered, until the vegetables are tender, about 30 minutes more. If the stew becomes dry, add additional broth or water. Season to taste with salt and freshly ground black pepper. Ladle into 4 bowls and serve hot.

\bigskip

\section*{Notes}
\begin{itemize}
  \item For a deeper glaze, be patient when browning the beef—good fond makes great flavor.
  \item The recipe scales well; consider doubling for meal prep or a crowd.
  \item Substitute additional broth for the wine if preferred.
\end{itemize}

\vfill
{\footnotesize \textit{Formatting notes:} Unicode fractions have been standardized to \LaTeX{} math fractions (e.g., \texttt{\textbackslash tfrac\{1\}\{4\}}), and “inch cubes” was clarified to “1-inch cubes” for consistency.}
\end{document}
