\documentclass[11pt]{article}

% ---------- Common, stable packages ----------
\usepackage[margin=1in]{geometry}
\usepackage[T1]{fontenc}
\usepackage[utf8]{inputenc}
\usepackage{lmodern}
\usepackage{microtype}
\usepackage{hyperref}
\usepackage{booktabs}
\usepackage{array}
\usepackage{tabularx}
\usepackage{enumitem}
\usepackage{amsmath}

\hypersetup{
  colorlinks=true,
  linkcolor=black,
  urlcolor=blue
}

% ---------- Helpers ----------
\setlist[itemize]{itemsep=2pt, topsep=4pt}
\setlist[enumerate]{itemsep=4pt, topsep=6pt}
\newcommand{\Section}[1]{\vspace{0.6em}\noindent\textbf{\Large #1}\par\vspace{0.25em}}
\newcommand{\Subsection}[1]{\vspace{0.4em}\noindent\textbf{\large #1}\par\vspace{0.2em}}
\newcolumntype{L}{>{\raggedright\arraybackslash}X}

\begin{document}

\begin{center}
  {\LARGE \textbf{Indian Chicken Curry (Murgh Kari)}}\\[4pt]
  \small Submitted by Ayshren \quad\textbullet\quad Updated: October 3, 2025 \quad\textbullet\quad Tested by Allrecipes Test Kitchen\\
  \small Rating: 4.8 (1{,}246) \;|\; 991 Reviews \;|\; 190 Photos
\end{center}

\Section{Overview}
A flavorful, weeknight-friendly chicken curry with a tomato--yogurt base and warm spices. Serve over basmati rice or with warm naan to soak up the sauce.

\Section{At a Glance}
\begin{tabularx}{\textwidth}{@{} l L @{}}
\toprule
\textbf{Yield} & 6 servings \\
\textbf{Prep Time} & 20 minutes \\
\textbf{Cook Time} & 40 minutes \\
\textbf{Total Time} & 1 hour \\
\textbf{Heat Level} & Adjustable (cayenne to taste) \\
\bottomrule
\end{tabularx}

\Section{Ingredients}
\Subsection{Chicken \& Aromatics}
\begin{itemize}
  \item 2 lb skinless, boneless chicken breast halves
  \item 2 teaspoons kosher salt (for seasoning chicken)
  \item 3 tablespoons neutral cooking oil, more as needed
  \item 1\,$\tfrac{1}{2}$ cups chopped onion
  \item 1 tablespoon minced garlic
  \item 1\,$\tfrac{1}{2}$ teaspoons minced fresh ginger root
\end{itemize}

\Subsection{Spices \& Base}
\begin{itemize}
  \item 1 tablespoon curry powder
  \item 1 teaspoon ground cumin
  \item 1 teaspoon ground turmeric
  \item 1 teaspoon ground coriander
  \item 1 teaspoon cayenne pepper
  \item 1 tablespoon water (to bloom spices)
  \item 1 (15 oz) can crushed tomatoes
  \item 1 cup plain yogurt
  \item 1 tablespoon chopped fresh cilantro
  \item 1 teaspoon kosher salt (for sauce)
  \item $\tfrac{1}{2}$ cup water
  \item 1 teaspoon garam masala
  \item 1 tablespoon chopped fresh cilantro (for finishing)
  \item 1 tablespoon fresh lemon juice
\end{itemize}

\Section{Equipment}
Large skillet (with lid), tongs, instant-read thermometer.

\Section{Instructions}
\begin{enumerate}
  \item \textbf{Season \& sear.} Sprinkle chicken with 2 teaspoons salt. Heat 1--2 tablespoons oil in a large skillet over high heat. Brown chicken in batches, adding oil as needed, until golden on all sides and partially cooked. Transfer to a plate.
  \item \textbf{Sauté aromatics.} Reduce heat to medium. In the remaining oil, cook onion, garlic, and ginger, stirring, until onion is soft and translucent, 5--8 minutes.
  \item \textbf{Bloom spices.} Stir in curry powder, cumin, turmeric, coriander, cayenne, and 1 tablespoon water; cook 1 minute, stirring, until fragrant.
  \item \textbf{Build the sauce.} Add crushed tomatoes, yogurt, 1 tablespoon cilantro, and 1 teaspoon salt; stir to combine.
  \item \textbf{Return chicken.} Add chicken and any accumulated juices. Pour in $\tfrac{1}{2}$ cup water; bring to a boil, turning chicken to coat. Sprinkle garam masala and 1 tablespoon cilantro over the chicken.
  \item \textbf{Simmer through.} Cover and simmer gently until chicken is no longer pink and juices run clear, about 20 minutes. An instant-read thermometer inserted into the center should read at least $165^\circ$F (74$^\circ$C).
  \item \textbf{Finish.} Drizzle with lemon juice. Adjust seasoning to taste.
  \item \textbf{Serve.} Spoon over basmati rice or serve with warm naan. Garnish with additional cilantro if desired.
\end{enumerate}

\Section{Cook's Notes}
\begin{itemize}
  \item \textbf{Chicken swap:} Boneless, skinless thighs work well in place of breasts (similar timing).
  \item \textbf{Dairy-free:} Unsweetened coconut milk may be used instead of yogurt.
  \item \textbf{Heat control:} Reduce cayenne for milder curry; add a pinch more for extra heat.
\end{itemize}

\Section{Nutrition (Approximate, per serving; 6 servings)}
\begin{tabular}{@{}ll@{}}
\toprule
Calories & 427 \\
Fat & 24 g \\
Carbohydrates & 15 g \\
Protein & 38 g \\
\bottomrule
\end{tabular}

\vfill
\begin{center}
  \footnotesize
  \textit{Food safety: Cook poultry to an internal temperature of $165^\circ$F (74$^\circ$C).}
\end{center}

\end{document}

