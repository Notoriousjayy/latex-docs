\documentclass[11pt]{article}

% --- Page + typography ---
\usepackage[margin=1in]{geometry}
\usepackage[T1]{fontenc}
\usepackage[utf8]{inputenc}
\usepackage{lmodern}
\usepackage{microtype}
\usepackage{setspace}
\setstretch{1.08}

% --- Color + links ---
\usepackage[dvipsnames]{xcolor}
\definecolor{CardFrame}{RGB}{33,37,41}
\definecolor{CardBack}{RGB}{249,250,252}
\definecolor{Accent}{RGB}{0,102,153}
\definecolor{Soft}{RGB}{120,120,120}
\usepackage{hyperref}
\hypersetup{colorlinks=true, linkcolor=MidnightBlue, urlcolor=MidnightBlue}

% --- Structure + tables + lists ---
\usepackage{enumitem}
\setlist[itemize]{itemsep=4pt, topsep=2pt, leftmargin=1.2em}
\setlist[enumerate]{itemsep=6pt, topsep=4pt, leftmargin=1.4em}
\usepackage{array,booktabs,tabularx}
\newcolumntype{Y}{>{\raggedright\arraybackslash}X}

% --- Nicely styled boxes ---
\usepackage[most]{tcolorbox}
\tcbset{
  sharp corners,
  boxrule=0.6pt,
  colframe=CardFrame,
  colback=CardBack
}

% --- Convenience ---
\newcommand{\F}[1]{#1\textdegree F}
\newcommand{\C}[1]{#1\textdegree C}

\begin{document}

{\Large\bfseries Baguette and French Bread with 100\% Freshly Milled Wheat}\par
\vspace{0.25em}
\textit{Two recipes in one—same dough, different shaping.}\par
\vspace{0.5em}
{\small \textcolor{Soft}{4.50 from 8 votes}}

\vspace{0.8em}

\begin{tcolorbox}
  \begin{tabularx}{\textwidth}{@{} l l l X @{}}
    \textbf{Prep} & \textbf{Bake} & \textbf{Yield} & \textbf{Notes} \\
    2 hr 30 min    & 25 min        & 3 baguettes \emph{or} 2 French loaves &
    100\% freshly milled hard white wheat. Kamut works; active dry yeast allowed (longer rise). \\
  \end{tabularx}
\end{tcolorbox}

\vspace{0.6em}

\begin{tcolorbox}[title=\textbf{Ingredients (1x batch)}]
\textbf{Flour} \\
\quad 5 cups freshly milled hard white wheat flour\footnotesize{} (mill about 3{\,}\(\tfrac{1}{2}\) cups wheat berries). \normalsize

\vspace{0.25em}
\textbf{Liquids \& Add-ins} 
\begin{itemize}
  \item 2 cups warm water
  \item 1{\,}\(\tfrac{1}{2}\) Tbsp cane sugar \;(\emph{or} honey)
  \item 1 Tbsp instant yeast \;(\emph{active dry} also works; allow longer rise)
  \item 1 Tbsp olive oil
  \item 2{\,}\(\tfrac{1}{2}\) tsp salt
\end{itemize}

\end{tcolorbox}

\vspace{0.4em}

\begin{tcolorbox}[title=\textbf{Approx. Metric Equivalents (guideline)}]
\begin{tabularx}{\textwidth}{@{} l l X @{}}
  \toprule
  \textbf{Item} & \textbf{Metric (approx.)} & \textbf{Notes} \\
  \midrule
  Warm water & \(\sim\)480\,mL & 2 US cups \\
  Sugar / honey & \(\sim\)22–25\,mL & 1.5 Tbsp (volume) \\
  Instant yeast & \(\sim\)9–10\,g & 1 Tbsp \\
  Olive oil & 15\,mL & 1 Tbsp \\
  Salt & \(\sim\)12–13\,mL & 2.5 tsp (volume; weight varies by type) \\
  Freshly milled flour & \(\sim\)625–700\,g & 5 cups; whole wheat density varies—mix to a \emph{tacky} dough \\
  \bottomrule
\end{tabularx}

{\footnotesize Flour weights are approximate; freshly milled whole wheat absorbs differently. Add flour gradually and stop when dough is soft and tacky.}
\end{tcolorbox}

\vspace{0.6em}

\begin{tcolorbox}[title=\textbf{Method}]
\begin{enumerate}
  \item \textbf{Build a sponge (15–20 min).} In a mixer bowl or large bowl, combine: warm water, sugar (or honey), and about \(\mathbf{2{\,}\tfrac{1}{2}}\) cups of the freshly milled flour. Stir smooth, then add yeast and mix in. Let stand until slightly risen and bubbly (15–20 minutes).\\
        \emph{No bubbles? Your yeast may be inactive—start over with fresh yeast.}

  \item \textbf{Mix the dough.} Add olive oil, salt, and \(\sim\)2 more cups flour. Mix to combine. Then add \emph{just enough} additional flour, a little at a time, until the dough begins to clean the bowl sides yet remains \emph{tacky} to the touch.

  \item \textbf{Knead to develop gluten.} Knead in mixer for \(\sim\)9 minutes (or by hand 15–20 minutes) until smooth, elastic, and still \emph{tacky}, not dry.\\
        \emph{Avoid over-flouring; too much flour yields a dense crumb.}

  \item \textbf{First rise (bulk ferment).} Place dough in a lightly oiled bowl (or leave in mixer bowl), cover, and rise until doubled, about 30–45 minutes.

  \item \textbf{Divide \& pre-shape.} Turn the still-tacky dough onto a lightly floured surface. For \textbf{baguettes}: divide into 3 equal pieces. For \textbf{French bread loaves}: divide into 2 equal pieces. Lightly flour or wet hands if needed.

  \item \textbf{Final shaping.} For each piece, roll into a rectangle about 10\,in wide and 6–7\,in tall. Fold the top third to the center and seal the seam. Fold again to the edge and seal. Roll into a cylinder: \textbf{baguette} = long and slender; \textbf{French bread} = slightly shorter and plumper.

  \item \textbf{Proof.} Place seam-side down on baguette/French bread pans, or on a floured tea towel (couche) with pleats between loaves. Cover and rise until nearly doubled (\(\sim\)30 minutes).

  \item \textbf{Preheat \& steam.} While loaves proof, preheat oven to \F{425}. Just before baking, score each loaf (diagonal slashes) and brush or mist with water (egg wash yields a darker, shinier crust).

  \item \textbf{Bake.} Bake at \F{425} for 10 minutes, then reduce to \F{375} and bake about 15 minutes more, or until an internal temperature of \F{190} is reached.

  \item \textbf{Cool \& serve.} Transfer to a rack and cool about 10 minutes. Tear apart or slice with a bread knife.
\end{enumerate}
\end{tcolorbox}

\vspace{0.6em}

\begin{tcolorbox}[title=\textbf{Shaping Guide \& Tips}]
\textbf{Baguette vs. French Bread:} Same dough—\emph{shape} sets the style. Baguettes are longer and slimmer; French loaves are a bit shorter and fuller.

\vspace{0.4em}
\textbf{Flour choice:} Hard white wheat gives the lightest result. If needed, use a blend of \(\tfrac{2}{3}\) hard red wheat + \(\tfrac{1}{3}\) soft white wheat.

\vspace{0.4em}
\textbf{Crust control:} For a crustier crust, extend the initial \F{425} phase by 1–2 minutes (ovens vary—experiment).

\vspace{0.4em}
\textbf{Active dry yeast:} Works fine; expect slightly longer rise times.

\vspace{0.4em}
\textbf{Hydration matters:} Freshly milled flour absorbs more. Keep the dough \emph{tacky}; resist adding excess flour.
\end{tcolorbox}

\vfill
{\footnotesize \textcolor{Soft}{\emph{Scaling:} Ingredients listed for 1x. For 2x or 3x, multiply quantities accordingly.}}

\end{document}
