\documentclass[11pt]{article}

\usepackage[margin=1in]{geometry}
\usepackage{hyperref}
\usepackage{xcolor}
\usepackage{enumitem}
\usepackage{booktabs}
\usepackage{longtable}
\usepackage{array}
\usepackage{titlesec}

% Colors
\definecolor{sectionblue}{HTML}{003366}
\definecolor{lightgray}{gray}{0.96}

% Section formatting
\titleformat{\section}
  {\normalfont\Large\bfseries\color{sectionblue}}{\thesection}{1em}{}
\titleformat{\subsection}
  {\normalfont\large\bfseries\color{sectionblue!90}}{\thesubsection}{1em}{}
\titleformat{\subsubsection}
  {\normalfont\normalsize\bfseries\color{sectionblue!80}}{\thesubsubsection}{1em}{}

% Table column helpers
\newcolumntype{L}[1]{>{\raggedright\arraybackslash}p{#1}}
\newcolumntype{C}[1]{>{\centering\arraybackslash}p{#1}}

\title{Confluence Space Structure and Directory Rationale\\[4pt]
       \large AdjecTex Technology}
\author{}
\date{\today}

\begin{document}

\maketitle
\tableofcontents
\newpage

\section{Purpose and Scope}

This document provides the rationale for the proposed Confluence
space structure and directory (section page) model for AdjecTex
Technology. It explains why spaces, section pages, and leaf pages
are organized as described in the \emph{Space Structure and
Directories} guide and how this information architecture supports
day-to-day work across product, platform, architecture, and
application security functions.

The scope of this rationale covers:

\begin{itemize}[leftmargin=*]
  \item The general pattern of \emph{Space $\rightarrow$ Home $\rightarrow$
        Section (directory) pages $\rightarrow$ Leaf pages}.
  \item The specific roles of core spaces:
        ADJT-HOME, ADJT-APPSEC, ADJT-ARCH, ADJT-PLATFORM,
        ADJT-DELIVERY, and product spaces (e.g., SBM-APP).
  \item The key design drivers, alternatives, tradeoffs, and risks
        that influenced the chosen structure.
  \item Governance and evolution of the structure over time.
\end{itemize}

\section{Background and Problem Statement}

Confluence can either act as a clear and navigable knowledge base
or devolve into a ``junk drawer'' where content is hard to locate
and even harder to maintain. Historically, common failure modes
include:

\begin{itemize}[leftmargin=*]
  \item Flat page hierarchies with no clear boundaries between
        domains (e.g., architecture, application security, platform).
  \item Inconsistent use of spaces versus pages, leading to unclear
        ownership and permissions.
  \item Mixed content on the same page (strategy, reference,
        runbooks, and ephemeral notes), making content hard to
        consume and even harder to evolve.
\end{itemize}

AdjecTex is organizing architecture, AppSec, cloud platform, SDLC,
and product documentation in a way that:

\begin{itemize}[leftmargin=*]
  \item Keeps knowledge discoverable for new and existing team
        members.
  \item Aligns with teams, responsibilities, and governance
        structures.
  \item Supports reuse of architectural and security assets across
        multiple products and platforms.
\end{itemize}

The proposed space and directory model is intended to prevent the
``junk drawer'' outcome by enforcing simple, repeatable patterns.

\section{Design Drivers}

\subsection{Information Architecture Drivers}

The structure is driven by the following information architecture
goals:

\begin{itemize}[leftmargin=*]
  \item \textbf{Clarity of Home Pages:}
        Each space has a \emph{Home} page with a short overview and
        curated links, not a dumping ground of all content.
  \item \textbf{Predictable Hierarchy:}
        Within each space, section (directory) pages clearly act as
        \emph{buckets} for related leaf content pages.
  \item \textbf{Separation of Concerns:}
        Strategy, standards, templates, reference architectures,
        product-specific details, and runbooks live in distinct
        places.
  \item \textbf{Cross-Space Reuse:}
        Architecture templates, reference architectures, and
        AppSec program materials are reused across multiple product
        spaces via links rather than duplication.
\end{itemize}

\subsection{Organizational Drivers}

The structure also reflects organizational realities:

\begin{itemize}[leftmargin=*]
  \item \textbf{Program vs.\ Product:}
        There are cross-cutting programs (e.g., AppSec, cloud
        platform, SDLC standards) and product-specific initiatives
        (e.g., Smart Building Manager, ValueVision).
  \item \textbf{Ownership and Accountability:}
        Each space has an implicit owning team and a clear
        responsibility for keeping it current.
  \item \textbf{Stakeholder Perspectives:}
        Different stakeholders (engineers, architects, security,
        product owners, management) need predictable locations for
        their questions and documents.
\end{itemize}

\subsection{Tool and Permission Drivers}

Confluence-specific constraints also influence the design:

\begin{itemize}[leftmargin=*]
  \item \textbf{Spaces as Permission Boundaries:}
        Spaces are used where differentiated access control or a
        distinct audience is needed (e.g., AppSec vs.\ product).
  \item \textbf{Pages as Content Units:}
        Pages are used to host actual artifacts (charters, views,
        runbooks, etc.), not as the only structuring element.
  \item \textbf{Section Pages as Directories:}
        Section pages group leaf pages and carry light explanatory
        text, but not large amounts of mixed content.
\end{itemize}

\section{General Pattern: Home--Section--Leaf}

Across all spaces, the following pattern is used:

\begin{itemize}[leftmargin=*]
  \item \textbf{Space Home:} A real page providing a short overview,
        contacts, and links to major sections.
  \item \textbf{Section (Directory) Pages:} Pages whose primary
        purpose is to group child pages, e.g.,
        \emph{``1. Strategy \& Governance''}, \emph{``2. Templates''},
        \emph{``3. Architecture''}. These pages contain brief
        descriptions and navigation, not dense reference content.
  \item \textbf{Leaf Pages:} Pages that represent actual artifacts,
        such as \emph{``AppSec Program Charter''},
        \emph{``Smart Building Manager -- Context View''}, or
        \emph{``Threat Modeling Playbook''}.
\end{itemize}

Simple rule of thumb (implemented across the structure):

\begin{itemize}[leftmargin=*]
  \item If the title is a \textbf{category} or \textbf{bucket}
        (e.g., ``Templates'', ``Playbooks \& Runbooks'',
        ``Architecture''), it is a \emph{directory} page.
  \item If the title is a \textbf{specific artifact} (e.g.,
        ``Threat Modeling Playbook'', ``ADR-0001 -- Choose
        Microservices''), it is a \emph{leaf} page.
\end{itemize}

\section{Space Overview}

Table~\ref{tab:spaces-overview} summarizes the key spaces and their
primary rationale.

\begin{longtable}{L{2.5cm}L{4cm}L{5.5cm}L{3cm}}
\caption{Confluence Spaces and Primary Rationale}
\label{tab:spaces-overview}\\
\toprule
\textbf{Space Key} & \textbf{Name} &
\textbf{Primary Purpose} &
\textbf{Primary Owner(s)} \\
\midrule
\endfirsthead
\toprule
\textbf{Space Key} & \textbf{Name} &
\textbf{Primary Purpose} &
\textbf{Primary Owner(s)} \\
\midrule
\endhead
\midrule
\multicolumn{4}{r}{\emph{Continued on next page}}\\
\bottomrule
\endfoot
\bottomrule
\endlastfoot
ADJT-HOME &
Adjectex Home &
Landing space for ``how to use Confluence at AdjecTex'', high-level
standards and space directory. &
Org-wide / Architecture \\
ADJT-APPSEC &
Application Security Program &
Program-level AppSec strategy, governance, playbooks, tooling,
and per-product engagements. &
AppSec Team \\
ADJT-ARCH &
Architecture \& Engineering Standards &
Architecture documentation meta, templates, reference
architectures, and styles/patterns catalog. &
Architecture Guild / Lead Architect \\
ADJT-PLATFORM &
Cloud Platform \& Infrastructure &
Cloud landing zone, environments, platform services, and
platform runbooks/SOPs. &
Cloud Platform / SRE \\
ADJT-DELIVERY &
SDLC, CI/CD \& Dev Practices &
SDLC model, Git workflows, CI/CD standards, testing strategy,
and tooling guides. &
Engineering Enablement \\
SBM-APP (example) &
Smart Building Manager Application &
Product-specific vision, requirements, architecture, security,
operations, and knowledge base. &
SBM Product Team \\
VV-APP (example) &
ValueVision Application &
Product-specific vision, requirements, and architecture for the
ValueVision platform. &
ValueVision Product Team \\
\end{longtable}

\section{Space-by-Space Rationale}

\subsection{ADJT-HOME -- AdjecTex Home}

\subsubsection{Intent}

ADJT-HOME is the organizational landing zone. Its rationale is to:

\begin{itemize}[leftmargin=*]
  \item Provide a single place where new members learn how AdjecTex
        uses Confluence.
  \item Expose a concise \emph{space directory} linking to the main
        program and product spaces.
  \item Host only a small number of high-level pages such as
        \emph{``How to Use Confluence at AdjecTex''} and
        \emph{``Standards \& Policies (high-level)''}.
\end{itemize}

\subsubsection{Why Not a Deep Hierarchy Here}

ADJT-HOME is intentionally kept mostly flat:

\begin{itemize}[leftmargin=*]
  \item Deep hierarchies are delegated to program and product spaces.
  \item The home space should not become another ``everything''
        location; it acts as a router, not a repository.
\end{itemize}

\subsection{ADJT-APPSEC -- Application Security Program}

\subsubsection{Intent}

ADJT-APPSEC centralizes cross-cutting AppSec knowledge that serves
every product:

\begin{itemize}[leftmargin=*]
  \item \textbf{Space Home:} Overview of the AppSec program, charter
        summary, and ``How to engage AppSec''.
  \item \textbf{Strategy \& Governance:} Houses the
        \emph{AppSec Program Charter}, \emph{Policy \& Standards Index},
        and \emph{Risk \& Compliance Alignment}.
  \item \textbf{SDLC \& CI/CD Integration:} Contains the
        \emph{16-Gate CI/CD Security View}, gate groupings, and
        security tooling pages (CodeQL, secret scanning, dependency
        review, ticketing integration).
  \item \textbf{Playbooks \& Runbooks:} Threat modeling, secure code
        review, vulnerability management, incident response, and
        secret scanning triage.
  \item \textbf{Patterns \& Guidelines:} Secure coding guidelines and
        AppSec patterns for common concerns (auth, input validation,
        logging, etc.).
  \item \textbf{Engagements by Product:} Per-product AppSec
        engagement subtrees (e.g., Smart Building Manager, ValueVision,
        Learning Platform) with context, threat models, risk registers,
        and findings.
  \item \textbf{Templates:} Reusable templates for threat models,
        security reviews, security user stories, and acceptance
        criteria.
\end{itemize}

\subsubsection{Rationale for Directories}

Each section page under ADJT-APPSEC is a directory by design:

\begin{itemize}[leftmargin=*]
  \item \emph{``1. Strategy \& Governance''} is a bucket for program
        definition artifacts, not mixed with runbooks or tooling.
  \item \emph{``2. SDLC \& CI/CD Integration''} groups all materials
        about how security gates integrate into pipelines.
  \item \emph{``3. Playbooks \& Runbooks''} groups operational guides,
        which differ in audience and lifecycle from strategy documents.
  \item \emph{``5. Engagements by Product''} is a directory with one
        child page per product engagement, which themselves act as
        directories for leaf artifacts like \emph{Threat Model} or
        \emph{Risk Register}.
\end{itemize}

\subsection{ADJT-ARCH -- Architecture \& Engineering Standards}

\subsubsection{Intent}

ADJT-ARCH is the central place for how AdjecTex documents and thinks
about architecture:

\begin{itemize}[leftmargin=*]
  \item \textbf{Architecture Documentation Meta:} Explains views vs.\
        beyond views, stakeholder needs, and documentation lifecycle.
  \item \textbf{Templates:} Provides canonical templates for
        architecture overview documents, context views, module views,
        C\&C views, deployment views, data model views, behavior views,
        work assignment views, rationale/decisions, roadmaps,
        variability, interfaces, quality scenarios, and ADRs.
  \item \textbf{Reference Architectures:} Holds reusable
        architectures (web/microservices, event-driven messaging, data
        \& analytics/lakehouse, cloud governance, security references).
  \item \textbf{Styles \& Patterns Catalog:} Documents architectural
        styles and patterns used at AdjecTex.
  \item \textbf{System/Project Architecture Index:} Index pages that
        link from this standards space to the actual product-specific
        architecture spaces.
\end{itemize}

The rationale is to avoid each product reinventing templates and
styles, and to centralize the ``how we do architecture'' narrative.

\subsection{ADJT-PLATFORM -- Cloud Platform \& Infrastructure}

\subsubsection{Intent}

ADJT-PLATFORM documents the shared platform on which products run:

\begin{itemize}[leftmargin=*]
  \item \textbf{Cloud Landing Zone \& Governance:} High-level cloud
        governance, landing zone concepts, policies, and reference
        diagrams.
  \item \textbf{Environments:} Patterns for development, test/staging,
        and production environments.
  \item \textbf{Networking \& Identity:} Network topology, identity
        structure, and related diagrams.
  \item \textbf{Platform Services:} Shared services such as logging
        \& observability, monitoring \& alerting, secrets management,
        and CI runners.
  \item \textbf{Platform Runbooks \& SOPs:} Operational runbooks,
        incident response, backup/restore procedures.
  \item \textbf{Infrastructure Architecture \& Diagrams:} Platform
        architecture views that product teams can reference.
\end{itemize}

Rationale: platform concerns are cross-cutting and must not be buried
inside individual product spaces. This space provides a single source
of truth.

\subsection{ADJT-DELIVERY -- SDLC, CI/CD \& Dev Practices}

\subsubsection{Intent}

ADJT-DELIVERY is the home for engineering process and delivery
practices:

\begin{itemize}[leftmargin=*]
  \item \textbf{SDLC Model \& Policies:} Overall SDLC and policies
        (feature development, hotfixes, emergency changes).
  \item \textbf{Git Workflow \& Branch Strategy:} GitFlow and
        trunk-based guidelines, branch naming conventions.
  \item \textbf{CI/CD Standards:} CI and CD standards, with cross-links
        to the AppSec 16-gate view.
  \item \textbf{Testing Strategy:} Unit, integration, performance, and
        security testing strategies.
  \item \textbf{Definition of Ready / Done:} Shared criteria to
        harmonize work intake and completion.
  \item \textbf{Tooling Guides:} GitHub, Jira, and code review best
        practices.
\end{itemize}

Having a dedicated delivery space separates process guidance from
product-specific implementation details and from platform
infrastructure specifics.

\subsection{Product Spaces (e.g., SBM-APP -- Smart Building Manager)}

\subsubsection{Intent}

Each product space (e.g., SBM-APP for Smart Building Manager) is the
canonical home for that product's documentation:

\begin{itemize}[leftmargin=*]
  \item \textbf{Vision \& Strategy:} Product vision, roadmap, and
        goals/OKRs.
  \item \textbf{Stakeholders \& Requirements:} Personas, high-level
        requirements, epics, and backlog links.
  \item \textbf{Architecture:} Context, module, C\&C, deployment, data
        model, behavior views, quality scenarios, rationale, roadmap,
        and ADRs.
  \item \textbf{Security \& Compliance:} AppSec engagement link,
        product threat model, findings \& remediation summaries.
  \item \textbf{Operations \& SRE:} SLIs/SLOs, runbooks, monitoring,
        and alerting configuration.
  \item \textbf{Implementation Notes \& Testing:} Design notes,
        technical decisions, test strategies, and results.
  \item \textbf{Knowledge Base:} How-to guides, FAQs, and known issues.
\end{itemize}

The rationale is to:

\begin{itemize}[leftmargin=*]
  \item Give each product team a coherent, end-to-end space that
        mirrors the lifecycle of their system.
  \item Avoid mixing product-specific details into program spaces
        like ADJT-ARCH, ADJT-APPSEC, or ADJT-PLATFORM.
  \item Reinforce the pattern that cross-cutting standards and
        templates live centrally, while concrete instantiations live
        within product spaces.
\end{itemize}

\section{Alternatives Considered}

Several alternative structures were considered and rejected:

\subsection{Single Monolithic ``Engineering'' Space}

\paragraph{Description.}
All content (architecture, AppSec, platform, SDLC, and products)
would reside in a single space with a deep page tree.

\paragraph{Reasons Rejected.}

\begin{itemize}[leftmargin=*]
  \item Poor permission boundaries; hard to grant/limit access per
        program or product.
  \item Difficult to maintain a clear mental model as the page tree
        grows.
  \item Increases the ``junk drawer'' risk and makes onboarding
        harder.
\end{itemize}

\subsection{Spaces Only Per Team, No Program-Level Spaces}

\paragraph{Description.}
Each team (e.g., AppSec, Architecture, Platform) would own one or
more spaces without a clear program-oriented split (e.g., no distinct
architecture standards or AppSec program spaces).

\paragraph{Reasons Rejected.}

\begin{itemize}[leftmargin=*]
  \item Blurs the distinction between cross-cutting programs and
        project work.
  \item Makes it harder to have a single canonical location for
        standards and templates.
  \item Encourages duplication of patterns across multiple team
        spaces.
\end{itemize}

\subsection{Spaces Only Per Product, No Shared Program Spaces}

\paragraph{Description.}
Only product spaces would exist, with standards and patterns copied
into each product space as needed.

\paragraph{Reasons Rejected.}

\begin{itemize}[leftmargin=*]
  \item Strong duplication of templates and standards across
        products.
  \item Increased risk of divergence and inconsistency (e.g., multiple
        versions of an architecture overview template).
  \item Harder for cross-cutting teams (AppSec, architecture,
        platform) to drive consistent practices.
\end{itemize}

\section{Tradeoffs and Implications}

The chosen structure introduces several tradeoffs:

\subsection{Cross-Space Navigation vs.\ Duplication}

\begin{itemize}[leftmargin=*]
  \item \textbf{Benefit:} Standards, patterns, and templates are
        centralized (ADJT-ARCH, ADJT-APPSEC, ADJT-DELIVERY).
  \item \textbf{Cost:} Users occasionally navigate between multiple
        spaces (e.g., from a product space to AppSec engagement pages
        or architecture templates).
\end{itemize}

This is mitigated by using cross-links from product spaces to relevant
program pages.

\subsection{More Spaces vs.\ Simpler Space List}

\begin{itemize}[leftmargin=*]
  \item \textbf{Benefit:} Each space has a clear mission and owning
        team, making stewardship easier.
  \item \textbf{Cost:} The global space list is slightly longer, but
        this is mitigated by the \emph{Space Directory} in ADJT-HOME.
\end{itemize}

\subsection{Structured Directories vs.\ Ad-Hoc Trees}

\begin{itemize}[leftmargin=*]
  \item \textbf{Benefit:} The pattern of Home $\rightarrow$ Section
        $\rightarrow$ Leaf pages allows users to predict where to
        place and find content.
  \item \textbf{Cost:} Some content reorganizations may be required
        when new categories emerge; however, the pattern itself
        remains stable.
\end{itemize}

\section{Governance and Evolution}

\subsection{Ownership}

Each space has a clear owner:

\begin{itemize}[leftmargin=*]
  \item ADJT-HOME: Architecture leadership or an agreed central
        documentation owner.
  \item ADJT-APPSEC: AppSec lead or AppSec program owner.
  \item ADJT-ARCH: Lead architect or architecture guild.
  \item ADJT-PLATFORM: Cloud platform lead / SRE lead.
  \item ADJT-DELIVERY: Engineering enablement or SDLC process owner.
  \item Product spaces (e.g., SBM-APP): Product owner and,
        operationally, the product team.
\end{itemize}

\subsection{Change Process}

\begin{itemize}[leftmargin=*]
  \item Structural changes at the \emph{space} or \emph{section}
        level should be discussed and agreed upon by the owning team
        and key stakeholders.
  \item Significant changes (e.g., new spaces, renaming spaces,
        collapsing/expanding major sections) should be documented and,
        if appropriate, captured in ADRs in ADJT-ARCH.
  \item Leaf pages can be added more freely, provided they follow the
        naming and placement conventions described in the
        \emph{Space Structure and Directories} guide.
\end{itemize}

\subsection{Name and Label Conventions}

\begin{itemize}[leftmargin=*]
  \item Numbered section pages (e.g., \emph{``1. Strategy \& Governance''})
        make ordering explicit and stable.
  \item Consistent suffixes (e.g., ``-- Context View'', ``-- Module
        View'') make it clear what kind of artifact a page represents.
  \item Labels can be used to tag pages by product, domain,
        technology, or concern (e.g., \texttt{appsec},
        \texttt{architecture-view}, \texttt{runbook}).
\end{itemize}

\section{Traceability and Usage Scenarios}

\subsection{Stakeholder--Question--Location Examples}

\begin{itemize}[leftmargin=*]
  \item \textbf{Question:} ``Where is the architecture overview for
        Smart Building Manager?''
  
        \emph{Answer:} SBM-APP $\rightarrow$
        \emph{3.~Architecture} section (context, module, C\&C,
        deployment views, etc.).
  \item \textbf{Question:} ``What is our standard architecture
        overview template?''
  
        \emph{Answer:} ADJT-ARCH $\rightarrow$
        \emph{2.~Templates} $\rightarrow$
        \emph{Architecture Overview Document Template}.
  \item \textbf{Question:} ``Where do I find the AppSec threat
        modeling playbook?''
  
        \emph{Answer:} ADJT-APPSEC $\rightarrow$
        \emph{3.~Playbooks \& Runbooks} $\rightarrow$
        \emph{Threat Modeling Playbook}.
  \item \textbf{Question:} ``What are the CI/CD security gates and
        how do they integrate with GitHub?''
  
        \emph{Answer:} ADJT-APPSEC $\rightarrow$
        \emph{2.~SDLC \& CI/CD Integration} $\rightarrow$
        \emph{16-Gate CI/CD Security View (Overview)}.
  \item \textbf{Question:} ``Where are the cloud environment
        patterns?''
  
        \emph{Answer:} ADJT-PLATFORM $\rightarrow$
        \emph{Environments} section.
\end{itemize}

This mapping ensures that the structure is not only theoretically
sound but also practical in answering real questions.

\section{Risks and Open Issues}

\subsection{Risks}

\begin{itemize}[leftmargin=*]
  \item \textbf{Drift from Intended Structure:} Over time, users may
        start creating ad-hoc sections or mixing content on directory
        pages.
  \item \textbf{Stale Landing Pages:} Home pages and section pages
        that are not kept up to date can mislead users.
  \item \textbf{Over-Segmentation:} There is a risk of creating too
        many spaces or sections; this is mitigated by using ADJT-HOME
        as a curated directory.
\end{itemize}

\subsection{Open Issues}

Potential areas for future refinement include:

\begin{itemize}[leftmargin=*]
  \item Finalizing labeling standards across all spaces.
  \item Agreeing on a formal process for deprecating spaces, sections,
        or templates when they are superseded.
  \item Deciding whether some highly related spaces (e.g.,
        ADJT-ARCH and ADJT-APPSEC) should share joint reference
        pages for security architecture.
\end{itemize}

\section{Conclusion}

The proposed Confluence space structure and directory model for
AdjecTex is intentionally simple, repeatable, and aligned with both
organizational responsibilities and architectural documentation
principles. By clearly separating program-level standards and
templates from product-specific implementations, and by using a
consistent Home--Section--Leaf pattern, this structure aims to keep
Confluence usable, navigable, and sustainable as the organization
grows.

\end{document}

