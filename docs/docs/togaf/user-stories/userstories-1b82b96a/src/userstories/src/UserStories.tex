\documentclass[11pt,a4paper]{article}

% ---------- Page & typography ----------
\usepackage[a4paper,margin=1in]{geometry}
\usepackage{lmodern}
\usepackage[T1]{fontenc}
\usepackage[utf8]{inputenc}
\usepackage{microtype}
\usepackage{parskip}

% ---------- Structure & lists ----------
\usepackage{titlesec}
\titlespacing*{\section}{0pt}{8pt plus 2pt}{6pt}
\titlespacing*{\subsection}{0pt}{6pt}{4pt}
\usepackage{enumitem}
\setlist{itemsep=2pt, topsep=4pt, leftmargin=1.2em}

% ---------- Color & links ----------
\usepackage[dvipsnames]{xcolor}
\usepackage{hyperref}
\hypersetup{
  colorlinks=true,
  linkcolor=MidnightBlue,
  urlcolor=MidnightBlue,
  citecolor=MidnightBlue,
  pdftitle={Study Plan — TOGAF: Complete User Story Cards},
  pdfauthor={},
  pdfsubject={TOGAF Study Plan — User Story Card Template and Full Chapter Cards}
}

% ---------- Boxes, tables, symbols ----------
\usepackage[most]{tcolorbox}
\usepackage{tikz}
\usepackage{tabularx,array}
\usepackage{amssymb}

% ===== Card look & feel (to match the reference image) =====
\definecolor{cardframe}{RGB}{200,205,210} % soft gray frame
\definecolor{cardback}{RGB}{248,249,250}  % near-white background
\definecolor{chipback}{RGB}{238,239,241}  % tag "pill" background
\definecolor{chipborder}{RGB}{180,184,188}

% Inline "pill" tag
\newcommand{\Tag}[1]{%
  \tcbox[
    on line, rounded corners,
    colback=chipback, colframe=chipborder,
    boxrule=0.3pt, left=5pt, right=5pt, top=2pt, bottom=2pt]{\footnotesize #1}%
}

% Left (labels) and right (values) columns
\newcolumntype{L}{>{\raggedleft\arraybackslash\bfseries}p{0.27\textwidth}}
\newcolumntype{Y}{>{\raggedright\arraybackslash}X}

% Story Card (details + BDD)
\newtcolorbox{StoryCard}[1]{  enhanced, breakable,
  colback=gray!2, colframe=gray!50,
  boxrule=0.4pt, arc=2pt,
  colbacktitle=gray!6, coltitle=black,
  fonttitle=\bfseries\large,
  title={#1},
  left=8pt, right=8pt, top=8pt, bottom=8pt,
  frame code={%
  \path[draw=MidnightBlue,line width=2pt]
    (frame.south west) -- (frame.north west);
}

}

% Tasks box (checkbox list)
\newtcolorbox{TasksBox}{  enhanced, breakable,
  colback=gray!1, colframe=gray!35,
  boxrule=0.4pt, arc=2pt,
  colbacktitle=gray!6, coltitle=black,
  fonttitle=\bfseries,
  title=Tasks,
  left=10pt, right=10pt, top=6pt, bottom=6pt,
  frame code={%
  \path[draw=MidnightBlue,line width=2pt]
    (frame.south west) -- (frame.north west);
}

}

% Convenience macros
\newcommand{\Field}[2]{\textbf{#1} & #2\\}
\newcommand{\StoryText}[1]{\textbf{Story}\quad \textit{#1}}
\newcommand{\NonFunctional}[1]{\textbf{Non-Functional}\quad #1}
\newcommand{\BDDHeader}{\textbf{Acceptance Criteria (BDD)}}
\newcommand{\BDDRow}[2]{\textbf{#1}\quad #2}
\newcommand{\DOR}[1]{\textit{\footnotesize Definition of Ready:} #1}
\newcommand{\DOD}[1]{\textit{\footnotesize Definition of Done:} #1}
\newcommand{\Task}[1]{\item[$\square$] #1}

% ===========================================================
% Document
% ===========================================================


% ==== ChatGPT Patch: helpers only (no conflicting \StoryCard macro) ====
\makeatletter
\@ifpackageloaded{array}{}{\usepackage{array}}
\@ifpackageloaded{tabularx}{}{\usepackage{tabularx}}
\@ifpackageloaded{ragged2e}{}{\usepackage{ragged2e}}
\@ifpackageloaded{tcolorbox}{}{\usepackage{tcolorbox}}
\tcbuselibrary{most}
\@ifpackageloaded{xcolor}{}{\usepackage[dvipsnames,table]{xcolor}}
\@ifundefinedcolor{MidnightBlue}{\definecolor{MidnightBlue}{RGB}{25,25,112}}{}
\providecommand{\cb}{\(\square\)}
\providecommand{\DoR}{\textbf{Definition of Ready:} Persona clear; AC drafted; Dependencies known; Estimate set.}
\providecommand{\DoD}{\textbf{Definition of Done:} All ACs pass; Tests green; Security/a11y checks; Docs updated; Deployed/flagged.}
\providecommand{\badge}[1]{\fbox{\footnotesize #1}}
% Column types used by cards
\@ifundefined{Lbl}{\newcolumntype{Lbl}[1]{>{\raggedleft\arraybackslash\bfseries}p{#1}}}{}
\@ifundefined{Y}{\newcolumntype{Y}{>{\RaggedRight\arraybackslash}X}}{}
% Field helper for tabular rows
\providecommand{\Field}[2]{\textbf{#1} & #2\\}
\makeatother
% ==== End Patch ====


\begin{document}
\begin{center}
  {\Large \bfseries Study Plan — TOGAF: User Story Card Template \& Full Chapter Cards}\\[2pt]
  {\small Standalone workbook for planning, executing, and evidencing TOGAF study via user stories.}
\end{center}

% -------------------------
\section*{How to Use This Template}
Create one card per \textbf{chapter/phase}. Each card captures purpose, value, risks, acceptance criteria, and hands-on evidence you will produce.

\subsection*{Story Card Definition (required fields)}
\begin{itemize}[leftmargin=1.2em]
  \item \textbf{ID \& Title} (e.g., \texttt{TOGAF-A1 — Architecture Vision})
  \item \textbf{Epic / Feature}: capability this story contributes to (e.g., \emph{ADM Mastery})
  \item \textbf{Business Value}: why mastering this chapter matters
  \item \textbf{Priority / Estimate}: e.g., \emph{Priority: Must}, \emph{SP: 3}
  \item \textbf{Persona}: learner role (EA, solution architect, candidate)
  \item \textbf{Dependencies}: readings or concepts to complete first
  \item \textbf{Assumptions}: context and constraints
  \item \textbf{Risks}: pitfalls (e.g., confusing artifacts vs deliverables)
  \item \textbf{Story}: \emph{As a \textless persona\textgreater, I want \textless capability\textgreater\ so that \textless outcome\textgreater.}
  \item \textbf{Non-Functional}: quality bars for your output (e.g., \Tag{Accuracy} \Tag{Traceability})
  \item \textbf{Acceptance Criteria (BDD)}: \emph{Scenario / Given / When / Then}
  \item \textbf{Definition of Ready}: entry conditions
  \item \textbf{Definition of Done}: exit conditions and evidence
\end{itemize}

\subsection*{User Story Template (copy/paste)}
\noindent\emph{As a \underline{[persona]} I want \underline{[capability]} so that \underline{[business outcome]}.}\\
\textbf{Scenario} \underline{[name]} \\
\textbf{Given} \underline{[preconditions]} \\
\textbf{When} \underline{[action you take]} \\
\textbf{Then} \underline{[verifiable result]}.

\clearpage

% ===========================================================
% PART 0 — INTRODUCTION & CORE CONCEPTS
% ===========================================================

\begin{StoryCard}{TOGAF-00 — Getting Started}
\begin{tabularx}{\textwidth}{L Y}
  \Field{Epic / Feature}{TOGAF Foundations \& ADM Overview}
  \Field{Business Value}{Create a mental model of the standard so later chapters connect cleanly; reduce confusion between phases, artifacts, and deliverables.}
  \Field{Priority / Estimate}{Priority: Must \quad\quad SP: 3}
  \Field{Persona}{New EA practitioner / certification candidate}
  \Field{Dependencies}{None}
  \Field{Assumptions}{Access to TOGAF PDFs and note repository}
  \Field{Risks}{Over-memorizing terms without understanding inputs/steps/outputs}
\end{tabularx}

\vspace{4pt}
\StoryText{As a learner, I want an ADM ``at-a-glance'' so that I can place each chapter in context and study efficiently.}

\vspace{2pt}
\NonFunctional{\Tag{Accuracy} \Tag{Clarity} \Tag{Traceability}}

\vspace{6pt}
\BDDHeader

\BDDRow{Scenario}{Happy path}\\
\BDDRow{Given}{the PDFs and a study workspace are available}\\
\BDDRow{When}{I create a one-page ADM map showing purpose, inputs, outputs for each phase}\\
\BDDRow{Then}{I can explain each phase in one sentence and cite one artifact per phase}

\vspace{6pt}
\DOR{Materials available; 60 minutes blocked.}\quad
\DOD{One-page ADM map saved; glossary started; next chapter chosen.}
\end{StoryCard}

\begin{TasksBox}
\begin{itemize}[label={}, leftmargin=1.2em]
  \Task{List phases: Preliminary, A–H, Requirements; jot one-sentence purpose each.}
  \Task{Sketch phase $\rightarrow$ inputs/steps/outputs (simple table or diagram).}
  \Task{Write a 3–line note: \emph{deliverable vs artifact vs building block}.}
  \Task{Create repo/folder; add \texttt{README.md} with links to all cards.}
  \Task{Plan next session: \emph{Phase A — Architecture Vision}.}
\end{itemize}
\end{TasksBox}
\clearpage

\begin{StoryCard}{TOGAF-01 — Core Concepts}
\begin{tabularx}{\textwidth}{L Y}
  \Field{Epic / Feature}{Foundations Mastery}
  \Field{Business Value}{Distinguish Business, Data, Application, Technology domains and understand how they interrelate within TOGAF.}
  \Field{Priority / Estimate}{Priority: Must \quad\quad SP: 3}
  \Field{Persona}{Learner seeking big-picture clarity}
  \Field{Dependencies}{TOGAF-00}
  \Field{Assumptions}{Basic familiarity with enterprise change initiatives}
  \Field{Risks}{Mixing up ``view'' vs ``viewpoint''; artifacts vs deliverables}
\end{tabularx}

\vspace{4pt}
\StoryText{As a learner, I want to summarize the four architecture domains and key TOGAF terms so that I can navigate later chapters precisely.}

\vspace{2pt}
\NonFunctional{\Tag{Terminology} \Tag{Clarity} \Tag{Traceability}}

\vspace{6pt}
\BDDHeader

\BDDRow{Scenario}{Happy path}\\
\BDDRow{Given}{the glossary and Part 0 are available}\\
\BDDRow{When}{I write concise definitions and an example for each key term}\\
\BDDRow{Then}{I can map a real project element to its correct domain and artifact type}

\vspace{6pt}
\DOR{Glossary indexed; note template ready.}\quad
\DOD{One-page glossary saved with examples; quiz self-test (10 Q) passed.}
\end{StoryCard}

\begin{TasksBox}
\begin{itemize}[label={}, leftmargin=1.2em]
  \Task{Define: domain, view, viewpoint, deliverable, artifact, building block.}
  \Task{Produce a table: domain $\rightarrow$ concerns $\rightarrow$ typical artifacts.}
  \Task{Add two project examples and classify them into domains/artifacts.}
\end{itemize}
\end{TasksBox}
\clearpage

% ===========================================================
% PART 1 — THE ADM (PRELIMINARY + A–H + REQUIREMENTS)
% ===========================================================

\begin{StoryCard}{TOGAF-P — Preliminary Phase}
\begin{tabularx}{\textwidth}{L Y}
  \Field{Epic / Feature}{Architecture Capability Setup}
  \Field{Business Value}{Tailor method, define principles, and set up tools so ADM work is consistent and governable.}
  \Field{Priority / Estimate}{Priority: Must \quad\quad SP: 5}
  \Field{Persona}{EA lead / candidate}
  \Field{Dependencies}{TOGAF-01}
  \Field{Assumptions}{Org context (hypothetical is fine)}
  \Field{Risks}{Skipping principle quality (rationale/implications)}
\end{tabularx}

\vspace{4pt}
\StoryText{As an EA practitioner, I want to tailor the method and draft principles so that delivery is aligned and repeatable.}

\vspace{2pt}
\NonFunctional{\Tag{Governance} \Tag{Repeatability} \Tag{Traceability}}

\vspace{6pt}
\BDDHeader

\BDDRow{Scenario}{Happy path}\\
\BDDRow{Given}{a target organization context}\\
\BDDRow{When}{I define 8–12 principles with rationale \& implications and a tooling/repo approach}\\
\BDDRow{Then}{a mini ``Architecture Practice Setup'' pack exists and is reusable in later phases}

\vspace{6pt}
\DOR{Context chosen; template ready.}\quad
\DOD{Principles, tailoring notes, and repo structure saved.}
\end{StoryCard}

\begin{TasksBox}
\begin{itemize}[label={}, leftmargin=1.2em]
  \Task{Draft principles (business/data/app/tech) with rationale \& implications.}
  \Task{Describe repo \& tooling (modeling, catalog, publishing).}
  \Task{List governance touchpoints (reviews, checkpoints).}
\end{itemize}
\end{TasksBox}
\clearpage

\begin{StoryCard}{TOGAF-A — Phase A: Architecture Vision}
\begin{tabularx}{\textwidth}{L Y}
  \Field{Epic / Feature}{ADM Mastery}
  \Field{Business Value}{Align stakeholders, define scope and success, and set the change program up for traceable outcomes.}
  \Field{Priority / Estimate}{Priority: Must \quad\quad SP: 5}
  \Field{Persona}{Enterprise Architect}
  \Field{Dependencies}{TOGAF-P}
  \Field{Assumptions}{Practice initiative identified}
  \Field{Risks}{Confusing Vision with detailed solution; skipping KPIs}
\end{tabularx}

\vspace{4pt}
\StoryText{As an EA, I want to produce an Architecture Vision so that stakeholders share a common picture of scope, value, and KPIs.}

\vspace{2pt}
\NonFunctional{\Tag{Stakeholder Alignment} \Tag{KPIs} \Tag{Concise}}

\vspace{6pt}
\BDDHeader

\BDDRow{Scenario}{Happy path}\\
\BDDRow{Given}{initiative and stakeholders are known}\\
\BDDRow{When}{I draft problem, objectives, scope, constraints, KPIs, and risks}\\
\BDDRow{Then}{a one-page Vision and draft Statement of Architecture Work are produced}

\vspace{6pt}
\DOR{Stakeholders listed; session booked.}\quad
\DOD{Vision + SoAW saved; KPIs accepted; risks logged.}
\end{StoryCard}

\begin{TasksBox}
\begin{itemize}[label={}, leftmargin=1.2em]
  \Task{Map stakeholders and key concerns; capture KPIs.}
  \Task{Write one-page Vision; draft Statement of Architecture Work outline.}
  \Task{Prepare a 3–5 slide readout.}
\end{itemize}
\end{TasksBox}
\clearpage

\begin{StoryCard}{TOGAF-B — Phase B: Business Architecture}
\begin{tabularx}{\textwidth}{L Y}
  \Field{Epic / Feature}{Target Business Design}
  \Field{Business Value}{Clarify capabilities, processes, and organizational changes required to realize the vision.}
  \Field{Priority / Estimate}{Priority: Must \quad\quad SP: 8}
  \Field{Persona}{Business/Enterprise Architect}
  \Field{Dependencies}{TOGAF-A}
  \Field{Assumptions}{Value streams identifiable}
  \Field{Risks}{Jumping to tech solutions before business capability gaps are clear}
\end{tabularx}

\vspace{4pt}
\StoryText{As an architect, I want to define target business capabilities and processes so that value realization is explicit and testable.}

\vspace{2pt}
\NonFunctional{\Tag{Clarity} \Tag{Capability-Based} \Tag{Traceability}}

\vspace{6pt}
\BDDHeader

\BDDRow{Scenario}{Capability-led}\\
\BDDRow{Given}{current business model and value streams}\\
\BDDRow{When}{I document baseline/target business capabilities, processes, org mappings}\\
\BDDRow{Then}{gaps and candidate work packages are identified with business owners}

\vspace{6pt}
\DOR{Value stream sketch ready.}\quad
\DOD{Capability map + process high-level model + gap list produced.}
\end{StoryCard}

\begin{TasksBox}
\begin{itemize}[label={}, leftmargin=1.2em]
  \Task{Create capability map (baseline/target).}
  \Task{Draft key process/context diagrams and stakeholder RACI.}
  \Task{Record gaps \& candidate work packages with business value statements.}
\end{itemize}
\end{TasksBox}
\clearpage

\begin{StoryCard}{TOGAF-C1 — Phase C: Data Architecture}
\begin{tabularx}{\textwidth}{L Y}
  \Field{Epic / Feature}{Information Systems — Data}
  \Field{Business Value}{Ensure data entities, flows, and qualities support business capabilities and compliance.}
  \Field{Priority / Estimate}{Priority: Must \quad\quad SP: 5}
  \Field{Persona}{Data/Information Architect}
  \Field{Dependencies}{TOGAF-B}
  \Field{Assumptions}{Authoritative sources are discoverable}
  \Field{Risks}{Underspecified ownership and quality constraints}
\end{tabularx}

\vspace{4pt}
\StoryText{As an architect, I want to model baseline/target data and constraints so that interoperability and governance are explicit.}

\vspace{2pt}
\NonFunctional{\Tag{Lineage} \Tag{Interoperability} \Tag{Quality}}

\vspace{6pt}
\BDDHeader

\BDDRow{Scenario}{Entity/flow coverage}\\
\BDDRow{Given}{priority capabilities from Phase B}\\
\BDDRow{When}{I define data entities, relationships, flows, ownership, and quality constraints}\\
\BDDRow{Then}{a data view and constraint list exist and link to application/services}

\vspace{6pt}
\DOR{Critical entities identified.}\quad
\DOD{Data views + constraints + governance/ownership recorded.}
\end{StoryCard}

\begin{TasksBox}
\begin{itemize}[label={}, leftmargin=1.2em]
  \Task{Create data entity-relationship view for priority scope.}
  \Task{Document data flows and authoritative sources; list quality constraints.}
  \Task{Map entities to owning teams and compliance requirements.}
\end{itemize}
\end{TasksBox}
\clearpage

\begin{StoryCard}{TOGAF-C2 — Phase C: Application Architecture}
\begin{tabularx}{\textwidth}{L Y}
  \Field{Epic / Feature}{Information Systems — Applications}
  \Field{Business Value}{Define application/services landscape and interactions to meet business and data needs.}
  \Field{Priority / Estimate}{Priority: Must \quad\quad SP: 5}
  \Field{Persona}{Application/Integration Architect}
  \Field{Dependencies}{TOGAF-C1}
  \Field{Assumptions}{Integration constraints known}
  \Field{Risks}{Hidden interoperability constraints}
\end{tabularx}

\vspace{4pt}
\StoryText{As an architect, I want to model baseline/target applications and interactions so that interoperability constraints are explicit.}

\vspace{2pt}
\NonFunctional{\Tag{Interoperability} \Tag{Resilience} \Tag{Security}}

\vspace{6pt}
\BDDHeader

\BDDRow{Scenario}{Service interaction}\\
\BDDRow{Given}{data and capability needs}\\
\BDDRow{When}{I map application components, interfaces, and interaction patterns}\\
\BDDRow{Then}{target application views and interface constraints are baselined}

\vspace{6pt}
\DOR{Data constraints documented.}\quad
\DOD{App interaction views + interface/interop constraints recorded.}
\end{StoryCard}

\begin{TasksBox}
\begin{itemize}[label={}, leftmargin=1.2em]
  \Task{Draw baseline vs target application/service maps.}
  \Task{Capture key interfaces and NFRs (latency, availability, security).}
  \Task{List interoperability constraints and standards.}
\end{itemize}
\end{TasksBox}
\clearpage

\begin{StoryCard}{TOGAF-D — Phase D: Technology Architecture}
\begin{tabularx}{\textwidth}{L Y}
  \Field{Epic / Feature}{Platform Enablement}
  \Field{Business Value}{Provide platforms and tech services that enable application and data designs.}
  \Field{Priority / Estimate}{Priority: Must \quad\quad SP: 5}
  \Field{Persona}{Technology Architect}
  \Field{Dependencies}{TOGAF-C2}
  \Field{Assumptions}{Hosting patterns (cloud/on-prem) in scope}
  \Field{Risks}{Underestimating NFRs like DR, observability, security}
\end{tabularx}

\vspace{4pt}
\StoryText{As a tech architect, I want to define target tech services/components so that workloads are supported and governable.}

\vspace{2pt}
\NonFunctional{\Tag{Scalability} \Tag{Availability} \Tag{Security} \Tag{Observability}}

\vspace{6pt}
\BDDHeader

\BDDRow{Scenario}{Service inventory}\\
\BDDRow{Given}{application and data needs}\\
\BDDRow{When}{I define platform services, standards, and constraints}\\
\BDDRow{Then}{a target tech service model and standards list are approved for planning}

\vspace{6pt}
\DOR{Hosting strategy chosen.}\quad
\DOD{Tech services diagram + standards catalog saved.}
\end{StoryCard}

\begin{TasksBox}
\begin{itemize}[label={}, leftmargin=1.2em]
  \Task{Document compute, storage, network, security, observability services.}
  \Task{Capture standards (OS, runtime, DB, integration, IAM).}
  \Task{Map NFRs to platform capabilities and test hooks.}
\end{itemize}
\end{TasksBox}
\clearpage

\begin{StoryCard}{TOGAF-E — Phase E: Opportunities \& Solutions}
\begin{tabularx}{\textwidth}{L Y}
  \Field{Epic / Feature}{Solution Shaping}
  \Field{Business Value}{Bundle work into feasible work packages and outline the initial roadmap.}
  \Field{Priority / Estimate}{Priority: Must \quad\quad SP: 5}
  \Field{Persona}{EA / Portfolio Architect}
  \Field{Dependencies}{TOGAF-B,C,D}
  \Field{Assumptions}{Dependencies and constraints captured}
  \Field{Risks}{Over-committing without value/risk trade-off}
\end{tabularx}

\vspace{4pt}
\StoryText{As an EA, I want to identify opportunities and group them into work packages so that a value-focused roadmap emerges.}

\vspace{2pt}
\NonFunctional{\Tag{Value} \Tag{Feasibility} \Tag{Traceability}}

\vspace{6pt}
\BDDHeader

\BDDRow{Scenario}{Initial roadmap}\\
\BDDRow{Given}{gaps and candidate packages}\\
\BDDRow{When}{I group packages, define transitions, and outline the roadmap}\\
\BDDRow{Then}{an initial Architecture Roadmap exists with rationale and dependencies}

\vspace{6pt}
\DOR{Gap list ready.}\quad
\DOD{Initial roadmap + transition states documented.}
\end{StoryCard}

\begin{TasksBox}
\begin{itemize}[label={}, leftmargin=1.2em]
  \Task{Bundle gaps into coherent work packages.}
  \Task{Sketch transition architectures/states.}
  \Task{Draft initial Architecture Roadmap with dependencies.}
\end{itemize}
\end{TasksBox}
\clearpage

\begin{StoryCard}{TOGAF-F — Phase F: Migration Planning}
\begin{tabularx}{\textwidth}{L Y}
  \Field{Epic / Feature}{Executable Plan}
  \Field{Business Value}{Prioritize and schedule work into an actionable program integrated with PMO/finance.}
  \Field{Priority / Estimate}{Priority: Must \quad\quad SP: 5}
  \Field{Persona}{EA / Program Planner}
  \Field{Dependencies}{TOGAF-E}
  \Field{Assumptions}{Value/risk and cost drivers known}
  \Field{Risks}{Ignoring risk-weighted value; lack of funding alignment}
\end{tabularx}

\vspace{4pt}
\StoryText{As a planner, I want an Implementation \& Migration Plan so that change is funded, sequenced, and measurable.}

\vspace{2pt}
\NonFunctional{\Tag{Feasible} \Tag{Measurable} \Tag{Aligned}}

\vspace{6pt}
\BDDHeader

\BDDRow{Scenario}{Portfolio prioritzation}\\
\BDDRow{Given}{work packages and constraints}\\
\BDDRow{When}{I score value/risk, build a timeline, and align with PMO}\\
\BDDRow{Then}{an approved Implementation \& Migration Plan is baselined}

\vspace{6pt}
\DOR{Roadmap in place.}\quad
\DOD{Plan baselined; funding checkpoints identified; KPIs tied to releases.}
\end{StoryCard}

\begin{TasksBox}
\begin{itemize}[label={}, leftmargin=1.2em]
  \Task{Score packages by value, risk, cost; produce priority matrix.}
  \Task{Create release plan/timeline with dependencies.}
  \Task{Map plan items to KPIs and budget lines.}
\end{itemize}
\end{TasksBox}
\clearpage

\begin{StoryCard}{TOGAF-G — Phase G: Implementation Governance}
\begin{tabularx}{\textwidth}{L Y}
  \Field{Epic / Feature}{Delivery Oversight}
  \Field{Business Value}{Ensure delivery remains aligned to architecture via reviews and waivers.}
  \Field{Priority / Estimate}{Priority: Should \quad\quad SP: 3}
  \Field{Persona}{Architecture Governance Lead}
  \Field{Dependencies}{TOGAF-F}
  \Field{Assumptions}{Delivery projects underway}
  \Field{Risks}{Token compliance without evidence}
\end{tabularx}

\vspace{4pt}
\StoryText{As a governance lead, I want a compliance review approach so that deviations are visible and decisions recorded.}

\vspace{2pt}
\NonFunctional{\Tag{Accountability} \Tag{Evidence} \Tag{Traceability}}

\vspace{6pt}
\BDDHeader

\BDDRow{Scenario}{Architecture compliance}\\
\BDDRow{Given}{project plans and designs}\\
\BDDRow{When}{I run compliance checkpoints and log decisions/waivers}\\
\BDDRow{Then}{a compliant status and action list exist with artifacts as evidence}

\vspace{6pt}
\DOR{Review calendar set.}\quad
\DOD{Checklists, minutes, and decisions filed in governance repo.}
\end{StoryCard}

\begin{TasksBox}
\begin{itemize}[label={}, leftmargin=1.2em]
  \Task{Create compliance checklist aligned to artifacts.}
  \Task{Schedule reviews (design, pre-implement, post-implement).}
  \Task{Set waiver process and recording location.}
\end{itemize}
\end{TasksBox}
\clearpage

\begin{StoryCard}{TOGAF-H — Phase H: Architecture Change Management}
\begin{tabularx}{\textwidth}{L Y}
  \Field{Epic / Feature}{Sustained Alignment}
  \Field{Business Value}{Adapt architecture responsibly as drivers change; trigger re-entry to ADM when needed.}
  \Field{Priority / Estimate}{Priority: Should \quad\quad SP: 3}
  \Field{Persona}{EA Practice Lead}
  \Field{Dependencies}{TOGAF-G}
  \Field{Assumptions}{Change drivers monitored}
  \Field{Risks}{Uncontrolled drift; stale standards}
\end{tabularx}

\vspace{4pt}
\StoryText{As a practice lead, I want clear re-entry triggers and maintenance processes so that change remains governed.}

\vspace{2pt}
\NonFunctional{\Tag{Controlled Change} \Tag{Auditability}}

\vspace{6pt}
\BDDHeader

\BDDRow{Scenario}{Trigger-based}\\
\BDDRow{Given}{strategy, standards, and production feedback}\\
\BDDRow{When}{I evaluate impact and decide on maintenance vs re-architecture}\\
\BDDRow{Then}{re-entry to appropriate ADM phases and updates are documented}

\vspace{6pt}
\DOR{Change log feed enabled.}\quad
\DOD{Trigger list, impact assessment template, and decision records saved.}
\end{StoryCard}

\begin{TasksBox}
\begin{itemize}[label={}, leftmargin=1.2em]
  \Task{Define triggers (strategy shifts, regulatory, tech end-of-life).}
  \Task{Create impact assessment template and RACI.}
  \Task{Set cadence for standards review and repository updates.}
\end{itemize}
\end{TasksBox}
\clearpage

\begin{StoryCard}{TOGAF-R — Requirements Management}
\begin{tabularx}{\textwidth}{L Y}
  \Field{Epic / Feature}{Continuous Requirements}
  \Field{Business Value}{Provide a single requirements thread through all phases for traceable decisions.}
  \Field{Priority / Estimate}{Priority: Must \quad\quad SP: 3}
  \Field{Persona}{All architects}
  \Field{Dependencies}{TOGAF-A through H}
  \Field{Assumptions}{Simple catalog format available}
  \Field{Risks}{Uncontrolled changes; poor traceability}
\end{tabularx}

\vspace{4pt}
\StoryText{As an architect, I want a living requirements catalog so that changes are traceable to phases and decisions.}

\vspace{2pt}
\NonFunctional{\Tag{Single Source} \Tag{Traceability} \Tag{Auditability}}

\vspace{6pt}
\BDDHeader

\BDDRow{Scenario}{Catalog lifecycle}\\
\BDDRow{Given}{incoming/changed requirements}\\
\BDDRow{When}{I log, baseline, and assess impacts}\\
\BDDRow{Then}{decisions and affected artifacts/phases are recorded}

\vspace{6pt}
\DOR{Catalog template ready.}\quad
\DOD{Catalog exists with status fields; two sample changes processed.}
\end{StoryCard}

\begin{TasksBox}
\begin{itemize}[label={}, leftmargin=1.2em]
  \Task{Create requirements catalog (ID, description, source, status, phase links).}
  \Task{Define change control and impact assessment steps.}
  \Task{Process two example changes end-to-end.}
\end{itemize}
\end{TasksBox}
\clearpage

% ===========================================================
% PART 2 — ADM TECHNIQUES
% ===========================================================

\begin{StoryCard}{TOGAF-T — ADM Techniques (Principles, Gaps, Risk, Migration, Stakeholders)}
\begin{tabularx}{\textwidth}{L Y}
  \Field{Epic / Feature}{Technique Proficiency}
  \Field{Business Value}{Apply supporting techniques to produce higher-quality, defensible architectures.}
  \Field{Priority / Estimate}{Priority: Should \quad\quad SP: 5}
  \Field{Persona}{Practicing Architect}
  \Field{Dependencies}{TOGAF-P \& A–D}
  \Field{Assumptions}{Templates available}
  \Field{Risks}{Principles without implications; superficial risk analysis}
\end{tabularx}

\vspace{4pt}
\StoryText{As an architect, I want to master ADM techniques so that my artifacts are consistent, risk-aware, and implementable.}

\vspace{2pt}
\NonFunctional{\Tag{Consistency} \Tag{Rigor} \Tag{Defensibility}}

\vspace{6pt}
\BDDHeader

\BDDRow{Scenario}{Technique application}\\
\BDDRow{Given}{a sample initiative}\\
\BDDRow{When}{I apply principles, stakeholder mapping, gap/risk, and migration techniques}\\
\BDDRow{Then}{each technique yields a concrete artifact linked to the roadmap}

\vspace{6pt}
\DOR{Initiative selected.}\quad
\DOD{Principles set, stakeholder map, gap list, risk register, and migration matrix saved.}
\end{StoryCard}

\begin{TasksBox}
\begin{itemize}[label={}, leftmargin=1.2em]
  \Task{Refine principles (add rationale \& implications).}
  \Task{Complete stakeholder map with concerns $\rightarrow$ viewpoints.}
  \Task{Perform gap analysis and risk assessment (mitigations, owners).}
  \Task{Draft migration matrix (building blocks: baseline $\rightarrow$ target).}
\end{itemize}
\end{TasksBox}
\clearpage

% ===========================================================
% PART 3 — APPLYING THE ADM
% ===========================================================

\begin{StoryCard}{TOGAF-AP — Applying the ADM (Iteration, Levels, Tailoring)}
\begin{tabularx}{\textwidth}{L Y}
  \Field{Epic / Feature}{Contextualization}
  \Field{Business Value}{Use the ADM effectively across enterprise/segment/capability scopes and agile delivery.}
  \Field{Priority / Estimate}{Priority: Should \quad\quad SP: 3}
  \Field{Persona}{EA / Method Engineer}
  \Field{Dependencies}{TOGAF-A–H}
  \Field{Assumptions}{Multiple delivery teams exist}
  \Field{Risks}{One-size-fits-all method; unclear iteration strategy}
\end{tabularx}

\vspace{4pt}
\StoryText{As a method engineer, I want a tailoring and iteration approach so that the ADM fits delivery cadence and scope.}

\vspace{2pt}
\NonFunctional{\Tag{Pragmatic} \Tag{Lightweight} \Tag{Repeatable}}

\vspace{6pt}
\BDDHeader

\BDDRow{Scenario}{Tailored ADM}\\
\BDDRow{Given}{organization constraints}\\
\BDDRow{When}{I define scopes, iteration patterns, and touchpoints with agile/PMO}\\
\BDDRow{Then}{a ``How We Use the ADM'' guide exists and is referenced by teams}

\vspace{6pt}
\DOR{Delivery practices reviewed.}\quad
\DOD{Tailoring guide approved; example iteration plan attached.}
\end{StoryCard}

\begin{TasksBox}
\begin{itemize}[label={}, leftmargin=1.2em]
  \Task{Choose scope levels (enterprise/segment/capability) and artifacts per level.}
  \Task{Define iteration style (capability- or layer-based) and review cadence.}
  \Task{Document handshakes with agile ceremonies and PMO stages.}
\end{itemize}
\end{TasksBox}
\clearpage

% ===========================================================
% PART 4 — ARCHITECTURE CONTENT
% ===========================================================

\begin{StoryCard}{TOGAF-CONT — Content (Deliverables, Artifacts, Repository, Metamodel)}
\begin{tabularx}{\textwidth}{L Y}
  \Field{Epic / Feature}{Content Mastery}
  \Field{Business Value}{Standardize outputs; enable reuse and governance via a clear repository and metamodel.}
  \Field{Priority / Estimate}{Priority: Must \quad\quad SP: 5}
  \Field{Persona}{All architects}
  \Field{Dependencies}{TOGAF-A–D}
  \Field{Assumptions}{Tooling supports catalogs/matrices/diagrams}
  \Field{Risks}{Confusing deliverables vs artifacts; inconsistent metadata}
\end{tabularx}

\vspace{4pt}
\StoryText{As an architect, I want a content model and repository structure so that artifacts are consistent and discoverable.}

\vspace{2pt}
\NonFunctional{\Tag{Findability} \Tag{Consistency} \Tag{Reuse}}

\vspace{6pt}
\BDDHeader

\BDDRow{Scenario}{Repository setup}\\
\BDDRow{Given}{selected tools and templates}\\
\BDDRow{When}{I define deliverables, artifact types, and repository sections}\\
\BDDRow{Then}{artifacts from each phase are filed with consistent metadata}

\vspace{6pt}
\DOR{Template set chosen.}\quad
\DOD{Repository map + sample artifacts (catalog, matrix, diagram) saved.}
\end{StoryCard}

\begin{TasksBox}
\begin{itemize}[label={}, leftmargin=1.2em]
  \Task{Define naming/metadata conventions (ID, owner, version, links).}
  \Task{Create ``Artifacts by Phase'' checklist (catalogs, matrices, diagrams).}
  \Task{Populate repository with at least one sample artifact per phase.}
\end{itemize}
\end{TasksBox}
\clearpage

% ===========================================================
% PART 5 — EA CAPABILITY & GOVERNANCE
% ===========================================================

\begin{StoryCard}{TOGAF-CAP — EA Capability \& Governance}
\begin{tabularx}{\textwidth}{L Y}
  \Field{Epic / Feature}{Operating the Practice}
  \Field{Business Value}{Stand up a durable architecture practice with metrics, accountability, and governance calendar.}
  \Field{Priority / Estimate}{Priority: Should \quad\quad SP: 5}
  \Field{Persona}{EA Practice Lead}
  \Field{Dependencies}{TOGAF-P \& G}
  \Field{Assumptions}{Leadership sponsorship exists}
  \Field{Risks}{Missing metrics; unclear accountability}
\end{tabularx}

\vspace{4pt}
\StoryText{As a practice lead, I want a charter, operating model, and metrics so that the EA function delivers measurable value.}

\vspace{2pt}
\NonFunctional{\Tag{Accountability} \Tag{Transparency} \Tag{Metrics}}

\vspace{6pt}
\BDDHeader

\BDDRow{Scenario}{Practice charter}\\
\BDDRow{Given}{organizational context}\\
\BDDRow{When}{I define roles, processes, metrics, and governance calendar}\\
\BDDRow{Then}{an EA Charter and accountability matrix are published}

\vspace{6pt}
\DOR{Sponsor identified.}\quad
\DOD{Charter approved; metrics baseline set; governance calendar live.}
\end{StoryCard}

\begin{TasksBox}
\begin{itemize}[label={}, leftmargin=1.2em]
  \Task{Draft EA Charter (vision, objectives, scope, interfaces).}
  \Task{Create accountability matrix (roles $\times$ decisions/artifacts).}
  \Task{Publish governance calendar (reviews, refresh cycles, metrics).}
\end{itemize}
\end{TasksBox}
\clearpage

% ===========================================================
% BLANK CARD — COPY FOR ANY EXTRA CHAPTERS/DEEP DIVES
% ===========================================================

\section*{Blank Card Template (copy and reuse)}
\begin{StoryCard}{TOGAF-\#\# — [Chapter / Phase Title]}
\begin{tabularx}{\textwidth}{L Y}
  \Field{Epic / Feature}{[e.g., ADM Mastery / Content Mastery / Governance]}
  \Field{Business Value}{[Outcome you gain by completing this unit]}
  \Field{Priority / Estimate}{Priority: [Must|Should|Could] \quad\quad SP: [1--8]}
  \Field{Persona}{[your role]}
  \Field{Dependencies}{[prior readings or cards]}
  \Field{Assumptions}{[context]}
  \Field{Risks}{[pitfalls to watch]}
\end{tabularx}

\vspace{4pt}
\StoryText{As a \underline{[persona]}, I want \underline{[capability]} so that \underline{[business outcome]}.}

\vspace{2pt}
\NonFunctional{\Tag{[Tag 1]} \Tag{[Tag 2]} \Tag{[Tag 3]}}

\vspace{6pt}
\BDDHeader

\BDDRow{Scenario}{[name]}\\
\BDDRow{Given}{[preconditions]}\\
\BDDRow{When}{[action]}\\
\BDDRow{Then}{[verifiable result]}

\vspace{6pt}
\DOR{[entry conditions]}\quad
\DOD{[exit checks and artifacts]}
\end{StoryCard}

\begin{TasksBox}
\begin{itemize}[label={}, leftmargin=1.2em]
  \Task{[hands-on task 1]}
  \Task{[hands-on task 2]}
  \Task{[hands-on task 3]}
  \Task{[artifact to produce]}
  \Task{[checkpoint to verify AC/DoD]}
\end{itemize}
\end{TasksBox}

% ===========================================================
% TIPS
% ===========================================================
\section*{Tips for Writing Effective Study Stories}
\begin{itemize}[leftmargin=1.2em]
  \item \textbf{Make outcomes measurable:} e.g., ``identify three artifacts produced in Phase B'' beats ``understand Phase B''.
  \item \textbf{Trace to TOGAF:} each task should map to inputs/steps/outputs of the chapter/phase.
  \item \textbf{Keep cards concise:} one page per story; file deep details as artifacts in your repo.
  \item \textbf{Use tags as quality bars:} \Tag{Accuracy}, \Tag{Traceability}, \Tag{Clarity}, etc.
  \item \textbf{Close the loop:} update dependencies for the next card and check DoD before moving on.
\end{itemize}

\end{document}