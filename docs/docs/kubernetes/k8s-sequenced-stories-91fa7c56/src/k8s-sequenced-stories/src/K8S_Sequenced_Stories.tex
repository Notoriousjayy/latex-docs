% !TEX TS-program = pdflatex
\documentclass[11pt]{article}

% -------------------- Packages --------------------
\usepackage[T1]{fontenc}
\usepackage{lmodern}                    % readable serif/sans defaults
\usepackage{inconsolata}                % clearer monospace for code
\usepackage{upquote}                    % straight quotes in verbatim/listings
\usepackage{microtype}
\usepackage[margin=1in]{geometry}
\usepackage[hyphens]{url}               % better URL breaks
\usepackage{hyperref}
\usepackage{bookmark}                   % robust PDF outlines (load after hyperref)
\usepackage{enumitem}
\usepackage{xcolor}
\usepackage{booktabs}
\usepackage{tabularx}
\usepackage{array}
\usepackage[most]{tcolorbox}
% Fallback: define MidnightBlue if not provided by xcolor (dvipsnames)
\makeatletter
\@ifundefinedcolor{MidnightBlue}{\definecolor{MidnightBlue}{RGB}{25,25,112}}{}
\makeatother

\usepackage{amssymb}
\usepackage{titlesec}
\usepackage{graphicx}
\usepackage{listings}

% -------------------- Readability Tweaks --------------------
\linespread{1.03}                       % slight line-height increase
\setlength{\parindent}{0pt}             % block paragraphs
\setlength{\parskip}{0.55em}            % space between paragraphs
\setlength{\emergencystretch}{2em}      % fewer overfull lines
\renewcommand{\arraystretch}{1.12}      % roomier tables
\raggedbottom                           % avoid vertical stretch on pages

% Prevent widows/orphans for smoother page breaks
\clubpenalty=10000
\widowpenalty=10000
\displaywidowpenalty=10000

% Tighter, consistent spacing for headings
\titlespacing*{\section}{0pt}{0.9em}{0.35em}
\titlespacing*{\subsection}{0pt}{0.75em}{0.25em}

% Lists: compact but readable and aligned with body text
\setlist{leftmargin=*,itemsep=2pt,topsep=4pt}
\setlist[itemize]{itemsep=2pt}
\setlist[enumerate]{itemsep=2pt}

% -------------------- Colors --------------------
\definecolor{Primary}{HTML}{0E7490}   % teal-700
\definecolor{Accent}{HTML}{0EA5E9}    % sky-500
\definecolor{Soft}{HTML}{F1F5F9}      % slate-100
\definecolor{Ink}{HTML}{0F172A}       % slate-900
\definecolor{Meta}{HTML}{475569}      % slate-600
\definecolor{OK}{HTML}{16A34A}        % green-600
\definecolor{Warn}{HTML}{EA580C}      % orange-600
\definecolor{Bad}{HTML}{DC2626}       % red-600

\hypersetup{
  colorlinks=true,
  linkcolor=Primary,
  urlcolor=Primary,
  citecolor=Primary,
  breaklinks=true,
  pdfauthor={Jordan Suber},
  pdftitle={Kubernetes Stories — Phased End-to-End Workflow}
}
\urlstyle{same}

\titleformat{\section}{\large\bfseries\color{Ink}}{\thesection}{0.6em}{}
\titleformat{\subsection}{\normalsize\bfseries\color{Ink}}{\thesubsection}{0.6em}{}

\newcommand{\checkbox}{\(\square\)}
\newcommand{\checkedbox}{\(\blacksquare\)}
\newcommand{\eg}{e.g.\ }
\newcommand{\ie}{i.e.\ }

% -------------------- Gherkin (listings) --------------------
\lstdefinelanguage{Gherkin}{
  morekeywords={Feature,Background,Scenario,Scenario\ Outline,Examples,Given,When,Then,And,But},
  sensitive=true,
}
\lstset{
  language=Gherkin,
  basicstyle=\ttfamily\small,
  keywordstyle=\color{Primary}\bfseries,
  commentstyle=\itshape\color{Meta},
  showstringspaces=false,
  frame=single,
  framerule=0.4pt,
  rulecolor=\color{Soft},
  backgroundcolor=\color{Soft},
  tabsize=2,
  columns=fullflexible,
  keepspaces=true,                      % preserve alignment
  breaklines=true,                      % wrap long steps
  breakatwhitespace=true,
  xleftmargin=1ex,                      % breathing room inside frame
  framexleftmargin=1ex,
  framesep=0.6ex,
  aboveskip=3pt,                        % tighter vertical rhythm
  belowskip=6pt
}

% -------------------- Story Card Box --------------------
\tcbset{
  colback=white,
  colframe=Primary,
  coltitle=white,
  fonttitle=\bfseries,
  sharp corners,
  boxrule=0.8pt,
  left=8pt,right=8pt,top=8pt,bottom=8pt,
  title filled,
  enhanced,
  breakable                              % allow page breaks inside cards
}

\newtcolorbox{StoryCard}[2][]{title={#2},#1}

% A compact label column for TabularX
\newcolumntype{L}{>{\raggedleft\arraybackslash\footnotesize\color{Meta}}p{0.22\linewidth}}
\newcolumntype{V}{>{\raggedright\arraybackslash}X}

% -------------------- Upgraded StoryCard helpers --------------------
% Consume 9 args and render a meta table like the original detailed cards:
% 1: ID   2: Title   3: Epic/Feature   4: Business Value
% 5: Priority   6: Estimate(SP)   7: Persona   8: Dependencies   9: Assumptions/Risks
\makeatletter
\@ifundefined{BeginCard}{
  \newcommand{\BeginCard}[9]{%
    \clearpage
    \begin{StoryCard}{\textbf{#1} ~|~ \textit{#2}}%
      \vspace{2pt}
      \begin{tabularx}{\linewidth}{L V}
        \textbf{Epic / Feature} & #3 \\
        \textbf{Business Value} & #4 \\
        \textbf{Priority} & #5 \quad \textbf{Estimate (SP):}~#6 \\
        \textbf{Persona} & #7 \\
        \textbf{Dependencies} & #8 \\
        \textbf{Assumptions / Risks} & #9 \\
      \end{tabularx}
      \vspace{6pt}
      {\footnotesize
        \textbf{Definition of Ready:} Persona clear; AC drafted; Dependencies known; Estimate set.\\
        \textbf{Definition of Done:} All ACs pass; Tests green; Security/observability checks; Docs updated; Evidence attached.
      }
      \vspace{4pt}
  }%
}{
  \renewcommand{\BeginCard}[9]{%
    \clearpage
    \begin{StoryCard}{\textbf{#1} ~|~ \textit{#2}}%
      \vspace{2pt}
      \begin{tabularx}{\linewidth}{L V}
        \textbf{Epic / Feature} & #3 \\
        \textbf{Business Value} & #4 \\
        \textbf{Priority} & #5 \quad \textbf{Estimate (SP):}~#6 \\
        \textbf{Persona} & #7 \\
        \textbf{Dependencies} & #8 \\
        \textbf{Assumptions / Risks} & #9 \\
      \end{tabularx}
      \vspace{6pt}
      {\footnotesize
        \textbf{Definition of Ready:} Persona clear; AC drafted; Dependencies known; Estimate set.\\
        \textbf{Definition of Done:} All ACs pass; Tests green; Security/observability checks; Docs updated; Evidence attached.
      }
      \vspace{4pt}
  }%
}
\makeatother
\providecommand{\EndCard}{\end{StoryCard}}

% -------------------- Tasks Box (list wrapper) --------------------
% Provides \begin{TasksBox} ... \item ... \end{TasksBox}
\newenvironment{TasksBox}[1][]{
  \begin{tcolorbox}[colback=Soft, colframe=Primary, title=\textbf{Tasks}, #1]
  \begin{itemize}[leftmargin=1.2em,itemsep=2pt,topsep=4pt]
}{
  \end{itemize}
  \end{tcolorbox}
}


% ===== Screenshot-style card + tasks styling patch (drop-in) =====
\usepackage{ragged2e}
% Badges/pills for metadata and NFR chips
\newtcbox{\pill}{on line, arc=3pt, boxsep=0.8pt, left=4pt,right=4pt,top=1pt,bottom=1pt,
  colframe=gray!50, colback=gray!15, boxrule=0.3pt}
\newcommand{\badge}[1]{\pill{\footnotesize #1}}

% Match screenshot look for tcolorbox defaults
\tcbset{
  colback=gray!2,
  colframe=gray!50,
  colbacktitle=gray!6,
  coltitle=black,
  fonttitle=\bfseries\large,
  arc=2pt,
  boxrule=0.4pt,
  left=8pt,right=8pt,top=8pt,bottom=8pt,
  enhanced,
  breakable,
  borderline west={2pt}{0pt}{draw=MidnightBlue}
}

% Robust two-column meta table (no tabularx dependency inside cards)
\newlength{\StoryLabelW}
\setlength{\StoryLabelW}{3.2cm}
\newlength{\StoryValueW}
\setlength{\StoryValueW}{\dimexpr\linewidth-\StoryLabelW-2\tabcolsep\relax}

% Re-style \BeginCard and \EndCard to match screenshot
\makeatletter
\renewcommand{\BeginCard}[9]{%
  \begin{StoryCard}{\textbf{#1}\ \textemdash\ #2}%
  \small
  \begin{tabular}{@{}>{\raggedleft\arraybackslash\bfseries}p{\StoryLabelW} >{\RaggedRight\arraybackslash}p{\StoryValueW}@{}}
    Epic / Feature          & #3 \\
    Business Value          & #4 \\
    Priority / Estimate     & \badge{Priority: #5}\ \badge{SP: #6} \\
    Persona                 & #7 \\
    Dependencies            & #8 \\
    Assumptions / Risks     & #9 \\
  \end{tabular}
  \medskip
  \textbf{Non-Functional}\quad
  \badge{Performance}\ \badge{Security}\ \badge{Reliability}\ \badge{Accessibility}\ \badge{Privacy}\ \badge{i18n}
  \medskip
}
\makeatother

\renewcommand{\EndCard}{%
  \vspace{0.2\baselineskip}
  {\footnotesize\color{gray!60}\textbf{Definition of Ready:} Persona clear; AC drafted; Dependencies known; Estimate set.\ \textbullet\ \textbf{Definition of Done:} All ACs pass; Tests green; Security/a11y checks; Docs updated; Deployed/flagged.}
  \end{StoryCard}
}

% Tasks box styling to match screenshot
\renewenvironment{TasksBox}[1][Tasks]{%
  \begin{tcolorbox}[
    enhanced,breakable,
    colback=gray!1, colframe=gray!35,
    colbacktitle=gray!6, coltitle=black,
    title={#1}, fonttitle=\bfseries,
    borderline west={1.8pt}{0pt}{draw=MidnightBlue},
    arc=2pt, boxrule=0.4pt,
    left=10pt,right=10pt,top=6pt,bottom=6pt,
    before skip=6pt, after skip=10pt
  ]
  \small
  \begin{itemize}[label=\checkbox, leftmargin=*, labelsep=0.6em, itemsep=4pt, topsep=2pt, parsep=0pt]
}{%
  \end{itemize}
  \end{tcolorbox}
}
% ===== End patch =====

\begin{document}
\begin{center}
  {\huge \textbf{Kubernetes Stories — Phased End-to-End Workflow}}\\[2pt]
  \textcolor{Meta}{Foundations → Visibility → Automation → Release → Scale → Security → Traffic → Depth → Advanced}\\[6pt]
\end{center}

% ============================================================
\section{Workflow Phases (Overview)}
% ============================================================
\subsection*{Foundations \& Dev Experience}
\begin{itemize}
  \item \textbf{KBP-01} — Basic service + Helm + ingress (hello world end-to-end).
  \item \textbf{KBP-02} — Developer workflows (namespaces, RBAC, inner loop).
  \item \textbf{KBP-04} — Config \& secrets baseline (separate config, non-secret vs secret).
\end{itemize}

\subsection*{Observability First (make issues visible early)}
\begin{itemize}
  \item \textbf{KBP-03} — Monitoring + logging (Prometheus, Grafana, Loki/EFK).
\end{itemize}

\subsection*{Ship Automatically}
\begin{itemize}
  \item \textbf{KBP-05} — CI pipeline to build/test/deploy.
  \item \textbf{KBP-18} — Adopt GitOps for deployments (pipeline hands off to GitOps).
\end{itemize}

\subsection*{Release Discipline}
\begin{itemize}
  \item \textbf{KBP-06} — Versioning, releases, rollout patterns (blue/green, canary).
\end{itemize}

\subsection*{Efficiency \& Stability}
\begin{itemize}
  \item \textbf{KBP-08} — Requests/limits, HPA, PDB (right-size + graceful disruptions).
  \item \textbf{KBP-20} — Resilience \& performance tests (chaos/load with SLO checks).
\end{itemize}

\subsection*{Security Baseline \& Guardrails}
\begin{itemize}
  \item \textbf{KBP-10} — Workload hardening (PSA, seccomp, non-root).
  \item \textbf{KBP-17} — Admission control \& authorization (API governance levers).
  \item \textbf{KBP-11} — Policy-as-code (Gatekeeper/Kyverno) to enforce standards.
  \item \textbf{KBP-19} — Holistic security posture (images, SBOMs, scanners, posture views).
\end{itemize}

\subsection*{Networking \& Traffic Control}
\begin{itemize}
  \item \textbf{KBP-09} — Networking hardening, Gateway API / service mesh (mTLS, traffic splits).
\end{itemize}

\subsection*{Scale Out \& Multi-Everything}
\begin{itemize}
  \item \textbf{KBP-07} — Multi-region staging \& controlled rollout (regional values/canaries).
  \item \textbf{KBP-12} — Multi-cluster management with GitOps (fleet-level practices).
\end{itemize}

\subsection*{App/Platform Depth}
\begin{itemize}
  \item \textbf{KBP-16} — Stateful services (storage, backups, operators).
  \item \textbf{KBP-13} — External integrations (safe service-to-service, egress controls).
  \item \textbf{KBP-15} — Higher-level app/platform patterns (golden paths, templates).
\end{itemize}

\subsection*{Advanced \& Extensibility}
\begin{itemize}
  \item \textbf{KBP-21} — Build a simple operator (Kubebuilder) to encode ops runbooks.
  \item \textbf{KBP-14} — ML inference on K8s (only after observability, security, scaling are in).
\end{itemize}

\subsection*{Wrap-Up \& Forward Plan}
\begin{itemize}
  \item \textbf{KBP-22} — Document conclusions \& next-90-day roadmap (what to double-down on).
\end{itemize}

% ============================================================
\section{Sequenced Story Index by Phase}
% ============================================================
% Map your actual story cards (K8S-xx) into these phases.
\subsection*{Foundations \& Dev Experience}
\begin{enumerate}
  \item \textbf{K8S-01} — Get a Local Cluster Running
  \item \textbf{K8S-02} — Provision Clusters (kubeadm + managed)
  \item \textbf{K8S-03} — Master \texttt{kubectl} Fundamentals
  \item \textbf{K8S-04} — Deploy Core Workloads
  \item \textbf{K8S-06} — Package with Helm \& Friends
  \item \textbf{K8S-07} — Govern with Namespaces/Quotas
  \item \textbf{K8S-08} — Persist \& Configure Safely
\end{enumerate}

\subsection*{Observability First}
\begin{enumerate}
  \item \textbf{K8S-11} — Observe Health \& Behavior (metrics/logs/traces, dashboards/alerts)
\end{enumerate}

\subsection*{Ship Automatically}
\begin{itemize}
  \item \emph{Add your CI \& GitOps stories here if present (e.g., KBP-05, KBP-18).}
\end{itemize}

\subsection*{Release Discipline}
\begin{enumerate}
  \item \textbf{K8S-05} — Expose Applications Reliably (Ingress/Gateway, TLS, rollout patterns)
\end{enumerate}

\subsection*{Efficiency \& Stability}
\begin{enumerate}
  \item \textbf{K8S-09} — Autoscale Workloads (HPA, right-size requests/limits)
  \item \textbf{K8S-12} — Diagnose \& Repair Fast (troubleshooting runbooks, SLO-driven fixes)
\end{enumerate}

\subsection*{Security Baseline \& Guardrails}
\begin{enumerate}
  \item \textbf{K8S-10} — Enforce Least Privilege (RBAC, securityContext, PSA/Pod Security)
\end{enumerate}

\subsection*{Networking \& Traffic Control}
\begin{enumerate}
  \item \textbf{K8S-05} — Expose Applications Reliably (applies here for traffic policy)
  \item \textbf{K8S-13} — Introduce Mesh Traffic Control (mTLS, retries, timeouts, splits)
\end{enumerate}

\subsection*{Scale Out \& Multi-Everything}
\begin{itemize}
  \item \emph{Add multi-region/multi-cluster GitOps stories here if present (e.g., KBP-07, KBP-12).}
\end{itemize}

\subsection*{App/Platform Depth}
\begin{enumerate}
  \item \textbf{K8S-14} — Scale-to-Zero with Knative (eventing/serving)
  \item \textbf{K8S-15} — Build/Extend the Platform (golden paths, templates)
\end{enumerate}

\subsection*{Advanced \& Extensibility}
\begin{itemize}
  \item \emph{Add operator/ML stories here if present (e.g., KBP-21, KBP-14).}
\end{itemize}

\subsection*{Wrap-Up \& Forward Plan}
\begin{itemize}
  \item \emph{Retrospective, platform scorecard, and next-90-day roadmap.}
\end{itemize}

% =======================
% Original Story Cards
% =======================

\begin{center}
  {\huge \textbf{Kubernetes Stories — End-to-End Workflow}}\\[2pt]
  \textcolor{Meta}{Sequenced, dependency-aware path from local lab to platform operation}\\[6pt]
\end{center}

\section*{Workflow at a Glance}
\begin{enumerate}
  \item K8S-01 — Get a Local Cluster Running
  \item K8S-02 — Provision Clusters (kubeadm + managed)
  \item K8S-03 — Master kubectl Fundamentals
  \item K8S-04 — Deploy Core Workloads
  \item K8S-05 — Expose Applications Reliably
  \item K8S-06 — Package with Helm \& Friends
  \item K8S-08 — Persist \& Configure Safely
  \item K8S-07 — Govern with Namespaces/Quotas
  \item K8S-10 — Enforce Least Privilege
  \item K8S-11 — Observe Health \& Behavior
  \item K8S-09 — Autoscale Workloads
  \item K8S-12 — Diagnose \& Repair Fast
  \item K8S-13 — Introduce Mesh Traffic Control
  \item K8S-14 — Scale-to-Zero with Knative
  \item K8S-15 — Build/Extend the Platform
\end{enumerate}
\clearpage

% --- Sequenced Stories (sourced from original, preserved content) ---

\section{Getting Started with Kubernetes}
\BeginCard{K8S-01}{Get a Local Cluster Running}{Kubernetes Basics}
{Establish a reproducible local lab to safely experiment and learn}
{Must}{3}
{developer}
{Docker, \texttt{kubectl}, \texttt{kind} or \texttt{minikube}}
{Resource constraints on laptop; network/proxy issues}
    % --- Auto-generated Acceptance Criteria for K8S-01 ---
    \textbf{Story:} As a \emph{developer}, I want to get a Local Cluster Running so that \emph{Establish a reproducible local lab to safely experiment and learn}.

    \textbf{Acceptance Criteria (BDD)}
    \begin{lstlisting}
    Scenario: Install \texttt{kubectl} and \texttt{kind} or \texttt{miniku
  Given Docker, \texttt{kubectl}, \texttt{kind} or \texttt{minikube}
  When Install \texttt{kubectl} and \texttt{kind} or \texttt{minikube}; verify \texttt{kubectl version} and context
  Then expected outcome is observable in logs/CLI/UI

Scenario: Create a cluster; enable metrics-server (minikube addon or Y
  Given Docker, \texttt{kubectl}, \texttt{kind} or \texttt{minikube}
  When Create a cluster; enable metrics-server (minikube addon or YAML)
  Then expected outcome is observable in logs/CLI/UI

Scenario: Deploy a sample Deployment + Service; confirm Pod readiness 
  Given Docker, \texttt{kubectl}, \texttt{kind} or \texttt{minikube}
  When Deploy a sample Deployment + Service; confirm Pod readiness and Service reachability
  Then expected outcome is observable in logs/CLI/UI
    \end{lstlisting}
\begin{TasksBox}
  \item Install \texttt{kubectl} and \texttt{kind} or \texttt{minikube}; verify \texttt{kubectl version} and context.
  \item Create a cluster; enable metrics-server (minikube addon or YAML).
  \item Deploy a sample Deployment + Service; confirm Pod readiness and Service reachability.
  \item Capture a cheatsheet of 20 \texttt{kubectl} commands in \texttt{/labs/ch01/README.md}.
\end{TasksBox}
\EndCard
\clearpage


\section{Creating a Kubernetes Cluster}
\BeginCard{K8S-02}{Provision Clusters (kubeadm + managed)}{Cluster Provisioning}
{Understand DIY vs. managed tradeoffs and create a repeatable runbook}
{Must}{5}
{platform engineer}
{Linux VMs, cloud account (GKE/EKS/AKS), CNI}
{Quota/permissions in cloud; VM CPU/mem limits}
    % --- Auto-generated Acceptance Criteria for K8S-02 ---
    \textbf{Story:} As a \emph{platform engineer}, I want to provision Clusters so that \emph{Understand DIY vs. managed tradeoffs and create a repeatable runbook}.

    \textbf{Acceptance Criteria (BDD)}
    \begin{lstlisting}
    Scenario: Bootstrap a single-control-plane cluster via \texttt{kubeadm
  Given Linux VMs, cloud account (GKE/EKS/AKS), CNI
  When Bootstrap a single-control-plane cluster via \texttt{kubeadm}; install a CNI
  Then expected outcome is observable in logs/CLI/UI

Scenario: Join a worker; validate node readiness; label/taint as neede
  Given Linux VMs, cloud account (GKE/EKS/AKS), CNI
  When Join a worker; validate node readiness; label/taint as needed
  Then expected outcome is observable in logs/CLI/UI

Scenario: Create one managed cluster (pick a cloud); install metrics-s
  Given Linux VMs, cloud account (GKE/EKS/AKS), CNI
  When Create one managed cluster (pick a cloud); install metrics-server \& Dashboard (protected)
  Then expected outcome is observable in logs/CLI/UI
    \end{lstlisting}
\begin{TasksBox}
  \item Bootstrap a single-control-plane cluster via \texttt{kubeadm}; install a CNI.
  \item Join a worker; validate node readiness; label/taint as needed.
  \item Create one managed cluster (pick a cloud); install metrics-server \& Dashboard (protected).
  \item Write a \texttt{create/destroy} runbook for both environments.
\end{TasksBox}
\EndCard
\clearpage

\section{Learning to Use the Kubernetes Client}

\BeginCard{K8S-03}{Master kubectl Fundamentals}{Developer Experience}
{Reduce MTTR and increase flow via fluent CLI usage}
{Must}{2}
{developer}
{Context/namespace helpers}
{Risk of destructive commands; use \texttt{--dry-run=client}}
    % --- Auto-generated Acceptance Criteria for K8S-03 ---
    \textbf{Story:} As a \emph{developer}, I want to master kubectl Fundamentals so that \emph{Reduce MTTR and increase flow via fluent CLI usage}.

    \textbf{Acceptance Criteria (BDD)}
    \begin{lstlisting}
    Scenario: Practice \texttt{get}, \texttt{describe}, \texttt{logs}, \te
  Given Context/namespace helpers
  When Practice \texttt{get}, \texttt{describe}, \texttt{logs}, \texttt{exec}, \texttt{delete --cascade}
  Then expected outcome is observable in logs/CLI/UI

Scenario: Use \texttt{kubectl explain} and JSONPath queries; export YA
  Given Context/namespace helpers
  When Use \texttt{kubectl explain} and JSONPath queries; export YAML via \texttt{-o yaml}
  Then expected outcome is observable in logs/CLI/UI

Scenario: Create namespace shortcuts (\texttt{kubens}) and context swi
  Given Context/namespace helpers
  When Create namespace shortcuts (\texttt{kubens}) and context switches
  Then expected outcome is observable in logs/CLI/UI
    \end{lstlisting}
\begin{TasksBox}
  \item Practice \texttt{get}, \texttt{describe}, \texttt{logs}, \texttt{exec}, \texttt{delete --cascade}.
  \item Use \texttt{kubectl explain} and JSONPath queries; export YAML via \texttt{-o yaml}.
  \item Create namespace shortcuts (\texttt{kubens}) and context switches.
\end{TasksBox}
\EndCard
\clearpage

\section{Creating and Modifying Fundamental Workloads}
\BeginCard{K8S-04}{Deploy Core Workloads}{Workload Primitives}
{Safely roll out, pause, and roll back application changes}
{Must}{3}
{app developer}
{Container image registry}
{Image pull limits; tag discipline}
    % --- Auto-generated Acceptance Criteria for K8S-04 ---
    \textbf{Story:} As a \emph{app developer}, I want to deploy Core Workloads so that \emph{Safely roll out, pause, and roll back application changes}.

    \textbf{Acceptance Criteria (BDD)}
    \begin{lstlisting}
    Scenario: Create Pod, Deployment (with rolling update), Job, CronJob, 
  Given Container image registry
  When Create Pod, Deployment (with rolling update), Job, CronJob, DaemonSet examples
  Then expected outcome is observable in logs/CLI/UI

Scenario: Trigger a rollout; verify \texttt{rollout status/history}; p
  Given Container image registry
  When Trigger a rollout; verify \texttt{rollout status/history}; perform rollback
  Then expected outcome is observable in logs/CLI/UI

Scenario: Add \texttt{readiness/liveness} probes to one Deployment
  Given Container image registry
  When Add \texttt{readiness/liveness} probes to one Deployment
  Then expected outcome is observable in logs/CLI/UI
    \end{lstlisting}
\begin{TasksBox}
  \item Create Pod, Deployment (with rolling update), Job, CronJob, DaemonSet examples.
  \item Trigger a rollout; verify \texttt{rollout status/history}; perform rollback.
  \item Add \texttt{readiness/liveness} probes to one Deployment.
\end{TasksBox}
\EndCard
\clearpage

\section{Working with Services}
\BeginCard{K8S-05}{Expose Applications Reliably}{Networking \& Discovery}
{Provide stable service discovery and ingress to users}
{Must}{3}
{application SRE}
{CoreDNS, Ingress controller}
{Ingress misconfig; path conflicts}
    % --- Auto-generated Acceptance Criteria for K8S-05 ---
    \textbf{Story:} As a \emph{application SRE}, I want to expose Applications Reliably so that \emph{Provide stable service discovery and ingress to users}.

    \textbf{Acceptance Criteria (BDD)}
    \begin{lstlisting}
    Scenario: Create ClusterIP/NodePort/LoadBalancer Services and compare
  Given CoreDNS, Ingress controller
  When Create ClusterIP/NodePort/LoadBalancer Services and compare
  Then expected outcome is observable in logs/CLI/UI

Scenario: Install NGINX Ingress; route \texttt{/} to an app; verify fr
  Given CoreDNS, Ingress controller
  When Install NGINX Ingress; route \texttt{/} to an app; verify from host
  Then expected outcome is observable in logs/CLI/UI

Scenario: Validate DNS inside Pods using \texttt{nslookup} or \texttt{
  Given CoreDNS, Ingress controller
  When Validate DNS inside Pods using \texttt{nslookup} or \texttt{dig}
  Then expected outcome is observable in logs/CLI/UI
    \end{lstlisting}
\begin{TasksBox}
  \item Create ClusterIP/NodePort/LoadBalancer Services and compare.
  \item Install NGINX Ingress; route \texttt{/} to an app; verify from host.
  \item Validate DNS inside Pods using \texttt{nslookup} or \texttt{dig}.
\end{TasksBox}
\EndCard
\clearpage

\section{Managing Application Manifests}
\BeginCard{K8S-06}{Package with Helm \& Friends}{Deployment Packaging}
{Enable repeatable, parameterized deployments across envs}
{Must}{5}
{platform engineer}
{Helm, optional: Kompose/Carvel}
{Values drift; document overrides}
    % --- Auto-generated Acceptance Criteria for K8S-06 ---
    \textbf{Story:} As a \emph{platform engineer}, I want to package with Helm \& Friends so that \emph{Enable repeatable, parameterized deployments across envs}.

    \textbf{Acceptance Criteria (BDD)}
    \begin{lstlisting}
    Scenario: Install Helm; deploy a public chart with custom \texttt{valu
  Given Helm, optional: Kompose/Carvel
  When Install Helm; deploy a public chart with custom \texttt{values.yaml}
  Then expected outcome is observable in logs/CLI/UI

Scenario: Convert a simple docker-compose app using \texttt{kompose}; 
  Given Helm, optional: Kompose/Carvel
  When Convert a simple docker-compose app using \texttt{kompose}; compare output
  Then expected outcome is observable in logs/CLI/UI

Scenario: Author a tiny Helm chart for your sample app; include Notes 
  Given Helm, optional: Kompose/Carvel
  When Author a tiny Helm chart for your sample app; include Notes and README
  Then expected outcome is observable in logs/CLI/UI
    \end{lstlisting}
\begin{TasksBox}
  \item Install Helm; deploy a public chart with custom \texttt{values.yaml}.
  \item Convert a simple docker-compose app using \texttt{kompose}; compare output.
  \item Author a tiny Helm chart for your sample app; include Notes and README.
\end{TasksBox}
\EndCard
\clearpage

\section{Volumes and Configuration Data}

\BeginCard{K8S-08}{Persist \& Configure Safely}{State \& Config}
{Separate config/secrets from code and preserve state across restarts}
{Must}{5}
{app developer}
{ConfigMap, Secret, PV/PVC}
{Secret sprawl; adopt rotation practices}
    % --- Auto-generated Acceptance Criteria for K8S-08 ---
    \textbf{Story:} As a \emph{app developer}, I want to persist \& Configure Safely so that \emph{Separate config/secrets from code and preserve state across restarts}.

    \textbf{Acceptance Criteria (BDD)}
    \begin{lstlisting}
    Scenario: Mount ConfigMap values; inject a Secret (env or volume)
  Given ConfigMap, Secret, PV/PVC
  When Mount ConfigMap values; inject a Secret (env or volume)
  Then expected outcome is observable in logs/CLI/UI

Scenario: Create a PVC; verify data survives Pod restarts
  Given ConfigMap, Secret, PV/PVC
  When Create a PVC; verify data survives Pod restarts
  Then expected outcome is observable in logs/CLI/UI

Scenario: Document secret handling (at-rest encryption, \texttt{.docke
  Given ConfigMap, Secret, PV/PVC
  When Document secret handling (at-rest encryption, \texttt{.dockerconfigjson}, \texttt{imagePullSecrets})
  Then expected outcome is observable in logs/CLI/UI
    \end{lstlisting}
\begin{TasksBox}
  \item Mount ConfigMap values; inject a Secret (env or volume).
  \item Create a PVC; verify data survives Pod restarts.
  \item Document secret handling (at-rest encryption, \texttt{.dockerconfigjson}, \texttt{imagePullSecrets}).
\end{TasksBox}
\EndCard
\clearpage

\section{Exploring the Kubernetes API and Key Metadata}
\BeginCard{K8S-07}{Govern with Namespaces/Quotas}{API \& Metadata}
{Constrain resource usage and organize multi-team tenancy}
{Should}{3}
{platform engineer}
{ResourceQuota, LimitRange}
{Overly strict quotas can block deploys}
    % --- Auto-generated Acceptance Criteria for K8S-07 ---
    \textbf{Story:} As a \emph{platform engineer}, I want to govern with Namespaces/Quotas so that \emph{Constrain resource usage and organize multi-team tenancy}.

    \textbf{Acceptance Criteria (BDD)}
    \begin{lstlisting}
    Scenario: List resources via \texttt{kubectl api-resources}; inspect v
  Given ResourceQuota, LimitRange
  When List resources via \texttt{kubectl api-resources}; inspect versions
  Then expected outcome is observable in logs/CLI/UI

Scenario: Create Namespaces with \texttt{ResourceQuota} and \texttt{Li
  Given ResourceQuota, LimitRange
  When Create Namespaces with \texttt{ResourceQuota} and \texttt{LimitRange}
  Then expected outcome is observable in logs/CLI/UI

Scenario: Label/annotate resources; query with selectors
  Given ResourceQuota, LimitRange
  When Label/annotate resources; query with selectors
  Then expected outcome is observable in logs/CLI/UI
    \end{lstlisting}
\begin{TasksBox}
  \item List resources via \texttt{kubectl api-resources}; inspect versions.
  \item Create Namespaces with \texttt{ResourceQuota} and \texttt{LimitRange}.
  \item Label/annotate resources; query with selectors.
\end{TasksBox}
\EndCard
\clearpage

\section{Security}
\BeginCard{K8S-10}{Enforce Least Privilege}{Platform Security}
{Reduce blast radius through RBAC and Pod hardening}
{Must}{5}
{security champion}
{ServiceAccount, Role/Binding, Pod Security Standards}
{Permissions confusion; validate with \texttt{can-i}}
    % --- Auto-generated Acceptance Criteria for K8S-10 ---
    \textbf{Story:} As a \emph{security champion}, I want to enforce Least Privilege so that \emph{Reduce blast radius through RBAC and Pod hardening}.

    \textbf{Acceptance Criteria (BDD)}
    \begin{lstlisting}
    Scenario: Create a dedicated ServiceAccount for an app; bind minimal R
  Given ServiceAccount, Role/Binding, Pod Security Standards
  When Create a dedicated ServiceAccount for an app; bind minimal Role
  Then expected outcome is observable in logs/CLI/UI

Scenario: Add \texttt{securityContext}: \texttt{runAsNonRoot}, \texttt
  Given ServiceAccount, Role/Binding, Pod Security Standards
  When Add \texttt{securityContext}: \texttt{runAsNonRoot}, \texttt{readOnlyRootFilesystem}, drop caps
  Then expected outcome is observable in logs/CLI/UI

Scenario: Apply Pod Security admission (baseline/restricted) at namesp
  Given ServiceAccount, Role/Binding, Pod Security Standards
  When Apply Pod Security admission (baseline/restricted) at namespace level
  Then expected outcome is observable in logs/CLI/UI
    \end{lstlisting}
\begin{TasksBox}
  \item Create a dedicated ServiceAccount for an app; bind minimal Role.
  \item Add \texttt{securityContext}: \texttt{runAsNonRoot}, \texttt{readOnlyRootFilesystem}, drop caps.
  \item Apply Pod Security admission (baseline/restricted) at namespace level.
\end{TasksBox}
\EndCard
\clearpage

\section{Monitoring and Logging}
\BeginCard{K8S-11}{Observe Health \& Behavior}{Observability}
{Shorten detection time with probes, metrics, and dashboards}
{Must}{5}
{SRE}
{Probes, kube-prometheus-stack/Grafana}
{Dashboard noise; focus on SLI panels}
    % --- Auto-generated Acceptance Criteria for K8S-11 ---
    \textbf{Story:} As a \emph{SRE}, I want to observe Health \& Behavior so that \emph{Shorten detection time with probes, metrics, and dashboards}.

    \textbf{Acceptance Criteria (BDD)}
    \begin{lstlisting}
    Scenario: Add liveness/readiness/startup probes; induce failures to se
  Given Probes, kube-prometheus-stack/Grafana
  When Add liveness/readiness/startup probes; induce failures to see effects
  Then expected outcome is observable in logs/CLI/UI

Scenario: Deploy Prometheus+Grafana on local cluster; import a simple 
  Given Probes, kube-prometheus-stack/Grafana
  When Deploy Prometheus+Grafana on local cluster; import a simple dashboard
  Then expected outcome is observable in logs/CLI/UI

Scenario: Collect and link logs for a failing Pod in \texttt{/labs/ch1
  Given Probes, kube-prometheus-stack/Grafana
  When Collect and link logs for a failing Pod in \texttt{/labs/ch11/README.md}
  Then expected outcome is observable in logs/CLI/UI
    \end{lstlisting}
\begin{TasksBox}
  \item Add liveness/readiness/startup probes; induce failures to see effects.
  \item Deploy Prometheus+Grafana on local cluster; import a simple dashboard.
  \item Collect and link logs for a failing Pod in \texttt{/labs/ch11/README.md}.
\end{TasksBox}
\EndCard
\clearpage

\section{Scaling}
\BeginCard{K8S-09}{Autoscale Workloads}{Capacity \& Efficiency}
{Match resources to demand and control costs}
{Should}{3}
{SRE}
{Metrics Server; optional Cluster Autoscaler}
{HPA signals noisy; smooth with requests/limits}
    % --- Auto-generated Acceptance Criteria for K8S-09 ---
    \textbf{Story:} As a \emph{SRE}, I want to autoscale Workloads so that \emph{Match resources to demand and control costs}.

    \textbf{Acceptance Criteria (BDD)}
    \begin{lstlisting}
    Scenario: Set CPU/memory requests and limits; run a small load test
  Given Metrics Server; optional Cluster Autoscaler
  When Set CPU/memory requests and limits; run a small load test
  Then expected outcome is observable in logs/CLI/UI

Scenario: Configure HPA; observe scale-out/back with \texttt{kubectl t
  Given Metrics Server; optional Cluster Autoscaler
  When Configure HPA; observe scale-out/back with \texttt{kubectl top}
  Then expected outcome is observable in logs/CLI/UI

Scenario: (Cloud) Enable Cluster Autoscaler; capture event timeline
  Given Metrics Server; optional Cluster Autoscaler
  When (Cloud) Enable Cluster Autoscaler; capture event timeline
  Then expected outcome is observable in logs/CLI/UI
    \end{lstlisting}
\begin{TasksBox}
  \item Set CPU/memory requests and limits; run a small load test.
  \item Configure HPA; observe scale-out/back with \texttt{kubectl top}.
  \item (Cloud) Enable Cluster Autoscaler; capture event timeline.
\end{TasksBox}
\EndCard
\clearpage

\section{Maintenance and Troubleshooting}
\BeginCard{K8S-12}{Diagnose \& Repair Fast}{Ops Readiness}
{Reduce MTTR with systematic debugging and safe maintenance}
{Must}{5}
{SRE}
{kubectl debug / drain}
{Node drains can disrupt; use PodDisruptionBudgets}
    % --- Auto-generated Acceptance Criteria for K8S-12 ---
    \textbf{Story:} As a \emph{SRE}, I want to diagnose \& Repair Fast so that \emph{Reduce MTTR with systematic debugging and safe maintenance}.

    \textbf{Acceptance Criteria (BDD)}
    \begin{lstlisting}
    Scenario: Reproduce common failures: CrashLoopBackOff, Pending PVC, Im
  Given kubectl debug / drain
  When Reproduce common failures: CrashLoopBackOff, Pending PVC, ImagePullBackOff
  Then expected outcome is observable in logs/CLI/UI

Scenario: Use \texttt{kubectl debug} or ephemeral containers to inspec
  Given kubectl debug / drain
  When Use \texttt{kubectl debug} or ephemeral containers to inspect
  Then expected outcome is observable in logs/CLI/UI

Scenario: Practice \texttt{cordon/drain/uncordon}; snapshot cluster st
  Given kubectl debug / drain
  When Practice \texttt{cordon/drain/uncordon}; snapshot cluster state
  Then expected outcome is observable in logs/CLI/UI
    \end{lstlisting}
\begin{TasksBox}
  \item Reproduce common failures: CrashLoopBackOff, Pending PVC, ImagePullBackOff.
  \item Use \texttt{kubectl debug} or ephemeral containers to inspect.
  \item Practice \texttt{cordon/drain/uncordon}; snapshot cluster state.
\end{TasksBox}
\EndCard
\clearpage

\section{Service Meshes}
\BeginCard{K8S-13}{Introduce Mesh Traffic Control}{Service Mesh}
{Gain mTLS, traffic shaping, and better service insights}
{Could}{5}
{platform engineer}
{Istio or Linkerd}
{Sidecar overhead; start small}
    % --- Auto-generated Acceptance Criteria for K8S-13 ---
    \textbf{Story:} As a \emph{platform engineer}, I want to introduce Mesh Traffic Control so that \emph{Gain mTLS, traffic shaping, and better service insights}.

    \textbf{Acceptance Criteria (BDD)}
    \begin{lstlisting}
    Scenario: Install Istio or Linkerd; enable automatic sidecar injection
  Given Istio or Linkerd
  When Install Istio or Linkerd; enable automatic sidecar injection
  Then expected outcome is observable in logs/CLI/UI

Scenario: Implement a canary (90/10 \textrightarrow{} 50/50 \textright
  Given Istio or Linkerd
  When Implement a canary (90/10 \textrightarrow{} 50/50 \textrightarrow{} 0/100); confirm mTLS
  Then expected outcome is observable in logs/CLI/UI

Scenario: Capture latency/error-rate before/after in notes
  Given Istio or Linkerd
  When Capture latency/error-rate before/after in notes
  Then expected outcome is observable in logs/CLI/UI
    \end{lstlisting}
\begin{TasksBox}
  \item Install Istio or Linkerd; enable automatic sidecar injection.
  \item Implement a canary (90/10 \textrightarrow{} 50/50 \textrightarrow{} 0/100); confirm mTLS.
  \item Capture latency/error-rate before/after in notes.
\end{TasksBox}
\EndCard
\clearpage

\section{Serverless and Event-Driven Applications}
\BeginCard{K8S-14}{Scale-to-Zero with Knative}{Serverless \& Events}
{Lower infra costs and simplify event plumbing}
{Could}{5}
{app developer}
{Knative Serving/Eventing; optional TriggerMesh}
{Cold starts; set expectations}
    % --- Auto-generated Acceptance Criteria for K8S-14 ---
    \textbf{Story:} As a \emph{app developer}, I want to scale-to-Zero with Knative so that \emph{Lower infra costs and simplify event plumbing}.

    \textbf{Acceptance Criteria (BDD)}
    \begin{lstlisting}
    Scenario: Install Knative; deploy a Knative Service; validate scale-to
  Given Knative Serving/Eventing; optional TriggerMesh
  When Install Knative; deploy a Knative Service; validate scale-to-zero
  Then expected outcome is observable in logs/CLI/UI

Scenario: Wire an event source \textrightarrow{} broker \textrightarro
  Given Knative Serving/Eventing; optional TriggerMesh
  When Wire an event source \textrightarrow{} broker \textrightarrow{} trigger \textrightarrow{} consumer
  Then expected outcome is observable in logs/CLI/UI

Scenario: Diagram the event flow and save with manifests
  Given Knative Serving/Eventing; optional TriggerMesh
  When Diagram the event flow and save with manifests
  Then expected outcome is observable in logs/CLI/UI
    \end{lstlisting}
\begin{TasksBox}
  \item Install Knative; deploy a Knative Service; validate scale-to-zero.
  \item Wire an event source \textrightarrow{} broker \textrightarrow{} trigger \textrightarrow{} consumer.
  \item Diagram the event flow and save with manifests.
\end{TasksBox}
\EndCard
\clearpage

\section{Extending Kubernetes}
\BeginCard{K8S-15}{Build/Extend the Platform}{Platform Extension}
{Tailor Kubernetes via clients, builds, and CRDs}
{Should}{5}
{platform engineer}
{Go/Python client; CRD scaffolding}
{API changes; pin versions}
    % --- Auto-generated Acceptance Criteria for K8S-15 ---
    \textbf{Story:} As a \emph{platform engineer}, I want to build/Extend the Platform so that \emph{Tailor Kubernetes via clients, builds, and CRDs}.

    \textbf{Acceptance Criteria (BDD)}
    \begin{lstlisting}
    Scenario: Compile \texttt{kubectl} locally or build a component from s
  Given Go/Python client; CRD scaffolding
  When Compile \texttt{kubectl} locally or build a component from source
  Then expected outcome is observable in logs/CLI/UI

Scenario: Write a short Python client that watches Pod events
  Given Go/Python client; CRD scaffolding
  When Write a short Python client that watches Pod events
  Then expected outcome is observable in logs/CLI/UI

Scenario: Define a simple CRD; create/list instances via \texttt{kubec
  Given Go/Python client; CRD scaffolding
  When Define a simple CRD; create/list instances via \texttt{kubectl}
  Then expected outcome is observable in logs/CLI/UI
    \end{lstlisting}
\begin{TasksBox}
  \item Compile \texttt{kubectl} locally or build a component from source.
  \item Write a short Python client that watches Pod events.
  \item Define a simple CRD; create/list instances via \texttt{kubectl}.
\end{TasksBox}

\EndCard
\end{document}
