
\documentclass[11pt,a4paper]{article}

% ---- Encoding & fonts ----
\usepackage[T1]{fontenc}
\usepackage[utf8]{inputenc}
\usepackage{lmodern}
\usepackage{microtype}

% ---- Layout & links ----
\usepackage[a4paper,margin=1in]{geometry}
\usepackage{hyperref}
\hypersetup{
  colorlinks=true,
  linkcolor=blue!60!black,
  urlcolor=blue!60!black,
  citecolor=blue!60!black,
  pdftitle={Study Plan — Computer Animation (Rick Parent, 3e)},
  pdfauthor={},
  pdfsubject={User Stories Template},
  pdfcreator={LaTeX}
}

% ---- Colors & symbols ----
\usepackage{xcolor}
\definecolor{cardframe}{HTML}{1F2937}   % gray-800
\definecolor{cardback}{HTML}{F8FAFC}    % slate-50
\definecolor{MidnightBlue}{HTML}{003366}% define to avoid missing color
\definecolor{TagGray}{HTML}{E5E7EB}
\definecolor{TagText}{HTML}{111827}

% ---- Tables & lists ----
\usepackage{array}
\usepackage{tabularx}
\newcolumntype{L}{>{\raggedright\arraybackslash}X}
\usepackage{enumitem}
\setlist{itemsep=2pt, topsep=4pt, leftmargin=1.1em}

% ---- Boxes ----
\usepackage[most]{tcolorbox}
\tcbuselibrary{skins, breakable, hooks}

% ---- Math / checkboxes ----
\usepackage{amssymb} % for \square

% ---- Tiny utility tags ----
\newtcbox{\Tag}{on line, arc=3pt, colback=TagGray, colframe=TagGray,
  boxsep=1pt, left=2pt, right=2pt, top=1pt, bottom=1pt, tcbox raise base,
  coltext=TagText}

% ---- Story card + tasks styles ----
\newtcolorbox{StoryCard}[2][]{%
  enhanced, breakable,
  colback=cardback, colframe=cardframe,
  title=\textbf{#2}, fonttitle=\bfseries,
  attach boxed title to top left={yshift=-2mm,xshift=2mm},
  boxed title style={colback=white, colframe=cardframe},
  left=2mm, right=2mm, top=2mm, bottom=2mm,
  sharp corners, % cleaner look
  #1
}

\newtcolorbox{TasksBox}[1][]{%
  enhanced, breakable,
  colback=white, colframe=cardframe,
  title=\textbf{Tasks}, fonttitle=\bfseries,
  left=2mm, right=2mm, top=2mm, bottom=2mm,
  #1
}

% ---- Reusable snippets ----
\newcommand{\DoR}{\textit{Definition of Ready:} Persona clear; AC drafted; Dependencies known; Estimate set.}
\newcommand{\DoD}{\textit{Definition of Done:} All ACs pass; Tests green; Security/a11y checks; Docs updated; Delivered/flagged.}

\newcommand{\NFtags}{\Tag{Performance} \Tag{Security} \Tag{Reliability} \Tag{Accessibility} \Tag{Privacy} \Tag{i18n}}

\newcommand{\ACBlock}[3]{%
\noindent\textbf{Acceptance Criteria (BDD)}\\
\textbf{Scenario}~#1\\
\textbf{Given}~#2\\
\textbf{When}~#3\\
\textbf{Then} the stated Outcomes/Deliverables for this chapter are produced and verifiable.
}

% ---- Title ----
\title{\textbf{Study Plan — Computer Animation (3rd ed.)}\\
\large User Story Template \& Chapter Cards}
\author{}
\date{\vspace{-0.5em}}

\begin{document}
\maketitle

\section*{How to Use This Document}
Each chapter of Rick Parent's \emph{Computer Animation, 3rd Edition} is represented as a \textbf{user story card}. Cards include the business value, persona, dependencies, acceptance criteria, and a concrete \textbf{Tasks} checklist. Duplicate a card when you need variants (e.g., an advanced path using C++/OpenGL instead of Python/Blender). Compile with \texttt{pdflatex} (no special packages beyond those in this file).

\subsection*{Required Data for a Good Story}
\begin{itemize}
  \item \textbf{ID \& Title} (e.g., \texttt{CA-03 --- Interpolating Values}).
  \item \textbf{Epic/Feature} the story rolls up to (e.g., ``Animation Core'').
  \item \textbf{Business Value} stated in stakeholder language.
  \item \textbf{Priority \& Estimate} (e.g., Must/Should + story points).
  \item \textbf{Persona} performing the work (e.g., ``technical animator'').
  \item \textbf{Dependencies} (tools, rigs, prior chapters).
  \item \textbf{Assumptions/Risks} that might affect scope or timing.
  \item \textbf{Story sentence}: \emph{As a \underline{persona}, I want \underline{capability} so that \underline{value}.}
  \item \textbf{Non-Functional tags} (performance, reliability, security, \dots).
  \item \textbf{Acceptance Criteria} in \emph{Given--When--Then} form.
  \item \textbf{Tasks} as a bite-size, checkable list.
\end{itemize}

\subsection*{User Story Template (Example)}
\begin{StoryCard}{TEMPLATE --- Write an Effective User Story}
\begin{tabularx}{\linewidth}{@{}lL@{}}
\textbf{Epic / Feature} & Production Foundations \\
\textbf{Business Value} & Shared understanding of scope and success criteria to reduce rework. \\
\textbf{Priority / Estimate} & \textbf{Priority:} Must \quad \textbf{SP:} 2 \\
\textbf{Persona} & developer on a new repo \\
\textbf{Dependencies} & build tooling, unit test framework \\
\textbf{Assumptions / Risks} & schedule risk if team lacks a common story format; ambiguity risk \\
\end{tabularx}

\medskip
\textbf{Story} \emph{As a developer, I want a consistent user story template so that the team can plan and verify work objectively.}

\medskip
\textbf{Non-Functional} \NFtags

\medskip
\ACBlock{Happy path}{the repository with this \LaTeX{} template is available}{the author completes the \emph{Tasks} below}

\medskip
{\footnotesize \DoR\quad\textbullet\quad\DoD}

\begin{TasksBox}
\begin{itemize}[label=$\square$, leftmargin=*, itemsep=2pt]
  \item Fill in ID/Title, Business Value, Persona, Dependencies, Assumptions/Risks.
  \item Draft the story sentence using ``As a \dots\ I want \dots\ so that \dots''.
  \item Write 1--3 \emph{Given--When--Then} acceptance criteria.
  \item Break work into 4--7 tasks (each 15--90 minutes).
  \item Review with a peer; commit the story card to the project docs.
\end{itemize}
\end{TasksBox}
\end{StoryCard}
\clearpage

%%%%%%%%%%%%%%%%%%%%%%%%%%%%%%%%%%%%%%%%%%%%%%%%%%%%%%%%%%%%%%%%%%%%%%%%%%%%%%%
%                            CHAPTER CARDS
%%%%%%%%%%%%%%%%%%%%%%%%%%%%%%%%%%%%%%%%%%%%%%%%%%%%%%%%%%%%%%%%%%%%%%%%%%%%%%%

% CHAPTER 1
\begin{StoryCard}{CA-01 --- Introduction}
\begin{tabularx}{\linewidth}{@{}lL@{}}
\textbf{Epic / Feature} & Orientation \& Pipeline \\
\textbf{Business Value} & Establish shared understanding of modern CG animation workflow and course outcomes. \\
\textbf{Priority / Estimate} & \textbf{Priority:} Must \quad \textbf{SP:} 2 \\
\textbf{Persona} & student/TD onboarding to the course plan \\
\textbf{Dependencies} & Blender or equivalent DCC; Python 3.x; Git repo for notes/clips \\
\textbf{Assumptions / Risks} & time-boxed to one week; risk of tool setup delays \\
\end{tabularx}

\medskip
\textbf{Story} \emph{As a learner, I want to map the CG animation pipeline and pick a final-shot concept so that weekly work aligns to a coherent end goal.}

\medskip
\textbf{Non-Functional} \NFtags

\medskip
\ACBlock{Happy path}{the toolchain installs successfully}{the learner produces a brief animatic and a pipeline diagram}

\medskip
{\footnotesize \DoR\quad\textbullet\quad\DoD}

\begin{TasksBox}
\begin{itemize}[label=$\square$, leftmargin=*, itemsep=2pt]
  \item Sketch the pipeline: assets $\rightarrow$ rig $\rightarrow$ animation $\rightarrow$ sim $\rightarrow$ lighting/render $\rightarrow$ comp.
  \item Create a 10--15s animatic (stepped keys or storyboard with timing).
  \item Set up project repo folders for \texttt{notes/}, \texttt{clips/}, \texttt{refs/}.
  \item Write risks \& constraints for your capstone shot (1 page).
\end{itemize}
\end{TasksBox}
\end{StoryCard}
\clearpage

% CHAPTER 2
\begin{StoryCard}{CA-02 --- Technical Background}
\begin{tabularx}{\linewidth}{@{}lL@{}}
\textbf{Epic / Feature} & Math \& Transforms \\
\textbf{Business Value} & Reliable transforms/orientations prevent gimbal issues and rig instability. \\
\textbf{Priority / Estimate} & \textbf{Priority:} Must \quad \textbf{SP:} 3 \\
\textbf{Persona} & technical animator / TD \\
\textbf{Dependencies} & Python notebooks or C++ utility library; test meshes \\
\textbf{Assumptions / Risks} & numeric instability if conventions (handedness, units) are inconsistent \\
\end{tabularx}

\medskip
\textbf{Story} \emph{As a TD, I want robust transform/orientation utilities so that rigs and cameras behave predictably.}

\medskip
\textbf{Non-Functional} \NFtags

\medskip
\ACBlock{Happy path}{reference tests are available}{matrix/quaternion conversions and parent/child transforms pass unit tests}

\medskip
{\footnotesize \DoR\quad\textbullet\quad\DoD}

\begin{TasksBox}
\begin{itemize}[label=$\square$]
  \item Implement 4x4 homogeneous transforms; verify inverse and composition.
  \item Implement quaternion $\leftrightarrow$ matrix / axis-angle; add unit tests.
  \item Build a demo rig (2-bone chain) to visualize local vs world transforms.
  \item Document conventions (axes, degrees/radians, units).
\end{itemize}
\end{TasksBox}
\end{StoryCard}
\clearpage

% CHAPTER 3
\begin{StoryCard}{CA-03 --- Interpolating Values}
\begin{tabularx}{\linewidth}{@{}lL@{}}
\textbf{Epic / Feature} & Curves \& Timing \\
\textbf{Business Value} & Smooth, controllable motion and camera paths with consistent speed profiling. \\
\textbf{Priority / Estimate} & \textbf{Priority:} Must \quad \textbf{SP:} 3 \\
\textbf{Persona} & animator / tools engineer \\
\textbf{Dependencies} & curve evaluation utilities; plotting \\
\textbf{Assumptions / Risks} & time slippage from arc-length reparametrization if not cached \\
\end{tabularx}

\medskip
\textbf{Story} \emph{As an animator, I want spline and orientation interpolation so that paths and rotations are smooth and predictable.}

\medskip
\textbf{Non-Functional} \NFtags

\medskip
\ACBlock{Happy path}{test paths and keyframes exist}{constant-speed motion along a spline and correct SLERP orientation}

\medskip
{\footnotesize \DoR\quad\textbullet\quad\DoD}

\begin{TasksBox}
\begin{itemize}[label=$\square$]
  \item Implement Cubic Hermite, Catmull--Rom, and B-spline evaluation.
  \item Implement SLERP; compare with normalized LERP (error plot).
  \item Reparametrize a path by arc length; demonstrate constant-speed fly-through.
  \item Render a 5s camera move before/after reparam (side-by-side).
\end{itemize}
\end{TasksBox}
\end{StoryCard}
\clearpage

% CHAPTER 4
\begin{StoryCard}{CA-04 --- Interpolation-Based Animation}
\begin{tabularx}{\linewidth}{@{}lL@{}}
\textbf{Epic / Feature} & Keyframing \& Shape Interp \\
\textbf{Business Value} & Artist-friendly controls for timing and deformations. \\
\textbf{Priority / Estimate} & \textbf{Priority:} Should \quad \textbf{SP:} 3 \\
\textbf{Persona} & animator \\
\textbf{Dependencies} & blendshape targets; keyframe editor \\
\textbf{Assumptions / Risks} & topology mismatch breaks shape interpolation \\
\end{tabularx}

\medskip
\textbf{Story} \emph{As an animator, I want a keyframe editor and blendshape mixer so that I can sculpt timing and shape changes.}

\medskip
\textbf{Non-Functional} \NFtags

\medskip
\ACBlock{Happy path}{targets and rig are available}{morph and keyed timing match the reference beat sheet}

\medskip
{\footnotesize \DoR\quad\textbullet\quad\DoD}

\begin{TasksBox}
\begin{itemize}[label=$\square$]
  \item Build a mini keyframe editor (stepped/linear/bezier tangents).
  \item Create 3--5 blendshape targets; implement normalized weight mixing.
  \item Animate a 10s morph sequence with clean in-betweens.
  \item Export a turntable clip of neutral vs extreme shapes.
\end{itemize}
\end{TasksBox}
\end{StoryCard}
\clearpage

% CHAPTER 5
\begin{StoryCard}{CA-05 --- Kinematic Linkages}
\begin{tabularx}{\linewidth}{@{}lL@{}}
\textbf{Epic / Feature} & FK/IK Rigs \\
\textbf{Business Value} & Fast posing with constraints to reduce foot sliding and joint breakage. \\
\textbf{Priority / Estimate} & \textbf{Priority:} Must \quad \textbf{SP:} 4 \\
\textbf{Persona} & rigger / animator \\
\textbf{Dependencies} & test character; solver utilities \\
\textbf{Assumptions / Risks} & solver divergence near singularities \\
\end{tabularx}

\medskip
\textbf{Story} \emph{As a rigger, I want stable FK/IK with joint limits so that animators can reach targets without artifacts.}

\medskip
\textbf{Non-Functional} \NFtags

\medskip
\ACBlock{Happy path}{rig joint limits are set}{IK solver reaches target within tolerance and without popping}

\medskip
{\footnotesize \DoR\quad\textbullet\quad\DoD}

\begin{TasksBox}
\begin{itemize}[label=$\square$]
  \item Implement CCD and Jacobian-transpose IK; log iteration counts.
  \item Add joint limits, pole vector, preferred angles.
  \item Animate a 5--10s reach-and-grasp on a moving object.
  \item Plot end-effector error over time; ensure monotonic convergence.
\end{itemize}
\end{TasksBox}
\end{StoryCard}
\clearpage

% CHAPTER 6
\begin{StoryCard}{CA-06 --- Motion Capture}
\begin{tabularx}{\linewidth}{@{}lL@{}}
\textbf{Epic / Feature} & MoCap Pipeline \\
\textbf{Business Value} & High-fidelity base motion retargeted to house rigs with minimal cleanup. \\
\textbf{Priority / Estimate} & \textbf{Priority:} Should \quad \textbf{SP:} 4 \\
\textbf{Persona} & motion TD \\
\textbf{Dependencies} & BVH/FBX clips; OpenCV; retarget tool \\
\textbf{Assumptions / Risks} & foot drift and scale mismatches require cleanup \\
\end{tabularx}

\medskip
\textbf{Story} \emph{As a motion TD, I want to retarget and blend MoCap so that I can quickly block complex performances.}

\medskip
\textbf{Non-Functional} \NFtags

\medskip
\ACBlock{Happy path}{calibration data and clips exist}{retargeted animation passes foot-contact checks and timing constraints}

\medskip
{\footnotesize \DoR\quad\textbullet\quad\DoD}

\begin{TasksBox}
\begin{itemize}[label=$\square$]
  \item Calibrate camera(s) and reconstruct a simple 3D point set.
  \item Retarget BVH to course rig; fix contacts with constraints.
  \item Blend two clips; add time-warp to match beats.
  \item Produce an 8--12s walk-to-reach composite before/after cleanup.
\end{itemize}
\end{TasksBox}
\end{StoryCard}
\clearpage

% CHAPTER 7
\begin{StoryCard}{CA-07 --- Physically Based Animation}
\begin{tabularx}{\linewidth}{@{}lL@{}}
\textbf{Epic / Feature} & Particles, Rigid, Cloth \\
\textbf{Business Value} & Realistic secondary motion increases production value. \\
\textbf{Priority / Estimate} & \textbf{Priority:} Must \quad \textbf{SP:} 5 \\
\textbf{Persona} & VFX TD \\
\textbf{Dependencies} & physics integrators; collision library \\
\textbf{Assumptions / Risks} & instability with large time steps; tuning time \\
\end{tabularx}

\medskip
\textbf{Story} \emph{As a VFX TD, I want stable particle/rigid/cloth sims so that shots look physically plausible.}

\medskip
\textbf{Non-Functional} \NFtags

\medskip
\ACBlock{Happy path}{collision proxies exist}{sims run without explosion and meet timing budgets}

\medskip
{\footnotesize \DoR\quad\textbullet\quad\DoD}

\begin{TasksBox}
\begin{itemize}[label=$\square$]
  \item Implement particle forces and emitters; add lifetime and randomness.
  \item Build mass--spring cloth; compare explicit vs semi-implicit integration.
  \item Add rigid bodies with impulse collisions (restitution, friction).
  \item Render a 10s composite: ball hits boxes; cloth banner reacts.
\end{itemize}
\end{TasksBox}
\end{StoryCard}
\clearpage

% CHAPTER 8
\begin{StoryCard}{CA-08 --- Fluids: Liquids and Gases}
\begin{tabularx}{\linewidth}{@{}lL@{}}
\textbf{Epic / Feature} & Fluid Effects \\
\textbf{Business Value} & Believable smoke/water interactions for hero shots. \\
\textbf{Priority / Estimate} & \textbf{Priority:} Should \quad \textbf{SP:} 5 \\
\textbf{Persona} & FX artist / TD \\
\textbf{Dependencies} & grid solver (2D); SPH prototype; render volumes \\
\textbf{Assumptions / Risks} & cost constraints on resolution/time; coupling to colliders \\
\end{tabularx}

\medskip
\textbf{Story} \emph{As an FX TD, I want smoke and liquid sims so that I can art-direct turbulent motion efficiently.}

\medskip
\textbf{Non-Functional} \NFtags

\medskip
\ACBlock{Happy path}{domain and sources are set}{smoke shows vorticity confinement and liquid shows plausible splashes}

\medskip
{\footnotesize \DoR\quad\textbullet\quad\DoD}

\begin{TasksBox}
\begin{itemize}[label=$\square$]
  \item Implement a 2D stable fluids grid (advect, diffuse, project).
  \item Prototype SPH for splashes; tune kernel radius/viscosity.
  \item Add collider coupling; emitters for smoke and pour.
  \item Render an 8--10s flipbook comparing parameter sweeps.
\end{itemize}
\end{TasksBox}
\end{StoryCard}
\clearpage

% CHAPTER 9
\begin{StoryCard}{CA-09 --- Modeling \& Animating Human Figures}
\begin{tabularx}{\linewidth}{@{}lL@{}}
\textbf{Epic / Feature} & Virtual Humans \\
\textbf{Business Value} & Solid deformations and locomotion for character shots. \\
\textbf{Priority / Estimate} & \textbf{Priority:} Must \quad \textbf{SP:} 5 \\
\textbf{Persona} & character TD \\
\textbf{Dependencies} & skinned mesh; skin weights; terrain asset \\
\textbf{Assumptions / Risks} & skinning artifacts at joints; foot sliding on uneven terrain \\
\end{tabularx}

\medskip
\textbf{Story} \emph{As a character TD, I want robust skinning and a walk cycle so that human motion reads believably.}

\medskip
\textbf{Non-Functional} \NFtags

\medskip
\ACBlock{Happy path}{weights and controls are defined}{walk cycle maintains COM over support polygon; no interpenetration}

\medskip
{\footnotesize \DoR\quad\textbullet\quad\DoD}

\begin{TasksBox}
\begin{itemize}[label=$\square$]
  \item Paint skin weights; test extreme poses; fix elbow/shoulder artifacts.
  \item Animate a gait cycle with contact and passing phases.
  \item Add simple garment (shirt/skirt) interacting with body.
  \item Render a 10s walk over uneven terrain; measure stride and cadence.
\end{itemize}
\end{TasksBox}
\end{StoryCard}
\clearpage

% CHAPTER 10
\begin{StoryCard}{CA-10 --- Facial Animation}
\begin{tabularx}{\linewidth}{@{}lL@{}}
\textbf{Epic / Feature} & Face Rig \& Lip-Sync \\
\textbf{Business Value} & Expressive dialogue and emotions for storytelling. \\
\textbf{Priority / Estimate} & \textbf{Priority:} Should \quad \textbf{SP:} 4 \\
\textbf{Persona} & facial rigger / animator \\
\textbf{Dependencies} & blendshapes or FACS AUs; audio clip \\
\textbf{Assumptions / Risks} & coarticulation timing; uncanny valley risk if eyes/eyelids misbehave \\
\end{tabularx}

\medskip
\textbf{Story} \emph{As a facial animator, I want a viseme-driven rig so that lip-sync and expressions feel natural.}

\medskip
\textbf{Non-Functional} \NFtags

\medskip
\ACBlock{Happy path}{audio and transcript exist}{lip closures on labials are correct; eye focus is consistent; no popping}

\medskip
{\footnotesize \DoR\quad\textbullet\quad\DoD}

\begin{TasksBox}
\begin{itemize}[label=$\square$]
  \item Build facial controls (brows, lids, lips, jaw); map visemes to shapes.
  \item Time-align phonemes; add coarticulation smoothing.
  \item Animate 10--15s dialogue with micro-motions (eye saccades, blinks).
  \item Export with audio waveform overlay for review.
\end{itemize}
\end{TasksBox}
\end{StoryCard}
\clearpage

% CHAPTER 11
\begin{StoryCard}{CA-11 --- Behavioral Animation}
\begin{tabularx}{\linewidth}{@{}lL@{}}
\textbf{Epic / Feature} & Agents \& Crowds \\
\textbf{Business Value} & Scalable background motion and intelligent navigation. \\
\textbf{Priority / Estimate} & \textbf{Priority:} Could \quad \textbf{SP:} 4 \\
\textbf{Persona} & gameplay/AI engineer \\
\textbf{Dependencies} & navmesh; pathfinding; steering behaviors \\
\textbf{Assumptions / Risks} & congestion at bottlenecks; performance drops with agent count \\
\end{tabularx}

\medskip
\textbf{Story} \emph{As an AI engineer, I want steering and pathfinding so that crowds navigate scenes convincingly.}

\medskip
\textbf{Non-Functional} \NFtags

\medskip
\ACBlock{Happy path}{map and obstacles are defined}{agents reach goals with low collision rate and stable frame time}

\medskip
{\footnotesize \DoR\quad\textbullet\quad\DoD}

\begin{TasksBox}
\begin{itemize}[label=$\square$]
  \item Implement seek/arrive/wander and obstacle avoidance; blend behaviors.
  \item Build a navmesh; pathfind with A*; add local avoidance.
  \item Simulate 50--200 agents through a choke point; measure throughput.
  \item Render a 15s crowd flow; record collision metrics.
\end{itemize}
\end{TasksBox}
\end{StoryCard}
\clearpage

% CHAPTER 12
\begin{StoryCard}{CA-12 --- Special Models for Animation}
\begin{tabularx}{\linewidth}{@{}lL@{}}
\textbf{Epic / Feature} & Procedural Models \\
\textbf{Business Value} & Quickly generate complex detail (plants, blobs, smooth surfaces). \\
\textbf{Priority / Estimate} & \textbf{Priority:} Could \quad \textbf{SP:} 4 \\
\textbf{Persona} & look-dev / TD \\
\textbf{Dependencies} & metaball/implicit surface tools; L-system module \\
\textbf{Assumptions / Risks} & temporal coherence during deformation; performance \\
\end{tabularx}

\medskip
\textbf{Story} \emph{As a look-dev TD, I want implicit surfaces and plant growth so that I can generate rich motion without manual modeling.}

\medskip
\textbf{Non-Functional} \NFtags

\medskip
\ACBlock{Happy path}{procedural modules are wired}{garden growth is temporally coherent and controllable}

\medskip
{\footnotesize \DoR\quad\textbullet\quad\DoD}

\begin{TasksBox}
\begin{itemize}[label=$\square$]
  \item Implement metaballs and preview via marching cubes/squares.
  \item Build an L-system with stochastic rules; add wind sway.
  \item Apply subdivision (Catmull--Clark/Loop) with crease control.
  \item Render a 10s procedural garden pullback (wireframe + shaded passes).
\end{itemize}
\end{TasksBox}
\end{StoryCard}
\clearpage

% APPENDIX A
\begin{StoryCard}{CA-A --- Rendering Issues (Integration Week)}
\begin{tabularx}{\linewidth}{@{}lL@{}}
\textbf{Epic / Feature} & Rendering \& Delivery \\
\textbf{Business Value} & Predictable render times and delivery-ready media. \\
\textbf{Priority / Estimate} & \textbf{Priority:} Must \quad \textbf{SP:} 2 \\
\textbf{Persona} & lighting/compositing TD \\
\textbf{Dependencies} & render farm or local batch; FFmpeg \\
\textbf{Assumptions / Risks} & render time vs quality trade-offs \\
\end{tabularx}

\medskip
\textbf{Story} \emph{As a lighting TD, I want sampling/motion-blur controls so that finals balance quality and cost.}

\medskip
\textbf{Non-Functional} \NFtags

\medskip
\ACBlock{Happy path}{scenes are renderable}{finals and preview renders differ only in controlled quality parameters}

\medskip
{\footnotesize \DoR\quad\textbullet\quad\DoD}

\begin{TasksBox}
\begin{itemize}[label=$\square$]
  \item Compare sample counts and reconstruction filters; enable motion blur.
  \item Batch-render best shots at ``preview'' and ``final'' settings; log timings.
  \item Package output with correct color space and bitrate using FFmpeg.
\end{itemize}
\end{TasksBox}
\end{StoryCard}
\clearpage

% APPENDIX B
\begin{StoryCard}{CA-B --- Background Math \& Techniques (Companion)}
\begin{tabularx}{\linewidth}{@{}lL@{}}
\textbf{Epic / Feature} & Math Companion \\
\textbf{Business Value} & Faster debugging and verifiable numerics across chapters. \\
\textbf{Priority / Estimate} & \textbf{Priority:} Should \quad \textbf{SP:} 2 \\
\textbf{Persona} & TD / engineer \\
\textbf{Dependencies} & Jupyter or C++ test harness \\
\textbf{Assumptions / Risks} & overfitting tests to specific scenes \\
\end{tabularx}

\medskip
\textbf{Story} \emph{As a TD, I want a math/test companion so that I can validate formulas and catch regressions quickly.}

\medskip
\textbf{Non-Functional} \NFtags

\medskip
\ACBlock{Happy path}{reference equations are captured}{unit tests for each formula pass and are linked to chapter cards}

\medskip
{\footnotesize \DoR\quad\textbullet\quad\DoD}

\begin{TasksBox}
\begin{itemize}[label=$\square$]
  \item Create a notebook per chapter; derive and validate key equations.
  \item Add finite-difference checks for Jacobians and gradients.
  \item Track assumptions (units, handedness, conventions) alongside tests.
\end{itemize}
\end{TasksBox}
\end{StoryCard}
\clearpage

\end{document}
