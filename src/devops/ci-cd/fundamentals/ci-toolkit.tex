
%========================================================
% Continuous Integration Toolkit (GitHub-first, Minted Edition + C/C++)
%========================================================
\documentclass[11pt]{article}

% ---------- Encoding & layout ----------
\usepackage[T1]{fontenc}
\usepackage[utf8]{inputenc}
\usepackage{lmodern}
\usepackage{geometry}
\geometry{margin=1in}
\usepackage{microtype}
\usepackage{setspace}
\setstretch{1.12}

% ---------- Colors, links, boxes ----------
\usepackage{xcolor}
\definecolor{ink}{HTML}{111827}      % gray-900
\definecolor{soft}{HTML}{F9FAFB}     % gray-50
\definecolor{accent}{HTML}{2563EB}   % blue-600
\definecolor{ok}{HTML}{059669}       % emerald-600
\definecolor{warn}{HTML}{D97706}     % amber-600
\definecolor{bad}{HTML}{DC2626}      % red-600
\definecolor{muted}{HTML}{6B7280}    % gray-500

\usepackage[hidelinks]{hyperref}
\hypersetup{
  colorlinks=true,
  linkcolor=accent,
  urlcolor=accent,
  citecolor=accent
}

\usepackage{titlesec}
\titleformat{\section}{\large\bfseries\color{ink}}{\thesection}{0.5em}{}
\titleformat{\subsection}{\bfseries\color{ink}}{\thesubsection}{0.5em}{}
\titleformat{\subsubsection}{\itshape\color{ink}}{\thesubsubsection}{0.5em}{}

\usepackage{tcolorbox}
\tcbset{colframe=ink, colback=soft, boxrule=0.6pt, sharp corners, arc=2pt}

% ---------- Minted (requires -shell-escape) ----------
\usepackage[cache=false]{minted} % no ; avoid style file errors
\usemintedstyle{friendly}
\setminted{
  breaklines=true,
  fontsize=\small,
  linenos=true,
  frame=single,
  framesep=6pt
}

% ---------- Title ----------
\title{\textbf{Continuous Integration Toolkit}\\\large GitHub-first (Minted Edition) with C and C++}
\author{}
\date{\today}

\begin{document}
\maketitle
\tableofcontents
\newpage

\begin{tcolorbox}
\textbf{How to build.} This document uses \texttt{minted}, which calls \texttt{pygmentize}. Compile with:
\begin{minted}[breaklines=false,linenos=false]{bash}
latexmk -pdf -shell-escape ci_toolkit_minted_ccpp.tex
\end{minted}
If you cannot enable \texttt{-shell-escape}, I can ship a \textit{} variant with pre-generated styles.
\end{tcolorbox}

\section{Philosophy \& Goals}
\begin{itemize}
  \item \textbf{Fast feedback:} run on every push/PR; show coverage and lint results in the PR.
  \item \textbf{Quality gates:} enforce coverage thresholds and required checks before merge.
  \item \textbf{Secure by default:} enable CodeQL and Secret Scanning; keep dependencies healthy.
  \item \textbf{Simple to adopt:} copy-paste scaffolds for Node, Python, Java, C, and C\texttt{++}.
  \item \textbf{Maintainable:} reusable workflows, caching, and matrices.
\end{itemize}

% =============================
% Node + Jest
% =============================
\section{Scaffold: Node + Jest}
\paragraph{Init project}
\begin{minted}{bash}
npm init -y
npm i -D jest
mkdir -p src __tests__
\end{minted}

\paragraph{src/appOperations.js}
\begin{minted}{javascript}
function multiply(a, b) { return a * b; }
function add(a, b) { return a + b; }
function subtract(a, b) { return a - b; }
module.exports = { multiply, add, subtract };
\end{minted}
\clearpage

\paragraph{\texttt{\_\_tests\_\_/appOperations.test.js}}
\begin{minted}{javascript}
const { multiply, add, subtract } = require('../src/appOperations');

test('multiply 5 x 0 = 0', () => {
  expect(multiply(5, 0)).toBe(0);
});

test('add 5 + 5 = 10', () => {
  expect(add(5, 5)).toBe(10);
});

test('subtract 15 - 5 = 10', () => {
  expect(subtract(15, 5)).toBe(10);
});
\end{minted}

\paragraph{package.json (scripts)}
\begin{minted}{json}
{
  "scripts": {
    "test": "jest --coverage"
  }
}
\end{minted}

\paragraph{jest.config.js}
\begin{minted}{javascript}
module.exports = {
  testEnvironment: 'node',
  collectCoverage: true,
  coverageThreshold: {
    global: { lines: 80, branches: 80, functions: 80, statements: 80 }
  }
};
\end{minted}
\clearpage

\section{CI Workflow: Tests \& Coverage on PRs}
\paragraph{.github/workflows/ci.yml}
\begin{minted}{yaml}
name: Unit tests & coverage

on:
  pull_request:
    branches: [ main ]
  push:
    branches: [ main ]

jobs:
  test:
    runs-on: ubuntu-latest
    permissions:
      contents: write
      checks: write
      pull-requests: write
    concurrency:
      group: ci-${{ github.ref }}
      cancel-in-progress: true
    steps:
      - uses: actions/checkout@v4
        with:
          fetch-depth: 1
      - uses: actions/setup-node@v4
        with:
          node-version: '20'
          cache: 'npm'
      - run: npm ci
      - run: npm test -- --coverage
      - name: Publish Jest coverage to PR
        uses: ArtiomTr/jest-coverage-report-action@v2
        with:
          github-token: ${{ secrets.GITHUB_TOKEN }}
      - name: Upload coverage artifact
        uses: actions/upload-artifact@v4
        with:
          name: coverage-${{ github.sha }}
          path: coverage
          retention-days: 7
\end{minted}

\begin{tcolorbox}
\textbf{Why this matters.} The coverage thresholds in \texttt{jest.config.js} enforce a minimum quality gate. The action posts a coverage summary on the PR so reviewers can spot regressions quickly.
\end{tcolorbox}

% =============================
% Python + pytest
% =============================
\section{Python Variant (pytest)}
\paragraph{requirements.txt}
\begin{minted}{text}
pytest
pytest-cov
\end{minted}

\paragraph{app.py}
\begin{minted}{python}
def multiply(a, b):
    return a * b

def add(a, b):
    return a + b

def subtract(a, b):
    return a - b
\end{minted}

\paragraph{tests/test\_app.py}
\begin{minted}{python}
from app import multiply, add, subtract

def test_multiply():
    assert multiply(5, 0) == 0

def test_add():
    assert add(5, 5) == 10

def test_subtract():
    assert subtract(15, 5) == 10
\end{minted}

\paragraph{pytest.ini}
\begin{minted}{ini}
[pytest]
addopts = -q --cov=. --cov-report=term-missing
\end{minted}
\clearpage

\paragraph{.github/workflows/ci-python.yml}
\begin{minted}{yaml}
name: Pytests

on:
  pull_request:
  push:
    branches: [ main ]

jobs:
  test:
    runs-on: ubuntu-latest
    steps:
      - uses: actions/checkout@v4
      - uses: actions/setup-python@v5
        with:
          python-version: "3.12"
          cache: "pip"
      - run: pip install -r requirements.txt
      - run: pytest
      - name: Upload coverage XML (optional)
        run: pytest --cov=. --cov-report=xml:coverage.xml
      - uses: actions/upload-artifact@v4
        with:
          name: coverage-xml
          path: coverage.xml
\end{minted}
\clearpage

% =============================
% Java + Maven
% =============================
\section{Java Variant (Maven + JaCoCo)}
\paragraph{pom.xml (snippets)}
\begin{minted}{xml}
<project>
  <properties>
    <maven.compiler.source>21</maven.compiler.source>
    <maven.compiler.target>21</maven.compiler.target>
  </properties>
  <build>
    <plugins>
      <plugin>
        <groupId>org.apache.maven.plugins</groupId>
        <artifactId>maven-surefire-plugin</artifactId>
        <version>3.2.5</version>
      </plugin>
      <plugin>
        <groupId>org.jacoco</groupId>
        <artifactId>jacoco-maven-plugin</artifactId>
        <version>0.8.11</version>
        <executions>
          <execution>
            <goals><goal>prepare-agent</goal></goals>
          </execution>
          <execution>
            <id>report</id>
            <phase>test</phase>
            <goals><goal>report</goal></goals>
          </execution>
        </executions>
      </plugin>
    </plugins>
  </build>
</project>
\end{minted}
\clearpage

\paragraph{.github/workflows/ci-java.yml}
\begin{minted}{yaml}
name: Maven tests

on:
  pull_request:
  push:
    branches: [ main ]

jobs:
  test:
    runs-on: ubuntu-latest
    steps:
      - uses: actions/checkout@v4
      - uses: actions/setup-java@v4
        with:
          distribution: temurin
          java-version: "21"
          cache: maven
      - run: mvn -B -DskipTests=false test
      - uses: actions/upload-artifact@v4
        with:
          name: jacoco
          path: target/site/jacoco
\end{minted}
\clearpage

% =============================
% C Variant (CMake + CTest + gcovr)
% =============================
\section{C Variant (CMake + CTest + gcovr)}
\paragraph{Directory layout}
\begin{minted}{text}
include/mathops.h
src/mathops.c
tests/test_mathops.c
CMakeLists.txt
\end{minted}

\paragraph{include/mathops.h}
\begin{minted}{c}
#ifndef MATHOPS_H
#define MATHOPS_H

int add(int a, int b);
int subtract(int a, int b);
int multiply(int a, int b);

#endif
\end{minted}

\paragraph{src/mathops.c}
\begin{minted}{c}
#include "mathops.h"

int add(int a, int b)      { return a + b; }
int subtract(int a, int b) { return a - b; }
int multiply(int a, int b) { return a * b; }
\end{minted}

\paragraph{tests/test\_mathops.c}
\begin{minted}{c}
#include <stdio.h>
#include "mathops.h"

int main(void) {
    if (multiply(5, 0) != 0) return 1;
    if (add(5, 5) != 10) return 1;
    if (subtract(15, 5) != 10) return 1;
    printf("All C tests passed\n");
    return 0;
}
\end{minted}
\clearpage

\paragraph{CMakeLists.txt}
\begin{minted}{cmake}
cmake_minimum_required(VERSION 3.20)
project(ci_c C)

set(CMAKE_C_STANDARD 11)

option(ENABLE_COVERAGE "Enable coverage flags" ON)

if (ENABLE_COVERAGE AND CMAKE_C_COMPILER_ID MATCHES "GNU|Clang")
  add_compile_options(--coverage -O0 -g)
  add_link_options(--coverage)
endif()

add_library(mathops src/mathops.c)
target_include_directories(mathops PUBLIC include)

add_executable(test_math tests/test_mathops.c)
target_link_libraries(test_math PRIVATE mathops)

include(CTest)
add_test(NAME mathops_tests COMMAND test_math)
\end{minted}
\clearpage

\paragraph{.github/workflows/ci-c.yml (with coverage gate)}
\begin{minted}{yaml}
name: C (CMake + CTest + gcovr)

on:
  pull_request:
  push:
    branches: [ main ]

jobs:
  build:
    runs-on: ubuntu-latest
    steps:
      - uses: actions/checkout@v4
      - name: Install tools
        run: |
          sudo apt-get update
          sudo apt-get install -y cmake gcc gcovr lcov
      - name: Configure
        run: cmake -S . -B build -DENABLE_COVERAGE=ON
      - name: Build
        run: cmake --build build --config Debug -- -j2
      - name: Test
        run: ctest --test-dir build --output-on-failure
      - name: Coverage (gcovr, fail under 80%)
        run: |
          gcovr -r . --filter 'src/' --xml -o build/coverage.xml \
                --html --html-details -o build/coverage.html \
                --fail-under-lines 80
      - uses: actions/upload-artifact@v4
        if: always()
        with:
          name: c-coverage
          path: build/coverage.*
\end{minted}

\begin{tcolorbox}
\textbf{Notes (C).} \texttt{gcovr} \emph{fails the job} if line coverage drops below 80\% (adjust as needed). Keep optimization off (\texttt{-O0}) for accurate coverage.
\end{tcolorbox}
\clearpage

% =============================
% C++ Variant (CMake + GoogleTest + gcovr)
% =============================
\section{C\texttt{++} Variant (CMake + GoogleTest + gcovr)}
\paragraph{Directory layout}
\begin{minted}{text}
include/mathops.hpp
src/mathops.cpp
tests/mathops_test.cpp
CMakeLists.txt
\end{minted}

\paragraph{include/mathops.hpp}
\begin{minted}{cpp}
#pragma once
namespace mathops {
  inline int add(int a, int b)      { return a + b; }
  inline int subtract(int a, int b) { return a - b; }
  inline int multiply(int a, int b) { return a * b; }
}
\end{minted}

\paragraph{src/mathops.cpp}
\begin{minted}{cpp}
#include "mathops.hpp"
// All functions are inline in the header for brevity; translation unit kept for structure.
\end{minted}

\paragraph{tests/mathops\_test.cpp}
\begin{minted}{cpp}
#include <gtest/gtest.h>
#include "mathops.hpp"

TEST(MathOps, Multiply) { EXPECT_EQ(mathops::multiply(5, 0), 0); }
TEST(MathOps, Add)      { EXPECT_EQ(mathops::add(5, 5), 10); }
TEST(MathOps, Subtract) { EXPECT_EQ(mathops::subtract(15, 5), 10); }

int main(int argc, char** argv) {
  ::testing::InitGoogleTest(&argc, argv);
  return RUN_ALL_TESTS();
}
\end{minted}
\clearpage

\paragraph{CMakeLists.txt (FetchContent + coverage flags)}
\begin{minted}{cmake}
cmake_minimum_required(VERSION 3.20)
project(ci_cpp CXX)

set(CMAKE_CXX_STANDARD 20)
set(CMAKE_CXX_STANDARD_REQUIRED ON)

option(ENABLE_COVERAGE "Enable coverage flags" ON)

if (ENABLE_COVERAGE AND CMAKE_CXX_COMPILER_ID MATCHES "GNU|Clang")
  add_compile_options(--coverage -O0 -g)
  add_link_options(--coverage)
endif()

add_library(mathops_cpp src/mathops.cpp)
target_include_directories(mathops_cpp PUBLIC include)

include(FetchContent)
FetchContent_Declare(
  googletest
  URL https://github.com/google/googletest/archive/refs/tags/v1.14.0.zip
)
FetchContent_MakeAvailable(googletest)

add_executable(mathops_test tests/mathops_test.cpp)
target_link_libraries(mathops_test PRIVATE mathops_cpp GTest::gtest GTest::gtest_main)

include(GoogleTest)
gtest_discover_tests(mathops_test)
\end{minted}
\clearpage

\paragraph{.github/workflows/ci-cpp.yml (matrix + coverage gate)}
\begin{minted}{yaml}
name: C++ (CMake + GTest + gcovr)

on:
  pull_request:
  push:
    branches: [ main ]

jobs:
  build:
    runs-on: ubuntu-latest
    strategy:
      matrix:
        compiler: [gcc, clang]
    steps:
      - uses: actions/checkout@v4
      - name: Install tools
        run: |
          sudo apt-get update
          sudo apt-get install -y cmake g++ clang gcovr lcov
      - name: Select compiler
        run: |
          if [ "${{ matrix.compiler }}" = "clang" ]; then
            export CC=clang
            export CXX=clang++
          else
            export CC=gcc
            export CXX=g++
          fi
          echo "CC=$CC"  >> $GITHUB_ENV
          echo "CXX=$CXX" >> $GITHUB_ENV
      - name: Configure
        run: cmake -S . -B build -DENABLE_COVERAGE=ON
      - name: Build
        run: cmake --build build --config Debug -- -j2
      - name: Test
        run: ctest --test-dir build --output-on-failure
      - name: Coverage (gcovr, fail under 80%)
        run: |
          gcovr -r . --filter 'src/' --xml -o build/coverage.xml \
                --html --html-details -o build/coverage.html \
                --fail-under-lines 80
      - uses: actions/upload-artifact@v4
        if: always()
        with:
          name: cpp-coverage-${{ matrix.compiler }}
          path: build/coverage.*
\end{minted}

\begin{tcolorbox}
\textbf{Notes (C\texttt{++}).} The matrix validates both \texttt{gcc} and \texttt{clang}. GoogleTest is pulled with CMake \texttt{FetchContent}. Coverage is enforced via \texttt{gcovr} with an 80\% line gate.
\end{tcolorbox}

% =============================
% Security and Quality Gates (shared)
% =============================
\section{Security and Quality Gates (Shared Patterns)}
\subsection{CodeQL (Code Scanning)}
\paragraph{.github/workflows/codeql.yml}
\begin{minted}{yaml}
name: CodeQL

on:
  push:
    branches: [ main ]
  pull_request:
    branches: [ main ]
  schedule:
    - cron: '0 2 * * 1'

jobs:
  analyze:
    permissions:
      actions: read
      contents: read
      security-events: write
    runs-on: ubuntu-latest
    steps:
      - uses: actions/checkout@v4
      - uses: github/codeql-action/init@v3
        with:
          languages: javascript, cpp
      - uses: github/codeql-action/autobuild@v3
      - uses: github/codeql-action/analyze@v3
\end{minted}

\subsection{Secret Scanning and Push Protection}
\paragraph{.github/secret\_scanning.yml (custom patterns \& ignores)}
\begin{minted}{yaml}
custom-patterns:
  - name: Internal API token
    pattern: 'sbm_[A-Za-z0-9]{24,}'
    secret-group: internal

paths-ignore:
  - "docs/**"
  - "*.md"
\end{minted}

\subsection{Dependency Hygiene (Dependabot)}
\paragraph{.github/dependabot.yml}
\begin{minted}{yaml}
version: 2
updates:
  - package-ecosystem: npm
    directory: "/"
    schedule: { interval: weekly }
    open-pull-requests-limit: 5
  - package-ecosystem: pip
    directory: "/"
    schedule: { interval: weekly }
  - package-ecosystem: maven
    directory: "/"
    schedule: { interval: weekly }
  - package-ecosystem: github-actions
    directory: "/"
    schedule: { interval: monthly }
\end{minted}

% =============================
% Linting / Formatting
% =============================
\section{Linting \& Formatting (Optional but Recommended)}
\subsection{Node (ESLint)}
\paragraph{Install}
\begin{minted}{bash}
npm i -D eslint
npx eslint --init
\end{minted}

\paragraph{CI step}
\begin{minted}{yaml}
- run: npm run lint
\end{minted}

\subsection{Python (flake8 + black)}
\paragraph{Install}
\begin{minted}{bash}
pip install flake8 black
\end{minted}

\paragraph{CI steps}
\begin{minted}{yaml}
- run: flake8 .
- run: black --check .
\end{minted}

\subsection{C/C\texttt{++} (clang-tidy / clang-format)}
\paragraph{Minimal config files}
\begin{minted}{yaml}
# .clang-tidy (example)
Checks: '-*,bugprone-*,performance-*,readability-*'
WarningsAsErrors: 'bugprone-*,performance-*'
\end{minted}

\begin{minted}{yaml}
# .clang-format (example)
BasedOnStyle: LLVM
IndentWidth: 2
ColumnLimit: 100
\end{minted}

\paragraph{CI snippets}
\begin{minted}{yaml}
- name: clang-tidy
  run: clang-tidy **/*.cpp -- -std=c++20

- name: clang-format (diff check)
  run: |
    if ! clang-format --dry-run --Werror $(git ls-files '*.c' '*.cpp' '*.hpp'); then
      echo "Formatting issues found"; exit 1; fi
\end{minted}

% =============================
% Reuse, Performance, and Badges
% =============================
\section{Monorepos \& Reuse}
\subsection{Reusable Workflows}
\paragraph{.github/workflows/reuse-node.yml}
\begin{minted}{yaml}
name: Reusable Node CI
on:
  workflow_call:
    inputs:
      node-version:
        required: true
        type: string
jobs:
  test:
    runs-on: ubuntu-latest
    steps:
      - uses: actions/checkout@v4
      - uses: actions/setup-node@v4
        with:
          node-version: ${{ inputs.node-version }}
          cache: npm
      - run: npm ci && npm test -- --coverage
\end{minted}

\paragraph{Consumer}
\begin{minted}{yaml}
jobs:
  use-reusable:
    uses: ./.github/workflows/reuse-node.yml
    with:
      node-version: "20"
\end{minted}

\section{Matrices, Caching, and Speed}
\subsection{Selective runs (paths filters)}
\begin{minted}{yaml}
on:
  pull_request:
    paths:
      - "apps/web/**"
      - ".github/workflows/ci.yml"
\end{minted}

\subsection{General caching tips}
\begin{itemize}
  \item Node: use \texttt{setup-node} with \texttt{cache: npm}; prefer \texttt{npm ci}.
  \item Python: \texttt{setup-python} with \texttt{cache: pip}.
  \item Java: \texttt{setup-java} with \texttt{cache: maven}.
  \item C/C\texttt{++}: enable ccache if needed; keep \texttt{fetch-depth: 1} for speed.
\end{itemize}

\section{Badges (Optional)}
\begin{minted}{text}
![CI](https://github.com/OWNER/REPO/actions/workflows/ci.yml/badge.svg)
[![CodeQL](https://github.com/OWNER/REPO/actions/workflows/codeql.yml/badge.svg)](https://github.com/OWNER/REPO/actions/workflows/codeql.yml)
\end{minted}

\section{Troubleshooting (Minted)}
\begin{itemize}
  \item \textbf{Missing \$ inserted / caption issues:} avoid fragile macros in captions. Use paragraph labels instead.
  \item \textbf{``Missing Pygments output'' or ``Cannot find style'':} ensure \texttt{pygmentize} is installed and build with \texttt{-shell-escape}. We select the \texttt{friendly} style to avoid custom style files.
  \item \textbf{``Missing \textbackslash end\{minted\}'':} verify each \texttt{minted} block is properly closed.
  \item \textbf{Coverage gates for C/C\texttt{++}:} adjust \texttt{gcovr} thresholds with \texttt{--fail-under-lines} and optionally \texttt{--fail-under-branches}.
\end{itemize}
\end{document}
