%========================================================
% GitHub Profile & Organization READMEs
% Practical Cheat Sheet + Copy-Paste Templates
%========================================================
\documentclass[11pt]{article}

% ---------- Encoding & layout ----------
\usepackage[T1]{fontenc}
\usepackage[utf8]{inputenc} % keep for broad compatibility
\usepackage{lmodern}        % crisp Latin fonts
\usepackage{geometry}
\geometry{margin=1in}
\usepackage{microtype}
\usepackage{setspace}
\setstretch{1.1}

% ---------- Colors, links, boxes ----------
\usepackage{xcolor}
\definecolor{ink}{HTML}{111827}      % gray-900
\definecolor{soft}{HTML}{F9FAFB}     % gray-50
\definecolor{accent}{HTML}{2563EB}   % blue-600
\definecolor{ok}{HTML}{059669}       % emerald-600
\definecolor{warn}{HTML}{D97706}     % amber-600
\definecolor{bad}{HTML}{DC2626}      % red-600
\usepackage{hyperref}
% ---------- Silence first-run .toc warnings (latexmk treats them as missing input) ----------
\usepackage{etoolbox}
\makeatletter
\patchcmd{\@starttoc}
  {\InputIfFileExists{\jobname.#1}{}{\typeout{No file \jobname.#1.}}}
  {\InputIfFileExists{\jobname.#1}{}{}}
  {}{}
\makeatother
% ---------- End .toc silence ----------
\hypersetup{
  colorlinks=true,
  linkcolor=accent,
  citecolor=accent,
  urlcolor=accent
}
\usepackage{tcolorbox}
% keep tcolorbox simple (avoid 'enhanced' to prevent key errors)
\tcbset{boxrule=0.6pt, colframe=ink, colback=soft, sharp corners, arc=2pt}
\usepackage{enumitem}
\setlist[itemize]{topsep=2pt,itemsep=2pt,parsep=0pt}
\setlist[enumerate]{topsep=2pt,itemsep=2pt,parsep=0pt}

% ---------- Code blocks (minted if available; fall back to listings) ----------
% ---------- Minted compatibility layer (CI-safe) ----------
% This document can render code with minted when -shell-escape is enabled, and
% falls back to listings when it is not (so CI builds do not hard-fail).
\newif\ifUseMinted
\UseMintedfalse
\begingroup
\ifdefined\pdfshellescape
  \ifnum\pdfshellescape=1\relax
    \global\UseMintedtrue
  \fi
\fi
\endgroup

\ifUseMinted
  \usepackage[newfloat,cache=false]{minted}
\else
  \usepackage{listings}
  \usepackage{xcolor}
  \usepackage{newfloat}
  \usepackage{fancyvrb} % provides \VerbatimEnvironment used by some wrappers

  % Provide a "listing" float compatible with minted's newfloat option
  \makeatletter
  \@ifundefined{c@listing}{%
    \DeclareFloatingEnvironment[name=Listing]{listing}
  }{}
  \makeatother

  % Minimal language definitions for common "minted" lexers / labels used in docs.
  % These are intentionally lightweight; they exist primarily to avoid build failures.
  \lstdefinelanguage{yaml}{
    sensitive=true,
    morecomment=[l]{\#},
    morestring=[b]",
    morestring=[b]',
  }
  \lstdefinelanguage{json}{
    sensitive=true,
    morestring=[b]",
    showstringspaces=false,
  }
  \lstdefinelanguage{ini}{
    sensitive=true,
    morecomment=[l]{;},
    morecomment=[l]{\#},
    morestring=[b]",
    morestring=[b]',
  }
  \lstdefinelanguage{cmake}{
    sensitive=true,
    morecomment=[l]{\#},
    morestring=[b]",
    morestring=[b]',
  }
  \lstdefinelanguage{powershell}{
    sensitive=true,
    morecomment=[l]{\#},
    morestring=[b]",
    morestring=[b]',
    morekeywords={param,begin,process,end,function,filter,return,if,elseif,else,foreach,for,while,do,until,break,continue,try,catch,finally,throw,switch},
  }
  \lstdefinelanguage{bash}{
    sensitive=true,
    morecomment=[l]{\#},
    morestring=[b]",
    morestring=[b]',
    morekeywords={if,then,else,elif,fi,for,do,done,while,in,case,esac,function,local,return,export,unset,echo,printf,read,cd,exit},
  }
  \lstdefinelanguage{sh}{sensitive=true, morecomment=[l]{\#}, morestring=[b]", morestring=[b]'}
  \lstdefinelanguage{shell}{sensitive=true, morecomment=[l]{\#}, morestring=[b]", morestring=[b]'}
  \lstdefinelanguage{console}{sensitive=false}
  \lstdefinelanguage{terminal}{sensitive=false}
  \lstdefinelanguage{md}{sensitive=false}
  \lstdefinelanguage{markdown}{sensitive=false}
  \lstdefinelanguage{text}{sensitive=false}

  % Reasonable defaults; keep this conservative to avoid surprises.
  \lstset{
    basicstyle=\ttfamily\small,
    breaklines=true,
    columns=fullflexible,
    keepspaces=true,
    showstringspaces=false,
    upquote=true,
    frame=single,
    framerule=0.2pt,
    aboveskip=0.75\baselineskip,
    belowskip=0.75\baselineskip,
    % Common Unicode glyphs seen in snippets
    literate=
      {•}{{\textbullet}}1
      {—}{{---}}1
      {–}{{--}}1
      {→}{{$\rightarrow$}}1
      {←}{{$\leftarrow$}}1
      {≥}{{$\ge$}}1
      {≤}{{$\le$}}1
  }

  % Minted command shims (ignore style/options in fallback)
  \providecommand{\usemintedstyle}[1]{}
  \providecommand{\setminted}[1]{}
  \providecommand{\setmintedinline}[1]{}

  % NOTE: minted's optional key-value options are not 1:1 with listings'
  % key-value options. To avoid hard failures, we intentionally ignore the
  % optional options argument in the fallback path.

  % minted: \begin{minted}[<opts>]{<lang>} ... \end{minted}
  \lstnewenvironment{minted}[2][]%
    {\lstset{language=#2}}%
    {}

  % minted: \inputminted[<opts>]{<lang>}{<file>}
  \newcommand{\inputminted}[3][]{\lstinputlisting[language=#2]{#3}}

  % minted: \mintinline{<lang>}{<code>}
  \newcommand{\mintinline}[2]{\texttt{#2}}

  % minted: \newminted[<envname>]{<lang>}{<opts>}
  \newcommand{\newminted}[3][]{%
    \def\MintedEnvName{#1}%
    \if\relax\detokenize{#1}\relax
      \edef\MintedEnvName{#2code}%
    \fi
    \expandafter\lstnewenvironment\expandafter{\MintedEnvName}[1][]%
      {\lstset{language=#2}}%
      {}%
  }

  % minted: \newmintedfile{<lang>}{<opts>} -> \input<lang>{file}
  \newcommand{\newmintedfile}[2]{%
    \expandafter\newcommand\csname input#1\endcsname[2][]{\lstinputlisting[language=#1]{##2}}%
  }

  % minted exposes \listoflistings; keep it defined for compatibility
  \providecommand{\listoflistings}{\listof{listing}{List of Listings}}
\fi
% ---------- End minted compatibility layer ----------
  \setminted{cache=false,
    breaklines=true,
    breakanywhere=true,
    fontsize=\small,
    linenos,
    numbersep=6pt,
    tabsize=2
  }
\ifUseMinted
  \newenvironment{code}[1]{%
    \VerbatimEnvironment
    \begin{minted}{#1}%
  }{%
    \end{minted}%
  }
\else
  \usepackage{listings}
  \lstset{
    basicstyle=\ttfamily\small,
    breaklines=true,
    numbers=left,
    numberstyle=\tiny,
    numbersep=6pt,
    tabsize=2,
    frame=single,
    rulecolor=\color{ink}
  }
  \lstnewenvironment{code}[1]{\lstset{language=#1}}{}
\fi

% ---------- Helpful tiny macros ----------
\newcommand{\kbd}[1]{\texttt{#1}}

%========================================================
\begin{document}

\begin{center}
  {\LARGE \textbf{GitHub Profile \& Organization READMEs}}\\[4pt]
  {\large Practical Cheat Sheet + Copy-Paste Templates}
\end{center}

\vspace{0.5em}

\begin{tcolorbox}
\textbf{What this is.} A quick, no-fluff reference for creating \emph{personal profile READMEs} and \emph{organization profile READMEs} (both public and members-only), plus ready-to-paste templates and an optional auto-refresh Action.
\end{tcolorbox}

%--------------------------------------------------------
\section*{1. Quick Rules}

\subsection*{1.1 Personal Profile README}
\begin{enumerate}
  \item Create a \textbf{public} repository named \textbf{exactly your GitHub handle} (e.g., \kbd{octocat}).
  \item Put your profile content in \kbd{README.md} on the default branch (usually \kbd{main}).
  \item That README renders on your user profile page.
\end{enumerate}

\subsection*{1.2 Organization README -- Public}
\begin{enumerate}
  \item In the org, create a \textbf{public} repository named \textbf{\kbd{.github}}.
  \item Add \kbd{profile/README.md} on the default branch.
  \item This README shows on the organization's public homepage.
\end{enumerate}

\subsection*{1.3 Organization README -- Members-Only}
\begin{enumerate}
  \item In the org, create a \textbf{private} repository named \textbf{\kbd{.github-private}}.
  \item Add \kbd{profile/README.md} on the default branch.
  \item Members can view this internal profile on the org page (member view).
\end{enumerate}

\begin{tcolorbox}
\textbf{Naming \& Branch Tips}
\begin{itemize}
  \item Use \textbf{exact} repo names and the \textbf{default branch}.
  \item Keep images in \kbd{assets/} and reference with relative paths.
  \item Prefer \texttt{.md} over \texttt{.rst} for broad compatibility.
\end{itemize}
\end{tcolorbox}
\clearpage

%--------------------------------------------------------
\section*{2. Minimal Structures (so you don't have to think)}
\begin{code}{text}
# Personal profile
<your-handle>/
  README.md

# Org public profile
.github/
  profile/
    README.md

# Org members-only profile
.github-private/
  profile/
    README.md
\end{code}

%--------------------------------------------------------
\section*{3. Copy-Paste Templates}

\subsection*{3.1 Personal Profile \texttt{README.md}}
\begin{code}{markdown}
# Hi, I'm <Your Name>
[![GitHub followers](https://img.shields.io/github/followers/<your-handle>?style=social)](https://github.com/<your-handle>)
[![Twitter Follow](https://img.shields.io/twitter/follow/<your-handle>?style=social)](https://twitter.com/<your-handle>)

- I'm working on: ...
- Tech: Go, Rust, TypeScript, AWS, Kubernetes, Terraform
- Ask me about: DevSecOps, CI/CD, GHAS, CodeQL
- Website/Blog: https://example.com

## Highlights
- Recent talk: [Title] -- slides / video
- Maintainer of: [project] (short tagline)

## Projects
- [repo-1](https://github.com/<your-handle>/repo-1) -- one-liner
- [repo-2](https://github.com/<your-handle>/repo-2) -- one-liner

<!-- recent:start -->
<!-- This section can be auto-updated by the Action below -->
<!-- recent:end -->

## Contact
- Email: me@example.com - LinkedIn: https://www.linkedin.com/in/<your-handle>/
\end{code}

\subsection*{3.2 Organization Public \texttt{profile/README.md}}
\begin{code}{markdown}
# Welcome to <Org Name>

We build <what you build>. This profile highlights our core repos, docs, and community spaces.

## Start Here
- Contributing: https://github.com/<org>/.github/blob/main/CONTRIBUTING.md
- Code of Conduct: https://github.com/<org>/.github/blob/main/CODE_OF_CONDUCT.md
- Security Policy: https://github.com/<org>/.github/blob/main/SECURITY.md

## Repos to Explore
- [repo-1](https://github.com/<org>/repo-1) -- one-liner
- [repo-2](https://github.com/<org>/repo-2) -- one-liner

## Community
- Discussions: https://github.com/<org>/.github/discussions
- Website: https://<org>.example.com
\end{code}

\subsection*{3.3 Organization Members-Only \texttt{profile/README.md}}
\begin{code}{markdown}
# <Org Name> -- Member Handbook

Welcome, members! This internal profile surfaces our most-used links.

## Onboarding
- Checklist, role guides, and calendars

## Engineering
- Environments, on-call runbooks, escalation paths

## Security
- Reporting, secret handling, dependency policies

## Operations
- Weekly rituals, roadmaps, planning docs
\end{code}

%--------------------------------------------------------
\section*{4. Optional: Auto-Refresh Your Personal Profile}

\noindent
This GitHub Actions workflow updates a \emph{``Recently active''} section in your personal \kbd{README.md} daily using the GitHub API (no external services).

\subsection*{4.1 Add markers to your \texttt{README.md}}
\begin{code}{markdown}
<!-- recent:start -->
<!-- recent:end -->
\end{code}

\subsection*{4.2 Workflow file \texttt{.github/workflows/refresh-profile.yml}}
\begin{code}{yaml}
name: Refresh Profile
on:
  schedule:
    - cron: "0 9 * * *"   # daily 09:00 UTC
  workflow_dispatch: {}
permissions:
  contents: write
  pull-requests: write
  actions: read
  checks: read
  statuses: read

jobs:
  update:
    runs-on: ubuntu-latest
    steps:
      - uses: actions/checkout@v4

      - uses: actions/github-script@v7
        id: gen
        with:
          script: |
            const res = await github.rest.repos.listForUser({
              username: context.repo.owner,
              sort: "updated",
              per_page: 5
            });
            const items = res.data
              .map(r => `- [${r.name}](${r.html_url}) -- ${r.description ?? ""}`)
              .join("\n");
            core.setOutput("block", `\n### Recently active\n${items}\n`);

      - name: Inject section
        shell: bash
        run: |
          BLOCK="${{ steps.gen.outputs.block }}"
          awk -v repl="$BLOCK" '
            BEGIN{printed=0}
            /<!-- recent:start -->/ {print; print repl; skip=1; next}
            /<!-- recent:end -->/ && skip {print; skip=0; printed=1; next}
            !skip {print}
            END{
              if (!printed)
                print "\n<!-- recent:start -->\n" repl "\n<!-- recent:end -->\n"
            }
          ' README.md > README.md.new
          mv README.md.new README.md

      - name: Commit
        run: |
          git config user.name  "github-actions[bot]"
          git config user.email "41898282+github-actions[bot]@users.noreply.github.com"
          git add README.md
          git commit -m "chore: refresh profile" || echo "No changes"
          git push
\end{code}

\begin{tcolorbox}
\textbf{Notes}
\begin{itemize}
  \item This workflow runs on your \textbf{personal profile repo} (named exactly like your handle).
  \item You can add more sections (e.g., latest blog posts) by expanding the \kbd{github-script} step.
\end{itemize}
\end{tcolorbox}

%--------------------------------------------------------
\section*{5. Good Practices Checklist}

\begin{itemize}
  \item Keep READMEs \textbf{short, scannable}, and \textbf{link} to deeper docs.
  \item Use \textbf{relative image paths} (e.g., \kbd{assets/banner.png}) for portability.
  \item Prefer \textbf{badges} for status: CI passing, license, coverage, latest release.
  \item Pin 6--8 best repositories on your profile for fast orientation.
  \item Revisit quarterly: update links, clean up stale sections, refresh screenshots.
\end{itemize}

%--------------------------------------------------------
\section*{6. Badge Snippets (Shields.io)}

\begin{code}{markdown}
![CI](https://img.shields.io/github/actions/workflow/status/<owner>/<repo>/ci.yml)
![License](https://img.shields.io/github/license/<owner>/<repo>)
![Release](https://img.shields.io/github/v/release/<owner>/<repo>)
![Issues](https://img.shields.io/github/issues/<owner>/<repo>)
\end{code}

%--------------------------------------------------------
\section*{7. Troubleshooting (Fast)}

\begin{itemize}
  \item \textbf{README not showing?} Verify repo name and path:
  \begin{itemize}
    \item Personal: \kbd{<handle>/README.md} on default branch.
    \item Org public: \kbd{.github/profile/README.md} (public repo).
    \item Org members: \kbd{.github-private/profile/README.md} (private repo).
  \end{itemize}
  \item \textbf{Images broken?} Use relative paths and ensure files are committed to the same repo.
  \item \textbf{Action didn't run?} Check workflow path and permissions; make a manual \emph{workflow\_dispatch}.
\end{itemize}

\vfill
\hrule
\smallskip
{\footnotesize \textit{Tip:} This PDF uses \texttt{minted} if available and falls back to \texttt{listings}. If you prefer minted, compile with \texttt{-shell-escape} and ensure Pygments is installed.}

\end{document}
