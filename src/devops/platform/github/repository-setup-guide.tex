\documentclass[11pt]{article}

% ---------- Encoding & layout ----------
\usepackage[T1]{fontenc}
\usepackage[utf8]{inputenc}
\usepackage[margin=1in]{geometry}
\usepackage{setspace}
\setstretch{1.1}

% ---------- Colors, hyperlinks ----------
\usepackage{xcolor}
\definecolor{ink}{HTML}{1F2937}      % neutral-800
\definecolor{soft}{HTML}{F9FAFB}     % neutral-50
\definecolor{accent}{HTML}{2563EB}   % blue-600
\definecolor{ok}{HTML}{047857}       % emerald-700
\definecolor{warn}{HTML}{B45309}     % amber-700
\usepackage[hidelinks]{hyperref}

% ---------- Boxes ----------
\usepackage[most]{tcolorbox}
\tcbset{enhanced, boxrule=0.6pt, colframe=ink, colback=soft, sharp corners, arc=2pt}

% ---------- Lists ----------
\usepackage{enumitem}
\setlist[itemize]{topsep=4pt,itemsep=3pt,parsep=0pt}
\setlist[enumerate]{topsep=4pt,itemsep=3pt,parsep=0pt}

% ---------- Code highlighting (minted) ----------
% Compile with: latexmk -pdf -shell-escape repository-setup.tex
\usepackage[cache=false]{minted}
\setminted{cache=false,fontsize=\small, breaklines=true, autogobble, tabsize=2}

% ---------- Convenience ----------
\newcommand{\file}[1]{\texttt{#1}}
\newcommand{\dir}[1]{\texttt{#1/}}

% ==============================================
%                 Document
% ==============================================
\begin{document}

\begin{center}
{\LARGE \textbf{Repository Setup Guide}}\\[4pt]
{\large GitHub Pages, Community Health, Sponsorship, and Metadata}\\[10pt]
Version 1.0 \quad|\quad \today
\end{center}

\begin{tcolorbox}
\textbf{Purpose.} This document turns your repository setup notes into a ready-to-run, repeatable playbook. It covers GitHub Pages (\file{CNAME} and \dir{docs/}), community health files (\file{CONTRIBUTING.md}, \file{CODE\_OF\_CONDUCT.md}, \file{SECURITY.md}, \file{SUPPORT.md}), sponsorship via \file{.github/FUNDING.yml}, licensing, citation metadata (\file{CITATION.cff}), and contribution templates (PR and Issues).
\end{tcolorbox}

\tableofcontents
\newpage

% ----------------------------------------------
\section{Quick Start Checklist}
\begin{enumerate}
  \item Create community files: \file{CONTRIBUTING.md}, \file{CODE\_OF\_CONDUCT.md}, \file{.github/SECURITY.md}, \file{.github/SUPPORT.md}.
  \item Enable sponsorship with \file{.github/FUNDING.yml}.
  \item Add \file{CITATION.cff} for citation metadata.
  \item Choose and add a license (MIT shown here) as \file{LICENSE.md}.
  \item For GitHub Pages: add \dir{docs/} and optional root-level \file{CNAME}, then enable Pages in \textbf{Settings \textrightarrow{} Pages}.
  \item Add PR template and issue templates in \dir{.github/}.
  \item Optional: centralize org-wide versions in a dedicated \file{.github} repository. Repo-local files override org defaults.
\end{enumerate}

\begin{tcolorbox}
\textbf{Compile note (LaTeX).} This guide uses the \texttt{minted} package. Compile with: \texttt{latexmk -pdf -shell-escape repository-setup.tex}
\end{tcolorbox}

% ----------------------------------------------
\section{Repository Layout (Starter Pack)}
Recommended minimal structure:
\begin{minted}{text}
.
|-- .github/
|   |-- FUNDING.yml
|   |-- ISSUE_TEMPLATE/
|   |   |-- bug_report.yml
|   |   `-- feature_request.yml
|   |-- PULL_REQUEST_TEMPLATE.md
|   |-- SECURITY.md
|   `-- SUPPORT.md
|-- CITATION.cff
|-- CNAME                  (optional; for custom domain)
|-- CODE_OF_CONDUCT.md
|-- CONTRIBUTING.md
|-- LICENSE.md
|-- README.md
`-- docs/
    `-- index.html
\end{minted}
\clearpage

% ----------------------------------------------
\section{GitHub Pages \& Custom Domain (\file{CNAME})}
\subsection{Basics}
GitHub Pages can serve static sites from the repository. Common setups:
\begin{itemize}
  \item Serve from \dir{docs/} on your main branch (recommended for app repos).
  \item Serve from the root if the repository is dedicated to a site.
\end{itemize}

\subsection{Custom Domain}
Add a root-level \file{CNAME} file containing your custom domain:
\begin{minted}{text}
your.custom.domain
\end{minted}

Then:
\begin{enumerate}
  \item In your DNS registrar, create a CNAME record pointing \texttt{your.custom.domain} to \texttt{<user-or-org>.github.io}.
  \item In the repository: \textbf{Settings \textrightarrow{} Pages}, choose the branch/folder and set the custom domain.
\end{enumerate}

\subsection{Minimal \dir{docs/} Site}
\begin{minted}{html}
<!doctype html>
<html>
  <head>
    <meta charset="utf-8">
    <meta name="viewport" content="width=device-width,initial-scale=1">
    <title>Your GitHub Pages Site</title>
  </head>
  <body>
    <h1>It works!</h1>
    <p>Configure GitHub Pages to serve from <code>/docs</code> on the main branch.</p>
  </body>
</html>
\end{minted}
\clearpage

% ----------------------------------------------
\section{Community Health Files}
These files clarify expectations, collaboration norms, and support boundaries.

\subsection{\file{CONTRIBUTING.md}}
\begin{minted}{markdown}
# Contributing

Thanks for considering a contribution!

1. Propose first: open an Issue describing the change and approach.
2. Fork & branch: use feature branches (feat/xyz, fix/abc).
3. Coding standards: include tests, run linters/formatters, update docs.
4. PR: link the Issue, describe change & risks, add screenshots if UI.
5. Review: respond to feedback; squash on merge if requested.

## Development setup
- Document prereqs and `make test` or `npm test`.

## Code of Conduct
By participating, you agree to the [Code of Conduct](CODE_OF_CONDUCT.md).
\end{minted}

\subsection{\file{CODE\_OF\_CONDUCT.md}}
\begin{minted}{markdown}
# Code of Conduct

We are committed to a welcoming, harassment-free experience for everyone.

- Be respectful and inclusive.
- No harassment, discrimination, or trolling.
- Follow maintainer guidance during reviews.

For serious concerns, contact conduct@example.com (private).

Consider adopting the Contributor Covenant.
\end{minted}

\clearpage
\subsection{\file{.github/SECURITY.md}}
\begin{minted}{markdown}
# Security Policy

We appreciate responsible disclosures.

## Reporting
Email security@example.com with details. Do not open public issues.

Include:
- Affected versions/commit SHA
- Reproduction steps or PoC (if available)
- Impact assessment

We'll acknowledge within 3 business days.
\end{minted}

\subsection{\file{.github/SUPPORT.md}}
\begin{minted}{markdown}
# Support

- How to ask: open an Issue with a minimal repro (steps, versions, logs).
- Where: repository Issues. Use Discussions for Q&A if enabled.
- Before filing: search existing issues and docs/FAQ.
\end{minted}

% ----------------------------------------------
\section{Sponsorship: \file{.github/FUNDING.yml}}
This file enables the Sponsors button and links funding platforms.
\begin{minted}{yaml}
# .github/FUNDING.yml
github: ["YOUR_GITHUB_USERNAME_OR_ORG"]
patreon: YOUR_PATREON
ko_fi: YOUR_KOFI
custom: "https://example.com/sponsor"
# tidelift: "npm/your-package-name"
\end{minted}

% ----------------------------------------------
\section{Citation Metadata: \file{CITATION.cff}}
GitHub renders this file to provide citation formats (APA, BibTeX, etc.).
\begin{minted}{yaml}
cff-version: 1.2.0
message: "If you use this software, please cite it as below."
title: "PROJECT_TITLE"
version: "0.1.0"
date-released: "2025-11-02"
authors:
  - family-names: "Suber"
    given-names: "Jordan"
# identifiers:
#   - type: doi
#     value: "10.1234/your-doi"
\end{minted}

% ----------------------------------------------
\section{License: \file{LICENSE.md} (MIT example)}
\begin{minted}{markdown}
MIT License

Copyright (c) 2025 YOUR NAME

Permission is hereby granted, free of charge, to any person obtaining a copy
of this software and associated documentation files (the "Software"), to deal
in the Software without restriction...

THE SOFTWARE IS PROVIDED "AS IS", WITHOUT WARRANTY OF ANY KIND...
\end{minted}

Choose a license suitable for your project or organization as needed.

% ----------------------------------------------
\section{Contribution Templates}
\subsection{Pull Request Template: \file{.github/PULL\_REQUEST\_TEMPLATE.md}}
\begin{minted}{markdown}
## Summary
Explain the motivation and context.

## Changes
- [ ] Item 1
- [ ] Item 2

## Testing
- [ ] Unit tests
- [ ] Manual QA notes

## Screenshots
(if UI)

## Checklist
- [ ] Linked to issue
- [ ] Added/updated tests
- [ ] Updated docs
\end{minted}
\clearpage

\subsection{Issue Templates: \dir{.github/ISSUE\_TEMPLATE/}}
\paragraph{Bug report (\file{bug\_report.yml})}
\begin{minted}{yaml}
name: Bug report
description: File a bug report
title: "[Bug]: "
labels: [bug]
body:
  - type: textarea
    attributes:
      label: Description
      description: What happened? What did you expect?
  - type: textarea
    attributes:
      label: Steps to Reproduce
  - type: input
    attributes:
      label: Version/Commit
  - type: textarea
    attributes:
      label: Logs
\end{minted}

\paragraph{Feature request (\file{feature\_request.yml})}
\begin{minted}{yaml}
name: Feature request
description: Suggest an idea
title: "[Feature]: "
labels: [enhancement]
body:
  - type: textarea
    attributes:
      label: Problem
  - type: textarea
    attributes:
      label: Proposed solution
  - type: textarea
    attributes:
      label: Alternatives
\end{minted}
\clearpage

% ----------------------------------------------
\section{README Basics}
Keep the root \file{README.md} crisp and actionable:
\begin{itemize}
  \item What the project does and who it is for.
  \item Quickstart commands and prerequisites.
  \item How to contribute and where to get help.
  \item License and citation pointers.
\end{itemize}

\begin{minted}{markdown}
# Project Name

Short description of purpose and scope.

## Quickstart
- Prereqs...
- Install...
- Run...

## Contributing
See CONTRIBUTING.md and CODE_OF_CONDUCT.md.

## Security
See .github/SECURITY.md.

## License
MIT (see LICENSE.md).

## Citation
See CITATION.cff.
\end{minted}

% ----------------------------------------------
\section{Org-wide vs Repo-local Files}
You can host org-wide defaults in a special \file{.github} repository:
\begin{itemize}
  \item Org defaults apply to repositories across the org.
  \item Any repo-local file overrides the org-wide default.
\end{itemize}

% ----------------------------------------------
\section{Next Steps}
\begin{itemize}
  \item Add a minimal CI workflow (e.g., lint, test) in \dir{.github/workflows/}.
  \item Gate PRs with required checks and branch protection rules.
  \item Document release process in \file{CONTRIBUTING.md}.
  \item If using GitHub Pages: add more content to \dir{docs/} and configure a theme.
\end{itemize}

% ----------------------------------------------
\section*{Appendix: Copy-and-Paste Snippets}
\subsection*{All in one place}
For convenience, here are the main snippets again:

\paragraph{\file{CNAME}}
\begin{minted}{text}
your.custom.domain
\end{minted}

\paragraph{\file{docs/index.html}}
\begin{minted}{html}
<!doctype html>
<html>
  <head><meta charset="utf-8"><meta name="viewport" content="width=device-width,initial-scale=1">
    <title>Your GitHub Pages Site</title></head>
  <body>
    <h1>It works!</h1>
    <p>Serve from /docs on the main branch.</p>
  </body>
</html>
\end{minted}

\paragraph{\file{.github/FUNDING.yml}}
\begin{minted}{yaml}
github: ["YOUR_GITHUB_USERNAME_OR_ORG"]
patreon: YOUR_PATREON
ko_fi: YOUR_KOFI
custom: "https://example.com/sponsor"
\end{minted}

\paragraph{\file{CITATION.cff}}
\begin{minted}{yaml}
cff-version: 1.2.0
message: "If you use this software, please cite it as below."
title: "PROJECT_TITLE"
version: "0.1.0"
date-released: "2025-11-02"
authors:
  - family-names: "Suber"
    given-names: "Jordan"
\end{minted}

\paragraph{\file{LICENSE.md} (MIT excerpt)}
\begin{minted}{markdown}
MIT License

Copyright (c) 2025 YOUR NAME

Permission is hereby granted, free of charge...
\end{minted}

\paragraph{\file{.github/PULL\_REQUEST\_TEMPLATE.md}}
\begin{minted}{markdown}
## Summary
...
\end{minted}

\paragraph{Issue templates}
\begin{minted}{yaml}
# bug_report.yml and feature_request.yml (see sections above)
\end{minted}

\end{document}

