% !TEX TS-program = pdflatex
\documentclass[11pt,a4paper]{article}

% -------------------- Packages --------------------
\usepackage[T1]{fontenc}
\usepackage{lmodern}
\usepackage{inconsolata}
\usepackage{upquote}
\usepackage{microtype}
\usepackage[margin=1in]{geometry}
\usepackage{parskip}
\usepackage[hyphens]{url}
\usepackage{hyperref}
\usepackage{bookmark}
\usepackage{enumitem}
\usepackage[dvipsnames]{xcolor}
\usepackage{booktabs}
\usepackage{array}
\usepackage{ragged2e}
\usepackage[most]{tcolorbox}
\usepackage{amsmath,amssymb}
\usepackage{titlesec}
\usepackage{graphicx}
\usepackage{listings}

% -------------------- Readability Tweaks --------------------
\linespread{1.03}
\setlength{\parindent}{0pt}
\setlength{\parskip}{0.55em}
\setlength{\emergencystretch}{2em}
\renewcommand{\arraystretch}{1.12}
\raggedbottom
\clubpenalty=10000
\widowpenalty=10000
\displaywidowpenalty=10000
\titlespacing*{\section}{0pt}{0.9em}{0.35em}
\titlespacing*{\subsection}{0pt}{0.75em}{0.25em}
\setlist{leftmargin=*,itemsep=2pt,topsep=4pt}
\setlist[itemize]{itemsep=2pt}
\setlist[enumerate]{itemsep=2pt}

% -------------------- Colors --------------------
\definecolor{Primary}{HTML}{0E7490}
\definecolor{Accent}{HTML}{0EA5E9}
\definecolor{Soft}{HTML}{F1F5F9}
\definecolor{Ink}{HTML}{0F172A}
\definecolor{Meta}{HTML}{475569}
\definecolor{OK}{HTML}{16A34A}
\definecolor{Warn}{HTML}{EA580C}
\definecolor{Bad}{HTML}{DC2626}

\hypersetup{
  colorlinks=true,
  linkcolor=Primary,
  urlcolor=Primary,
  citecolor=Primary,
  breaklinks=true,
  pdfauthor={},
  pdftitle={User Story Template & Guide}
}
\urlstyle{same}
\titleformat{\section}{\large\bfseries\color{Ink}}{\thesection}{0.6em}{}
\titleformat{\subsection}{\normalsize\bfseries\color{Ink}}{\thesubsection}{0.6em}{}

\newcommand{\checkbox}{\(\square\)}
\newcommand{\checkedbox}{\(\blacksquare\)}
\newcommand{\eg}{e.g.\ }
\newcommand{\ie}{i.e.\ }

% -------------------- Gherkin (listings) --------------------
\lstdefinelanguage{Gherkin}{
  morekeywords={Feature,Background,Scenario,Scenario\ Outline,Examples,Given,When,Then,And,But},
  sensitive=true,
}
\lstset{
  language=Gherkin,
  basicstyle=\ttfamily\small,
  keywordstyle=\color{Primary}\bfseries,
  commentstyle=\itshape\color{Meta},
  showstringspaces=false,
  frame=single,
  framerule=0.4pt,
  rulecolor=\color{Soft},
  backgroundcolor=\color{Soft},
  tabsize=2,
  columns=fullflexible,
  keepspaces=true,
  breaklines=true,
  breakatwhitespace=true,
  xleftmargin=1ex,
  framexleftmargin=1ex,
  framesep=0.6ex,
  aboveskip=3pt,
  belowskip=6pt
}

% -------------------- Card Look & Feel (matches screenshot) --------------------
\tcbset{
  colback=gray!2,
  colframe=gray!50,
  colbacktitle=gray!6,
  coltitle=black,
  fonttitle=\bfseries\large,
  arc=2pt,
  boxrule=0.4pt,
  left=8pt,right=8pt,top=8pt,bottom=8pt,
  enhanced,
  breakable,
  borderline west={2pt}{0pt}{MidnightBlue}
}

% Badges/pills
\newtcbox{\pill}{on line, arc=3pt, boxsep=0.8pt, left=4pt,right=4pt,top=1pt,bottom=1pt,
  colframe=gray!50, colback=gray!15, boxrule=0.3pt}
\newcommand{\badge}[1]{\pill{\footnotesize #1}}

% Footer helpers
\newcommand{\DoR}{\textbf{Definition of Ready:} Persona clear; AC drafted; Dependencies known; Estimate set.}
\newcommand{\DoD}{\textbf{Definition of Done:} All ACs pass; Tests green; Security/a11y checks; Docs updated; Deployed/flagged.}
\let\cb\checkbox

% Widths for robust tables (no tabularx needed)
\newlength{\StoryLabelW}
\setlength{\StoryLabelW}{3.2cm}
\newlength{\StoryValueW}
\setlength{\StoryValueW}{\dimexpr\linewidth-\StoryLabelW-2\tabcolsep\relax}

% -------------------- Story Card macro (exact screenshot layout) --------------------
% 1: ID   2: Title   3: Epic/Feature   4: Business Value
% 5: Priority   6: Estimate(SP)   7: Persona   8: Dependencies   9: Assumptions/Risks
\newcommand{\StoryCard}[9]{%
  \newpage
  \begin{tcolorbox}[title={\textbf{#1}\ \textemdash\ #2}]
  \small
  \begin{tabular}{@{}>{\raggedleft\arraybackslash\bfseries}p{\StoryLabelW} >{\RaggedRight\arraybackslash}p{\StoryValueW}@{}}
    Epic / Feature          & #3 \\
    Business Value          & #4 \\
    Priority / Estimate     & \badge{Priority: #5}\ \badge{SP: #6} \\
    Persona                 & #7 \\
    Dependencies            & #8 \\
    Assumptions / Risks     & #9 \\
  \end{tabular}

  \medskip
  \textbf{Story}\quad
  \emph{As a #7, I want to #2 so that #4.}

  \medskip
  \textbf{Non-Functional}\quad
  \badge{Performance}\ \badge{Security}\ \badge{Reliability}\ \badge{Accessibility}\ \badge{Privacy}\ \badge{i18n}

  \medskip
  \textbf{Acceptance Criteria (BDD)}
  \begin{description}[leftmargin=2.4cm, labelwidth=2.3cm, style=nextline, itemsep=2pt, topsep=2pt]
    \item[\textbf{Scenario}] Happy path
    \item[\textbf{Given}] the target repository and pipeline configuration are available
    \item[\textbf{When}] the user completes the \emph{Hands-on Objective}
    \item[\textbf{Then}] the stated \emph{Outcome} is observable and recorded in the pipeline/job summary
  \end{description}

  \vspace{0.2\baselineskip}
  {\footnotesize\color{gray!60}\DoR\ \textbullet\ \DoD}
  \end{tcolorbox}
}

% -------------------- Tasks box (matches screenshot style) --------------------
\newenvironment{TasksBox}[1][Tasks]{%
  \begin{tcolorbox}[
    enhanced,breakable,
    colback=gray!1, colframe=gray!35,
    colbacktitle=gray!6, coltitle=black,
    title={#1}, fonttitle=\bfseries,
    borderline west={1.8pt}{0pt}{MidnightBlue},
    arc=2pt, boxrule=0.4pt,
    left=10pt,right=10pt,top=6pt,bottom=6pt,
    before skip=6pt, after skip=10pt
  ]
  \small
  \begin{itemize}[label=\cb, leftmargin=*, labelsep=0.6em, itemsep=4pt, topsep=2pt, parsep=0pt]
}{%
  \end{itemize}
  \end{tcolorbox}
}

% -------------------- Document --------------------
\begin{document}
\begin{center}
  {\huge \textbf{User Story Template \& Guide}}\\[2pt]
  \textcolor{Meta}{Concise, user-centered requirements for Agile delivery}\\[6pt]
\end{center}

\noindent\textbf{What is a user story?}\\
A brief, informal description of a feature from the end-user's perspective that emphasizes value: \emph{``As a \([type~of~user]\), I want to \([action]\) so that \([benefit]\).''}

\section{How to Use This Template}
\begin{enumerate}
  \item Start with the \textbf{one-sentence story} (persona, goal, value).
  \item Add \textbf{acceptance criteria} in Gherkin (Given/When/Then).
  \item Capture \textbf{non-functional requirements} (performance, security, accessibility, \ldots).
  \item Confirm \textbf{Definition of Ready} (DoR) before sprint pull; confirm \textbf{Definition of Done} (DoD) before acceptance.
  \item Keep stories \textbf{INVEST}: Independent, Negotiable, Valuable, Estimable, Small, Testable.
\end{enumerate}

\clearpage
% ============================================================
\section{Story Card Template (Duplicate for Each Story)}
% ============================================================
\StoryCard{ID-XXXX}{Short, Action-Oriented Title}{Epic/Feature or Capability}
{Concise business value/outcome statement (why this matters)}
{Must}{3}
{persona (e.g., developer, SRE, analyst)}
{key upstream/downstream dependencies}
{assumptions, risks, constraints}

\begin{TasksBox}
  \item \cb First concrete task (commands, paths, or files where useful).
  \item \cb Second concrete task.
  \item \cb Third concrete task.
  \item \cb Validation step (e.g., “job summary shows metrics A/B/C”).
\end{TasksBox}

\clearpage
% ============================================================
\section{Quick Story Skeletons (Copy \& Fill)}
% ============================================================
\noindent\textbf{Classic:}\\
\emph{As a \lbrack type of user\rbrack, I want to \lbrack action\rbrack\ so that \lbrack benefit\rbrack.}

\noindent\textbf{Job-to-be-Done (alternative phrasing):}\\
\emph{When \lbrack situation\rbrack, as \lbrack persona\rbrack, I want to \lbrack motivation\rbrack so I can \lbrack expected outcome\rbrack.}

\noindent\textbf{API-facing (service consumer):}\\
\emph{As an integrating service, I want an endpoint to \lbrack task\rbrack so that my workflow can \lbrack outcome\rbrack.}

\clearpage
% ============================================================
\section{Examples (Good, with AC)}
% ============================================================
\subsection*{Example 1 — Resume Autofill (Web App)}
\StoryCard{US-101}{Save Resume Information}{Profile/Resume}
{Reduce re-entry time by storing resume data once and reusing in new applications}
{Must}{3}
{job applicant}
{User profile service; secure storage}
{PII handling; consent toggles}
\begin{lstlisting}
Scenario: Save resume profile
  Given I am logged in
  And I have filled out my resume fields
  When I click "Save profile"
  Then my resume is stored securely
  And I see a confirmation message

Scenario: Autofill on new application
  Given I am starting a new job application
  And I have a saved resume profile
  When I open the "Resume" step
  Then fields are pre-populated from my profile
  And I can edit any field before submission

Scenario: Privacy and consent
  Given I am updating my profile
  When I toggle "Allow autofill"
  Then future applications use my profile only if enabled
\end{lstlisting}

\clearpage
\subsection*{Example 2 — Mobile Accessibility Setting}
\StoryCard{US-142}{Increase Text Size Preference}{Accessibility}
{Improve readability for visually impaired users by allowing larger default text}
{Should}{2}
{mobile app user}
{Settings service; theming}
{Dynamic type support across screens}
\begin{lstlisting}
Scenario: Persisted text size
  Given I set text size to "Large"
  When I close and reopen the app
  Then all textual UI respects my "Large" setting

Scenario: Minimum contrast enforced
  Given the current theme
  When "Large" text is enabled
  Then contrast ratio meets WCAG 2.1 AA
\end{lstlisting}

\clearpage
\subsection*{Example 3 — API Rate Limiting (Platform)}
\StoryCard{US-230}{Fair-Use API Rate Limits}{Platform Governance}
{Predictable throttling so integrators can handle limits gracefully}
{Must}{3}
{API consumer}
{Gateway; headers; telemetry}
{Client retries; visibility}
\begin{lstlisting}
Scenario: Exceeded request limit
  Given the limit is 100 requests/minute
  And I have sent 100 requests within the last minute
  When I send one additional request
  Then I receive HTTP 429 with "Retry-After"

Scenario: Headers advertised
  Given I call any public endpoint
  Then response includes "X-RateLimit-Limit",
  And "X-RateLimit-Remaining" and "X-RateLimit-Reset"
\end{lstlisting}

\clearpage
% ============================================================
\section{Quality Bar: INVEST \& Friends}
% ============================================================
\subsection*{INVEST Checklist}
\begin{itemize}
  \item \textbf{Independent} — Minimal coupling to other stories.
  \item \textbf{Negotiable} — Details emerge through conversation.
  \item \textbf{Valuable} — Clear, user-facing benefit.
  \item \textbf{Estimable} — Team can size the work.
  \item \textbf{Small} — Completes within a sprint; slice if not.
  \item \textbf{Testable} — ACs enable objective verification.
\end{itemize}

\subsection*{The 3 Cs}
\textbf{Card}, \textbf{Conversation}, \textbf{Confirmation}.

\subsection*{Definition of Ready (suggested)}
\checkbox Persona known \quad
\checkbox ACs drafted \quad
\checkbox Dependencies known \quad
\checkbox Estimate done \quad
\checkbox NFRs considered

\subsection*{Definition of Done (suggested)}
\checkbox All ACs pass \quad
\checkbox Tests written \& green \quad
\checkbox Security/a11y checks \quad
\checkbox Docs updated \quad
\checkbox Deployed/flagged

\clearpage
% ============================================================
\section{Good vs. Weak Stories (Rewrite Patterns)}
% ============================================================
\begin{tcolorbox}[title=Weak]
``As a developer, I want to refactor the payment module so that the code is cleaner.''
\end{tcolorbox}
\begin{tcolorbox}[title=Rewrite (User Value)]
``As a shopper, I want checkout to load under 2s so that I can complete purchases without frustration.''\\
(Now track refactor tasks as sub-tasks under this user-value story or as an enabling technical story linked to the epic.)
\end{tcolorbox}

\clearpage
% ============================================================
\section{Slicing Techniques (to Make Stories Smaller)}
% ============================================================
Slice by:
\begin{itemize}
  \item \textbf{Workflow Step} (create \(\rightarrow\) edit \(\rightarrow\) submit)
  \item \textbf{Data Subset} (top fields first, then advanced)
  \item \textbf{Channel} (web, then mobile)
  \item \textbf{Happy Path} first, then \textbf{edge cases \& errors}
  \item \textbf{Non-functional} (baseline perf, then stretch goals)
  \item \textbf{Integration Depth} (mock \(\rightarrow\) sandbox \(\rightarrow\) production)
\end{itemize}

% ============================================================
\section{Backlog Hygiene \& Planning Aids}
% ============================================================
\subsection*{Story Metadata (recommended fields)}
\begin{tabular}{@{}>{\raggedleft\arraybackslash\bfseries}p{\StoryLabelW} >{\RaggedRight\arraybackslash}p{\StoryValueW}@{}}
\textbf{ID / Title} & Concise verb-noun title (\eg ``Autofill Resume'').\\
\textbf{Epic / Feature} & Parent work item for grouping.\\
\textbf{Priority} & Must/Should/Could (or numeric rank).\\
\textbf{Estimate} & Story points or t-shirt size.\\
\textbf{Business Value} & Revenue, risk, compliance, satisfaction.\\
\textbf{Dependencies} & Predecessors, external systems, approvals.\\
\textbf{NFRs} & Performance, security, a11y, privacy, reliability.\\
\textbf{Links} & Designs, APIs, docs, analytics dashboards.\\
\end{tabular}

% ============================================================
\section{Reference: Writing Effective Acceptance Criteria}
% ============================================================
\begin{itemize}
  \item Prefer \textbf{observable outcomes}: what the user/system sees.
  \item One behavior per scenario; keep steps short.
  \item Cover \textbf{happy path}, \textbf{negative path}, \textbf{edge cases}.
  \item Include data boundaries, permissions, and error messages.
  \item Tie to \textbf{analytics} and \textbf{operational signals} when useful.
\end{itemize}

\vfill
\begin{center}
\textcolor{Meta}{\footnotesize Template v1.0 --- Duplicate the \texttt{\textbackslash StoryCard} for each backlog item. Keep it user-centered, small, and testable.}
\end{center}

\end{document}
