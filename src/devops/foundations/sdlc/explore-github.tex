%========================================================
% Explore GitHub — Ultra-Practical Cheat Sheet (Drop-in Replacement)
% Detailed study notes with minted code examples
%========================================================
\documentclass[11pt]{article}

% ---------- Encoding & layout ----------
\usepackage[T1]{fontenc}
\usepackage[utf8]{inputenc}
\usepackage[margin=1in]{geometry}
\usepackage{parskip}
\usepackage{microtype}
\usepackage{enumitem}
\setlist{topsep=4pt,itemsep=3pt,parsep=0pt}

% ---------- Colors & links ----------
\usepackage{xcolor}
\definecolor{ink}{HTML}{111827}      % gray-900
\definecolor{accent}{HTML}{2563EB}   % blue-600
\definecolor{soft}{HTML}{F3F4F6}     % gray-100
\usepackage[colorlinks=true,linkcolor=accent,urlcolor=accent]{hyperref}

% ---------- Monospace & quotes ----------
\usepackage{inconsolata}
\usepackage{upquote}

% ---------- Code highlighting ----------
% IMPORTANT: compile with -shell-escape (required by minted)
% ---------- Code (minted; CI-safe fallback) ----------
% If compiled with *unrestricted* -shell-escape and Pygments is available, minted will be used.
% Otherwise, we fall back to listings-based shims that compile in CI (no syntax highlighting).
\usepackage{xparse}

\newif\ifuseminted
\ifnum\pdfshellescape=1\relax
  \usemintedtrue
\else
  \usemintedfalse
\fi

% Floating code listings (optional; keeps \SetupFloatingEnvironment{listing}{...} working)
\usepackage{newfloat}
\usepackage{caption}
\usepackage{float}
\makeatletter
\@ifundefined{c@listing}{\DeclareFloatingEnvironment[fileext=lol,placement={!ht},name=Listing]{listing}}{}
\makeatother

\ifuseminted
  \usepackage[cache=false]{minted}
\else
  \usepackage{listings}

  % Global listings defaults (plain, robust)
  \lstset{
    basicstyle=\ttfamily\small,
    breaklines=true,
    columns=fullflexible,
    frame=single,
    tabsize=2,
    % Common Unicode seen in docs (renders as ASCII / LaTeX equivalents)
    literate=
      {—}{{---}}1
      {–}{{--}}1
      {•}{{$\bullet$}}1
      {…}{{\ldots}}1
      {→}{{$\rightarrow$}}1
      {⇒}{{$\Rightarrow$}}1
      {✓}{{\checkmark}}1
      {✗}{{$\times$}}1
      {“}{{``}}1
      {”}{{''}}1
      {‘}{{`}}1
      {’}{{'}}1
  }

  % minted-compatible shims (options/lang are accepted but intentionally ignored)
  \providecommand{\usemintedstyle}[1]{}
  \providecommand{\setminted}[2][]{}
  \providecommand{\setmintedinline}[2][]{}

  \lstnewenvironment{minted}[2][]{\lstset{}}{}

  \NewDocumentCommand{\inputminted}{ O{} m m }{\lstinputlisting{#3}}

  \NewDocumentCommand{\mintinline}{ O{} m m }{\texttt{#3}}

  % Support for \newminted / \newmintedfile (define environments/commands; options/lang ignored)
  \makeatletter
  \NewDocumentCommand{\newminted}{ O{} m m }{%
    \def\minted@envname{#1}%
    \ifx\minted@envname\@empty
      \edef\minted@envname{#2code}%
    \fi
    \expandafter\lstnewenvironment\expandafter{\minted@envname}[1][]%
      {\lstset{}}{}%
  }
  \NewDocumentCommand{\newmintedfile}{ O{} m m }{%
    \def\minted@cmdname{#1}%
    \ifx\minted@cmdname\@empty
      \edef\minted@cmdname{input#2}%
    \fi
    \expandafter\NewDocumentCommand\csname \minted@cmdname\endcsname{ O{} m }{%
      \lstinputlisting{##2}%
    }%
  }
  \makeatother
\fi
\usemintedstyle{friendly}
\setminted{
  fontsize=\small,
  breaklines,
  breakanywhere,
  autogobble,
  linenos,
  numbersep=6pt,
  frame=lines,
  framesep=2mm
}

% ---------- Simple callout box (no extra packages) ----------
\newcommand{\callout}[1]{%
  \noindent\begingroup
  \fboxsep=8pt
  \colorbox{soft}{\parbox{\dimexpr\linewidth-2\fboxsep}{#1}}%
  \endgroup
}

% ---------- Convenience ----------
\newcommand{\kbd}[1]{\texttt{#1}}
\newcommand{\term}[1]{\textbf{#1}}

\begin{document}
\begin{center}
  {\LARGE \textbf{Explore GitHub — Ultra-Practical Cheat Sheet}}\\[4pt]
  \small Repos, PRs, Actions, Releases, \texttt{gh} CLI, and SemVer you can use today
\end{center}

\section*{1) Platform overview}
\begin{description}[leftmargin=1.5cm,style=nextline]
  \item[Code] Branches, tags, files, clone URLs, Codespaces.
  \item[Issues] Tasks with labels/milestones; link to PRs and commits.
  \item[Pull Requests] Code review, checks, merge strategies.
  \item[Actions] CI/CD: build, test, package, deploy, automate.
  \item[Projects] Boards/tables/timelines across repos.
  \item[Wiki/Pages] Documentation and static sites.
  \item[Security] GHAS: code scanning, Dependabot, secrets, advisories.
  \item[Insights] Contributors, traffic, forks, clones.
  \item[Settings] Permissions, branch protection, secrets/variables, webhooks, apps.
\end{description}

\callout{\textbf{Pro tips.} Star \& Watch important repos; use template repos; initialize with \kbd{README}, language \kbd{.gitignore}, and a license.}
\clearpage

\section*{2) Create a repository quickly}
\begin{enumerate}
  \item \textbf{New repository} $\rightarrow$ name, description, visibility.
  \item Add \kbd{README}, choose language \kbd{.gitignore}, select a license.
  \item \textbf{Create repository}, then use \textbf{Code} button for HTTPS/SSH/CLI.
\end{enumerate}

\subsection*{CLI equivalents (GitHub CLI)}
\begin{minted}{bash}
# Install: https://cli.github.com
gh --version
gh auth login
gh auth status

# Create from scratch (public) with README, license, .gitignore
gh repo create my-repo \
  --public --add-readme \
  --license apache-2.0 \
  --gitignore Python

# Clone (HTTPS or SSH depending on your auth)
gh repo clone OWNER/REPO
\end{minted}

\section*{3) Branching \& commits}
\subsection*{Branch naming}
\begin{itemize}
  \item \kbd{feature/short-description}, \kbd{fix/issue-123}, \kbd{docs/update-contrib}.
  \item Keep branches short-lived; rebase or merge from \kbd{main} frequently.
\end{itemize}

\subsection*{Conventional Commits (recommended)}
\begin{minted}{text}
<type>[optional scope]: <description>

feat: add search box to header
fix(auth): handle expired refresh tokens
docs(readme): add quickstart
refactor(api): collapse duplicate handlers
test(web): add unit tests for navbar
chore(deps): bump axios from 1.6.7 to 1.7.0
\end{minted}
\clearpage

\section*{4) Pull Request workflow (end-to-end)}
\begin{enumerate}
  \item \textbf{Open an issue} or link an existing one.
  \item \textbf{Create a topic branch}, commit in small chunks.
  \item \textbf{Open PR}: crisp title \& body; link issues (\kbd{Fixes \#123}).
  \item \textbf{Request review}; address comments; keep PR focused.
  \item \textbf{Checks green}: unit/integration tests, linters, scanners.
  \item \textbf{Merge}: \emph{Squash} (clean history), \emph{Merge commit} (context), or \emph{Rebase} (linear).
  \item \textbf{Delete branch} if done; confirm deployment/observability.
\end{enumerate}

\subsection*{Useful PR CLI}
\begin{minted}{bash}
# Create and view
gh pr create --base main --title "feat: search box" --body "Adds quick search."
gh pr view --web

# Status, checkout, and merge
gh pr status
gh pr checkout 123
gh pr merge 123 --squash --delete-branch
\end{minted}
\clearpage

\section*{5) Actions: from zero to useful}
\subsection*{Minimal CI for Node (drop-in)}
\begin{minted}{yaml}
# .github/workflows/ci.yml
name: ci
on:
  push:
    branches: [ main ]
  pull_request:
    branches: [ main ]

jobs:
  test:
    runs-on: ubuntu-latest
    strategy:
      matrix:
        node-version: [18, 20]
    steps:
      - uses: actions/checkout@v4
      - uses: actions/setup-node@v4
        with:
          node-version: ${{ matrix.node-version }}
          cache: npm
      - run: npm ci
      - run: npm test --if-present
\end{minted}
\clearpage

\subsection*{Reusable patterns (matrix, cache, artifacts, concurrency)}
\begin{minted}{yaml}
# .github/workflows/build.yml
name: build
on: [push, pull_request]

concurrency:
  group: ${{ github.workflow }}-${{ github.ref }}
  cancel-in-progress: true

jobs:
  build:
    runs-on: ubuntu-latest
    steps:
      - uses: actions/checkout@v4

      # Language toolchain example (Node)
      - uses: actions/setup-node@v4
        with:
          node-version: 20
          cache: npm

      # Build
      - run: npm ci && npm run build

      # Save build output
      - uses: actions/upload-artifact@v4
        with:
          name: web-dist
          path: dist/
          retention-days: 7
\end{minted}
\clearpage

\subsection*{Environment secrets \& variables}
\begin{itemize}
  \item \term{Secrets}: encrypted (e.g., \kbd{NPM\_TOKEN}); use \kbd{\$\{\{ secrets.NAME \}\}}.
  \item \term{Variables}: non-secret config (e.g., \kbd{APP\_ENV}); use \kbd{\$\{\{ vars.NAME \}\}}.
  \item Prefer environment-scoped secrets with required reviewers for production.
\end{itemize}

\subsection*{Manual deploy gate with environments}
\begin{minted}{yaml}
# .github/workflows/deploy.yml
name: deploy
on:
  workflow_dispatch:

jobs:
  deploy-prod:
    runs-on: ubuntu-latest
    environment:
      name: production
      url: https://example.com
    steps:
      - uses: actions/checkout@v4
      - run: ./scripts/deploy.sh
        env:
          API_TOKEN: ${{ secrets.PROD_API_TOKEN }}
\end{minted}
\clearpage

\section*{6) Releases \& tagging}
\subsection*{Tags vs Releases}
\begin{itemize}
  \item \term{Tag}: pointer to a commit (\kbd{git tag v1.2.0}).
  \item \term{Release}: tag \textit{plus} notes, assets, visibility in UI.
\end{itemize}

\subsection*{Create release with CLI}
\begin{minted}{bash}
# From a prepared CHANGELOG and a tag:
git tag v1.2.0
git push origin v1.2.0

# Create a GitHub release (notes from file)
gh release create v1.2.0 \
  --title "v1.2.0" \
  --notes-file CHANGELOG-1.2.0.md \
  ./builds/app-linux-x64.tar.gz#linux \
  ./builds/app-darwin-arm64.tar.gz#mac-arm64
\end{minted}

\subsection*{Auto-generate release notes}
\begin{minted}{bash}
# Let GitHub generate notes based on PRs and commits
gh release create v1.3.0 --generate-notes
\end{minted}

\section*{7) Semantic Versioning (SemVer) in practice}
\begin{tabular}{@{}l l@{}}
\texttt{MAJOR} & breaking changes \\
\texttt{MINOR} & backwards-compatible features \\
\texttt{PATCH} & bug fixes / small improvements \\
\end{tabular}

\subsection*{Dependency ranges (common patterns)}
\begin{itemize}
  \item \kbd{$\geq$1.4.0} \textemdash{} allow anything newer than/equal to baseline.
  \item \kbd{\textasciicircum1.4.0} \textemdash{} allow MINOR/PATCH updates (no new MAJOR).
  \item \kbd{\~{}1.4.0} \textemdash{} allow PATCH updates within the same MINOR.
\end{itemize}
\clearpage

\section*{8) Issues, labels, templates, and Projects}
\subsection*{Issue template (Markdown)}
\begin{minted}{markdown}
---
name: Bug report
about: Create a report to help us improve
labels: bug, needs-triage
---

### Describe the bug
A clear and concise description...

### Steps to reproduce
1. Go to '...'
2. Click on '...'

### Expected behavior
...
\end{minted}

\subsection*{Issue forms (YAML)}
\begin{minted}{yaml}
# .github/ISSUE_TEMPLATE/bug.yml
name: Bug report
description: Report a reproducible problem
labels: [bug, needs-triage]
body:
  - type: textarea
    id: description
    attributes:
      label: Bug description
      placeholder: What happened?
    validations:
      required: true
  - type: input
    id: version
    attributes:
      label: Affected version
      placeholder: e.g., 1.2.3
\end{minted}

\subsection*{PR template (Markdown)}
\begin{minted}{markdown}
## Summary
Brief description of changes and why.

## Checklist
- [ ] Tests added/updated
- [ ] Docs updated
- [ ] Linked issue: Fixes #123
\end{minted}

\subsection*{Labels and triage}
\begin{itemize}
  \item Keep a small, meaningful set (e.g., \kbd{bug}, \kbd{enhancement}, \kbd{docs}, \kbd{deps}).
  \item Use \term{Projects} for planning and triage dashboards across repos.
\end{itemize}

\section*{9) Security essentials (quick wins)}
\begin{itemize}
  \item Enable \term{Dependabot alerts} and \term{security updates}.
  \item Turn on \term{Secret scanning} (push protection where available).
  \item Configure \term{Code scanning} (CodeQL) for primary languages.
  \item Use \term{branch protection}: required reviews, required checks, \kbd{CODEOWNERS}.
\end{itemize}

\subsection*{CODEOWNERS example}
\begin{minted}{text}
# .github/CODEOWNERS
# Order matters; last match wins.
*                  @org/security-team
docs/**            @docs-writers
src/web/**         @frontend-core
src/api/**         @backend-core @api-reviewers
\end{minted}
\clearpage

\section*{10) Advanced \texttt{gh} CLI you’ll actually use}
\subsection*{Auth, repos, issues, PRs}
\begin{minted}{bash}
# Auth
gh auth login
gh auth status

# Repos
gh repo list my-org --limit 100 --visibility internal
gh repo view my-org/my-repo --web

# Issues
gh issue create --title "feat: search facet" \
  --body "Adds facet by category" --label enhancement
gh issue list --label bug --state open

# PRs
gh pr list --search "label:needs-review" --state open
gh pr checks 123
\end{minted}

\subsection*{Actions (workflows \& runs)}
\begin{minted}{bash}
gh workflow list
gh workflow run ci.yml --ref my-feature-branch
gh run list --limit 20
gh run watch --exit-status
\end{minted}

\subsection*{Raw REST API via \texttt{gh api}}
\begin{minted}{bash}
# List open issues as JSON (then filter with jq if installed)
gh api repos/OWNER/REPO/issues --method GET --paginate \
  -F state=open -H "Accept: application/vnd.github+json"
\end{minted}
\clearpage

\section*{11) Repository hygiene}
\subsection*{Recommended \texttt{.gitattributes}}
\begin{minted}{text}
# Normalize line endings
* text=auto eol=lf

# Treat these as text
*.md text
*.yml text
*.yaml text
*.json text

# Binary assets (no diff)
*.png binary
*.jpg binary
*.zip binary
\end{minted}

\subsection*{Useful \texttt{README.md} scaffold (no fenced blocks inside)}
\begin{minted}{markdown}
# Project Name

Short description. One or two sentences.

## Quickstart
    npm ci
    npm run dev

## Scripts
- `npm test` — run test suite
- `npm run build` — production build

## Contributing
See [CONTRIBUTING.md](CONTRIBUTING.md).

## License
Apache-2.0
\end{minted}

\section*{12) Troubleshooting}
\begin{itemize}
  \item \term{PR checks failing}: open the \textbf{Checks} tab; re-run \& read logs.
  \item \term{Auth errors}: \kbd{gh auth status}; verify PAT scopes (\kbd{repo}, \kbd{workflow}).
  \item \term{Missing permissions}: ask a maintainer; check branch protection \& CODEOWNERS.
  \item \term{Actions not running}: confirm \kbd{on:} triggers and branch filters; check concurrency.
  \item \term{Release not visible}: ensure tag exists on origin; use \kbd{gh release create}.
\end{itemize}

\bigskip
\noindent\textit{Adapt this sheet with your team’s conventions (branch names, required checks, deploy envs). Keep PRs small, reviews fast, and releases frequent.}
\end{document}
