% ci-cd-starter-github-actions-ghcr.tex
% CI-safe: No minted, no Pygments, no shell-escape required
\documentclass[11pt]{article}

% === Encoding & Fonts ===
\usepackage[T1]{fontenc}
\usepackage[utf8]{inputenc}
\usepackage{lmodern}
\usepackage{microtype}

% === Page Layout ===
\usepackage[a4paper,margin=1in]{geometry}
\usepackage{parskip}
\usepackage{setspace}
\setstretch{1.1}

% === Colors ===
\usepackage{xcolor}
\definecolor{CodeBg}{HTML}{F9FAFB}
\definecolor{Ink}{HTML}{111827}
\definecolor{Accent}{HTML}{2563EB}

% === Hyperlinks ===
\usepackage{hyperref}
\hypersetup{colorlinks=true, linkcolor=Accent, citecolor=Accent, urlcolor=Accent}

% === Headings ===
\usepackage{titlesec}
\titleformat{\section}{\large\bfseries\color{Ink}}{\thesection}{0.6em}{}
\titleformat{\subsection}{\normalsize\bfseries\color{Ink}}{\thesubsection}{0.6em}{}
\titlespacing*{\section}{0pt}{12pt}{6pt}
\titlespacing*{\subsection}{0pt}{8pt}{4pt}

% === Lists ===
\usepackage{enumitem}
\setlist{itemsep=2pt, topsep=4pt, leftmargin=1.2em}

% === Framed Boxes ===
\usepackage{mdframed}
\mdfdefinestyle{codebox}{
  linecolor=Ink!30,
  linewidth=0.5pt,
  backgroundcolor=CodeBg,
  innerleftmargin=6pt,
  innerrightmargin=6pt,
  innertopmargin=6pt,
  innerbottommargin=6pt,
  skipabove=8pt,
  skipbelow=8pt
}

% === Code Blocks (listings only) ===
\usepackage{listings}
\usepackage{upquote}

\lstset{
  basicstyle=\ttfamily\small,
  backgroundcolor=\color{CodeBg},
  breaklines=true,
  columns=fullflexible,
  keepspaces=true,
  showstringspaces=false,
  frame=none,
  tabsize=2,
  aboveskip=0pt,
  belowskip=0pt
}

\lstnewenvironment{bashcode}{\lstset{language=bash}}{}
\lstnewenvironment{yamlcode}{\lstset{}}{}
\lstnewenvironment{pythoncode}{\lstset{language=Python}}{}
\lstnewenvironment{dockercode}{\lstset{}}{}
\lstnewenvironment{mdcode}{\lstset{}}{}

% === Silence missing .toc on first run ===
\makeatletter
\let\orig@starttoc\@starttoc
\def\@starttoc#1{\IfFileExists{\jobname.#1}{\orig@starttoc{#1}}{}}
\makeatother

% === Document ===
\title{\textbf{CI/CD Starter: GitHub Actions \& GHCR}}
\author{}
\date{}

\begin{document}
\maketitle

\paragraph{What you get}
A clean, drop-in pipeline that mirrors standard stages (lint $\rightarrow$ build/push $\rightarrow$ pull/test), publishes multi-arch images to GitHub Container Registry (GHCR), and uses safe defaults (concurrency, least-privileged permissions, pinned action versions).

\section{Quick Start}

\subsection*{Repository layout}

\begin{mdframed}[style=codebox]
\begin{bashcode}
.
|- hello.py
|- Dockerfile
|- .dockerignore
`- .github/
   `- workflows/
      `- python-pipeline.yml
\end{bashcode}
\end{mdframed}

\subsection*{Files to copy}

\paragraph{hello.py}
\begin{mdframed}[style=codebox]
\begin{pythoncode}
print("Hello, world!")
\end{pythoncode}
\end{mdframed}

\paragraph{Dockerfile}
\begin{mdframed}[style=codebox]
\begin{dockercode}
FROM python:3.13-slim

WORKDIR /app
COPY hello.py .

CMD ["python", "hello.py"]
\end{dockercode}
\end{mdframed}

\paragraph{.dockerignore}
\begin{mdframed}[style=codebox]
\begin{bashcode}
.github
.git
.gitignore
README.md
\end{bashcode}
\end{mdframed}

\paragraph{.github/workflows/python-pipeline.yml}
\begin{mdframed}[style=codebox]
\begin{yamlcode}
name: Python CI/CD Pipeline

on:
  push:
    branches: [ "main" ]
  workflow_dispatch:

concurrency:
  group: ci-${{ github.ref }}
  cancel-in-progress: true

jobs:
  lint-and-test:
    name: Lint and Test
    runs-on: ubuntu-latest
    permissions:
      contents: read
    steps:
      - name: Checkout code
        uses: actions/checkout@v4

      - name: Set up Python
        uses: actions/setup-python@v5
        with:
          python-version: "3.13"

      - name: Install linters
        run: pip install flake8 pylint

      - name: Lint hello.py
        run: |
          flake8 hello.py
          pylint --disable=R,C hello.py

      - name: Run quick test
        run: python hello.py | grep -i 'hello'

  build-and-push:
    name: Build and Push Container
    needs: lint-and-test
    runs-on: ubuntu-latest
    permissions:
      packages: write
      contents: read
    steps:
      - name: Checkout code
        uses: actions/checkout@v4

      - name: Set up Docker Buildx
        uses: docker/setup-buildx-action@v3

      - name: Log in to GHCR
        uses: docker/login-action@v3
        with:
          registry: ghcr.io
          username: ${{ github.actor }}
          password: ${{ secrets.GITHUB_TOKEN }}

      - name: Extract metadata
        id: meta
        uses: docker/metadata-action@v5
        with:
          images: ghcr.io/${{ github.repository }}
          tags: |
            type=sha,format=short
            type=sha,format=long
            type=ref,event=branch

      - name: Build and push
        uses: docker/build-push-action@v5
        with:
          context: .
          push: true
          platforms: linux/amd64,linux/arm64
          tags: ${{ steps.meta.outputs.tags }}
          labels: ${{ steps.meta.outputs.labels }}

  test-image:
    name: Pull and Test Container
    needs: build-and-push
    runs-on: ubuntu-latest
    permissions:
      packages: read
    steps:
      - name: Log in to GHCR
        uses: docker/login-action@v3
        with:
          registry: ghcr.io
          username: ${{ github.actor }}
          password: ${{ secrets.GITHUB_TOKEN }}

      - name: Pull image
        run: docker pull ghcr.io/${{ github.repository }}:${{ github.sha }}

      - name: Run and assert
        run: |
          docker run --rm ghcr.io/${{ github.repository }}:${{ github.sha }} \
            | grep -i 'hello'
\end{yamlcode}
\end{mdframed}

\section{Why This Pipeline}

\subsection{Stages}

\textbf{Lint \& Test} ensure code quality early with fast feedback (flake8, pylint, smoke test). \textbf{Build \& Push} creates a portable artifact (container image) and publishes it to GHCR with metadata tags. \textbf{Pull \& Test} validates the published artifact in a clean runner.

\subsection{Design choices}

\begin{itemize}
  \item \textbf{Concurrency} cancels overlapping runs on the same ref to reduce wasted compute.
  \item \textbf{Least-privilege permissions}: jobs only request what they need.
  \item \textbf{Multi-arch images}: \texttt{linux/amd64,linux/arm64} broadens runtime targets.
  \item \textbf{Deterministic tagging}: tag by commit SHA and branch for traceability.
\end{itemize}

\section{How to Use}

\begin{enumerate}
  \item Create a new repository (optionally include the Python \texttt{.gitignore} template).
  \item Add the four files exactly as shown and commit to \texttt{main}.
  \item Open the \textbf{Actions} tab: confirm the three jobs run and complete successfully.
  \item Open the repo homepage sidebar \textbf{Packages} to see your GHCR image and tags.
\end{enumerate}

\section{Security Notes}

\begin{itemize}
  \item The \texttt{GITHUB\_TOKEN} is scoped to the repository and permits pushing to GHCR.
  \item Avoid broad \texttt{permissions}; do not grant \texttt{packages: write} in jobs that don't push.
  \item Prefer pinning actions to immutable commit SHAs for stricter supply-chain control.
\end{itemize}

\section{Optional: README Badge}

\begin{mdframed}[style=codebox]
\begin{mdcode}
![CI](https://github.com/OWNER/REPO/actions/workflows/python-pipeline.yml/badge.svg?branch=main)
\end{mdcode}
\end{mdframed}

\section{Troubleshooting}

\begin{itemize}
  \item \textbf{403 pushing to GHCR}: ensure job has \texttt{packages: write} and uses \texttt{ghcr.io} as registry.
  \item \textbf{Image not found in test step}: confirm the tag uses \texttt{\$\{\{ github.sha \}\}} and build job succeeded.
  \item \textbf{Docker buildx fails on arm64}: runners provide QEMU; transient errors usually resolve on rerun.
\end{itemize}

\end{document}
