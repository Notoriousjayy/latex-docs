%========================================================
% CI/CD Starter – GitHub Actions + GHCR
% A minimal, production-lean pipeline you can paste into a new repo.
%========================================================
\documentclass[11pt]{article}

% ---------- Encoding & layout ----------
\usepackage[T1]{fontenc}
\usepackage[utf8]{inputenc}
\usepackage[a4paper,margin=1in]{geometry}
\usepackage{microtype}
\usepackage{setspace}
\setstretch{1.12}

% ---------- Colors, links, headings ----------
\usepackage{xcolor}
\definecolor{ink}{HTML}{111827}      % gray-900
\definecolor{soft}{HTML}{F9FAFB}     % gray-50
\definecolor{accent}{HTML}{2563EB}   % blue-600
\definecolor{ok}{HTML}{059669}       % emerald-600
\definecolor{warn}{HTML}{D97706}     % amber-600
\definecolor{bad}{HTML}{DC2626}      % red-600

\usepackage[colorlinks=true,linkcolor=accent,citecolor=accent,urlcolor=accent]{hyperref}
\usepackage{titlesec}
\titleformat{\section}{\large\bfseries\color{ink}}{\thesection}{0.6em}{}
\titleformat{\subsection}{\normalsize\bfseries\color{ink}}{\thesubsection}{0.6em}{}

% ---------- Code blocks (no external pygments required) ----------
\usepackage{listings}
\lstdefinestyle{boxed}{
  basicstyle=\ttfamily\small,
  backgroundcolor=\color{soft},
  columns=flexible,
  keepspaces=true,
  showstringspaces=false,
  breaklines=true,
  breakatwhitespace=false,
  postbreak=\mbox{\textcolor{accent}{$\hookrightarrow$}\space},
  tabsize=2,
  numbers=none,
  frame=none
}
% YAML (simple)
\lstdefinelanguage{yaml}{
  keywords={true,false,null},
  sensitive=false,
  comment=[l]{\#},
  moredelim=[l][\color{accent}]{:},
  morestring=[b]',
  morestring=[b]"
}
% Dockerfile (simple)
\lstdefinelanguage{Docker}{
  keywords={FROM,RUN,COPY,ADD,WORKDIR,CMD,ENTRYPOINT,ENV,EXPOSE,USER,ARG,LABEL,SHELL,STOPSIGNAL,VOLUME,ONBUILD,HEALTHCHECK},
  sensitive=true,
  comment=[l]{\#}
}
% Markdown (for the README badge example)
\lstdefinelanguage{md}{
  morecomment=[l]{\#}
}

\lstset{style=boxed}

% ---------- Framed boxes ----------
\usepackage{mdframed}
\mdfdefinestyle{codebox}{
  linecolor=ink,
  linewidth=0.5pt,
  backgroundcolor=soft,
  innerleftmargin=5pt,
  innerrightmargin=5pt,
  innertopmargin=5pt,
  innerbottommargin=5pt,
  skipabove=0pt,
  skipbelow=0pt,
  leftmargin=0pt,
  rightmargin=0pt
}

% ---------- Title ----------

% ---------- Code blocks (listings only - CI safe) ----------
\usepackage{upquote}

\lstdefinelanguage{yaml}{
  keywords={true,false,null},
  sensitive=false,
  comment=[l]{\#},
  morestring=[b]',
  morestring=[b]"
}

\lstset{
  basicstyle=\ttfamily\small,
  backgroundcolor=\color{soft},
  breaklines=true,
  breakatwhitespace=false,
  columns=fullflexible,
  keepspaces=true,
  showstringspaces=false,
  frame=single,
  framerule=0.4pt,
  tabsize=2,
  aboveskip=6pt,
  belowskip=6pt,
  literate=
    {—}{{---}}1
    {–}{{--}}1
    {→}{{$\rightarrow$}}1
}

% Minted-compatible environments using listings
\lstnewenvironment{minted}[2][]{\lstset{}}{}
\newcommand{\mintinline}[3][]{\texttt{#3}}
\lstnewenvironment{bashcode}{\lstset{language=bash}}{}
\lstnewenvironment{yamlcode}{\lstset{language=yaml}}{}
\lstnewenvironment{jsoncode}{\lstset{}}{}
\lstnewenvironment{cmakecode}{\lstset{}}{}
\lstnewenvironment{textcode}{\lstset{}}{}
\lstnewenvironment{cppcode}{\lstset{language=C++}}{}
\lstnewenvironment{ccode}{\lstset{language=C}}{}
\lstnewenvironment{inicode}{\lstset{}}{}
\lstnewenvironment{pythoncode}{\lstset{language=Python}}{}

\title{\textbf{CI/CD Starter: GitHub Actions \& GHCR}}
\author{}
\date{}

\begin{document}
\maketitle

\paragraph{What you get}
A clean, drop-in pipeline that mirrors standard stages (lint \textrightarrow{} build/push \textrightarrow{} pull/test), publishes multi-arch images to GitHub Container Registry (GHCR), and uses safe defaults (concurrency, least-privileged permissions, pinned action versions by major).

\section{Quick Start}
\subsection*{Repository layout}
\begin{mdframed}[style=codebox]
\begin{lstlisting}[language=bash]
.
|- hello.py
|- Dockerfile
|- .dockerignore
`- .github/
   `- workflows/
      `- python-pipeline.yml
\end{lstlisting}
\end{mdframed}

\subsection*{Files to copy}

\paragraph{hello.py}
\begin{mdframed}[style=codebox]
\begin{lstlisting}[language=Python]
print("Hello, world!")
\end{lstlisting}
\end{mdframed}

\paragraph{Dockerfile}
\begin{mdframed}[style=codebox]
\begin{lstlisting}[language=Docker]
FROM python:3.13-slim

WORKDIR /app
COPY hello.py .

CMD ["python", "hello.py"]
\end{lstlisting}
\end{mdframed}

\paragraph{.dockerignore}
\begin{mdframed}[style=codebox]
\begin{lstlisting}[language=bash]
.github
.git
.gitignore
README.md
\end{lstlisting}
\end{mdframed}

\paragraph{.github/workflows/python-pipeline.yml}
\begin{mdframed}[style=codebox]
\begin{lstlisting}[language=yaml]
name: Python CI/CD Pipeline

on:
  push:
    branches: [ "main" ]
  workflow_dispatch:

# Prevent overlapping runs on the same ref
concurrency:
  group: ci-${{ github.ref }}
  cancel-in-progress: true

jobs:
  lint-and-test:
    name: Lint and Test
    runs-on: ubuntu-latest
    permissions:
      contents: read
    steps:
      - name: Checkout code
        uses: actions/checkout@v4

      - name: Set up Python
        uses: actions/setup-python@v5
        with:
          python-version: "3.13"

      - name: Install linters
        run: pip install flake8 pylint

      - name: Lint hello.py
        run: |
          flake8 hello.py
          pylint --disable=R,C hello.py

      - name: Run quick test
        shell: bash
        run: |
          set -euo pipefail
          python hello.py | grep -i 'hello'

  build-and-push:
    name: Build and Push Container
    needs: lint-and-test
    runs-on: ubuntu-latest
    permissions:
      packages: write     # push to GHCR
      contents: read
    steps:
      - name: Checkout code
        uses: actions/checkout@v4

      - name: Set up Docker Buildx
        uses: docker/setup-buildx-action@v3

      - name: Log in to GitHub Container Registry
        uses: docker/login-action@v3
        with:
          registry: ghcr.io
          username: ${{ github.actor }}
          password: ${{ secrets.GITHUB_TOKEN }}

      - name: Extract Docker metadata
        id: meta
        uses: docker/metadata-action@v5
        with:
          images: ghcr.io/${{ github.repository }}
          tags: |
            type=sha,format=short
            type=sha,format=long
            type=ref,event=branch

      - name: Build and push image
        uses: docker/build-push-action@v5
        with:
          context: .
          push: true
          platforms: linux/amd64,linux/arm64
          tags: ${{ steps.meta.outputs.tags }}
          labels: ${{ steps.meta.outputs.labels }}

  test-image:
    name: Pull and Test Container
    needs: build-and-push
    runs-on: ubuntu-latest
    permissions:
      packages: read       # pull from GHCR
    steps:
      - name: Log in to GitHub Container Registry
        uses: docker/login-action@v3
        with:
          registry: ghcr.io
          username: ${{ github.actor }}
          password: ${{ secrets.GITHUB_TOKEN }}

      - name: Pull image by commit SHA
        run: docker pull ghcr.io/${{ github.repository }}:${{ github.sha }}

      - name: Run image and assert output
        shell: bash
        run: |
          set -euo pipefail
          docker run --rm ghcr.io/${{ github.repository }}:${{ github.sha }} | grep -i 'hello'
\end{lstlisting}
\end{mdframed}

\section{Why this pipeline}
\subsection{Stages}
\textbf{Lint \& Test} ensure code quality early with fast feedback (flake8, pylint, smoke test). \textbf{Build \& Push} creates a portable artifact (container image) and publishes it to GHCR with metadata tags (short SHA, long SHA, branch ref). \textbf{Pull \& Test} validates the published artifact in a clean runner to guarantee deployable integrity.

\subsection{Design choices}
\begin{itemize}
  \item \textbf{Concurrency} cancels overlapping runs on the same ref to reduce wasted compute.
  \item \textbf{Least-privilege permissions}: jobs only request what they need (e.g., \texttt{packages: write} only where pushing).
  \item \textbf{Multi-arch images}: \texttt{linux/amd64,linux/arm64} broadens runtime targets.
  \item \textbf{Deterministic tagging}: tag by commit SHA and branch for traceability.
\end{itemize}

\section{How to use}
\begin{enumerate}
  \item Create a new repository (optionally include the Python \texttt{.gitignore} template).
  \item Add the four files exactly as shown and commit to \texttt{main}.
  \item Open the \textbf{Actions} tab: confirm the three jobs run and complete successfully.
  \item Open the repo homepage sidebar \textbf{Packages} to see your GHCR image and tags.
\end{enumerate}

\section{Security notes}
\begin{itemize}
  \item The \texttt{GITHUB\_TOKEN} is scoped to the repository and permits pushing to GHCR under \texttt{ghcr.io/\{owner\}/\{repo\}} by default.
  \item Avoid broad \texttt{permissions}; do not grant \texttt{packages: write} in jobs that don't push.
  \item Prefer pinning actions to immutable commit SHAs for stricter supply-chain control when you leave the prototype phase.
\end{itemize}

\section{Optional: README badge}
\begin{mdframed}[style=codebox]
\begin{lstlisting}[language=md]
![CI](https://github.com/OWNER/REPO/actions/workflows/python-pipeline.yml/badge.svg?branch=main)
\end{lstlisting}
\end{mdframed}

\section{Troubleshooting}
\begin{itemize}
  \item \textbf{403 pushing to GHCR}: ensure job has \texttt{packages: write} and you use \texttt{ghcr.io} as the registry.
  \item \textbf{Image not found in test step}: confirm the tag uses \texttt{\$\{\{ github.sha \}\}} and the build job succeeded.
  \item \textbf{Docker buildx fails on arm64}: runners provide QEMU; transient errors usually resolve on rerun. Ensure \texttt{setup-buildx-action} ran.
\end{itemize}

\end{document}
