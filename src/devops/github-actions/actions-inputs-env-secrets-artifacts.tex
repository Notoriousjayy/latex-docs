%========================================================
% Using GitHub Actions: actions, inputs, env, secrets & artifacts
% Polished Quick-Reference + Complete Working Example
%========================================================
% Compile with: pdflatex -shell-escape main.tex
% (xelatex -shell-escape or lualatex -shell-escape also work)
\documentclass[11pt]{article}

% ---------- Encoding & layout ----------
\usepackage[T1]{fontenc}
\usepackage[utf8]{inputenc}
\usepackage{lmodern}
\usepackage[a4paper,margin=1in]{geometry}
\usepackage{microtype}
\usepackage{setspace}
\setstretch{1.08}

% ---------- Colors & links ----------
\usepackage{xcolor}
\definecolor{ink}{HTML}{111827}      % gray-900
\definecolor{soft}{HTML}{F9FAFB}     % gray-50
\definecolor{accent}{HTML}{2563EB}   % blue-600
\definecolor{ok}{HTML}{059669}       % emerald-600
\definecolor{warn}{HTML}{D97706}     % amber-600
\definecolor{bad}{HTML}{DC2626}      % red-600
\usepackage[colorlinks=true,linkcolor=accent,urlcolor=accent]{hyperref}

% ---------- Headings ----------
\usepackage{titlesec}
\titleformat*{\section}{\large\bfseries\color{ink}}
\titleformat*{\subsection}{\normalsize\bfseries\color{ink}}
\titleformat*{\subsubsection}{\bfseries\color{ink}}

% ---------- Code (minted) ----------
% Requires: -shell-escape (to run pygmentize)
% ---------- Code (minted; CI-safe fallback) ----------
% If compiled with *unrestricted* -shell-escape and Pygments is available, minted will be used.
% Otherwise, we fall back to listings-based shims that compile in CI (no syntax highlighting).
\usepackage{xparse}

\newif\ifuseminted
\ifnum\pdfshellescape=1\relax
  \usemintedtrue
\else
  \usemintedfalse
\fi

% Floating code listings (optional; keeps \SetupFloatingEnvironment{listing}{...} working)
\usepackage{newfloat}
\usepackage{caption}
\usepackage{float}
\makeatletter
\@ifundefined{c@listing}{\DeclareFloatingEnvironment[fileext=lol,placement={!ht},name=Listing]{listing}}{}
\makeatother

\ifuseminted
  \usepackage{minted-config}
\else
  \usepackage{listings}

  % Global listings defaults (plain, robust)
  \lstset{
    basicstyle=\ttfamily\small,
    breaklines=true,
    columns=fullflexible,
    frame=single,
    tabsize=2,
    % Common Unicode seen in docs (renders as ASCII / LaTeX equivalents)
    literate=
      {—}{{---}}1
      {–}{{--}}1
      {•}{{$\bullet$}}1
      {…}{{\ldots}}1
      {→}{{$\rightarrow$}}1
      {⇒}{{$\Rightarrow$}}1
      {✓}{{\checkmark}}1
      {✗}{{$\times$}}1
      {“}{{``}}1
      {”}{{''}}1
      {‘}{{`}}1
      {’}{{'}}1
  }

  % minted-compatible shims (options/lang are accepted but intentionally ignored)
  \providecommand{\usemintedstyle}[1]{}
  \providecommand{\setminted}[2][]{}
  \providecommand{\setmintedinline}[2][]{}

  \lstnewenvironment{minted}[2][]{\lstset{}}{}

  \NewDocumentCommand{\inputminted}{ O{} m m }{\lstinputlisting{#3}}

  \NewDocumentCommand{\mintinline}{ O{} m m }{\texttt{#3}}

  % Support for \newminted / \newmintedfile (define environments/commands; options/lang ignored)
  \makeatletter
  \NewDocumentCommand{\newminted}{ O{} m m }{%
    \def\minted@envname{#1}%
    \ifx\minted@envname\@empty
      \edef\minted@envname{#2code}%
    \fi
    \expandafter\lstnewenvironment\expandafter{\minted@envname}[1][]%
      {\lstset{}}{}%
  }
  \NewDocumentCommand{\newmintedfile}{ O{} m m }{%
    \def\minted@cmdname{#1}%
    \ifx\minted@cmdname\@empty
      \edef\minted@cmdname{input#2}%
    \fi
    \expandafter\NewDocumentCommand\csname \minted@cmdname\endcsname{ O{} m }{%
      \lstinputlisting{##2}%
    }%
  }
  \makeatother
\fi
\setminted{
  style=friendly,
  autogobble=true,
  breaklines=true,
  fontsize=\small
}

% ---------- Document meta ----------
\title{\textbf{Using GitHub Actions:}\\Actions, Inputs, Env, Secrets \& Artifacts}
\author{Quick-Reference + Paste-and-Run Example}
\date{}

\begin{document}
\maketitle
\vspace{-1.0em}

\noindent\textbf{What you get.} A concise reference for daily GitHub Actions work (using marketplace/local/container actions; passing inputs; working with \texttt{env}, secrets, and artifacts) \emph{plus} a complete, runnable sample you can paste into any repo to verify your pipeline end-to-end.

\section{Overview}
\begin{itemize}
  \item \textbf{Actions}: Reusable steps packaged as Docker containers, JavaScript, or Composite actions.
  \item \textbf{Triggers}: Defined under \texttt{on:} (e.g., \texttt{push}, \texttt{pull\_request}, \texttt{workflow\_dispatch}).
  \item \textbf{Runners}: \texttt{runs-on:} labels such as \texttt{ubuntu-latest}.
  \item \textbf{Context}: Read runtime data via expressions like \texttt{\$\{\{ github.actor \}\}} or \texttt{\$\{\{ env.MY\_VAR \}\}}.
\end{itemize}

\section{Using Actions}
\subsection{Marketplace action}
\begin{minted}[tabsize=2]{yaml}
- uses: actions/checkout@v4
\end{minted}

\subsection{Action from another repo (pin versions)}
\begin{minted}[tabsize=2]{yaml}
- uses: octocat/my-cool-action@v1        # tag
- uses: octocat/my-cool-action@9fceb02   # commit SHA (strongest pin)
- uses: octocat/my-cool-action@main      # branch (avoid in prod)
\end{minted}

\subsection{Action from a subdirectory in a repo}
\begin{minted}[tabsize=2]{yaml}
- uses: octocat/my-cool-action/path/to/action@v2
\end{minted}

\subsection{Local action (same repository)}
\begin{minted}[tabsize=2]{yaml}
- uses: ./.github/actions/lint
\end{minted}

\subsection{Container action (Docker image)}
\begin{minted}[tabsize=2]{yaml}
- uses: docker://python:3.12
# or a registry image:
# - uses: docker://ghcr.io/owner/image:1.2.3
\end{minted}

\section{Inputs (with)}
Pass inputs to an action with \texttt{with:}.
\begin{minted}[tabsize=2]{yaml}
- uses: actions/checkout@v4
  with:
    repository: apache/tomcat
    ref: main
    path: tomcat
\end{minted}

\section{Environment Variables (\texttt{env})}
Define variables at workflow, job, or step scope. Read in shells or expressions.
\subsection{Define at different levels}
\begin{minted}[tabsize=2]{yaml}
name: Example
on: [push]
env:
  PROJECT_NAME: my-app           # workflow-level default

jobs:
  build:
    runs-on: ubuntu-latest
    env:
      NODE_ENV: production       # job-level
    steps:
      - uses: actions/checkout@v4
      - name: Show env
        run: |
          echo "PROJECT_NAME=${PROJECT_NAME}"
          echo "NODE_ENV=${NODE_ENV}"
      - name: Step override
        env:
          NODE_ENV: test         # step-level
        run: echo "NODE_ENV=${NODE_ENV}"
\end{minted}

\subsection{Read syntax}
\begin{itemize}
  \item Bash: \texttt{\$PROJECT\_NAME}
  \item PowerShell: \texttt{\$Env:PROJECT\_NAME}
  \item YAML expression: \texttt{\$\{\{ env.PROJECT\_NAME \}\}}
\end{itemize}

\subsection{Useful built-in env/context}
\texttt{GITHUB\_EVENT\_NAME}, \texttt{GITHUB\_ACTOR}, \texttt{GITHUB\_WORKSPACE}, \texttt{\$\{\{ github.repository \}\}}, \texttt{\$\{\{ github.sha \}\}}.
\clearpage

\section{Secrets}
\begin{itemize}
  \item Store under \textbf{Settings \textrightarrow{} Secrets and variables \textrightarrow{} Actions}.
  \item Use in YAML via \texttt{\$\{\{ secrets.MY\_SECRET \}\}}.
  \item Pass as \texttt{env} or as action inputs. Values are masked in logs.
\end{itemize}

\subsection{Example}
\begin{minted}[tabsize=2]{yaml}
- name: Login to AWS
  env:
    AWS_ACCESS_KEY_ID:     ${{ secrets.AWS_ACCESS_KEY_ID }}
    AWS_SECRET_ACCESS_KEY: ${{ secrets.AWS_SECRET_ACCESS_KEY }}
    AWS_DEFAULT_REGION:    us-east-1
  run: aws sts get-caller-identity
\end{minted}

\section{Artifacts}
Upload outputs to share across jobs or to download from the run.
\subsection{Upload}
\begin{minted}[tabsize=2]{yaml}
- uses: actions/upload-artifact@v4
  with:
    name: build-outputs
    path: ./dist/**
\end{minted}

\subsection{Download}
\begin{minted}[tabsize=2]{yaml}
- uses: actions/download-artifact@v4
  with:
    name: build-outputs
\end{minted}

\subsection{Notes}
Retention defaults (often 90 days) and storage quotas apply. Names must match between upload and download.

% -------------------------------------------------------
\section{Complete, Working Example (Paste \& Run)}
This example uses \textbf{Bun} to build a tiny executable and demonstrates:
\emph{actions}, \emph{env}, \emph{artifacts}, and \emph{job dependencies}. Paste these files into your repo and push.

\subsection*{Project files}
\begin{itemize}
  \item \texttt{random-number-generator.js}
  \item \texttt{package.json}
  \item \texttt{.gitignore}
  \item \texttt{.github/workflows/create-artifacts.yml}
\end{itemize}

\subsection{\texttt{random-number-generator.js}}
\begin{minted}[tabsize=2]{javascript}
#!/usr/bin/env bun
/**
 * Deterministic-ish RNG runner:
 *  - No arg      -> random numbers with time seed
 *  - String arg  -> hashed to seed
 *  - Number arg  -> used as seed
 * Appends a short report to ./report.txt
 */

function xfnv1a(str) {
  let h = 2166136261 >>> 0;
  for (let i = 0; i < str.length; i++) {
    h ^= str.charCodeAt(i);
    h = Math.imul(h, 16777619);
  }
  return h >>> 0;
}
function mulberry32(seed) {
  let a = seed >>> 0;
  return function () {
    a |= 0;
    a = (a + 0x6D2B79F5) | 0;
    let t = Math.imul(a ^ (a >>> 15), 1 | a);
    t ^= t + Math.imul(t ^ (t >>> 7), 61 | t);
    return ((t ^ (t >>> 14)) >>> 0) / 4294967296;
  };
}

const arg = process.argv[2];
let seed;
if (arg === undefined) {
  seed = (Date.now() ^ (Math.random() * 1e9)) >>> 0;
} else if (!Number.isNaN(Number(arg))) {
  seed = Number(arg) >>> 0;
} else {
  seed = xfnv1a(String(arg));
}
const rand = mulberry32(seed);

// Produce a tiny report (3 numbers, 0–100)
const nums = Array.from({ length: 3 }, () => Math.floor(rand() * 101));
const mode = arg === undefined ? "no-seed" : (Number.isNaN(Number(arg)) ? `seed:${arg}` : `seed:${Number(arg)}`);
const line = `[${new Date().toISOString()}] (${mode}) -> ${nums.join(", ")}\n`;

await Bun.write("report.txt", line, { append: true });
console.log(line.trim());
\end{minted}
\clearpage

\subsection{\texttt{package.json}}
\begin{minted}[tabsize=2]{json}
{
  "name": "create-artifacts",
  "version": "1.0.0",
  "private": true,
  "scripts": {
    "build": "bun build --compile ./random-number-generator.js --outfile random-number-generator-linux",
    "test": "bun run random-number-generator.js && bun run random-number-generator.js \"test-seed\" && bun run random-number-generator.js 1906"
  }
}
\end{minted}

\subsection{\texttt{.github/workflows/create-artifacts.yml}}
\begin{minted}[tabsize=2]{yaml}
name: Create Artifacts
on: [push, workflow_dispatch]

jobs:
  test:
    name: Test Code
    runs-on: ubuntu-latest
    steps:
      - name: Checkout
        uses: actions/checkout@v4

      - name: Setup Bun
        uses: oven-sh/setup-bun@v2
        with:
          bun-version: latest

      - name: Install dependencies
        run: bun install

      - name: Test with different seeds
        run: |
          echo "Testing with no seed..."
          bun run random-number-generator.js
          echo

          echo "Testing with string seed..."
          bun run random-number-generator.js "test-seed"
          echo

          echo "Testing with numeric seed..."
          bun run random-number-generator.js 1906
          echo "Generated report:"
          cat report.txt

      - name: Upload test report
        uses: actions/upload-artifact@v4
        with:
          name: test-report
          path: report.txt

  build:
    name: Build Executables
    runs-on: ubuntu-latest
    needs: test
    steps:
      - name: Checkout
        uses: actions/checkout@v4

      - name: Setup Bun
        uses: oven-sh/setup-bun@v2
        with:
          bun-version: latest

      - name: Install dependencies
        run: bun install

      - name: Build Linux executable
        run: bun run build

      - name: List executables
        run: ls -la random-number-generator-*

      - name: Upload executables
        uses: actions/upload-artifact@v4
        with:
          name: random-number-generator-executables
          path: ./random-number-generator-*

  test-linux:
    name: Test Linux Executable
    runs-on: ubuntu-latest
    needs: build
    steps:
      - name: Download executables
        uses: actions/download-artifact@v4
        with:
          name: random-number-generator-executables

      - name: Make binary executable
        run: chmod +x random-number-generator-linux

      - name: Run smoke tests
        run: |
          echo "Testing with no seed..."
          ./random-number-generator-linux
          echo

          echo "Testing with string seed..."
          ./random-number-generator-linux "test-seed"
          echo

          echo "Testing with numeric seed..."
          ./random-number-generator-linux 1906
          echo

      - name: Upload linux test report
        uses: actions/upload-artifact@v4
        with:
          name: linux-test-report
          path: report.txt
\end{minted}

\subsection{\texttt{.gitignore}}
\begin{minted}[tabsize=2]{text}
# Prevent project files from being tracked by git
*.bun-build
bun.lock
report.txt
package-lock.json
random-number-generator-*
\end{minted}

\subsection*{Run it}
\begin{enumerate}
  \item Commit the four files and push to any branch.
  \item Open \textbf{Actions} in GitHub; select the latest run.
  \item Verify three artifacts exist:
  \begin{itemize}
    \item \texttt{test-report}
    \item \texttt{random-number-generator-executables}
    \item \texttt{linux-test-report}
  \end{itemize}
\end{enumerate}

% -------------------------------------------------------
\section{Common Pitfalls \& Quick Fixes}
\begin{itemize}
  \item \textbf{Indentation / misplaced steps:} Ensure steps are under \texttt{jobs.<job>.steps}.
  \item \textbf{Unpinned actions:} Use a tag or SHA; avoid \texttt{@main} for reliability.
  \item \textbf{Secrets not visible:} Expected---GitHub masks values. Pass via \texttt{env} or \texttt{with}.
  \item \textbf{Artifacts missing:} Confirm matching \texttt{name} on upload/download and \texttt{needs:} between jobs.
  \item \textbf{Windows env syntax:} Use \texttt{\$Env:NAME} in PowerShell vs \texttt{\$NAME} in Bash.
  \item \textbf{Multiple OS binaries:} Start with Linux; add matrix builds later if you need macOS/Windows.
\end{itemize}

% -------------------------------------------------------
\section{Build Notes (minted)}
\begin{itemize}
  \item Compile with \texttt{-shell-escape} so \texttt{minted} can call \texttt{pygmentize}.
  \item If your LaTeX toolchain caches styles, avoid \texttt{frozencache} unless you manage the cache files.
  \item The examples use only ASCII to avoid Unicode issues in some LaTeX setups.
\end{itemize}

\vfill
\noindent\textit{Tip:} Swap \texttt{actions/checkout@v4} for a pinned SHA in regulated environments, and pin \texttt{oven-sh/setup-bun@v2} to a tag/commit as well.

\end{document}

