%========================================================
% GitHub Actions & Workflows – Key Patterns and Recipes
% Drop-in, production-ready workflow examples
%========================================================
\documentclass[11pt]{article}

% ---------- Encoding & layout ----------
\usepackage[T1]{fontenc}
\usepackage[utf8]{inputenc}
\usepackage{lmodern}
\usepackage[a4paper,margin=1in]{geometry}
\usepackage{microtype}
\usepackage{parskip}      % nicer paragraphs

% ---------- Links ----------
\usepackage[hidelinks]{hyperref}
% ---------- Silence first-run .toc warnings (latexmk treats them as missing input) ----------
\usepackage{etoolbox}
\makeatletter
\patchcmd{\@starttoc}
  {\InputIfFileExists{\jobname.#1}{}{\typeout{No file \jobname.#1.}}}
  {\InputIfFileExists{\jobname.#1}{}{}}
  {}{}
\makeatother
% ---------- End .toc silence ----------

% ---------- Minted (code highlighting) ----------
% NOTE: compile with -shell-escape (or --shell-escape)
% ---------- Minted compatibility layer (CI-safe) ----------
% This document can render code with minted when -shell-escape is enabled, and
% falls back to listings when it is not (so CI builds do not hard-fail).
\newif\ifUseMinted
\UseMintedfalse
\begingroup
\ifdefined\pdfshellescape
  \ifnum\pdfshellescape=1\relax
    \global\UseMintedtrue
  \fi
\fi
\endgroup

\ifUseMinted
  \usepackage[newfloat,cache=false]{minted}
\else
  \usepackage{listings}
  \usepackage{xcolor}
  \usepackage{newfloat}
  \usepackage{fancyvrb} % provides \VerbatimEnvironment used by some wrappers

  % Provide a "listing" float compatible with minted's newfloat option
  \makeatletter
  \@ifundefined{c@listing}{%
    \DeclareFloatingEnvironment[name=Listing]{listing}
  }{}
  \makeatother

  % Minimal language definitions for common "minted" lexers / labels used in docs.
  % These are intentionally lightweight; they exist primarily to avoid build failures.
  \lstdefinelanguage{yaml}{
    sensitive=true,
    morecomment=[l]{\#},
    morestring=[b]",
    morestring=[b]',
  }
  \lstdefinelanguage{json}{
    sensitive=true,
    morestring=[b]",
    showstringspaces=false,
  }
  \lstdefinelanguage{ini}{
    sensitive=true,
    morecomment=[l]{;},
    morecomment=[l]{\#},
    morestring=[b]",
    morestring=[b]',
  }
  \lstdefinelanguage{cmake}{
    sensitive=true,
    morecomment=[l]{\#},
    morestring=[b]",
    morestring=[b]',
  }
  \lstdefinelanguage{powershell}{
    sensitive=true,
    morecomment=[l]{\#},
    morestring=[b]",
    morestring=[b]',
    morekeywords={param,begin,process,end,function,filter,return,if,elseif,else,foreach,for,while,do,until,break,continue,try,catch,finally,throw,switch},
  }
  \lstdefinelanguage{bash}{
    sensitive=true,
    morecomment=[l]{\#},
    morestring=[b]",
    morestring=[b]',
    morekeywords={if,then,else,elif,fi,for,do,done,while,in,case,esac,function,local,return,export,unset,echo,printf,read,cd,exit},
  }
  \lstdefinelanguage{sh}{sensitive=true, morecomment=[l]{\#}, morestring=[b]", morestring=[b]'}
  \lstdefinelanguage{shell}{sensitive=true, morecomment=[l]{\#}, morestring=[b]", morestring=[b]'}
  \lstdefinelanguage{console}{sensitive=false}
  \lstdefinelanguage{terminal}{sensitive=false}
  \lstdefinelanguage{md}{sensitive=false}
  \lstdefinelanguage{markdown}{sensitive=false}
  \lstdefinelanguage{text}{sensitive=false}

  % Reasonable defaults; keep this conservative to avoid surprises.
  \lstset{
    basicstyle=\ttfamily\small,
    breaklines=true,
    columns=fullflexible,
    keepspaces=true,
    showstringspaces=false,
    upquote=true,
    frame=single,
    framerule=0.2pt,
    aboveskip=0.75\baselineskip,
    belowskip=0.75\baselineskip,
    % Common Unicode glyphs seen in snippets
    literate=
      {•}{{\textbullet}}1
      {—}{{---}}1
      {–}{{--}}1
      {→}{{$\rightarrow$}}1
      {←}{{$\leftarrow$}}1
      {≥}{{$\ge$}}1
      {≤}{{$\le$}}1
  }

  % Minted command shims (ignore style/options in fallback)
  \providecommand{\usemintedstyle}[1]{}
  \providecommand{\setminted}[1]{}
  \providecommand{\setmintedinline}[1]{}

  % NOTE: minted's optional key-value options are not 1:1 with listings'
  % key-value options. To avoid hard failures, we intentionally ignore the
  % optional options argument in the fallback path.

  % minted: \begin{minted}[<opts>]{<lang>} ... \end{minted}
  \lstnewenvironment{minted}[2][]%
    {\lstset{language=#2}}%
    {}

  % minted: \inputminted[<opts>]{<lang>}{<file>}
  \newcommand{\inputminted}[3][]{\lstinputlisting[language=#2]{#3}}

  % minted: \mintinline{<lang>}{<code>}
  \newcommand{\mintinline}[2]{\texttt{#2}}

  % minted: \newminted[<envname>]{<lang>}{<opts>}
  \newcommand{\newminted}[3][]{%
    \def\MintedEnvName{#1}%
    \if\relax\detokenize{#1}\relax
      \edef\MintedEnvName{#2code}%
    \fi
    \expandafter\lstnewenvironment\expandafter{\MintedEnvName}[1][]%
      {\lstset{language=#2}}%
      {}%
  }

  % minted: \newmintedfile{<lang>}{<opts>} -> \input<lang>{file}
  \newcommand{\newmintedfile}[2]{%
    \expandafter\newcommand\csname input#1\endcsname[2][]{\lstinputlisting[language=#1]{##2}}%
  }

  % minted exposes \listoflistings; keep it defined for compatibility
  \providecommand{\listoflistings}{\listof{listing}{List of Listings}}
\fi
% ---------- End minted compatibility layer ----------
\usepackage{float}  % for [H] placement
\usemintedstyle{tango} % e.g., friendly, monokai, tango
\setminted{
  fontsize=\small,
  breaklines=true,
  breaksymbolleft={}
}
\SetupFloatingEnvironment{listing}{name=Listing}

% Convenience environments (no caption option here)
\newminted[yamlcode]{yaml}{
  linenos,
  numbersep=6pt,
  tabsize=2
}
\newminted[bashcode]{bash}{
  linenos,
  numbersep=6pt
}
\newminted[pscode]{powershell}{
  linenos,
  numbersep=6pt
}
\newminted[gocode]{go}{
  linenos,
  numbersep=6pt
}

% ---------- Title ----------
\title{\textbf{GitHub Actions \& Workflows}\\Key Patterns \& Drop-in Recipes}
\author{}
\date{}

\begin{document}
\maketitle

\section*{How to use this document}
Copy the snippets into your repository under \texttt{.github/workflows/}. Each recipe is crafted to be production-ready and highlights a specific pattern (matrix builds, path filters, concurrency, reusable workflows, etc.). All code blocks use \texttt{minted}; compile with \texttt{-shell-escape}.

\section{Workflow Anatomy (90-second refresher)}
\begin{itemize}
  \item \textbf{\texttt{name}}: Human-readable workflow label (optional).
  \item \textbf{\texttt{on}}: Triggers (e.g., \texttt{push}, \texttt{pull\_request}, \texttt{workflow\_dispatch}, \texttt{schedule}, \texttt{workflow\_run}).
  \item \textbf{\texttt{jobs}}: One or more jobs; each runs on a runner (e.g., \texttt{ubuntu-latest}).
  \item \textbf{\texttt{steps}}: Inside jobs; \texttt{uses} an action or \texttt{run}s shell commands.
  \item \textbf{\texttt{needs}}: Job dependencies (enforces order).
  \item \textbf{\texttt{strategy.matrix}}: Expand a job across OS/language versions etc.
  \item \textbf{\texttt{env}/\texttt{secrets}}: Environment variables and secrets.
  \item \textbf{\texttt{permissions}}: Token scopes (\emph{least privilege}).
  \item \textbf{\texttt{concurrency}}: Cancel in-flight runs for a branch/PR.
\end{itemize}

\newpage
\section{Recipe 1 – First Workflow: two jobs, multi-OS sanity check}
\begin{listing}[H]
\caption{Basic two-job workflow to verify runners, shells, and checkout}
\begin{yamlcode}
name: First Workflow
on: [push]

jobs:
  linux:
    name: Linux env
    runs-on: ubuntu-latest
    steps:
      - uses: actions/checkout@v4
      - name: Print env (bash)
        run: env | sort

  windows:
    name: Windows env
    runs-on: windows-latest
    steps:
      - uses: actions/checkout@v4
      - name: Print env (PowerShell)
        run: Get-ChildItem Env: | Sort-Object Name
\end{yamlcode}
\end{listing}

\newpage
\section{Recipe 2 – Matrix Build \& Test (Go example)}
\begin{listing}[H]
\caption{Matrix build across Ubuntu/macOS/Windows with Go toolchain cache}
\begin{yamlcode}
name: CI (Go matrix)
on:
  push:
    branches: [main]
  pull_request:

jobs:
  build:
    name: Build & Test (${{ matrix.os }})
    runs-on: ${{ matrix.os }}
    strategy:
      fail-fast: false
      matrix:
        os: [ubuntu-latest, macos-latest, windows-latest]
        include:
          - os: ubuntu-latest
            bin: main
            runbin: ./main
          - os: macos-latest
            bin: main
            runbin: ./main
          - os: windows-latest
            bin: main.exe
            runbin: .\main.exe
    steps:
      - uses: actions/checkout@v4
      - uses: actions/setup-go@v5
        with:
          go-version: '1.21'
          cache: true
      - name: Build
        run: go build -o ${{ matrix.bin }} ./...
      - name: Unit tests
        run: go test ./... -count=1 -race -v
      - name: Run binary smoke test
        run: ${{ matrix.runbin }} --help
\end{yamlcode}
\end{listing}
\newpage

\paragraph{Optional \texttt{main.go} for smoke test}
\begin{gocode}
package main

import (
  "flag"
  "fmt"
)

func main() {
  help := flag.Bool("help", false, "show help")
  flag.Parse()
  if *help {
    fmt.Println("demo app: flags: --help")
    return
  }
  fmt.Println("hello from CI")
}
\end{gocode}

\newpage
\section{Recipe 3 – Cross-compile \& Publish Artifacts}
\begin{listing}[H]
\caption{Cross-compile after matrix (\texttt{needs: build}) and upload artifacts}
\begin{yamlcode}
name: Build + Cross-Compile
on:
  push:
    branches: [main]

jobs:
  build:
    runs-on: ubuntu-latest
    steps:
      - uses: actions/checkout@v4
      - uses: actions/setup-go@v5
        with:
          go-version: '1.21'
          cache: true
      - run: go build -o main ./...

  cross:
    runs-on: ubuntu-latest
    needs: build
    steps:
      - uses: actions/checkout@v4
      - uses: actions/setup-go@v5
        with:
          go-version: '1.21'
          cache: true
      - name: Cross-compile
        shell: bash
        run: |
          set -euo pipefail
          mkdir -p dist
          GOOS=linux   GOARCH=amd64 go build -o dist/app-linux-amd64 ./...
          GOOS=darwin  GOARCH=arm64 go build -o dist/app-macos-arm64 ./...
          GOOS=windows GOARCH=amd64 go build -o dist/app-windows-amd64.exe ./...
      - name: Upload artifacts
        uses: actions/upload-artifact@v4
        with:
          name: app-${{ github.sha }}
          path: dist/*
\end{yamlcode}
\end{listing}

\newpage
\section{Recipe 4 – Branch, Tag, and Path Filters}
\begin{listing}[H]
\caption{Precise event filters for branches, tags, and paths}
\begin{yamlcode}
name: Filtered CI
on:
  push:
    branches: ["main", "release/*"]
    tags: ["v*"]
    paths:
      - "cmd/**"
      - "internal/**"
      - "!docs/**"
  pull_request:
    branches: ["main"]
    paths-ignore:
      - "docs/**"
\end{yamlcode}
\end{listing}

\newpage
\section{Recipe 5 – Concurrency (Cancel In-Progress)}
\begin{listing}[H]
\caption{Concurrency to avoid duplicate long-running jobs}
\begin{yamlcode}
name: Lint & Test
on: [push, pull_request]

concurrency:
  group: ${{ github.workflow }}-${{ github.ref }}
  cancel-in-progress: true

jobs:
  ci:
    runs-on: ubuntu-latest
    steps:
      - uses: actions/checkout@v4
      - run: echo "do work"
\end{yamlcode}
\end{listing}

\newpage
\section{Recipe 6 – Least-Privilege \texttt{permissions}}
\begin{listing}[H]
\caption{Lock down \texttt{GITHUB\_TOKEN} scopes}
\begin{yamlcode}
name: Secure Permissions
on: [pull_request]

permissions:
  contents: read
  pull-requests: write   # e.g., workflow needs to comment on PRs

jobs:
  annotate:
    runs-on: ubuntu-latest
    steps:
      - uses: actions/checkout@v4
      - name: Add PR comment (example)
        uses: marocchino/sticky-pull-request-comment@v2
        with:
          message: "CI results are in!"
\end{yamlcode}
\end{listing}

\newpage
\section{Recipe 7 – Cache Dependencies (generic)}
\begin{listing}[H]
\caption{Generic caching with a robust key and restore keys}
\begin{yamlcode}
name: Node CI
on: [push, pull_request]

jobs:
  ci:
    runs-on: ubuntu-latest
    steps:
      - uses: actions/checkout@v4

      - name: Use Node
        uses: actions/setup-node@v4
        with:
          node-version: '20.x'

      - name: Cache npm
        uses: actions/cache@v4
        with:
          path: ~/.npm
          key: npm-${{ runner.os }}-${{ hashFiles('**/package-lock.json') }}
          restore-keys: |
            npm-${{ runner.os }}-

      - run: npm ci
      - run: npm test -- --ci
\end{yamlcode}
\end{listing}

\newpage
\section{Recipe 8 – Artifacts and Build Outputs Between Jobs}
\begin{listing}[H]
\caption{Upload in one job, download in a dependent job}
\begin{yamlcode}
name: Build → E2E
on: [pull_request]

jobs:
  build:
    runs-on: ubuntu-latest
    steps:
      - uses: actions/checkout@v4
      - run: npm ci && npm run build
      - uses: actions/upload-artifact@v4
        with:
          name: web-dist
          path: dist/

  e2e:
    runs-on: ubuntu-latest
    needs: build
    steps:
      - uses: actions/download-artifact@v4
        with:
          name: web-dist
          path: dist/
      - run: npx playwright install --with-deps
      - run: npx playwright test
\end{yamlcode}
\end{listing}

\newpage
\section{Recipe 9 – Reusable Workflows (\texttt{workflow\_call})}
\paragraph{1) Reusable workflow (in same repo)} Save as \texttt{.github/workflows/reusable-ci.yml}.
\begin{yamlcode}
name: Reusable CI
on:
  workflow_call:
    inputs:
      node:
        required: true
        type: string

jobs:
  ci:
    runs-on: ubuntu-latest
    steps:
      - uses: actions/checkout@v4
      - uses: actions/setup-node@v4
        with:
          node-version: ${{ inputs.node }}
      - run: npm ci && npm test
\end{yamlcode}

\paragraph{2) Caller workflow}
\begin{yamlcode}
name: App CI
on: [push, pull_request]

jobs:
  call-shared:
    uses: ./.github/workflows/reusable-ci.yml
    with:
      node: "20"
\end{yamlcode}

\newpage
\section{Recipe 10 – Scheduled and Manual Triggers}
\begin{listing}[H]
\caption{Nightly maintenance and manual \texttt{workflow\_dispatch}}
\begin{yamlcode}
name: Maintenance
on:
  schedule:
    - cron: "17 3 * * *"   # 03:17 UTC daily
  workflow_dispatch:
    inputs:
      task:
        description: "Choose a task"
        required: true
        type: choice
        options: [vacuum-db, refresh-caches]

jobs:
  run-task:
    runs-on: ubuntu-latest
    steps:
      - run: echo "Running ${{ github.event.inputs.task }} ..."
\end{yamlcode}
\end{listing}

\newpage
\section{Recipe 11 – Safer PRs from Forks (\texttt{pull\_request\_target})}
\begin{listing}[H]
\caption{Guarded use of \texttt{pull\_request\_target}}
\begin{yamlcode}
name: PR Labeler (safe)
on:
  pull_request_target:
    types: [opened, synchronize, reopened]

permissions:
  contents: read
  pull-requests: write

jobs:
  label:
    runs-on: ubuntu-latest
    steps:
      - name: Label by title
        uses: actions-ecosystem/action-add-labels@v1
        with:
          github_token: ${{ secrets.GITHUB_TOKEN }}
          labels: |
            needs-triage
\end{yamlcode}
\end{listing}

\newpage
\section{Recipe 12 – Monorepo Path Filters (per package)}
\begin{listing}[H]
\caption{Monorepo CI per package directory}
\begin{yamlcode}
name: Package A CI
on:
  push:
    branches: [main]
    paths:
      - "packages/pkg-a/**"
  pull_request:
    paths:
      - "packages/pkg-a/**"

jobs:
  test-a:
    runs-on: ubuntu-latest
    steps:
      - uses: actions/checkout@v4
      - run: cd packages/pkg-a && npm ci && npm test
\end{yamlcode}
\end{listing}

\newpage
\section{Recipe 13 – Conditional Steps/Jobs (\texttt{if:})}
\begin{listing}[H]
\caption{Conditionally skip when only docs changed}
\begin{yamlcode}
name: Conditional Work
on: [push, pull_request]

jobs:
  build:
    runs-on: ubuntu-latest
    steps:
      - uses: actions/checkout@v4

      - name: Detect docs-only
        id: changed
        run: |
          git fetch --depth=2 origin ${{ github.base_ref || 'HEAD~1' }}
          CHANGED=$(git diff --name-only HEAD^ HEAD | tr -d '\r')
          echo "changed=$CHANGED" >> $GITHUB_OUTPUT

      - name: Skip if docs-only
        if: ${{ startsWith(steps.changed.outputs.changed, 'docs/') }}
        run: echo "Docs-only change; skipping build."

      - name: Build
        if: ${{ !startsWith(steps.changed.outputs.changed, 'docs/') }}
        run: echo "Do the real build..."
\end{yamlcode}
\end{listing}

\newpage
\section{Recipe 14 – Self-Hosted Runners (labels and timeouts)}
\begin{listing}[H]
\caption{Self-hosted with custom labels and timeout}
\begin{yamlcode}
name: Self-Hosted CI
on: [push]

jobs:
  heavy:
    runs-on: [self-hosted, linux, x64, gpu]
    timeout-minutes: 60
    steps:
      - uses: actions/checkout@v4
      - run: nvidia-smi || true
\end{yamlcode}
\end{listing}

\newpage
\section{Common Gotchas (quick checklist)}
\begin{itemize}
  \item \textbf{\texttt{needs}} references \emph{job IDs} (the YAML keys), not \texttt{name}.
  \item Default shells differ: bash on Linux/macOS; PowerShell on Windows.
  \item Put per-event branch/path filters under each event key inside \texttt{on:}.
  \item Use \texttt{concurrency} to cancel old runs on the same branch/PR.
  \item Lock down \texttt{permissions}; grant elevated scopes only where needed.
  \item For forked PRs, prefer \texttt{pull\_request}; use \texttt{pull\_request\_target} only for trusted, no-checkout workflows.
  \item Artifact size and log size are limited—upload just what you need.
\end{itemize}

\section*{Appendix – Local commands you may find handy}
\begin{bashcode}
# Validate YAML locally (requires yq or yamllint)
yamllint .github/workflows

# Render your matrix to see what would run (pseudo; GitHub handles expansion)
# Tip: keep matrix small and explicit when in doubt.
\end{bashcode}

\section*{PowerShell snippet (Windows runner tips)}
\begin{pscode}
# Show environment variables
Get-ChildItem Env: | Sort-Object Name

# Fail a step explicitly
Write-Error "Stopping this step on purpose."
\end{pscode}

\end{document}