%========================================================
% GitHub Actions — Practical Cheat Sheet
%========================================================
\documentclass[11pt]{article}

% ---------- Encoding & layout ----------
\usepackage[T1]{fontenc}
\usepackage[utf8]{inputenc}
\usepackage{lmodern}
\usepackage[margin=1in]{geometry}
\usepackage{microtype}
\usepackage{enumitem}
\setlist{nosep,leftmargin=1.4em}
\usepackage{array}
\usepackage{booktabs}
\usepackage{pifont}

% ---------- Colors, links, boxes ----------
\usepackage{xcolor}
\definecolor{Ink}{HTML}{111827}      % gray-900
\definecolor{Soft}{HTML}{F9FAFB}     % gray-50
\definecolor{Accent}{HTML}{2563EB}   % blue-600
\definecolor{OK}{HTML}{059669}       % emerald-600
\definecolor{Warn}{HTML}{D97706}     % amber-600
\definecolor{Bad}{HTML}{DC2626}      % red-600

\usepackage{hyperref}
\hypersetup{
  colorlinks=true,
  linkcolor=Accent,
  urlcolor=Accent,
  citecolor=Accent
}

\usepackage[most]{tcolorbox}
\tcbset{boxrule=0.5pt, colframe=Ink, colback=Soft, arc=2pt, left=8pt,right=8pt,top=6pt,bottom=6pt}

% ---------- Code (minted) ----------
% NOTE: Compile with: latexmk -pdf -shell-escape "<filename>.tex"
% ---------- Code (minted; CI-safe fallback) ----------
% If compiled with *unrestricted* -shell-escape and Pygments is available, minted will be used.
% Otherwise, we fall back to listings-based shims that compile in CI (no syntax highlighting).
\usepackage{xparse}

\newif\ifuseminted
\ifnum\pdfshellescape=1\relax
  \usemintedtrue
\else
  \usemintedfalse
\fi

% Floating code listings (optional; keeps \SetupFloatingEnvironment{listing}{...} working)
\usepackage{newfloat}
\usepackage{caption}
\usepackage{float}
\makeatletter
\@ifundefined{c@listing}{\DeclareFloatingEnvironment[fileext=lol,placement={!ht},name=Listing]{listing}}{}
\makeatother

\ifuseminted
  \usepackage[cache=false]{minted}
\else
  \usepackage{listings}

  % Global listings defaults (plain, robust)
  \lstset{
    basicstyle=\ttfamily\small,
    breaklines=true,
    columns=fullflexible,
    frame=single,
    tabsize=2,
    % Common Unicode seen in docs (renders as ASCII / LaTeX equivalents)
    literate=
      {—}{{---}}1
      {–}{{--}}1
      {•}{{$\bullet$}}1
      {…}{{\ldots}}1
      {→}{{$\rightarrow$}}1
      {⇒}{{$\Rightarrow$}}1
      {✓}{{\checkmark}}1
      {✗}{{$\times$}}1
      {“}{{``}}1
      {”}{{''}}1
      {‘}{{`}}1
      {’}{{'}}1
  }

  % minted-compatible shims (options/lang are accepted but intentionally ignored)
  \providecommand{\usemintedstyle}[1]{}
  \providecommand{\setminted}[2][]{}
  \providecommand{\setmintedinline}[2][]{}

  \lstnewenvironment{minted}[2][]{\lstset{}}{}

  \NewDocumentCommand{\inputminted}{ O{} m m }{\lstinputlisting{#3}}

  \NewDocumentCommand{\mintinline}{ O{} m m }{\texttt{#3}}

  % Support for \newminted / \newmintedfile (define environments/commands; options/lang ignored)
  \makeatletter
  \NewDocumentCommand{\newminted}{ O{} m m }{%
    \def\minted@envname{#1}%
    \ifx\minted@envname\@empty
      \edef\minted@envname{#2code}%
    \fi
    \expandafter\lstnewenvironment\expandafter{\minted@envname}[1][]%
      {\lstset{}}{}%
  }
  \NewDocumentCommand{\newmintedfile}{ O{} m m }{%
    \def\minted@cmdname{#1}%
    \ifx\minted@cmdname\@empty
      \edef\minted@cmdname{input#2}%
    \fi
    \expandafter\NewDocumentCommand\csname \minted@cmdname\endcsname{ O{} m }{%
      \lstinputlisting{##2}%
    }%
  }
  \makeatother
\fi
\usemintedstyle{default}
\setminted{
  fontsize=\small,
  breaklines,
  breaksymbolleft=\raisebox{0.8ex}{\tiny\color{Ink}\ding{229}}\ ,
  linenos,
  numbersep=8pt,
  framesep=2mm,
  frame=single
}
\newmintedfile[yamlfile]{yaml}{}
\newminted[yamlcode]{yaml}{}
\newminted[bashcode]{bash}{}
\newminted[jsfile]{javascript}{}
\newminted[jsoncode]{json}{}

% ---------- Title ----------
\usepackage{titling}
\setlength{\droptitle}{-1em}
\title{\textbf{GitHub Actions — Practical Cheat Sheet}}
\author{}
\date{}

\begin{document}
\maketitle

\begin{tcolorbox}
\textbf{Scope.} A rapid, no-fluff reference for: (1) workflow placement \& triggers, (2) reusable workflows, (3) authoring custom actions, (4) job summaries, (5) CODEOWNERS, and (6) org workflow templates—plus common gotchas and a setup checklist.
\end{tcolorbox}

\section{Workflows \& Triggers (Placement That \emph{Actually} Works)}
\begin{itemize}
  \item Put workflow files in \texttt{.github/workflows/} (lower-case, plural). Extensions: \texttt{.yml} or \texttt{.yaml}.
  \item Many events (e.g., \texttt{pull\_request}, \texttt{issues}, \texttt{release}, \texttt{schedule}, UI \texttt{workflow\_dispatch}) are only recognized when the workflow file exists on the repository's \emph{default branch}.
  \item File name does not matter; the \texttt{name:} field sets the display name in the UI.
\end{itemize}

\begin{yamlcode}
# .github/workflows/ci.yml
name: CI
on:
  push:
    branches: [ main ]
  workflow_dispatch:
jobs:
  build:
    runs-on: ubuntu-latest
    steps:
      - uses: actions/checkout@v4
      - run: echo "Build stuff"
\end{yamlcode}

\paragraph{Event filters that save time.}
\begin{itemize}
  \item Use \texttt{paths}/\texttt{paths-ignore} to scope when a workflow runs.
  \item For \texttt{pull\_request}, prefer \texttt{pull\_request\_target} only when you \emph{must} run with base-branch permissions (be mindful of security).
\end{itemize}
\clearpage

\section{Reusable Workflows (Calling \& Inputs)}
\begin{itemize}
  \item Reusable workflows are also stored under \texttt{.github/workflows/}.
  \item Call with \texttt{jobs.<id>.uses: <owner>/<repo>/.github/workflows/<file>@<ref>}.
  \item Pass parameters via \texttt{with:} and secrets via \texttt{secrets:}.
\end{itemize}

\begin{yamlcode}
# .github/workflows/reuse-caller.yml
name: Reuse CI
on: workflow_dispatch
jobs:
  call-reusable:
    uses: your-org/your-repo/.github/workflows/ci.yml@v1
    with:
      some-input: "value"
    secrets:
      TOKEN: ${{ secrets.MY_TOKEN }}
\end{yamlcode}

\begin{tcolorbox}
\textbf{Tip.} Version reusable workflows with moving tags (\texttt{v1}) plus immutable releases (\texttt{v1.2.3}). Update the moving tag on each release to provide stability with opt-in upgrades.
\end{tcolorbox}

\section{Authoring Custom Actions (JavaScript or Docker)}
\begin{itemize}
  \item Prefer one published action per repo with definition at repo root as \texttt{action.yml} (or \texttt{action.yaml}).
  \item You \emph{can} keep multiple actions in subfolders, but that repository cannot be published to the Marketplace as a single action.
  \item Set \texttt{runs.using} to \texttt{node20} for JavaScript actions; point \texttt{main} at your compiled entry (e.g., \texttt{dist/index.js}).
\end{itemize}

\begin{yamlcode}
# action.yml (JavaScript action)
name: "Hello Action"
description: "Greets and exits"
runs:
  using: "node20"
  main: "dist/index.js"
inputs:
  who:
    description: "Who to greet"
    default: "world"
\end{yamlcode}

\noindent\textbf{Using your action:}
\begin{yamlcode}
- uses: your-org/hello-action@v1
  with:
    who: "Jordan"
\end{yamlcode}

\begin{tcolorbox}
\textbf{Release hygiene.} Create a release (e.g., \texttt{v1.2.3}), then move the major tag (e.g., \texttt{v1}) to that commit. Consumers can pin either \texttt{@v1} or \texttt{@v1.2.3}.
\end{tcolorbox}

\section{Job Summaries (Fast, Pretty Reports)}
Write Markdown to \texttt{\$GITHUB\_STEP\_SUMMARY} to show a per-job summary in the UI.
\begin{itemize}
  \item One summary file per job, up to 20 writes, about 1 MB total content.
  \item Supports tables, code fences, and Mermaid diagrams in GitHub Markdown.
\end{itemize}

\begin{bashcode}
# step: Summarize test results
echo "## Test Summary" >> "$GITHUB_STEP_SUMMARY"
echo "" >> "$GITHUB_STEP_SUMMARY"
echo "| Metric | Value |" >> "$GITHUB_STEP_SUMMARY"
echo "|---|---:|" >> "$GITHUB_STEP_SUMMARY"
echo "| Total | 120 |" >> "$GITHUB_STEP_SUMMARY"
echo "| Passed | 118 |" >> "$GITHUB_STEP_SUMMARY"
echo "| Failed | 2 |" >> "$GITHUB_STEP_SUMMARY"
\end{bashcode}

\section{CODEOWNERS (Code Owner Reviews That Stick)}
\begin{itemize}
  \item File must be named \texttt{CODEOWNERS} (exact name). Valid locations: repo root, \texttt{.github/}, or \texttt{docs/}.
  \item The file must exist on the \emph{base branch} for a PR to apply owners. Combine with Branch Protection: ``Require review from Code Owners.''
  \item Pattern order matters; the \emph{first} match wins.
  \item Each pattern must list at least one owner (users or teams); owners must have \emph{write} access to count as code owners.
  \item No negation (\texttt{!}) and no bracket character classes in patterns.
\end{itemize}

\begin{minted}[fontsize=\small,breaklines,linenos,frame=single]{text}
# CODEOWNERS (place at /.github/CODEOWNERS or at repo root)
# Comments begin with '#'
/docs/           @org/docs-team
/apps/           @octocat
/apps/secret/    @org/security-reviewers
\end{minted}

\begin{tcolorbox}
\textbf{Enforcement.} In \emph{Branch protection rules}, enable: Require a pull request before merging \(\rightarrow\) Require review from Code Owners. Owners cannot auto-approve their own PRs unless allowed by policy.
\end{tcolorbox}
\clearpage

\section{Org Workflow Templates (Show Up in ``New Workflow'')}
To offer curated starter workflows across the org:
\begin{enumerate}
  \item Create a \emph{public} org repository named \texttt{.github}.
  \item Add templates under \texttt{workflow-templates/}.
  \item Each template needs a workflow file and a matching \texttt{properties/*.properties.json}.
  \item Icon can be an Octicon name or an SVG in the folder (reference by filename without \texttt{.svg}).
  \item Templates are discovered from the default branch.
\end{enumerate}

\begin{yamlcode}
# .github/workflow-templates/node-ci.yml
name: Node CI
on: [push]
jobs:
  build:
    runs-on: ubuntu-latest
    steps:
      - uses: actions/checkout@v4
      - uses: actions/setup-node@v4
        with: { node-version: 20 }
      - run: npm ci
      - run: npm test --silent
\end{yamlcode}

\begin{jsoncode}
// .github/workflow-templates/properties/node-ci.properties.json
{
  "name": "Node CI (Org Standard)",
  "description": "Lint, test, and build using our org baseline.",
  "iconName": "repo-push",
  "categories": ["JavaScript"]
}
\end{jsoncode}
\clearpage

\section{Security, Permissions, Caching (Quick Wins)}
\begin{itemize}
  \item \textbf{GITHUB\_TOKEN permissions:} set least privilege at the workflow or job level.
\end{itemize}

\begin{yamlcode}
permissions:
  contents: read
  pull-requests: write  # only if you need to comment/update PRs
\end{yamlcode}

\begin{itemize}
  \item \textbf{Concurrency:} cancel superseded builds to save minutes.
\end{itemize}

\begin{yamlcode}
concurrency:
  group: ${{ github.workflow }}-${{ github.ref }}
  cancel-in-progress: true
\end{yamlcode}

\begin{itemize}
  \item \textbf{Caching:} pair cache keys with a fallback to avoid cache misses.
\end{itemize}

\begin{yamlcode}
- uses: actions/cache@v4
  with:
    path: ~/.npm
    key: npm-${{ runner.os }}-${{ hashFiles('**/package-lock.json') }}
    restore-keys: |
      npm-${{ runner.os }}-
\end{yamlcode}

\section{Gotchas \& Troubleshooting}
\begin{itemize}
  \item \textbf{Default branch rule:} Schedules and many UI-driven triggers use the workflow from the default branch.
  \item \textbf{Path filters:} When using \texttt{paths}, remember that a workflow still appears in the run list even if all jobs are skipped by conditions.
  \item \textbf{Reusable workflows:} Inputs are strings by default; coerce types in the called workflow if needed.
  \item \textbf{Marketplace pinning:} Pin third-party actions to a tag and, ideally, a commit SHA for supply-chain safety.
  \item \textbf{Matrix limits:} Large matrices can hit concurrency or minutes limits—chunk by \texttt{include}/\texttt{exclude}.
\end{itemize}

\section{New Repository Setup — Mini Checklist}
\begin{tcolorbox}
\begin{itemize}
  \item Create \texttt{.github/workflows/ci.yml} with \texttt{push} and \texttt{workflow\_dispatch}.
  \item Set \textbf{Branch protection}: require PRs, require status checks, and require review from Code Owners.
  \item Add \texttt{CODEOWNERS} at repo root or \texttt{.github/}.
  \item Lock down \texttt{GITHUB\_TOKEN} permissions to least privilege.
  \item Configure cache for your package manager and test runner.
  \item (Org-wide) Keep curated templates in the public \texttt{.github} repo under \texttt{workflow-templates/}.
\end{itemize}
\end{tcolorbox}

\vfill
\begin{center}
\footnotesize This cheat sheet is intended as a fast operational reference. For deeper context, consult the official GitHub Actions documentation for events, expressions, and security guidance.
\end{center}

\end{document}
