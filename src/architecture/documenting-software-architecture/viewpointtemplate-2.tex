\documentclass[11pt]{article}

\usepackage[margin=1in]{geometry}
\usepackage{enumitem}
\usepackage{setspace}
\setstretch{1.1}

\begin{document}

\title{Viewpoint Template}
\author{}
\date{}
\maketitle

%--------------------------------------------------
\section*{Viewpoint Name}
\textit{The name for the viewpoint, and any synonyms for the viewpoint.}

\vspace{1em}
\noindent
\textbf{Name:} \hrulefill

\vspace{0.5em}
\noindent
\textbf{Synonyms / Alternative Names:}

\vspace{4em}

%--------------------------------------------------
\section*{Overview}
\textit{An abstract or brief overview of the viewpoint and its key features.}

\vspace{6em}

%--------------------------------------------------
\section*{Concerns}
\textit{List the architecture-related concerns framed by this viewpoint. This
helps readers decide whether this viewpoint will be useful to them.}

\vspace{0.5em}
\begin{itemize}[nosep]
  \item
  \item
  \item
\end{itemize}

\vspace{2em}

%--------------------------------------------------
\section*{Anti-Concerns (Optional)}
\textit{Document the kinds of issues this viewpoint is \textbf{not} appropriate
for. Articulating anti-concerns can be useful for certain viewpoint notations.}

\vspace{4em}

%--------------------------------------------------
\section*{Typical Stakeholders (Optional)}
\textit{The typical audiences for views prepared using this viewpoint. Who are
the usual stakeholders for this kind of view?}

\vspace{0.5em}
\begin{itemize}[nosep]
  \item
  \item
  \item
\end{itemize}

\vspace{2em}

%--------------------------------------------------
\section*{Model Types}
\textit{Identify each type of model used by the viewpoint.}

\vspace{0.5em}
\begin{itemize}[nosep]
  \item
  \item
  \item
\end{itemize}

\vspace{2em}

%--------------------------------------------------
\section*{Model Languages}
\textit{For each model type, describe the language, notation, or modeling
techniques to be used. Model languages provide the vocabularies for constructing
the view. A model language may be:}
\begin{itemize}[nosep]
  \item an existing modeling language (e.g., UML, SysML, SADT),
  \item a mathematical or analytical technique (e.g., queuing theory),
  \item a metamodel that defines the language’s core constructs,
  \item a template that users fill in,
  \item or a combination of these.
\end{itemize}

\vspace{4em}

%--------------------------------------------------
\section*{Viewpoint Metamodels (Optional)}
\textit{Describe the conceptual entities, their attributes, and the relations
that make up the vocabulary of this kind of model (its metamodel). A metamodel
should capture at least:}
\begin{itemize}[nosep]
  \item \textbf{Entities} – major sorts of elements present in this model type.
  \item \textbf{Attributes} – properties of those entities.
  \item \textbf{Relationships} – relations defined among entities.
  \item \textbf{Constraints} – constraints on entities, attributes, or
        relationships.
\end{itemize}

\vspace{4em}

%--------------------------------------------------
\section*{Conforming Notations}
\textit{Identify any existing notations or model languages that may be used for
this model type.}

\vspace{4em}

%--------------------------------------------------
\section*{Model Correspondence Rules}
\textit{Specify any model correspondence rules defined by this viewpoint (e.g.,
how elements in one model correspond to elements in another). Document each
rule here.}

\vspace{6em}

%--------------------------------------------------
\section*{Operations on Views}
\textit{Define the methods that may be applied to views and their models.}

\subsection*{Creation Methods}
\textit{How are views prepared using this viewpoint? Include process guidance
(how to start, what to do next), work product guidance (templates, heuristics,
styles, patterns, idioms).}

\vspace{4em}

\subsection*{Interpretive Methods}
\textit{How should readers and stakeholders interpret and understand views of
this type?}

\vspace{4em}

\subsection*{Analysis Methods}
\textit{Methods for checking, reasoning about, transforming, predicting, or
evaluating architecture results from this view.}

\vspace{4em}

\subsection*{Implementation Methods}
\textit{Methods for realizing or constructing systems using information from
this view.}

\vspace{4em}

%--------------------------------------------------
\section*{Examples (Optional)}
\textit{Provide one or more example views prepared using this viewpoint.}

\vspace{8em}

%--------------------------------------------------
\section*{Notes (Optional)}
\textit{Any additional information users of the viewpoint may need.}

\vspace{6em}

%--------------------------------------------------
\section*{Sources}
\textit{List the sources for this viewpoint, if any. This may include author,
history, literature references, prior art, and more.}

\vspace{6em}

\end{document}