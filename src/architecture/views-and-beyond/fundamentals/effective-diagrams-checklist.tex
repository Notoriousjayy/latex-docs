% !TeX program = lualatex
\documentclass[11pt]{article}

\usepackage[margin=1in]{geometry}
\usepackage{microtype}
\usepackage[hidelinks]{hyperref}
\usepackage{enumitem}
\usepackage{booktabs}
\usepackage{xcolor}

\setlist[itemize]{leftmargin=*, itemsep=0.35em, topsep=0.35em}
\setlist[enumerate]{leftmargin=*, itemsep=0.35em, topsep=0.35em}

\title{\textbf{Checklist for Effective Diagrams}\\
\large Turning complex information into clear, compelling visuals}
\author{%
  % Replace with your name / team
  }
\date{\today}

\begin{document}
\maketitle

\begin{abstract}
A strong diagram is \emph{immediately understandable}: it communicates one clear message with minimal cognitive load, using disciplined layout, purposeful visual encoding, and careful final validation. This document consolidates a practical checklist for producing effective diagrams, with supporting references for deeper reading.\cite{ref1,ref2}
\end{abstract}

\tableofcontents
\newpage

\section{How to Use This Checklist}
Use this checklist as:
\begin{itemize}
  \item a \textbf{design guide} while drafting diagrams, and
  \item a \textbf{review gate} before publishing to stakeholders.
\end{itemize}

If you have limited time, start with:
\begin{enumerate}
  \item \textbf{Purpose \& Focus} (single goal + one key message),
  \item \textbf{Structure \& Layout} (flow, grouping, alignment), and
  \item \textbf{Final Polish \& Accuracy} (truthfulness, legibility, squint test).
\end{enumerate}

\section{Purpose \& Focus}
A diagram should answer a specific “so what?” question and highlight a single primary insight.\cite{ref2,ref3}

\begin{itemize}
  \item \textbf{Clear Goal:} The diagram has one well-defined purpose and a specific decision or insight it supports.
  \item \textbf{Audience-Centric:} The content and terminology match the viewers’ knowledge level; unnecessary jargon is avoided.
  \item \textbf{One Key Message:} The most important takeaway is the most visually prominent element.\cite{ref2,ref3,ref4,ref5}
\end{itemize}

\section{Structure \& Layout}
Structure determines how quickly a viewer can parse meaning. Favor consistent patterns and reduce visual effort.\cite{ref6,ref7}

\begin{itemize}
  \item \textbf{Logical Flow:} Reading order is natural (left-to-right, top-to-bottom, or clockwise).
  \item \textbf{White Space:} Adequate spacing separates elements to reduce cognitive load.
  \item \textbf{Clustering:} Related items are grouped to convey stages, functions, or domains.
  \item \textbf{Consistency:} Similar elements share size, shape, and style (e.g., all process steps look alike).
  \item \textbf{Alignment:} Elements align vertically and horizontally for a clean, professional layout.\cite{ref6,ref7,ref8,ref9}
\end{itemize}

\section{Visual Elements \& Clarity}
Clarity is achieved through precise labeling, minimal text, and connectors that communicate relationships unambiguously.\cite{ref9,ref12}

\begin{itemize}
  \item \textbf{Descriptive Title:} Clearly states what the diagram represents.
  \item \textbf{Clear Labels:} Key parts are labeled directly; legends are used sparingly and only when needed.
  \item \textbf{Concise Text:} Labels and annotations are brief (ideally 1--3 words).
  \item \textbf{Meaningful Connectors:} Lines/arrows show direction and relationship; add connector labels when ambiguity is possible.
  \item \textbf{No “Chartjunk”:} Remove non-essential decoration (heavy grids, 3D effects, excessive shadows).
  \item \textbf{Start/End Points:} Particularly for flowcharts, clearly define the start and end states.\cite{ref6,ref7,ref9,ref10,ref11,ref12,ref13}
\end{itemize}

\section{Color \& Contrast}
Color should encode meaning and improve comprehension, not add noise.\cite{ref6,ref7,ref11}

\begin{itemize}
  \item \textbf{Purposeful Color:} Use color to highlight, separate, or encode categories.
  \item \textbf{Limited Palette:} Prefer 2--3 colors plus grayscale to avoid visual overload.
  \item \textbf{Accessibility:} Favor colorblind-friendly combinations; avoid red/green dependence.
  \item \textbf{High Contrast:} Ensure text remains legible against backgrounds in both screen and print contexts.\cite{ref6,ref7,ref8,ref11}
\end{itemize}

\section{Final Polish \& Accuracy}
The final review is where many diagrams succeed or fail: check legibility, correctness, and truthful representation.\cite{ref1,ref14}

\begin{itemize}
  \item \textbf{High Resolution:} Export images sharply (avoid pixelation); choose an appropriate file format for the medium.
  \item \textbf{Accuracy Check:} Validate that values, relationships, and claims are correct and faithfully represented (avoid distortion and misleading scales).
  \item \textbf{The “Squint Test”:} When viewing the diagram at a glance (or slightly blurred), the main message should still dominate.\cite{ref2,ref11,ref14,ref15}
\end{itemize}

\section{Quick-Check by Diagram Type}
Use these fast checks as a minimal acceptance gate.\cite{ref12,ref16,ref17}

\subsection{Flowcharts / Process Diagrams}
\begin{itemize}
  \item Clear \textbf{start} and \textbf{end} states
  \item Decision points use \textbf{diamonds} and are \textbf{labeled}
  \item Directional flow is unambiguous; arrows do not create confusing crossings
\end{itemize}

\subsection{Charts (Bar/Line/Scatter/etc.)}
\begin{itemize}
  \item Axes are labeled and include \textbf{units}
  \item Baselines and scales are appropriate (including starting at zero when required for honest comparison)
  \item Legends and annotations clarify meaning without duplicating every label
\end{itemize}

\section{One-Page Review Table}
\begin{center}
\small
\begin{tabular}{@{}p{0.23\linewidth} p{0.72\linewidth}@{}}
\toprule
\textbf{Category} & \textbf{Pass/Fail Criteria (minimal)} \\
\midrule
Purpose \& Focus &
One goal; one key message is most prominent; audience-appropriate wording \\
Structure \& Layout &
Logical flow; consistent shapes; grouped stages; aligned objects; sufficient whitespace \\
Visual Clarity &
Clear title; direct labels; concise text; meaningful arrows; minimal chartjunk; start/end visible \\
Color \& Contrast &
Limited palette; purposeful encoding; accessible combinations; readable contrast \\
Polish \& Accuracy &
High-resolution export; data/logic validated; squint test passes \\
\bottomrule
\end{tabular}
\end{center}

\section{Note on Reliability}
AI-generated or rapidly assembled diagrams and checklists can contain errors or omit context. Treat this checklist as a review aid, not a substitute for domain validation.

\newpage
\begin{thebibliography}{17}

\bibitem{ref1}
Dataversity. \emph{Data Visualization Best Practices}.
\url{https://www.dataversity.net/articles/data-visualization-best-practices/}

\bibitem{ref2}
Software Ideas. \emph{The Art of Diagramming: How to Create Engaging and Effective Visuals}.
\url{https://www.softwareideas.net/a/1816/the-art-of-diagramming--how-to-create-engaging-and-effective-visuals}

\bibitem{ref3}
UC Berkeley BPMO. \emph{Data Visualization Checklist (PDF)}.
\url{https://bpm.berkeley.edu/sites/default/files/bpmo_data_viz_checklist_v4f.pdf}

\bibitem{ref4}
UX Magazine (Medium). \emph{The Ultimate Data Visualization Handbook for Designers}.
\url{https://uxmag.medium.com/the-ultimate-data-visualization-handbook-for-designers-efa7d6e0b6fe}

\bibitem{ref5}
Inforiver. \emph{Popular Charts for Storytelling (Power BI) --- eBook}.
\url{https://inforiver.com/ebooks/popular-charts-for-storytelling-power-bi/}

\bibitem{ref6}
draw.io. \emph{What Makes a Diagram a Good Diagram?}
\url{https://drawio-app.com/blog/what-makes-a-diagram-a-good-diagram/}

\bibitem{ref7}
Vexlio. \emph{Five Simple Things That Will Immediately Improve Your Diagrams}.
\url{https://vexlio.com/blog/five-simple-things-that-will-immediately-improve-your-diagrams/}

\bibitem{ref8}
LinkedIn Pulse. \emph{Golden Rules: Data Visualisation}.
\url{https://www.linkedin.com/pulse/golden-rules-data-visualisation-turning-o1lrf}

\bibitem{ref9}
Miro. \emph{Diagram Design}.
\url{https://miro.com/blog/diagram-design/}

\bibitem{ref10}
Study.com. \emph{Diagrams Lesson for Kids: Definition, Components \& Example}.
\url{https://study.com/academy/lesson/video/diagrams-lesson-for-kids-definition-components-example.html}

\bibitem{ref11}
resolution.de. \emph{Data Visualization Best Practices}.
\url{https://www.resolution.de/post/data-visualization-best-practices/}

\bibitem{ref12}
Collaboard. \emph{Process Diagram}.
\url{https://www.collaboard.app/en/blog/process-diagram}

\bibitem{ref13}
LinkedIn. \emph{Designing Effective Diagrams and Illustrations}.
\url{https://www.linkedin.com/top-content/writing/writing-user-manuals/designing-effective-diagrams-and-illustrations/}

\bibitem{ref14}
University of Michigan Library Guides. \emph{Resource on Infographics/Data Visualization (Guide)}.
\url{https://guides.umd.umich.edu/c.php?g=934040&p=6732793}

\bibitem{ref15}
Health and Learning. \emph{Checklist for Reviewing Infographics (PDF)}.
\url{https://healthandlearning.org/wp-content/uploads/2018/01/Checklist-for-Reviewing-Infographics-2.pdf}

\bibitem{ref16}
WPI (Claypool). \emph{Graph Checklist}.
\url{https://web.cs.wpi.edu/~claypool/one-pagers/graph-checklist.html}

\bibitem{ref17}
Stephanie Evergreen. \emph{Data Visualization Checklist (PDF)}.
\url{https://stephanieevergreen.com/wp-content/uploads/2016/10/DataVizChecklist_May2016.pdf}

\end{thebibliography}

\end{document}
