\documentclass[11pt]{article}

\usepackage[margin=1in]{geometry}
\usepackage{array}
\usepackage{longtable}
\usepackage{hyperref}

\title{<System Name> \\[4pt]
\large Architecture Documentation Beyond Views}
\author{<Author>}
\date{<Date>}

\begin{document}
\maketitle

\section{Purpose and Audience}
Describe the purpose of this document and how it relates to the
set of architectural views. Identify primary stakeholders and
their concerns (e.g., architects, developers, testers, managers).

\section{Documentation Roadmap}
Explain how the architecture documentation is organized and how
to navigate it.

\subsection*{View Catalog}
\setlength{\extrarowheight}{2pt}
\begin{longtable}{>{\bfseries}p{0.24\textwidth} p{0.25\textwidth} p{0.40\textwidth}}
View Name & View Type & Primary Concerns / Stakeholders \\
\hline
<Module View> & Module & <e.g., code structure, ownership> \\
<C\&C View> & Component-and-Connector & <e.g., runtime structure, performance> \\
<Deployment View> & Deployment & <e.g., hosting, capacity, ops> \\
% Add additional views as needed
\end{longtable}

\subsection*{Document Map}
List the main documents (views, rationale, interface docs, etc.) and
describe what each contains and how they relate.

\section{System Overview}
Give a short, technology-neutral summary of the system:
\begin{itemize}
  \item Business goals and context.
  \item Major capabilities and users.
  \item High-level architecture (one or two paragraphs and,
        optionally, a simple diagram referenced from a view).
\end{itemize}

\section{Architectural Drivers}
Summarize the forces that shaped the architecture.

\subsection*{Business and Mission Goals}
\begin{itemize}
  \item Goal 1: <description>
  \item Goal 2: <description>
\end{itemize}

\subsection*{Quality Attribute Priorities}
Describe the most important quality attributes and their scenarios
(e.g., performance, availability, security, modifiability).

\subsection*{Constraints and Assumptions}
List technical, organizational, and regulatory constraints, plus
major assumptions.

\section{Mapping Between Requirements and Views}
Explain how requirements are realized across views.

\subsection*{Requirements-to-View Matrix}
\begin{longtable}{>{\bfseries}p{0.22\textwidth} p{0.22\textwidth} p{0.40\textwidth}}
Requirement ID & Realizing Views & Notes \\
\hline
REQ-1 & <Module View, C\&C View> & <brief explanation> \\
REQ-2 & <Deployment View> & <brief explanation> \\
\end{longtable}

\section{Mapping Between Views and Implementation}
Describe how architectural elements map to implementation artifacts.

\subsection*{Element Mapping Table}
\begin{longtable}{>{\bfseries}p{0.20\textwidth} p{0.25\textwidth} p{0.35\textwidth}}
Architectural Element & Views Containing Element & Implementation Mapping \\
\hline
<Element ID> & <Module, C\&C> & <packages, services, repos, deployment units> \\
\end{longtable}

\section{Crosscutting Concerns}
Document crosscutting structures and policies that span multiple views
(e.g., error handling, logging, security, configuration).

\subsection*{Concern: <Security>}
\begin{itemize}
  \item Policies and assumptions.
  \item Mechanisms used (e.g., authN/authZ, encryption).
  \item References to views or components that implement them.
\end{itemize}

% Repeat subsection for additional concerns (logging, monitoring, etc.)

\section{Architecture Evolution and Roadmap}
Summarize:
\begin{itemize}
  \item Planned major changes or migrations.
  \item Deprecations and replacement strategies.
  \item Compatibility and versioning strategy.
\end{itemize}

\section{Glossary and Acronyms}
Define key terms and abbreviations used throughout the architecture docs.

\begin{longtable}{>{\bfseries}p{0.25\textwidth} p{0.60\textwidth}}
Term & Definition \\
\hline
<term> & <definition> \\
\end{longtable}

\section{Related Documents and References}
List related specifications, standards, design docs, tickets, or
external references.

\section{Revision History}
\begin{longtable}{>{\bfseries}p{0.15\textwidth} p{0.20\textwidth} p{0.20\textwidth} p{0.35\textwidth}}
Version & Date & Author & Summary of Changes \\
\hline
0.1 & <yyyy-mm-dd> & <name> & Initial draft. \\
\end{longtable}

\end{document}

