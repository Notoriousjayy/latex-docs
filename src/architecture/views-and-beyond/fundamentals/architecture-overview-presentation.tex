\documentclass[aspectratio=169]{beamer}
% Possible aspect ratios: 43, 149, 1610, 169. 16:9 is common for modern displays.

% ---------------------------------------------------------------------------
% Presentation Metadata (Fill these in)
% ---------------------------------------------------------------------------
\newcommand{\ProjectName}{\textit{<Project Name>}}
\newcommand{\SystemName}{\textit{<System Name>}}
\newcommand{\AuthorName}{\textit{<Your Name>}}
\newcommand{\ReviewDate}{\today}
\newcommand{\Version}{v0.1 (Draft)}

% ---------------------------------------------------------------------------
% Beamer Theme & Styling
% ---------------------------------------------------------------------------
\usetheme{Madrid}          % Simple, widely available theme
\usecolortheme{default}
\usefonttheme{professionalfonts}

\setbeamertemplate{navigation symbols}{}  % Remove navigation buttons
\setbeamertemplate{footline}[frame number] % Show slide numbers

\usepackage[T1]{fontenc}
\usepackage{lmodern}
\usepackage{xcolor}

% Highlight TODOs in red
\newcommand{\TODO}[1]{\textcolor{red}{[TODO: #1]}}

% Optional: a subtle brand color
\definecolor{BrandPrimary}{RGB}{25,76,153}
\setbeamercolor{title}{fg=BrandPrimary}
\setbeamercolor{frametitle}{fg=BrandPrimary}

% Automatically show mini ToC at each section start
\AtBeginSection[]{
  \begin{frame}{Outline}
    \tableofcontents[currentsection]
  \end{frame}
}

% ---------------------------------------------------------------------------
% Title Page
% ---------------------------------------------------------------------------
\title[Architecture Overview]{Architecture Overview for \SystemName}
\subtitle{\ProjectName}
\author{\AuthorName}
\institute{\textit{<Organization / Team>}}
\date{\ReviewDate \\ \Version}

% ===========================================================================

\begin{document}

% ---------------------------------------------------------------------------
% Title Slide
% ---------------------------------------------------------------------------
\begin{frame}
  \titlepage
\end{frame}

% ---------------------------------------------------------------------------
% High-Level Outline
% ---------------------------------------------------------------------------
\begin{frame}{Outline}
  \tableofcontents
\end{frame}

% ===========================================================================
\section{Problem Statement}
% ===========================================================================

% Slide 1: Business / Mission Problem
\begin{frame}{Problem Statement: Business / Mission Context}
  \textbf{Goal of this section:} Explain \emph{why} \SystemName exists.

  \begin{itemize}
    \item \textbf{Business context}
      \begin{itemize}
        \item \TODO{Summarize the domain and business environment.}
        \item \TODO{Identify key customers / users / stakeholders.}
      \end{itemize}
    \item \textbf{Problems / pain points}
      \begin{itemize}
        \item \TODO{Describe main problems the system addresses.}
        \item \TODO{Highlight current limitations of existing solutions (if any).}
      \end{itemize}
    \item \textbf{Desired outcomes}
      \begin{itemize}
        \item \TODO{List measurable business or mission outcomes.}
      \end{itemize}
  \end{itemize}
\end{frame}

% Slide 2: Key Requirements (Functional)
\begin{frame}{Problem Statement: Key Functional Requirements}
  \textbf{Goal of this slide:} Capture the essence of what the system must \emph{do}.

  \begin{itemize}
    \item \textbf{Core capabilities}
      \begin{itemize}
        \item \TODO{Capability 1 (short phrase).}
        \item \TODO{Capability 2.}
        \item \TODO{Capability 3.}
      \end{itemize}
    \item \textbf{Critical use cases / scenarios}
      \begin{itemize}
        \item \TODO{Scenario A (1 line).}
        \item \TODO{Scenario B (1 line).}
       \end{itemize}
    \item \textbf{Out of scope (for this release)}
      \begin{itemize}
        \item \TODO{Briefly note key items that are intentionally out of scope.}
      \end{itemize}
  \end{itemize}
\end{frame}

% Slide 3: Key Quality Attributes and Constraints
\begin{frame}{Problem Statement: Qualities and Constraints}
  \textbf{Goal of this slide:} Make non-functional drivers explicit.

  \begin{columns}[T]
    \column{0.48\textwidth}
      \textbf{Quality Attributes}
      \begin{itemize}
        \item \TODO{Availability (e.g., 99.9\% uptime).}
        \item \TODO{Performance (e.g., response time, throughput).}
        \item \TODO{Security (e.g., compliance, threat model).}
        \item \TODO{Scalability / elasticity.}
        \item \TODO{Modifiability / extensibility.}
      \end{itemize}

    \column{0.48\textwidth}
      \textbf{Constraints}
      \begin{itemize}
        \item \TODO{Technology constraints (e.g., mandated platforms, languages).}
        \item \TODO{Organizational constraints (e.g., team structure, vendor policies).}
        \item \TODO{Regulatory / legal constraints.}
        \item \TODO{Operational constraints (e.g., deployment environments).}
      \end{itemize}
  \end{columns}
\end{frame}

% ===========================================================================
\section{Architecture Strategy}
% ===========================================================================

% Slide 4: Architecture Drivers and Challenges
\begin{frame}{Architecture Strategy: Drivers and Challenges}
  \textbf{Goal of this slide:} Show the main forces shaping the architecture.

  \begin{itemize}
    \item \textbf{Top architectural drivers}
      \begin{itemize}
        \item \TODO{Driver 1 (e.g., extreme scalability).}
        \item \TODO{Driver 2 (e.g., strict regulatory compliance).}
      \end{itemize}
    \item \textbf{Key challenges}
      \begin{itemize}
        \item \TODO{Challenge 1 (e.g., multi-tenant isolation).}
        \item \TODO{Challenge 2.}
      \end{itemize}
    \item \textbf{Success criteria}
      \begin{itemize}
        \item \TODO{How will we know the architecture is “good enough”?}
      \end{itemize}
  \end{itemize}
\end{frame}

% Slide 5: Architecture Approaches and Styles
\begin{frame}{Architecture Strategy: Styles and Patterns}
  \textbf{Goal of this slide:} Summarize the big structural ideas.

  \begin{itemize}
    \item \textbf{Primary architectural styles}
      \begin{itemize}
        \item \TODO{E.g., layered architecture.}
        \item \TODO{E.g., microservices / SOA.}
        \item \TODO{E.g., event-driven / pub-sub.}
      \end{itemize}
    \item \textbf{Major patterns / tactics}
      \begin{itemize}
        \item \TODO{Example: CQRS, Saga, Circuit Breaker, etc.}
      \end{itemize}
    \item \textbf{Rationale}
      \begin{itemize}
        \item \TODO{Explain how the chosen styles respond to the main drivers.}
      \end{itemize}
  \end{itemize}
\end{frame}

% ===========================================================================
\section{System Context}
% ===========================================================================

% Slide 6: High-Level System Context
\begin{frame}{System Context Overview}
  \textbf{Goal of this slide:} Show the system as a black box and its environment.

  \begin{itemize}
    \item \textbf{\SystemName as a black box}
      \begin{itemize}
        \item \TODO{One-sentence description of what \SystemName does.}
      \end{itemize}
    \item \textbf{External actors / systems}
      \begin{itemize}
        \item \TODO{Actor / System 1.}
        \item \TODO{Actor / System 2.}
        \item \TODO{Actor / System 3.}
      \end{itemize}
    \item \textbf{Interaction types}
      \begin{itemize}
        \item \TODO{E.g., HTTP APIs, message queues, file exchange, UI.}
      \end{itemize}
  \end{itemize}
\end{frame}

% Slide 7: System Context Diagram
\begin{frame}{System Context Diagram}
  \textbf{Goal of this slide:} Visualize boundaries and external interfaces.

  \begin{itemize}
    \item \TODO{Insert system context diagram (image / exported figure).}
    \item \TODO{Clearly mark system boundary and externals.}
    \item \TODO{Annotate key interfaces (names, protocols).}
  \end{itemize}
\end{frame}

% ===========================================================================
\section{Architecture Views}
% ===========================================================================

% NOTE: Duplicate / adapt the following slide templates for each view you want to present.
% Example views: Decomposition (module), Runtime C&C, Deployment, Data model, etc.

% Slide 8: View Overview (all views)
\begin{frame}{Architecture Views: Overview}
  \textbf{Goal of this slide:} Introduce which views will be shown.

  \begin{itemize}
    \item \textbf{Selected views}
      \begin{itemize}
        \item \TODO{Decomposition view (module structure).}
        \item \TODO{Runtime C\&C view (components and connectors).}
        \item \TODO{Deployment / allocation view.}
        \item \TODO{Data model view (if applicable).}
      \end{itemize}
    \item \textbf{Selection rationale}
      \begin{itemize}
        \item \TODO{Explain briefly why these views are sufficient for this overview.}
      \end{itemize}
  \end{itemize}
\end{frame}

% ---------------------- Decomposition / Module View ----------------------
\subsection*{Decomposition / Module View}

\begin{frame}{Decomposition View: Primary Diagram}
  \textbf{Goal of this slide:} Show major modules / subsystems and their relations.

  \begin{itemize}
    \item \TODO{Insert module/decomposition diagram.}
    \item \TODO{Add a brief legend explaining symbols and line styles.}
    \item \TODO{Highlight few key modules (e.g., core domain, integration).}
  \end{itemize}
\end{frame}

\begin{frame}{Decomposition View: Responsibilities and Structure}
  \textbf{Goal of this slide:} Explain what the major modules \emph{do}.

  \begin{columns}[T]
    \column{0.48\textwidth}
      \textbf{Key Modules}
      \begin{itemize}
        \item \TODO{Module A (1-line responsibility summary).}
        \item \TODO{Module B.}
        \item \TODO{Module C.}
      \end{itemize}

    \column{0.48\textwidth}
      \textbf{Design Intent}
      \begin{itemize}
        \item \TODO{Explain decomposition principles (e.g., cohesion, separation of concerns).}
        \item \TODO{Mention layering or dependency rules.}
      \end{itemize}
  \end{columns}
\end{frame}

% ---------------------- Runtime C&C View ----------------------
\subsection*{Runtime Component \& Connector View}

\begin{frame}{Runtime C\&C View: Primary Diagram}
  \textbf{Goal of this slide:} Show runtime components and their communication.

  \begin{itemize}
    \item \TODO{Insert C\&C diagram (services, processes, connectors).}
    \item \TODO{Include legend for component types and connector types.}
    \item \TODO{Highlight key runtime pathways (e.g., request path, event flow).}
  \end{itemize}
\end{frame}

\begin{frame}{Runtime C\&C View: Qualities and Tactics}
  \textbf{Goal of this slide:} Link structure to qualities (performance, reliability, etc.).

  \begin{itemize}
    \item \textbf{Performance / scalability}
      \begin{itemize}
        \item \TODO{Explain scaling strategy (e.g., stateless services, sharding).}
      \end{itemize}
    \item \textbf{Reliability / resilience}
      \begin{itemize}
        \item \TODO{Describe redundancy, failover, retry patterns.}
      \end{itemize}
    \item \textbf{Security}
      \begin{itemize}
        \item \TODO{Mention key security boundaries, authN/Z components.}
      \end{itemize}
  \end{itemize}
\end{frame}

% ---------------------- Deployment / Allocation View ----------------------
\subsection*{Deployment / Allocation View}

\begin{frame}{Deployment View: Environment Topology}
  \textbf{Goal of this slide:} Show where components run physically / virtually.

  \begin{itemize}
    \item \TODO{Insert deployment diagram (nodes, zones, clusters).}
    \item \TODO{Distinguish environments (dev / test / prod) if relevant.}
  \end{itemize}
\end{frame}

\begin{frame}{Deployment View: Concerns and Constraints}
  \textbf{Goal of this slide:} Explain deployment-related constraints and reasoning.

  \begin{itemize}
    \item \textbf{Infrastructure}
      \begin{itemize}
        \item \TODO{Cloud provider(s), on-prem, hybrid, etc.}
      \end{itemize}
    \item \textbf{Network and security}
      \begin{itemize}
        \item \TODO{Network zones, firewalls, VPNs, TLS, etc.}
      \end{itemize}
    \item \textbf{Operational concerns}
      \begin{itemize}
        \item \TODO{Monitoring, logging, backup/restore strategy (at high level).}
      \end{itemize}
  \end{itemize}
\end{frame}

% ===========================================================================
\section{How the Architecture Works}
% ===========================================================================

% NOTE: Duplicate the scenario slides for 2–3 key scenarios.

% Slide: Scenario Overview
\begin{frame}{Key Scenarios Overview}
  \textbf{Goal of this slide:} Introduce which scenarios will be walked through.

  \begin{itemize}
    \item \TODO{Scenario 1: e.g., “User registration and login”.}
    \item \TODO{Scenario 2: e.g., “Background batch processing”.}
    \item \TODO{Scenario 3: e.g., “Failure and recovery from node outage”.}
  \end{itemize}
\end{frame}

% Slide: Scenario Template
\begin{frame}{Scenario \#1: \TODO{Scenario Name}}
  \textbf{Goal of this slide:} Show how the architecture behaves end-to-end.

  \begin{itemize}
    \item \textbf{Trigger / preconditions}
      \begin{itemize}
        \item \TODO{What starts the scenario? What must be true first?}
      \end{itemize}
    \item \textbf{Main flow}
      \begin{itemize}
        \item \TODO{Step-by-step description of the flow across components.}
        \item \TODO{Optionally reference a sequence diagram or state machine.}
      \end{itemize}
    \item \textbf{Qualities exercised}
      \begin{itemize}
        \item \TODO{Which qualities this scenario demonstrates (e.g., performance, availability).}
      \end{itemize}
  \end{itemize}
\end{frame}

% Slide: Change / Impact Scenario
\begin{frame}{Change Scenario: \TODO{Example Change}}
  \textbf{Goal of this slide:} Show how architecture responds to future change.

  \begin{itemize}
    \item \textbf{Change description}
      \begin{itemize}
        \item \TODO{Describe an anticipated change (e.g., add new integration).}
      \end{itemize}
    \item \textbf{Affected views / components}
      \begin{itemize}
        \item \TODO{Which modules, components, and deployments are impacted?}
      \end{itemize}
    \item \textbf{Effort and risk}
      \begin{itemize}
        \item \TODO{Qualitative assessment of cost, risk, and complexity.}
      \end{itemize}
  \end{itemize}
\end{frame}

% Slide: Failure / Recovery Scenario
\begin{frame}{Failure Scenario: \TODO{Example Failure}}
  \textbf{Goal of this slide:} Show how the system behaves under failure.

  \begin{itemize}
    \item \textbf{Failure event}
      \begin{itemize}
        \item \TODO{E.g., node outage, network partition, dependency failure.}
      \end{itemize}
    \item \textbf{Detection and response}
      \begin{itemize}
        \item \TODO{How is the failure detected? Which components respond and how?}
      \end{itemize}
    \item \textbf{User impact and recovery}
      \begin{itemize}
        \item \TODO{What is the visible impact? How and when is normal operation restored?}
      \end{itemize}
  \end{itemize}
\end{frame}

% ===========================================================================
\section{Risks, Decisions, and Open Questions}
% ===========================================================================

% Slide: Major Architectural Decisions
\begin{frame}{Major Architectural Decisions}
  \textbf{Goal of this slide:} Make key decisions and their rationale explicit.

  \begin{itemize}
    \item \TODO{Decision 1: short statement (e.g., “Adopt microservices”).}
      \begin{itemize}
        \item \TODO{Rationale: why this is chosen.}
        \item \TODO{Implications: pros / cons.}
      \end{itemize}
    \item \TODO{Decision 2.}
    \item \TODO{Decision 3.}
  \end{itemize}
\end{frame}

% Slide: Risks and Mitigations
\begin{frame}{Key Risks and Mitigations}
  \textbf{Goal of this slide:} Highlight major architecture-related risks.

  \begin{itemize}
    \item \TODO{Risk 1: description.}
      \begin{itemize}
        \item \TODO{Probability / impact (qualitative).}
        \item \TODO{Mitigation / contingency.}
      \end{itemize}
    \item \TODO{Risk 2.}
    \item \TODO{Risk 3.}
  \end{itemize}
\end{frame}

% Slide: Open Questions
\begin{frame}{Open Questions and Next Steps}
  \textbf{Goal of this slide:} Be explicit about unresolved issues.

  \begin{itemize}
    \item \textbf{Open questions}
      \begin{itemize}
        \item \TODO{Question 1 (e.g., unresolved technology choice).}
        \item \TODO{Question 2.}
      \end{itemize}
    \item \textbf{Next steps}
      \begin{itemize}
        \item \TODO{What will happen after this review?}
        \item \TODO{Decisions needed and by when.}
      \end{itemize}
  \end{itemize}
\end{frame}

% ===========================================================================
\section{Backup / Reference Slides}
% ===========================================================================

% Slide: Stakeholder View Mapping (Backup)
\begin{frame}{Stakeholders and Views (Reference)}
  \textbf{Goal of this slide:} Show mapping from stakeholders to views.

  \begin{itemize}
    \item \TODO{Stakeholder A: relevant views / concerns.}
    \item \TODO{Stakeholder B: relevant views / concerns.}
    \item \TODO{Stakeholder C: relevant views / concerns.}
  \end{itemize}
\end{frame}

% Slide: Glossary (Backup)
\begin{frame}{Glossary (Reference)}
  \textbf{Goal of this slide:} Clarify important terms and acronyms.

  \begin{itemize}
    \item \TODO{Term 1 --- short definition.}
    \item \TODO{Term 2 --- short definition.}
    \item \TODO{Acronym 1 --- expanded form.}
  \end{itemize}
\end{frame}

% Slide: Additional Diagrams (Backup)
\begin{frame}{Additional Diagrams (Reference)}
  \textbf{Goal of this slide:} Provide extra diagrams for deep dives.

  \begin{itemize}
    \item \TODO{Diagram 1: e.g., detailed subsystem view.}
    \item \TODO{Diagram 2: e.g., data model / schema snip.}
  \end{itemize}
\end{frame}

% ---------------------------------------------------------------------------
\end{document}

