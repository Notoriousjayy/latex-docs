\documentclass[11pt]{article}

\usepackage[margin=1in]{geometry}
\usepackage{array}
\usepackage{booktabs}
\usepackage{graphicx}
\usepackage{enumitem}
\usepackage{pdflscape}

\newcommand{\rot}[1]{\rotatebox{90}{\parbox{2.4cm}{\centering #1}}}

\begin{document}
\section{Stakeholders and Their Documentation Needs}

The central idea of this section is that you do not select architecture views by taste or by a fixed recipe.  
Instead, you \emph{systematically} choose, combine, and stage views based on:

\begin{itemize}[noitemsep]
  \item the \textbf{stakeholders} who must use the documentation,
  \item the \textbf{project constraints} (budget, schedule, skills, standards),
  \item and the \textbf{structures actually present} in the architecture.
\end{itemize}

The following subsections summarize the main stakeholder roles and the kinds of documentation each typically needs.

%--------------------------------------------------------------------
\subsection{Project Managers}

\textbf{Concerns.} Schedule, resource assignment, risk, contingency plans, the system's purpose and constraints, and how the system interacts with hardware and external systems.

\medskip
\noindent
\textbf{Documentation they need.}
\begin{itemize}[noitemsep]
  \item \textbf{Module views:} decomposition (modules to implement, responsibilities, complexity hints); uses and/or layered views (dependencies and likely implementation order).
  \item \textbf{Allocation views:} deployment (hardware, environments, external systems); work assignment (which organization or team builds what).
  \item \textbf{Other:} top-level context diagrams and system overview material that supports planning and communication.
\end{itemize}

\noindent
\textbf{Pattern.} Documentation that lets them build and maintain a credible project plan and coordinate organizations.

%--------------------------------------------------------------------
\subsection{Members of the Development Team}

\textbf{Concerns.} What they are building, how responsibilities are partitioned, constraints, data model, interfaces, and opportunities for reuse.

\medskip
\noindent
\textbf{Documentation they need.}
\begin{itemize}[noitemsep]
  \item \textbf{Module views:} decomposition (where their element fits), uses/layered (dependencies and protocols), and generalization (inheritance and specialization structure).
  \item \textbf{C\&C views:} structures involving their components and their neighbors at runtime.
  \item \textbf{Allocation views:} deployment (where the code runs), implementation (files, repositories, directories), and install (how the software is packaged and delivered).
  \item \textbf{Other:} interface documentation for their elements and those they interact with; variability guides for configurable or product-line systems; rationale and constraints that explain trade-offs.
\end{itemize}

\noindent
\textbf{Pattern.} Enough detail to implement correctly and predict the consequences of design choices.

%--------------------------------------------------------------------
\subsection{Testers and Integrators}

\textbf{Concerns.} Correctness and fit of the pieces, incremental integration strategy, and where to find build artifacts, interfaces, and constraints.

\medskip
\noindent
\textbf{Documentation they need.}
\begin{itemize}[noitemsep]
  \item \textbf{Module views:} decomposition (components to test and integrate), uses (dependency chains that affect test order), and the data model.
  \item \textbf{C\&C views:} all relevant runtime structures and interactions.
  \item \textbf{Allocation views:} deployment (where elements run), install (how to assemble a testable configuration), and implementation (where to find code and artifacts).
  \item \textbf{Other:} interface and behavioral specifications for modules under test and their neighbors; context diagrams focused on the test boundary.
\end{itemize}

\noindent
\textbf{Pattern.} Similar to developers, but with extra emphasis on behavior, integration paths, and observable interfaces.

%--------------------------------------------------------------------
\subsection{Designers of Other Systems}

\textbf{Concerns.} Interoperability: what services are provided and required, protocols, and data models at system boundaries.

\medskip
\noindent
\textbf{Documentation they need.}
\begin{itemize}[noitemsep]
  \item Precise \textbf{interface specifications} of the elements they will interact with.
  \item The relevant \textbf{data model} of the system they integrate with.
  \item Top-level \textbf{context diagrams} showing external interactions.
\end{itemize}

\noindent
\textbf{Pattern.} They do not need the entire architecture, only the parts that form the system-to-system contracts.

%--------------------------------------------------------------------
\subsection{Maintainers}

\textbf{Concerns.} Understanding the impact of changes, localizing modifications, and preserving original intent and rationale.

\medskip
\noindent
\textbf{Documentation they need.}
\begin{itemize}[noitemsep]
  \item \textbf{Module views:} decomposition, layered structure, and data model (to understand where responsibilities and state live).
  \item \textbf{C\&C views:} runtime structures for impact analysis.
  \item \textbf{Allocation views:} deployment, implementation, and install views to understand operational consequences of changes.
  \item \textbf{Other:} rationale and constraints explaining why the system is the way it is.
\end{itemize}

\noindent
\textbf{Pattern.} Developers' needs plus support for impact analysis and reconstruction of design intent.

%--------------------------------------------------------------------
\subsection{Product-Line Application Builders}

\textbf{Concerns.} Building products in a software product line by tailoring core assets, instantiating variability, and adding product-specific code.

\medskip
\noindent
\textbf{Documentation they need.}
\begin{itemize}[noitemsep]
  \item Everything integrators need for understanding components and deployment.
  \item \textbf{Variability guides} (in module and/or C\&C views) that show where and how to customize or configure.
\end{itemize}

\noindent
\textbf{Pattern.} Integrator-level information plus clear documentation of product-line variation mechanisms.

%--------------------------------------------------------------------
\subsection{Customers}

\textbf{Concerns.} Cost, progress, and confidence that the architecture satisfies required qualities and functionality, as well as fit with their environment.

\medskip
\noindent
\textbf{Documentation they need.}
\begin{itemize}[noitemsep]
  \item \textbf{C\&C views} that support analysis results (performance, reliability, and other qualities).
  \item \textbf{Allocation views:} deployment view (fit with hardware and environment); work-assignment view in filtered form.
  \item High-level \textbf{context diagrams} that illustrate scope, external interfaces, and key responsibilities.
\end{itemize}

\noindent
\textbf{Pattern.} High-level assurances and evidence of soundness rather than detailed design artifacts.

%--------------------------------------------------------------------
\subsection{End Users}

\textbf{Concerns.} How the system behaves from a user's perspective and whether performance and reliability will be acceptable.

\medskip
\noindent
\textbf{Documentation they may use.}
\begin{itemize}[noitemsep]
  \item Selected \textbf{C\&C views} that emphasize flow of control and data transformation from inputs to outputs.
  \item \textbf{Analysis results} related to performance, reliability, and usability.
  \item \textbf{Deployment views} that show where functionality resides on platforms they interact with.
\end{itemize}

\noindent
\textbf{Pattern.} Architecture diagrams are mainly helpful when they clarify behavior and usability implications.

%--------------------------------------------------------------------
\subsection{Analysts (Quality Attribute Specialists)}

\textbf{Concerns.} Evaluating the architecture against specific quality attributes such as performance, accuracy, modifiability, security, availability, and usability.

\medskip
\noindent
\textbf{Documentation they need (general).}
\begin{itemize}[noitemsep]
  \item \textbf{Module views} (especially decomposition and uses) for understanding structure and dependencies.
  \item \textbf{C\&C views} showing processes, communication paths, and data flow.
  \item \textbf{Allocation views} (especially deployment) to understand nodes, networks, and redundancy.
\end{itemize}

\noindent
\textbf{Attribute-specific examples.}
\begin{itemize}[noitemsep]
  \item \emph{Performance:} communicating-processes C\&C view, deployment view, and behavioral documentation with timing and concurrency.
  \item \emph{Accuracy:} C\&C and data-flow views that show where numerical or algorithmic errors can accumulate.
  \item \emph{Modifiability:} uses and decomposition views for change-impact analysis; C\&C views to expose runtime side effects.
  \item \emph{Security:} deployment and context views (external connections); C\&C and decomposition views highlighting where security controls are applied.
  \item \emph{Availability:} communicating-processes and deployment views for identifying failure modes and redundancy.
  \item \emph{Usability:} decomposition and C\&C views that show how user-visible state and error handling are managed.
\end{itemize}

\noindent
\textbf{Pattern.} Analysts treat architecture documentation as a model to feed formal or semi-formal analyses.

%--------------------------------------------------------------------
\subsection{Infrastructure Support Personnel}

\textbf{Concerns.} Setting up and maintaining development, build, and production infrastructure: environments, continuous integration and delivery, source control management, and configuration.

\medskip
\noindent
\textbf{Documentation they need.}
\begin{itemize}[noitemsep]
  \item \textbf{Module views} (decomposition and uses) describing major artifacts and dependencies.
  \item \textbf{C\&C views} indicating what runs where and how components interact at runtime.
  \item \textbf{Allocation views:} deployment, install, and implementation views, which map software assets to infrastructure.
  \item \textbf{Other:} variability guides used for environment- or configuration-specific setups.
\end{itemize}

\noindent
\textbf{Pattern.} An operational view of software assets and their mapping to infrastructure components.

%--------------------------------------------------------------------
\subsection{New Stakeholders}

\textbf{Concerns.} Onboarding: understanding the system at a high level before diving into role-specific details.

\medskip
\noindent
\textbf{Documentation they need.}
\begin{itemize}[noitemsep]
  \item Introductory, broad-scope material: top-level context diagrams, architectural constraints, overall rationale, and root-level views.
  \item The same kinds of views as their ultimate role requires, but initially at lower levels of detail.
\end{itemize}

%--------------------------------------------------------------------
\subsection{Current and Future Architects}

\textbf{Concerns.} Design decisions, rationale, trade-offs, and the complete architectural history of the system.

\medskip
\noindent
\textbf{Documentation they need.}
\begin{itemize}[noitemsep]
  \item Essentially \emph{all} views in substantial detail.
  \item Rationale and constraints for all major decisions.
  \item Traceability between requirements, architectural choices, and quality-attribute evaluations.
\end{itemize}

\noindent
\textbf{Pattern.} They are the most demanding readers: documentation should enable them to evolve and re-evaluate the architecture over time.

%--------------------------------------------------------------------
\clearpage
\begin{landscape}
\subsection{Stakeholder--View Table (Filled Version)}

The qualitative mapping above can be summarized in a stakeholder--view table.  
Entries use the following key:

\medskip
\noindent
\textbf{Key:}
d = detailed information,\quad
s = some details,\quad
o = overview information,\quad
x = anything.

\begin{center}
\scriptsize
\renewcommand{\arraystretch}{1.15}
\setlength{\tabcolsep}{3pt}
\resizebox{\linewidth}{!}{%
\begin{tabular}{l|*{5}{c}|c|*{4}{c}|*{6}{c}}
\toprule
 & \multicolumn{5}{c|}{Module Views}
 & \multicolumn{1}{c|}{C\&C Views}
 & \multicolumn{4}{c|}{Allocation Views}
 & \multicolumn{6}{c}{Other Documentation} \\
\cline{2-17}
Stakeholder
 & \rot{Decomposition}
 & \rot{Uses}
 & \rot{Generalization}
 & \rot{Layered}
 & \rot{Data Model}
 & \rot{Various}
 & \rot{Deployment}
 & \rot{Implementation}
 & \rot{Install}
 & \rot{Work Assignment}
 & \rot{Interface Documentation}
 & \rot{Context Diagrams}
 & \rot{Mapping Between Views}
 & \rot{Variability Guides}
 & \rot{Analysis Results}
 & \rot{Rationale and Constraints} \\
\midrule
Project managers
 & s & s & s & d & d & d & o & s & s & d &   & o &   &   &   &   \\
Members of development team
 & d & d & d & d & d & d & s & s & s & d & d & d & d & s &   & d \\
Testers and integrators
 & d & d & d & d & d & s & s & s & s & d & d & s & d & s &   & d \\
Designers of other systems
 &   &   &   &   & d & s &   &   &   &   & d & d &   &   &   &   \\
Maintainers
 & d & d & d & d & d & d & s & s & s & d & d & d & d & d &   & d \\
Product-line application builders
 & d & d & s & o & o & s & s & s & s & s & d & s & d & d &   & d \\
Customers
 & o & o & o & s & s & s & s &   &   & o &   & d &   &   & s &   \\
End users
 &   &   &   &   &   & s & s &   &   &   &   & s &   &   & s &   \\
Analysts
 & d & d & s & d & d & d & d &   &   &   & s & d & s &   & d &   \\
Infrastructure support personnel
 & s & s & s & d & d & s & d & d & d &   &   &   &   & s &   &   \\
New stakeholders
 & x & x & x & x & x & x & x & x & x & x & x & x & x & x & x & x \\
Current and future architects
 & d & d & d & d & d & d & d & s & s & s & d & d & d & d & d & d \\
\bottomrule
\end{tabular}%
}
\end{center}
\end{landscape}

%--------------------------------------------------------------------
\clearpage
\begin{landscape}
\subsection{Stakeholder--View Table Template (Empty Version)}

For use on new projects, an empty version of the same table can be used as a worksheet.  
Cells can be filled in after interviewing stakeholders about their information needs.

\medskip
\noindent
\textbf{Key:}
d = detailed information,\quad
s = some details,\quad
o = overview information,\quad
x = anything.

\begin{center}
\scriptsize
\renewcommand{\arraystretch}{1.15}
\setlength{\tabcolsep}{3pt}
\resizebox{\linewidth}{!}{%
\begin{tabular}{l|*{5}{c}|c|*{4}{c}|*{6}{c}}
\toprule
 & \multicolumn{5}{c|}{Module Views}
 & \multicolumn{1}{c|}{C\&C Views}
 & \multicolumn{4}{c|}{Allocation Views}
 & \multicolumn{6}{c}{Other Documentation} \\
\cline{2-17}
Stakeholder
 & \rot{Decomposition}
 & \rot{Uses}
 & \rot{Generalization}
 & \rot{Layered}
 & \rot{Data Model}
 & \rot{Various}
 & \rot{Deployment}
 & \rot{Implementation}
 & \rot{Install}
 & \rot{Work Assignment}
 & \rot{Interface Documentation}
 & \rot{Context Diagrams}
 & \rot{Mapping Between Views}
 & \rot{Variability Guides}
 & \rot{Analysis Results}
 & \rot{Rationale and Constraints} \\
\midrule
Project managers                     &  &  &  &  &  &  &  &  &  &  &  &  &  &  &  &  \\
Members of development team          &  &  &  &  &  &  &  &  &  &  &  &  &  &  &  &  \\
Testers and integrators              &  &  &  &  &  &  &  &  &  &  &  &  &  &  &  &  \\
Designers of other systems           &  &  &  &  &  &  &  &  &  &  &  &  &  &  &  &  \\
Maintainers                          &  &  &  &  &  &  &  &  &  &  &  &  &  &  &  &  \\
Product-line application builders    &  &  &  &  &  &  &  &  &  &  &  &  &  &  &  &  \\
Customers                            &  &  &  &  &  &  &  &  &  &  &  &  &  &  &  &  \\
End users                            &  &  &  &  &  &  &  &  &  &  &  &  &  &  &  &  \\
Analysts                             &  &  &  &  &  &  &  &  &  &  &  &  &  &  &  &  \\
Infrastructure support personnel     &  &  &  &  &  &  &  &  &  &  &  &  &  &  &  &  \\
New stakeholders                     &  &  &  &  &  &  &  &  &  &  &  &  &  &  &  &  \\
Current and future architects        &  &  &  &  &  &  &  &  &  &  &  &  &  &  &  &  \\
\bottomrule
\end{tabular}%
}
\end{center}
\end{landscape}

\end{document}
