\nonstopmode

\errorcontextlines=999
\hbadness=9999
\hfuzz=1pt
\documentclass[11pt,a4paper]{article}

% --- Page + typography ---
\usepackage[a4paper,margin=1in]{geometry}
\usepackage{lmodern}
\usepackage[T1]{fontenc}
\usepackage[utf8]{inputenc}
\usepackage{microtype}
\emergencystretch=2em
\usepackage{parskip}

% --- Structure + lists ---
\usepackage{enumitem}
\setlist{itemsep=2pt, topsep=4pt, leftmargin=1.2em}

\usepackage{titlesec}
\titlespacing*{\section}{0pt}{6pt plus 2pt}{4pt}
\titlespacing*{\subsection}{0pt}{5pt}{3pt}
\titlespacing*{\subsubsection}{0pt}{4pt}{2pt}

% --- Colors + links ---
\usepackage[dvipsnames]{xcolor}
\usepackage{url}
\usepackage{hyperref}
\expandafter\def\expandafter\UrlBreaks\expandafter{\UrlBreaks\do\?\do\=\do\&\do\_\do\-}\relax
\def\Urlmuskip{0mu plus 1mu}
\usepackage[most]{tcolorbox}
\usepackage[most]{tcolorbox}
\makeatletter
\@ifundefinedcolor{MidnightBlue}{\definecolor{MidnightBlue}{RGB}{25,25,112}}{}
\makeatother
\hypersetup{
  colorlinks=true,
  linkcolor=black,
  urlcolor=MidnightBlue,
  citecolor=black,
  pdftitle={Study Plan — The Design of Sites (2nd ed.) User Stories (Full Set)},
  pdfauthor={Jordan Suber}
}
\urlstyle{same}

% --- Tables + boxes + symbols ---
\usepackage{tabularx,array,ragged2e}
\newcolumntype{Y}{>{\RaggedRight\arraybackslash}X}
\newcolumntype{L}[1]{>{\raggedleft\arraybackslash\bfseries}p{#1}}
\setlength{\tabcolsep}{6pt}

\usepackage[most]{tcolorbox}
\tcbset{enhanced, breakable}

\usepackage{amsmath,amssymb}

% --- Listings (for BDD\path{/Gherkin} examples) ---
\usepackage{listings}
\lstdefinelanguage{Gherkin}{
  morekeywords={Feature,Scenario,Given,When,Then,And,Background,But,Examples,Scenario Outline},
  sensitive=false,
}
\lstset{
basicstyle=\ttfamily\small,
  keywordstyle=\bfseries,
  columns=fullflexible,
  keepspaces=true,
  showstringspaces=false,
  breaklines=true
}
% ==============================
% Embedded Story Card Template (standalone)
% ==============================

% Badges\path{/pills} for metadata
\newtcbox{\pill}{on line, arc=3pt, boxsep=0.8pt, left=4pt,right=4pt,top=1pt,bottom=1pt,
  colframe=gray!50, colback=gray!15, boxrule=0.3pt}
\newcommand{\badge}[1]{\pill{\footnotesize #1}}

% Ready\path{/Done} helpers
\newcommand{\DoR}{\textbf{Definition of Ready:} Persona clear; AC drafted; Dependencies known; Estimate set.}
\newcommand{\DoD}{\textbf{Definition of Done:} All ACs pass; KPIs improve or stay green; Security\path{/a11y} checks; Docs updated; Deployed\path{/flagged.}}

% Tasks box environment used for AC\path{/Evidence}
% === Clean TasksBox environment (breakable; no nobreak warnings) ===
\newenvironment{TasksBox}[1]{%
  \begin{tcolorbox}[
    enhanced jigsaw,
    breakable,
    pad at break=2mm,
    before skip=6pt,
    after skip=8pt,
    lines before break=0,
    colback=gray!2,
    colframe=gray!50,
    arc=2pt,
    boxrule=0.4pt,
    colbacktitle=gray!6,
    coltitle=black,
    fonttitle=\bfseries\large,
    left=8pt,right=8pt,top=8pt,bottom=8pt,
    borderline west={2pt}{0pt}{MidnightBlue},
    title={\textbf{#1}},
    title after break={\tcbtitle\ (cont.)}
  ]
  \small
}{%
  \end{tcolorbox}
}



% 9-arg Story Card
% 1: ID   2: Title   3: Epic\path{/Feature}   4: Business Value
% 5: Priority   6: SP   7: Persona   8: Dependencies   9: Assumptions\path{/Risks}
% === Clean StoryCard command ===
\newcommand{\StoryCard}[9]{%
  \begin{tcolorbox}[
    enhanced jigsaw,
    breakable,
    pad at break=2mm,
    before skip=8pt,
    after skip=10pt,
    lines before break=0,
    colback=white,
    colframe=gray!50,
    arc=2pt,
    boxrule=0.6pt,
    colbacktitle=gray!10,
    coltitle=black,
    fonttitle=\bfseries\Large,
    left=10pt,right=10pt,top=10pt,bottom=10pt,
    borderline west={2pt}{0pt}{MidnightBlue},
    title={ID-#1 \textemdash\ \textbf{#2}}
  ]
  \small
  \begin{tabularx}{\textwidth}{@{}L{3.2cm}Y@{}}
    \textbf{Epic / Feature}        & #3 \\
    \textbf{Business Value}        & #4 \\
    \textbf{Priority / Estimate}   & \badge{Priority: #5}\ \badge{SP: #6} \\
    \textbf{Persona}               & #7 \\
    \textbf{Dependencies}          & #8 \\
    \textbf{Assumptions / Risks}   & #9 \\
  \end{tabularx}

  \medskip
  \end{tcolorbox}
}

% Convenience checkboxes (optional for AC lists)
\newcommand{\checkbox}{\(\square\)}
\newcommand{\checkedbox}{\(\blacksquare\)}

% ==============================
% Document content
% ==============================

\title{\textbf{Study Plan — \emph{The Design of Sites} (2nd ed.)}\\
\vspace{0.25em}\large Full User Story Set by Chapter\path{/Pattern} Group}
\author{Jordan Suber}
\date{\today}

\begin{document}
\maketitle

\section*{How to Use This Template}

\section*{Customer-Centered Sites: End-to-End Workflow}
\section*{Discovery \& Strategy}

\texttt{\textbackslash StoryCard} with metadata and a pair of \textbf{Acceptance Criteria} scenarios (BDD) plus \textbf{Evidence Links}. Duplicate any card to expand your backlog.
\clearpage

% ==============================
% Foundations (Ch.1--5)
% ==============================


\StoryCard{DOS-CH3-001}{Define Personas, Top Tasks, \& Scenarios}{Ch.3 — Knowing Your Customers}{Decisions grounded in representative users and measurable tasks}{High}{8}{Researcher}{Interview pool; survey tool}{Risk: small sample; Assumption: tasks stable for 2 quarters}
\begin{TasksBox}{Acceptance Criteria}
% ensure height
\vspace{1pt}\strut
\begin{lstlisting}[language=Gherkin]
Scenario: Personas cover major segments
  Given interviews and survey data
  When personas are written
  Then there are 3-5 personas with goals, constraints, and context of use

Scenario: Tasks are measurable
  Given the top tasks list
  When success criteria are defined
  Then each task has a target KPI and measurement plan
\end{lstlisting}
\end{TasksBox}
\begin{TasksBox}{Evidence Links}
- Personas: \path{/research/personas-v1.pdf} \\
- Top tasks: \path{/research/top-tasks.csv} \\
- Scenarios: \path{/research/scenarios.md}
\end{TasksBox}
\clearpage


\StoryCard{DOS-PG-AECOM-001}{Ethical Personalization \& Recommendations}{Advanced E-Commerce}{Higher AOV without harming trust or relevance}{Medium}{8}{Returning Buyer}{Product catalog tags; rec engine}{Risk: filter bubble; Assumption: explainable rules}
\begin{TasksBox}{Acceptance Criteria}
% ensure height
\vspace{1pt}\strut
\begin{lstlisting}[language=Gherkin]
Scenario: Cross-sell placement
  Given a product detail page
  When related items are displayed
  Then no more than two modules appear and are clearly labeled "Recommended"

Scenario: Controls and transparency
  Given personalization is active
  When a user opens settings
  Then there are controls to mute categories and a link "Why these?"
\end{lstlisting}
\end{TasksBox}
\begin{TasksBox}{Evidence Links}
- Merch rules: \path{/commerce/merchandising.md} \\
- A\path{/B} plan: \path{/experiments/recs-plan.md}
\end{TasksBox}
\clearpage


\section*{Information Architecture \& Content Strategy}

\StoryCard{DOS-PG-CONTENT-001}{Define Content Model \& Editorial Rules}{Writing and Managing Content}{Consistent, scannable content that supports reuse and SEO}{High}{8}{Editors}{Style guide base; content inventory}{Risk: legacy content debt; Assumption: modular CMS}
\begin{TasksBox}{Acceptance Criteria}
% ensure height
\vspace{1pt}\strut
\begin{lstlisting}[language=Gherkin]
Scenario: Content model defined
  Given page types and modules
  When modeling content
  Then templates list required fields, helper text, and governance rules

Scenario: Scannability standards
  Given the editorial guide
  When a page is authored
  Then headings, summaries, and links conform to the guide and pass linting
\end{lstlisting}
\end{TasksBox}
\begin{TasksBox}{Evidence Links}
- Content model: \path{/content/model.yaml} \\
- Editorial guide: \path{/content/style-guide.md} \\
- Lint rules: \path{/content/lint-rules.yml}
\end{TasksBox}
\clearpage

\StoryCard{DOS-PG-IA-001}{Establish Global + Local Navigation}{Creating a Navigation Framework}{Customers can reach critical tasks in \textless2 clicks from the homepage}{High}{8}{All Users}{Final IA labels; search plan}{Risk: menu bloat; Assumption: 6--8 global items}
\begin{TasksBox}{Acceptance Criteria}
% ensure height
\vspace{1pt}\strut
\begin{lstlisting}[language=Gherkin]
Scenario: Global nav exposes top tasks
  Given the global navigation
  When evaluating the first read
  Then it lists 6-8 items with descriptive labels (no jargon)

Scenario: Local nav supports deep browsing
  Given a second-level content page
  When scanning the left rail
  Then contextual links to sibling\path{/child} pages are present
\end{lstlisting}
\end{TasksBox}
\begin{TasksBox}{Evidence Links}
- IA diagram: \path{/ia/site-map-v3.drawio} \\
- Copy deck: \path{/content/nav-labels.md}
\end{TasksBox}
\clearpage

\StoryCard{DOS-PG-MOBILE-001}{Deliver Mobile-First Variants}{The Mobile Web}{Parity of critical tasks with mobile-friendly inputs and layouts}{High}{8}{Mobile Visitor}{Breakpoint rules; mobile components}{Risk: hidden content; Assumption: touch-first targets}
\begin{TasksBox}{Acceptance Criteria}
% ensure height
\vspace{1pt}\strut
\begin{lstlisting}[language=Gherkin]
Scenario: Viewport and sizing
  Given a mobile device
  When a page loads
  Then viewport meta is correct and tap targets meet 44px guidelines

Scenario: Task parity
  Given the list of critical tasks
  When testing on mobile
  Then each task is possible without desktop-only steps
\end{lstlisting}
\end{TasksBox}
\begin{TasksBox}{Evidence Links}
- Mobile guide: \path{/design-system/mobile.md} \\
- Component variants: \path{/design-system/components/mobile-variants.md}
\end{TasksBox}
\clearpage

% ==============================
% Blank Card to Copy
% ==============================

\section*{Blank Story Card}
% Fill the 9 arguments in order: 
% ID, Title, Epic\path{/Feature}, Business Value, Priority, SP, Persona, Dependencies, Assumptions\path{/Risks}


\StoryCard{DOS-PG-NAV-001}{Standardize Link Labels \& Error Handling}{Making Navigation Easy}{Predictable navigation with clear labels and resilient 404s}{Medium}{5}{All Users}{Link inventory; 404 design}{Risk: jargon; Assumption: descriptive labels allowed}
\begin{TasksBox}{Acceptance Criteria}
% ensure height
\vspace{1pt}\strut
\begin{lstlisting}[language=Gherkin]
Scenario: Descriptive links
  Given menus and inline links
  When labels are audited
  Then vague terms are replaced with descriptive phrases

Scenario: Useful 404
  Given a broken link
  When a 404 page is served
  Then it presents search, top paths, and a reporting link
\end{lstlisting}
\end{TasksBox}
\begin{TasksBox}{Evidence Links}
- Link language guide: \path{/content/link-language.md} \\
- 404 spec: \path{/ux/errors/404-spec.pdf}
\end{TasksBox}
\clearpage

\section*{Interaction Design \& Prototyping}

\StoryCard{DOS-CH2-001}{Create Pattern Shortlist \& Composition Map}{Ch.2 — Using Design Patterns}{Faster, consistent design with a vetted pattern set and page\path{/flow} mapping}{High}{8}{UX Lead}{Access to component library; page inventory}{Risk: overfitting; Assumption: cross-team adoption}
\begin{TasksBox}{Acceptance Criteria}
% ensure height
\vspace{1pt}\strut
\begin{lstlisting}[language=Gherkin]
Scenario: Shortlist approved
  Given a review with design and engineering
  When the shortlist is presented
  Then 8-12 patterns are approved with rationale and examples

Scenario: Composition map covers top flows
  Given the top 5 user journeys
  When patterns are mapped to pages and states
  Then each step references a chosen pattern and fallback
\end{lstlisting}
\end{TasksBox}
\begin{TasksBox}{Evidence Links}
- Pattern shortlist: \path{/patterns/shortlist.md} \\
- Composition map: \path{/patterns/composition-map.pdf}
\end{TasksBox}
\clearpage

\StoryCard{DOS-CH4-001}{Establish Iterative Prototype \& Usability Cadence}{Ch.4 — Involving Customers Iteratively}{Regular feedback reduces risk before build}{High}{5}{Design \& PM}{Prototype tool; recruiting panel}{Risk: low participation; Assumption: 5-users-per-iteration}
\begin{TasksBox}{Acceptance Criteria}
% ensure height
\vspace{1pt}\strut
\begin{lstlisting}[language=Gherkin]
Scenario: Cadence defined
  Given the roadmap for the next 6 weeks
  When scheduling evaluation sessions
  Then there is a 2-week cadence with goals for each session and roles assigned

Scenario: Actionable findings
  Given a completed test session
  When insights are logged
  Then issues are prioritized with owners and deadlines
\end{lstlisting}
\end{TasksBox}
\begin{TasksBox}{Evidence Links}
- Test protocol: \path{/ux/tests/protocol-iter1.md} \\
- Insight log: \path{/ux/tests/findings.csv}
\end{TasksBox}
\clearpage

\StoryCard{DOS-PG-GEN-001}{Identify Site Genre \& Benchmarks}{Pattern Group — Site Genres}{Clarity on conventions and constraints per genre with 2-3 benchmarks}{Medium}{3}{UX Lead}{Competitor list}{Risk: copying without fit; Assumption: mixed-genre site}
\begin{TasksBox}{Acceptance Criteria}
% ensure height
\vspace{1pt}\strut
\begin{lstlisting}[language=Gherkin]
Scenario: Genre classified
  Given the product vision
  When evaluating common genres
  Then the site is tagged with one primary and up to two secondary genres

Scenario: Benchmarks extracted
  Given 2-3 exemplar sites per genre
  When patterns are analyzed
  Then a list of reusable patterns with pros\path{/cons} is published
\end{lstlisting}
\end{TasksBox}
\begin{TasksBox}{Evidence Links}
- Benchmark deck: \path{/competitive/genre-benchmarks.pdf}
\end{TasksBox}
\clearpage

\StoryCard{DOS-PG-HP-001}{State the Value Proposition on Homepage}{Creating a Powerful Homepage}{Visitors immediately understand what the site does and why it matters}{High}{5}{New Visitor}{Brand messaging finalized; IA primary paths}{Assumption: 1--2 CTAs; Risk: overcrowded hero}
\begin{TasksBox}{Acceptance Criteria}
% ensure height
\vspace{1pt}\strut
\begin{lstlisting}[language=Gherkin]
Scenario: Above-the-fold communicates key value
  Given a new visitor lands on the homepage
  When the page loads on a standard laptop viewport
  Then the primary value proposition, one primary CTA, and 2-3 key paths are visible without scrolling

Scenario: Mobile first-read is clear
  Given a mobile visitor on 375px width
  When the hero content renders
  Then the value proposition is readable and the primary CTA is tappable without zoom
\end{lstlisting}
\end{TasksBox}
\begin{TasksBox}{Evidence Links}
- Wireframe: \path{/design/homepage/hero-v2.fig} \\
- Prototype: \url{https://example.com/proto/homepage-v2} \\
- Analytics goal: Time-to-first-CTA, scroll depth dashboard
\end{TasksBox}
\clearpage

\StoryCard{DOS-PG-LAYOUT-001}{Publish Grid \& Layout Specification}{Designing Effective Page Layouts}{Consistent, readable pages across breakpoints}{High}{5}{Design System Team}{Token system; breakpoint guidelines}{Risk: component drift; Assumption: responsive-first}
\begin{TasksBox}{Acceptance Criteria}
% ensure height
\vspace{1pt}\strut
\begin{lstlisting}[language=Gherkin]
Scenario: Grid tokens
  Given the design system
  When layout tokens are published
  Then grid columns, gutters, and margins are defined for mobile\path{/tablet/desktop}

Scenario: Above-the-fold guidance
  Given content-heavy pages
  When layout templates are used
  Then critical content appears within the first read band
\end{lstlisting}
\end{TasksBox}
\begin{TasksBox}{Evidence Links}
- Layout spec: \path{/design-system/layout.md} \\
- Templates: \path{/design-system/templates/}
\end{TasksBox}
\clearpage

\StoryCard{DOS-PG-PERF-001}{Adopt a Performance Budget}{Speeding Up Your Site}{Faster loads that improve conversion and SEO}{High}{5}{Engineering}{RUM tooling; asset pipeline}{Risk: third-party bloat; Assumption: budgets enforced in CI}
\begin{TasksBox}{Acceptance Criteria}
% ensure height
\vspace{1pt}\strut
\begin{lstlisting}[language=Gherkin]
Scenario: Budget defined
  Given target device classes and network profiles
  When budgets are set
  Then page weight, LCP, and requests limits are defined per template

Scenario: Budget enforced
  Given the CI pipeline
  When a PR increases bundle size beyond budget
  Then the build fails with guidance to remediate
\end{lstlisting}
\end{TasksBox}
\begin{TasksBox}{Evidence Links}
- Perf budget: \path{/perf/budgets.yml} \\
- CI checks: \path{/ci/perf-checks.md} \\
- RUM dashboard: \path{/dashboards/web-vitals}
\end{TasksBox}
\clearpage

\StoryCard{DOS-PG-SEARCH-001}{Design Effective Site Search}{Making Site Search Fast and Relevant}{Search is a first-class navigation path with relevant, fast results}{Medium}{8}{Information Seeker}{Indices seeded; content model complete}{Risk: noisy synonyms; Assumption: facetable content}
\begin{TasksBox}{Acceptance Criteria}
% ensure height
\vspace{1pt}\strut
\begin{lstlisting}[language=Gherkin]
Scenario: Search interaction
  Given the site header on any page
  When the user focuses the search input and types a query
  Then recent queries and typeahead suggestions appear within 150ms

Scenario: Results quality and resilience
  Given a query with zero exact matches
  When results render
  Then show helpful fallbacks (did-you-mean, popular content) and no dead ends
\end{lstlisting}
\end{TasksBox}
\begin{TasksBox}{Evidence Links}
- Relevance tests: \path{/search/eval/queries.csv} \\
- Telemetry: \path{/dashboards/search-latency}
\end{TasksBox}
\clearpage

\StoryCard{DOS-PG-TASKS-001}{Reduce Friction in Forms \& Help}{Helping Customers Complete Tasks}{Higher task completion with clear forms, errors, and assistance}{High}{8}{All Users}{Form library; help center CMS}{Risk: long forms; Assumption: progressive disclosure}
\begin{TasksBox}{Acceptance Criteria}
% ensure height
\vspace{1pt}\strut
\begin{lstlisting}[language=Gherkin]
Scenario: Form clarity
  Given a multi-field form
  When validation fails
  Then errors are inline, specific, and focus returns to the first invalid field

Scenario: Assistive help
  Given a complex task page
  When a user hesitates
  Then contextual help or a "Need help?" module is available
\end{lstlisting}
\end{TasksBox}
\begin{TasksBox}{Evidence Links}
- Form standards: \path{/ux/forms/standards.md} \\
- Help patterns: \path{/help/patterns.md}
\end{TasksBox}
\clearpage

\StoryCard{ID-XXXX}{Short, Action-Oriented Title}{Epic\path{/Feature} or Chapter\path{/Pattern}}{Outcome for the customer / metric to move}{Priority}{SP}{Persona}{Dependencies}{Assumptions / Risks}
\begin{TasksBox}{Acceptance Criteria}
% ensure height
\vspace{1pt}\strut
\begin{lstlisting}[language=Gherkin]
Scenario: Happy path
  Given ...
  When ...
  Then ...

Scenario: Edge case
  Given ...
  When ...
  Then ...
\end{lstlisting}
\end{TasksBox}
\begin{TasksBox}{Evidence Links}
- Designs: ... \\
- Prototype\path{/PR}: ... \\
- Dashboards\path{/Reports}: ...
\end{TasksBox}
\clearpage

\section*{Accessibility \& Usability}

\section*{Content Production}

\section*{Implementation}

\section*{Testing \& QA}

\section*{Launch}

\section*{Post-Launch \& Optimization}

\StoryCard{DOS-CH1-001}{Establish Customer-Centered Design Principles}{Ch.1 — Customer-Centered Web Design}{Team aligns on principles that tie design choices to customer outcomes and metrics}{High}{5}{Product Team}{Stakeholder availability; access to analytics}{Risk: principles too generic; Assumption: leadership buy-in}
\begin{TasksBox}{Acceptance Criteria}
% ensure height
\vspace{1pt}\strut
\begin{lstlisting}[language=Gherkin]
Scenario: Principles are documented and actionable
  Given a cross-functional workshop was conducted
  When the team finalizes the principles
  Then there are 5-7 principles, each mapped to a customer task and KPI

Scenario: Principles drive decisions
  Given a design review for a new flow
  When a tradeoff is discussed
  Then the decision cites at least one agreed principle and expected KPI impact
\end{lstlisting}
\end{TasksBox}
\begin{TasksBox}{Evidence Links}
- Principles doc: \path{/docs/design-principles.md} \\
- Workshop notes: \path{/meetings/ccwd/notes.pdf} \\
- KPI mapping: \path{/analytics/kpi-map.xlsx}
\end{TasksBox}
\clearpage

\StoryCard{DOS-CH5-001}{Publish Process Playbook \& Phase Gates}{Ch.5 — Development Processes}{Predictable delivery with clear entry\path{/exit} criteria}{Medium}{5}{PMO}{SDLC references; RACI templates}{Risk: heavy process; Assumption: lightweight gates}
\begin{TasksBox}{Acceptance Criteria}
% ensure height
\vspace{1pt}\strut
\begin{lstlisting}[language=Gherkin]
Scenario: Gates defined
  Given discovery, design, build, and launch phases
  When defining gates
  Then each phase has entry\path{/exit} criteria and required artifacts

Scenario: Rollback readiness
  Given a launch plan
  When risk is assessed
  Then rollback criteria and owner are documented
\end{lstlisting}
\end{TasksBox}
\begin{TasksBox}{Evidence Links}
- Playbook: \path{/process/playbook.pdf} \\
- Gate checklist: \path{/process/gates.md} \\
- Rollback plan: \path{/release/rollback.md}
\end{TasksBox}
\clearpage

% ==============================
% Pattern Groups
% ==============================


\StoryCard{DOS-PG-BECOM-001}{Implement Cart \& Streamlined Checkout}{Basic E-Commerce}{Frictionless purchase flow that minimizes drop-off}{High}{13}{Buyer}{Payment gateway; tax\path{/shipping} rules}{Risk: address validation; Assumption: guest checkout enabled}
\begin{TasksBox}{Acceptance Criteria}
% ensure height
\vspace{1pt}\strut
\begin{lstlisting}[language=Gherkin]
Scenario: Cart preserves state
  Given an anonymous session
  When an item is added to cart
  Then cart state persists across pages and refreshes

Scenario: Checkout flow
  Given a filled cart
  When proceeding to checkout
  Then a single-page or guided multi-step checkout completes within 2 minutes
\end{lstlisting}
\end{TasksBox}
\begin{TasksBox}{Evidence Links}
- Checkout wireflow: \path{/commerce/checkout-flow.pdf} \\
- Drop-off dashboard: \path{/dashboards/checkout-funnel}
\end{TasksBox}
\clearpage

\StoryCard{DOS-PG-TRUST-001}{Expose Trust \& Credibility Signals}{Building Trust and Credibility}{Lower perceived risk to improve conversion and retention}{High}{5}{Privacy-Conscious Visitor}{Policy docs; security text}{Risk: dark-pattern suspicion; Assumption: transparent wording}
\begin{TasksBox}{Acceptance Criteria}
% ensure height
\vspace{1pt}\strut
\begin{lstlisting}[language=Gherkin]
Scenario: Key trust pages present
  Given an about, privacy, and security overview
  When a visitor scans the footer and main nav
  Then links to these pages are clearly visible and crawlable

Scenario: Consent clarity
  Given a cookie and data consent prompt
  When the prompt is shown
  Then options are clear, non-coercive, and stored for audit
\end{lstlisting}
\end{TasksBox}
\begin{TasksBox}{Evidence Links}
- Policies: \path{/legal/} \\
- Consent spec: \path{/privacy/consent-spec.md}
\end{TasksBox}

\end{document}
