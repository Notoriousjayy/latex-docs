
% !TEX TS-program = pdflatex
\documentclass[11pt,a4paper]{article}

% -------------------- Packages --------------------
\usepackage[T1]{fontenc}
\usepackage[utf8]{inputenc}
\usepackage{lmodern}
\usepackage{microtype}
\usepackage[a4paper,margin=1in]{geometry}
\usepackage{parskip}
\usepackage{enumitem}
\usepackage{titlesec}
\usepackage[dvipsnames]{xcolor} % gives MidnightBlue
\usepackage[hidelinks]{hyperref}
\usepackage{ragged2e}
\usepackage{tabularx}
\usepackage{array}
\usepackage{amsmath,amssymb}
\usepackage[most]{tcolorbox}
\usepackage{etoolbox}

\hypersetup{
  pdftitle={Study Plan — Building Micro-Frontends (Mezzalira 2e) — User Story Template Pack},
  pdfauthor={Jordan Suber},
  pdfsubject={User Story Template + Study Plan},
  pdfcreator={LaTeX}
}

% -------------------- Typography + spacing --------------------
\setlist{itemsep=2pt, topsep=4pt, leftmargin=1.2em}
\titlespacing*{\section}{0pt}{8pt plus 2pt}{4pt}
\titlespacing*{\subsection}{0pt}{6pt}{3pt}
\newcommand{\Meta}{MidnightBlue}

% -------------------- Pills / badges / helpers --------------------
\newtcbox{\pill}{on line, arc=3pt, boxsep=0.8pt, left=4pt,right=4pt,top=1pt,bottom=1pt,
  colframe=gray!50, colback=gray!15, boxrule=0.3pt}
\newcommand{\badge}[1]{\pill{\footnotesize #1}}
\newcommand{\cb}{\(\square\)}

% Definitions of Ready / Done (feel free to edit)
\newcommand{\DoR}{\textbf{Definition of Ready:} Persona clear; AC drafted; Dependencies known; Estimate set.}
\newcommand{\DoD}{\textbf{Definition of Done:} All ACs pass; tests green; security/a11y checks; docs updated; deployed/flagged.}

% -------------------- Card layout constants --------------------
\newlength{\StoryLabelW}\setlength{\StoryLabelW}{3.2cm}
\newlength{\StoryValueW}\setlength{\StoryValueW}{\dimexpr\textwidth-\StoryLabelW-1em\relax}

% -------------------- Story Card (adapts screenshot style) --------------------
% Args: 1 ID, 2 Title, 3 Epic/Feature, 4 Business Value, 5 Priority, 6 SP, 7 Persona, 8 Dependencies, 9 Assumptions/Risks
\newcommand{\StoryCard}[9]{%
  \newpage
  \begin{tcolorbox}[
    enhanced,breakable,
    colback=gray!2, colframe=gray!50, arc=2pt, boxrule=0.4pt,
    colbacktitle=gray!6, coltitle=black, fonttitle=\bfseries\large,
    left=8pt,right=8pt,top=8pt,bottom=8pt,
    title={{\textbf{#1} --- #2}},
    borderline west={2pt}{0pt}{MidnightBlue}
  ]
  \small
  \begin{tabularx}{\textwidth}{@{}>{\raggedleft\arraybackslash\bfseries}p{\StoryLabelW}>{\RaggedRight\arraybackslash}X@{}}
    Epic / Feature & #3 \\
    Business Value & #4 \\
    Priority / Estimate & \badge{Priority: #5}\ \badge{SP: #6} \\
    Persona & #7 \\
    Dependencies & #8 \\
    Assumptions / Risks & #9 \\
  \end{tabularx}

  \medskip
  \textbf{Story}\quad
  \emph{As a #7, I want to #2 so that #4.}

  \medskip
  \textbf{Non-Functional}\quad
  \badge{Performance}\ \badge{Security}\ \badge{Reliability}\ \badge{Accessibility}\ \badge{Privacy}\ \badge{i18n}

  \medskip
  \textbf{Acceptance Criteria (BDD)}
  \begin{description}[leftmargin=2.4cm, labelwidth=2.3cm, style=nextline, itemsep=2pt, topsep=2pt]
    \item[\textbf{Scenario}] Happy path
    \item[\textbf{Given}] the target repositories, environments, and study plan context are available
    \item[\textbf{When}] the \emph{Hands-on Objectives} below are executed
    \item[\textbf{Then}] the stated \emph{Outcomes/Deliverables} for this chapter are produced, reviewed, and published
  \end{description}

  \vspace{0.2\baselineskip}
  {\footnotesize\color{gray!60}\DoR\ \textbullet\ \DoD}
  \end{tcolorbox}
}

% -------------------- Tasks box --------------------
\newenvironment{TasksBox}[1][Tasks]{%
  \begin{tcolorbox}[
    enhanced,breakable,
    colback=gray!1, colframe=gray!35,
    colbacktitle=gray!6, coltitle=black,
    title={#1}, fonttitle=\bfseries,
    borderline west={1.8pt}{0pt}{MidnightBlue},
    arc=2pt, boxrule=0.4pt,
    left=10pt,right=10pt,top=6pt,bottom=6pt,
    before skip=6pt, after skip=10pt
  ]
  \small
  \begin{itemize}[label=\cb, leftmargin=*, labelsep=0.6em, itemsep=4pt, topsep=2pt, parsep=0pt]
}{%
  \end{itemize}
  \end{tcolorbox}
}

% -------------------- Document --------------------
\begin{document}

\begin{center}
  {\Large \textbf{Study Plan — Building Micro-Frontends (Mezzalira, 2nd Edition)}}\\[2pt]
  \textcolor{\Meta}{User Story Template Pack with Examples}\\[8pt]
  \normalsize \textbf{Author:} Jordan Suber \quad|\quad \textbf{Version:} \today
\end{center}

\section*{How to Write Effective User Stories}
\textbf{User story formula}\,: ``\emph{As a \textbf{persona}, I want to \textbf{goal} so that \textbf{business value}.}'' Keep stories small (1--3 days), independent, and verifiable via acceptance criteria.
\begin{itemize}
  \item \textbf{Persona:} Who benefits? e.g., Frontend Architect, Platform Engineer, Product Manager.
  \item \textbf{Goal:} What \emph{capability} is delivered? Avoid solutions in the story; keep implementation in tasks.
  \item \textbf{Business Value:} Why does it matter? Tie to speed, reliability, cost, compliance, or user experience.
  \item \textbf{Acceptance Criteria (BDD):} Describe observable behavior using Given/When/Then.
  \item \textbf{Non-Functional:} Add performance, security, accessibility, privacy, and reliability expectations.
  \item \textbf{Sizing/Priority:} Estimate in story points; label Must/Should/Could to aid sequencing.
\end{itemize}

\section*{Story Card Definition (Required Data)}
Each card contains:
\begin{itemize}
  \item \textbf{ID} (stable), \textbf{Title} (action + outcome), \textbf{Epic/Feature} (traceability), \textbf{Business Value}.
  \item \textbf{Priority} and \textbf{Estimate}, \textbf{Persona}, \textbf{Dependencies}, \textbf{Assumptions/Risks}.
  \item \textbf{Story} sentence, \textbf{Non-Functional} tags, \textbf{Acceptance Criteria}, and a \textbf{Tasks} checklist.
\end{itemize}

\section*{User Story Template (Copy \& Fill)}
\StoryCard{ID-XXX}{<Concise action/outcome>}{<Epic/Feature>}{<Value outcome>}{Must}{3}{<Primary persona>}{<Key deps>}{<Assumptions/Risks>}

\begin{TasksBox}[Hands-on Objectives]
  \item Define the outcome in one sentence (what success looks like).
  \item Draft 3--6 acceptance criteria in Given/When/Then form.
  \item List constraints and dependencies (environments, repos, approvals).
  \item Create deliverables (ADR, diagrams, code spike, dashboard, or README).
  \item Capture metrics (perf, error rate, cycle time) and attach evidence links.
\end{TasksBox}

% ======================================================================
% Study Plan Cards — Mezzalira 2e (14 chapters)
% ======================================================================

% 1 Principles
\StoryCard{MFE-1}{Establish micro-frontend principles}{Program Foundations}{Shared language, decision framework, and measurable quality attributes}{Must}{3}{Frontend Architect}{Architecture review; example repo}{Scope creep; unclear success metrics}
\begin{TasksBox}[Hands-on Objectives]
  \item Write a one-page position paper: when micro-frontends are justified here; list non-goals.
  \item Identify 3--5 bounded contexts (domains) and ownership candidates.
  \item Define 7 quality attributes (e.g., independent deploys, isolated failures) and metrics.
  \item Create repo scaffolding and a hello-world shell (placeholder nav + slot).
  \item Deliverable: ADR-000 (vision) and a metrics sheet.
\end{TasksBox}

% 2 Architectures & Challenges
\StoryCard{MFE-2}{Select composition style}{Architecture Choices}{Trade-off clarity; predictable runtime/composition behavior}{Must}{5}{Frontend Architect}{Design system draft; routing plan}{Over-coupling; runtime version drift}
\begin{TasksBox}[Hands-on Objectives]
  \item Compare client-, server-, and edge-side composition against your domains (table of trade-offs).
  \item Define inter-MF comms (request/response, pub/sub, events) and routing boundaries.
  \item Spike both CSR (Module Federation) and SSR (Next.js) rendering the same layout.
  \item Deliverable: ADR-001 (composition + comms) with decision record and diagrams.
\end{TasksBox}

% 3 Discovering Architectures
\StoryCard{MFE-3}{Choose split strategy + shell}{Domain Split \& Shell}{Stable shell and domain-aligned slices improve autonomy and UX}{Must}{5}{Tech Lead}{Design system strategy}{Inconsistent UX; shared pkg churn}
\begin{TasksBox}[Hands-on Objectives]
  \item Evaluate vertical vs horizontal splits with 5 criteria (team, deploy, perf, UX, SEO).
  \item Define shell responsibilities: routing, auth contour, design tokens, feature flags.
  \item Decide design system delivery (pkg vs runtime import) and version policy.
  \item Implement one vertical MF and one horizontal MF for contrast.
\end{TasksBox}

% 4 CSR MFs
\StoryCard{MFE-4}{Build CSR micro-frontends}{Client-Side Composition}{Fast local dev; independent deploys with MF runtime sharing}{Should}{8}{Frontend Engineer}{Module Federation config}{Shared deps bloat; cache issues}
\begin{TasksBox}[Hands-on Objectives]
  \item Create Home, Catalog, and Account as separate builds; expose/consume modules.
  \item Define shared dependencies and versioning; add error boundaries per MF.
  \item Integrate design system; measure TTFB/LCP as baseline.
  \item Deliverable: running CSR env + perf snapshot.
\end{TasksBox}

% 5 SSR MFs
\StoryCard{MFE-5}{Deliver SSR multi-zones}{Server-Side Composition}{Improved SEO/perf with independent deploy zones}{Should}{8}{Platform Engineer}{Next.js zones; caching}{Cache invalidation; data consistency}
\begin{TasksBox}[Hands-on Objectives]
  \item Stand up 2+ Next.js zones with shared nav and independent deploys.
  \item Choose data fetching (RSC/API routes) and caching (CDN, ISR/SSR, memory).
  \item Compare Lighthouse metrics vs CSR; document deltas.
\end{TasksBox}

% 6 Automation
\StoryCard{MFE-6}{Harden CI/CD with fitness functions}{Automation Pipeline}{Safe, fast change with guardrails}{Must}{5}{DevEx Engineer}{CI runners; test envs}{Flaky tests; slow builds}
\begin{TasksBox}[Hands-on Objectives]
  \item Create pipeline templates per MF (lint, unit, build, e2e smoke, deploy).
  \item Add fitness functions: bundle budget, route perf budget, contract tests.
  \item Wire logs/traces/metrics with MF tags (OpenTelemetry); publish dashboards.
\end{TasksBox}

% 7 Discovery & Deploy
\StoryCard{MFE-7}{Implement discovery registry + canary}{Discovery \& Delivery}{Safer releases and dynamic assembly of UI}{Must}{5}{Platform Engineer}{Registry service; CDN}{Rollback complexity}
\begin{TasksBox}[Hands-on Objectives]
  \item Implement a registry describing each MF (name, route, version, endpoint, deps).
  \item Wire runtime lookup to assemble UI on client/server/edge.
  \item Run a 10\% canary for the Catalog MF; define rollback and success criteria.
\end{TasksBox}

% 8 Pipeline Case Study
\StoryCard{MFE-8}{Adapt reference pipeline}{Case Study}{Operational consistency and provenance}{Could}{3}{DevEx Engineer}{Artifacts store; SBOM tool}{Supply chain gaps}
\begin{TasksBox}[Hands-on Objectives]
  \item Replicate reference pipeline stages and adapt to org needs.
  \item Add SBOM + checksums; enable preview env per PR.
\end{TasksBox}

% 9 Backend Patterns
\StoryCard{MFE-9}{Stabilize backends for frontends}{BFF / API / GraphQL}{Resilient contracts and optimized data access}{Should}{8}{Backend Engineer}{Gateway; BFF service}{Breaking schema changes}
\begin{TasksBox}[Hands-on Objectives]
  \item Choose API Gateway vs BFF per domain; publish a service dictionary.
  \item Implement one BFF aggregating 2 services for Catalog with caching/pagination.
  \item Try GraphQL federation or stitching; swap UI without code changes.
\end{TasksBox}

% 10 Anti-patterns
\StoryCard{MFE-10}{Eliminate common anti-patterns}{Quality \& Governance}{Reduced coupling and clearer ownership}{Must}{3}{Tech Lead}{Refactor capacity}{Hidden shared state}
\begin{TasksBox}[Hands-on Objectives]
  \item Identify 5 anti-patterns (state sharing, anarchy repos, premature abstraction).
  \item Introduce an anti-corruption layer when crossing bounded contexts.
  \item Refactor one global store into events + local state per MF.
\end{TasksBox}

% 11 Migration
\StoryCard{MFE-11}{Plan incremental migration}{Strangler \& Rollout}{Lower risk adoption with measured value}{Must}{5}{Program Manager}{Legacy system SMEs}{Scope drift; dual-running cost}
\begin{TasksBox}[Hands-on Objectives]
  \item Write a 3-phase plan (pilot $\rightarrow$ expand $\rightarrow$ decommission) with KPIs and exit criteria.
  \item Define shared concerns: auth, routing, feature flags, localization.
  \item Migrate one feature end-to-end with a strangler pattern.
\end{TasksBox}

% 12 Case Study
\StoryCard{MFE-12}{Execute a mini case study}{Applied Learning}{Demonstrate before/after impact}{Should}{5}{Team}{Test accounts; staging env}{Uncaptured metrics}
\begin{TasksBox}[Hands-on Objectives]
  \item Apply split, shell, comms, auth, and canary to the capstone app.
  \item Document architecture before/after and team velocity metrics.
  \item Record a 5-minute demo; link evidence.
\end{TasksBox}

% 13 Org Adoption
\StoryCard{MFE-13}{Create an org adoption pack}{Governance \& Ways of Working}{Sustainable scaling across teams}{Should}{3}{Eng Manager}{Guilds/CoPs; RFC/ADR process}{Over-governance; tooling mismatch}
\begin{TasksBox}[Hands-on Objectives]
  \item Draft trade-offs and a “when not to use MFs” section for leadership.
  \item Define governance (RFC/ADR templates, guild cadence, dependency policy).
  \item Map domains to teams with decision rights and ownership boundaries.
\end{TasksBox}

% 14 AI + MFE
\StoryCard{MFE-14}{Pilot AI-assisted delivery}{AI Enablement}{Faster scaffolding and tests with human-in-the-loop}{Could}{3}{Engineers}{Guardrails; model access}{Low-quality suggestions}
\begin{TasksBox}[Hands-on Objectives]
  \item Use AI to scaffold a new MF (repo, CI, basic tests) and generate E2E tests from Gherkin.
  \item Establish a review checklist; document failure modes and redlines.
  \item Compare AI vs manual effort; capture learning.
\end{TasksBox}

\end{document}
