
% !TEX TS-program = pdflatex
\documentclass[11pt,a4paper]{article}

% -------------------- Packages --------------------
\usepackage[T1]{fontenc}
\usepackage[utf8]{inputenc}
\usepackage{lmodern}
\usepackage{microtype}
\usepackage[a4paper,margin=1in]{geometry}
\usepackage{parskip}
\usepackage{enumitem}
\usepackage{hyperref}
\usepackage{bookmark}
\usepackage[dvipsnames]{xcolor}
\usepackage{array}
\usepackage{ragged2e}
\usepackage{tabularx}
\usepackage[most]{tcolorbox}
\usepackage{amsmath,amssymb}

% -------------------- Colors --------------------
\definecolor{Primary}{HTML}{0B3D91}   % deep blue for frames
\definecolor{Accent}{HTML}{005EB8}    % brighter blue
\definecolor{Soft}{HTML}{F8FAFC}      % card background
\definecolor{Ink}{HTML}{0F172A}       % near-black text
\definecolor{Meta}{HTML}{475569}      % muted meta text
\definecolor{OK}{HTML}{16A34A}
\definecolor{Warn}{HTML}{EA580C}
\definecolor{Bad}{HTML}{DC2626}
\colorlet{Border}{Primary!65!black}

\hypersetup{colorlinks=true, linkcolor=Accent, urlcolor=Accent, citecolor=Accent,
  pdftitle={Effective TypeScript — Study Plan (User Story Template)},
  pdfauthor={}, pdfsubject={User Story Cards}, pdfcreator={LaTeX}}

% -------------------- Typographic helpers --------------------
\newcommand{\Meta}[1]{\textcolor{Meta}{#1}}
\newcommand{\Small}[1]{{\footnotesize #1}}
\newcommand{\Tight}{\setlength{\itemsep}{2pt}\setlength{\topsep}{2pt}}
\newcommand{\key}[1]{\textbf{#1}}

% Checkbox item macro (robust in pdfLaTeX)
\newcommand{\citem}{\item[$\square$]}

% Inline pill/badge for tags
\newtcbox{\pill}{on line, arc=3pt, boxsep=0.6pt, left=4pt,right=4pt,top=1pt,bottom=1pt,
  colframe=black!15, colback=black!5, boxrule=0.3pt}
\newcommand{\badge}[1]{\pill{\footnotesize #1}}

% Key/Value table helpers
\newcolumntype{L}{>{\raggedright\arraybackslash\hspace{0pt}}p{0.22\textwidth}}
\newcolumntype{Y}{>{\RaggedRight\arraybackslash}X}

% -------------------- Story Card environment --------------------
\tcbset{enhanced, sharp corners, arc=2mm, boxsep=1mm, left=2mm, right=2mm, top=2mm, bottom=2mm}

\newtcolorbox{StoryCard}[2][]{%
  breakable,
  colback=white,
  colframe=Border,
  boxrule=0.6pt,
  title=\textbf{#2},
  fonttitle=\bfseries,
  attach boxed title to top left={yshift=-2mm, xshift=2mm},
  boxed title style={colback=Primary!6, colframe=Border},
  #1
}

% -------------------- Footer ribbons --------------------
\newcommand{\DoR}{\textbf{Definition of Ready:} Persona clear; AC drafted; Dependencies known; Estimate set.}
\newcommand{\DoD}{\textbf{Definition of Done:} All ACs pass; Tests green; Security/a11y checks; Docs updated; Deployed/flagged.}

% -------------------- Document --------------------
\begin{document}
\pagestyle{plain}
\begin{center}
  {\LARGE \textbf{Study Plan — Effective TypeScript (2nd Edition)}}\\[4pt]
  \Meta{User Story Template \& Examples mapped to the 10 chapters}\\[6pt]
  \Small{Assumes TypeScript 5.x, Node 18+, pnpm/npm, VS Code + TS extension.}
\end{center}
\vspace{0.5em}

\section*{How to use this template}
For each chapter, duplicate the \textbf{Template Card} and tailor the fields.
Keep cards \emph{user-centered}, \emph{small}, and \emph{testable}. Use the badges to mark non-functional focus areas.

\subsection*{Required data on every card}
\begin{itemize}\Tight
  \item \key{Epic / Feature} \(\rightarrow\) where this card rolls up (e.g., ``Type System Mastery'').
  \item \key{Business Value} \(\rightarrow\) outcomes this study work enables (e.g., safer APIs, faster PR reviews).
  \item \key{Priority / Estimate} \(\rightarrow\) Must/Should/Could + SP (3--5 ideal for a study chapter).
  \item \key{Persona} \(\rightarrow\) who benefits or acts (e.g., ``TS app dev on a new repo'').
  \item \key{Dependencies} \(\rightarrow\) tools, repos, or pre-reading.
  \item \key{Assumptions / Risks} \(\rightarrow\) things that could block you.
  \item \key{Story} \(\rightarrow\) \emph{As a <persona>, I want <action> so that <value>.}
  \item \key{Non-Functional} \(\rightarrow\) tags like \badge{Performance} \badge{Security} \badge{Reliability} \badge{Accessibility} \badge{Privacy} \badge{i18n}.
  \item \key{Acceptance Criteria (BDD)} \(\rightarrow\) Scenario with \textbf{Given/When/Then} that proves outcomes.
  \item \key{Tasks} \(\rightarrow\) 4--8 concrete steps using checkboxes \(\square\).
\end{itemize}

\subsection*{Writing effective stories (quick guide)}
\begin{itemize}\Tight
  \item Prefer \textbf{observable outcomes} over internal activity.
  \item One behavior per scenario; keep steps short and readable.
  \item Cover \textbf{happy path}, \textbf{negative path}, and key \textbf{edge cases}.
  \item Include boundaries, error handling, and permissions where relevant.
  \item Tie outcomes to \textbf{DX}, \textbf{safety}, or \textbf{runtime correctness}.
\end{itemize}

% -------------------- Template Card --------------------
\begin{StoryCard}{TEMPLATE — <Concise Card Title>}
\begin{tabularx}{\textwidth}{L Y}
  \key{Epic / Feature} & <Parent initiative or feature>\\
  \key{Business Value} & <Concrete value users/teams get from this learning>\\
  \key{Priority / Estimate} & Priority: <Must/Should/Could> \quad \badge{SP: <3--5>}\\
  \key{Persona} & <Who does this work?>\\
  \key{Dependencies} & <Repos, tools, readings>\\
  \key{Assumptions / Risks} & <What might go wrong or be unclear?>\\
\end{tabularx}

\medskip
\key{Story}\; \emph{As a <persona>, I want <outcome> so that <value>.}

\medskip
\key{Non-Functional}\; \badge{Performance}\; \badge{Security}\; \badge{Reliability}\; \badge{Accessibility}\; \badge{Privacy}\; \badge{i18n}

\medskip
\key{Acceptance Criteria (BDD)}
\begin{description}[style=unboxed,leftmargin=0pt,labelsep=8pt]
  \item[\key{Scenario}] Happy path
  \item[\key{Given}] <Preconditions, repos and context available>
  \item[\key{When}] <The hands-on objectives are executed>
  \item[\key{Then}] <Observable outcome of the chapter appears in code, CI, or docs>
\end{description}

\Small{\Meta{\DoR\;\; \textbullet\;\; \DoD}}

\medskip
\key{Tasks}
\begin{itemize}\Tight
  \citem <Task 1 (concrete, 15--60 minutes)>
  \citem <Task 2>
  \citem <Task 3>
  \citem <Task 4>
\end{itemize}
\end{StoryCard}

\clearpage

% -----------------------------------------------------------------------------
%                        CHAPTER CARDS (Effective TypeScript)
% -----------------------------------------------------------------------------

\begin{StoryCard}{TS-1 — Getting to Know TypeScript}
\begin{tabularx}{\textwidth}{L Y}
  \key{Epic / Feature} & Onboarding and Fundamentals\\
  \key{Business Value} & Shared understanding of TS vs JS, strictness posture, and tooling; reduces integration risk and confusion.\\
  \key{Priority / Estimate} & Priority: Must \quad \badge{SP: 3}\\
  \key{Persona} & Developer on a new repo\\
  \key{Dependencies} & Node 18+, pnpm/npm, VS Code, TypeScript 5.x\\
  \key{Assumptions / Risks} & Local vs CI toolchain drift; risk of initial failures after enabling strict options\\
\end{tabularx}

\medskip
\key{Story}\; \emph{As a developer, I want to configure and explore TypeScript so that I correctly understand what the compiler checks and how to keep builds green.}

\medskip
\key{Non-Functional}\; \badge{Reliability}\; \badge{Security}\; \badge{DX}

\medskip
\key{Acceptance Criteria (BDD)}
\begin{description}[style=unboxed,leftmargin=0pt,labelsep=8pt]
  \item[\key{Scenario}] Happy path
  \item[\key{Given}] a fresh repo with TypeScript installed
  \item[\key{When}] strict compiler options are enabled and a sample file is compiled
  \item[\key{Then}] a short \texttt{tsconfig-rationale.md} explains each option; build is green in CI
\end{description}

\Small{\Meta{\DoR\;\; \textbullet\;\; \DoD}}

\medskip
\key{Tasks}
\begin{itemize}\Tight
  \citem Initialize repo; run \texttt{tsc --init}; enable \texttt{strict}, \texttt{noUncheckedIndexedAccess}, \texttt{exactOptionalPropertyTypes}.
  \citem Convert a small JS utility to TS; record surprises and fixes.
  \citem Capture a ``TS myths vs reality'' note (5 bullets).
  \citem Configure VS Code TS language service tips: ``Go to Type Definition'', quick fixes, refactors.
\end{itemize}
\end{StoryCard}

\clearpage

\begin{StoryCard}{TS-2 — Type System (Structural Types and Assignability)}
\begin{tabularx}{\textwidth}{L Y}
  \key{Epic / Feature} & Type System Mastery\\
  \key{Business Value} & Fewer runtime errors and clearer APIs via correct assignability and literal/widening control.\\
  \key{Priority / Estimate} & Priority: Must \quad \badge{SP: 5}\\
  \key{Persona} & Application developer\\
  \key{Dependencies} & ESLint + \texttt{@typescript-eslint} configured\\
  \key{Assumptions / Risks} & Misunderstanding variance or excess property checks can leak bugs\\
\end{tabularx}

\medskip
\key{Story}\; \emph{As an app developer, I want to internalize structural typing and assignability so that functions and objects compose safely.}

\medskip
\key{Non-Functional}\; \badge{Reliability}\; \badge{Maintainability}

\medskip
\key{Acceptance Criteria (BDD)}
\begin{description}[style=unboxed,leftmargin=0pt,labelsep=8pt]
  \item[\key{Scenario}] Happy path
  \item[\key{Given}] a module with object and function types
  \item[\key{When}] assignability puzzles and ESLint strict boolean checks are addressed
  \item[\key{Then}] a \texttt{Result<T,E>} helper is in use and misuse causes compile errors
\end{description}

\Small{\Meta{\DoR\;\; \textbullet\;\; \DoD}}

\medskip
\key{Tasks}
\begin{itemize}\Tight
  \citem Create 10 small ``type gym'' puzzles covering unions, intersections, and excess property checks.
  \citem Implement \texttt{Result<T,E>} and a function returning success/failure.
  \citem Enable \texttt{strict-boolean-expressions}; fix violations.
  \citem Document parameter vs return variance guidance in \texttt{type-design-checklist.md}.
\end{itemize}
\end{StoryCard}

\clearpage

\begin{StoryCard}{TS-3 — Inference \& Control Flow Analysis}
\begin{tabularx}{\textwidth}{L Y}
  \key{Epic / Feature} & Ergonomic Typing\\
  \key{Business Value} & Less annotation noise; safer narrowing paths and clearer code.\\
  \key{Priority / Estimate} & Priority: Must \quad \badge{SP: 3}\\
  \key{Persona} & App developer refactoring an existing module\\
  \key{Dependencies} & TypeScript 5.x, ESLint rules enabled\\
  \key{Assumptions / Risks} & Over-annotation harms readability; brittle guards can be unsound\\
\end{tabularx}

\medskip
\key{Story}\; \emph{As a developer, I want to rely on inference and control-flow narrowing so that code stays concise and correct.}

\medskip
\key{Non-Functional}\; \badge{DX}\; \badge{Reliability}

\medskip
\key{Acceptance Criteria (BDD)}
\begin{description}[style=unboxed,leftmargin=0pt,labelsep=8pt]
  \item[\key{Scenario}] Happy path
  \item[\key{Given}] a module with many explicit types
  \item[\key{When}] redundant annotations are removed and guards/``satisfies'' are introduced
  \item[\key{Then}] the module compiles cleanly and has fewer lines of type noise with equal or better safety
\end{description}

\Small{\Meta{\DoR\;\; \textbullet\;\; \DoD}}

\medskip
\key{Tasks}
\begin{itemize}\Tight
  \citem Replace unnecessary annotations with inference; keep only clarifying ones.
  \citem Implement user-defined guards: \texttt{isNonEmptyString}, \texttt{isISODate}, \texttt{isRecord<K,V>}.
  \citem Use \texttt{satisfies} to prevent widening on literals.
  \citem Refactor a discriminated-union \texttt{switch}; enable \texttt{noFallthroughCasesInSwitch}.
\end{itemize}
\end{StoryCard}

\clearpage

\begin{StoryCard}{TS-4 — Type Design (APIs as Types)}
\begin{tabularx}{\textwidth}{L Y}
  \key{Epic / Feature} & API Design\\
  \key{Business Value} & Stable, evolvable APIs; fewer breaking changes and safer refactors.\\
  \key{Priority / Estimate} & Priority: Must \quad \badge{SP: 5}\\
  \key{Persona} & Library/API author\\
  \key{Dependencies} & Project with domain types (Users/Projects/Tickets)\\
  \key{Assumptions / Risks} & Overly broad types leak implementation; missing read-only/optional semantics\\
\end{tabularx}

\medskip
\key{Story}\; \emph{As an API author, I want to design composable, minimal types so that consumers enjoy a stable, discoverable surface.}

\medskip
\key{Non-Functional}\; \badge{Reliability}\; \badge{Maintainability}\; \badge{DX}

\medskip
\key{Acceptance Criteria (BDD)}
\begin{description}[style=unboxed,leftmargin=0pt,labelsep=8pt]
  \item[\key{Scenario}] Happy path
  \item[\key{Given}] DTOs and service interfaces for a small domain
  \item[\key{When}] branded IDs, \texttt{NonEmptyArray<T>}, and separate input/output types are introduced
  \item[\key{Then}] an API review finds no accidental widenings; examples compile in consumer code
\end{description}

\Small{\Meta{\DoR\;\; \textbullet\;\; \DoD}}

\medskip
\key{Tasks}
\begin{itemize}\Tight
  \citem Model Users/Projects/Tickets with DTOs and service interfaces.
  \citem Add branded/opaque IDs and a \texttt{NonEmptyArray<T>} helper.
  \citem Introduce \texttt{readonly} and \texttt{exactOptionalPropertyTypes} where appropriate.
  \citem Draft a ``type API review'' checklist and run it.
\end{itemize}
\end{StoryCard}

\clearpage

\begin{StoryCard}{TS-5 — Unsoundness \& the \texttt{any} Type}
\begin{tabularx}{\textwidth}{L Y}
  \key{Epic / Feature} & Risk Management\\
  \key{Business Value} & Contain escape hatches; improve safety without blocking delivery.\\
  \key{Priority / Estimate} & Priority: Must \quad \badge{SP: 3}\\
  \key{Persona} & Maintainer migrating legacy code\\
  \key{Dependencies} & ESLint rules; optional runtime validator (e.g., Zod)\\
  \key{Assumptions / Risks} & Overuse of \texttt{any} and unsafe casts\\
\end{tabularx}

\medskip
\key{Story}\; \emph{As a maintainer, I want to fence and document unsoundness so that risks are visible and limited.}

\medskip
\key{Non-Functional}\; \badge{Security}\; \badge{Reliability}

\medskip
\key{Acceptance Criteria (BDD)}
\begin{description}[style=unboxed,leftmargin=0pt,labelsep=8pt]
  \item[\key{Scenario}] Happy path
  \item[\key{Given}] a codebase with \texttt{any}-heavy areas
  \item[\key{When}] a quarantine adapter and runtime validation are introduced
  \item[\key{Then}] all remaining \texttt{any} usages are documented and isolated; \texttt{noImplicitAny} enforced
\end{description}

\Small{\Meta{\DoR\;\; \textbullet\;\; \DoD}}

\medskip
\key{Tasks}
\begin{itemize}\Tight
  \citem Scan for \texttt{any} hotspots; move interop behind a small typed adapter.
  \citem Replace \texttt{any} with \texttt{unknown}+refinement or generics where feasible.
  \citem Add a minimal runtime validator for external data.
  \citem Enable \texttt{noImplicitAny} and fix top offenders.
\end{itemize}
\end{StoryCard}

\clearpage

\begin{StoryCard}{TS-6 — Generics \& Type-Level Programming}
\begin{tabularx}{\textwidth}{L Y}
  \key{Epic / Feature} & Reusable Abstractions\\
  \key{Business Value} & Safer, reusable helpers without IDE slowdowns.\\
  \key{Priority / Estimate} & Priority: Should \quad \badge{SP: 5}\\
  \key{Persona} & Library and app developers\\
  \key{Dependencies} & TS 5.x features; benchmarking editor responsiveness\\
  \key{Assumptions / Risks} & Overly clever types harm readability or performance\\
\end{tabularx}

\medskip
\key{Story}\; \emph{As a developer, I want to write constrained generics and moderate type-level code so that helpers are powerful yet maintainable.}

\medskip
\key{Non-Functional}\; \badge{DX}\; \badge{Maintainability}

\medskip
\key{Acceptance Criteria (BDD)}
\begin{description}[style=unboxed,leftmargin=0pt,labelsep=8pt]
  \item[\key{Scenario}] Happy path
  \item[\key{Given}] a utilities module
  \item[\key{When}] helpers like \texttt{DeepReadonly<T>} and a typed event emitter are implemented
  \item[\key{Then}] examples compile; editor responsiveness remains good; code is documented with examples
\end{description}

\Small{\Meta{\DoR\;\; \textbullet\;\; \DoD}}

\medskip
\key{Tasks}
\begin{itemize}\Tight
  \citem Implement \texttt{DeepReadonly<T>}, \texttt{PickByValue<T,V>}, \texttt{TupleToObject<T>} with tests.
  \citem Build \texttt{typedEventEmitter<TEvents>} with per-event handler types.
  \citem Record compile/editor times before/after; refactor if too slow.
  \citem Add docs with usage snippets.
\end{itemize}
\end{StoryCard}

\clearpage

\begin{StoryCard}{TS-7 — Recipes (Applied Patterns)}
\begin{tabularx}{\textwidth}{L Y}
  \key{Epic / Feature} & Practical Patterns\\
  \key{Business Value} & Fewer runtime surprises in configs, async flows, and state management.\\
  \key{Priority / Estimate} & Priority: Should \quad \badge{SP: 5}\\
  \key{Persona} & App developer\\
  \key{Dependencies} & Fetch wrapper, runtime validator, state mgmt (Redux/RTK or TanStack Query)\\
  \key{Assumptions / Risks} & Loss of type information across boundaries without care\\
\end{tabularx}

\medskip
\key{Story}\; \emph{As an app dev, I want applied TS recipes so that common app paths remain type-safe without casts.}

\medskip
\key{Non-Functional}\; \badge{Reliability}\; \badge{Security}

\medskip
\key{Acceptance Criteria (BDD)}
\begin{description}[style=unboxed,leftmargin=0pt,labelsep=8pt]
  \item[\key{Scenario}] Happy path
  \item[\key{Given}] config, async fetch, and reducer examples
  \item[\key{When}] typed config loader, typed fetch wrapper, and action unions are implemented
  \item[\key{Then}] no \texttt{as} casts exist in those paths and misuse fails to compile
\end{description}

\Small{\Meta{\DoR\;\; \textbullet\;\; \DoD}}

\medskip
\key{Tasks}
\begin{itemize}\Tight
  \citem Implement a validated \texttt{Config} loader returning precise inferred types.
  \citem Create a \texttt{createReducer} helper using discriminated unions.
  \citem Add a typed fetch wrapper returning \texttt{Result<T,E>}; handle errors idiomatically.
  \citem Add unit tests for edge cases (missing keys, bad JSON, network failures).
\end{itemize}
\end{StoryCard}

\clearpage

\begin{StoryCard}{TS-8 — Declarations \& \texttt{@types}}
\begin{tabularx}{\textwidth}{L Y}
  \key{Epic / Feature} & Public Typings\\
  \key{Business Value} & Consumers get great IntelliSense and fail fast on misuse.\\
  \key{Priority / Estimate} & Priority: Should \quad \badge{SP: 3}\\
  \key{Persona} & Library author\\
  \key{Dependencies} & \texttt{tsd} or \texttt{dtslint}; sample package\\
  \key{Assumptions / Risks} & Incorrect augmentations or breakage across versions\\
\end{tabularx}

\medskip
\key{Story}\; \emph{As a library author, I want solid declaration files so that users have accurate types and documentation.}

\medskip
\key{Non-Functional}\; \badge{DX}\; \badge{Maintainability}

\medskip
\key{Acceptance Criteria (BDD)}
\begin{description}[style=unboxed,leftmargin=0pt,labelsep=8pt]
  \item[\key{Scenario}] Happy path
  \item[\key{Given}] a package with runtime JS and \texttt{index.d.ts}
  \item[\key{When}] typings are tested and a third-party augmentation is added
  \item[\key{Then}] consumers see accurate types; bad usage fails in \texttt{tsd} tests
\end{description}

\Small{\Meta{\DoR\;\; \textbullet\;\; \DoD}}

\medskip
\key{Tasks}
\begin{itemize}\Tight
  \citem Create a small library with runtime JS and matching declarations.
  \citem Add \texttt{typesVersions} if needed; write \texttt{tsd} tests to lock behavior.
  \citem Augment a third-party lib for a missing method/type; cover with tests.
  \citem Document your public types in README/API docs.
\end{itemize}
\end{StoryCard}

\clearpage

\begin{StoryCard}{TS-9 — Writing \& Running Code (Toolchain)}
\begin{tabularx}{\textwidth}{L Y}
  \key{Epic / Feature} & Build, Test, Ship\\
  \key{Business Value} & Boring, reliable DX and fast CI; fewer integration surprises.\\
  \key{Priority / Estimate} & Priority: Should \quad \badge{SP: 5}\\
  \key{Persona} & Maintainer\\
  \key{Dependencies} & Vitest/Jest, Vite/ESBuild/Rspack; CI pipeline\\
  \key{Assumptions / Risks} & Misaligned module/target settings, slow CI\\
\end{tabularx}

\medskip
\key{Story}\; \emph{As a maintainer, I want a stable toolchain so that builds, tests, and types are consistent locally and in CI.}

\medskip
\key{Non-Functional}\; \badge{Performance}\; \badge{Reliability}

\medskip
\key{Acceptance Criteria (BDD)}
\begin{description}[style=unboxed,leftmargin=0pt,labelsep=8pt]
  \item[\key{Scenario}] Happy path
  \item[\key{Given}] build, typecheck, lint, and test scripts wired to CI
  \item[\key{When}] project references and caching are enabled
  \item[\key{Then}] CI is consistently green and runs in under 2 minutes for the sample monorepo
\end{description}

\Small{\Meta{\DoR\;\; \textbullet\;\; \DoD}}

\medskip
\key{Tasks}
\begin{itemize}\Tight
  \citem Add \texttt{build}, \texttt{typecheck}, \texttt{test}, \texttt{lint} scripts with fast feedback.
  \citem Configure path aliases and project references; split into two small packages.
  \citem Enable incremental builds and caching in CI; measure run times.
  \citem Ensure source maps and declaration files are emitted as needed.
\end{itemize}
\end{StoryCard}

\clearpage

\begin{StoryCard}{TS-10 — Modernization \& Migration}
\begin{tabularx}{\textwidth}{L Y}
  \key{Epic / Feature} & Evolution\\
  \key{Business Value} & Controlled migration of legacy code; repeatable upgrade playbook.\\
  \key{Priority / Estimate} & Priority: Should \quad \badge{SP: 3}\\
  \key{Persona} & Maintainer on a legacy app\\
  \key{Dependencies} & JSDoc \texttt{// @ts-check}, codemod tooling\\
  \key{Assumptions / Risks} & Risk of churn or breaking changes during upgrades\\
\end{tabularx}

\medskip
\key{Story}\; \emph{As a maintainer, I want stepwise migrations so that legacy JS becomes strict TS with minimal disruption.}

\medskip
\key{Non-Functional}\; \badge{Reliability}\; \badge{Maintainability}

\medskip
\key{Acceptance Criteria (BDD)}
\begin{description}[style=unboxed,leftmargin=0pt,labelsep=8pt]
  \item[\key{Scenario}] Happy path
  \item[\key{Given}] a legacy JS folder
  \item[\key{When}] code is migrated via JSDoc checks and codemods, then moved to TS files
  \item[\key{Then}] \texttt{any}/\texttt{@ts-ignore} counts drop measurably; an upgrade checklist is published
\end{description}

\Small{\Meta{\DoR\;\; \textbullet\;\; \DoD}}

\medskip
\key{Tasks}
\begin{itemize}\Tight
  \citem Start with JSDoc \texttt{// @ts-check}; fix surfaced issues.
  \citem Convert to TS files; turn on strict flags progressively.
  \citem Write a codemod for a deprecated pattern and run repo-wide.
  \citem Publish an ``upgrade playbook'' for minor/major TS versions.
\end{itemize}
\end{StoryCard}

\clearpage

\end{document}
