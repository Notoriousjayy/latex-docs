\documentclass[11pt]{report}
\usepackage[margin=1in]{geometry}
\usepackage[T1]{fontenc}
\usepackage[utf8]{inputenc}
\usepackage{lmodern}
\usepackage{hyperref}
\usepackage{longtable}
\usepackage{booktabs}
\usepackage{enumitem}
\usepackage{titlesec}
\setlist[description]{style=nextline,leftmargin=0pt,itemsep=0.35\baselineskip}
\titleformat{\chapter}[display]{\normalfont\bfseries\Large}{\chaptername\ \thechapter}{10pt}{\LARGE}
\hypersetup{colorlinks=true,linkcolor=blue,urlcolor=blue,citecolor=blue}
\title{Rule Catalog: CERT Oracle Secure Coding Standard for Java}
\author{Generated Catalog}
\date{2026-02-04}
\begin{document}
\maketitle
\thispagestyle{empty}
\begin{center}
\large Complete list of rules (ID + title), grouped by topic mnemonic\\[6pt]
\normalsize Source: Fred Long, Dhruv Mohindra, Robert C. Seacord, Dean F. Sutherland, The CERT\textregistered{} Oracle\textregistered{} Secure Coding Standard for Java\texttrademark{} (Addison-Wesley, 2011).\\
\end{center}
\vspace{12pt}
\noindent\textbf{How to use this document.} This catalog lists every rule in the standard by \emph{rule identifier} (e.g., \texttt{IDS07-J}) and \emph{rule title}. Rules are grouped by the standard's mnemonic topic codes. The page references shown in parentheses are the rule's location as listed in the book's table of contents.
\tableofcontents
\clearpage
\chapter*{Summary}
\addcontentsline{toc}{chapter}{Summary}
\noindent\textbf{Total rules:} 156\\
\noindent\textbf{Topic distribution:}
\begin{longtable}{@{}llr@{}}
\toprule
Mnemonic & Topic & Count\\
\midrule
\endfirsthead
\toprule
Mnemonic & Topic & Count\\
\midrule
\endhead
\bottomrule
\endfoot
IDS & Input Validation and Data Sanitization & 14\\
DCL & Declarations and Initialization & 3\\
EXP & Expressions & 7\\
NUM & Numeric Types and Operations & 14\\
OBJ & Object Orientation & 12\\
MET & Methods & 13\\
ERR & Exceptional Behavior & 10\\
VNA & Visibility and Atomicity & 6\\
LCK & Locking & 12\\
THI & Thread APIs & 6\\
TPS & Thread Pools & 5\\
TSM & Thread-Safety Miscellaneous & 4\\
FIO & Input Output & 15\\
SER & Serialization & 12\\
SEC & Platform Security & 9\\
ENV & Runtime Environment & 6\\
MSC & Miscellaneous & 8\\
\end{longtable}
\clearpage
\chapter{IDS — Input Validation and Data Sanitization}
\begin{description}
\item[\textbf{IDS00-J}] Sanitize untrusted data passed across a trust boundary (ToC p.~24)
\item[\textbf{IDS01-J}] Normalize strings before validating them (ToC p.~34)
\item[\textbf{IDS02-J}] Canonicalize path names before validating them (ToC p.~36)
\item[\textbf{IDS03-J}] Do not log unsanitized user input (ToC p.~41)
\item[\textbf{IDS04-J}] Limit the size of files passed to ZipInputStream (ToC p.~43)
\item[\textbf{IDS05-J}] Use a subset of ASCII for file and path names (ToC p.~46)
\item[\textbf{IDS06-J}] Exclude user input from format strings (ToC p.~48)
\item[\textbf{IDS07-J}] Do not pass untrusted, unsanitized data to the Runtime.exec() method (ToC p.~50)
\item[\textbf{IDS08-J}] Sanitize untrusted data passed to a regex (ToC p.~54)
\item[\textbf{IDS09-J}] Do not use locale-dependent methods on locale-dependent data without specifying the appropriate locale (ToC p.~59)
\item[\textbf{IDS10-J}] Do not split characters between two data structures (ToC p.~60)
\item[\textbf{IDS11-J}] Eliminate noncharacter code points before validation (ToC p.~66)
\item[\textbf{IDS12-J}] Perform lossless conversion of String data between differing character encodings (ToC p.~68)
\item[\textbf{IDS13-J}] Use compatible encodings on both sides of file or network I/O 71 Chapter 3 Declarations and Initialization (DCL) 75 Rules 75 Risk Assessment Summary (ToC p.~75)
\end{description}
\clearpage
\chapter{DCL — Declarations and Initialization}
\begin{description}
\item[\textbf{DCL00-J}] Prevent class initialization cycles (ToC p.~75)
\item[\textbf{DCL01-J}] Do not reuse public identifiers from the Java Standard Library (ToC p.~79)
\item[\textbf{DCL02-J}] Declare all enhanced for statement loop variables final 81 Chapter 4 Expressions (EXP) 85 Rules 85 Risk Assessment Summary (ToC p.~85)
\end{description}
\clearpage
\chapter{EXP — Expressions}
\begin{description}
\item[\textbf{EXP00-J}] Do not ignore values returned by methods (ToC p.~86)
\item[\textbf{EXP01-J}] Never dereference null pointers (ToC p.~88)
\item[\textbf{EXP02-J}] Use the two-argument Arrays.equals() method to compare the contents of arrays (ToC p.~90)
\item[\textbf{EXP03-J}] Do not use the equality operators when comparing values of boxed primitives (ToC p.~91)
\item[\textbf{EXP04-J}] Ensure that autoboxed values have the intended type (ToC p.~97)
\item[\textbf{EXP05-J}] Do not write more than once to the same variable within an expression (ToC p.~100)
\item[\textbf{EXP06-J}] Do not use side-effecting expressions in assertions 103 ix Chapter 5 Numeric Types and Operations (NUM) 105 Rules 105 Risk Assessment Summary (ToC p.~106)
\end{description}
\clearpage
\chapter{NUM — Numeric Types and Operations}
\begin{description}
\item[\textbf{NUM00-J}] Detect or prevent integer overflow (ToC p.~106)
\item[\textbf{NUM01-J}] Do not perform bitwise and arithmetic operations on the same data (ToC p.~114)
\item[\textbf{NUM02-J}] Ensure that division and modulo operations do not result in divide-by-zero errors (ToC p.~119)
\item[\textbf{NUM03-J}] Use integer types that can fully represent the possible range of unsigned data (ToC p.~121)
\item[\textbf{NUM04-J}] Do not use floating-point numbers if precise computation is required (ToC p.~122)
\item[\textbf{NUM05-J}] Do not use denormalized numbers (ToC p.~125)
\item[\textbf{NUM06-J}] Use the strictfp modifier for floating-point calculation consistency across platforms (ToC p.~128)
\item[\textbf{NUM07-J}] Do not attempt comparisons with NaN (ToC p.~132)
\item[\textbf{NUM08-J}] Check floating-point inputs for exceptional values (ToC p.~134)
\item[\textbf{NUM09-J}] Do not use floating-point variables as loop counters (ToC p.~136)
\item[\textbf{NUM10-J}] Do not construct BigDecimal objects from floating-point literals (ToC p.~138)
\item[\textbf{NUM11-J}] Do not compare or inspect the string representation of floating-point values (ToC p.~139)
\item[\textbf{NUM12-J}] Ensure conversions of numeric types to narrower types do not result in lost or misinterpreted data (ToC p.~141)
\item[\textbf{NUM13-J}] Avoid loss of precision when converting primitive integers to floating-point 146 Chapter 6 Object Orientation (OBJ) 151 Rules 151 Risk Assessment Summary (ToC p.~152)
\end{description}
\clearpage
\chapter{OBJ — Object Orientation}
\begin{description}
\item[\textbf{OBJ00-J}] Limit extensibility of classes and methods with invariants to trusted subclasses only (ToC p.~152)
\item[\textbf{OBJ01-J}] Declare data members as private and provide accessible wrapper methods (ToC p.~159)
\item[\textbf{OBJ02-J}] Preserve dependencies in subclasses when changing superclasses (ToC p.~162)
\item[\textbf{OBJ03-J}] Do not mix generic with nongeneric raw types in new code (ToC p.~169)
\item[\textbf{OBJ04-J}] Provide mutable classes with copy functionality to safely allow passing instances to untrusted code (ToC p.~175)
\item[\textbf{OBJ05-J}] Defensively copy private mutable class members before returning their references (ToC p.~180)
\item[\textbf{OBJ06-J}] Defensively copy mutable inputs and mutable internal components (ToC p.~185)
\item[\textbf{OBJ07-J}] Sensitive classes must not let themselves be copied (ToC p.~189)
\item[\textbf{OBJ08-J}] Do not expose private members of an outer class from within a nested class (ToC p.~192)
\item[\textbf{OBJ09-J}] Compare classes and not class names (ToC p.~194)
\item[\textbf{OBJ10-J}] Do not use public static nonfinal variables (ToC p.~197)
\item[\textbf{OBJ11-J}] Be wary of letting constructors throw exceptions 199 Chapter 7 Methods (MET) 209 Rules 209 Risk Assessment Summary (ToC p.~210)
\end{description}
\clearpage
\chapter{MET — Methods}
\begin{description}
\item[\textbf{MET00-J}] Validate method arguments (ToC p.~210)
\item[\textbf{MET01-J}] Never use assertions to validate method arguments (ToC p.~213)
\item[\textbf{MET02-J}] Do not use deprecated or obsolete classes or methods (ToC p.~215)
\item[\textbf{MET03-J}] Methods that perform a security check must be declared private or final (ToC p.~217)
\item[\textbf{MET04-J}] Do not increase the accessibility of overridden or hidden methods (ToC p.~218)
\item[\textbf{MET05-J}] Ensure that constructors do not call overridable methods (ToC p.~220)
\item[\textbf{MET06-J}] Do not invoke overridable methods in clone() (ToC p.~223)
\item[\textbf{MET07-J}] Never declare a class method that hides a method declared in a superclass or superinterface (ToC p.~226)
\item[\textbf{MET08-J}] Ensure objects that are equated are equatable (ToC p.~229)
\item[\textbf{MET09-J}] Classes that define an equals() method must also define a hashCode() method (ToC p.~238)
\item[\textbf{MET10-J}] Follow the general contract when implementing thecompareTo() method (ToC p.~241)
\item[\textbf{MET11-J}] Ensure that keys used in comparison operations are immutable (ToC p.~243)
\item[\textbf{MET12-J}] Do not use finalizers 248 Chapter 8 Exceptional Behavior (ERR) 255 Rules 255 Risk Assessment Summary (ToC p.~255)
\end{description}
\clearpage
\chapter{ERR — Exceptional Behavior}
\begin{description}
\item[\textbf{ERR00-J}] Do not suppress or ignore checked exceptions (ToC p.~256)
\item[\textbf{ERR01-J}] Do not allow exceptions to expose sensitive information (ToC p.~263)
\item[\textbf{ERR02-J}] Prevent exceptions while logging data (ToC p.~268)
\item[\textbf{ERR03-J}] Restore prior object state on method failure (ToC p.~270)
\item[\textbf{ERR04-J}] Do not exit abruptly from a finally block (ToC p.~275)
\item[\textbf{ERR05-J}] Do not let checked exceptions escape from a finally block (ToC p.~277)
\item[\textbf{ERR06-J}] Do not throw undeclared checked exceptions (ToC p.~280)
\item[\textbf{ERR07-J}] Do not throw RuntimeException ,Exception , orThrowable (ToC p.~285)
\item[\textbf{ERR08-J}] Do not catch NullPointerException or any of its ancestors (ToC p.~288)
\item[\textbf{ERR09-J}] Do not allow untrusted code to terminate the JVM 296 Chapter 9 Visibility and Atomicity (VNA) 301 Rules 301 Risk Assessment Summary (ToC p.~301)
\end{description}
\clearpage
\chapter{VNA — Visibility and Atomicity}
\begin{description}
\item[\textbf{VNA00-J}] Ensure visibility when accessing shared primitive variables (ToC p.~302)
\item[\textbf{VNA01-J}] Ensure visibility of shared references to immutable objects (ToC p.~306)
\item[\textbf{VNA02-J}] Ensure that compound operations on shared variables are atomic (ToC p.~309)
\item[\textbf{VNA03-J}] Do not assume that a group of calls to independently atomic methods is atomic (ToC p.~317)
\item[\textbf{VNA04-J}] Ensure that calls to chained methods are atomic (ToC p.~323)
\item[\textbf{VNA05-J}] Ensure atomicity when reading and writing 64-bit values 328 Chapter 10 Locking (LCK) 331 Rules 331 Risk Assessment Summary (ToC p.~332)
\end{description}
\clearpage
\chapter{LCK — Locking}
\begin{description}
\item[\textbf{LCK00-J}] Use private final lock objects to synchronize classes that may interact with untrusted code (ToC p.~332)
\item[\textbf{LCK01-J}] Do not synchronize on objects that may be reused (ToC p.~339)
\item[\textbf{LCK02-J}] Do not synchronize on the class object returned by getClass() (ToC p.~343)
\item[\textbf{LCK03-J}] Do not synchronize on the intrinsic locks of high-level concurrency objects (ToC p.~347)
\item[\textbf{LCK04-J}] Do not synchronize on a collection view if the backing collection is accessible (ToC p.~348)
\item[\textbf{LCK05-J}] Synchronize access to static fields that can be modified by untrusted code (ToC p.~351)
\item[\textbf{LCK06-J}] Do not use an instance lock to protect shared static data (ToC p.~352)
\item[\textbf{LCK07-J}] Avoid deadlock by requesting and releasing locks in the same order (ToC p.~355)
\item[\textbf{LCK08-J}] Ensure actively held locks are released on exceptional conditions (ToC p.~365)
\item[\textbf{LCK09-J}] Do not perform operations that can block while holding a lock (ToC p.~370)
\item[\textbf{LCK10-J}] Do not use incorrect forms of the double-checked locking idiom (ToC p.~375)
\item[\textbf{LCK11-J}] Avoid client-side locking when using classes that do not commit to their locking strategy 381 Chapter 11 Thread APIs (THI) 387 Rules 387 Risk Assessment Summary (ToC p.~387)
\end{description}
\clearpage
\chapter{THI — Thread APIs}
\begin{description}
\item[\textbf{THI00-J}] Do not invoke Thread.run() (ToC p.~388)
\item[\textbf{THI01-J}] Do not invoke ThreadGroup methods (ToC p.~390)
\item[\textbf{THI02-J}] Notify all waiting threads rather than a single thread (ToC p.~394)
\item[\textbf{THI03-J}] Always invoke wait() and await() methods inside a loop (ToC p.~401)
\item[\textbf{THI04-J}] Ensure that threads performing blocking operations can be terminated (ToC p.~404)
\item[\textbf{THI05-J}] Do not use Thread.stop() to terminate threads 412 Chapter 12 Thread Pools (TPS) 417 Rules 417 Risk Assessment Summary (ToC p.~417)
\end{description}
\clearpage
\chapter{TPS — Thread Pools}
\begin{description}
\item[\textbf{TPS00-J}] Use thread pools to enable graceful degradation of service during traffic bursts (ToC p.~418)
\item[\textbf{TPS01-J}] Do not execute interdependent tasks in a bounded thread pool (ToC p.~421)
\item[\textbf{TPS02-J}] Ensure that tasks submitted to a thread pool are interruptible (ToC p.~428)
\item[\textbf{TPS03-J}] Ensure that tasks executing in a thread pool do not fail silently (ToC p.~431)
\item[\textbf{TPS04-J}] Ensure ThreadLocal variables are reinitialized when using thread pools 436 xiii Chapter 13 Thread-Safety Miscellaneous (TSM) 441 Rules 441 Risk Assessment Summary (ToC p.~441)
\end{description}
\clearpage
\chapter{TSM — Thread-Safety Miscellaneous}
\begin{description}
\item[\textbf{TSM00-J}] Do not override thread-safe methods with methods that are not thread-safe (ToC p.~442)
\item[\textbf{TSM01-J}] Do not let the this reference escape during object construction (ToC p.~445)
\item[\textbf{TSM02-J}] Do not use background threads during class initialization (ToC p.~454)
\item[\textbf{TSM03-J}] Do not publish partially initialized objects 459 Chapter 14 Input Output (FIO) 467 Rules 467 Risk Assessment Summary (ToC p.~468)
\end{description}
\clearpage
\chapter{FIO — Input Output}
\begin{description}
\item[\textbf{FIO00-J}] Do not operate on files in shared directories (ToC p.~468)
\item[\textbf{FIO01-J}] Create files with appropriate access permissions (ToC p.~478)
\item[\textbf{FIO02-J}] Detect and handle file-related errors (ToC p.~481)
\item[\textbf{FIO03-J}] Remove temporary files before termination (ToC p.~483)
\item[\textbf{FIO04-J}] Close resources when they are no longer needed (ToC p.~487)
\item[\textbf{FIO05-J}] Do not expose buffers created using the wrap() orduplicate() methods to untrusted code (ToC p.~493)
\item[\textbf{FIO06-J}] Do not create multiple buffered wrappers on a single InputStream (ToC p.~496)
\item[\textbf{FIO07-J}] Do not let external processes block on input and output streams (ToC p.~500)
\item[\textbf{FIO08-J}] Use an int to capture the return value of methods that read a character or byte (ToC p.~504)
\item[\textbf{FIO09-J}] Do not rely on the write() method to output integers outside the range 0 to 255 (ToC p.~507)
\item[\textbf{FIO10-J}] Ensure the array is filled when using read() to fill an array (ToC p.~509)
\item[\textbf{FIO11-J}] Do not attempt to read raw binary data as character data (ToC p.~511)
\item[\textbf{FIO12-J}] Provide methods to read and write little-endian data (ToC p.~513)
\item[\textbf{FIO13-J}] Do not log sensitive information outside a trust boundary (ToC p.~516)
\item[\textbf{FIO14-J}] Perform proper cleanup at program termination 519 Chapter 15 Serialization (SER) 527 Rules 527 Risk Assessment Summary (ToC p.~528)
\end{description}
\clearpage
\chapter{SER — Serialization}
\begin{description}
\item[\textbf{SER00-J}] Maintain serialization compatibility during class evolution (ToC p.~528)
\item[\textbf{SER01-J}] Do not deviate from the proper signatures of serialization methods (ToC p.~531)
\item[\textbf{SER02-J}] Sign then seal sensitive objects before sending them across a trust boundary (ToC p.~534)
\item[\textbf{SER03-J}] Do not serialize unencrypted, sensitive data (ToC p.~541)
\item[\textbf{SER04-J}] Do not allow serialization and deserialization to bypass the security manager (ToC p.~546)
\item[\textbf{SER05-J}] Do not serialize instances of inner classes (ToC p.~549)
\item[\textbf{SER06-J}] Make defensive copies of private mutable components during deserialization (ToC p.~551)
\item[\textbf{SER07-J}] Do not use the default serialized form for implementation-defined invariants (ToC p.~553)
\item[\textbf{SER08-J}] Minimize privileges before deserializing from a privileged context (ToC p.~558)
\item[\textbf{SER09-J}] Do not invoke overridable methods from thereadObject() method (ToC p.~562)
\item[\textbf{SER10-J}] Avoid memory and resource leaks during serialization (ToC p.~563)
\item[\textbf{SER11-J}] Prevent overwriting of externalizable objects 566 Chapter 16 Platform Security (SEC) 569 Rules 569 Risk Assessment Summary (ToC p.~570)
\end{description}
\clearpage
\chapter{SEC — Platform Security}
\begin{description}
\item[\textbf{SEC00-J}] Do not allow privileged blocks to leak sensitive information across a trust boundary (ToC p.~570)
\item[\textbf{SEC01-J}] Do not allow tainted variables in privileged blocks (ToC p.~574)
\item[\textbf{SEC02-J}] Do not base security checks on untrusted sources (ToC p.~577)
\item[\textbf{SEC03-J}] Do not load trusted classes after allowing untrusted code to load arbitrary classes (ToC p.~579)
\item[\textbf{SEC04-J}] Protect sensitive operations with security manager checks (ToC p.~582)
\item[\textbf{SEC05-J}] Do not use reflection to increase accessibility of classes, methods, or fields (ToC p.~585)
\item[\textbf{SEC06-J}] Do not rely on the default automatic signature verification provided by URLClassLoader and java.util.jar (ToC p.~592)
\item[\textbf{SEC07-J}] Call the superclass’s getPermissions() method when writing a custom class loader (ToC p.~597)
\item[\textbf{SEC08-J}] Define wrappers around native methods 599xiv (ToC p.~603)
\end{description}
\clearpage
\chapter{ENV — Runtime Environment}
\begin{description}
\item[\textbf{ENV00-J}] Do not sign code that performs only unprivileged operations (ToC p.~604)
\item[\textbf{ENV01-J}] Place all security-sensitive code in a single jar and sign and seal it (ToC p.~606)
\item[\textbf{ENV02-J}] Do not trust the values of environment variables (ToC p.~610)
\item[\textbf{ENV03-J}] Do not grant dangerous combinations of permissions (ToC p.~613)
\item[\textbf{ENV04-J}] Do not disable bytecode verification (ToC p.~617)
\item[\textbf{ENV05-J}] Do not deploy an application that can be remotely monitored 618 Chapter 18 Miscellaneous (MSC) 625 Rules 625 Risk Assessment Summary (ToC p.~625)
\end{description}
\clearpage
\chapter{MSC — Miscellaneous}
\begin{description}
\item[\textbf{MSC00-J}] Use SSLSocket rather than Socket for secure data exchange (ToC p.~626)
\item[\textbf{MSC01-J}] Do not use an empty infinite loop (ToC p.~630)
\item[\textbf{MSC02-J}] Generate strong random numbers (ToC p.~632)
\item[\textbf{MSC03-J}] Never hard code sensitive information (ToC p.~635)
\item[\textbf{MSC04-J}] Do not leak memory (ToC p.~638)
\item[\textbf{MSC05-J}] Do not exhaust heap space (ToC p.~647)
\item[\textbf{MSC06-J}] Do not modify the underlying collection when an iteration is in progress (ToC p.~653)
\item[\textbf{MSC07-J}] Prevent multiple instantiations of singleton objects (ToC p.~657)
\end{description}
\clearpage
\appendix
\chapter{Alphabetical Rule Index}
\begin{longtable}{@{}llr@{}}
\toprule
Rule ID & Title & ToC page\\
\midrule
\endfirsthead
\toprule
Rule ID & Title & ToC page\\
\midrule
\endhead
\bottomrule
\endfoot
DCL00-J & Prevent class initialization cycles & 75\\
DCL01-J & Do not reuse public identifiers from the Java Standard Library & 79\\
DCL02-J & Declare all enhanced for statement loop variables final 81 Chapter 4 Expressions (EXP) 85 Rules 85 Risk Assessment Summary & 85\\
ENV00-J & Do not sign code that performs only unprivileged operations & 604\\
ENV01-J & Place all security-sensitive code in a single jar and sign and seal it & 606\\
ENV02-J & Do not trust the values of environment variables & 610\\
ENV03-J & Do not grant dangerous combinations of permissions & 613\\
ENV04-J & Do not disable bytecode verification & 617\\
ENV05-J & Do not deploy an application that can be remotely monitored 618 Chapter 18 Miscellaneous (MSC) 625 Rules 625 Risk Assessment Summary & 625\\
ERR00-J & Do not suppress or ignore checked exceptions & 256\\
ERR01-J & Do not allow exceptions to expose sensitive information & 263\\
ERR02-J & Prevent exceptions while logging data & 268\\
ERR03-J & Restore prior object state on method failure & 270\\
ERR04-J & Do not exit abruptly from a finally block & 275\\
ERR05-J & Do not let checked exceptions escape from a finally block & 277\\
ERR06-J & Do not throw undeclared checked exceptions & 280\\
ERR07-J & Do not throw RuntimeException ,Exception , orThrowable & 285\\
ERR08-J & Do not catch NullPointerException or any of its ancestors & 288\\
ERR09-J & Do not allow untrusted code to terminate the JVM 296 Chapter 9 Visibility and Atomicity (VNA) 301 Rules 301 Risk Assessment Summary & 301\\
EXP00-J & Do not ignore values returned by methods & 86\\
EXP01-J & Never dereference null pointers & 88\\
EXP02-J & Use the two-argument Arrays.equals() method to compare the contents of arrays & 90\\
EXP03-J & Do not use the equality operators when comparing values of boxed primitives & 91\\
EXP04-J & Ensure that autoboxed values have the intended type & 97\\
EXP05-J & Do not write more than once to the same variable within an expression & 100\\
EXP06-J & Do not use side-effecting expressions in assertions 103 ix Chapter 5 Numeric Types and Operations (NUM) 105 Rules 105 Risk Assessment Summary & 106\\
FIO00-J & Do not operate on files in shared directories & 468\\
FIO01-J & Create files with appropriate access permissions & 478\\
FIO02-J & Detect and handle file-related errors & 481\\
FIO03-J & Remove temporary files before termination & 483\\
FIO04-J & Close resources when they are no longer needed & 487\\
FIO05-J & Do not expose buffers created using the wrap() orduplicate() methods to untrusted code & 493\\
FIO06-J & Do not create multiple buffered wrappers on a single InputStream & 496\\
FIO07-J & Do not let external processes block on input and output streams & 500\\
FIO08-J & Use an int to capture the return value of methods that read a character or byte & 504\\
FIO09-J & Do not rely on the write() method to output integers outside the range 0 to 255 & 507\\
FIO10-J & Ensure the array is filled when using read() to fill an array & 509\\
FIO11-J & Do not attempt to read raw binary data as character data & 511\\
FIO12-J & Provide methods to read and write little-endian data & 513\\
FIO13-J & Do not log sensitive information outside a trust boundary & 516\\
FIO14-J & Perform proper cleanup at program termination 519 Chapter 15 Serialization (SER) 527 Rules 527 Risk Assessment Summary & 528\\
IDS00-J & Sanitize untrusted data passed across a trust boundary & 24\\
IDS01-J & Normalize strings before validating them & 34\\
IDS02-J & Canonicalize path names before validating them & 36\\
IDS03-J & Do not log unsanitized user input & 41\\
IDS04-J & Limit the size of files passed to ZipInputStream & 43\\
IDS05-J & Use a subset of ASCII for file and path names & 46\\
IDS06-J & Exclude user input from format strings & 48\\
IDS07-J & Do not pass untrusted, unsanitized data to the Runtime.exec() method & 50\\
IDS08-J & Sanitize untrusted data passed to a regex & 54\\
IDS09-J & Do not use locale-dependent methods on locale-dependent data without specifying the appropriate locale & 59\\
IDS10-J & Do not split characters between two data structures & 60\\
IDS11-J & Eliminate noncharacter code points before validation & 66\\
IDS12-J & Perform lossless conversion of String data between differing character encodings & 68\\
IDS13-J & Use compatible encodings on both sides of file or network I/O 71 Chapter 3 Declarations and Initialization (DCL) 75 Rules 75 Risk Assessment Summary & 75\\
LCK00-J & Use private final lock objects to synchronize classes that may interact with untrusted code & 332\\
LCK01-J & Do not synchronize on objects that may be reused & 339\\
LCK02-J & Do not synchronize on the class object returned by getClass() & 343\\
LCK03-J & Do not synchronize on the intrinsic locks of high-level concurrency objects & 347\\
LCK04-J & Do not synchronize on a collection view if the backing collection is accessible & 348\\
LCK05-J & Synchronize access to static fields that can be modified by untrusted code & 351\\
LCK06-J & Do not use an instance lock to protect shared static data & 352\\
LCK07-J & Avoid deadlock by requesting and releasing locks in the same order & 355\\
LCK08-J & Ensure actively held locks are released on exceptional conditions & 365\\
LCK09-J & Do not perform operations that can block while holding a lock & 370\\
LCK10-J & Do not use incorrect forms of the double-checked locking idiom & 375\\
LCK11-J & Avoid client-side locking when using classes that do not commit to their locking strategy 381 Chapter 11 Thread APIs (THI) 387 Rules 387 Risk Assessment Summary & 387\\
MET00-J & Validate method arguments & 210\\
MET01-J & Never use assertions to validate method arguments & 213\\
MET02-J & Do not use deprecated or obsolete classes or methods & 215\\
MET03-J & Methods that perform a security check must be declared private or final & 217\\
MET04-J & Do not increase the accessibility of overridden or hidden methods & 218\\
MET05-J & Ensure that constructors do not call overridable methods & 220\\
MET06-J & Do not invoke overridable methods in clone() & 223\\
MET07-J & Never declare a class method that hides a method declared in a superclass or superinterface & 226\\
MET08-J & Ensure objects that are equated are equatable & 229\\
MET09-J & Classes that define an equals() method must also define a hashCode() method & 238\\
MET10-J & Follow the general contract when implementing thecompareTo() method & 241\\
MET11-J & Ensure that keys used in comparison operations are immutable & 243\\
MET12-J & Do not use finalizers 248 Chapter 8 Exceptional Behavior (ERR) 255 Rules 255 Risk Assessment Summary & 255\\
MSC00-J & Use SSLSocket rather than Socket for secure data exchange & 626\\
MSC01-J & Do not use an empty infinite loop & 630\\
MSC02-J & Generate strong random numbers & 632\\
MSC03-J & Never hard code sensitive information & 635\\
MSC04-J & Do not leak memory & 638\\
MSC05-J & Do not exhaust heap space & 647\\
MSC06-J & Do not modify the underlying collection when an iteration is in progress & 653\\
MSC07-J & Prevent multiple instantiations of singleton objects & 657\\
NUM00-J & Detect or prevent integer overflow & 106\\
NUM01-J & Do not perform bitwise and arithmetic operations on the same data & 114\\
NUM02-J & Ensure that division and modulo operations do not result in divide-by-zero errors & 119\\
NUM03-J & Use integer types that can fully represent the possible range of unsigned data & 121\\
NUM04-J & Do not use floating-point numbers if precise computation is required & 122\\
NUM05-J & Do not use denormalized numbers & 125\\
NUM06-J & Use the strictfp modifier for floating-point calculation consistency across platforms & 128\\
NUM07-J & Do not attempt comparisons with NaN & 132\\
NUM08-J & Check floating-point inputs for exceptional values & 134\\
NUM09-J & Do not use floating-point variables as loop counters & 136\\
NUM10-J & Do not construct BigDecimal objects from floating-point literals & 138\\
NUM11-J & Do not compare or inspect the string representation of floating-point values & 139\\
NUM12-J & Ensure conversions of numeric types to narrower types do not result in lost or misinterpreted data & 141\\
NUM13-J & Avoid loss of precision when converting primitive integers to floating-point 146 Chapter 6 Object Orientation (OBJ) 151 Rules 151 Risk Assessment Summary & 152\\
OBJ00-J & Limit extensibility of classes and methods with invariants to trusted subclasses only & 152\\
OBJ01-J & Declare data members as private and provide accessible wrapper methods & 159\\
OBJ02-J & Preserve dependencies in subclasses when changing superclasses & 162\\
OBJ03-J & Do not mix generic with nongeneric raw types in new code & 169\\
OBJ04-J & Provide mutable classes with copy functionality to safely allow passing instances to untrusted code & 175\\
OBJ05-J & Defensively copy private mutable class members before returning their references & 180\\
OBJ06-J & Defensively copy mutable inputs and mutable internal components & 185\\
OBJ07-J & Sensitive classes must not let themselves be copied & 189\\
OBJ08-J & Do not expose private members of an outer class from within a nested class & 192\\
OBJ09-J & Compare classes and not class names & 194\\
OBJ10-J & Do not use public static nonfinal variables & 197\\
OBJ11-J & Be wary of letting constructors throw exceptions 199 Chapter 7 Methods (MET) 209 Rules 209 Risk Assessment Summary & 210\\
SEC00-J & Do not allow privileged blocks to leak sensitive information across a trust boundary & 570\\
SEC01-J & Do not allow tainted variables in privileged blocks & 574\\
SEC02-J & Do not base security checks on untrusted sources & 577\\
SEC03-J & Do not load trusted classes after allowing untrusted code to load arbitrary classes & 579\\
SEC04-J & Protect sensitive operations with security manager checks & 582\\
SEC05-J & Do not use reflection to increase accessibility of classes, methods, or fields & 585\\
SEC06-J & Do not rely on the default automatic signature verification provided by URLClassLoader and java.util.jar & 592\\
SEC07-J & Call the superclass’s getPermissions() method when writing a custom class loader & 597\\
SEC08-J & Define wrappers around native methods 599xiv & 603\\
SER00-J & Maintain serialization compatibility during class evolution & 528\\
SER01-J & Do not deviate from the proper signatures of serialization methods & 531\\
SER02-J & Sign then seal sensitive objects before sending them across a trust boundary & 534\\
SER03-J & Do not serialize unencrypted, sensitive data & 541\\
SER04-J & Do not allow serialization and deserialization to bypass the security manager & 546\\
SER05-J & Do not serialize instances of inner classes & 549\\
SER06-J & Make defensive copies of private mutable components during deserialization & 551\\
SER07-J & Do not use the default serialized form for implementation-defined invariants & 553\\
SER08-J & Minimize privileges before deserializing from a privileged context & 558\\
SER09-J & Do not invoke overridable methods from thereadObject() method & 562\\
SER10-J & Avoid memory and resource leaks during serialization & 563\\
SER11-J & Prevent overwriting of externalizable objects 566 Chapter 16 Platform Security (SEC) 569 Rules 569 Risk Assessment Summary & 570\\
THI00-J & Do not invoke Thread.run() & 388\\
THI01-J & Do not invoke ThreadGroup methods & 390\\
THI02-J & Notify all waiting threads rather than a single thread & 394\\
THI03-J & Always invoke wait() and await() methods inside a loop & 401\\
THI04-J & Ensure that threads performing blocking operations can be terminated & 404\\
THI05-J & Do not use Thread.stop() to terminate threads 412 Chapter 12 Thread Pools (TPS) 417 Rules 417 Risk Assessment Summary & 417\\
TPS00-J & Use thread pools to enable graceful degradation of service during traffic bursts & 418\\
TPS01-J & Do not execute interdependent tasks in a bounded thread pool & 421\\
TPS02-J & Ensure that tasks submitted to a thread pool are interruptible & 428\\
TPS03-J & Ensure that tasks executing in a thread pool do not fail silently & 431\\
TPS04-J & Ensure ThreadLocal variables are reinitialized when using thread pools 436 xiii Chapter 13 Thread-Safety Miscellaneous (TSM) 441 Rules 441 Risk Assessment Summary & 441\\
TSM00-J & Do not override thread-safe methods with methods that are not thread-safe & 442\\
TSM01-J & Do not let the this reference escape during object construction & 445\\
TSM02-J & Do not use background threads during class initialization & 454\\
TSM03-J & Do not publish partially initialized objects 459 Chapter 14 Input Output (FIO) 467 Rules 467 Risk Assessment Summary & 468\\
VNA00-J & Ensure visibility when accessing shared primitive variables & 302\\
VNA01-J & Ensure visibility of shared references to immutable objects & 306\\
VNA02-J & Ensure that compound operations on shared variables are atomic & 309\\
VNA03-J & Do not assume that a group of calls to independently atomic methods is atomic & 317\\
VNA04-J & Ensure that calls to chained methods are atomic & 323\\
VNA05-J & Ensure atomicity when reading and writing 64-bit values 328 Chapter 10 Locking (LCK) 331 Rules 331 Risk Assessment Summary & 332\\
\end{longtable}
\end{document}