\documentclass[11pt]{article}

% ---------- Safe, common packages ----------
\usepackage[margin=1in]{geometry}
\usepackage[T1]{fontenc}
\usepackage[utf8]{inputenc}
\usepackage{lmodern}
\usepackage{microtype}
\usepackage{setspace}
\usepackage{hyperref}
\usepackage{graphicx}
\usepackage{booktabs}
\usepackage{array}
\usepackage{tabularx}
\usepackage{enumitem}
\usepackage{amsmath}

\hypersetup{
  colorlinks=true,
  linkcolor=black,
  urlcolor=blue
}

% ---------- Simple helpers ----------
\setlist[itemize]{itemsep=2pt, topsep=4pt}
\setlist[enumerate]{itemsep=4pt, topsep=6pt}
\newcommand{\Section}[1]{\vspace{0.6em}\noindent\textbf{\Large #1}\par\vspace{0.25em}}
\newcommand{\Subsection}[1]{\vspace{0.4em}\noindent\textbf{\large #1}\par\vspace{0.2em}}
\newcommand{\meta}[2]{\noindent\textbf{#1:} #2\par}
\newcolumntype{L}{>{\raggedright\arraybackslash}X}

\begin{document}
\begin{center}
  {\LARGE \textbf{New York-Style Bagel Recipe}}\\[2pt]
  \small Last updated on: August 26, 2024
\end{center}

\Section{Overview}
This easy homemade New York-Style Bagel recipe uses basic pantry ingredients to make classic, chewy bagels with a soft, shiny crust. The dough is boiled, then baked for authentic New York texture and flavor. Same-day method; ready in about 2 hours.

\Section{At a Glance}
\begin{tabularx}{\textwidth}{@{} l L @{}}
\toprule
\textbf{Yield} & Makes 8 medium-sized bagels \\
\textbf{Prep Time} & 20 minutes \\
\textbf{Cook Time} & 20 minutes \\
\textbf{Additional Time} & 1 hour 20 minutes (rise and rest) \\
\textbf{Total Time} & About 2 hours \\
\bottomrule
\end{tabularx}

\Section{Ingredients}

\Subsection{Bagel Dough}
\begin{itemize}
  \item 2 teaspoons (6 g) active dry yeast
  \item 4\,$\tfrac{1}{2}$ teaspoons (19 g) granulated sugar
  \item 1\,$\tfrac{1}{4}$ cups (300 ml) warm water, plus up to $\pm$\,\,$\tfrac{1}{4}$ cup (60 ml) more as needed
  \item 3\,$\tfrac{1}{2}$ cups (440 g) bread or high-gluten flour \\
        \hspace*{1.5em}\emph{(up to an additional $\tfrac{1}{2}$ cup / 60 g for kneading)}
  \item 1\,$\tfrac{1}{2}$ teaspoons (6 g) fine salt
\end{itemize}

\Subsection{Optional Toppings}
Caraway seeds, cinnamon sugar, coarse salt, minced fresh garlic, minced fresh onion, poppy seeds, sesame seeds, everything bagel seasoning, or any combination you like.

\Section{Instructions}

\begin{enumerate}
  \item \textbf{Proof the yeast.} In a small bowl, combine 1/2 cup (120 ml) of the warm water (about $105$--$115^\circ$F / $40.5$--$46^\circ$C) with the sugar and yeast. Do not stir for 5 minutes. Then stir to dissolve completely.
  \item \textbf{Make the dough.} In a large bowl, mix flour and salt. Make a well and pour in the yeast mixture. Add 1/3 cup (80 ml) warm water and begin mixing, adding the remaining scant 1/2 cup (about 100 ml) as needed to form a moist, firm dough. Depending on flour, humidity, and altitude, you may need an additional 2--4 tablespoons up to about 1/4 cup (60 ml).
  \item \textbf{Knead.} Turn onto a lightly floured surface and knead about 10 minutes, working in as much flour as needed to achieve a smooth, elastic, and fairly stiff dough.
  \item \textbf{First rise.} Place in a lightly oiled bowl, turn to coat, cover with a damp towel, and let rise in a warm place for about 1 hour, until doubled. Punch down and rest 10 minutes.
  \item \textbf{Divide and pre-shape.} Divide into 8 equal pieces. Shape each into a taut round by cupping your hand in a ``C'' shape and rotating against the unfloured work surface to create surface tension.
  \item \textbf{Shape bagels.} Dust a finger with flour, poke a hole through the center of each round, then gently stretch the hole until it is about one-third the diameter of the bagel. Place on a lightly oiled or parchment-lined sheet.
  \item \textbf{Rest and preheat.} Cover with a damp towel and rest 10 minutes. Meanwhile, preheat the oven to 425$^\circ$F / 220$^\circ$C (Gas Mark 7) and bring a large pot of water to a gentle boil.
  \item \textbf{Boil.} Reduce the boil to a simmer. Using a slotted spoon or skimmer, lower in as many bagels as fit comfortably without crowding. They will float quickly. Simmer 1 minute, flip, and simmer 1 minute more. For an extra-chewy, classic New York texture, boil 2 minutes per side.
  \item \textbf{Top.} While still tacky from the water, add desired toppings. (Alternatively, brush with a light egg wash before topping, if you prefer extra sheen.)
  \item \textbf{Bake.} Transfer to an oiled or parchment-lined baking sheet and bake 20--25 minutes until uniformly golden brown (start checking at 20).
  \item \textbf{Cool (a little) and serve.} Cool on a wire rack for a few minutes---or slice warm and add a generous schmear of cream cheese.
\end{enumerate}

\Section{Serving Suggestions}
\begin{itemize}
  \item \textbf{Classic Schmear:} Plain or flavored cream cheese.
  \item \textbf{Butter or Butter \& Jam:} Simple and great, especially with high-quality butter.
  \item \textbf{Lox Bagel:} Cream cheese, lox, tomato, red onion, cucumber, and capers.
  \item \textbf{Avocado:} Toasted bagel with smashed avocado; add bacon if desired.
  \item \textbf{Egg \& Cheese:} Deli-style egg-and-cheese (with bacon optional); season and dress to taste.
\end{itemize}

\Section{Notes}
\Subsection{Water Amount \& Dough Feel}
Water quantities are guidelines. Aim for a smooth, cohesive dough---not dry or crumbly. Adjust during mixing (not after rising).

\Subsection{Shinier, Slightly Sweeter Bagels}
For extra sheen and gentle sweetness, add about 1 teaspoon barley malt syrup to the dough and a generous tablespoon to the boiling water. (Brown sugar or honey in the boil is a reasonable substitute.)

\Subsection{Flour Options}
Bread or high-gluten flour yields the chewiest texture. All-purpose flour works well (slightly less chewy). For whole-wheat, use half whole-wheat and half bread flour. Spelt variations can work; sifted spelt lightens the crumb.

\Subsection{Boiling Tips}
Keep the water at a simmer rather than a rolling boil for smoother crusts. Handle shaped dough gently to avoid wrinkles.

\Subsection{Pan Prep}
Use lightly oiled parchment or a silicone mat to minimize sticking; a light oiling of the parchment adds extra insurance.

\Subsection{Altitude}
At higher elevations, dough rises faster. Consider reducing yeast to 1 teaspoon and allow a full rise until doubled (timing may vary).

\Section{Make-Ahead \& Storage}
\begin{itemize}
  \item \textbf{Cold Rise:} After kneading, cover and refrigerate overnight for deeper flavor. Bring to room temp for about 30 minutes before shaping.
  \item \textbf{Par-Bake:} Bake 10--15 minutes until just faintly golden. Cool, freeze airtight. From frozen, bake an additional 10--15 minutes until golden.
  \item \textbf{Freezing:} Cool completely, slice, and freeze in a freezer bag. Toast straight from frozen if desired.
  \item \textbf{Room Temp Storage:} In a resealable bag up to 2 days.
\end{itemize}

\Section{Nutrition (Approximate, per 1 of 8 bagels)}
\begin{tabular}{@{}ll@{}}
\toprule
Calories & 228.4 \\
Carbohydrates & 44.4 g \\
Protein & 6 g \\
Total Fat & --- \\
Trans Fat & 1.4 g \\
Cholesterol & 0 mg \\
Sodium & 441 mg \\
Fiber & 1.7 g \\
Sugar & 2.4 g \\
\bottomrule
\end{tabular}

\Section{FAQs}

\Subsection{Yeast Choices}
\textbf{Fresh yeast:} Use approximately 2.5:1 fresh-to-active-dry by weight (about 15 g fresh for this recipe). Fresh yeast need not be proofed.\\
\textbf{Instant yeast:} Substitute 1:1 by weight; no proofing required---add directly with flour and sugar.\\
\textbf{Milder yeast flavor:} Reduce active dry yeast to 1 teaspoon and double the first rise.

\Subsection{Temperature}
Ideal water temperature for yeast activation is roughly $105$--$115^\circ$F ($40.5$--$46^\circ$C).

\Subsection{Mixer/Bread Maker}
A stand mixer with dough hook on the lowest speed for 5--6 minutes works well. For double batches, check the manufacturer's capacity guidance.

\Subsection{Smoothness}
Handle gently and keep the poach at a simmer rather than a rolling boil to avoid a wrinkly crust.

\Subsection{Egg Wash}
Optional. Toppings adhere fine to wet, freshly boiled dough. Skip egg wash for a vegan-friendly result.

\Subsection{Sizing Up}
Larger bagels need a few extra minutes in the oven; bake to uniform golden brown.

\vfill
\begin{center}
  \footnotesize
  \textit{Cuisine: American \quad Category: How-To \quad Source adaptation: Home kitchen notes.}
\end{center}

\end{document}

