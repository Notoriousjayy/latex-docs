\documentclass[11pt]{article}

% ---------- Stable packages ----------
\usepackage[margin=1in]{geometry}
\usepackage[T1]{fontenc}
\usepackage[utf8]{inputenc}
\usepackage{lmodern}
\usepackage{microtype}
\usepackage{amsmath} % for \tfrac, aligned math
\usepackage{booktabs}
\usepackage{tabularx}
\usepackage{enumitem}
\usepackage{hyperref}
\usepackage{xcolor}

\hypersetup{
  colorlinks=true,
  linkcolor=black,
  urlcolor=blue
}

% ---------- Helpers ----------
\newcommand{\Section}[1]{\vspace{0.8em}\noindent{\Large\bfseries #1}\par\vspace{0.35em}}
\newcommand{\Subsection}[1]{\vspace{0.4em}\noindent{\large\bfseries #1}\par\vspace{0.15em}}
\setlist[itemize]{itemsep=3pt, topsep=4pt}
\setlist[enumerate]{itemsep=5pt, topsep=6pt}
\newcolumntype{L}{>{\raggedright\arraybackslash}X}

\begin{document}

\begin{center}
  {\LARGE \textbf{Vegetable Soup}}\\[2pt]
  \small By Lidey Heuck \quad\textbullet\quad Updated Oct.\ 15, 2024 \quad\textbullet\quad Rating: 5.0 (3{,}234)
\end{center}

\Section{Overview}
A simple, one-pot vegetable soup that is highly customizable. Consider this a template: swap in what's on hand and add vegetables according to their cooking times. Hearty greens like kale or chard can replace spinach; rosemary or an Italian seasoning blend may stand in for the oregano \& thyme. A pinch of cumin brings subtle smokiness, and a 15~oz can of drained, rinsed white beans makes it more filling. Finish as-is or with Parmesan, thinly sliced scallions, and torn basil.

\Section{At a Glance}
\begin{tabularx}{0.96\textwidth}{@{} l L @{}}
\toprule
\textbf{Yield} & 6--8 servings \\
\textbf{Total Time} & About 1 hour \\
\textbf{Prep Time} & 20 minutes \\
\textbf{Cook Time} & 45 minutes \\
\textbf{Method} & One-pot simmer (stovetop) \\
\bottomrule
\end{tabularx}

\Section{Ingredients}
\begin{itemize}
  \item 3~Tbsp extra-virgin olive oil
  \item 1 large yellow onion, chopped
  \item 3 medium carrots, diced (about 2 cups)
  \item 2--3 large celery stalks, diced (about \(1\tfrac{1}{2}\) cups)
  \item 4 garlic cloves, minced (about 2~Tbsp)
  \item 2~tsp fresh thyme \emph{or} 1~tsp dried
  \item \(1\tfrac{1}{2}\)~tsp dried oregano
  \item Pinch of crushed red pepper flakes
  \item Kosher salt and freshly ground black pepper
  \item 2--3 Yukon Gold potatoes, diced (about 2 cups)
  \item 1~Tbsp tomato paste
  \item 8 cups (2 quarts) vegetable broth
  \item 1 (15~oz) can diced tomatoes, with juices
  \item 1~cup frozen or fresh chopped green beans
  \item 1~cup frozen or fresh corn kernels
  \item 1~cup frozen or fresh green peas
  \item 2~cups baby spinach
  \item \( \tfrac{1}{3} \)~cup chopped fresh parsley, plus more for serving
  \item 1~Tbsp red wine vinegar
\end{itemize}

\Subsection{Optional Add-Ins \& Finishes}
\begin{itemize}
  \item 1 (15~oz) can white beans, drained and rinsed (for more body)
  \item Fresh rosemary or Italian seasoning (in place of oregano \& thyme)
  \item Grated Parmesan, thinly sliced scallions, and torn basil for serving
\end{itemize}

\Section{Preparation}
\begin{enumerate}
  \item \textbf{Sweat aromatics.} In a large pot or Dutch oven, heat the oil over medium. Add onion, carrots, and celery; cook, stirring occasionally, until crisp-tender, about 10 minutes. Add garlic, thyme, oregano, red pepper flakes, 1~tsp kosher salt (use 2~tsp if using low-sodium broth), and 1~tsp black pepper; cook until fragrant, about 1 minute.
  \item \textbf{Toast tomato paste.} Add potatoes and tomato paste; cook, stirring often, until the paste begins to brown on the bottom, 2--3 minutes.
  \item \textbf{Simmer base.} Pour in broth and diced tomatoes with their juices; bring to a boil over medium-high. Reduce heat and simmer, partially covered, until potatoes are fork-tender, 20--25 minutes.
  \item \textbf{Add tender veg.} Stir in green beans, corn, and peas; return to a simmer and cook until green beans are tender, 3--5 minutes.
  \item \textbf{Finish \& season.} Off heat, add spinach, parsley, and vinegar; stir until spinach wilts. Taste and adjust salt and pepper. Serve hot, topped with more parsley (and optional garnishes).
\end{enumerate}

\Section{Customization Guide}
\begin{itemize}
  \item \textbf{Vegetable swaps:} Fennel, zucchini, or broccoli are excellent additions. Add firmer veg earlier; tender veg later.
  \item \textbf{Greens:} Substitute kale or Swiss chard for spinach; simmer kale 5 minutes, chard 2--3 minutes.
  \item \textbf{Spice profile:} Replace oregano/thyme with 1--2~tsp Italian seasoning, or add \(\tfrac{1}{2}\)~tsp ground cumin for subtle smoke.
  \item \textbf{Heft:} Add white beans with the broth; simmer 10 minutes.
\end{itemize}

\Section{Storage \& Freezer Tip}
Soup keeps up to 5 days refrigerated or 3 months frozen.
\begin{itemize}
  \item \textbf{To freeze:} Cool soup to room temperature in the pot. Transfer to airtight containers and freeze up to 3 months.
  \item \textbf{To reheat:} Thaw overnight in the refrigerator, or unmold by running the container under hot tap water and transfer to a saucepan. Simmer gently, partially covered, until heated through, adding water or broth as needed.
\end{itemize}

\vfill
\begin{center}
  \small \emph{Serving suggestions:} Parmesan, scallions, and basil elevate the bowl; crusty bread on the side never hurts.
\end{center}

\end{document}
