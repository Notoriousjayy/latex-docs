\documentclass[11pt]{article}

% ---------- Safe, common packages ----------
\usepackage[margin=1in]{geometry}
\usepackage[T1]{fontenc}
\usepackage[utf8]{inputenc}
\usepackage{lmodern}
\usepackage{microtype}
\usepackage{hyperref}
\usepackage{graphicx}
\usepackage{booktabs}
\usepackage{array}
\usepackage{tabularx}
\usepackage{enumitem}
\usepackage{amsmath}
\usepackage{setspace}

\hypersetup{
  colorlinks=true,
  linkcolor=black,
  urlcolor=blue
}

% ---------- Simple helpers ----------
\setlist[itemize]{itemsep=2pt, topsep=4pt}
\setlist[enumerate]{itemsep=4pt, topsep=6pt}
\newcommand{\Section}[1]{\vspace{0.6em}\noindent\textbf{\Large #1}\par\vspace{0.25em}}
\newcommand{\Subsection}[1]{\vspace{0.4em}\noindent\textbf{\large #1}\par\vspace{0.2em}}
\newcolumntype{L}{>{\raggedright\arraybackslash}X}

\begin{document}

\begin{center}
  {\LARGE \textbf{Easy Homemade Biscuits}}\\[4pt]
  \small by Sam Merritt \quad\textbullet\quad Published: April 25, 2018
\end{center}

\Section{Overview}
Buttery, soft, flaky biscuits made completely from scratch with everyday ingredients. This all-butter, no-shortening dough comes together quickly, uses a simple folding (laminating) technique for layers, and bakes up tall and tender.

\Section{At a Glance}
\begin{tabularx}{\textwidth}{@{} l L @{}}
\toprule
\textbf{Yield} & 6 biscuits (recipe may be doubled for 12) \\
\textbf{Prep Time} & 15 minutes \\
\textbf{Bake Time} & 12 minutes \\
\textbf{Total Time} & 27 minutes \\
\textbf{Oven Temp} & 425$^\circ$F (220$^\circ$C) \\
\bottomrule
\end{tabularx}

\Section{Ingredients}
\begin{itemize}
  \item 2 cups (250 g) all-purpose flour
  \item 1 tablespoon baking powder
  \item 1 tablespoon granulated sugar
  \item 1 teaspoon salt
  \item 6 tablespoons (85 g) \textbf{very cold} unsalted butter \\
        \hspace*{1.5em}\emph{(European-style preferred but not required)}
  \item $\tfrac{3}{4}$ cup (177 ml) whole milk\footnote{Buttermilk or 2\% milk also work.}
\end{itemize}

\Subsection{Recommended Equipment}
Box grater; biscuit cutter (about $2\,\tfrac{3}{4}$ in / 7 cm); mixing bowls; parchment-lined baking sheet.

\Section{Instructions}
\begin{enumerate}
  \item \textbf{Chill the butter.} For best results, place the butter in the freezer 10--20 minutes before starting.
  \item \textbf{Preheat \& pan.} Heat oven to 425$^\circ$F (220$^\circ$C). Line a baking sheet with parchment paper.
  \item \textbf{Dry mix.} In a large bowl whisk together flour, baking powder, sugar, and salt.
  \item \textbf{Cut in butter (grater method preferred).} Grate the cold butter on a box grater directly into the dry mixture, or cut in with a pastry cutter, until the mixture resembles coarse crumbs with visible pea-sized bits of butter.
  \item \textbf{Add milk.} Pour in the milk and stir with a wooden spoon or spatula just until the dough comes together. Do not overmix.
  \item \textbf{Bring together \& fold for layers.} Turn the dough onto a well-floured surface. If sticky, dust with a little flour. Gently press together, then fold the dough in half, press to flatten; rotate 90$^\circ$ and repeat. Perform 5--6 total folds to \emph{laminate} (create flaky layers), taking care not to overwork or warm the butter.
  \item \textbf{Shape.} Using hands (not a rolling pin), pat the dough to about 1 in (2.5 cm) thick.
  \item \textbf{Cut.} Lightly flour a $2\,\tfrac{3}{4}$-inch (7 cm) biscuit cutter. Press straight down without twisting to cut as many rounds as possible, placing them less than 1/2 in (1 cm) apart on the prepared sheet.
  \item \textbf{Re-roll scraps.} Gently press the scraps together and cut additional biscuits to yield at least 6.
  \item \textbf{Bake.} Bake 12 minutes, or until the tops just begin to turn lightly golden brown.
  \item \textbf{Finish.} If desired, brush hot biscuits with melted salted butter. Serve warm.
\end{enumerate}

\Section{Key Techniques for Tall, Flaky Biscuits}
\begin{itemize}
  \item \textbf{Keep ingredients cold.} Cold butter and cold milk = better lift and flakier layers.
  \item \textbf{Grate the butter.} Freezing, then grating butter distributes it quickly without overworking.
  \item \textbf{Laminate lightly.} Fold the dough 5--6 times for defined layers; avoid heavy kneading.
  \item \textbf{Hands, not a pin.} Pat to thickness with your hands; a rolling pin can compress layers.
  \item \textbf{Cut straight down.} Do not twist the cutter; twisting seals edges and limits rise.
  \item \textbf{Close placement.} Setting biscuits less than 1/2 in apart helps them climb upward.
\end{itemize}

\Section{Notes}
\begin{itemize}
  \item \textbf{Butter choice.} Unsalted butter is recommended; if using salted butter, reduce added salt slightly.
  \item \textbf{Milk swaps.} Whole milk gives the richest flavor; buttermilk adds tang; 2\% works well.
  \item \textbf{Serving.} Best warm from the oven; brush with melted butter after baking if desired.
\end{itemize}

\Section{Nutrition (Approximate, per biscuit; 6 biscuits)}
\begin{tabular}{@{}ll@{}}
\toprule
Calories & 280 kcal \\
Carbohydrates & 36 g \\
Protein & 5 g \\
Fat & 13 g \\
Saturated Fat & 8 g \\
Cholesterol & 33 mg \\
Sodium & 405 mg \\
Potassium & 287 mg \\
Fiber & 1 g \\
Sugar & 4 g \\
Vitamin A & 399 IU \\
Calcium & 131 mg \\
Iron & 2 mg \\
\bottomrule
\end{tabular}

\vfill
\begin{center}
  \footnotesize
  \textit{Tip: Visible butter specks in the dough before baking are a good sign---they create steam and lift in the oven.}
\end{center}

\end{document}

