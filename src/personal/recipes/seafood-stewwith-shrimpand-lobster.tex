\documentclass[11pt]{article}
\usepackage{amsmath}% for \tfrac

% ---------- Common, stable packages ----------
\usepackage[margin=1in]{geometry}
\usepackage[T1]{fontenc}
\usepackage[utf8]{inputenc}
\usepackage{lmodern}
\usepackage{microtype}
\usepackage{hyperref}
\usepackage{booktabs}
\usepackage{array}
\usepackage{tabularx}
\usepackage{enumitem}

\hypersetup{
  colorlinks=true,
  linkcolor=black,
  urlcolor=blue
}

% ---------- Helpers ----------
\setlist[itemize]{itemsep=2pt, topsep=4pt}
\setlist[enumerate]{itemsep=4pt, topsep=6pt}
\newcommand{\Section}[1]{\vspace{0.6em}\noindent\textbf{\Large #1}\par\vspace{0.25em}}
\newcommand{\Subsection}[1]{\vspace{0.4em}\noindent\textbf{\large #1}\par\vspace{0.2em}}
\newcolumntype{L}{>{\raggedright\arraybackslash}X}

\begin{document}

\begin{center}
  {\LARGE \textbf{Seafood Stew with Shrimp and Lobster}}\\[2pt]
  \small Fragrant, Mediterranean-leaning broth with coriander, garlic, white wine, lemon, baby greens, and tender shrimp \& lobster.
\end{center}

\Section{Overview}
Inspired by Mediterranean coastal cooking, this one-pot stew layers aromatics, white wine, and lemon with quick-poached lobster and succulent shrimp. Baby kale and fresh herbs finish the broth for a bright, comforting bowl that feels special but cooks fast.

\Section{At a Glance}
\begin{tabularx}{\textwidth}{@{} l L @{}}
\toprule
\textbf{Yield} & Serves 4 \\
\textbf{Prep Time} & 20 minutes \\
\textbf{Cook Time} & 20 minutes \\
\textbf{Total Time} & 40 minutes \\
\textbf{Cuisine} & Mediterranean \quad \textbf{Course} \, Entree (Soup/Stew) \\
\bottomrule
\end{tabularx}

\Section{Ingredients}

\Subsection{Seafood \& Aromatics}
\begin{itemize}
  \item 1 lb (450 g) large shrimp or prawns, peeled \& deveined, tail-on
  \item 1 lb (about two) lobster tails
  \item 1 \(\tfrac{1}{2}\) lemons, divided (half for shrimp; remainder for broth)
  \item 1/2 red onion, roughly chopped
  \item 2 Roma tomatoes, diced
  \item 2 cups baby kale (lightly packed)
  \item 2 green onions, chopped
  \item 1 cup fresh parsley leaves
  \item 4 large garlic cloves, chopped \emph{(divided; 1 for poaching, 3 for base)}
\end{itemize}

\Subsection{Pantry \& Seasoning}
\begin{itemize}
  \item 2 tbsp extra-virgin olive oil
  \item 1 tsp red pepper flakes (to taste)
  \item 1 tsp ground coriander
  \item 1/2 cup dry white wine (e.g., Sauvignon Blanc, Pinot Grigio)
  \item 2 \(\times\) 15-oz (425 g) cans low-sodium chicken broth \emph{(or seafood/vegetable stock)}
  \item 4 cups water
  \item 1 bay leaf
  \item 1/2 tsp ground ginger
  \item Kosher salt \& freshly ground black pepper
\end{itemize}

\Section{Equipment}
Large Dutch oven or soup pot; medium pot or deep sauté pan; fine-mesh strainer; tongs; kitchen shears; ladle.

\Section{Method}

\Subsection{1) Prepare the Seafood}
\begin{enumerate}
  \item \textbf{Make the poaching liquid.} In a medium pot, combine 4 cups water, bay leaf, ground ginger, 1 chopped garlic clove, and a good pinch each of salt and pepper. Bring to a boil over high heat.
  \item \textbf{Season the shrimp.} In a bowl, toss shrimp with the juice of 1/2 lemon, plus a pinch of salt and pepper; set aside.
  \item \textbf{Poach the lobster.} Add lobster tails to the boiling liquid, reduce to medium-high, cover, and cook until the shells turn bright red, about 3 minutes. Transfer tails to a board to cool. Turn off the heat and reserve the poaching liquid.
  \item \textbf{Shell and chop.} When cool enough to handle, use kitchen shears to cut the shells lengthwise; remove the meat and coarsely chop into bite-size pieces.
\end{enumerate}

\Subsection{2) Build the Stew Base}
\begin{enumerate}
  \item \textbf{Sauté aromatics.} In a large pot, heat olive oil over medium-high. Add red onion and red pepper flakes; cook until softened, about 5 minutes. Stir in the remaining 3 chopped garlic cloves; cook 1 minute until fragrant.
  \item \textbf{Season \& reduce.} Add diced tomatoes, ground coriander, and a pinch of salt and pepper. Stir 1--2 minutes. Pour in the white wine and simmer 3--5 minutes to reduce slightly.
  \item \textbf{Finish the broth.} Add the chicken (or seafood/vegetable) broth. Strain the reserved lobster poaching liquid through a fine-mesh strainer into the pot. Squeeze in the remaining lemon (to taste). Bring to a lively simmer.
\end{enumerate}

\Subsection{3) Add Seafood and Finish}
\begin{enumerate}
  \item \textbf{Cook shrimp \& lobster.} With the broth at a high simmer, add the seasoned shrimp. After 1 minute, gently stir in the chopped lobster.
  \item \textbf{Greens \& herbs.} Stir in baby kale, green onions, and parsley. Remove from heat, cover, and let stand 5 minutes to soften the greens.
  \item \textbf{Serve.} Taste and adjust salt, pepper, lemon, and heat. Ladle into warm bowls and serve with crusty bread.
\end{enumerate}

\Section{Chef's Notes \& Tips}
\begin{itemize}
  \item \textbf{Wine choices:} Dry, unoaked whites with citrus notes (Sauvignon Blanc, Pinot Grigio, Chablis) complement the lemony broth. For alcohol-free, use additional stock and a splash of white wine vinegar or extra lemon to balance acidity.
  \item \textbf{Stock upgrades:} Substitute seafood stock or add a few tablespoons of bottled clam juice for extra brine. Save shrimp and lobster shells for a quick house stock.
  \item \textbf{Heat \& brightness:} Adjust red pepper flakes to taste; finish with extra lemon or a drizzle of good olive oil at the table.
  \item \textbf{Don’t overcook seafood:} Shrimp turn pink and curl quickly; lobster is already cooked from poaching. Keep the stew at a simmer (not a boil) once seafood is added.
\end{itemize}

\Section{Make-Ahead, Storage \& Reheating}
\begin{itemize}
  \item \textbf{Prep ahead:} Poach and chop lobster, peel/devein shrimp, and chop vegetables up to 1 day in advance. Refrigerate separately (keep lobster in a bit of cooled poaching liquid).
  \item \textbf{Leftovers:} Store cooled stew in an airtight container up to 2 days, ensuring seafood is submerged in broth.
  \item \textbf{Reheat gently:} Warm on the stovetop over medium or lower just until heated through. Avoid boiling to prevent rubbery seafood.
\end{itemize}

\Section{Serving Suggestions}
Crusty bread (baguette, sourdough) for dipping is highly recommended. Pair with a simple citrusy salad or roasted vegetables; pour the same wine used in cooking alongside.

\vfill
\begin{center}
  \footnotesize
  \textit{All quantities are scalable; keep poaching and simmer times brief to protect seafood texture.}
\end{center}

\end{document}

