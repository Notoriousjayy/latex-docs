\documentclass[11pt]{article}

% ---------- Safe, common packages ----------
\usepackage[margin=1in]{geometry}
\usepackage[T1]{fontenc}
\usepackage[utf8]{inputenc}
\usepackage{lmodern}
\usepackage{microtype}
\usepackage{hyperref}
\usepackage{graphicx}
\usepackage{booktabs}
\usepackage{array}
\usepackage{tabularx}
\usepackage{enumitem}
\usepackage{amsmath}
\usepackage{setspace}

\hypersetup{
  colorlinks=true,
  linkcolor=black,
  urlcolor=blue
}

% ---------- Simple helpers ----------
\setlist[itemize]{itemsep=2pt, topsep=4pt}
\setlist[enumerate]{itemsep=4pt, topsep=6pt}
\newcommand{\Section}[1]{\vspace{0.6em}\noindent\textbf{\Large #1}\par\vspace{0.25em}}
\newcommand{\Subsection}[1]{\vspace{0.4em}\noindent\textbf{\large #1}\par\vspace{0.2em}}
\newcolumntype{L}{>{\raggedright\arraybackslash}X}

\begin{document}

\begin{center}
  {\LARGE \textbf{Creamy Vegetable Soup}}\\[4pt]
  \small By Yumna Jawad \quad\textbullet\quad Updated: March 3, 2025 \quad\textbullet\quad Rating: 4.76/5 (95 votes)
\end{center}

\Section{Overview}
A cozy, comforting vegan soup that is naturally gluten free, easy to make, and loaded with vegetables. Blend as smooth or as chunky as you like, and finish with plant-based milk for creamy richness without heaviness.

\Section{At a Glance}
\begin{tabularx}{\textwidth}{@{} l L @{}}
\toprule
\textbf{Yield} & Serves 6 \\
\textbf{Diet} & Vegan (with plant milk \& vegetable broth), Gluten Free \\
\textbf{Prep Time} & 10 minutes \\
\bottomrule
\end{tabularx}

\Section{Ingredients}
\Subsection{Base}
\begin{itemize}
  \item 2 cloves garlic, minced
  \item 1 medium onion, finely chopped
  \item 2 tablespoons extra-virgin olive oil
  \item 2 medium potatoes, peeled and roughly chopped (Yukon Gold preferred)
  \item 6 celery stalks, divided (4 for simmering; 2 added later for texture)
  \item 6 cups vegetable broth \emph{(sub low-sodium chicken or beef broth if not vegetarian)}
  \item 1--2 teaspoons dried thyme \emph{(or about 2 tablespoons fresh thyme)}
  \item Salt and freshly ground black pepper, to taste
\end{itemize}

\Subsection{Finish \& Mix-ins}
\begin{itemize}
  \item 1\;1/2 cups unsweetened almond milk \emph{(or other plant milk; regular milk or a splash of cream also work)}
  \item Frozen peas and frozen corn, to taste \emph{(or other frozen veg such as green beans, cauliflower, broccoli)}
  \item Fresh parsley or green onions (scallions), thinly sliced, for garnish
\end{itemize}

\Section{Popular Additions}
\begin{itemize}
  \item \textbf{Tomatoes:} Stir in one 15-ounce can chopped tomatoes with the broth for gentle acidity.
  \item \textbf{Greens:} Add a few handfuls of kale (with peas/corn) or baby spinach in the last 2--3 minutes.
  \item \textbf{Extra creaminess:} Blend in up to 1/4 cup room-temperature sour cream, or dollop at serving (non-vegan).
\end{itemize}

\Section{How to Make Creamy Vegetable Soup}
\begin{enumerate}
  \item \textbf{Sweat the mirepoix.} Heat olive oil in a large Dutch oven over medium heat until shimmering. Add onion, carrots, and celery (use 4 stalks; reserve 2) with a pinch of salt and pepper. Cook, stirring, until softened but not browned.
  \item \textbf{Add potatoes, garlic, and thyme.} Stir in potatoes, minced garlic, and thyme; cook 1--2 minutes until fragrant.
  \item \textbf{Simmer.} Add vegetable broth, bring to a boil, then reduce to a gentle simmer. Cover and cook until potatoes are very tender when pierced with a fork.
  \item \textbf{Purée most of the soup.} Use an immersion blender to blend until mostly smooth, leaving some texture. \emph{No immersion blender?} Carefully blend in batches in a countertop blender, then return to the pot, reserving about one-third unblended for chunkiness if desired.
  \item \textbf{Finish with milk and veggies.} Stir in almond milk, the remaining 2 celery stalks (thinly sliced and par-cooked if you like them very tender), and frozen peas and corn. Simmer gently, covered, until heated through and slightly thickened. Do not boil after adding milk.
  \item \textbf{Season \& serve.} Adjust salt and pepper. Ladle into bowls and garnish with parsley or scallions. Serve hot with crusty bread.
\end{enumerate}

\Section{My Best Tips}
\begin{itemize}
  \item Cut vegetables small (especially carrots) for faster cooking and better flavor extraction.
  \item No need to peel Yukon Gold potatoes; the skins add nutrients and body.
  \item Season lightly at each step; adjust at the end to avoid oversalting.
  \item Keep heat moderate when sweating onions, carrots, and celery to draw out sweetness without browning.
\end{itemize}

\Section{What to Serve}
\begin{itemize}
  \item Warm pita or crusty bread
  \item White bean \& avocado sandwich
  \item Zucchini fries
  \item Toast with hummus
  \item Grilled cheese (if not vegan)
\end{itemize}

\vfill
\begin{center}
  \footnotesize
  \textit{Notes: Use vegetable broth and plant milk for a fully vegan soup. For thicker soup, blend more; for chunkier texture, blend less.}
\end{center}

\end{document}

