\documentclass[11pt]{article}

% ---------- Safe, common packages ----------
\usepackage[margin=1in]{geometry}
\usepackage[T1]{fontenc}
\usepackage[utf8]{inputenc}
\usepackage{lmodern}
\usepackage{microtype}
\usepackage{hyperref}
\usepackage{graphicx}
\usepackage{booktabs}
\usepackage{array}
\usepackage{tabularx}
\usepackage{enumitem}
\usepackage{amsmath}
\usepackage{setspace}

\hypersetup{
  colorlinks=true,
  linkcolor=black,
  urlcolor=blue
}

% ---------- Simple helpers ----------
\setlist[itemize]{itemsep=2pt, topsep=4pt}
\setlist[enumerate]{itemsep=4pt, topsep=6pt}
\newcommand{\Section}[1]{\vspace{0.6em}\noindent\textbf{\Large #1}\par\vspace{0.25em}}
\newcommand{\Subsection}[1]{\vspace{0.4em}\noindent\textbf{\large #1}\par\vspace{0.2em}}
\newcommand{\meta}[2]{\noindent\textbf{#1:} #2\par}
\newcolumntype{L}{>{\raggedright\arraybackslash}X}

\begin{document}

\begin{center}
  {\LARGE \textbf{Cioppino (Fisherman’s Stew)}}\\[4pt]
  \small By Jennifer Segal \quad\textbullet\quad Updated: February 7, 2024
\end{center}

\Section{Overview}
Cioppino is a rustic Italian--American seafood stew brimming with shrimp, clams, and tender fish in a rich tomato--wine broth. Serve with plenty of bread for soaking up the flavorful juices, and set out a bowl for shells.

\Section{At a Glance}
\begin{tabularx}{\textwidth}{@{} l L @{}}
\toprule
\textbf{Cuisine} & Italian--American \\
\textbf{Serve With} & Garlic bread, focaccia, or a baguette \\
\textbf{Notes} & Fish is baked separately so it stays tender and intact \\
\bottomrule
\end{tabularx}

\Section{What You’ll Need}
\begin{itemize}
  \item \textbf{Shallots \& Garlic:} Aromatics that build the flavor base.
  \item \textbf{Dry White Wine:} Adds brightness and acidity.
  \item \textbf{Canned Crushed Tomatoes:} For body, tang, and color.
  \item \textbf{Clam Juice:} Briny depth that makes the broth taste of the sea.
  \item \textbf{Crushed Red Pepper, Oregano, Thyme:} Gentle heat, earthiness, and herbal aroma.
  \item \textbf{Firm Fish Fillets:} Halibut, cod, salmon, or snapper, cut into chunks.
  \item \textbf{Butter:} Stirred in at the end for a velvety finish.
  \item \textbf{Littleneck Clams:} They open in the pot and enrich the broth.
  \item \textbf{Shrimp:} Meaty sweetness; cooks quickly in the stew.
\end{itemize}

\Section{Ingredients}
\Subsection{Broth Base}
\begin{itemize}
  \item Olive oil (use 1/4 cup to start)
  \item Shallots, finely chopped
  \item Garlic, minced
  \item Dry white wine
  \item Canned crushed tomatoes
  \item Clam juice
  \item Sugar (a pinch, to balance acidity)
  \item Kosher salt (1 tsp to season the broth; adjust to taste)
  \item Crushed red pepper flakes
  \item Dried oregano
  \item Fresh thyme sprigs
  \item Water (about 1 cup)
  \item Unsalted butter
\end{itemize}

\Subsection{Seafood \& Finish}
\begin{itemize}
  \item Firm-fleshed fish fillets, cut into large chunks
  \item Olive oil (about 2 tbsp to toss with fish)
  \item Kosher salt (about 3/4 tsp to season fish)
  \item Littleneck clams, scrubbed
  \item Large shrimp, peeled and deveined
  \item Fresh thyme, chopped (for finishing)
  \item Flat-leaf parsley, chopped (optional garnish)
\end{itemize}

\Section{Step-by-Step Instructions}
\begin{enumerate}
  \item \textbf{Start the aromatics.} Heat 1/4 cup olive oil in a large pot over medium heat. Add shallots and cook until soft and translucent, about 5 minutes. Add garlic; cook 1 minute more without browning.
  \item \textbf{Deglaze.} Add white wine and increase heat to high. Boil until reduced by about half, 3--4 minutes.
  \item \textbf{Build the broth.} Add crushed tomatoes, clam juice, a pinch of sugar, 1 teaspoon salt, red pepper flakes, oregano, thyme sprigs, and 1 cup water. Bring to a boil, then reduce heat and simmer covered for 25 minutes.
  \item \textbf{Roast the fish.} While the broth simmers, toss fish chunks with 2 tablespoons olive oil and 3/4 teaspoon salt. Arrange on a lined baking sheet and bake at 400\,$^\circ$F (200\,$^\circ$C) for about 10 minutes, just until cooked through. Cover to keep warm. \emph{(Baking separately keeps the fish tender and intact.)}
  \item \textbf{Enrich.} Remove thyme sprigs from the pot; stir in a knob of butter until melted.
  \item \textbf{Cook the clams.} Add clams; return to a gentle simmer. Cover and cook about 6 minutes, until most clams have opened.
  \item \textbf{Add shrimp.} Gently stir in shrimp; cover and cook until just opaque and clams are fully opened, about 5 minutes. Discard any unopened clams. Stir in chopped fresh thyme; taste and adjust seasoning.
  \item \textbf{Serve.} Divide warm baked fish among bowls. Ladle stew over the fish, distributing clams and shrimp evenly. Garnish with parsley if using. Serve with plenty of crusty bread.
\end{enumerate}

\Section{Notes \& Tips}
\begin{itemize}
  \item \textbf{Variations:} Add mussels, crab, or lobster when available.
  \item \textbf{Heat level:} Adjust crushed red pepper to taste.
  \item \textbf{Make-ahead:} The tomato--wine base can be made several hours ahead and reheated; add seafood just before serving.
\end{itemize}

\Section{Frequently Asked Questions}
\Subsection{How do I check that clams are alive?}
Tap any open shells; discard those that don’t close. Discard cracked or broken shells.

\Subsection{How do I scrub clams?}
Rinse under cold water and scrub shells with a stiff brush to remove grit. Avoid soaking in fresh water for long periods (it can kill them).

\Subsection{How should clams be stored?}
Refrigerate in a breathable container (mesh bag or bowl covered with a damp towel). Do not seal in airtight containers or submerge in water. Cook within a day if possible.

\Subsection{Can I make cioppino ahead of time?}
Make the broth in advance, then add clams and shrimp (and any additional shellfish) just before serving. Bake fish separately and add to bowls when serving to keep it from overcooking.

\vfill
\begin{center}
  \footnotesize
  \textit{Serve with: Garlic bread, focaccia, or a baguette for soaking up the broth.}
\end{center}

\end{document}

