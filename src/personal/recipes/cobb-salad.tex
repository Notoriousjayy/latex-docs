\documentclass[11pt]{article}

% --- Packages ---
\usepackage[T1]{fontenc}
\usepackage[utf8]{inputenc}
\usepackage[a4paper,margin=1in]{geometry}
\usepackage{lmodern}
\usepackage{microtype}
\usepackage{enumitem}
\usepackage{hyperref}
\usepackage{xcolor}
\usepackage{booktabs}
\usepackage{array}

% --- Link setup ---
\hypersetup{
  colorlinks=true,
  linkcolor=blue!60!black,
  urlcolor=blue!60!black,
  citecolor=blue!60!black,
  pdftitle={Cobb Salad — Recipe},
  pdfauthor={},
  pdfsubject={Recipe},
  pdfcreator={LaTeX}
}

% --- Formatting tweaks ---
\setlist[itemize]{itemsep=2pt, topsep=4pt, leftmargin=1.2em}
\setlist[enumerate]{itemsep=6pt, topsep=6pt, leftmargin=1.4em}

\begin{document}

\begin{center}
  {\LARGE \textbf{Cobb Salad}}\\[4pt]
  \normalsize A polished, printable recipe.
\end{center}

\vspace{0.5em}

\noindent
\begin{tabular}{@{}>{\bfseries}p{2.2cm}p{12cm}@{}}
Total Time: & 50 minutes \\
Yield: & 4 servings \\
\end{tabular}

\vspace{1em}

\section*{Ingredients}

\subsection*{For the Dressing}
\begin{itemize}
  \item 1 small shallot, thinly sliced into rings
  \item 3 tablespoons red-wine vinegar
  \item Kosher salt and freshly ground black pepper
  \item 1 tablespoon whole-grain or Dijon mustard
  \item 3 tablespoons olive oil (plus more as needed)
\end{itemize}

\subsection*{For the Salad and Assembly}
\begin{itemize}
  \item 4 large eggs
  \item 10 ounces thick-cut bacon (about 8 strips)
  \item 12 ounces boneless, skinless chicken breast (about 2 medium breasts)
  \item 1 head romaine lettuce, torn into bite-sized pieces or coarsely chopped
  \item 6 ounces small to medium tomatoes (about 6), sliced or quartered
  \item 1 avocado, thinly sliced or chopped
  \item 4 ounces blue cheese, crumbled
  \item 3 tablespoons finely chopped chives
\end{itemize}

\section*{Preparation}

\begin{enumerate}
  \item \textbf{Make the dressing.} In a small bowl, cover the shallot rings with the red-wine vinegar and season with salt and pepper. Let sit for 5 minutes to lightly pickle the shallots and infuse the vinegar. Add the mustard and 3 tablespoons olive oil and, using a fork, whisk to blend. Taste and season with additional salt and pepper if needed.
  \item \textbf{Cook the eggs.} Bring a small pot of water to a boil. Gently lower in the 4 large eggs and boil for 8 minutes. Remove from heat and run cold water over the eggs to quickly chill (you can add a few ice cubes as well). Once chilled, peel the eggs and set aside until ready to assemble.
  \item \textbf{Cook the bacon.} Meanwhile, cook the bacon in a large skillet over medium heat until crispy on both sides, 8 to 10 minutes. Transfer to a paper towel-lined plate and let cool. Once cool enough to handle, coarsely chop and set aside.
  \item \textbf{Cook the chicken.} Drain all but roughly 2 tablespoons of bacon grease from the skillet (discard or reserve the excess). Season the chicken with salt and pepper and cook in the same skillet over medium-high heat until well browned on both sides and cooked through, 12 to 15 minutes. Transfer to a large plate or cutting board.
  \item \textbf{Finish the vinaigrette.} Transfer any drippings from the skillet (you should have at least 2 tablespoons) to the bowl with the vinaigrette and whisk to blend, adding more olive oil if desired.
  \item \textbf{Prep proteins.} Once the chicken is cool enough to handle, shred it into bite-sized pieces (or chop/slice). Slice or chop the hard-boiled eggs.
  \item \textbf{Dress the lettuce.} Arrange the romaine in your largest serving bowl or on a platter. Drizzle about half the dressing over the lettuce and toss to combine; season with salt and pepper.
  \item \textbf{Assemble and serve.} Arrange the chicken in the center of the bowl or platter in a straight line. Place the tomatoes on one side and the eggs on the other. Place the avocado next to the eggs and the blue cheese next to the tomatoes. Sprinkle the bacon in the center. Spoon the remaining dressing over the top and finish with the chopped chives.
\end{enumerate}

\vspace{0.5em}
\noindent\textit{Tip:} For jammy eggs, reduce the boil time to about 7 minutes; for firmer yolks, go to 9--10 minutes.

\end{document}
