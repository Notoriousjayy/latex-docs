\documentclass[11pt]{article}

% ---------- Common, stable packages ----------
\usepackage[margin=1in]{geometry}
\usepackage[T1]{fontenc}
\usepackage[utf8]{inputenc}
\usepackage{lmodern}
\usepackage{microtype}
\usepackage{hyperref}
\usepackage{booktabs}
\usepackage{array}
\usepackage{tabularx}
\usepackage{enumitem}

\hypersetup{
  colorlinks=true,
  linkcolor=black,
  urlcolor=blue
}

% ---------- Helpers ----------
\setlist[itemize]{itemsep=2pt, topsep=4pt}
\setlist[enumerate]{itemsep=4pt, topsep=6pt}
\newcommand{\Section}[1]{\vspace{0.6em}\noindent\textbf{\Large #1}\par\vspace{0.25em}}
\newcommand{\Subsection}[1]{\vspace{0.4em}\noindent\textbf{\large #1}\par\vspace{0.2em}}
\newcolumntype{L}{>{\raggedright\arraybackslash}X}

\begin{document}

\begin{center}
  {\LARGE \textbf{Seafood Gumbo}}\\[2pt]
  \small A Creole classic with a deep roux, andouille, shrimp, and crab simmered in rich seafood stock.
\end{center}

\Section{Overview}
Seafood gumbo is comfort in a bowl: a dark, nutty roux builds body; the ``holy trinity'' of onion, bell pepper, and celery lays the aromatic base; andouille adds smoke; shrimp and crab bring sweet brine. Serve generously over hot white rice.

\Section{At a Glance}
\begin{tabularx}{\textwidth}{@{} l L @{}}
\toprule
\textbf{Yield} & About 3 quarts (serves 10) \\
\textbf{Prep Time} & 30 minutes \\
\textbf{Cook Time} & 2 hours \\
\textbf{Total Time} & 2 hours 30 minutes \\
\textbf{Course} & Main Course \qquad \textbf{Cuisine} \, American, Cajun, Creole \\
\textbf{Estimated Calories} & \(\sim\)315 per serving (see note) \\
\bottomrule
\end{tabularx}

\Section{Ingredients}

\Subsection{Meats \& Seafood}
\begin{itemize}
  \item 1--2 lb andouille sausage, sliced and browned
  \item 1 lb Dungeness crab legs
  \item 1 lb shrimp, peeled and deveined
\end{itemize}

\Subsection{Roux \& Aromatics}
\begin{itemize}
  \item \(3/4\) cup vegetable oil (or butter)
  \item 1 cup all-purpose flour
  \item \(1\frac{1}{2}\) cups onion, chopped
  \item \(3/4\) cup red bell pepper, chopped
  \item \(3/4\) cup celery, chopped
  \item 4 cloves garlic, minced
  \item 1 cup green onion, sliced (divided; some for finish/garnish)
  \item Fresh parsley, chopped (for garnish)
\end{itemize}

\Subsection{Liquids}
\begin{itemize}
  \item 8 cups seafood stock
\end{itemize}

\Subsection{Seasonings}
\begin{itemize}
  \item 2 tsp hot sauce
  \item 2 bay leaves
  \item \(1/4\) tsp dried thyme
  \item 2 tsp salt, plus more to taste
  \item 1 tsp cayenne
  \item 1 tbsp Bayou City All Purpose Seasoning
  \item 1 tbsp Bayou City Garlic Pepper
  \item 1 tbsp gumbo fil\'e (ground sassafras)
\end{itemize}

\Subsection{To Serve}
\begin{itemize}
  \item Cooked white rice
\end{itemize}

\Section{Equipment}
8-quart heavy-bottomed stockpot or Dutch oven; wooden spoon or roux whisk; ladle; long-handled skimmer; heatproof spatula.

\Section{Method}

\Subsection{1) Brown the Andouille}
\begin{enumerate}
  \item In a large heavy 8-qt pot over medium heat, brown the andouille on both cut sides. Remove to a plate and reserve drippings in the pot.
\end{enumerate}

\Subsection{2) Make a Dark Roux}
\begin{enumerate}
  \item Add the vegetable oil to the pot (with drippings) over medium heat. Sprinkle in the flour and stir continuously with a wooden spoon, scraping the corners and bottom, until the roux turns the color of milk chocolate, 20--30 minutes. \emph{Do not stop stirring; adjust heat as needed to avoid scorching.}
\end{enumerate}

\Subsection{3) Build the Base}
\begin{enumerate}
  \item Stir in the onion, bell pepper, and celery (the ``holy trinity''); cook 5 minutes until softened.
  \item Add the garlic; cook 30 seconds until fragrant.
  \item Return the browned andouille to the pot; cook 5 minutes, stirring occasionally.
\end{enumerate}

\Subsection{4) Simmer the Gumbo}
\begin{enumerate}
  \item Pour in the seafood stock; bring to a boil.
  \item Reduce to medium-low; add hot sauce, bay leaves, thyme, salt, cayenne, Bayou City All Purpose Seasoning, and Bayou City Garlic Pepper. Simmer gently 45 minutes, skimming foam and excess oil from the surface as needed.
  \item Stir in the gumbo fil\'e; continue to simmer 15 minutes.
\end{enumerate}

\Subsection{5) Finish with Seafood \& Greens}
\begin{enumerate}
  \item Add most of the green onions (reserve a handful for garnish), the crab legs, and the shrimp. Cook 5--7 minutes, just until shrimp are pink and opaque.
  \item Taste and adjust seasoning with additional salt, cayenne, or hot sauce. Discard bay leaves.
\end{enumerate}

\Subsection{6) Serve}
\begin{enumerate}
  \item Spoon hot cooked white rice into shallow bowls, ladle gumbo over top, and garnish with reserved green onions and parsley.
\end{enumerate}

\Section{Notes \& Tips}
\begin{itemize}
  \item \textbf{Roux color:} The nutty depth of gumbo comes from a well-developed dark roux. Be patient and stir constantly; if any black specks appear or it smells acrid, discard and start over.
  \item \textbf{Fil\'e powder:} Adds body and a distinct herbal note. Avoid boiling vigorously after adding fil\'e to prevent stringiness.
  \item \textbf{Seafood timing:} Add shrimp near the end to prevent overcooking; adjust simmer time for larger shrimp.
  \item \textbf{Stock swap:} If seafood stock is unavailable, combine low-sodium chicken stock with a splash of clam juice for brine.
  \item \textbf{Heat control:} Tailor spice with cayenne and hot sauce to taste.
\end{itemize}

\Section{Best Sides}
Fluffy white rice (essential), buttery cornbread, sautéed greens (collards or spinach), or Southern-style potato salad.

\Section{Storage \& Make-Ahead}
\begin{itemize}
  \item \textbf{Refrigerate:} Cool completely; store airtight up to 3 days.
  \item \textbf{Reheat:} Warm gently on the stovetop over low heat to preserve texture. Add a splash of stock if needed.
  \item \textbf{Flavor improves:} Gumbo deepens in flavor on day 2; seafood should still be reheated gently.
\end{itemize}

\Section{FAQ}
\begin{itemize}
  \item \textbf{Can I use frozen shrimp?} Yes; thaw fully and drain before cooking.
  \item \textbf{Can I adjust the spice level?} Absolutely. Reduce or omit cayenne/hot sauce for milder gumbo; add more to taste for heat.
  \item \textbf{Make ahead for a party?} Yes. Prepare gumbo base (through Step 4) a day ahead; chill. Reheat, then add fil\'e, seafood, and green onions just before serving.
\end{itemize}

\vfill
\begin{center}
  \footnotesize
  \textit{Nutrition note: 315 kcal per serving is an estimate and will vary with brands, sausage fat content, and serving size.}
\end{center}

\end{document}

