\documentclass[11pt]{article}
\usepackage{amsmath}

% ---------- Common, stable packages ----------
\usepackage[margin=1in]{geometry}
\usepackage[T1]{fontenc}
\usepackage[utf8]{inputenc}
\usepackage{lmodern}
\usepackage{microtype}
\usepackage{hyperref}
\usepackage{booktabs}
\usepackage{array}
\usepackage{tabularx}
\usepackage{enumitem}

\hypersetup{
  colorlinks=true,
  linkcolor=black,
  urlcolor=blue
}

% ---------- Helpers ----------
\setlist[itemize]{itemsep=2pt, topsep=4pt}
\setlist[enumerate]{itemsep=4pt, topsep=6pt}
\newcommand{\Section}[1]{\vspace{0.6em}\noindent\textbf{\Large #1}\par\vspace{0.25em}}
\newcommand{\Subsection}[1]{\vspace{0.4em}\noindent\textbf{\large #1}\par\vspace{0.2em}}
\newcolumntype{L}{>{\raggedright\arraybackslash}X}

\begin{document}

\begin{center}
  {\LARGE \textbf{Shrimp Po' Boys}}\\[4pt]
  \small Crispy fried shrimp piled into toasted garlic-buttered rolls with a spicy, tangy r\'emoulade.
\end{center}

\Section{Overview}
A New Orleans favorite made weeknight-easy: toast split French rolls with garlic butter, whisk a quick r\'emoulade, fry panko-crusted shrimp until golden, then build sandwiches with plenty of shredded lettuce. Keep the shrimp hot and the bread crisp by assembling just before serving.

\Section{At a Glance}
\begin{tabularx}{\textwidth}{@{} l L @{}}
\toprule
\textbf{Yield} & 4 sandwiches (serves 4) \\
\textbf{Prep Time} & 35 minutes \\
\textbf{Cook Time} & 10 minutes \\
\textbf{Total Time} & 45 minutes \\
\textbf{Key Temps} & Toast at 350\,$^\circ$F (175\,$^\circ$C); fry at 360\,$^\circ$F (182\,$^\circ$C) \\
\bottomrule
\end{tabularx}

\Section{Ingredients}

\Subsection{R\'emoulade Sauce}
\begin{itemize}
  \item $\tfrac{1}{2}$ cup mayonnaise
  \item 2 tablespoons ponzu (lime) \,\footnotesize(e.g., Kikkoman Ponzu Lime)\normalsize
  \item 1 tablespoon horseradish
  \item 1 teaspoon pickle relish
  \item 1 teaspoon minced garlic
  \item $\tfrac{1}{2}$ teaspoon cayenne pepper
\end{itemize}

\Subsection{Sandwiches}
\begin{itemize}
  \item 4 tablespoons melted butter
  \item 1 teaspoon minced garlic
  \item 4 French rolls, split and hinged
  \item Neutral oil for frying (about 2 cups / 480\,ml), as needed
  \item $\tfrac{3}{4}$ cup all-purpose flour
  \item 2 tablespoons Creole seasoning
  \item 3 large eggs, beaten
  \item 2 cups panko breadcrumbs
  \item 2 pounds (900\,g) jumbo shrimp, peeled and deveined
  \item 2 cups shredded lettuce
\end{itemize}

\Section{Equipment}
Rimmed baking sheet; pastry brush; three shallow bowls; large heavy saucepan or deep fryer; thermometer (clip-on or instant-read); wire rack or paper towels; tongs; spider or slotted spoon.

\Section{Instructions}
\begin{enumerate}
  \item \textbf{Heat oven.} Preheat to 350\,$^\circ$F (175\,$^\circ$C). Line a baking sheet.
  \item \textbf{Make the r\'emoulade.} In a small bowl whisk together mayonnaise, ponzu, horseradish, relish, garlic, and cayenne. Cover and refrigerate.
  \item \textbf{Toast the rolls.} Stir melted butter with 1~teaspoon minced garlic. Place rolls cut-side up on the sheet and brush interiors with garlic butter. Toast 1--3 minutes until lightly golden. Set aside.
  \item \textbf{Heat the oil.} In a deep fryer or heavy pot, heat oil to 360\,$^\circ$F (182\,$^\circ$C). Maintain this temperature between batches.
  \item \textbf{Set up dredging station.} Bowl~1: flour mixed with Creole seasoning. Bowl~2: beaten eggs. Bowl~3: panko.
  \item \textbf{Bread the shrimp.} Pat shrimp dry. Dredge in seasoned flour (shake off excess), dip in egg (let excess drip), then press into panko to coat. Arrange breaded shrimp in a single layer on a plate.
  \item \textbf{Fry.} Working in batches, fry shrimp until coating is deep golden and meat is opaque, about 2 minutes per side. Transfer to a rack or paper towels to drain.
  \item \textbf{Assemble.} Spread r\'emoulade on toasted rolls. Mound hot fried shrimp inside and top with shredded lettuce. Serve immediately.
\end{enumerate}

\Section{Notes \& Tips}
\begin{itemize}
  \item \textbf{Keep it crisp:} Fry in small batches and let the oil return to 360\,$^\circ$F before the next. Drain on a wire rack, not just paper towels.
  \item \textbf{Rolls:} New Orleans--style French bread has a thin, crackly crust and tender crumb; any light French rolls work.
  \item \textbf{Heat level:} Adjust cayenne in the sauce to taste; add hot sauce at the table if desired.
  \item \textbf{Sauce too thin?} Whisk in a little more mayonnaise to thicken.
  \item \textbf{Add-ons (optional):} Sliced tomato, dill pickle chips, or thinly sliced red onion.
\end{itemize}

\vfill
\begin{center}
  \footnotesize
  \textit{Safety:} Hot oil is hazardous—use a deep pan, avoid overcrowding, and keep children away from the stove.\\
  \textit{Serving idea:} Pair with kettle chips, fries, or a simple slaw.
\end{center}

\end{document}
