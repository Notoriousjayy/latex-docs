\documentclass[11pt,letterpaper]{article}

% --- Page + typography ---
\usepackage[letterpaper,margin=1in]{geometry}
\usepackage{lmodern}
\usepackage[T1]{fontenc}
\usepackage[utf8]{inputenc}
\usepackage{microtype}
\usepackage{setspace}
\usepackage{parskip}

% --- Color, boxes, links ---
\usepackage[dvipsnames,table]{xcolor}
\definecolor{Peach}{RGB}{255,218,185} % ensure Peach exists on older setups
\usepackage[most]{tcolorbox}          % <-- important for 'enhanced'
\tcbset{enhanced,boxrule=0.5pt,arc=2mm}

\usepackage{hyperref}
\hypersetup{
  colorlinks=true,
  linkcolor=MidnightBlue,
  urlcolor=MidnightBlue,
  citecolor=MidnightBlue
}

% --- Lists & formatting ---
\usepackage{enumitem}
\setlist{itemsep=2pt,topsep=6pt,leftmargin=1.2em}

% --- Symbols (for \textdegree) ---
\usepackage{textcomp}

% --- Title meta helpers ---
\newcommand{\RecipeTitle}{Ultimate Peanut Butter Cheesecake}
\newcommand{\RecipeAuthor}{Tessa Arias}
\newcommand{\RecipeDate}{October 9, 2023}
\newcommand{\RecipeComments}{43 Comments}

\begin{document}

\begin{center}
  {\LARGE\bfseries \RecipeTitle}\\[4pt]
  {\large by \RecipeAuthor}\\[2pt]
  {\small Published: \RecipeDate \quad\textbullet\quad \RecipeComments}
\end{center}

\vspace{0.5em}

\begin{tcolorbox}[colback=Gray!4,colframe=Gray!40,title={Recipe Snapshot}]
\begin{minipage}[t]{0.48\linewidth}
\textbf{Taste}:\; Loaded with nutty peanut flavor balancing cheesecake's sweet tang.\\
\textbf{Texture}:\; Crunchy crust, velvety filling, rich fudgy topping.\\
\textbf{Ease}:\; Weekend baking project --- very doable with the tips below.\\
\textbf{Pros}:\; Perfect for peanut butter lovers.\\
\textbf{Cons}:\; None noted.
\end{minipage}\hfill
\begin{minipage}[t]{0.48\linewidth}
\textbf{Great for}: Family dinners, Friendsgiving, holiday parties.\\
\textbf{Method}: Baked cheesecake with water bath.\\
\textbf{Components}: Peanut--graham crust; peanut butter cheesecake filling; peanut butter ganache topping.\\
\end{minipage}
\end{tcolorbox}

\section*{Overview}
This is truly the \emph{Ultimate Peanut Butter Cheesecake}: peanuts in the crust, peanut butter in the filling, and a luscious peanut butter ganache topping for absolute nutty decadence.

\begin{tcolorbox}[colback=Peach!10,colframe=Peach!60!black,title={Reader Love}]
\emph{``Tessa, I am to the point where I just won't make a cheesecake recipe that isn't yours. This peanut butter cheesecake is a dream\ldots{} total perfection!! Thank you!''} --- Mary M
\end{tcolorbox}

\section*{Ingredients Overview}
\textbf{Crust}
\begin{itemize}
  \item Graham cracker and peanut crumb base (crunchy foundation).
  \item Melted butter to bind.
\end{itemize}

\textbf{Filling}
\begin{itemize}
  \item High--quality, full--fat brick cream cheese (fully softened).
  \item Granulated sugar.
  \item Sour cream (room temperature).
  \item Conventional creamy peanut butter (see note below).
  \item Eggs (room temperature).
  \item Vanilla extract.
\end{itemize}

\textbf{Topping}
\begin{itemize}
  \item Peanut butter ganache (peanut butter gently combined with hot cream; prepare to desired consistency).
\end{itemize}

\begin{tcolorbox}[colback=Yellow!8,colframe=Yellow!40!black,title={Sprinkle of Science: Peanut Butter Choice}]
Use \textbf{conventional creamy peanut butter}. It typically contains a stabilizing oil that keeps it emulsified. Natural peanut butter can separate; if using it, mix \emph{extremely well} so there are no dry bits or oily pockets.
\end{tcolorbox}

\section*{Equipment}
Springform pan; roasting pan (for the water bath); wide heavy--duty aluminum foil (wrap pan at least 3 times) or an oven bag; electric mixer (stand or hand); kettle for boiling water; instant--read thermometer; thin knife; metal pie server; sharp chef's knife.

\section*{Method Overview}
\textbf{1) Prepare the pan and crust}
\begin{itemize}
  \item Wrap the springform pan base and sides in wide heavy--duty foil (3 layers) to protect from the water bath.
  \item Press the graham--peanut crust firmly into the pan (base; sides optional). Chill while making the filling.
\end{itemize}

\textbf{2) Make the filling}
\begin{itemize}
  \item Beat softened cream cheese \textbf{thoroughly} until completely smooth before adding anything else.
  \item Add sugar, sour cream, and peanut butter; beat smooth, scraping bowl and beater \emph{often}.
  \item Add eggs \textbf{last} and mix just until combined (avoid overbeating at this stage).
\end{itemize}

\textbf{3) Set up the water bath}
\begin{itemize}
  \item Place the foil--wrapped springform pan into a larger roasting pan on the oven rack.
  \item Carefully pour boiling water into the roasting pan until it reaches halfway up the sides of the springform pan.
\end{itemize}

\textbf{4) Bake gently}
\begin{itemize}
  \item Bake until the top looks set/dry but the \emph{center still jiggles gently}.
  \item Internal temperature at the center should read about \textbf{150\textdegree F}.
\end{itemize}

\textbf{5) Cool slowly}
\begin{itemize}
  \item Turn the oven off; crack the door and let the cheesecake cool gradually in the warm oven.
  \item Immediately after removing from the oven, run a thin knife around the edge to help prevent cracks as it cools.
  \item Cool to room temperature, then chill completely in the refrigerator until fully set.
\end{itemize}

\textbf{6) Finish with ganache \& slice cleanly}
\begin{itemize}
  \item Pour the peanut butter ganache over the chilled cheesecake; let it set.
  \item For clean slices: unlatch and remove the springform ring, warm a sharp knife under hot water, wipe dry, and clean between cuts. Slide a metal server under the crust to lift slices without crumbling.
\end{itemize}

\section*{Why Use a Water Bath?}
A water bath provides a gentle, even heat that:
\begin{itemize}
  \item Helps prevent cracks and sunken centers.
  \item Ensures an ultra--smooth, even texture.
  \item Keeps edges from overbaking before the center is done.
\end{itemize}

\section*{Preventing Cracks: Three Common Causes}
\textbf{Overbeating}\; Beat cream cheese, sugar, sour cream, and peanut butter very well, scraping often. Once eggs are added, \textbf{mix only until combined}.\\[4pt]
\textbf{Overcooking}\; Residual heat continues cooking after the oven is off. Stop when the center still wobbles and the top looks dry.\\[4pt]
\textbf{Rapid temperature change}\; Avoid opening the oven frequently. Cool gradually in the turned--off oven with the door cracked. Run a thin knife around the edge right after baking.

\section*{Doneness Cues}
Stop baking when the cheesecake is dry on top but still gently wobbly in the very center. The center should register about \textbf{150\textdegree F}. It will finish setting as it cools and chills.

\section*{Serving Tips}
If time allows, rest at room temperature for about 30 minutes before serving for best texture and flavor.

\section*{Storage}
Loosely cover and refrigerate for up to \textbf{5 days}. Note: the crust will gradually lose some crunch over time.

\section*{Freezing}
Cheesecake freezes beautifully:
\begin{enumerate}
  \item Freeze the whole cake or individual slices on a baking sheet until firm.
  \item Wrap in plastic wrap and place in a freezer bag; freeze up to \textbf{2 months}.
  \item Thaw a whole cake overnight in the fridge. Individual slices can thaw overnight in the fridge or at room temperature for about 30 minutes.
\end{enumerate}

\vfill
\begin{center}
  {\small\color{Gray}Add exact quantities, oven temperature, and timings where you store those details.}
\end{center}

\end{document}
