\documentclass[11pt]{article}

% ---------- Common, stable packages ----------
\usepackage[margin=1in]{geometry}
\usepackage[T1]{fontenc}
\usepackage[utf8]{inputenc}
\usepackage{lmodern}
\usepackage{microtype}
\usepackage{hyperref}
\usepackage{booktabs}
\usepackage{array}
\usepackage{tabularx}
\usepackage{enumitem}
\usepackage{amsmath}

\hypersetup{
  colorlinks=true,
  linkcolor=black,
  urlcolor=blue
}

% ---------- Helpers ----------
\setlist[itemize]{itemsep=2pt, topsep=4pt}
\setlist[enumerate]{itemsep=4pt, topsep=6pt}
\newcommand{\Section}[1]{\vspace{0.6em}\noindent\textbf{\Large #1}\par\vspace{0.25em}}
\newcommand{\Subsection}[1]{\vspace{0.4em}\noindent\textbf{\large #1}\par\vspace{0.2em}}
\newcolumntype{L}{>{\raggedright\arraybackslash}X}

\begin{document}

\begin{center}
  {\LARGE \textbf{Chicken Curry Recipe}}\\[4pt]
  \small By Swasthi \quad\textbullet\quad Updated: September 14, 2023
\end{center}

\Section{Overview}
A classic Indian chicken curry of tender pieces simmered in a bright onion--tomato gravy with warm spices. No stock, cornstarch, or canned puree required. Serve with steamed rice (plain, turmeric, or jeera) or with naan/chapati.

\Section{At a Glance}
\begin{tabularx}{\textwidth}{@{} l L @{}}
\toprule
\textbf{Yield} & 3 servings \\
\textbf{Prep Time} & 15 minutes \\
\textbf{Cook Time} & 35 minutes \\
\textbf{Total Time} & 50 minutes \\
\textbf{Heat Level} & Adjustable (use less/more chili to taste) \\
\bottomrule
\end{tabularx}

\Section{Ingredients (US cup = 240 ml)}

\Subsection{Chicken \& Liquids}
\begin{itemize}
  \item $\tfrac{1}{2}$ kg (1.1 lb) chicken, preferably bone-in (boneless ok)
  \item $\tfrac{1}{2}$ to 1 cup hot water \emph{(or light coconut milk)}%
  \footnote{Using hot water helps keep the chicken tender.}
  \item $\tfrac{1}{4}$ cup yogurt \emph{(or 1.5 tbsp cashew butter, or 12 cashews soaked and blended; see Notes)}
\end{itemize}

\Subsection{Aromatics \& Base}
\begin{itemize}
  \item 2 to 3 tablespoons oil
  \item 1 cup (about 3 medium) onions, very finely chopped
  \item 1 to 2 green chilies, slit (omit for less heat)
  \item 1 tablespoon ginger--garlic paste \emph{(or 3 cloves garlic + 3/4 inch ginger, minced)}
  \item $\tfrac{1}{2}$ cup (about 2 medium) tomatoes, finely chopped or pureed
  \item 2 tablespoons coriander (cilantro) leaves or mint, finely chopped (plus more to garnish)
\end{itemize}

\Subsection{Spice Powders}
\begin{itemize}
  \item $\tfrac{1}{4}$ teaspoon turmeric powder
  \item 1 teaspoon Kashmiri red chili powder or smoked paprika (use $\tfrac{1}{2}$ tsp for less spicy)
  \item 1 teaspoon garam masala (plus up to $\tfrac{1}{2}$ tsp more at the end, to taste)
  \item 1 teaspoon coriander powder
\end{itemize}

\Subsection{Whole Spices (optional but recommended)}
\begin{itemize}
  \item 1 bay leaf \emph{(or 1 sprig curry leaves)}
  \item 4 cloves
  \item 2-inch cinnamon piece
  \item 3 green cardamom pods
\end{itemize}

\Section{Equipment}
Heavy pan or Dutch oven with lid; wooden spoon; ladle; optional blender.

\Section{Instructions}

\Subsection{Preparation}
\begin{enumerate}
  \item \textbf{Bloom whole spices.} Heat oil over medium in a deep pan. Add bay leaf, cinnamon, cloves, and cardamoms; sauté a few seconds until fragrant.
  \item \textbf{Sweat onions.} Add finely chopped onions and slit green chilies; sauté until deep golden, 7--8 minutes.
  \item \textbf{Add aromatics.} Stir in ginger--garlic paste; cook about 1 minute until the raw smell disappears.
  \item \textbf{Tomatoes \& spices.} Add tomatoes, turmeric, and salt; cook until completely soft and mushy. Reduce heat to low; stir in yogurt (or cashew paste), red chili powder, garam masala, and coriander powder. Cook gently until the masala is thick and aromatic.
\end{enumerate}

\Subsection{How to Make the Curry}
\begin{enumerate}
  \item \textbf{Add chicken.} Stir in chicken and chopped coriander/mint. Fry on medium until chicken turns pale, about 3 minutes.
  \item \textbf{Brief rest.} Cover and cook on low 3--4 minutes so the meat absorbs the masala.
  \item \textbf{Add hot water.} Pour in just enough hot water (about $\tfrac{1}{2}$ cup) to partially cover the chicken. Avoid cold water.
  \item \textbf{Simmer.} Cover and cook over medium heat until chicken is soft and cooked through and the gravy thickens. Timing varies by size/age of chicken; cook gently rather than boiling hard.
  \item \textbf{Finish.} Check salt and adjust. If desired, add up to $\tfrac{1}{2}$ teaspoon more garam masala. Garnish with coriander, cover, and rest off heat a couple of minutes.
  \item \textbf{Serve.} With plain rice, turmeric rice, jeera rice, ghee rice, naan, or chapati. Add raita and onion salad on the side if you like.
\end{enumerate}

\Section{Pro Tips}
\begin{itemize}
  \item \textbf{Fine-chop onions:} Small pieces cook faster and become jammy, enriching the gravy.
  \item \textbf{Tomatoes:} Fresh chopped or pureed both work. Canned tomatoes or passata require slightly more spice.
  \item \textbf{Low \& slow:} Gentle simmering yields the best flavor and tender meat.
\end{itemize}

\Section{Ingredients \& Substitutes}
\begin{itemize}
  \item \textbf{Chicken:} Bone-in pieces develop the best body in the gravy; boneless also works.
  \item \textbf{Whole spices:} Optional but deepen aroma; use when available.
  \item \textbf{Garam masala:} Quality varies; adjust to taste. Sub with meat masala, biryani masala, kitchen king masala, or curry powder (note aroma/heat differ).
  \item \textbf{Natural thickeners:} Yogurt, coconut milk, or nut/poppy seed paste enhance body without flour or starch.
\end{itemize}

\Section{Using Yogurt in Curry (to avoid splitting)}
\begin{itemize}
  \item Whisk the yogurt smooth. Temper it with 2 tablespoons of hot onion--tomato masala, then add over low heat.
  \item Use yogurt with low whey (Greek or strained). If runny, strain before use.
\end{itemize}

\Section{Variations}
\begin{itemize}
  \item \textbf{Smooth gravy:} Cool the masala and blend with about 1 cup water until smooth; strain back if desired.
  \item \textbf{Creamy finish:} Stir in 3--4 tablespoons heavy cream or thick coconut milk at the end (off heat).
  \item \textbf{Coconut-forward:} Replace some or all water with warm light coconut milk (do not add cold).
\end{itemize}

\Section{Notes}
\begin{itemize}
  \item Whole spices may be omitted, but they add depth.
  \item Use hot water when extending the gravy; cold water can toughen chicken.
  \item If you dislike onion/tomato texture, blend the masala with $\tfrac{1}{2}$ cup water until smooth.
  \item Choose a good garam masala; some brands are very pungent, so add gradually.
\end{itemize}

\Section{Nutrition (Approximate, per serving; 3 servings)}
\begin{tabular}{@{}ll@{}}
\toprule
Calories & 417 \\
Fat & 26 g \quad (Saturated 7 g) \\
Cholesterol & 126 mg \\
Sodium & 143 mg \\
Potassium & 521 mg \\
Carbohydrates & 10 g \quad (Fiber 3 g; Sugar 3 g) \\
Protein & 33 g \\
Vitamin A & 705 IU \\
Vitamin C & 10.4 mg \\
Calcium & 65 mg \\
Iron & 2.7 mg \\
\bottomrule
\end{tabular}

\vfill
\begin{center}
  \footnotesize
  \textit{Serving suggestion: garnish with fresh coriander; pair with jeera rice or butter naan.}
\end{center}

\end{document}

