\documentclass[11pt]{article}

% -----------------------------
% Packages
% -----------------------------
\usepackage[T1]{fontenc}
\usepackage[utf8]{inputenc}
\usepackage{lmodern}
\usepackage{microtype}
\usepackage[margin=1in]{geometry}
\usepackage{enumitem}
\usepackage{xcolor}
\usepackage{hyperref}
\usepackage{bookmark}
\usepackage{titlesec}

% -----------------------------
% Hyperref setup
% -----------------------------
\hypersetup{
  colorlinks=true,
  linkcolor=blue,
  urlcolor=blue,
  citecolor=blue,
  pdftitle={Optimizing Fiber Intake with Whole Grains and Seeds},
  pdfauthor={},
  pdfsubject={Nutrition},
  pdfkeywords={fiber, whole grains, oats, barley, chia, flax}
}

% -----------------------------
% Formatting
% -----------------------------
\setlist[itemize]{leftmargin=*, itemsep=2pt, topsep=4pt}
\setlist[enumerate]{leftmargin=*, itemsep=2pt, topsep=4pt}
\titleformat{\section}{\Large\bfseries}{\thesection}{0.6em}{}
\titleformat{\subsection}{\large\bfseries}{\thesubsection}{0.6em}{}

% -----------------------------
% Document
% -----------------------------
\title{Optimizing Fiber and Health with Whole Grains and Seeds}
\author{}
\date{}

\begin{document}
\maketitle
\tableofcontents
\clearpage

\section{Overview}
For optimal fiber intake and overall health, prioritize \textbf{diverse whole grains}---such as oats, barley, quinoa, brown rice, bulgur, and whole wheat---and complement them with \textbf{fiber-dense seeds} like chia or flax. This approach improves the balance of:
\begin{itemize}
  \item \textbf{Soluble fiber} (supports cholesterol reduction and promotes fullness), and
  \item \textbf{Insoluble fiber} (supports bowel regularity and healthy digestion).
\end{itemize}
In practice, this blend typically improves satiety, digestion, and cardiometabolic health compared with refined grains. Aim for \textbf{at least half} of grain intake to come from whole grains, and rotate several grain types across meals.

\section{Top Fiber-Rich Grains and Seeds}
\subsection{Core whole grains}
\begin{itemize}
  \item \textbf{Oats and barley}: Notable sources of \emph{beta-glucan} (soluble fiber) associated with cholesterol-lowering benefits.
  \item \textbf{Bulgur}: Cracked wheat; generally high in fiber and quick to prepare.
  \item \textbf{Quinoa}: A nutrient-dense grain-like seed; provides fiber and is commonly described as a ``complete'' protein source.
  \item \textbf{Brown rice and wild rice}: Whole-grain options with more intact nutrients than white rice.
  \item \textbf{Rye}: Commonly associated with increased fullness and potential blood sugar benefits.
  \item \textbf{Ancient grains}: Amaranth, millet, teff, and sorghum are nutrient-dense options that diversify dietary fiber sources.
\end{itemize}

\subsection{Seeds}
\begin{itemize}
  \item \textbf{Chia seeds and flaxseeds}: Very high in fiber; often highlighted for omega-3 fat content. Sprinkle into meals to increase fiber density.
\end{itemize}

\section{How to Combine for Best Results}
\begin{enumerate}
  \item \textbf{Breakfast:} Oatmeal topped with fruit (e.g., berries) plus a sprinkle of chia or ground flax.
  \item \textbf{Baking:} Replace roughly half of refined flour with whole-wheat flour or other whole-grain blends in breads, muffins, and cookies.
  \item \textbf{Meals:} Use brown rice, quinoa, farro, or barley as sides instead of white rice or refined grains.
  \item \textbf{Snacks:} Choose whole-grain crackers with nut butter, or popcorn.
  \item \textbf{Add-ins:} Mix cooked grains into salads, soups, or yogurt for added texture and fiber.
\end{enumerate}

\section{Key Benefits}
\begin{itemize}
  \item \textbf{Digestive health:} Insoluble fiber adds bulk and can help prevent constipation.
  \item \textbf{Heart health:} Soluble fiber (notably from oats and barley) can help lower cholesterol.
  \item \textbf{Blood sugar control:} Fiber slows carbohydrate absorption, helping reduce post-meal glucose spikes.
  \item \textbf{Weight management:} Fiber increases fullness, which can reduce overall calorie intake.
\end{itemize}

\section{References}
\begin{enumerate}
  \item WebMD: High-fiber foods. \url{https://www.webmd.com/diet/ss/slideshow-high-fiber-foods}
  \item Mayo Clinic: Dietary fiber. \url{https://www.mayoclinic.org/healthy-lifestyle/nutrition-and-healthy-eating/in-depth/fiber/art-20043983}
  \item Whole Grains Council: ``What's the healthiest whole grain?'' \url{https://wholegrainscouncil.org/blog/2025/12/whats-healthiest-whole-grain}
  \item Mayo Clinic: Whole grains. \url{https://www.mayoclinic.org/healthy-lifestyle/nutrition-and-healthy-eating/in-depth/whole-grains/art-20047826}
  \item EatingWell: Healthy whole grains for more fiber. \url{https://www.eatingwell.com/article/8031247/healthy-whole-grains-to-eat-more-fiber/}
  \item UF Health: High-fiber foods. \url{https://ufhealth.org/care-sheets/high-fiber-foods}
  \item King Arthur Baking: Super 10 Blend. \url{https://shop.kingarthurbaking.com/items/super-10-blend}
  \item Johns Hopkins Medicine: 9 whole grains to try. \url{https://www.hopkinsmedicine.org/health/wellness-and-prevention/barley-farro-sorghum-and-more-9-whole-grains-to-try}
  \item YouTube: Alternative grains (video). \url{https://www.youtube.com/watch?v=rT1q3HmRpSU}
  \item BBC Good Food: Alternative grains. \url{https://www.bbcgoodfood.com/health/nutrition/alternative-grains}
  \item UCSF Health: Increasing fiber intake. \url{https://www.ucsfhealth.org/education/increasing-fiber-intake}
  \item Nature's Garden: Blood sugar guide (dried fruit). \url{https://naturesgarden.net/blogs/tips/blood-sugar-dried-fruit-guide}
\end{enumerate}

\end{document}
