% !TeX program = lualatex
\documentclass[11pt]{article}

\usepackage[margin=1in]{geometry}
\usepackage{amsmath}
\usepackage{microtype}
\usepackage[hidelinks]{hyperref}
\usepackage{enumitem}
\usepackage{booktabs}
\usepackage{xcolor}

\setlist[itemize]{leftmargin=*, itemsep=0.35em, topsep=0.35em}
\setlist[enumerate]{leftmargin=*, itemsep=0.35em, topsep=0.35em}

\title{\textbf{Tools to Measure Direct Sunlight}\\
\large A practical guide for gardeners and professionals}
\author{%
  % Your name / team
}
\date{\today}

\begin{document}
\maketitle

\begin{abstract}
Measuring direct sunlight can be as simple as categorizing a spot as \emph{full sun}, \emph{partial sun}, or \emph{shade}, or as rigorous as quantifying solar irradiance in watts per square meter (W/m\textsuperscript{2}) for engineering and research. This guide summarizes common instruments and methods, what they measure, and how to select the right tool for plant placement, horticultural optimization (including DLI), or professional-grade solar measurement.
\end{abstract}

\section{What ``Direct Sunlight'' Measurement Usually Means}
In practice, people measure sunlight in one (or more) of these ways:
\begin{itemize}
  \item \textbf{Categorical exposure:} full sun / partial sun / shade across a day (sufficient for many home gardens).
  \item \textbf{Visible light intensity:} illuminance in \textbf{lux} or \textbf{foot-candles} (quick spot checks; not plant-spectrum aware).
  \item \textbf{Plant-relevant daily light:} \textbf{Daily Light Integral (DLI)} in \textbf{mol/m\textsuperscript{2}/day} (best for horticultural precision).
  \item \textbf{Solar radiation (energy):} irradiance in \textbf{W/m\textsuperscript{2}} for solar energy, meteorology, or climate studies.
  \item \textbf{Spectral measurements:} wavelength-specific intensity (research/controlled-environment agriculture).
\end{itemize}

\section{Plant-Light Fundamentals: PAR, PPFD, and DLI}
For growers, the most actionable measurements are based on \textbf{photosynthetically active radiation (PAR)} and its time integral:
\begin{itemize}
  \item \textbf{PAR range:} typically 400--700 nm (the traditional PAR band).
  \item \textbf{PPFD:} \emph{Photosynthetic Photon Flux Density} --- instantaneous photon flux in \(\mu\)mol m\(^{-2}\) s\(^{-1}\), measured by a \emph{quantum sensor}.
  \item \textbf{DLI:} \emph{Daily Light Integral} --- total photons delivered over 24 hours in mol m\(^{-2}\) day\(^{-1}\).
\end{itemize}

When PPFD is constant, DLI can be computed from photoperiod:
\[
\mathrm{DLI}\;(\mathrm{mol}\,\mathrm{m}^{-2}\,\mathrm{day}^{-1}) =
\frac{\mathrm{PPFD}\;(\mu\mathrm{mol}\,\mathrm{m}^{-2}\,\mathrm{s}^{-1}) \times \text{seconds of light per day}}{10^{6}}.
\]
Under \emph{variable} sunlight (clouds, shading, changing angles), a meter/logger that \textbf{samples and integrates} PPFD over time is the most reliable way to obtain DLI. Apogee notes that integrating capability is necessary for variable lighting and that certain MQ-series meters can log DLI automatically when deployed in logging mode \cite{apogee_dli_measuring,apogee_mq500_pdf}.

\subsection{ePAR (Extended PAR) and far-red considerations}
Some modern fixtures (and sunlight) include meaningful energy beyond 700 nm. Apogee's ePAR sensors are designed to measure an extended 400--750 nm band and were introduced to address emerging use cases where photons beyond 700 nm can be relevant to growth outcomes \cite{apogee_pq612}.

\section{For Gardeners and Home Use (Ease of Use)}
\subsection{Sunlight calculators / accumulated-light devices}
These devices are intended for basic gardening decisions (e.g., where to place a vegetable bed or container).
\begin{itemize}
  \item \textbf{Example:} Luster Leaf SunCalc (and similar products). Place it at the location for a full day; it records accumulated light over a window (often up to \(\sim\)12 hours) and reports categories such as full sun, partial sun, or shade.
  \item \textbf{Best for:} Plant placement and quick validation of site exposure.
  \item \textbf{Limitations:} Provides categorization rather than quantitative PAR/PPFD or DLI.
\end{itemize}

\subsection{Light meters (photometers)}
Handheld meters provide instantaneous readings of \textbf{illuminance} in lux or foot-candles.
\begin{itemize}
  \item \textbf{How to use:} Take measurements at representative times (morning, solar noon, late afternoon) across multiple days.
  \item \textbf{Best for:} Comparing spots (e.g., balcony corners, under trellis, near walls).
  \item \textbf{Limitations:} Lux is weighted to human vision, not plant photosynthesis; lux-to-PPFD conversions can be highly spectrum-dependent (especially for LEDs).
\end{itemize}

\subsection{Pen-and-paper ``shadow map'' method}
A low-tech approach for visualizing sun/shade patterns across time.
\begin{itemize}
  \item \textbf{Method:} Draw the area (bed, balcony, yard). At several times of day, trace shadow lines and shaded regions. Repeat over multiple days/seasons if needed.
  \item \textbf{Best for:} Understanding obstructions (buildings, railings, trees) and microclimates.
  \item \textbf{Limitations:} Not quantitative; effort increases with desired accuracy.
\end{itemize}

\section{DLI / PAR Meter Recommendations (Grower-Focused)}
This section integrates practical options for measuring and managing \textbf{DLI} with an emphasis on horticultural use.

\subsection{Top-tier and professional instrumentation}
\begin{itemize}
  \item \textbf{Apogee MQ-500 (handheld quantum meter):} A full-spectrum quantum meter intended to measure PAR (400--700 nm) and provide accurate PPFD under sunlight and LED sources, with improved LED accuracy relative to older ``original'' sensors \cite{apogee_mq500_page}. Apogee also documents MQ-series capabilities for logging/integrating DLI when deployed in logging mode \cite{apogee_dli_measuring,apogee_mq500_pdf}.
  \item \textbf{Apogee microCache packages (sensor + logger):}
    \begin{itemize}
      \item \textbf{PQ-500 package (SQ-500 + microCache):} Bundles a full-spectrum quantum sensor with a Bluetooth micro-logger for monitoring PPFD and capturing time-series data suitable for DLI analysis \cite{apogee_pq500}.
      \item \textbf{PQ-612 package (SQ-610 ePAR + microCache):} Designed to measure the 400--750 nm ePAR band; positioned by Apogee as appropriate when extended wavelengths are relevant \cite{apogee_pq612}.
    \end{itemize}
  \item \textbf{LI-COR LI-1500 (professional logger):} A field/logger platform designed to log PAR from LI-COR quantum sensors and compute DLI via automated sampling configurations; LI-COR provides guidance for DLI calculation workflows using the LI-1500 and LI-190R \cite{licor_calculating_dli,licor_li1500_product}.
\end{itemize}

\subsection{Mid-range and budget-friendly approaches (useful estimates)}
\begin{itemize}
  \item \textbf{Lower-cost ``plant light'' approaches:} Many growers start with inexpensive lux meters or consumer ``sun meters'' to map relative brightness across locations; this is effective for comparative placement decisions, but remains an approximation for photosynthesis \cite{gardenerspath_best_light_meters}.
  \item \textbf{Phone apps:} Apps can provide quick, rough estimates (usually lux-based) and are best treated as screening tools; accuracy depends on device sensors, calibration, and spectral conditions \cite{photone_app_article}.
  \item \textbf{Photography light meters (e.g., Sekonic L-308X):} These are designed to measure exposure-related illuminance (lux/foot-candles) for imaging and video workflows, not PPFD; they can help compare brightness, but do not directly measure PAR or integrate DLI \cite{sekonic_l308x}.
\end{itemize}

\subsection{Key features to look for (DLI-centric)}
\begin{itemize}
  \item \textbf{True quantum/PAR sensing:} Sensor should measure photon flux in the PAR band (typically 400--700 nm) as PPFD.
  \item \textbf{Time integration / logging:} For DLI, prefer instruments that \emph{sample and integrate} over the day, particularly under variable sunlight \cite{apogee_dli_measuring,licor_calculating_dli}.
  \item \textbf{Spectral response (LEDs):} For LED-heavy environments, prioritize sensors designed for improved LED spectral response (e.g., ``full-spectrum'' quantum sensors) \cite{apogee_mq500_page}.
  \item \textbf{Workflow and data access:} Bluetooth/USB connectivity, software, and export formats become decisive for serious use (multi-day comparisons, seasonal tracking, audits) \cite{apogee_pq500,licor_li1500_product}.
  \item \textbf{ePAR readiness (optional):} If fixtures use far-red content and you want to quantify extended wavelengths, consider ePAR-capable instrumentation \cite{apogee_pq612}.
\end{itemize}

\section{For Scientific and Professional Use (Precision)}
\subsection{Pyranometer (global solar irradiance)}
A pyranometer measures \textbf{total solar radiation} incident on a plane surface, including direct sun and diffuse sky radiation, typically in \textbf{W/m\textsuperscript{2}} \cite{hukseflux_instrument_select,noaa_instruments}.
\begin{itemize}
  \item \textbf{Best for:} Solar resource assessment, meteorology, climate monitoring, and engineering analyses.
  \item \textbf{Notes:} Usually installed with leveling, proper siting, and data logging; calibration and standards matter.
\end{itemize}

\subsection{Pyrheliometer (direct beam solar irradiance)}
A pyrheliometer measures \textbf{direct beam} solar radiation (the sun's direct component), excluding diffuse sky radiation \cite{hukseflux_instrument_select}.
\begin{itemize}
  \item \textbf{Best for:} High-precision direct-normal irradiance measurements and research-grade solar characterization.
  \item \textbf{Notes:} Typically requires sun-tracking to stay aligned with the solar disc.
\end{itemize}

\subsection{Spectral instruments and PAR-focused research sensors}
For plant research and controlled environments, measuring light in specific wavelength bands can be required.
\begin{itemize}
  \item \textbf{PAR (Photosynthetically Active Radiation):} Often used in plant science, focusing on the spectral range relevant to photosynthesis.
  \item \textbf{Best for:} Research, greenhouse lighting design, advanced horticultural optimization.
\end{itemize}

\section{Quick Selection Guide}
\begin{table}[h!]
\centering
\begin{tabular}{@{}p{0.30\linewidth}p{0.34\linewidth}p{0.30\linewidth}@{}}
\toprule
\textbf{Use case} & \textbf{Recommended tool(s)} & \textbf{Primary output} \\
\midrule
Basic plant placement & SunCalc-style device; shadow map & Categorical: sun/partial/shade \\
Comparing multiple micro-spots & Lux/foot-candle meter; repeated readings & Lux / foot-candles \\
DLI for serious growers (portable) & Quantum PAR meter with DLI/logging (e.g., Apogee MQ-500) & PPFD; DLI (mol m$^{-2}$ day$^{-1}$) \\
DLI for multi-day monitoring & Sensor + logger (Apogee PQ packages); LI-COR LI-1500 setups & Time series PPFD; computed DLI \\
Solar energy / engineering studies & Pyranometer; pyrheliometer (direct beam) & W/m\textsuperscript{2} \\
\bottomrule
\end{tabular}
\end{table}

\section{How to Choose (Practical Recommendations)}
\begin{itemize}
  \item \textbf{If your goal is plant placement:} Start with an accumulated-light device (SunCalc-style) or a shadow map. These directly answer ``Is this spot truly full sun?'' without complexity.
  \item \textbf{If you want repeatable comparisons across your space:} Use a handheld lux meter and measure multiple times per day; treat results as relative comparisons rather than plant-physiology truth.
  \item \textbf{If you are optimizing growth/yield:} Prefer \textbf{PAR/PPFD measurement with DLI integration/logging}. For many serious growers, a full-spectrum quantum meter (e.g., MQ-500) or a sensor+logger package (e.g., PQ-500) provides a strong balance between accuracy and workflow \cite{apogee_mq500_page,apogee_pq500,apogee_dli_measuring}.
  \item \textbf{If you use far-red-heavy fixtures and need to quantify extended wavelengths:} Consider ePAR-capable solutions (e.g., PQ-612) \cite{apogee_pq612}.
  \item \textbf{If you need research-grade logging and standardized workflows:} A professional logger platform (e.g., LI-1500 with appropriate quantum sensors) supports automated DLI computation and structured data handling \cite{licor_calculating_dli,licor_li1500_product}.
\end{itemize}

\begin{thebibliography}{99}
\bibitem{noaa_instruments}
NOAA Global Monitoring Laboratory, ``Instruments'' (overview of atmospheric and radiation-related instrumentation).
\href{https://gml.noaa.gov/grad/instruments.html}{https://gml.noaa.gov/grad/instruments.html}

\bibitem{hukseflux_instrument_select}
Hukseflux, ``Measuring sunlight: what instrument to use'' (instrument selection guidance).
\href{https://www.hukseflux.com/library/measuring-sunlight-what-instrument-to-use}{https://www.hukseflux.com/library/measuring-sunlight-what-instrument-to-use}

\bibitem{gardenerspath_best_light_meters}
Gardeners Path, ``Best Light Meters for Houseplants and Indoor Growing'' (consumer options and discussion, including MQ-500 mention).
\href{https://gardenerspath.com/gear/tools-and-supplies/best-light-meters/}{https://gardenerspath.com/gear/tools-and-supplies/best-light-meters/}

\bibitem{apogee_dli_measuring}
Apogee Instruments, ``Daily Light Integral: Measuring Light for Plants'' (DLI integration considerations; MQ-series logging notes).
\href{https://www.apogeeinstruments.com/daily-light-integral-measuring-light-for-plants/}{https://www.apogeeinstruments.com/daily-light-integral-measuring-light-for-plants/}

\bibitem{apogee_mq500_page}
Apogee Instruments, ``MQ-500: Full-Spectrum Quantum Meter'' (PAR range and LED-focused spectral response positioning).
\href{https://www.apogeeinstruments.com/mq-500-full-spectrum-quantum-meter/}{https://www.apogeeinstruments.com/mq-500-full-spectrum-quantum-meter/}

\bibitem{apogee_mq500_pdf}
Apogee Instruments, ``MQ-500 Quantum Meter Manual (PDF)'' (logging and DLI features).
\href{https://www.apogeeinstruments.com/content/MQ-500.pdf}{https://www.apogeeinstruments.com/content/MQ-500.pdf}

\bibitem{apogee_pq500}
Apogee Instruments, ``PQ-500 Package: microCache and Full-spectrum Quantum'' (package contents and intended monitoring workflow).
\href{https://www.apogeeinstruments.com/pq-500-package-microcache-and-full-spectrum-quantum-with-30-cm-cable/}{https://www.apogeeinstruments.com/pq-500-package-microcache-and-full-spectrum-quantum-with-30-cm-cable/}

\bibitem{apogee_pq612}
Apogee Instruments, ``PQ-612 Package: microCache and ePAR Sensor'' (400--750 nm ePAR positioning and package contents).
\href{https://www.apogeeinstruments.com/pq-612-package-microcache-and-epar-sensor-with-2-meter-cable/}{https://www.apogeeinstruments.com/pq-612-package-microcache-and-epar-sensor-with-2-meter-cable/}

\bibitem{licor_calculating_dli}
LI-COR, ``Measuring Daily Light Integral (DLI) with the LI-1500 Light Sensor Logger'' (application note / workflow guidance).
\href{https://www.licor.com/support/LI-1500/topics/calculating-DLI.html}{https://www.licor.com/support/LI-1500/topics/calculating-DLI.html}

\bibitem{licor_li1500_product}
LI-COR, ``LI-1500 Light Sensor Logger'' (product overview and DLI package positioning).
\href{https://www.licor.com/env/products/light/light-logger}{https://www.licor.com/env/products/light/light-logger}

\bibitem{sekonic_l308x}
Sekonic, ``L-308X Flashmate Light Meter'' (photography/video light meter; lux/foot-candle measurements).
\href{https://sekonic.com/sekonic-l-308x-u-flashmate-light-meter/}{https://sekonic.com/sekonic-l-308x-u-flashmate-light-meter/}

\bibitem{photone_app_article}
Grow Light Meter (Photone), ``The Best Light Meter App For Plants'' (example discussion of app-based measurement and variability).
\href{https://growlightmeter.com/the-best-light-meter-app-for-plants-in-2021/}{https://growlightmeter.com/the-best-light-meter-app-for-plants-in-2021/}
\end{thebibliography}

\end{document}
