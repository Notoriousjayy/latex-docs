% !TeX program = pdflatex
\documentclass[11pt,oneside]{article}

%=============================================================================
% PACKAGES
%=============================================================================
\usepackage[margin=1in]{geometry}
\usepackage[T1]{fontenc}
\usepackage[utf8]{inputenc}
\usepackage{amssymb}
%\usepackage{microtype} % disabled due to font expansion issues
\usepackage[hidelinks,colorlinks=true,linkcolor=darkgreen!70!black,urlcolor=blue!70!black]{hyperref}
\usepackage{enumitem}
\usepackage{booktabs}
\usepackage{xcolor}
\usepackage{array}
\usepackage{longtable}
\usepackage{tabularx}
\usepackage{colortbl}
\usepackage{multirow}
\usepackage{fancyhdr}
\usepackage{titlesec}
\usepackage{tcolorbox}
\usepackage{graphicx}
\usepackage{float}
\usepackage{caption}
\usepackage{subcaption}
\usepackage{parskip}
\usepackage{calc}
\usepackage{etoolbox}
\usepackage{pdflscape}
\usepackage{makecell}

%=============================================================================
% COLOR DEFINITIONS
%=============================================================================
\definecolor{darkgreen}{RGB}{34,139,34}
\definecolor{leafgreen}{RGB}{76,153,0}
\definecolor{earthbrown}{RGB}{139,90,43}
\definecolor{skyblue}{RGB}{135,206,235}
\definecolor{sunflower}{RGB}{255,204,0}
\definecolor{tomato}{RGB}{255,99,71}
\definecolor{lavender}{RGB}{230,230,250}
\definecolor{lightgreen}{RGB}{144,238,144}
\definecolor{tableheader}{RGB}{76,153,0}
\definecolor{tablerow1}{RGB}{245,255,245}
\definecolor{tablerow2}{RGB}{255,255,255}
\definecolor{warningbg}{RGB}{255,248,220}
\definecolor{tipbg}{RGB}{240,255,240}
\definecolor{notebg}{RGB}{240,248,255}

%=============================================================================
% TCOLORBOX ENVIRONMENTS
%=============================================================================
\tcbuselibrary{skins,breakable}

\newtcolorbox{warningbox}[1][]{
  enhanced,
  colback=warningbg,
  colframe=sunflower!80!black,
  fonttitle=\bfseries,
  title={\textcolor{sunflower!80!black}{$\triangle$} Warning},
  #1
}

\newtcolorbox{tipbox}[1][]{
  enhanced,
  colback=tipbg,
  colframe=darkgreen,
  fonttitle=\bfseries,
  title={\textcolor{darkgreen}{$\checkmark$} Pro Tip},
  #1
}

\newtcolorbox{notebox}[1][]{
  enhanced,
  colback=notebg,
  colframe=skyblue!70!black,
  fonttitle=\bfseries,
  title={\textcolor{skyblue!70!black}{$\circ$} Note},
  #1
}

\newtcolorbox{criticalbox}[1][]{
  enhanced,
  colback=tomato!10,
  colframe=tomato!80!black,
  fonttitle=\bfseries,
  title={\textcolor{tomato!80!black}{$\bigstar$} Critical Success Factor},
  #1
}

%=============================================================================
% FORMATTING CONFIGURATION
%=============================================================================
\setlist[itemize]{leftmargin=*, itemsep=0.35em, topsep=0.35em}
\setlist[enumerate]{leftmargin=*, itemsep=0.35em, topsep=0.35em}

% Section formatting
\titleformat{\section}
  {\normalfont\Large\bfseries\color{darkgreen}}
  {\thesection}{1em}{}
\titleformat{\subsection}
  {\normalfont\large\bfseries\color{darkgreen!80!black}}
  {\thesubsection}{1em}{}
\titleformat{\subsubsection}
  {\normalfont\normalsize\bfseries\color{darkgreen!60!black}}
  {\thesubsubsection}{1em}{}

% Header/Footer
\pagestyle{fancy}
\fancyhf{}
\fancyhead[L]{\textcolor{darkgreen!70!black}{\small Balcony Container Garden Run Book}}
\fancyhead[R]{\textcolor{darkgreen!70!black}{\small Lithia Springs, GA (Zone 8a)}}
\fancyfoot[C]{\thepage}
\renewcommand{\headrulewidth}{0.4pt}
\renewcommand{\footrulewidth}{0pt}

\setlength{\headheight}{14pt}
% Table configuration
\newcolumntype{L}[1]{>{\raggedright\arraybackslash}p{#1}}
\newcolumntype{C}[1]{>{\centering\arraybackslash}p{#1}}
\newcolumntype{R}[1]{>{\raggedleft\arraybackslash}p{#1}}

%=============================================================================
% DOCUMENT METADATA
%=============================================================================
\title{%
  \vspace{-1cm}
  {\Huge\bfseries\textcolor{darkgreen}{Balcony Container Garden Run Book}}\\[0.5em]
  {\Large\textcolor{earthbrown}{Lithia Springs, Georgia (USDA Zone 8a)}}\\[0.3em]
  {\large\textcolor{darkgreen!70!black}{Ground-Level Balcony --- Container-First Approach}}\\[0.5em]
  {\normalsize\textcolor{gray}{Comprehensive Operational Guide for Year-Round Container Gardening}}
}
\author{}
\date{\textcolor{darkgreen!70!black}{\today}}

%=============================================================================
% BEGIN DOCUMENT
%=============================================================================
\begin{document}
\maketitle
\thispagestyle{empty}

%-----------------------------------------------------------------------------
% ABSTRACT
%-----------------------------------------------------------------------------
\begin{abstract}
\noindent This comprehensive run book provides a container-first, balcony-optimized approach to growing nutrient-dense fruits, vegetables, herbs, and culinary ``spices'' in Lithia Springs, Georgia (USDA Hardiness Zone 8a). Designed specifically for ground-level balconies where shade, humidity, and pest pressure can be higher than on upper floors, this operational guide covers everything from material selection and container sizing to pest management and seasonal scheduling. The document maintains a practical, operational structure including: complete Materials and Tools specifications, detailed Balcony Bill of Materials (BOM), Pre-Season Readiness Reviews, comprehensive crop guidance organized by category, a location-aware seasonal schedule with monthly action items, troubleshooting guides, and a practical system for achieving plant diversity targets. Whether you're a beginner or experienced container gardener, this run book will serve as your complete reference for maximizing yield and plant health in a limited balcony space.
\end{abstract}

\vspace{1em}

%-----------------------------------------------------------------------------
% QUICK REFERENCE CARD
%-----------------------------------------------------------------------------
\begin{tcolorbox}[
  enhanced,
  colback=lightgreen!20,
  colframe=darkgreen,
  title={\textbf{Quick Reference --- Key Zone 8a Dates}},
  fonttitle=\bfseries\color{white},
  colbacktitle=darkgreen
]
\begin{tabularx}{\textwidth}{@{}l X@{}}
\textbf{Last Spring Frost (avg):} & March 15--April 1 \\
\textbf{First Fall Frost (avg):} & November 1--15 \\
\textbf{Growing Season:} & 210--240 days \\
\textbf{Peak Summer Temps:} & 85--95°F (often higher with humidity) \\
\textbf{USDA Zone:} & 8a (10--15°F minimum winter temperatures) \\
\textbf{Annual Rainfall:} & 50--55 inches (humid subtropical) \\
\end{tabularx}
\end{tcolorbox}

\newpage
\tableofcontents
\newpage

%=============================================================================
% SECTION 0: MATERIALS AND TOOLS
%=============================================================================
\section{Materials and Tools}
\label{sec:materials}

This section provides a comprehensive inventory of all materials, tools, and supplies needed for a successful container garden that includes strawberries, blueberries, dwarf citrus (optional), tomatoes, peppers, cucumbers, beans, leafy greens, chard, herbs, and potatoes. Items are organized by functional category with specific product recommendations where applicable.

%-----------------------------------------------------------------------------
\subsection{Containers and Structural Support}
%-----------------------------------------------------------------------------

Container selection is one of the most critical decisions affecting plant health, yield, and maintenance requirements. The right container provides adequate root space, proper drainage, and temperature regulation.

\subsubsection{Primary Container Types}

\begin{longtable}{@{}L{3cm}L{3.5cm}L{7cm}@{}}
\toprule
\rowcolor{tableheader}
\textcolor{white}{\textbf{Container Type}} & \textcolor{white}{\textbf{Best For}} & \textcolor{white}{\textbf{Characteristics \& Notes}} \\
\midrule
\endfirsthead
\rowcolor{tablerow1}
Fabric Grow Bags (5--15 gal) & Potatoes, tomatoes, peppers & Excellent drainage and aeration; promotes air-pruning of roots; lightweight; folds for storage; requires more frequent watering \\
\rowcolor{tablerow2}
Rigid Plastic Pots (3--25+ gal) & Long-lived crops, heavy fruiters & Durable; retains moisture better than fabric; heavier and more stable; may need additional drainage holes \\
\rowcolor{tablerow1}
Glazed Ceramic Pots (3--15 gal) & Herbs, ornamental peppers, citrus & Attractive appearance; excellent moisture retention; heavy (good for stability); can crack in freeze events \\
\rowcolor{tablerow2}
Self-Watering Containers & Herbs, greens, chard & Built-in reservoir maintains moisture; reduces watering frequency; ideal for consistent moisture crops \\
\rowcolor{tablerow1}
Hanging Baskets (10--14 in) & Strawberries, trailing herbs & Saves floor space; improves air circulation; useful for pest-prone crops; requires frequent watering \\
\bottomrule
\end{longtable}

\subsubsection{Container Accessories}

\begin{itemize}
  \item \textbf{Saucers and Drip Trays:} Essential for protecting balcony surfaces from water damage and mineral staining. Select trays 2--3 inches larger than pot diameter. Clear trays allow easy monitoring of drainage.
  
  \item \textbf{Pot Risers and Feet:} Elevate containers 1--2 inches above surfaces to improve drainage, prevent root rot, reduce pest access (slugs, pillbugs), and allow air circulation beneath pots. Types include:
  \begin{itemize}
    \item Plastic pot feet (economical, lightweight)
    \item Terracotta feet (decorative, stable)
    \item Wheeled caddies with integrated risers (for heavy pots)
  \end{itemize}
  
  \item \textbf{Wheeled Plant Caddies:} Essential for containers exceeding 15 gallons. Features to look for:
  \begin{itemize}
    \item Weight capacity rated for fully watered pot weight
    \item Locking wheels for stability
    \item UV-resistant materials
    \item Drainage-compatible design (perforated or with channels)
  \end{itemize}
\end{itemize}

\subsubsection{Vertical Support Structures}

\begin{itemize}
  \item \textbf{Tomato Cages:} Select heavy-duty galvanized steel cages rated for indeterminate varieties. Minimum specifications: 54--60 inches tall, 18--20 inch diameter at base. Avoid flimsy cone-style cages for full-size tomatoes.
  
  \item \textbf{Trellis Panels:} For cucumbers, pole beans, and vining crops. Options include:
  \begin{itemize}
    \item A-frame trellises (freestanding, portable)
    \item Wall-mounted lattice panels (space-efficient)
    \item Nylon trellis netting (5--6 inch mesh size ideal)
    \item DIY bamboo or wooden frames
  \end{itemize}
  
  \item \textbf{Railing Supports:} Adjustable brackets that clamp to balcony railings for mounting planters or trellis systems. Verify weight rating and railing compatibility.
  
  \item \textbf{Plant Ties and Clips:}
  \begin{itemize}
    \item Soft fabric ties (prevent stem damage)
    \item Velcro plant tape (adjustable, reusable)
    \item Tomato clips (quick attachment to stakes/strings)
    \item Jute twine (biodegradable, economical)
  \end{itemize}
\end{itemize}

%-----------------------------------------------------------------------------
\subsection{Growing Media and Amendments}
%-----------------------------------------------------------------------------

Container growing media is fundamentally different from garden soil. Never use native soil or ``topsoil'' in containers---it compacts, drains poorly, and may harbor pests and diseases.

\subsubsection{Base Growing Media}

\begin{itemize}
  \item \textbf{Premium Potting Mix:} Select a high-quality container mix with the following characteristics:
  \begin{itemize}
    \item Peat moss or coconut coir base (moisture retention)
    \item Perlite or pumice (drainage and aeration)
    \item Composted bark (structure)
    \item Slow-release fertilizer (optional, 3--6 month formula)
    \item pH range: 6.0--7.0 for most crops
  \end{itemize}
  
  \item \textbf{Specialized Mixes:}
  \begin{itemize}
    \item \textbf{Acidic blueberry mix:} pH 4.5--5.5; peat-heavy with sulfur
    \item \textbf{Cactus/citrus mix:} Extra perlite for improved drainage
    \item \textbf{Seed starting mix:} Fine-textured, sterile, low-nutrient
  \end{itemize}
\end{itemize}

\begin{warningbox}
Avoid potting mixes with high percentages of wood chips or uncomposted organic matter. These can tie up nitrogen as they decompose, causing temporary nutrient deficiency. Also avoid mixes with water-absorbing polymer gels for blueberries (they can alter pH).
\end{warningbox}

\subsubsection{Soil Amendments}

\begin{longtable}{@{}L{3cm}L{3cm}L{7.5cm}@{}}
\toprule
\rowcolor{tableheader}
\textcolor{white}{\textbf{Amendment}} & \textcolor{white}{\textbf{Application Rate}} & \textcolor{white}{\textbf{Purpose \& Usage Notes}} \\
\midrule
\endfirsthead
\rowcolor{tablerow1}
Compost & 10--20\% of mix by volume & Improves microbial activity, water retention, and nutrient availability. Use finished, dark, crumbly compost only. \\
\rowcolor{tablerow2}
Worm Castings & 10--15\% of mix; 1/4 inch top-dress & Rich in beneficial microbes and readily available nutrients. Excellent for container vegetables. \\
\rowcolor{tablerow1}
Perlite & 15--25\% of mix by volume & Volcanic glass that improves drainage and aeration. Essential for heavy-feeding crops. \\
\rowcolor{tablerow2}
Pumice & 15--25\% of mix by volume & Similar to perlite but heavier and doesn't float. Better for outdoor containers. \\
\rowcolor{tablerow1}
Coconut Coir & Substitute for peat & Sustainable peat alternative. Rinse before use to remove salts. Slightly alkaline. \\
\rowcolor{tablerow2}
Pine Bark Fines & 10--20\% of mix & Improves drainage and structure. Acidifies slightly over time (good for blueberries). \\
\bottomrule
\end{longtable}

\subsubsection{Mulch Materials}

Mulching containers reduces water loss by 25--50\%, moderates soil temperature, and suppresses weeds.

\begin{itemize}
  \item \textbf{Straw:} Excellent insulator; 2--3 inch layer; may contain weed seeds
  \item \textbf{Shredded Leaves:} Free and effective; 1--2 inch layer; breaks down quickly
  \item \textbf{Coco Coir Chips:} Attractive, long-lasting; 1--2 inch layer
  \item \textbf{Pine Straw:} Acidifying; 2--3 inch layer; ideal for blueberries
  \item \textbf{Compost:} Nutrient-rich; 1/2--1 inch layer; replenish monthly
\end{itemize}

%-----------------------------------------------------------------------------
\subsection{Fertilizers and Plant Nutrition}
%-----------------------------------------------------------------------------

Container plants have limited soil volume to draw nutrients from and require regular fertilization for optimal production.

\subsubsection{Primary Fertilizer Types}

\begin{itemize}
  \item \textbf{Slow-Release Granular Fertilizer:}
  \begin{itemize}
    \item Formula: Balanced (10-10-10) or tomato-specific (higher phosphorus/potassium)
    \item Application: Mix into potting soil at planting; reapply every 2--4 months
    \item Best for: Baseline nutrition throughout the season
  \end{itemize}
  
  \item \textbf{Liquid Fertilizer (Water-Soluble):}
  \begin{itemize}
    \item Formula: Fish emulsion (5-1-1), seaweed extract, or synthetic balanced
    \item Application: Weekly to biweekly during active growth and fruiting
    \item Dilution: Half-strength more frequently is better than full-strength occasionally
    \item Best for: Quick-response feeding during peak demand
  \end{itemize}
  
  \item \textbf{Specialty Fertilizers:}
  \begin{itemize}
    \item \textbf{Citrus fertilizer:} Contains iron, zinc, manganese for citrus micronutrient needs
    \item \textbf{Acid-loving plant fertilizer:} For blueberries; maintains low pH
    \item \textbf{Tomato/vegetable fertilizer:} Higher in potassium for fruit development
  \end{itemize}
\end{itemize}

\subsubsection{Micronutrient Support}

\begin{itemize}
  \item \textbf{Calcium:} Prevents blossom end rot in tomatoes/peppers. Sources include gypsum (calcium sulfate) and calcium nitrate foliar spray. Apply preventively if your water is soft or you've had issues before.
  
  \item \textbf{Magnesium:} Essential for chlorophyll production. Epsom salt (magnesium sulfate) at 1 tablespoon per gallon monthly during fruiting. Watch for interveinal yellowing as deficiency sign.
  
  \item \textbf{Iron:} Critical for blueberries and citrus in alkaline conditions. Use chelated iron (EDDHA form is most stable at high pH) if chlorosis develops.
\end{itemize}

\begin{tipbox}
Create a feeding calendar at the start of the season. Mark slow-release application dates, weekly liquid feeding reminders, and micronutrient applications. Consistent, scheduled feeding outperforms reactive feeding.
\end{tipbox}

%-----------------------------------------------------------------------------
\subsection{Watering and Moisture Management}
%-----------------------------------------------------------------------------

Proper watering is the single most important skill for container gardening success. Containers dry out faster than in-ground gardens and are less forgiving of inconsistency.

\subsubsection{Watering Equipment}

\begin{itemize}
  \item \textbf{Watering Can (2+ gallon):} Essential for targeted watering. Features to look for:
  \begin{itemize}
    \item Removable rose (breaker) for gentle watering
    \item Comfortable handle for repeated lifting
    \item Clear volume markings for fertilizer mixing
  \end{itemize}
  
  \item \textbf{Hose Wand (16--24 inch):} Extends reach and provides gentle water flow. Water breaker attachment prevents soil compaction and splash.
  
  \item \textbf{Drip Irrigation System:} Highly recommended for consistent moisture and water conservation.
  \begin{itemize}
    \item Components: Main tubing (1/2 inch), micro-tubing (1/4 inch), drip emitters (1--2 GPH), stakes
    \item Timer: Battery-powered with multiple start times per day
    \item Filters: Inline filter prevents clogging
    \item Pressure regulator: Maintains consistent flow (typically 25 PSI)
  \end{itemize}
  
  \item \textbf{Moisture Meter:} Inexpensive tool that removes guesswork. Insert probe to root zone depth (4--6 inches) before watering.
\end{itemize}

\subsubsection{Watering Guidelines by Season}

\begin{longtable}{@{}L{2.5cm}L{3cm}L{4cm}L{4cm}@{}}
\toprule
\rowcolor{tableheader}
\textcolor{white}{\textbf{Season}} & \textcolor{white}{\textbf{Frequency}} & \textcolor{white}{\textbf{Best Time}} & \textcolor{white}{\textbf{Special Considerations}} \\
\midrule
\endfirsthead
\rowcolor{tablerow1}
Spring & Every 2--3 days & Morning & Increase as plants size up \\
\rowcolor{tablerow2}
Early Summer & Daily to every other day & Morning & Monitor new plantings closely \\
\rowcolor{tablerow1}
Peak Summer & 1--2 times daily & Early morning + late afternoon & Critical period; don't let pots dry completely \\
\rowcolor{tablerow2}
Fall & Every 2--4 days & Morning & Reduce as growth slows \\
\rowcolor{tablerow1}
Winter & Every 1--2 weeks & Midday (warmer temps) & Only for overwintered perennials \\
\bottomrule
\end{longtable}

\begin{criticalbox}
\textbf{The Finger Test:} Insert your finger 2 inches into the soil. If it feels dry at that depth, water thoroughly until water flows from drainage holes. If moist, check again tomorrow. This simple test prevents both over- and under-watering.
\end{criticalbox}

%-----------------------------------------------------------------------------
\subsection{Pest and Disease Management}
%-----------------------------------------------------------------------------

Ground-level balconies face unique pest pressures. A proactive, integrated approach prevents most problems before they require treatment.

\subsubsection{Physical Barriers and Prevention}

\begin{itemize}
  \item \textbf{Insect Netting (Fine Mesh):}
  \begin{itemize}
    \item Mesh size: 0.6mm or smaller excludes most pests
    \item Uses: Cover seedlings, leafy greens, and brassicas
    \item Installation: Support with hoops or frames to prevent leaf contact
    \item Also protects from bird damage and reduces wind stress
  \end{itemize}
  
  \item \textbf{Row Cover/Floating Row Cover:}
  \begin{itemize}
    \item Lightweight fabric allows light and water penetration
    \item Provides 2--4°F frost protection as bonus
    \item Remove when plants flower if pollination is needed
  \end{itemize}
  
  \item \textbf{Copper Tape:}
  \begin{itemize}
    \item Barrier for slugs and snails (causes mild electric shock)
    \item Apply around pot rims or on pot risers
    \item Must be unbroken circuit; clean periodically
  \end{itemize}
  
  \item \textbf{Sticky Traps:}
  \begin{itemize}
    \item Yellow traps: Whiteflies, aphids, fungus gnats, leafminers
    \item Blue traps: Thrips
    \item Use for monitoring population levels, not primary control
  \end{itemize}
\end{itemize}

\subsubsection{Organic and Low-Toxicity Treatments}

\begin{longtable}{@{}L{3cm}L{3cm}L{7.5cm}@{}}
\toprule
\rowcolor{tableheader}
\textcolor{white}{\textbf{Product}} & \textcolor{white}{\textbf{Target Pests}} & \textcolor{white}{\textbf{Application Notes}} \\
\midrule
\endfirsthead
\rowcolor{tablerow1}
Insecticidal Soap & Aphids, whiteflies, spider mites, mealybugs & Spray directly on pests; reapply every 5--7 days; avoid mid-day application; rinse edibles before harvest \\
\rowcolor{tablerow2}
Horticultural Oil (Neem) & Soft-bodied insects, fungal issues & Apply in morning or evening; coat all surfaces; don't use above 85°F; avoid sulfur within 2 weeks \\
\rowcolor{tablerow1}
Bacillus thuringiensis (Bt) & Caterpillars, hornworms & Spray on foliage; must be ingested; reapply after rain; won't harm beneficial insects \\
\rowcolor{tablerow2}
Spinosad & Beetles, caterpillars, thrips & Evening application only (toxic to bees until dry); highly effective; OMRI listed \\
\rowcolor{tablerow1}
Diatomaceous Earth & Slugs, crawling insects & Apply to dry soil surface; reapply after rain or watering; use food-grade only \\
\rowcolor{tablerow2}
Iron Phosphate Bait & Slugs and snails & Scatter around base of plants; safe for pets and wildlife; reapply after rain \\
\bottomrule
\end{longtable}

\begin{warningbox}
\textbf{Spray Safety:} Always test any spray on a few leaves before full application. Wait 48 hours to check for leaf burn. Never spray in full sun or when temperatures exceed 85°F. Spray in the morning so foliage dries before evening (reduces disease risk).
\end{warningbox}

\subsubsection{Sanitation Kit}

Maintain a dedicated sanitation kit to prevent disease spread:

\begin{itemize}
  \item \textbf{Bypass Pruners:} Sharp, clean cuts heal faster. Dedicated pair for pruning---not for harvest.
  \item \textbf{Micro-tip Snips:} For deadheading, removing individual diseased leaves, and herb harvesting.
  \item \textbf{70\% Isopropyl Alcohol:} Disinfect tools between plants, especially after cutting diseased tissue.
  \item \textbf{Disposable Gloves:} Prevent contact transmission of viral and bacterial diseases.
  \item \textbf{Trash Bag:} Immediately bag and dispose of diseased plant material. Never compost diseased tissue.
\end{itemize}

%-----------------------------------------------------------------------------
\subsection{Ground-Level Balcony Specific Equipment}
%-----------------------------------------------------------------------------

Ground-level positions present unique challenges that require additional equipment and strategies.

\begin{itemize}
  \item \textbf{Elevated Growing Systems:}
  \begin{itemize}
    \item Tiered plant stands (raises lower plants into better light)
    \item Wall-mounted vertical planters (maximizes space, improves airflow)
    \item Hanging systems (protects crops from ground-dwelling pests)
  \end{itemize}
  
  \item \textbf{Rodent and Squirrel Deterrents:}
  \begin{itemize}
    \item Hardware cloth cages (1/2 inch mesh) for vulnerable containers
    \item Motion-activated sprinklers (if water access available)
    \item Commercial bitter sprays (reapply after rain)
    \item Physical barriers: mesh covers for ripening fruit
  \end{itemize}
  
  \item \textbf{Reflective Materials:}
  \begin{itemize}
    \item Reflective mulch (aluminum or white plastic) increases light to lower leaves
    \item Light-colored backdrop panels redirect ambient light
    \item Particularly valuable for balconies with partial shade
  \end{itemize}
  
  \item \textbf{Wind Protection:}
  \begin{itemize}
    \item Shade cloth (40\%) can double as wind barrier
    \item Positioning containers in corner clusters reduces wind exposure
    \item Stakes and supports must be rated for wind load
  \end{itemize}
\end{itemize}

%=============================================================================
% SECTION 0.1: BILL OF MATERIALS
%=============================================================================
\section{Balcony Bill of Materials (BOM)}
\label{sec:bom}

This Bill of Materials provides a practical, costed baseline for a 12-container setup. Use the scaling rules in \S\ref{sec:scaling} to adapt for larger or smaller gardens.

%-----------------------------------------------------------------------------
\subsection{Standard 12-Container Layout}
%-----------------------------------------------------------------------------

This layout is optimized for a ground-level balcony with 6+ hours of direct sunlight, approximately 80--100 square feet of floor space.

\begin{center}
\begin{tabularx}{\textwidth}{@{}C{1.5cm}L{3cm}L{3cm}X@{}}
\toprule
\rowcolor{tableheader}
\textcolor{white}{\textbf{Qty}} & \textcolor{white}{\textbf{Container Size}} & \textcolor{white}{\textbf{Crop Assignment}} & \textcolor{white}{\textbf{Container Notes}} \\
\midrule
2 & 10-gallon rigid & Tomatoes (indeterminate) & Heavy-duty with cage support \\
2 & 5-gallon rigid & Peppers (bell or hot) & Stable base; dark color for warmth \\
2 & 7-gallon grow bag & Bush cucumbers or beans & Fabric for drainage; with trellis \\
4 & 3-gallon various & Herbs + greens/chard & Mix of self-watering and standard \\
1 & 15-gallon rigid & Blueberry (dwarf) & Dedicated acidic media \\
1 & 20-gallon rigid & Dwarf citrus or fig & On wheeled caddy; brightest spot \\
\midrule
\textbf{12} & \multicolumn{3}{l}{\textbf{Total containers}} \\
\bottomrule
\end{tabularx}
\end{center}

%-----------------------------------------------------------------------------
\subsection{Complete BOM Quantities}
%-----------------------------------------------------------------------------

\subsubsection{Containers and Support (Hardware)}

\begin{center}
\begin{tabularx}{\textwidth}{@{}L{5cm}C{2cm}X@{}}
\toprule
\rowcolor{tableheader}
\textcolor{white}{\textbf{Item}} & \textcolor{white}{\textbf{Quantity}} & \textcolor{white}{\textbf{Specifications/Notes}} \\
\midrule
10-gallon rigid pots & 2 & Plastic or fabric; drainage holes \\
5-gallon rigid pots & 2 & Heavy-duty plastic \\
7-gallon grow bags & 2 & 300GSM fabric minimum \\
3-gallon pots & 4 & Mix types as needed \\
15-gallon pot & 1 & Deep profile for blueberry \\
20-gallon pot & 1 & Widest diameter available \\
Saucers/drip trays & 12 & Size-matched to each pot \\
Pot risers/feet & 12 sets & 3--4 feet per pot \\
Wheeled caddy & 1--2 & 22+ inch diameter; 100+ lb capacity \\
Tomato cages (heavy-duty) & 2 & 54--60 inch tall \\
Trellis panels & 2 & 4 ft × 6 ft or equivalent \\
Soft plant ties & 1 roll & 50+ feet; reusable \\
Tomato clips & 20--30 & Assorted sizes \\
\bottomrule
\end{tabularx}
\end{center}

\subsubsection{Growing Media}

Total pot volume for the 12-container layout: approximately \textbf{91 gallons} (12.2 cubic feet).

Purchase 15--20\% extra for settling, spillage, and top-off throughout the season.

\begin{center}
\begin{tabularx}{\textwidth}{@{}L{5cm}C{3cm}X@{}}
\toprule
\rowcolor{tableheader}
\textcolor{white}{\textbf{Item}} & \textcolor{white}{\textbf{Quantity}} & \textcolor{white}{\textbf{Notes}} \\
\midrule
Premium potting mix & 7--8 bags (2 cu ft each) & 14--16 cu ft total \\
Acidic blueberry mix & 1 bag (2 cu ft) & For blueberry container only \\
Compost/worm castings & 1--2 cu ft & Top-dressing through season \\
Perlite (optional) & 1 bag (4 qt) & For improving drainage in citrus \\
Mulch (straw/coir) & 2--3 cu ft & Surface covering for 8+ pots \\
\bottomrule
\end{tabularx}
\end{center}

\subsubsection{Fertilizers and Nutrients}

\begin{center}
\begin{tabularx}{\textwidth}{@{}L{5cm}C{3cm}X@{}}
\toprule
\rowcolor{tableheader}
\textcolor{white}{\textbf{Item}} & \textcolor{white}{\textbf{Quantity}} & \textcolor{white}{\textbf{Application Notes}} \\
\midrule
Slow-release granular (balanced) & 2--3 lbs & Initial mix-in + 1--2 reapplications \\
Liquid fertilizer concentrate & 32 oz & 16+ weekly applications \\
Citrus fertilizer & 1 lb & Per label schedule (citrus only) \\
Acidic fertilizer & 1 lb & Blueberry applications \\
Epsom salt & 1 lb & Magnesium supplementation \\
Calcium source (gypsum) & 1 lb & Blossom end rot prevention \\
\bottomrule
\end{tabularx}
\end{center}

\subsubsection{Watering and Irrigation}

\begin{center}
\begin{tabularx}{\textwidth}{@{}L{5cm}C{2cm}X@{}}
\toprule
\rowcolor{tableheader}
\textcolor{white}{\textbf{Item}} & \textcolor{white}{\textbf{Quantity}} & \textcolor{white}{\textbf{Specifications}} \\
\midrule
Watering can & 1 & 2+ gallon with removable rose \\
Hose wand (optional) & 1 & 24 inch with water breaker \\
Moisture meter & 1 & Analog probe type \\
Drip irrigation kit & 1 & Sized for 12--15 pots \\
Battery timer & 1 & 2+ programs per day \\
\bottomrule
\end{tabularx}
\end{center}

\subsubsection{Pest and Disease Control}

\begin{center}
\begin{tabularx}{\textwidth}{@{}L{5cm}C{2cm}X@{}}
\toprule
\rowcolor{tableheader}
\textcolor{white}{\textbf{Item}} & \textcolor{white}{\textbf{Quantity}} & \textcolor{white}{\textbf{Notes}} \\
\midrule
Insect netting & 10 ft × 10 ft & Fine mesh (0.6mm or smaller) \\
Sticky traps (yellow) & 10--20 pack & Monitoring \\
Insecticidal soap & 32 oz RTU & Ready-to-use spray \\
Neem oil concentrate & 16 oz & Mix as needed \\
Bt spray & 8 oz & For caterpillar issues \\
Copper tape & 1 roll & Slug/snail barrier \\
Iron phosphate bait & 1 lb & Slug/snail control \\
Pruners (bypass) & 1 pair & Sharp, quality brand \\
Micro-tip snips & 1 pair & For harvesting and detail work \\
Isopropyl alcohol (70\%) & 16 oz & Tool sanitation \\
\bottomrule
\end{tabularx}
\end{center}

%-----------------------------------------------------------------------------
\subsection{Scaling Rules}
\label{sec:scaling}
%-----------------------------------------------------------------------------

Use these calculations to adjust potting mix purchases for different container configurations:

\begin{center}
\begin{tabularx}{\textwidth}{@{}L{4cm}C{3cm}C{3cm}X@{}}
\toprule
\rowcolor{tableheader}
\textcolor{white}{\textbf{Container Size}} & \textcolor{white}{\textbf{Gallons}} & \textcolor{white}{\textbf{Cubic Feet}} & \textcolor{white}{\textbf{Liters (approx)}} \\
\midrule
3-gallon pot & 3 gal & 0.40 cu ft & 11.4 L \\
5-gallon pot & 5 gal & 0.67 cu ft & 19.0 L \\
7-gallon pot & 7 gal & 0.94 cu ft & 26.5 L \\
10-gallon pot & 10 gal & 1.34 cu ft & 37.9 L \\
15-gallon pot & 15 gal & 2.01 cu ft & 56.8 L \\
20-gallon pot & 20 gal & 2.67 cu ft & 75.7 L \\
25-gallon pot & 25 gal & 3.34 cu ft & 94.6 L \\
\bottomrule
\end{tabularx}
\end{center}

\begin{notebox}
\textbf{Cost Optimization:} Potting mix is the largest consumable expense. Buy in bulk (3+ cubic feet bags) when possible. Many garden centers offer spring sales in March--April. Store unused mix in a dry location in sealed containers.
\end{notebox}

%=============================================================================
% SECTION 0.2: PRE-SEASON READINESS REVIEW
%=============================================================================
\section{Pre-Season Readiness Review}
\label{sec:readiness}

Complete this comprehensive checklist \textbf{2--4 weeks before} your main spring planting window (late February to mid-March for Lithia Springs).

%-----------------------------------------------------------------------------
\subsection{Site Assessment Checklist}
%-----------------------------------------------------------------------------

\subsubsection{1. Sun Audit (Critical for Ground-Level Balconies)}

Ground-level positions often receive less direct sunlight due to adjacent buildings, fences, and vegetation.

\begin{enumerate}
  \item \textbf{Conduct a multi-day sun survey:}
  \begin{itemize}
    \item Track direct sunlight hours at different locations on your balcony
    \item Note times and durations (e.g., 8:00--11:30 AM direct sun; 11:30--2:00 PM filtered)
    \item Repeat for 2--3 consecutive days at intervals throughout the week
    \item Account for seasonal changes (sun angle increases through spring/summer)
  \end{itemize}
  
  \item \textbf{Map your microclimates:}
  \begin{itemize}
    \item Identify the ``hot spots'' receiving 6+ hours direct sun (reserve for tomatoes, peppers, citrus)
    \item Mark partial shade zones (4--6 hours) for greens, herbs, and root crops
    \item Note heavily shaded areas (under 4 hours)---limit to shade-tolerant herbs only
  \end{itemize}
  
  \item \textbf{Adjust crop selection based on results:}
\end{enumerate}

\begin{center}
\begin{tabularx}{\textwidth}{@{}C{3cm}L{4cm}X@{}}
\toprule
\rowcolor{tableheader}
\textcolor{white}{\textbf{Sun Hours}} & \textcolor{white}{\textbf{Classification}} & \textcolor{white}{\textbf{Suitable Crops}} \\
\midrule
8+ hours & Full sun & Tomatoes, peppers, cucumbers, squash, beans, citrus, strawberries \\
6--8 hours & Full sun (minimum) & All above (may have reduced yield), most herbs \\
4--6 hours & Partial shade & Leafy greens, chard, root crops, herbs, blueberries \\
2--4 hours & Shade & Mint, parsley, chives, lettuce (spring/fall only) \\
\bottomrule
\end{tabularx}
\end{center}

\subsubsection{2. Drainage Validation}

\begin{enumerate}
  \item \textbf{Container inspection:}
  \begin{itemize}
    \item Verify every pot has adequate drainage holes (minimum 3--4 holes per container)
    \item Check that holes are not blocked by roots from previous season
    \item Add additional holes to containers with poor drainage
  \end{itemize}
  
  \item \textbf{Perform a drainage test:}
  \begin{itemize}
    \item Fill each container with water and time how long until drainage stops
    \item Target: Majority of water should drain within 30--60 seconds
    \item Slow drainage indicates compacted or waterlogged media---refresh
  \end{itemize}
  
  \item \textbf{Elevation and airflow:}
  \begin{itemize}
    \item Confirm all containers are elevated on risers (1--2 inches minimum)
    \item Verify saucers drain properly and don't hold standing water
    \item Check spacing between containers (minimum 6--12 inches for airflow)
  \end{itemize}
\end{enumerate}

\subsubsection{3. Infrastructure Inspection}

\begin{enumerate}
  \item \textbf{Structural components:}
  \begin{itemize}
    \item Inspect trellises, cages, and supports for rust, rot, or damage
    \item Test stability---re-anchor or replace compromised structures
    \item Verify wheeled caddies roll freely and lock properly
  \end{itemize}
  
  \item \textbf{Irrigation system (if applicable):}
  \begin{itemize}
    \item Flush all lines with clean water
    \item Check for clogged emitters---soak in vinegar solution if needed
    \item Replace cracked or brittle tubing
    \item Test timer batteries and programming
  \end{itemize}
\end{enumerate}

%-----------------------------------------------------------------------------
\subsection{Media and Nutrition Staging}
%-----------------------------------------------------------------------------

\begin{enumerate}
  \item \textbf{Assess existing media:}
  \begin{itemize}
    \item Second-year potting mix can be refreshed: remove top 2--3 inches, replace with fresh mix plus compost
    \item Test pH if growing blueberries (target 4.5--5.5)
    \item Heavily compacted or waterlogged media should be completely replaced
  \end{itemize}
  
  \item \textbf{Pre-wet new potting mix:}
  \begin{itemize}
    \item Most bagged mixes are hydrophobic when dry
    \item In a wheelbarrow or large container, add water gradually while mixing
    \item Target: Mix should be evenly moist like a wrung-out sponge
    \item Let pre-wetted mix rest 24--48 hours before filling containers
  \end{itemize}
  
  \item \textbf{Stage fertilizers and amendments:}
  \begin{itemize}
    \item Mix slow-release granules into fresh potting mix before filling containers
    \item Prepare compost/castings for top-dressing
    \item Organize liquid fertilizers with clear labeling
  \end{itemize}
\end{enumerate}

%-----------------------------------------------------------------------------
\subsection{Pest Baseline Controls (Ground-Level Specific)}
%-----------------------------------------------------------------------------

Ground-level positions require proactive pest management before planting.

\begin{enumerate}
  \item \textbf{Environmental sanitation:}
  \begin{itemize}
    \item Remove all plant debris from previous season
    \item Clear area beneath and around containers
    \item Eliminate standing water, leaves, and mulch piles where pests overwinter
  \end{itemize}
  
  \item \textbf{Physical barrier staging:}
  \begin{itemize}
    \item Apply copper tape to pot rims before planting
    \item Have insect netting cut to size and ready for installation
    \item Position sticky traps at multiple locations for baseline monitoring
  \end{itemize}
  
  \item \textbf{Wildlife assessment:}
  \begin{itemize}
    \item Note evidence of rodent or squirrel activity
    \item Prepare mesh covers or cages for vulnerable crops (strawberries, tomatoes)
  \end{itemize}
\end{enumerate}

%-----------------------------------------------------------------------------
\subsection{Seed-Starting Plan (If Applicable)}
%-----------------------------------------------------------------------------

Starting seeds indoors extends your growing season and provides access to more varieties.

\begin{center}
\begin{tabularx}{\textwidth}{@{}L{3cm}C{2.5cm}C{2.5cm}X@{}}
\toprule
\rowcolor{tableheader}
\textcolor{white}{\textbf{Crop}} & \textcolor{white}{\textbf{Start Indoors}} & \textcolor{white}{\textbf{Transplant}} & \textcolor{white}{\textbf{Notes}} \\
\midrule
Peppers & 8--10 weeks early & After last frost & Slow germination; need warmth \\
Tomatoes & 6--8 weeks early & After last frost & Faster than peppers \\
Cucumbers & 3--4 weeks early & After soil warms & Direct sow also works \\
Herbs (basil) & 6--8 weeks early & After last frost & Frost-sensitive \\
Greens/Chard & 4--6 weeks early & 4 weeks before last frost & Cold-tolerant \\
\bottomrule
\end{tabularx}
\end{center}

\begin{tipbox}
\textbf{Hardening Off Protocol:} Begin 7--10 days before transplanting. Day 1--2: 1--2 hours in shade. Day 3--4: 3--4 hours with some sun. Day 5--6: 5--6 hours in increasing sun. Day 7+: Full day exposure. Bring indoors if frost threatens.
\end{tipbox}

%=============================================================================
% SECTION 1: FRUITS FOR CONTAINERS
%=============================================================================
\section{Fruits for Containers (Pots and Hanging Baskets)}
\label{sec:fruits}

Growing fruit in containers is highly rewarding but requires attention to variety selection, container sizing, and cultural requirements specific to each crop.

%-----------------------------------------------------------------------------
\subsection{Strawberries}
%-----------------------------------------------------------------------------

Strawberries are the quintessential container fruit---productive, beautiful, and perfect for small spaces.

\subsubsection{Variety Selection}

\begin{longtable}{@{}L{3cm}L{4cm}L{6.5cm}@{}}
\toprule
\rowcolor{tableheader}
\textcolor{white}{\textbf{Type}} & \textcolor{white}{\textbf{Recommended Varieties}} & \textcolor{white}{\textbf{Characteristics}} \\
\midrule
\endfirsthead
\rowcolor{tablerow1}
Day-Neutral & Seascape, Albion, Monterey, Evie-2 & Fruit continuously through the season; not triggered by daylength; best for containers \\
\rowcolor{tablerow2}
Everbearing & Ozark Beauty, Quinault, Fort Laramie & Two main crops (spring + fall); more heat-tolerant than June-bearers \\
\rowcolor{tablerow1}
June-Bearing & Chandler, Camarosa, Sweet Charlie & One large spring harvest; replace after 3 years; best for heavy single harvest \\
\bottomrule
\end{longtable}

\subsubsection{Container and Media Requirements}

\begin{itemize}
  \item \textbf{Container options:}
  \begin{itemize}
    \item Hanging baskets (12--14 inch): 3--5 plants per basket
    \item Strawberry pots (multi-pocket): 6--12 plants per pot
    \item Standard containers (3--5 gallon): 4--6 plants per container
    \item Grow bags or grow towers: Follow manufacturer spacing
  \end{itemize}
  
  \item \textbf{Media:} Standard potting mix with 10--15\% perlite for drainage. pH 5.5--6.8.
  
  \item \textbf{Spacing:} 8--12 inches between plants (crowns should not be buried).
\end{itemize}

\subsubsection{Operational Guidance}

\begin{itemize}
  \item \textbf{Watering:} Keep evenly moist but never waterlogged. Strawberries have shallow roots and dry out quickly. Drip irrigation or consistent hand-watering is essential.
  
  \item \textbf{Fertilization:} Light feeders. Apply balanced liquid fertilizer every 2--3 weeks during active growth. Reduce nitrogen during fruiting to prevent soft fruit.
  
  \item \textbf{Runner management:}
  \begin{itemize}
    \item For maximum fruit: Remove all runners as they appear
    \item For new plants: Allow 1--2 runners per plant to root, then sever
    \item Runners divert energy from fruit production
  \end{itemize}
  
  \item \textbf{Overwintering (Zone 8a):} Mulch heavily (4--6 inches straw) when temperatures drop below 25°F. Move containers to sheltered location. Plants can survive to approximately 15°F with protection.
\end{itemize}

\begin{warningbox}
\textbf{Bird Protection:} Ripening strawberries attract birds immediately. Install netting before fruit colors or you will lose your entire harvest within days.
\end{warningbox}

%-----------------------------------------------------------------------------
\subsection{Blueberries}
%-----------------------------------------------------------------------------

Container blueberries require more attention to cultural requirements than most fruits but reward with years of production.

\subsubsection{Variety Selection}

Choose southern highbush or rabbiteye varieties for Zone 8a. Most require a pollinizer (plant two different varieties).

\begin{longtable}{@{}L{3cm}L{3cm}L{7.5cm}@{}}
\toprule
\rowcolor{tableheader}
\textcolor{white}{\textbf{Type}} & \textcolor{white}{\textbf{Varieties}} & \textcolor{white}{\textbf{Notes for Containers}} \\
\midrule
\endfirsthead
\rowcolor{tablerow1}
Southern Highbush & Sunshine Blue, Top Hat, Misty, O'Neal, Jewel & Lower chill requirements; 150--500 hours; most adaptable \\
\rowcolor{tablerow2}
Rabbiteye & Tifblue, Climax, Premier, Brightwell & Higher vigor; need 400--600 chill hours; late-ripening \\
\rowcolor{tablerow1}
Half-High Hybrids & Northblue, Northsky, Polaris & Very compact; good for smallest spaces; cold-hardy \\
\bottomrule
\end{longtable}

\subsubsection{Container and Media Requirements}

\begin{criticalbox}
\textbf{Critical Success Factor:} Acidic soil (pH 4.5--5.5) is absolutely essential. Blueberries cannot absorb iron and other nutrients at higher pH levels, resulting in chlorosis (yellowing) and death. This is the number one cause of container blueberry failure.
\end{criticalbox}

\begin{itemize}
  \item \textbf{Container size:} 15--20 gallon minimum for mature plants. Start young plants in 7--10 gallon and up-pot as they grow.
  
  \item \textbf{Media recipe:}
  \begin{itemize}
    \item 60\% peat moss (naturally acidic)
    \item 20\% pine bark fines
    \item 20\% perlite
    \item Add sulfur per package directions to achieve target pH
  \end{itemize}
  
  \item \textbf{Water quality:} Municipal water is often alkaline (pH 7.5+) and will gradually raise soil pH. Solutions:
  \begin{itemize}
    \item Collect rainwater (naturally acidic)
    \item Add 1 tablespoon white vinegar per gallon of tap water
    \item Use citric acid to acidify irrigation water
  \end{itemize}
\end{itemize}

\subsubsection{Operational Guidance}

\begin{itemize}
  \item \textbf{pH monitoring:} Test soil pH monthly during growing season. Acidify if pH drifts above 5.5.
  
  \item \textbf{Fertilization:} Use acid-forming fertilizer (ammonium sulfate base) or dedicated azalea/blueberry fertilizer. Apply monthly during growing season at half recommended rate. Never use nitrate-form nitrogen.
  
  \item \textbf{Watering:} Blueberries need consistent moisture---more than most fruits. Never let containers dry out completely, but ensure excellent drainage.
  
  \item \textbf{Chlorosis response:} If leaves yellow between veins (interveinal chlorosis):
  \begin{enumerate}
    \item Test pH immediately
    \item Apply chelated iron (EDDHA form) as foliar spray for quick response
    \item Acidify soil with sulfur for long-term correction
    \item Review water source and amend if necessary
  \end{enumerate}
  
  \item \textbf{Pruning:} Remove crossing branches, dead wood, and oldest canes (gray bark). Best done in late winter while dormant.
\end{itemize}

%-----------------------------------------------------------------------------
\subsection{Dwarf Citrus (Optional)}
%-----------------------------------------------------------------------------

Container citrus is achievable in Zone 8a but requires commitment to cold protection and year-round care.

\subsubsection{Variety Selection}

Select only dwarf or semi-dwarf varieties grafted onto dwarfing rootstock.

\begin{itemize}
  \item \textbf{Most cold-hardy (best for 8a):} Satsuma mandarin, Kumquat, Meyer lemon
  \item \textbf{Moderately hardy:} Calamondin, Limequat, Improved Meyer lemon
  \item \textbf{Cold-sensitive (requires more protection):} True lemons, limes, grapefruit, oranges
\end{itemize}

\subsubsection{Container and Media Requirements}

\begin{itemize}
  \item \textbf{Container size:} Start in 10--15 gallon; mature trees need 20--25 gallon. Use wheeled caddies for all citrus containers.
  
  \item \textbf{Media:} Well-draining mix with extra perlite (25--30\%). Commercial citrus or cactus mix works well. pH 6.0--7.0.
  
  \item \textbf{Positioning:} Place in your absolute brightest location---citrus needs maximum sun for fruit production.
\end{itemize}

\subsubsection{Operational Guidance}

\begin{itemize}
  \item \textbf{Watering:} Allow top 2 inches to dry between waterings. Reduce significantly in winter (every 2--3 weeks). Overwatering is the most common cause of citrus decline.
  
  \item \textbf{Fertilization:} Use citrus-specific fertilizer with micronutrients (iron, zinc, manganese). Apply monthly during growing season (March--October). Suspend feeding November--February.
  
  \item \textbf{Cold protection protocol (critical):}
  \begin{enumerate}
    \item Monitor forecasts starting in October
    \item Move containers to sheltered location when temps drop below 35°F
    \item Below 28°F: Move indoors to bright, cool location (50--55°F ideal)
    \item If indoor light is limited, reduce watering further to induce semi-dormancy
    \item Gradually reacclimate to outdoor conditions in spring (harden off)
  \end{enumerate}
  
  \item \textbf{Indoor overwintering:} Place near brightest window. Keep away from heating vents. Maintain humidity with pebble tray or humidifier. Watch for spider mites (common indoors).
\end{itemize}

%-----------------------------------------------------------------------------
\subsection{Container Cane Berries (Raspberries/Blackberries)}
%-----------------------------------------------------------------------------

Compact cane berry varieties are well-suited to container culture.

\subsubsection{Variety Selection}

\begin{itemize}
  \item \textbf{Raspberries:} Raspberry Shortcake (thornless, compact), BrazelBerries Baby Cakes, Heritage (everbearing)
  \item \textbf{Blackberries:} BrazelBerries Baby Cakes Blackberry, Prime-Ark Freedom (primocane), Apache (thornless)
\end{itemize}

\subsubsection{Container and Media Requirements}

\begin{itemize}
  \item \textbf{Container size:} Minimum 10 gallon; 15--20 gallon preferred for long-term culture.
  \item \textbf{Media:} Standard potting mix with good drainage. pH 6.0--6.5.
  \item \textbf{Support:} Even compact varieties benefit from a small stake or support ring.
\end{itemize}

\subsubsection{Operational Guidance}

\begin{itemize}
  \item \textbf{Pruning (floricane/summer-bearing):} After harvest, cut fruited canes to ground. Leave new green canes for next year's fruit.
  
  \item \textbf{Pruning (primocane/everbearing):} Can cut all canes to ground in late winter for one fall crop, or leave some canes for early summer + fall crops.
  
  \item \textbf{Watering:} Consistent moisture during fruiting is critical. Mulch heavily.
  
  \item \textbf{Fertilization:} Apply balanced fertilizer in early spring and after first harvest flush.
\end{itemize}

%=============================================================================
% SECTION 2: VEGETABLES FOR SMALL SPACES
%=============================================================================
\section{Vegetables for Small Spaces (Vertical and Potting)}
\label{sec:vegetables}

Container vegetables require careful attention to variety selection, container sizing, and consistent care for optimal production.

%-----------------------------------------------------------------------------
\subsection{Tomatoes}
%-----------------------------------------------------------------------------

Tomatoes are the flagship container crop---highly productive when properly managed.

\subsubsection{Variety Selection}

\begin{longtable}{@{}L{2.5cm}L{4cm}L{7cm}@{}}
\toprule
\rowcolor{tableheader}
\textcolor{white}{\textbf{Type}} & \textcolor{white}{\textbf{Recommended Varieties}} & \textcolor{white}{\textbf{Container Notes}} \\
\midrule
\endfirsthead
\rowcolor{tablerow1}
Cherry/Grape & Sun Gold, Sweet 100, Juliet, Terenzo & Most productive for containers; continuous harvest; 5--10 gal \\
\rowcolor{tablerow2}
Determinate & Roma, Celebrity, Bush Early Girl & Compact growth; single harvest flush; 5--7 gal minimum \\
\rowcolor{tablerow1}
Indeterminate & Better Boy, Cherokee Purple, Brandywine & Require large containers (10+ gal) and strong support; season-long harvest \\
\rowcolor{tablerow2}
Patio/Dwarf & Tiny Tim, Tumbling Tom, Patio Princess & Smallest spaces; hanging baskets; 3--5 gal \\
\bottomrule
\end{longtable}

\subsubsection{Container and Media Requirements}

\begin{itemize}
  \item \textbf{Minimum container sizes:}
  \begin{itemize}
    \item Determinate/compact: 5--7 gallon
    \item Indeterminate: 10--15 gallon (minimum 10)
    \item Cherry/grape: 5--10 gallon depending on variety vigor
  \end{itemize}
  
  \item \textbf{Media:} Rich potting mix with 20\% compost. pH 6.2--6.8. Mix slow-release fertilizer into media at planting.
  
  \item \textbf{Support:} Install cage or stake at planting time to avoid root disturbance later.
\end{itemize}

\subsubsection{Operational Guidance}

\begin{itemize}
  \item \textbf{Planting depth:} Bury stem up to first set of true leaves---buried stem develops additional roots.
  
  \item \textbf{Watering:} Deep, consistent watering. Irregular watering causes blossom end rot and fruit cracking. Water in the morning at soil level.
  
  \item \textbf{Fertilization:}
  \begin{itemize}
    \item Slow-release at planting
    \item Liquid feed weekly once flowering begins (low-nitrogen, high-potassium)
    \item Calcium supplement if blossom end rot appears
  \end{itemize}
  
  \item \textbf{Pruning (indeterminate):}
  \begin{itemize}
    \item Remove suckers below first flower cluster
    \item Limit to 3--4 main stems for container culture
    \item Remove lower leaves as they yellow
    \item Top plants 4 weeks before first frost to redirect energy to ripening fruit
  \end{itemize}
  
  \item \textbf{Common issues:}
  \begin{itemize}
    \item \textbf{Blossom end rot:} Calcium deficiency from inconsistent watering
    \item \textbf{Cracking:} Heavy rain or irregular watering after dry period
    \item \textbf{Blossom drop:} Temperatures above 90°F or below 55°F at night
  \end{itemize}
\end{itemize}

%-----------------------------------------------------------------------------
\subsection{Peppers}
%-----------------------------------------------------------------------------

Peppers are excellent container crops---compact, productive, and heat-tolerant.

\subsubsection{Variety Selection}

\begin{itemize}
  \item \textbf{Sweet/Bell:} California Wonder, Gypsy, Lunchbox, Mini Bell mix
  \item \textbf{Hot:} Jalapeño, Serrano, Thai Dragon, Cayenne, Habanero
  \item \textbf{Ornamental/Edible:} NuMex Twilight, Explosive Ember, Black Pearl
\end{itemize}

\subsubsection{Container and Media Requirements}

\begin{itemize}
  \item \textbf{Container size:} 5-gallon minimum; 7--10 gallon for larger sweet pepper varieties.
  \item \textbf{Media:} Standard potting mix. pH 6.0--6.8. Peppers prefer slightly lower fertility than tomatoes.
\end{itemize}

\subsubsection{Operational Guidance}

\begin{itemize}
  \item \textbf{Temperature:} Peppers are heat-lovers. Don't transplant until soil is consistently warm (60°F+). They thrive in summer heat.
  
  \item \textbf{Watering:} Even moisture. Less prone to blossom end rot than tomatoes but still need consistency.
  
  \item \textbf{Fertilization:} Moderate feeder. Excess nitrogen produces foliage at expense of fruit. Use balanced or low-nitrogen fertilizer.
  
  \item \textbf{Staking:} Most peppers benefit from a single stake or small cage. Heavy fruit loads can topple plants.
  
  \item \textbf{Harvest timing:}
  \begin{itemize}
    \item Green peppers: harvest any time after reaching full size
    \item Colored peppers: leave on plant until fully colored (adds 2--3 weeks)
    \item Hot peppers: generally hotter when fully ripe
  \end{itemize}
\end{itemize}

%-----------------------------------------------------------------------------
\subsection{Cucumbers}
%-----------------------------------------------------------------------------

Bush cucumber varieties are ideal for container culture; vining types require significant trellis support.

\subsubsection{Variety Selection}

\begin{itemize}
  \item \textbf{Bush types (recommended):} Bush Champion, Spacemaster, Salad Bush, Picklebush
  \item \textbf{Compact vining:} Patio Snacker, Bush Slicer
  \item \textbf{Vining (with trellis):} Marketmore, Armenian, Lemon cucumber
\end{itemize}

\subsubsection{Container and Media Requirements}

\begin{itemize}
  \item \textbf{Container size:} 5--7 gallon minimum per plant. Cucumbers have large root systems.
  \item \textbf{Media:} Rich, moisture-retentive mix with compost. pH 6.0--6.8.
  \item \textbf{Trellis:} Install at planting. Even bush types benefit from support to keep fruit off soil.
\end{itemize}

\subsubsection{Operational Guidance}

\begin{itemize}
  \item \textbf{Direct sow or transplant:} Cucumbers dislike root disturbance. Either direct sow or transplant very carefully from peat pots.
  
  \item \textbf{Watering:} Heavy water users, especially during fruiting. Inconsistent water causes bitter fruit. Mulch heavily.
  
  \item \textbf{Fertilization:} Heavy feeders. Weekly liquid fertilizer during fruiting.
  
  \item \textbf{Harvest:} Pick frequently when fruits reach harvest size. Overripe cucumbers turn bitter and signal the plant to stop producing.
  
  \item \textbf{Succession planting:} Cucumbers decline after 4--6 weeks of heavy production. Plant a second crop in midsummer for fall harvest.
\end{itemize}

%-----------------------------------------------------------------------------
\subsection{Beans (Bush Type)}
%-----------------------------------------------------------------------------

Bush beans are low-maintenance, productive, and fix their own nitrogen.

\subsubsection{Variety Selection}

\begin{itemize}
  \item \textbf{Green snap beans:} Provider, Contender, Bush Blue Lake
  \item \textbf{Yellow wax beans:} Gold Rush, Rocdor
  \item \textbf{Purple beans:} Royal Burgundy (turns green when cooked)
  \item \textbf{Dry beans:} Black beans, pinto, navy (require longer season)
\end{itemize}

\subsubsection{Container and Media Requirements}

\begin{itemize}
  \item \textbf{Container size:} 5--7 gallon for 4--6 plants; can use longer window boxes
  \item \textbf{Spacing:} 4--6 inches between plants
  \item \textbf{Media:} Standard potting mix. Do not add nitrogen fertilizer at planting.
\end{itemize}

\subsubsection{Operational Guidance}

\begin{itemize}
  \item \textbf{Direct sow only:} Beans do not transplant well. Sow seeds 1 inch deep after last frost.
  
  \item \textbf{Fertilization:} As legumes, beans fix atmospheric nitrogen. Do not fertilize with nitrogen---it reduces production.
  
  \item \textbf{Watering:} Even moisture during flowering and pod set is critical.
  
  \item \textbf{Harvest:} Pick every 2--3 days when pods are pencil-thick and seeds are just barely visible. Regular picking extends harvest.
  
  \item \textbf{Succession planting:} Sow every 2--3 weeks through early August for continuous harvest.
\end{itemize}

%-----------------------------------------------------------------------------
\subsection{Swiss Chard and Leafy Greens}
%-----------------------------------------------------------------------------

Greens are the highest-return crops for container gardens---productive, fast-growing, and cut-and-come-again.

\subsubsection{Variety Selection}

\begin{itemize}
  \item \textbf{Swiss chard:} Bright Lights (multicolor), Fordhook Giant, Ruby Red
  \item \textbf{Lettuce:} Salad Bowl, Buttercrunch, Red Sails, mesclun mixes
  \item \textbf{Spinach:} Bloomsdale, Space, Tyee (bolt-resistant)
  \item \textbf{Asian greens:} Bok choy, tatsoi, mizuna, komatsuna
\end{itemize}

\subsubsection{Container and Media Requirements}

\begin{itemize}
  \item \textbf{Container size:} 3--5 gallon; wider is better than deep for greens
  \item \textbf{Spacing:} 4--6 inches for chard; 2--4 inches for baby greens
  \item \textbf{Media:} Standard potting mix with compost. pH 6.0--7.0.
\end{itemize}

\subsubsection{Operational Guidance}

\begin{itemize}
  \item \textbf{Planting windows:} Cool-season crops. Plant early spring (4--6 weeks before last frost) and late summer (6--8 weeks before first frost). Avoid midsummer heat.
  
  \item \textbf{Watering:} Consistent moisture. Stressed greens become bitter and bolt quickly.
  
  \item \textbf{Harvest technique:} Cut outer leaves first, leaving center to continue growing. This ``cut-and-come-again'' method extends harvest by weeks.
  
  \item \textbf{Shade tolerance:} Greens can produce in 4--6 hours of sun---ideal for shadier balcony spots.
  
  \item \textbf{Bolt prevention:} When temps exceed 80°F, provide afternoon shade. Bolt-resistant varieties extend summer harvest.
\end{itemize}

%-----------------------------------------------------------------------------
\subsection{Potatoes}
%-----------------------------------------------------------------------------

Grow bags are ideal for potatoes, and the harvest is exciting.

\subsubsection{Variety Selection}

\begin{itemize}
  \item \textbf{Early (fastest):} Yukon Gold, Red Norland (65--80 days)
  \item \textbf{Mid-season:} Kennebec, Purple Majesty (80--100 days)
  \item \textbf{Fingerling:} French Fingerling, Russian Banana
\end{itemize}

\subsubsection{Container and Media Requirements}

\begin{itemize}
  \item \textbf{Container:} 10--15 gallon grow bags work best (roll-down sides for ``hilling'')
  \item \textbf{Media:} Light, loose mix---50\% potting mix, 50\% compost
  \item \textbf{Seed potatoes:} Use certified disease-free seed potatoes, not grocery store potatoes
\end{itemize}

\subsubsection{Operational Guidance}

\begin{itemize}
  \item \textbf{Planting:} Fill container 1/3 full. Place 3--4 seed potato pieces (2+ eyes each) on surface. Cover with 4 inches of mix.
  
  \item \textbf{Hilling:} As plants grow, add more mix until container is full. This ``hilling'' encourages more tubers to form.
  
  \item \textbf{Watering:} Even moisture during tuber formation. Reduce watering when tops begin to yellow and die back.
  
  \item \textbf{Harvest:}
  \begin{itemize}
    \item ``New'' potatoes: Harvest a few when plants flower
    \item Full harvest: Wait 2 weeks after tops completely die back
    \item Dump container and sort through for tubers
  \end{itemize}
\end{itemize}

%=============================================================================
% SECTION 3: HERBS FOR CULINARY USE
%=============================================================================
\section{Herbs for Culinary Use and Health}
\label{sec:herbs}

Herbs deliver the highest return per square foot of any container crop. They require minimal space, provide continuous harvest, and dramatically expand culinary options.

%-----------------------------------------------------------------------------
\subsection{Essential Culinary Herbs}
%-----------------------------------------------------------------------------

\begin{longtable}{@{}L{2.5cm}L{2.5cm}L{2.5cm}L{6cm}@{}}
\toprule
\rowcolor{tableheader}
\textcolor{white}{\textbf{Herb}} & \textcolor{white}{\textbf{Container}} & \textcolor{white}{\textbf{Water Needs}} & \textcolor{white}{\textbf{Growing Notes}} \\
\midrule
\endfirsthead
\rowcolor{tablerow1}
Basil & 3--5 gal & High & Pinch flowers; frost-sensitive; succession plant \\
\rowcolor{tablerow2}
Rosemary & 3--5 gal & Low & Perennial; needs drainage; overwintering possible \\
\rowcolor{tablerow1}
Thyme & 2--3 gal & Low & Perennial; drought-tolerant; trim after flowering \\
\rowcolor{tablerow2}
Parsley & 3--5 gal & Medium & Biennial (harvest first year); slow germination \\
\rowcolor{tablerow1}
Cilantro & 3--5 gal & Medium & Bolts in heat; succession plant every 2--3 weeks \\
\rowcolor{tablerow2}
Chives & 3--5 gal & Medium & Perennial; divide every 2--3 years; edible flowers \\
\rowcolor{tablerow1}
Mint & 3--5 gal \textbf{alone} & High & Aggressive spreader---always containerize separately \\
\rowcolor{tablerow2}
Sage & 3--5 gal & Low & Perennial; woody over time; full sun \\
\rowcolor{tablerow1}
Oregano & 2--3 gal & Low & Perennial; spreads; harvest before flowering \\
\rowcolor{tablerow2}
Dill & 5--7 gal & Medium & Deep taproot; supports swallowtail butterflies \\
\bottomrule
\end{longtable}

%-----------------------------------------------------------------------------
\subsection{Herb Groupings by Water Needs}
%-----------------------------------------------------------------------------

Group herbs with similar water requirements in the same container for easier management:

\textbf{Dry/Mediterranean Group:} Rosemary, thyme, oregano, sage, lavender

\textbf{Moderate Moisture Group:} Parsley, chives, cilantro, dill

\textbf{High Moisture Group:} Basil, mint (keep separate)

%-----------------------------------------------------------------------------
\subsection{Operational Guidance}
%-----------------------------------------------------------------------------

\begin{itemize}
  \item \textbf{Harvest frequently:} Regular harvesting (pinching) promotes bushy growth and prevents flowering. Once herbs flower, leaf production slows and flavor diminishes.
  
  \item \textbf{Morning harvest:} Essential oils are highest in the morning after dew dries. Harvest before midday heat.
  
  \item \textbf{Pinching technique:} Cut just above a leaf node to encourage branching. Never remove more than 1/3 of the plant at once.
  
  \item \textbf{Perennial care:} Rosemary, thyme, oregano, sage, and chives are perennial in Zone 8a. Protect from extreme cold; may need mulching or relocation during freezes.
  
  \item \textbf{Annual succession:} Basil and cilantro bolt in heat. Plant successions every 2--3 weeks for continuous harvest.
\end{itemize}

\begin{warningbox}
\textbf{Mint Warning:} Never plant mint in shared containers---it will overtake all other plants within one season. Always grow mint in its own isolated pot.
\end{warningbox}

%=============================================================================
% SECTION 4: SPICES FOR BALCONY GROWING
%=============================================================================
\section{``Spices'' You Can Realistically Grow on a Balcony}
\label{sec:spices}

True spices (cinnamon, nutmeg, black pepper) are tropical tree products that require greenhouse conditions. However, several spice-adjacent crops thrive in Zone 8a container gardens.

%-----------------------------------------------------------------------------
\subsection{Recommended Options}
%-----------------------------------------------------------------------------

\begin{itemize}
  \item \textbf{Garlic Greens:} Plant individual cloves 1 inch deep in any container. Harvest greens repeatedly (like scallions). Will not produce bulbs without proper chilling and space, but greens have full garlic flavor.
  
  \item \textbf{Oregano (dried):} Excellent when dried. Harvest before flowering, hang in bundles, store in airtight containers. Flavor intensifies when dried.
  
  \item \textbf{Thyme/Rosemary/Sage (dried):} All dry well and store for months. Essential for seasoning blends.
  
  \item \textbf{Lemon Balm:} Citrusy herb that dries well for tea. Prolific grower---harvest frequently.
  
  \item \textbf{Hot Peppers (dried/flaked):} Cayenne, Thai, and other thin-walled peppers dry easily. String on thread and hang, or use dehydrator.
  
  \item \textbf{Chives (freeze-dried):} Snip and freeze for year-round use. Maintains color and flavor better than drying.
\end{itemize}

%-----------------------------------------------------------------------------
\subsection{Drying and Preservation}
%-----------------------------------------------------------------------------

\textbf{Air Drying:}
\begin{enumerate}
  \item Harvest mid-morning after dew dries
  \item Tie stems in small bundles (6--10 stems)
  \item Hang upside-down in warm, dry, well-ventilated area
  \item Protect from direct sunlight (fades color)
  \item Drying complete when leaves crumble easily (1--2 weeks)
\end{enumerate}

\textbf{Dehydrator Method:}
\begin{enumerate}
  \item Spread leaves in single layer on trays
  \item Set temperature to 95--105°F (lowest setting)
  \item Check every few hours; remove when crisp
  \item Store in airtight containers away from light
\end{enumerate}

%=============================================================================
% SECTION 5: RECOMMENDED BALCONY STRATEGY
%=============================================================================
\section{Recommended Balcony Strategy (Container-First, Yield-First)}
\label{sec:strategy}

This section provides a complete operational framework for your container garden, organized by seasonal timing and practical workflow.

%-----------------------------------------------------------------------------
\subsection{Seasonal Calendar for Lithia Springs, GA (Zone 8a)}
%-----------------------------------------------------------------------------

\subsubsection{Late January--February: Cool-Season Start and Seed Starting}

\textbf{Outdoor Direct Sowing (as conditions permit):}
\begin{itemize}
  \item Radishes
  \item Lettuce (under row cover if temps drop)
  \item Spinach
  \item Peas (mid-February onward)
\end{itemize}

\textbf{Indoor Seed Starting:}
\begin{itemize}
  \item Peppers: Start 8--10 weeks before last frost (late January--early February)
  \item Tomatoes: Start 6--8 weeks before last frost (mid-February--early March)
  \item Herbs: Basil, parsley, cilantro (6--8 weeks before transplant)
\end{itemize}

\textbf{Tasks:}
\begin{itemize}
  \item Complete Pre-Season Readiness Review (\S\ref{sec:readiness})
  \item Order seeds and plants
  \item Clean and prepare containers
  \item Pre-wet and stage potting mix
\end{itemize}

\subsubsection{March: Transition Month}

\textbf{Outdoor Activities:}
\begin{itemize}
  \item Continue cool-season crops
  \item Install trellises and cages
  \item Begin hardening off seedlings (see hardening protocol)
  \item Plant strawberries (bare root or transplants)
  \item Plant blueberries (container nursery stock)
\end{itemize}

\textbf{Tasks:}
\begin{itemize}
  \item Refresh mulch on all containers
  \item Top-dress with compost
  \item Install drip irrigation if using
  \item Final pest barrier setup (copper tape, netting)
\end{itemize}

\subsubsection{Early--Mid April: Main Warm-Season Planting Window}

\textbf{Plant/Transplant After Last Frost (around April 1--15):}
\begin{itemize}
  \item Tomatoes
  \item Peppers
  \item Cucumbers (late April if direct sowing)
  \item Beans (direct sow)
  \item Summer herbs (basil, etc.)
\end{itemize}

\textbf{Tasks:}
\begin{itemize}
  \item Keep frost cloth ready for late cold snaps
  \item Begin regular watering schedule
  \item Start liquid feeding as plants establish
\end{itemize}

\subsubsection{May--August: Peak Production Season}

\textbf{Ongoing Activities:}
\begin{itemize}
  \item Daily moisture monitoring (critical)
  \item Weekly liquid fertilization for fruiting crops
  \item Regular harvest (every 1--3 days for beans, cucumbers, cherry tomatoes)
  \item Pruning and training (tomatoes, cucumbers)
  \item Pest and disease monitoring
\end{itemize}

\textbf{Succession Planting:}
\begin{itemize}
  \item Beans: Sow every 2--3 weeks through early August
  \item Cucumbers: Plant second crop in June for fall harvest
  \item Basil/Cilantro: Sow every 2--3 weeks
\end{itemize}

\textbf{Special Considerations:}
\begin{itemize}
  \item June--August: Highest pest and disease pressure
  \item July--August: Heat stress---afternoon shade may help greens
  \item Water 1--2 times daily during heat waves
\end{itemize}

\subsubsection{Late August--October: Fall Transition}

\textbf{Fall Planting (6--8 weeks before first frost):}
\begin{itemize}
  \item Lettuce, spinach, chard
  \item Radishes
  \item Asian greens
  \item Garlic cloves (for greens or overwintering)
\end{itemize}

\textbf{Tasks:}
\begin{itemize}
  \item Clean up declining summer crops
  \item Refresh potting mix for fall containers
  \item Reduce fertilization on perennials as growth slows
  \item Prepare cold protection materials
\end{itemize}

\subsubsection{November--January: Dormancy and Maintenance}

\textbf{Activities:}
\begin{itemize}
  \item Harvest cold-hardy greens (chard, kale, lettuce under cover)
  \item Protect tender plants during freezes
  \item Move citrus indoors during cold spells
  \item Reduce watering significantly
  \item Plan next year's garden
\end{itemize}

%-----------------------------------------------------------------------------
\subsection{Ground-Level Balcony Specific Strategies}
%-----------------------------------------------------------------------------

\textbf{Managing Reduced Sunlight:}
\begin{itemize}
  \item Prioritize sunniest spots for tomatoes, peppers, and citrus
  \item Use reflective mulch and light-colored surfaces to increase available light
  \item Accept that partially shaded locations (4--6 hours sun) will have reduced fruit yields
  \item Maximize shade-tolerant crops: greens, chard, herbs, radishes
\end{itemize}

\textbf{Managing Increased Pest Pressure:}
\begin{itemize}
  \item Elevate ALL containers on risers (non-negotiable)
  \item Keep area beneath containers clean and dry
  \item Apply copper tape barriers to most vulnerable crops
  \item Use sticky traps for ongoing monitoring
  \item Scout daily---early intervention prevents infestations
\end{itemize}

\textbf{Managing Higher Humidity:}
\begin{itemize}
  \item Space containers adequately (12+ inches apart)
  \item Prune lower foliage on tomatoes and peppers
  \item Water in the morning so foliage dries before evening
  \item Remove and dispose of any diseased plant material immediately
  \item Choose disease-resistant varieties when available
\end{itemize}

%-----------------------------------------------------------------------------
\subsection{Operating Cadence (Weekly Rhythm)}
%-----------------------------------------------------------------------------

Establish a consistent routine to prevent problems and maximize production:

\textbf{Daily (Summer/Peak Season):}
\begin{itemize}
  \item Morning moisture check (finger test or moisture meter)
  \item Water as needed (morning preferred)
  \item Quick pest and disease scan (5 minutes)
  \item Harvest herbs and greens as needed
  \item Harvest ripe fruits and vegetables
\end{itemize}

\textbf{Weekly:}
\begin{itemize}
  \item Liquid fertilizer application to heavy feeders (tomatoes, peppers, cucumbers)
  \item Train and tie vines
  \item Remove suckers from tomatoes
  \item Check trellis stability
  \item Remove damaged, yellowed, or diseased leaves
  \item Renew sticky traps if needed
\end{itemize}

\textbf{Biweekly:}
\begin{itemize}
  \item Top-dress containers with compost or worm castings
  \item Refresh mulch as needed
  \item Check and adjust drip irrigation emitters
  \item Deep-clean trays and saucers if algae or mineral buildup
\end{itemize}

\textbf{Monthly:}
\begin{itemize}
  \item Review overall crop performance
  \item Replace underperforming or declining plants
  \item pH check for blueberries
  \item Reapply slow-release fertilizer if using
  \item Assess pest populations and adjust strategy
\end{itemize}

%=============================================================================
% SECTION 6: PLANT DIVERSITY TARGET
%=============================================================================
\section{``30 Plants per Week'' Made Operational}
\label{sec:diversity}

The ``30 plant foods per week'' concept comes from gut microbiome research suggesting diverse plant consumption supports beneficial gut bacteria. This is a \textbf{diversity target}: count unique plant foods consumed weekly (not servings or portions).

%-----------------------------------------------------------------------------
\subsection{Understanding the Target}
%-----------------------------------------------------------------------------

\begin{itemize}
  \item Each unique plant food counts \textbf{once per week} regardless of how often consumed
  \item Herbs count: basil, thyme, rosemary, parsley are 4 separate plants
  \item Different colors of the same vegetable count separately: red pepper, green pepper = 2 plants
  \item Fermented plants count: sauerkraut adds to diversity
\end{itemize}

%-----------------------------------------------------------------------------
\subsection{Execution Method}
%-----------------------------------------------------------------------------

\begin{enumerate}
  \item \textbf{Create a weekly tracking sheet:} Simple checkbox list of 30+ plant options
  
  \item \textbf{Maximize herb contribution:} This is where container gardens excel. Using 4--5 different herbs daily can contribute 15--20+ unique plants weekly with minimal space.
  
  \item \textbf{Rotate intentionally:} Don't use the same 3 herbs every day. Rotate through your full collection.
  
  \item \textbf{Add ``micro diversity'':} Green onion tops, chive flowers, garlic greens---small additions that count as unique plants.
\end{enumerate}

%-----------------------------------------------------------------------------
\subsection{Sample Weekly Diversity Plan}
%-----------------------------------------------------------------------------

\textbf{From Your Container Garden (potential 15--20 plants):}
\begin{itemize}
  \item Herbs: Basil, rosemary, thyme, oregano, parsley, cilantro, chives, mint, sage (9)
  \item Greens: Swiss chard, lettuce varieties (2--3)
  \item Vegetables: Tomatoes, peppers, cucumbers, beans (4)
  \item Fruits: Strawberries, blueberries (2)
\end{itemize}

\textbf{Supplement from Market (potential 10--15 plants):}
\begin{itemize}
  \item Additional vegetables: Onions, garlic, carrots, broccoli, etc.
  \item Fruits: Apples, bananas, berries not grown
  \item Whole grains: Oats, rice, quinoa (these count)
  \item Nuts and seeds: Almonds, sunflower seeds, flax
\end{itemize}

\textbf{Weekly Anchor Structure:}
\begin{itemize}
  \item Daily staples: 2--4 different herbs (rotate), 1 leafy green, 1 fruit or vegetable
  \item Weekly anchors: Tomatoes, peppers, cucumbers (when in season)
  \item Seasonal specials: Berries, citrus (when available)
\end{itemize}

%=============================================================================
% APPENDICES
%=============================================================================
\appendix
\newpage
\section{Appendix A: Quick Reference Tables}
\label{app:tables}

\subsection{Container Sizing Quick Reference}

\begin{center}
\begin{tabularx}{\textwidth}{@{}L{3.5cm}C{2.5cm}X@{}}
\toprule
\rowcolor{tableheader}
\textcolor{white}{\textbf{Crop}} & \textcolor{white}{\textbf{Min. Size (gal)}} & \textcolor{white}{\textbf{Notes}} \\
\midrule
Tomatoes (indet.) & 10--15 & Heavy-duty cage/support required \\
Tomatoes (det.) & 5--7 & Compact; single harvest flush \\
Peppers & 5 & 7--10 gal for larger bell types \\
Cucumbers & 5--7 & Trellis required \\
Beans (bush) & 5--7 & For 4--6 plants \\
Leafy greens/Chard & 3--5 & Wider containers preferred \\
Herbs (most) & 2--3 & Group by water needs \\
Potatoes & 10--15 & Grow bags ideal \\
Strawberries & 3--5 & Or hanging baskets \\
Blueberries & 15--20 & Acidic media essential \\
Citrus (dwarf) & 15--25 & Wheeled caddy for mobility \\
\bottomrule
\end{tabularx}
\end{center}

\subsection{Seasonal Action Calendar (One-Page Summary)}

\begin{center}
\begin{tabularx}{\textwidth}{@{}L{3.5cm}X@{}}
\toprule
\rowcolor{tableheader}
\textcolor{white}{\textbf{Period}} & \textcolor{white}{\textbf{Primary Actions}} \\
\midrule
Late Winter (Jan--Feb) & Sun audit; prep containers/media; start seeds indoors; plant cool-season greens; setup pest controls \\
\midrule
Early Spring (Mar) & Install trellises; harden off transplants; plant strawberries/blueberries; top-dress compost \\
\midrule
Late Spring (Apr) & Transplant warm-season crops after last frost; begin regular watering; have frost cloth ready \\
\midrule
Early Summer (May--Jun) & Daily watering; weekly feeding; succession sow beans/cucumbers; monitor pests closely \\
\midrule
Peak Summer (Jul--Aug) & Water 1--2x daily; prune/train tomatoes; manage heat stress; control disease \\
\midrule
Late Summer (Aug--Sep) & Plant fall greens; reduce fertilizer on perennials; clean up declining plants \\
\midrule
Fall (Oct--Nov) & Harvest fall crops; protect from early frost; prep for winter; plan next year \\
\midrule
Winter (Dec--Jan) & Minimal watering; protect perennials; move citrus during freezes; order seeds \\
\bottomrule
\end{tabularx}
\end{center}

\subsection{Watering Quick Reference}

\begin{center}
\begin{tabularx}{\textwidth}{@{}L{3cm}L{3cm}X@{}}
\toprule
\rowcolor{tableheader}
\textcolor{white}{\textbf{Crop Category}} & \textcolor{white}{\textbf{Water Needs}} & \textcolor{white}{\textbf{Notes}} \\
\midrule
Tomatoes & High, consistent & Inconsistency causes BER and cracking \\
Peppers & Moderate, consistent & More tolerant than tomatoes \\
Cucumbers & High & Mulch heavily; bitter fruit if stressed \\
Beans & Moderate & Critical during flowering/pod set \\
Greens/Chard & Moderate, consistent & Stress causes bolting and bitterness \\
Blueberries & High, consistent & Never let dry out completely \\
Citrus & Moderate & Allow partial dry-down between waterings \\
Mediterranean herbs & Low & Rosemary, thyme, sage prefer dry \\
Basil/Parsley & Moderate--High & Keep evenly moist \\
\bottomrule
\end{tabularx}
\end{center}

\newpage
\section{Appendix B: Troubleshooting Guide}
\label{app:troubleshooting}

\subsection{Common Problems and Solutions}

\begin{longtable}{@{}L{3cm}L{4cm}L{6.5cm}@{}}
\toprule
\rowcolor{tableheader}
\textcolor{white}{\textbf{Problem}} & \textcolor{white}{\textbf{Likely Cause}} & \textcolor{white}{\textbf{Solution}} \\
\midrule
\endfirsthead
\rowcolor{tablerow1}
Yellowing leaves (overall) & Nitrogen deficiency or overwatering & Check drainage; apply nitrogen fertilizer if soil is properly moist \\
\rowcolor{tablerow2}
Yellowing between veins & Iron deficiency (esp. blueberries) & Check pH; apply chelated iron; acidify if needed \\
\rowcolor{tablerow1}
Blossom end rot (tomatoes/peppers) & Calcium deficiency from inconsistent water & Stabilize watering; add calcium source; mulch \\
\rowcolor{tablerow2}
Blossom drop & Temperature extremes (>90°F or <55°F night) & Provide afternoon shade; wait for temps to moderate \\
\rowcolor{tablerow1}
Bitter cucumbers & Water stress & Increase watering; mulch heavily; harvest younger \\
\rowcolor{tablerow2}
Bolting (greens, cilantro) & Heat stress or root-bound & Provide shade; plant bolt-resistant varieties; succession plant \\
\rowcolor{tablerow1}
Leggy seedlings & Insufficient light & Move closer to light; reduce temps; increase air circulation \\
\rowcolor{tablerow2}
Stunted growth & Root-bound or nutrient deficiency & Check root ball; up-pot if needed; fertilize \\
\rowcolor{tablerow1}
White powdery coating & Powdery mildew & Improve airflow; neem oil spray; remove affected leaves \\
\rowcolor{tablerow2}
Sticky residue on leaves & Aphids or scale & Insecticidal soap; check undersides of leaves; blast with water \\
\rowcolor{tablerow1}
Holes in leaves & Caterpillars or slugs & Bt spray for caterpillars; iron phosphate bait for slugs \\
\rowcolor{tablerow2}
Wilting despite moist soil & Root rot from overwatering & Improve drainage; reduce watering; may need fresh media \\
\bottomrule
\end{longtable}

\newpage
\section{Appendix C: Companion Planting for Containers}
\label{app:companion}

While companion planting principles are primarily designed for in-ground gardens, some concepts apply to containers:

\subsection{Beneficial Combinations}

\begin{itemize}
  \item \textbf{Tomatoes + Basil:} Classic pairing. Basil may repel some tomato pests and both have similar water needs.
  
  \item \textbf{Peppers + Parsley:} Parsley attracts beneficial insects; both thrive in similar conditions.
  
  \item \textbf{Cucumbers + Dill:} Dill attracts beneficial predatory wasps; harvest dill before cucumbers flower.
  
  \item \textbf{Greens + Chives:} Chives may deter aphids from lettuce and other greens.
\end{itemize}

\subsection{Combinations to Avoid}

\begin{itemize}
  \item \textbf{Tomatoes + Brassicas:} Both are heavy feeders; compete for nutrients in limited container space.
  
  \item \textbf{Fennel + Almost anything:} Fennel inhibits growth of most plants; always grow alone.
  
  \item \textbf{Mint + Anything:} Mint overtakes shared containers; always isolate.
\end{itemize}

\newpage
\section{Appendix D: Zone 8a Climate Data for Lithia Springs, GA}
\label{app:climate}

\begin{center}
\begin{tabularx}{\textwidth}{@{}L{2.5cm}C{1.8cm}C{1.8cm}C{1.8cm}X@{}}
\toprule
\rowcolor{tableheader}
\textcolor{white}{\textbf{Month}} & \textcolor{white}{\textbf{Avg High}} & \textcolor{white}{\textbf{Avg Low}} & \textcolor{white}{\textbf{Rainfall}} & \textcolor{white}{\textbf{Garden Notes}} \\
\midrule
January & 51°F & 33°F & 4.5 in & Seed ordering; cool-season greens under cover \\
February & 56°F & 36°F & 4.7 in & Start seeds indoors; prep containers \\
March & 64°F & 42°F & 4.6 in & Harden off; plant cool-season crops \\
April & 72°F & 50°F & 3.6 in & Main transplanting month \\
May & 80°F & 58°F & 3.7 in & All warm crops in; regular harvest begins \\
June & 86°F & 66°F & 3.9 in & Peak growth; increase watering \\
July & 89°F & 70°F & 5.0 in & Heat management; daily watering \\
August & 88°F & 69°F & 3.8 in & Heat continues; plant fall crops late month \\
September & 83°F & 63°F & 3.5 in & Fall planting; heat begins easing \\
October & 73°F & 52°F & 3.1 in & Prime fall harvest; frost watch late month \\
November & 63°F & 42°F & 3.5 in & First frost; protect tender plants \\
December & 54°F & 35°F & 4.0 in & Winter dormancy; minimal activity \\
\bottomrule
\end{tabularx}
\end{center}

\vspace{1em}
\textbf{Key Dates:}
\begin{itemize}
  \item \textbf{Average last spring frost:} March 15--April 1
  \item \textbf{Average first fall frost:} November 1--15
  \item \textbf{Growing season length:} 210--240 days
  \item \textbf{Annual precipitation:} 50--55 inches
  \item \textbf{USDA Hardiness Zone:} 8a (minimum temps 10--15°F)
\end{itemize}

%=============================================================================
% END OF DOCUMENT
%=============================================================================
\vfill
\begin{center}
\rule{0.5\textwidth}{0.4pt}\\[0.5em]
{\small\textit{Document prepared for container gardening in Lithia Springs, Georgia.}}\\
{\small\textit{Adapt recommendations based on your specific microclimate and conditions.}}
\end{center}

\end{document}
