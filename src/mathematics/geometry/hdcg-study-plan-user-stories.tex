
\documentclass[11pt,a4paper]{article}

\usepackage[T1]{fontenc}
\usepackage[utf8]{inputenc}
\usepackage{lmodern}
\usepackage{microtype}
\usepackage[a4paper,margin=1in]{geometry}
\usepackage{parskip}

\usepackage[table]{xcolor}
\usepackage{hyperref}
\hypersetup{colorlinks=true, linkcolor=blue!60!black, urlcolor=blue!60!black, citecolor=blue!60!black}
\usepackage{enumitem}
\usepackage{array}
\usepackage{tabularx}
\usepackage{booktabs}
\usepackage{amssymb}
\usepackage[most]{tcolorbox}
\tcbuselibrary{skins,breakable}
\usetikzlibrary{shadings}

\definecolor{CardBorder}{RGB}{210,215,221}
\definecolor{CardShadeL}{RGB}{0,153,204}
\definecolor{CardShadeR}{RGB}{0,102,153}
\definecolor{PillBG}{RGB}{238,242,245}
\definecolor{MetaText}{gray}{0.45}

\tcbset{
  pillstyle/.style={
    on line,
    boxsep=1pt, left=5pt, right=5pt, top=2pt, bottom=2pt,
    colback=PillBG, colframe=PillBG, boxrule=0pt,
    enhanced, arc=2mm
  }
}
\newcommand{\pill}[1]{\tcbox[pillstyle]{\sffamily\footnotesize #1}}

\newlist{checklist}{itemize}{1}
\setlist[checklist]{label=\(\square\), left=1.6em, labelsep=0.6em, itemsep=0.25em, topsep=0.3em}

\newtcolorbox{StoryCardBox}[1][]{%
  enhanced, breakable,
  colback=white, colframe=CardBorder, boxrule=0.5pt,
  arc=2mm,
  frame hidden=false,
  interior style={fill=white},
  left=10pt, right=10pt, top=8pt, bottom=8pt,
  overlay={%
    \path[shade, left color=CardShadeL, right color=CardShadeR, shading angle=0]
      ([xshift=-0.5pt]frame.north west) rectangle ([xshift=6pt]frame.south west);
  },
  borderline west={0.5pt}{0pt}{CardBorder},
}
\newcommand{\metaline}[1]{\par\small\color{MetaText}#1\par}
\newcommand{\storyrow}[2]{%
  \noindent\begin{tabularx}{\linewidth}{@{}>{\bfseries}l X@{}}#1 & #2\end{tabularx}\par
}
\newcommand{\DoR}{\textbf{Definition of Ready:} Persona clear; AC drafted; Dependencies known; Estimate set.}
\newcommand{\DoD}{\textbf{Definition of Done:} All ACs pass; Tests green; Security/a11y checks; Docs updated; Deployed/flagged.}
\newcommand{\StoryBody}[1]{\vspace{0.25em}{\itshape #1}\par}
\newcommand{\NonFunctional}[1]{\vspace{0.35em}\storyrow{Non-Functional}{#1}}
\newcommand{\BDD}[4]{%
  \vspace{0.35em}
  {\bfseries Acceptance Criteria (BDD)}\par
  \storyrow{Scenario}{#1}
  \storyrow{Given}{#2}
  \storyrow{When}{#3}
  \storyrow{Then}{#4}
}
\newcommand{\Meta}{\vspace{0.2em}\metaline{\DoR\enspace\textbullet\enspace \DoD}}

\newtcolorbox{TaskBox}{%
  enhanced, breakable,
  colback=white, colframe=CardBorder, boxrule=0.5pt,
  arc=2mm,
  left=10pt, right=10pt, top=8pt, bottom=8pt,
  overlay={%
    \path[shade, left color=CardShadeL, right color=CardShadeR, shading angle=0]
      ([xshift=-0.5pt]frame.north west) rectangle ([xshift=6pt]frame.south west);
  },
  borderline west={0.5pt}{0pt}{CardBorder}
}
\newenvironment{StoryTasks}{\begin{TaskBox}{\large\bfseries Tasks}\par\vspace{-0.25em}\begin{checklist}}{\end{checklist}\end{TaskBox}}

\renewcommand{\arraystretch}{1.12}



% ---- Added by patch: StoryCard environment (simple) ----
\newenvironment{StoryCard}[2]{%
  \begin{tcolorbox}[enhanced,breakable,arc=2mm,boxrule=0.7pt,left=2mm,right=2mm,top=1mm,bottom=1mm]
  \noindent\texttt{#1}\hfill\textbf{\large #2}\par\medskip
}{%
  \end{tcolorbox}
}

\begin{document}
\begin{center}
  {\huge \textbf{HDCG Study Plan --- Full User Stories}}\\[0.15em]
  {\large One story card per chapter (visual style matches attached examples).}
\end{center}
\vspace{0.6em}
\section*{HDCG-01 --- Finite Point Configurations}
\begin{StoryCard}{HDCG-01}{Finite Point Configurations}
\storyrow{Epic / Feature}{Combinatorial \& Discrete Geometry}
\storyrow{Business Value}{Establish theoretical tools and combinatorial principles that underpin later algorithms and applications. (Finite Point Configurations).}
\storyrow{Priority / Estimate}{\textbf{Priority:} Must\hfill \textbf{SP:} 3}
\storyrow{Persona}{research student of discrete geometry}
\storyrow{Dependencies}{---}
\storyrow{Assumptions / Risks}{time to internalize proofs vs. breadth}

\StoryBody{As a research student of discrete geometry, I want to master \textit{Finite Point Configurations} so that I can apply it to real problems and communicate theoretical insights clearly and reproducibly.}

\NonFunctional{\pill{Reliability} \pill{Reproducibility}}

\BDD{Happy path}{the chapter, examples, and any tooling are available}{I complete the \emph{Hands-on Objective} and validations for this chapter}{the stated outcomes are produced (proof/code/summary) and recorded in the repo with passing checks}

\Meta
\end{StoryCard}

\begin{StoryTasks}
  \item Extract key definitions/lemmas; compile a one-page summary with references to the chapter.
  \item Recreate classic extremal/incidence examples; compute bounds on small instances.
  \item Generate synthetic point sets (random, grids, clustered) and measure incidence properties.
  \item Write a short note contrasting combinatorial vs. metric phenomena observed.
\end{StoryTasks}

\clearpage
\section*{HDCG-02 --- Packing and Covering}
\begin{StoryCard}{HDCG-02}{Packing and Covering}
\storyrow{Epic / Feature}{Combinatorial \& Discrete Geometry}
\storyrow{Business Value}{Establish theoretical tools and combinatorial principles that underpin later algorithms and applications. (Packing and Covering).}
\storyrow{Priority / Estimate}{\textbf{Priority:} Must\hfill \textbf{SP:} 3}
\storyrow{Persona}{research student of discrete geometry}
\storyrow{Dependencies}{prior chapters as referenced}
\storyrow{Assumptions / Risks}{time to internalize proofs vs. breadth}

\StoryBody{As a research student of discrete geometry, I want to master \textit{Packing and Covering} so that I can apply it to real problems and communicate theoretical insights clearly and reproducibly.}

\NonFunctional{\pill{Reliability} \pill{Reproducibility}}

\BDD{Happy path}{the chapter, examples, and any tooling are available}{I complete the \emph{Hands-on Objective} and validations for this chapter}{the stated outcomes are produced (proof/code/summary) and recorded in the repo with passing checks}

\Meta
\end{StoryCard}

\begin{StoryTasks}
  \item Extract key definitions/lemmas; compile a one-page summary with references to the chapter.
  \item Derive density bounds for circle/sphere packing in small domains; validate by experiment.
  \item Implement a greedy packing heuristic; compare to known optimal layouts for toy sizes.
  \item Compute simple covering numbers for intervals/disks; visualize uncovered mass.
\end{StoryTasks}

\clearpage
\section*{HDCG-03 --- Tilings}
\begin{StoryCard}{HDCG-03}{Tilings}
\storyrow{Epic / Feature}{Combinatorial \& Discrete Geometry}
\storyrow{Business Value}{Establish theoretical tools and combinatorial principles that underpin later algorithms and applications. (Tilings).}
\storyrow{Priority / Estimate}{\textbf{Priority:} Must\hfill \textbf{SP:} 3}
\storyrow{Persona}{research student of discrete geometry}
\storyrow{Dependencies}{prior chapters as referenced}
\storyrow{Assumptions / Risks}{time to internalize proofs vs. breadth}

\StoryBody{As a research student of discrete geometry, I want to master \textit{Tilings} so that I can apply it to real problems and communicate theoretical insights clearly and reproducibly.}

\NonFunctional{\pill{Reliability} \pill{Reproducibility}}

\BDD{Happy path}{the chapter, examples, and any tooling are available}{I complete the \emph{Hands-on Objective} and validations for this chapter}{the stated outcomes are produced (proof/code/summary) and recorded in the repo with passing checks}

\Meta
\end{StoryCard}

\begin{StoryTasks}
  \item Extract key definitions/lemmas; compile a one-page summary with references to the chapter.
  \item Classify tilings (periodic vs. aperiodic) for given prototiles; prove a simple property.
  \item Implement a substitution tiling generator; render several levels of refinement.
  \item Measure tile frequency and boundary growth; summarize findings.
\end{StoryTasks}

\clearpage
\section*{HDCG-04 --- Helly-type Theorems \& Transversals}
\begin{StoryCard}{HDCG-04}{Helly-type Theorems \& Transversals}
\storyrow{Epic / Feature}{Combinatorial \& Discrete Geometry}
\storyrow{Business Value}{Establish theoretical tools and combinatorial principles that underpin later algorithms and applications. (Helly-type Theorems \& Transversals).}
\storyrow{Priority / Estimate}{\textbf{Priority:} Must\hfill \textbf{SP:} 3}
\storyrow{Persona}{research student of discrete geometry}
\storyrow{Dependencies}{prior chapters as referenced}
\storyrow{Assumptions / Risks}{time to internalize proofs vs. breadth}

\StoryBody{As a research student of discrete geometry, I want to master \textit{Helly-type Theorems \& Transversals} so that I can apply it to real problems and communicate theoretical insights clearly and reproducibly.}

\NonFunctional{\pill{Reliability} \pill{Reproducibility}}

\BDD{Happy path}{the chapter, examples, and any tooling are available}{I complete the \emph{Hands-on Objective} and validations for this chapter}{the stated outcomes are produced (proof/code/summary) and recorded in the repo with passing checks}

\Meta
\end{StoryCard}

\begin{StoryTasks}
  \item Extract key definitions/lemmas; compile a one-page summary with references to the chapter.
  \item State Helly/Carath\'{e}odory/Tverberg precisely; prove a concrete low-dimensional instance.
  \item Model a convex-feasibility LP; empirically test Helly-style certificates of feasibility.
  \item Create counterexamples to naive generalizations; document assumptions.
\end{StoryTasks}

\clearpage
\section*{HDCG-05 --- Pseudoline Arrangements}
\begin{StoryCard}{HDCG-05}{Pseudoline Arrangements}
\storyrow{Epic / Feature}{Combinatorial \& Discrete Geometry}
\storyrow{Business Value}{Establish theoretical tools and combinatorial principles that underpin later algorithms and applications. (Pseudoline Arrangements).}
\storyrow{Priority / Estimate}{\textbf{Priority:} Must\hfill \textbf{SP:} 3}
\storyrow{Persona}{research student of discrete geometry}
\storyrow{Dependencies}{prior chapters as referenced}
\storyrow{Assumptions / Risks}{time to internalize proofs vs. breadth}

\StoryBody{As a research student of discrete geometry, I want to master \textit{Pseudoline Arrangements} so that I can apply it to real problems and communicate theoretical insights clearly and reproducibly.}

\NonFunctional{\pill{Reliability} \pill{Reproducibility}}

\BDD{Happy path}{the chapter, examples, and any tooling are available}{I complete the \emph{Hands-on Objective} and validations for this chapter}{the stated outcomes are produced (proof/code/summary) and recorded in the repo with passing checks}

\Meta
\end{StoryCard}

\begin{StoryTasks}
  \item Extract key definitions/lemmas; compile a one-page summary with references to the chapter.
  \item Draw small arrangements; compute cells/levels/zones and verify counts.
  \item Implement arrangement construction for segments/lines and enumerate faces.
  \item Explore zone theorem numerically by measuring average zone complexity.
\end{StoryTasks}

\clearpage
\section*{HDCG-06 --- Oriented Matroids}
\begin{StoryCard}{HDCG-06}{Oriented Matroids}
\storyrow{Epic / Feature}{Combinatorial \& Discrete Geometry}
\storyrow{Business Value}{Establish theoretical tools and combinatorial principles that underpin later algorithms and applications. (Oriented Matroids).}
\storyrow{Priority / Estimate}{\textbf{Priority:} Must\hfill \textbf{SP:} 3}
\storyrow{Persona}{research student of discrete geometry}
\storyrow{Dependencies}{prior chapters as referenced}
\storyrow{Assumptions / Risks}{time to internalize proofs vs. breadth}

\StoryBody{As a research student of discrete geometry, I want to master \textit{Oriented Matroids} so that I can apply it to real problems and communicate theoretical insights clearly and reproducibly.}

\NonFunctional{\pill{Reliability} \pill{Reproducibility}}

\BDD{Happy path}{the chapter, examples, and any tooling are available}{I complete the \emph{Hands-on Objective} and validations for this chapter}{the stated outcomes are produced (proof/code/summary) and recorded in the repo with passing checks}

\Meta
\end{StoryCard}

\begin{StoryTasks}
  \item Extract key definitions/lemmas; compile a one-page summary with references to the chapter.
  \item Work with sign vectors and chirotopes on small point sets; verify axioms.
  \item Relate realizable vs. non-realizable examples; find and reproduce a literature example.
  \item Map an arrangement to an oriented matroid; note dualities.
\end{StoryTasks}

\clearpage
\section*{HDCG-07 --- Lattice Points \& Lattice Polytopes}
\begin{StoryCard}{HDCG-07}{Lattice Points \& Lattice Polytopes}
\storyrow{Epic / Feature}{Combinatorial \& Discrete Geometry}
\storyrow{Business Value}{Establish theoretical tools and combinatorial principles that underpin later algorithms and applications. (Lattice Points \& Lattice Polytopes).}
\storyrow{Priority / Estimate}{\textbf{Priority:} Must\hfill \textbf{SP:} 3}
\storyrow{Persona}{research student of discrete geometry}
\storyrow{Dependencies}{prior chapters as referenced}
\storyrow{Assumptions / Risks}{time to internalize proofs vs. breadth}

\StoryBody{As a research student of discrete geometry, I want to master \textit{Lattice Points \& Lattice Polytopes} so that I can apply it to real problems and communicate theoretical insights clearly and reproducibly.}

\NonFunctional{\pill{Reliability} \pill{Reproducibility}}

\BDD{Happy path}{the chapter, examples, and any tooling are available}{I complete the \emph{Hands-on Objective} and validations for this chapter}{the stated outcomes are produced (proof/code/summary) and recorded in the repo with passing checks}

\Meta
\end{StoryCard}

\begin{StoryTasks}
  \item Extract key definitions/lemmas; compile a one-page summary with references to the chapter.
  \item Compute Ehrhart polynomials for small lattice polytopes; verify reciprocity numerically.
  \item Implement lattice-point counting for boxes/simplices; validate against closed forms.
  \item Investigate how dilation changes counts and coefficients.
\end{StoryTasks}

\clearpage
\section*{HDCG-08 --- Low-Distortion Embeddings}
\begin{StoryCard}{HDCG-08}{Low-Distortion Embeddings}
\storyrow{Epic / Feature}{Combinatorial \& Discrete Geometry}
\storyrow{Business Value}{Establish theoretical tools and combinatorial principles that underpin later algorithms and applications. (Low-Distortion Embeddings).}
\storyrow{Priority / Estimate}{\textbf{Priority:} Must\hfill \textbf{SP:} 3}
\storyrow{Persona}{research student of discrete geometry}
\storyrow{Dependencies}{prior chapters as referenced}
\storyrow{Assumptions / Risks}{time to internalize proofs vs. breadth}

\StoryBody{As a research student of discrete geometry, I want to master \textit{Low-Distortion Embeddings} so that I can apply it to real problems and communicate theoretical insights clearly and reproducibly.}

\NonFunctional{\pill{Reliability} \pill{Reproducibility}}

\BDD{Happy path}{the chapter, examples, and any tooling are available}{I complete the \emph{Hands-on Objective} and validations for this chapter}{the stated outcomes are produced (proof/code/summary) and recorded in the repo with passing checks}

\Meta
\end{StoryCard}

\begin{StoryTasks}
  \item Extract key definitions/lemmas; compile a one-page summary with references to the chapter.
  \item Implement a Johnson--Lindenstrauss (JL) embedding; measure distortion for varying k.
  \item Compare PCA vs. random projections on a real dataset; report reconstruction error.
  \item Document trade-offs (run time, memory, accuracy) across embedding choices.
\end{StoryTasks}

\clearpage
\section*{HDCG-09 --- Polygonal Linkages}
\begin{StoryCard}{HDCG-09}{Polygonal Linkages}
\storyrow{Epic / Feature}{Combinatorial \& Discrete Geometry}
\storyrow{Business Value}{Establish theoretical tools and combinatorial principles that underpin later algorithms and applications. (Polygonal Linkages).}
\storyrow{Priority / Estimate}{\textbf{Priority:} Must\hfill \textbf{SP:} 3}
\storyrow{Persona}{research student of discrete geometry}
\storyrow{Dependencies}{prior chapters as referenced}
\storyrow{Assumptions / Risks}{time to internalize proofs vs. breadth}

\StoryBody{As a research student of discrete geometry, I want to master \textit{Polygonal Linkages} so that I can apply it to real problems and communicate theoretical insights clearly and reproducibly.}

\NonFunctional{\pill{Reliability} \pill{Reproducibility}}

\BDD{Happy path}{the chapter, examples, and any tooling are available}{I complete the \emph{Hands-on Objective} and validations for this chapter}{the stated outcomes are produced (proof/code/summary) and recorded in the repo with passing checks}

\Meta
\end{StoryCard}

\begin{StoryTasks}
  \item Extract key definitions/lemmas; compile a one-page summary with references to the chapter.
  \item Simulate a simple linkage; visualize configuration space qualitatively.
  \item Test feasibility (realisability) for a small linkage with constraints.
  \item Identify singular configurations and discuss rigidity vs. flexibility.
\end{StoryTasks}

\clearpage
\section*{HDCG-10 --- Geometric Graph Theory}
\begin{StoryCard}{HDCG-10}{Geometric Graph Theory}
\storyrow{Epic / Feature}{Combinatorial \& Discrete Geometry}
\storyrow{Business Value}{Establish theoretical tools and combinatorial principles that underpin later algorithms and applications. (Geometric Graph Theory).}
\storyrow{Priority / Estimate}{\textbf{Priority:} Must\hfill \textbf{SP:} 3}
\storyrow{Persona}{research student of discrete geometry}
\storyrow{Dependencies}{prior chapters as referenced}
\storyrow{Assumptions / Risks}{time to internalize proofs vs. breadth}

\StoryBody{As a research student of discrete geometry, I want to master \textit{Geometric Graph Theory} so that I can apply it to real problems and communicate theoretical insights clearly and reproducibly.}

\NonFunctional{\pill{Reliability} \pill{Reproducibility}}

\BDD{Happy path}{the chapter, examples, and any tooling are available}{I complete the \emph{Hands-on Objective} and validations for this chapter}{the stated outcomes are produced (proof/code/summary) and recorded in the repo with passing checks}

\Meta
\end{StoryCard}

\begin{StoryTasks}
  \item Extract key definitions/lemmas; compile a one-page summary with references to the chapter.
  \item Compute visibility graphs and planarity tests on random point sets.
  \item Measure crossing numbers experimentally for small n; compare to bounds.
  \item Explore thickness and minors on selected graphs.
\end{StoryTasks}

\clearpage
\section*{HDCG-11 --- Euclidean Ramsey Theory}
\begin{StoryCard}{HDCG-11}{Euclidean Ramsey Theory}
\storyrow{Epic / Feature}{Combinatorial \& Discrete Geometry}
\storyrow{Business Value}{Establish theoretical tools and combinatorial principles that underpin later algorithms and applications. (Euclidean Ramsey Theory).}
\storyrow{Priority / Estimate}{\textbf{Priority:} Must\hfill \textbf{SP:} 3}
\storyrow{Persona}{research student of discrete geometry}
\storyrow{Dependencies}{prior chapters as referenced}
\storyrow{Assumptions / Risks}{time to internalize proofs vs. breadth}

\StoryBody{As a research student of discrete geometry, I want to master \textit{Euclidean Ramsey Theory} so that I can apply it to real problems and communicate theoretical insights clearly and reproducibly.}

\NonFunctional{\pill{Reliability} \pill{Reproducibility}}

\BDD{Happy path}{the chapter, examples, and any tooling are available}{I complete the \emph{Hands-on Objective} and validations for this chapter}{the stated outcomes are produced (proof/code/summary) and recorded in the repo with passing checks}

\Meta
\end{StoryCard}

\begin{StoryTasks}
  \item Extract key definitions/lemmas; compile a one-page summary with references to the chapter.
  \item Implement colorings for points/edges; search for monochromatic structures.
  \item Reproduce a small Euclidean Ramsey statement; provide a constructive or probabilistic proof sketch.
  \item Summarize growth rates and open directions.
\end{StoryTasks}

\clearpage
\section*{HDCG-12 --- Discrete Aspects of Stochastic Geometry}
\begin{StoryCard}{HDCG-12}{Discrete Aspects of Stochastic Geometry}
\storyrow{Epic / Feature}{Combinatorial \& Discrete Geometry}
\storyrow{Business Value}{Establish theoretical tools and combinatorial principles that underpin later algorithms and applications. (Discrete Aspects of Stochastic Geometry).}
\storyrow{Priority / Estimate}{\textbf{Priority:} Must\hfill \textbf{SP:} 3}
\storyrow{Persona}{research student of discrete geometry}
\storyrow{Dependencies}{prior chapters as referenced}
\storyrow{Assumptions / Risks}{time to internalize proofs vs. breadth; variance in experiments; need for many trials}

\StoryBody{As a research student of discrete geometry, I want to master \textit{Discrete Aspects of Stochastic Geometry} so that I can apply it to real problems and communicate theoretical insights clearly and reproducibly.}

\NonFunctional{\pill{Reliability} \pill{Reproducibility}}

\BDD{Happy path}{the chapter, examples, and any tooling are available}{I complete the \emph{Hands-on Objective} and validations for this chapter}{the stated outcomes are produced (proof/code/summary) and recorded in the repo with passing checks}

\Meta
\end{StoryCard}

\begin{StoryTasks}
  \item Extract key definitions/lemmas; compile a one-page summary with references to the chapter.
  \item Simulate Poisson point processes; estimate mean area/length of derived structures.
  \item Compute mean widths/coverage via Monte Carlo; include confidence intervals.
  \item Compare empirical findings to theoretical expectations where available.
\end{StoryTasks}

\clearpage
\section*{HDCG-13 --- Geometric Discrepancy \& Uniform Distribution}
\begin{StoryCard}{HDCG-13}{Geometric Discrepancy \& Uniform Distribution}
\storyrow{Epic / Feature}{Combinatorial \& Discrete Geometry}
\storyrow{Business Value}{Establish theoretical tools and combinatorial principles that underpin later algorithms and applications. (Geometric Discrepancy \& Uniform Distribution).}
\storyrow{Priority / Estimate}{\textbf{Priority:} Must\hfill \textbf{SP:} 3}
\storyrow{Persona}{research student of discrete geometry}
\storyrow{Dependencies}{prior chapters as referenced}
\storyrow{Assumptions / Risks}{time to internalize proofs vs. breadth}

\StoryBody{As a research student of discrete geometry, I want to master \textit{Geometric Discrepancy \& Uniform Distribution} so that I can apply it to real problems and communicate theoretical insights clearly and reproducibly.}

\NonFunctional{\pill{Reliability} \pill{Reproducibility}}

\BDD{Happy path}{the chapter, examples, and any tooling are available}{I complete the \emph{Hands-on Objective} and validations for this chapter}{the stated outcomes are produced (proof/code/summary) and recorded in the repo with passing checks}

\Meta
\end{StoryCard}

\begin{StoryTasks}
  \item Extract key definitions/lemmas; compile a one-page summary with references to the chapter.
  \item Compute discrepancy of random vs. low-discrepancy sequences (Halton/Sobol) on test sets.
  \item Integrate a smooth test function using QMC vs. MC; compare convergence rates.
  \item Relate discrepancy to VC-dimension or range spaces from the chapter.
\end{StoryTasks}

\clearpage
\section*{HDCG-14 --- Polyominoes}
\begin{StoryCard}{HDCG-14}{Polyominoes}
\storyrow{Epic / Feature}{Combinatorial \& Discrete Geometry}
\storyrow{Business Value}{Establish theoretical tools and combinatorial principles that underpin later algorithms and applications. (Polyominoes).}
\storyrow{Priority / Estimate}{\textbf{Priority:} Must\hfill \textbf{SP:} 3}
\storyrow{Persona}{research student of discrete geometry}
\storyrow{Dependencies}{prior chapters as referenced}
\storyrow{Assumptions / Risks}{time to internalize proofs vs. breadth}

\StoryBody{As a research student of discrete geometry, I want to master \textit{Polyominoes} so that I can apply it to real problems and communicate theoretical insights clearly and reproducibly.}

\NonFunctional{\pill{Reliability} \pill{Reproducibility}}

\BDD{Happy path}{the chapter, examples, and any tooling are available}{I complete the \emph{Hands-on Objective} and validations for this chapter}{the stated outcomes are produced (proof/code/summary) and recorded in the repo with passing checks}

\Meta
\end{StoryCard}

\begin{StoryTasks}
  \item Extract key definitions/lemmas; compile a one-page summary with references to the chapter.
  \item Use exact cover (DLX) to tile small regions; verify parity and area constraints.
  \item Classify tilings for a given board; produce enumerations for small sizes.
  \item Document invariants that quickly rule out impossible tilings.
\end{StoryTasks}

\clearpage
\section*{HDCG-15 --- Convex Polytopes: Basics}
\begin{StoryCard}{HDCG-15}{Convex Polytopes: Basics}
\storyrow{Epic / Feature}{Polytopes \& Polyhedra}
\storyrow{Business Value}{Build fluency with polytopal structures used across optimization, geometry processing, and combinatorics. (Convex Polytopes: Basics).}
\storyrow{Priority / Estimate}{\textbf{Priority:} Must\hfill \textbf{SP:} 3}
\storyrow{Persona}{research student of discrete geometry}
\storyrow{Dependencies}{prior chapters as referenced}
\storyrow{Assumptions / Risks}{time to internalize proofs vs. breadth}

\StoryBody{As a research student of discrete geometry, I want to master \textit{Convex Polytopes: Basics} so that I can apply it to real problems and communicate theoretical insights clearly and reproducibly.}

\NonFunctional{\pill{Reliability} \pill{Reproducibility}}

\BDD{Happy path}{the chapter, examples, and any tooling are available}{I complete the \emph{Hands-on Objective} and validations for this chapter}{the stated outcomes are produced (proof/code/summary) and recorded in the repo with passing checks}

\Meta
\end{StoryCard}

\begin{StoryTasks}
  \item Extract key definitions/lemmas; compile a one-page summary with references to the chapter.
  \item Convert small polytopes between H- and V-representations; check duals.
  \item Illustrate faces/facets/normal fans with a plotting tool; verify Euler.
  \item Summarize duality relationships with examples (simplex, cube, cross-polytope).
\end{StoryTasks}

\clearpage
\section*{HDCG-16 --- Subdivisions \& Triangulations of Polytopes}
\begin{StoryCard}{HDCG-16}{Subdivisions \& Triangulations of Polytopes}
\storyrow{Epic / Feature}{Polytopes \& Polyhedra}
\storyrow{Business Value}{Build fluency with polytopal structures used across optimization, geometry processing, and combinatorics. (Subdivisions \& Triangulations of Polytopes).}
\storyrow{Priority / Estimate}{\textbf{Priority:} Must\hfill \textbf{SP:} 3}
\storyrow{Persona}{research student of discrete geometry}
\storyrow{Dependencies}{prior chapters as referenced, convex hulls, orientation/circumcircle predicates}
\storyrow{Assumptions / Risks}{time to internalize proofs vs. breadth}

\StoryBody{As a research student of discrete geometry, I want to master \textit{Subdivisions \& Triangulations of Polytopes} so that I can apply it to real problems and communicate theoretical insights clearly and reproducibly.}

\NonFunctional{\pill{Reliability} \pill{Reproducibility} \pill{Performance} \pill{Scalability}}

\BDD{Happy path}{the chapter, examples, and any tooling are available}{I complete the \emph{Hands-on Objective} and validations for this chapter}{the stated outcomes are produced (proof/code/summary) and recorded in the repo with passing checks}

\Meta
\end{StoryCard}

\begin{StoryTasks}
  \item Extract key definitions/lemmas; compile a one-page summary with references to the chapter.
  \item Construct regular triangulations; implement flips and track secondary polytopes qualitatively.
  \item Evaluate simplex quality metrics (aspect ratio, minimum angle).
  \item Show effect of lifting on triangulation regularity.
\end{StoryTasks}

\clearpage
\section*{HDCG-17 --- Face Numbers: f-, h-, and g-vectors}
\begin{StoryCard}{HDCG-17}{Face Numbers: f-, h-, and g-vectors}
\storyrow{Epic / Feature}{Polytopes \& Polyhedra}
\storyrow{Business Value}{Build fluency with polytopal structures used across optimization, geometry processing, and combinatorics. (Face Numbers: f-, h-, and g-vectors).}
\storyrow{Priority / Estimate}{\textbf{Priority:} Must\hfill \textbf{SP:} 3}
\storyrow{Persona}{research student of discrete geometry}
\storyrow{Dependencies}{prior chapters as referenced}
\storyrow{Assumptions / Risks}{time to internalize proofs vs. breadth}

\StoryBody{As a research student of discrete geometry, I want to master \textit{Face Numbers: f-, h-, and g-vectors} so that I can apply it to real problems and communicate theoretical insights clearly and reproducibly.}

\NonFunctional{\pill{Reliability} \pill{Reproducibility}}

\BDD{Happy path}{the chapter, examples, and any tooling are available}{I complete the \emph{Hands-on Objective} and validations for this chapter}{the stated outcomes are produced (proof/code/summary) and recorded in the repo with passing checks}

\Meta
\end{StoryCard}

\begin{StoryTasks}
  \item Extract key definitions/lemmas; compile a one-page summary with references to the chapter.
  \item Compute f- and h-vectors for classic polytopes; check Dehn--Sommerville relations.
  \item Explore g-theorem examples; illustrate inequalities with plots.
  \item Prepare a cheat sheet summarizing identities and constraints.
\end{StoryTasks}

\clearpage
\section*{HDCG-18 --- Symmetry of Polytopes}
\begin{StoryCard}{HDCG-18}{Symmetry of Polytopes}
\storyrow{Epic / Feature}{Polytopes \& Polyhedra}
\storyrow{Business Value}{Build fluency with polytopal structures used across optimization, geometry processing, and combinatorics. (Symmetry of Polytopes).}
\storyrow{Priority / Estimate}{\textbf{Priority:} Must\hfill \textbf{SP:} 3}
\storyrow{Persona}{research student of discrete geometry}
\storyrow{Dependencies}{prior chapters as referenced}
\storyrow{Assumptions / Risks}{time to internalize proofs vs. breadth}

\StoryBody{As a research student of discrete geometry, I want to master \textit{Symmetry of Polytopes} so that I can apply it to real problems and communicate theoretical insights clearly and reproducibly.}

\NonFunctional{\pill{Reliability} \pill{Reproducibility}}

\BDD{Happy path}{the chapter, examples, and any tooling are available}{I complete the \emph{Hands-on Objective} and validations for this chapter}{the stated outcomes are produced (proof/code/summary) and recorded in the repo with passing checks}

\Meta
\end{StoryCard}

\begin{StoryTasks}
  \item Extract key definitions/lemmas; compile a one-page summary with references to the chapter.
  \item Identify automorphism groups for basic polytopes; compute group orders.
  \item Relate symmetry to orbits of faces; visualize symmetric embeddings.
  \item Discuss symmetry exploitation in algorithms (state-space reduction).
\end{StoryTasks}

\clearpage
\section*{HDCG-19 --- Polytope Skeletons \& Paths}
\begin{StoryCard}{HDCG-19}{Polytope Skeletons \& Paths}
\storyrow{Epic / Feature}{Polytopes \& Polyhedra}
\storyrow{Business Value}{Build fluency with polytopal structures used across optimization, geometry processing, and combinatorics. (Polytope Skeletons \& Paths).}
\storyrow{Priority / Estimate}{\textbf{Priority:} Must\hfill \textbf{SP:} 3}
\storyrow{Persona}{research student of discrete geometry}
\storyrow{Dependencies}{prior chapters as referenced}
\storyrow{Assumptions / Risks}{time to internalize proofs vs. breadth}

\StoryBody{As a research student of discrete geometry, I want to master \textit{Polytope Skeletons \& Paths} so that I can apply it to real problems and communicate theoretical insights clearly and reproducibly.}

\NonFunctional{\pill{Reliability} \pill{Reproducibility}}

\BDD{Happy path}{the chapter, examples, and any tooling are available}{I complete the \emph{Hands-on Objective} and validations for this chapter}{the stated outcomes are produced (proof/code/summary) and recorded in the repo with passing checks}

\Meta
\end{StoryCard}

\begin{StoryTasks}
  \item Extract key definitions/lemmas; compile a one-page summary with references to the chapter.
  \item Experiment with graph diameters; simulate pivot paths on polytope graphs.
  \item Implement simple pivot rules; measure paths vs. diameter bounds.
  \item Relate results to Hirsch-type questions qualitatively.
\end{StoryTasks}

\clearpage
\section*{HDCG-20 --- Polyhedral Maps}
\begin{StoryCard}{HDCG-20}{Polyhedral Maps}
\storyrow{Epic / Feature}{Polytopes \& Polyhedra}
\storyrow{Business Value}{Build fluency with polytopal structures used across optimization, geometry processing, and combinatorics. (Polyhedral Maps).}
\storyrow{Priority / Estimate}{\textbf{Priority:} Must\hfill \textbf{SP:} 3}
\storyrow{Persona}{research student of discrete geometry}
\storyrow{Dependencies}{prior chapters as referenced}
\storyrow{Assumptions / Risks}{time to internalize proofs vs. breadth}

\StoryBody{As a research student of discrete geometry, I want to master \textit{Polyhedral Maps} so that I can apply it to real problems and communicate theoretical insights clearly and reproducibly.}

\NonFunctional{\pill{Reliability} \pill{Reproducibility}}

\BDD{Happy path}{the chapter, examples, and any tooling are available}{I complete the \emph{Hands-on Objective} and validations for this chapter}{the stated outcomes are produced (proof/code/summary) and recorded in the repo with passing checks}

\Meta
\end{StoryCard}

\begin{StoryTasks}
  \item Extract key definitions/lemmas; compile a one-page summary with references to the chapter.
  \item Embed simple polyhedral maps on surfaces; verify Euler characteristic.
  \item Count faces/edges/vertices; confirm orientability effects.
  \item Construct a small map with specified degree sequence.
\end{StoryTasks}

\clearpage
\section*{HDCG-21 --- Topological Methods in Discrete Geometry}
\begin{StoryCard}{HDCG-21}{Topological Methods in Discrete Geometry}
\storyrow{Epic / Feature}{Combinatorial \& Computational Topology}
\storyrow{Business Value}{Leverage topological perspectives to reason about existence, structure, and invariants in geometric problems. (Topological Methods in Discrete Geometry).}
\storyrow{Priority / Estimate}{\textbf{Priority:} Must\hfill \textbf{SP:} 3}
\storyrow{Persona}{research student of discrete geometry}
\storyrow{Dependencies}{prior chapters as referenced}
\storyrow{Assumptions / Risks}{time to internalize proofs vs. breadth}

\StoryBody{As a research student of discrete geometry, I want to master \textit{Topological Methods in Discrete Geometry} so that I can apply it to real problems and communicate theoretical insights clearly and reproducibly.}

\NonFunctional{\pill{Reliability} \pill{Reproducibility}}

\BDD{Happy path}{the chapter, examples, and any tooling are available}{I complete the \emph{Hands-on Objective} and validations for this chapter}{the stated outcomes are produced (proof/code/summary) and recorded in the repo with passing checks}

\Meta
\end{StoryCard}

\begin{StoryTasks}
  \item Extract key definitions/lemmas; compile a one-page summary with references to the chapter.
  \item Use ham-sandwich or Borsuk--Ulam style arguments to prove a discrete claim.
  \item Illustrate topological proof vs. combinatorial alternative on the same problem.
  \item Document assumptions and generalization limits.
\end{StoryTasks}

\clearpage
\section*{HDCG-22 --- Random Simplicial Complexes}
\begin{StoryCard}{HDCG-22}{Random Simplicial Complexes}
\storyrow{Epic / Feature}{Combinatorial \& Computational Topology}
\storyrow{Business Value}{Leverage topological perspectives to reason about existence, structure, and invariants in geometric problems. (Random Simplicial Complexes).}
\storyrow{Priority / Estimate}{\textbf{Priority:} Must\hfill \textbf{SP:} 3}
\storyrow{Persona}{research student of discrete geometry}
\storyrow{Dependencies}{prior chapters as referenced}
\storyrow{Assumptions / Risks}{time to internalize proofs vs. breadth; variance in experiments; need for many trials}

\StoryBody{As a research student of discrete geometry, I want to master \textit{Random Simplicial Complexes} so that I can apply it to real problems and communicate theoretical insights clearly and reproducibly.}

\NonFunctional{\pill{Reliability} \pill{Reproducibility}}

\BDD{Happy path}{the chapter, examples, and any tooling are available}{I complete the \emph{Hands-on Objective} and validations for this chapter}{the stated outcomes are produced (proof/code/summary) and recorded in the repo with passing checks}

\Meta
\end{StoryCard}

\begin{StoryTasks}
  \item Extract key definitions/lemmas; compile a one-page summary with references to the chapter.
  \item Generate Linial--Meshulam or Vietoris--Rips models; sweep probability parameters.
  \item Measure thresholds for connectedness or vanishing homology.
  \item Plot Betti numbers across regimes; discuss finite-size effects.
\end{StoryTasks}

\clearpage
\section*{HDCG-23 --- Graphs on Surfaces (Embeddings \& Genus)}
\begin{StoryCard}{HDCG-23}{Graphs on Surfaces (Embeddings \& Genus)}
\storyrow{Epic / Feature}{Combinatorial \& Computational Topology}
\storyrow{Business Value}{Leverage topological perspectives to reason about existence, structure, and invariants in geometric problems. (Graphs on Surfaces (Embeddings \& Genus)).}
\storyrow{Priority / Estimate}{\textbf{Priority:} Must\hfill \textbf{SP:} 3}
\storyrow{Persona}{research student of discrete geometry}
\storyrow{Dependencies}{prior chapters as referenced}
\storyrow{Assumptions / Risks}{time to internalize proofs vs. breadth; numerical robustness and degeneracies}

\StoryBody{As a research student of discrete geometry, I want to master \textit{Graphs on Surfaces (Embeddings \& Genus)} so that I can apply it to real problems and communicate theoretical insights clearly and reproducibly.}

\NonFunctional{\pill{Reliability} \pill{Reproducibility} \pill{Robustness} \pill{Accuracy}}

\BDD{Happy path}{the chapter, examples, and any tooling are available}{I complete the \emph{Hands-on Objective} and validations for this chapter}{the stated outcomes are produced (proof/code/summary) and recorded in the repo with passing checks}

\Meta
\end{StoryCard}

\begin{StoryTasks}
  \item Extract key definitions/lemmas; compile a one-page summary with references to the chapter.
  \item Implement planarity testing; compute genus for small graphs.
  \item Route edges on a surface embedding; visualize crossings by handles.
  \item Compare embeddings with/without constraints.
\end{StoryTasks}

\clearpage
\section*{HDCG-24 --- Persistent Homology (Barcodes)}
\begin{StoryCard}{HDCG-24}{Persistent Homology (Barcodes)}
\storyrow{Epic / Feature}{Combinatorial \& Computational Topology}
\storyrow{Business Value}{Leverage topological perspectives to reason about existence, structure, and invariants in geometric problems. (Persistent Homology (Barcodes)).}
\storyrow{Priority / Estimate}{\textbf{Priority:} Must\hfill \textbf{SP:} 3}
\storyrow{Persona}{research student of discrete geometry}
\storyrow{Dependencies}{prior chapters as referenced, simplicial complexes, metrics, stability notion}
\storyrow{Assumptions / Risks}{time to internalize proofs vs. breadth}

\StoryBody{As a research student of discrete geometry, I want to master \textit{Persistent Homology (Barcodes)} so that I can apply it to real problems and communicate theoretical insights clearly and reproducibly.}

\NonFunctional{\pill{Reliability} \pill{Reproducibility}}

\BDD{Happy path}{the chapter, examples, and any tooling are available}{I complete the \emph{Hands-on Objective} and validations for this chapter}{the stated outcomes are produced (proof/code/summary) and recorded in the repo with passing checks}

\Meta
\end{StoryCard}

\begin{StoryTasks}
  \item Extract key definitions/lemmas; compile a one-page summary with references to the chapter.
  \item Build filtrations (Rips/\v{C}ech); compute 0--2D barcodes on a toy dataset.
  \item Study stability by adding noise and comparing bottleneck distance.
  \item Interpret features (lifetimes) with domain context.
\end{StoryTasks}

\clearpage
\section*{HDCG-25 --- High-dimensional Topological Data Analysis}
\begin{StoryCard}{HDCG-25}{High-dimensional Topological Data Analysis}
\storyrow{Epic / Feature}{Combinatorial \& Computational Topology}
\storyrow{Business Value}{Leverage topological perspectives to reason about existence, structure, and invariants in geometric problems. (High-dimensional Topological Data Analysis).}
\storyrow{Priority / Estimate}{\textbf{Priority:} Must\hfill \textbf{SP:} 3}
\storyrow{Persona}{data scientist working with geometric methods}
\storyrow{Dependencies}{prior chapters as referenced}
\storyrow{Assumptions / Risks}{time to internalize proofs vs. breadth; curse of dimensionality; memory pressure}

\StoryBody{As a data scientist working with geometric methods, I want to master \textit{High-dimensional Topological Data Analysis} so that I can apply it to real problems and communicate theoretical insights clearly and reproducibly.}

\NonFunctional{\pill{Reliability} \pill{Reproducibility}}

\BDD{Happy path}{the chapter, examples, and any tooling are available}{I complete the \emph{Hands-on Objective} and validations for this chapter}{the stated outcomes are produced (proof/code/summary) and recorded in the repo with passing checks}

\Meta
\end{StoryCard}

\begin{StoryTasks}
  \item Extract key definitions/lemmas; compile a one-page summary with references to the chapter.
  \item Construct sparse filtrations; apply landmarks/witness complexes.
  \item Use dimensionality reduction to visualize summaries; validate with metrics.
  \item Evaluate scalability (time/memory) vs. sample size and dimension.
\end{StoryTasks}

\clearpage
\section*{HDCG-26 --- Convex Hulls}
\begin{StoryCard}{HDCG-26}{Convex Hulls}
\storyrow{Epic / Feature}{Algorithms of Fundamental Geometric Objects}
\storyrow{Business Value}{Master core geometric algorithms that power search, reconstruction, meshing, and planning pipelines. (Convex Hulls).}
\storyrow{Priority / Estimate}{\textbf{Priority:} Must\hfill \textbf{SP:} 3}
\storyrow{Persona}{applied geometry engineer}
\storyrow{Dependencies}{prior chapters as referenced}
\storyrow{Assumptions / Risks}{time to internalize proofs vs. breadth}

\StoryBody{As a applied geometry engineer, I want to master \textit{Convex Hulls} so that I can apply it to real problems and communicate algorithmic insights clearly and reproducibly.}

\NonFunctional{\pill{Reliability} \pill{Reproducibility} \pill{Performance} \pill{Scalability}}

\BDD{Happy path}{the chapter, examples, and any tooling are available}{I complete the \emph{Hands-on Objective} and validations for this chapter}{the stated outcomes are produced (proof/code/summary) and recorded in the repo with passing checks}

\Meta
\end{StoryCard}

\begin{StoryTasks}
  \item Extract key definitions/lemmas; compile a one-page summary with references to the chapter.
  \item Implement 2D (Andrew) and 3D (incremental) hulls; include degeneracy handling.
  \item Add exact/filtered predicates; benchmark accuracy and speed.
  \item Validate by area/volume/facet orientation; export meshes.
\end{StoryTasks}

\clearpage
\section*{HDCG-27 --- Voronoi Diagrams \& Delaunay Triangulations}
\begin{StoryCard}{HDCG-27}{Voronoi Diagrams \& Delaunay Triangulations}
\storyrow{Epic / Feature}{Algorithms of Fundamental Geometric Objects}
\storyrow{Business Value}{Master core geometric algorithms that power search, reconstruction, meshing, and planning pipelines. (Voronoi Diagrams \& Delaunay Triangulations).}
\storyrow{Priority / Estimate}{\textbf{Priority:} Must\hfill \textbf{SP:} 3}
\storyrow{Persona}{applied geometry engineer}
\storyrow{Dependencies}{prior chapters as referenced, convex hulls, orientation/circumcircle predicates}
\storyrow{Assumptions / Risks}{time to internalize proofs vs. breadth}

\StoryBody{As a applied geometry engineer, I want to master \textit{Voronoi Diagrams \& Delaunay Triangulations} so that I can apply it to real problems and communicate algorithmic insights clearly and reproducibly.}

\NonFunctional{\pill{Reliability} \pill{Reproducibility} \pill{Performance} \pill{Scalability}}

\BDD{Happy path}{the chapter, examples, and any tooling are available}{I complete the \emph{Hands-on Objective} and validations for this chapter}{the stated outcomes are produced (proof/code/summary) and recorded in the repo with passing checks}

\Meta
\end{StoryCard}

\begin{StoryTasks}
  \item Extract key definitions/lemmas; compile a one-page summary with references to the chapter.
  \item Build Delaunay via edge flips/incremental insertion; output Voronoi by duality.
  \item Implement point location; compare query latency to kd-tree baseline.
  \item Stress-test co-circular/duplicate inputs; enable exact predicates.
\end{StoryTasks}

\clearpage
\section*{HDCG-28 --- Arrangements of Curves and Surfaces}
\begin{StoryCard}{HDCG-28}{Arrangements of Curves and Surfaces}
\storyrow{Epic / Feature}{Algorithms of Fundamental Geometric Objects}
\storyrow{Business Value}{Master core geometric algorithms that power search, reconstruction, meshing, and planning pipelines. (Arrangements of Curves and Surfaces).}
\storyrow{Priority / Estimate}{\textbf{Priority:} Must\hfill \textbf{SP:} 3}
\storyrow{Persona}{applied geometry engineer}
\storyrow{Dependencies}{prior chapters as referenced}
\storyrow{Assumptions / Risks}{time to internalize proofs vs. breadth; numerical robustness and degeneracies}

\StoryBody{As a applied geometry engineer, I want to master \textit{Arrangements of Curves and Surfaces} so that I can apply it to real problems and communicate algorithmic insights clearly and reproducibly.}

\NonFunctional{\pill{Reliability} \pill{Reproducibility} \pill{Robustness} \pill{Accuracy}}

\BDD{Happy path}{the chapter, examples, and any tooling are available}{I complete the \emph{Hands-on Objective} and validations for this chapter}{the stated outcomes are produced (proof/code/summary) and recorded in the repo with passing checks}

\Meta
\end{StoryCard}

\begin{StoryTasks}
  \item Extract key definitions/lemmas; compile a one-page summary with references to the chapter.
  \item Construct arrangements of segments/lines; enumerate cells and graph structure.
  \item Measure zone/level complexity empirically; compare to theory.
  \item Demonstrate applications (motion planning cells / point location).
\end{StoryTasks}

\clearpage
\section*{HDCG-29 --- Triangulations \& Mesh Generation}
\begin{StoryCard}{HDCG-29}{Triangulations \& Mesh Generation}
\storyrow{Epic / Feature}{Algorithms of Fundamental Geometric Objects}
\storyrow{Business Value}{Master core geometric algorithms that power search, reconstruction, meshing, and planning pipelines. (Triangulations \& Mesh Generation).}
\storyrow{Priority / Estimate}{\textbf{Priority:} Must\hfill \textbf{SP:} 3}
\storyrow{Persona}{applied geometry engineer}
\storyrow{Dependencies}{prior chapters as referenced, convex hulls, orientation/circumcircle predicates}
\storyrow{Assumptions / Risks}{time to internalize proofs vs. breadth}

\StoryBody{As a applied geometry engineer, I want to master \textit{Triangulations \& Mesh Generation} so that I can apply it to real problems and communicate algorithmic insights clearly and reproducibly.}

\NonFunctional{\pill{Reliability} \pill{Reproducibility} \pill{Performance} \pill{Scalability}}

\BDD{Happy path}{the chapter, examples, and any tooling are available}{I complete the \emph{Hands-on Objective} and validations for this chapter}{the stated outcomes are produced (proof/code/summary) and recorded in the repo with passing checks}

\Meta
\end{StoryCard}

\begin{StoryTasks}
  \item Extract key definitions/lemmas; compile a one-page summary with references to the chapter.
  \item Implement Delaunay refinement; track minimum angle and element size.
  \item Perform boundary recovery; assess quality metrics before/after refinement.
  \item Export mesh and run a simple PDE/graphics demo.
\end{StoryTasks}

\clearpage
\section*{HDCG-30 --- Polygons (Geometry \& Algorithms)}
\begin{StoryCard}{HDCG-30}{Polygons (Geometry \& Algorithms)}
\storyrow{Epic / Feature}{Algorithms of Fundamental Geometric Objects}
\storyrow{Business Value}{Master core geometric algorithms that power search, reconstruction, meshing, and planning pipelines. (Polygons (Geometry \& Algorithms)).}
\storyrow{Priority / Estimate}{\textbf{Priority:} Must\hfill \textbf{SP:} 3}
\storyrow{Persona}{applied geometry engineer}
\storyrow{Dependencies}{prior chapters as referenced}
\storyrow{Assumptions / Risks}{time to internalize proofs vs. breadth}

\StoryBody{As a applied geometry engineer, I want to master \textit{Polygons (Geometry \& Algorithms)} so that I can apply it to real problems and communicate algorithmic insights clearly and reproducibly.}

\NonFunctional{\pill{Reliability} \pill{Reproducibility} \pill{Performance} \pill{Scalability}}

\BDD{Happy path}{the chapter, examples, and any tooling are available}{I complete the \emph{Hands-on Objective} and validations for this chapter}{the stated outcomes are produced (proof/code/summary) and recorded in the repo with passing checks}

\Meta
\end{StoryCard}

\begin{StoryTasks}
  \item Extract key definitions/lemmas; compile a one-page summary with references to the chapter.
  \item Implement point-in-polygon (ray casting / winding number); triangulate simple polygons.
  \item Handle degeneracies (collinearity, repeated vertices); compute area/centroid.
  \item Demonstrate art-gallery style visibility on floorplan polygons.
\end{StoryTasks}

\clearpage
\section*{HDCG-31 --- Shortest Paths \& Networks}
\begin{StoryCard}{HDCG-31}{Shortest Paths \& Networks}
\storyrow{Epic / Feature}{Algorithms of Fundamental Geometric Objects}
\storyrow{Business Value}{Master core geometric algorithms that power search, reconstruction, meshing, and planning pipelines. (Shortest Paths \& Networks).}
\storyrow{Priority / Estimate}{\textbf{Priority:} Must\hfill \textbf{SP:} 3}
\storyrow{Persona}{applied geometry engineer}
\storyrow{Dependencies}{prior chapters as referenced}
\storyrow{Assumptions / Risks}{time to internalize proofs vs. breadth}

\StoryBody{As a applied geometry engineer, I want to master \textit{Shortest Paths \& Networks} so that I can apply it to real problems and communicate algorithmic insights clearly and reproducibly.}

\NonFunctional{\pill{Reliability} \pill{Reproducibility}}

\BDD{Happy path}{the chapter, examples, and any tooling are available}{I complete the \emph{Hands-on Objective} and validations for this chapter}{the stated outcomes are produced (proof/code/summary) and recorded in the repo with passing checks}

\Meta
\end{StoryCard}

\begin{StoryTasks}
  \item Extract key definitions/lemmas; compile a one-page summary with references to the chapter.
  \item Build a visibility graph and run Dijkstra for polygonal domains.
  \item Implement continuous Dijkstra or funnel algorithm on triangulations.
  \item Compare path lengths and runtimes across methods.
\end{StoryTasks}

\clearpage
\section*{HDCG-32 --- Proximity Algorithms}
\begin{StoryCard}{HDCG-32}{Proximity Algorithms}
\storyrow{Epic / Feature}{Algorithms of Fundamental Geometric Objects}
\storyrow{Business Value}{Master core geometric algorithms that power search, reconstruction, meshing, and planning pipelines. (Proximity Algorithms).}
\storyrow{Priority / Estimate}{\textbf{Priority:} Must\hfill \textbf{SP:} 3}
\storyrow{Persona}{applied geometry engineer}
\storyrow{Dependencies}{prior chapters as referenced}
\storyrow{Assumptions / Risks}{time to internalize proofs vs. breadth}

\StoryBody{As a applied geometry engineer, I want to master \textit{Proximity Algorithms} so that I can apply it to real problems and communicate algorithmic insights clearly and reproducibly.}

\NonFunctional{\pill{Reliability} \pill{Reproducibility} \pill{Performance} \pill{Scalability}}

\BDD{Happy path}{the chapter, examples, and any tooling are available}{I complete the \emph{Hands-on Objective} and validations for this chapter}{the stated outcomes are produced (proof/code/summary) and recorded in the repo with passing checks}

\Meta
\end{StoryCard}

\begin{StoryTasks}
  \item Extract key definitions/lemmas; compile a one-page summary with references to the chapter.
  \item Construct MST/\(\beta\)-skeletons and k-NN graphs; profile runtimes.
  \item Analyse stability under noise; compare exact vs. approximate structures.
  \item Summarize use-cases (clustering, skeletonization).
\end{StoryTasks}

\clearpage
\section*{HDCG-33 --- Visibility \& Art-Gallery Problems}
\begin{StoryCard}{HDCG-33}{Visibility \& Art-Gallery Problems}
\storyrow{Epic / Feature}{Algorithms of Fundamental Geometric Objects}
\storyrow{Business Value}{Master core geometric algorithms that power search, reconstruction, meshing, and planning pipelines. (Visibility \& Art-Gallery Problems).}
\storyrow{Priority / Estimate}{\textbf{Priority:} Must\hfill \textbf{SP:} 3}
\storyrow{Persona}{applied geometry engineer}
\storyrow{Dependencies}{prior chapters as referenced}
\storyrow{Assumptions / Risks}{time to internalize proofs vs. breadth}

\StoryBody{As a applied geometry engineer, I want to master \textit{Visibility \& Art-Gallery Problems} so that I can apply it to real problems and communicate algorithmic insights clearly and reproducibly.}

\NonFunctional{\pill{Reliability} \pill{Reproducibility}}

\BDD{Happy path}{the chapter, examples, and any tooling are available}{I complete the \emph{Hands-on Objective} and validations for this chapter}{the stated outcomes are produced (proof/code/summary) and recorded in the repo with passing checks}

\Meta
\end{StoryCard}

\begin{StoryTasks}
  \item Extract key definitions/lemmas; compile a one-page summary with references to the chapter.
  \item Compute visibility polygons; handle holes and reflex vertices.
  \item Formulate simple art-gallery coverage and test heuristics.
  \item Visualize guard placements and uncovered regions.
\end{StoryTasks}

\clearpage
\section*{HDCG-34 --- Geometric Reconstruction Problems}
\begin{StoryCard}{HDCG-34}{Geometric Reconstruction Problems}
\storyrow{Epic / Feature}{Algorithms of Fundamental Geometric Objects}
\storyrow{Business Value}{Master core geometric algorithms that power search, reconstruction, meshing, and planning pipelines. (Geometric Reconstruction Problems).}
\storyrow{Priority / Estimate}{\textbf{Priority:} Must\hfill \textbf{SP:} 3}
\storyrow{Persona}{applied geometry engineer}
\storyrow{Dependencies}{prior chapters as referenced}
\storyrow{Assumptions / Risks}{time to internalize proofs vs. breadth}

\StoryBody{As a applied geometry engineer, I want to master \textit{Geometric Reconstruction Problems} so that I can apply it to real problems and communicate algorithmic insights clearly and reproducibly.}

\NonFunctional{\pill{Reliability} \pill{Reproducibility}}

\BDD{Happy path}{the chapter, examples, and any tooling are available}{I complete the \emph{Hands-on Objective} and validations for this chapter}{the stated outcomes are produced (proof/code/summary) and recorded in the repo with passing checks}

\Meta
\end{StoryCard}

\begin{StoryTasks}
  \item Extract key definitions/lemmas; compile a one-page summary with references to the chapter.
  \item Recover shapes from projections or shadows; study minimal measurement sets.
  \item Quantify reconstruction error (Hausdorff/symmetric difference) on synthetic data.
  \item Discuss identifiability conditions and failure modes.
\end{StoryTasks}

\clearpage
\section*{HDCG-35 --- Curve \& Surface Reconstruction}
\begin{StoryCard}{HDCG-35}{Curve \& Surface Reconstruction}
\storyrow{Epic / Feature}{Algorithms of Fundamental Geometric Objects}
\storyrow{Business Value}{Master core geometric algorithms that power search, reconstruction, meshing, and planning pipelines. (Curve \& Surface Reconstruction).}
\storyrow{Priority / Estimate}{\textbf{Priority:} Must\hfill \textbf{SP:} 3}
\storyrow{Persona}{applied geometry engineer}
\storyrow{Dependencies}{prior chapters as referenced}
\storyrow{Assumptions / Risks}{time to internalize proofs vs. breadth; numerical robustness and degeneracies}

\StoryBody{As a applied geometry engineer, I want to master \textit{Curve \& Surface Reconstruction} so that I can apply it to real problems and communicate algorithmic insights clearly and reproducibly.}

\NonFunctional{\pill{Reliability} \pill{Reproducibility} \pill{Robustness} \pill{Accuracy}}

\BDD{Happy path}{the chapter, examples, and any tooling are available}{I complete the \emph{Hands-on Objective} and validations for this chapter}{the stated outcomes are produced (proof/code/summary) and recorded in the repo with passing checks}

\Meta
\end{StoryCard}

\begin{StoryTasks}
  \item Extract key definitions/lemmas; compile a one-page summary with references to the chapter.
  \item Implement a simple crust/ball-pivoting/Poisson pipeline on noisy samples.
  \item Tune parameters and measure Hausdorff distance to ground truth.
  \item Report topology errors and smoothing trade-offs.
\end{StoryTasks}

\clearpage
\section*{HDCG-36 --- Computational Convexity}
\begin{StoryCard}{HDCG-36}{Computational Convexity}
\storyrow{Epic / Feature}{Algorithms of Fundamental Geometric Objects}
\storyrow{Business Value}{Master core geometric algorithms that power search, reconstruction, meshing, and planning pipelines. (Computational Convexity).}
\storyrow{Priority / Estimate}{\textbf{Priority:} Must\hfill \textbf{SP:} 3}
\storyrow{Persona}{applied geometry engineer}
\storyrow{Dependencies}{prior chapters as referenced}
\storyrow{Assumptions / Risks}{time to internalize proofs vs. breadth}

\StoryBody{As a applied geometry engineer, I want to master \textit{Computational Convexity} so that I can apply it to real problems and communicate algorithmic insights clearly and reproducibly.}

\NonFunctional{\pill{Reliability} \pill{Reproducibility}}

\BDD{Happy path}{the chapter, examples, and any tooling are available}{I complete the \emph{Hands-on Objective} and validations for this chapter}{the stated outcomes are produced (proof/code/summary) and recorded in the repo with passing checks}

\Meta
\end{StoryCard}

\begin{StoryTasks}
  \item Extract key definitions/lemmas; compile a one-page summary with references to the chapter.
  \item Solve membership/separation via cutting-plane or ellipsoid on toy instances.
  \item Compare oracle-based methods vs. explicit H/V representations.
  \item Document complexity and numerical behavior.
\end{StoryTasks}

\clearpage
\section*{HDCG-37 --- Algorithmic Real Algebraic Geometry}
\begin{StoryCard}{HDCG-37}{Algorithmic Real Algebraic Geometry}
\storyrow{Epic / Feature}{Algorithms of Fundamental Geometric Objects}
\storyrow{Business Value}{Master core geometric algorithms that power search, reconstruction, meshing, and planning pipelines. (Algorithmic Real Algebraic Geometry).}
\storyrow{Priority / Estimate}{\textbf{Priority:} Must\hfill \textbf{SP:} 3}
\storyrow{Persona}{applied geometry engineer}
\storyrow{Dependencies}{prior chapters as referenced}
\storyrow{Assumptions / Risks}{time to internalize proofs vs. breadth; numerical robustness and degeneracies}

\StoryBody{As a applied geometry engineer, I want to master \textit{Algorithmic Real Algebraic Geometry} so that I can apply it to real problems and communicate algorithmic insights clearly and reproducibly.}

\NonFunctional{\pill{Reliability} \pill{Reproducibility} \pill{Robustness} \pill{Accuracy}}

\BDD{Happy path}{the chapter, examples, and any tooling are available}{I complete the \emph{Hands-on Objective} and validations for this chapter}{the stated outcomes are produced (proof/code/summary) and recorded in the repo with passing checks}

\Meta
\end{StoryCard}

\begin{StoryTasks}
  \item Extract key definitions/lemmas; compile a one-page summary with references to the chapter.
  \item Experiment with solving small semi-algebraic systems; visualize solution sets.
  \item Use cylindrical algebraic decomposition (CAD) conceptually or via a CAS for toy inputs.
  \item Discuss complexity blowups and practical workarounds.
\end{StoryTasks}

\clearpage
\section*{HDCG-38 --- Point Location}
\begin{StoryCard}{HDCG-38}{Point Location}
\storyrow{Epic / Feature}{Geometric Data Structures \& Searching}
\storyrow{Business Value}{Develop data structures for fast queries, intersections, and proximity at scale. (Point Location).}
\storyrow{Priority / Estimate}{\textbf{Priority:} Must\hfill \textbf{SP:} 3}
\storyrow{Persona}{applied geometry engineer}
\storyrow{Dependencies}{prior chapters as referenced}
\storyrow{Assumptions / Risks}{time to internalize proofs vs. breadth}

\StoryBody{As a applied geometry engineer, I want to master \textit{Point Location} so that I can apply it to real problems and communicate algorithmic insights clearly and reproducibly.}

\NonFunctional{\pill{Reliability} \pill{Reproducibility} \pill{Performance} \pill{Scalability}}

\BDD{Happy path}{the chapter, examples, and any tooling are available}{I complete the \emph{Hands-on Objective} and validations for this chapter}{the stated outcomes are produced (proof/code/summary) and recorded in the repo with passing checks}

\Meta
\end{StoryCard}

\begin{StoryTasks}
  \item Extract key definitions/lemmas; compile a one-page summary with references to the chapter.
  \item Build a randomized trapezoidal map; benchmark query vs. build time.
  \item Test degeneracies and dynamic updates (insertions).
  \item Compare to persistent search structures where applicable.
\end{StoryTasks}

\clearpage
\section*{HDCG-39 --- Collision Detection \& Proximity Queries}
\begin{StoryCard}{HDCG-39}{Collision Detection \& Proximity Queries}
\storyrow{Epic / Feature}{Geometric Data Structures \& Searching}
\storyrow{Business Value}{Develop data structures for fast queries, intersections, and proximity at scale. (Collision Detection \& Proximity Queries).}
\storyrow{Priority / Estimate}{\textbf{Priority:} Must\hfill \textbf{SP:} 3}
\storyrow{Persona}{applied geometry engineer}
\storyrow{Dependencies}{prior chapters as referenced}
\storyrow{Assumptions / Risks}{time to internalize proofs vs. breadth; numerical robustness and degeneracies}

\StoryBody{As a applied geometry engineer, I want to master \textit{Collision Detection \& Proximity Queries} so that I can apply it to real problems and communicate algorithmic insights clearly and reproducibly.}

\NonFunctional{\pill{Reliability} \pill{Reproducibility} \pill{Robustness} \pill{Accuracy}}

\BDD{Happy path}{the chapter, examples, and any tooling are available}{I complete the \emph{Hands-on Objective} and validations for this chapter}{the stated outcomes are produced (proof/code/summary) and recorded in the repo with passing checks}

\Meta
\end{StoryCard}

\begin{StoryTasks}
  \item Extract key definitions/lemmas; compile a one-page summary with references to the chapter.
  \item Implement BVH (AABB/OBB) construction and traversal; add narrow-phase tests.
  \item Evaluate continuous collision detection for moving segments/triangles.
  \item Profiling: queries/sec vs. object count; document worst-case scenes.
\end{StoryTasks}

\clearpage
\section*{HDCG-40 --- Range Searching}
\begin{StoryCard}{HDCG-40}{Range Searching}
\storyrow{Epic / Feature}{Geometric Data Structures \& Searching}
\storyrow{Business Value}{Develop data structures for fast queries, intersections, and proximity at scale. (Range Searching).}
\storyrow{Priority / Estimate}{\textbf{Priority:} Must\hfill \textbf{SP:} 3}
\storyrow{Persona}{applied geometry engineer}
\storyrow{Dependencies}{prior chapters as referenced}
\storyrow{Assumptions / Risks}{time to internalize proofs vs. breadth}

\StoryBody{As a applied geometry engineer, I want to master \textit{Range Searching} so that I can apply it to real problems and communicate algorithmic insights clearly and reproducibly.}

\NonFunctional{\pill{Reliability} \pill{Reproducibility} \pill{Performance} \pill{Scalability}}

\BDD{Happy path}{the chapter, examples, and any tooling are available}{I complete the \emph{Hands-on Objective} and validations for this chapter}{the stated outcomes are produced (proof/code/summary) and recorded in the repo with passing checks}

\Meta
\end{StoryCard}

\begin{StoryTasks}
  \item Extract key definitions/lemmas; compile a one-page summary with references to the chapter.
  \item Implement kd/interval/segment trees; test orthogonal range counting/reporting.
  \item Measure query/update trade-offs; visualize pruning behavior.
  \item Scale experiments to large n and report memory footprints.
\end{StoryTasks}

\clearpage
\section*{HDCG-41 --- Ray Shooting \& Lines in Space}
\begin{StoryCard}{HDCG-41}{Ray Shooting \& Lines in Space}
\storyrow{Epic / Feature}{Geometric Data Structures \& Searching}
\storyrow{Business Value}{Develop data structures for fast queries, intersections, and proximity at scale. (Ray Shooting \& Lines in Space).}
\storyrow{Priority / Estimate}{\textbf{Priority:} Must\hfill \textbf{SP:} 3}
\storyrow{Persona}{applied geometry engineer}
\storyrow{Dependencies}{prior chapters as referenced}
\storyrow{Assumptions / Risks}{time to internalize proofs vs. breadth}

\StoryBody{As a applied geometry engineer, I want to master \textit{Ray Shooting \& Lines in Space} so that I can apply it to real problems and communicate algorithmic insights clearly and reproducibly.}

\NonFunctional{\pill{Reliability} \pill{Reproducibility} \pill{Performance} \pill{Scalability}}

\BDD{Happy path}{the chapter, examples, and any tooling are available}{I complete the \emph{Hands-on Objective} and validations for this chapter}{the stated outcomes are produced (proof/code/summary) and recorded in the repo with passing checks}

\Meta
\end{StoryCard}

\begin{StoryTasks}
  \item Extract key definitions/lemmas; compile a one-page summary with references to the chapter.
  \item Implement ray--scene intersection with uniform grid and BVH; compare.
  \item Validate against analytic scenes; collect miss/hit statistics.
  \item Profile coherent vs. incoherent rays.
\end{StoryTasks}

\clearpage
\section*{HDCG-42 --- Geometric Intersection}
\begin{StoryCard}{HDCG-42}{Geometric Intersection}
\storyrow{Epic / Feature}{Geometric Data Structures \& Searching}
\storyrow{Business Value}{Develop data structures for fast queries, intersections, and proximity at scale. (Geometric Intersection).}
\storyrow{Priority / Estimate}{\textbf{Priority:} Must\hfill \textbf{SP:} 3}
\storyrow{Persona}{applied geometry engineer}
\storyrow{Dependencies}{prior chapters as referenced}
\storyrow{Assumptions / Risks}{time to internalize proofs vs. breadth; numerical robustness and degeneracies}

\StoryBody{As a applied geometry engineer, I want to master \textit{Geometric Intersection} so that I can apply it to real problems and communicate algorithmic insights clearly and reproducibly.}

\NonFunctional{\pill{Reliability} \pill{Reproducibility} \pill{Performance} \pill{Scalability}}

\BDD{Happy path}{the chapter, examples, and any tooling are available}{I complete the \emph{Hands-on Objective} and validations for this chapter}{the stated outcomes are produced (proof/code/summary) and recorded in the repo with passing checks}

\Meta
\end{StoryCard}

\begin{StoryTasks}
  \item Extract key definitions/lemmas; compile a one-page summary with references to the chapter.
  \item Write exact segment/triangle intersection predicates; fuzz edge cases.
  \item Build sweep-line for segment intersection; report complexity and statistics.
  \item Summarize robustness fixes (epsilon vs. exact arithmetic).
\end{StoryTasks}

\clearpage
\section*{HDCG-43 --- Nearest Neighbors in High Dimension}
\begin{StoryCard}{HDCG-43}{Nearest Neighbors in High Dimension}
\storyrow{Epic / Feature}{Geometric Data Structures \& Searching}
\storyrow{Business Value}{Develop data structures for fast queries, intersections, and proximity at scale. (Nearest Neighbors in High Dimension).}
\storyrow{Priority / Estimate}{\textbf{Priority:} Must\hfill \textbf{SP:} 3}
\storyrow{Persona}{applied geometry engineer}
\storyrow{Dependencies}{prior chapters as referenced}
\storyrow{Assumptions / Risks}{time to internalize proofs vs. breadth; curse of dimensionality; memory pressure}

\StoryBody{As a applied geometry engineer, I want to master \textit{Nearest Neighbors in High Dimension} so that I can apply it to real problems and communicate algorithmic insights clearly and reproducibly.}

\NonFunctional{\pill{Reliability} \pill{Reproducibility} \pill{Performance} \pill{Scalability}}

\BDD{Happy path}{the chapter, examples, and any tooling are available}{I complete the \emph{Hands-on Objective} and validations for this chapter}{the stated outcomes are produced (proof/code/summary) and recorded in the repo with passing checks}

\Meta
\end{StoryCard}

\begin{StoryTasks}
  \item Extract key definitions/lemmas; compile a one-page summary with references to the chapter.
  \item Compare kd-tree, LSH, and HNSW on real vector data; tune parameters.
  \item Measure recall/latency trade-offs; draw accuracy--speed curves.
  \item Discuss curse-of-dimensionality and mitigation strategies.
\end{StoryTasks}

\clearpage
\section*{HDCG-44 --- Randomization \& Derandomization in Geometry}
\begin{StoryCard}{HDCG-44}{Randomization \& Derandomization in Geometry}
\storyrow{Epic / Feature}{Computational Techniques}
\storyrow{Business Value}{Adopt practical computation techniques for speed, robustness, generalization, and summaries. (Randomization \& Derandomization in Geometry).}
\storyrow{Priority / Estimate}{\textbf{Priority:} Must\hfill \textbf{SP:} 3}
\storyrow{Persona}{applied geometry engineer}
\storyrow{Dependencies}{prior chapters as referenced}
\storyrow{Assumptions / Risks}{time to internalize proofs vs. breadth; variance in experiments; need for many trials}

\StoryBody{As a applied geometry engineer, I want to master \textit{Randomization \& Derandomization in Geometry} so that I can apply it to real problems and communicate algorithmic insights clearly and reproducibly.}

\NonFunctional{\pill{Reliability} \pill{Reproducibility}}

\BDD{Happy path}{the chapter, examples, and any tooling are available}{I complete the \emph{Hands-on Objective} and validations for this chapter}{the stated outcomes are produced (proof/code/summary) and recorded in the repo with passing checks}

\Meta
\end{StoryCard}

\begin{StoryTasks}
  \item Extract key definitions/lemmas; compile a one-page summary with references to the chapter.
  \item Implement random sampling/\(\varepsilon\)-nets for a range space; validate bounds empirically.
  \item Replace with a deterministic construction (derandomization) on small cases.
  \item Compare quality and runtime for both approaches.
\end{StoryTasks}

\clearpage
\section*{HDCG-45 --- Robust Geometric Computation}
\begin{StoryCard}{HDCG-45}{Robust Geometric Computation}
\storyrow{Epic / Feature}{Computational Techniques}
\storyrow{Business Value}{Adopt practical computation techniques for speed, robustness, generalization, and summaries. (Robust Geometric Computation).}
\storyrow{Priority / Estimate}{\textbf{Priority:} Must\hfill \textbf{SP:} 3}
\storyrow{Persona}{applied geometry engineer}
\storyrow{Dependencies}{prior chapters as referenced}
\storyrow{Assumptions / Risks}{time to internalize proofs vs. breadth; numerical robustness and degeneracies}

\StoryBody{As a applied geometry engineer, I want to master \textit{Robust Geometric Computation} so that I can apply it to real problems and communicate algorithmic insights clearly and reproducibly.}

\NonFunctional{\pill{Reliability} \pill{Reproducibility} \pill{Robustness} \pill{Accuracy}}

\BDD{Happy path}{the chapter, examples, and any tooling are available}{I complete the \emph{Hands-on Objective} and validations for this chapter}{the stated outcomes are produced (proof/code/summary) and recorded in the repo with passing checks}

\Meta
\end{StoryCard}

\begin{StoryTasks}
  \item Extract key definitions/lemmas; compile a one-page summary with references to the chapter.
  \item Integrate exact/filtered predicates into one pipeline; catalog failures with floats.
  \item Re-run earlier algorithms (hull, Delaunay, intersections) under robustness modes.
  \item Summarize cost of robustness vs. correctness benefits.
\end{StoryTasks}

\clearpage
\section*{HDCG-46 --- Parallel Algorithms in Geometry}
\begin{StoryCard}{HDCG-46}{Parallel Algorithms in Geometry}
\storyrow{Epic / Feature}{Computational Techniques}
\storyrow{Business Value}{Adopt practical computation techniques for speed, robustness, generalization, and summaries. (Parallel Algorithms in Geometry).}
\storyrow{Priority / Estimate}{\textbf{Priority:} Must\hfill \textbf{SP:} 3}
\storyrow{Persona}{applied geometry engineer}
\storyrow{Dependencies}{prior chapters as referenced, task parallelism basics, threading model}
\storyrow{Assumptions / Risks}{time to internalize proofs vs. breadth}

\StoryBody{As a applied geometry engineer, I want to master \textit{Parallel Algorithms in Geometry} so that I can apply it to real problems and communicate algorithmic insights clearly and reproducibly.}

\NonFunctional{\pill{Reliability} \pill{Reproducibility} \pill{Performance} \pill{Scalability}}

\BDD{Happy path}{the chapter, examples, and any tooling are available}{I complete the \emph{Hands-on Objective} and validations for this chapter}{the stated outcomes are produced (proof/code/summary) and recorded in the repo with passing checks}

\Meta
\end{StoryCard}

\begin{StoryTasks}
  \item Extract key definitions/lemmas; compile a one-page summary with references to the chapter.
  \item Parallelize a hull or Delaunay implementation; measure speedup and scalability.
  \item Identify contention hotspots; propose work partitioning.
  \item Test on varying cores; produce a scalability plot.
\end{StoryTasks}

\clearpage
\section*{HDCG-47 --- \(\varepsilon\)-nets \& \(\varepsilon\)-approximations}
\begin{StoryCard}{HDCG-47}{\(\varepsilon\)-nets \& \(\varepsilon\)-approximations}
\storyrow{Epic / Feature}{Computational Techniques}
\storyrow{Business Value}{Adopt practical computation techniques for speed, robustness, generalization, and summaries. (\(\varepsilon\)-nets \& \(\varepsilon\)-approximations).}
\storyrow{Priority / Estimate}{\textbf{Priority:} Must\hfill \textbf{SP:} 3}
\storyrow{Persona}{applied geometry engineer}
\storyrow{Dependencies}{prior chapters as referenced}
\storyrow{Assumptions / Risks}{time to internalize proofs vs. breadth}

\StoryBody{As a applied geometry engineer, I want to master \textit{\(\varepsilon\)-nets \& \(\varepsilon\)-approximations} so that I can apply it to real problems and communicate algorithmic insights clearly and reproducibly.}

\NonFunctional{\pill{Reliability} \pill{Reproducibility}}

\BDD{Happy path}{the chapter, examples, and any tooling are available}{I complete the \emph{Hands-on Objective} and validations for this chapter}{the stated outcomes are produced (proof/code/summary) and recorded in the repo with passing checks}

\Meta
\end{StoryCard}

\begin{StoryTasks}
  \item Extract key definitions/lemmas; compile a one-page summary with references to the chapter.
  \item Construct \(\varepsilon\)-nets/approximations for basic range spaces; verify hitting/approximation properties.
  \item Relate to VC-dimension; compute sample sizes for target \(\varepsilon\), \(\delta\).
  \item Apply to a small learning-like problem (set cover/active sampling).
\end{StoryTasks}

\clearpage
\section*{HDCG-48 --- Coresets \& Sketches}
\begin{StoryCard}{HDCG-48}{Coresets \& Sketches}
\storyrow{Epic / Feature}{Computational Techniques}
\storyrow{Business Value}{Adopt practical computation techniques for speed, robustness, generalization, and summaries. (Coresets \& Sketches).}
\storyrow{Priority / Estimate}{\textbf{Priority:} Must\hfill \textbf{SP:} 3}
\storyrow{Persona}{applied geometry engineer}
\storyrow{Dependencies}{prior chapters as referenced}
\storyrow{Assumptions / Risks}{time to internalize proofs vs. breadth}

\StoryBody{As a applied geometry engineer, I want to master \textit{Coresets \& Sketches} so that I can apply it to real problems and communicate algorithmic insights clearly and reproducibly.}

\NonFunctional{\pill{Reliability} \pill{Reproducibility} \pill{Performance} \pill{Scalability}}

\BDD{Happy path}{the chapter, examples, and any tooling are available}{I complete the \emph{Hands-on Objective} and validations for this chapter}{the stated outcomes are produced (proof/code/summary) and recorded in the repo with passing checks}

\Meta
\end{StoryCard}

\begin{StoryTasks}
  \item Extract key definitions/lemmas; compile a one-page summary with references to the chapter.
  \item Build k-means/median coresets; evaluate clustering error vs. exact.
  \item Profile construction time vs. coreset size; plot trade-offs.
  \item Demonstrate downstream speedups with negligible loss.
\end{StoryTasks}

\clearpage
\section*{HDCG-49 --- Linear Programming (Low-dimensional \& Randomized)}
\begin{StoryCard}{HDCG-49}{Linear Programming (Low-dimensional \& Randomized)}
\storyrow{Epic / Feature}{Applications of Discrete \& Computational Geometry}
\storyrow{Business Value}{Apply geometric concepts to real domains to deliver measurable outcomes and demos. (Linear Programming (Low-dimensional \& Randomized)).}
\storyrow{Priority / Estimate}{\textbf{Priority:} Must\hfill \textbf{SP:} 3}
\storyrow{Persona}{practitioner building geometry-driven applications}
\storyrow{Dependencies}{prior chapters as referenced}
\storyrow{Assumptions / Risks}{time to internalize proofs vs. breadth; variance in experiments; need for many trials}

\StoryBody{As a practitioner building geometry-driven applications, I want to master \textit{Linear Programming (Low-dimensional \& Randomized)} so that I can apply it to real problems and communicate theoretical insights clearly and reproducibly.}

\NonFunctional{\pill{Reliability} \pill{Reproducibility} \pill{Performance} \pill{Scalability}}

\BDD{Happy path}{the chapter, examples, and any tooling are available}{I complete the \emph{Hands-on Objective} and validations for this chapter}{the stated outcomes are produced (proof/code/summary) and recorded in the repo with passing checks}

\Meta
\end{StoryCard}

\begin{StoryTasks}
  \item Extract key definitions/lemmas; compile a one-page summary with references to the chapter.
  \item Implement randomized incremental LP in low dimension; visualize feasible region.
  \item Benchmark vs. simplex on small test sets; record degeneracy behavior.
  \item Apply to a geometric optimization mini-problem (smallest enclosing ball).
\end{StoryTasks}

\clearpage
\section*{HDCG-50 --- Algorithmic Motion Planning}
\begin{StoryCard}{HDCG-50}{Algorithmic Motion Planning}
\storyrow{Epic / Feature}{Applications of Discrete \& Computational Geometry}
\storyrow{Business Value}{Apply geometric concepts to real domains to deliver measurable outcomes and demos. (Algorithmic Motion Planning).}
\storyrow{Priority / Estimate}{\textbf{Priority:} Must\hfill \textbf{SP:} 3}
\storyrow{Persona}{practitioner building geometry-driven applications}
\storyrow{Dependencies}{prior chapters as referenced}
\storyrow{Assumptions / Risks}{time to internalize proofs vs. breadth}

\StoryBody{As a practitioner building geometry-driven applications, I want to master \textit{Algorithmic Motion Planning} so that I can apply it to real problems and communicate theoretical insights clearly and reproducibly.}

\NonFunctional{\pill{Reliability} \pill{Reproducibility}}

\BDD{Happy path}{the chapter, examples, and any tooling are available}{I complete the \emph{Hands-on Objective} and validations for this chapter}{the stated outcomes are produced (proof/code/summary) and recorded in the repo with passing checks}

\Meta
\end{StoryCard}

\begin{StoryTasks}
  \item Extract key definitions/lemmas; compile a one-page summary with references to the chapter.
  \item Implement PRM/RRT on a 2D environment; measure coverage and path quality.
  \item Add collision checks via BVH; compare planners across seeds.
  \item Export paths and visualize milestones/edges.
\end{StoryTasks}

\clearpage
\section*{HDCG-51 --- Robotics: Configuration Spaces}
\begin{StoryCard}{HDCG-51}{Robotics: Configuration Spaces}
\storyrow{Epic / Feature}{Applications of Discrete \& Computational Geometry}
\storyrow{Business Value}{Apply geometric concepts to real domains to deliver measurable outcomes and demos. (Robotics: Configuration Spaces).}
\storyrow{Priority / Estimate}{\textbf{Priority:} Must\hfill \textbf{SP:} 3}
\storyrow{Persona}{practitioner building geometry-driven applications}
\storyrow{Dependencies}{prior chapters as referenced}
\storyrow{Assumptions / Risks}{time to internalize proofs vs. breadth}

\StoryBody{As a practitioner building geometry-driven applications, I want to master \textit{Robotics: Configuration Spaces} so that I can apply it to real problems and communicate theoretical insights clearly and reproducibly.}

\NonFunctional{\pill{Reliability} \pill{Reproducibility}}

\BDD{Happy path}{the chapter, examples, and any tooling are available}{I complete the \emph{Hands-on Objective} and validations for this chapter}{the stated outcomes are produced (proof/code/summary) and recorded in the repo with passing checks}

\Meta
\end{StoryCard}

\begin{StoryTasks}
  \item Extract key definitions/lemmas; compile a one-page summary with references to the chapter.
  \item Model a planar arm's configuration space with obstacles; compute free space components.
  \item Plan a collision-free path; validate in a simple simulator.
  \item Discuss DOF scaling and sampling strategies.
\end{StoryTasks}

\clearpage
\section*{HDCG-52 --- Computer Graphics: Geometric Pipelines}
\begin{StoryCard}{HDCG-52}{Computer Graphics: Geometric Pipelines}
\storyrow{Epic / Feature}{Applications of Discrete \& Computational Geometry}
\storyrow{Business Value}{Apply geometric concepts to real domains to deliver measurable outcomes and demos. (Computer Graphics: Geometric Pipelines).}
\storyrow{Priority / Estimate}{\textbf{Priority:} Must\hfill \textbf{SP:} 3}
\storyrow{Persona}{practitioner building geometry-driven applications}
\storyrow{Dependencies}{prior chapters as referenced}
\storyrow{Assumptions / Risks}{time to internalize proofs vs. breadth}

\StoryBody{As a practitioner building geometry-driven applications, I want to master \textit{Computer Graphics: Geometric Pipelines} so that I can apply it to real problems and communicate theoretical insights clearly and reproducibly.}

\NonFunctional{\pill{Reliability} \pill{Reproducibility} \pill{Usability}}

\BDD{Happy path}{the chapter, examples, and any tooling are available}{I complete the \emph{Hands-on Objective} and validations for this chapter}{the stated outcomes are produced (proof/code/summary) and recorded in the repo with passing checks}

\Meta
\end{StoryCard}

\begin{StoryTasks}
  \item Extract key definitions/lemmas; compile a one-page summary with references to the chapter.
  \item Implement geometric clipping and rasterization of a simple scene.
  \item Run mesh simplification; report quality vs. decimation.
  \item Profile the pipeline stages you implemented.
\end{StoryTasks}

\clearpage
\section*{HDCG-53 --- Modeling Motion (Rigid \& Affine)}
\begin{StoryCard}{HDCG-53}{Modeling Motion (Rigid \& Affine)}
\storyrow{Epic / Feature}{Applications of Discrete \& Computational Geometry}
\storyrow{Business Value}{Apply geometric concepts to real domains to deliver measurable outcomes and demos. (Modeling Motion (Rigid \& Affine)).}
\storyrow{Priority / Estimate}{\textbf{Priority:} Must\hfill \textbf{SP:} 3}
\storyrow{Persona}{practitioner building geometry-driven applications}
\storyrow{Dependencies}{prior chapters as referenced}
\storyrow{Assumptions / Risks}{time to internalize proofs vs. breadth}

\StoryBody{As a practitioner building geometry-driven applications, I want to master \textit{Modeling Motion (Rigid \& Affine)} so that I can apply it to real problems and communicate theoretical insights clearly and reproducibly.}

\NonFunctional{\pill{Reliability} \pill{Reproducibility}}

\BDD{Happy path}{the chapter, examples, and any tooling are available}{I complete the \emph{Hands-on Objective} and validations for this chapter}{the stated outcomes are produced (proof/code/summary) and recorded in the repo with passing checks}

\Meta
\end{StoryCard}

\begin{StoryTasks}
  \item Extract key definitions/lemmas; compile a one-page summary with references to the chapter.
  \item Derive rigid/affine transforms; implement screw interpolation demo.
  \item Track errors under concatenation; verify invariants.
  \item Compare different interpolation schemes for stability.
\end{StoryTasks}

\clearpage
\section*{HDCG-54 --- Pattern Recognition (Geometric View)}
\begin{StoryCard}{HDCG-54}{Pattern Recognition (Geometric View)}
\storyrow{Epic / Feature}{Applications of Discrete \& Computational Geometry}
\storyrow{Business Value}{Apply geometric concepts to real domains to deliver measurable outcomes and demos. (Pattern Recognition (Geometric View)).}
\storyrow{Priority / Estimate}{\textbf{Priority:} Must\hfill \textbf{SP:} 3}
\storyrow{Persona}{practitioner building geometry-driven applications}
\storyrow{Dependencies}{prior chapters as referenced}
\storyrow{Assumptions / Risks}{time to internalize proofs vs. breadth}

\StoryBody{As a practitioner building geometry-driven applications, I want to master \textit{Pattern Recognition (Geometric View)} so that I can apply it to real problems and communicate theoretical insights clearly and reproducibly.}

\NonFunctional{\pill{Reliability} \pill{Reproducibility}}

\BDD{Happy path}{the chapter, examples, and any tooling are available}{I complete the \emph{Hands-on Objective} and validations for this chapter}{the stated outcomes are produced (proof/code/summary) and recorded in the repo with passing checks}

\Meta
\end{StoryCard}

\begin{StoryTasks}
  \item Extract key definitions/lemmas; compile a one-page summary with references to the chapter.
  \item Implement geometric classifiers (nearest-center/Voronoi) on a toy dataset.
  \item Compare to an SVM baseline; report decision boundary shapes.
  \item Analyze robustness to outliers using geometric medians.
\end{StoryTasks}

\clearpage
\section*{HDCG-55 --- Graph Drawing}
\begin{StoryCard}{HDCG-55}{Graph Drawing}
\storyrow{Epic / Feature}{Applications of Discrete \& Computational Geometry}
\storyrow{Business Value}{Apply geometric concepts to real domains to deliver measurable outcomes and demos. (Graph Drawing).}
\storyrow{Priority / Estimate}{\textbf{Priority:} Must\hfill \textbf{SP:} 3}
\storyrow{Persona}{practitioner building geometry-driven applications}
\storyrow{Dependencies}{prior chapters as referenced}
\storyrow{Assumptions / Risks}{time to internalize proofs vs. breadth}

\StoryBody{As a practitioner building geometry-driven applications, I want to master \textit{Graph Drawing} so that I can apply it to real problems and communicate theoretical insights clearly and reproducibly.}

\NonFunctional{\pill{Reliability} \pill{Reproducibility} \pill{Usability}}

\BDD{Happy path}{the chapter, examples, and any tooling are available}{I complete the \emph{Hands-on Objective} and validations for this chapter}{the stated outcomes are produced (proof/code/summary) and recorded in the repo with passing checks}

\Meta
\end{StoryCard}

\begin{StoryTasks}
  \item Extract key definitions/lemmas; compile a one-page summary with references to the chapter.
  \item Implement force-directed layout with planarity constraints where possible.
  \item Test layered/Sugiyama layout; manage crossings.
  \item Quantify edge length variance and crossing counts.
\end{StoryTasks}

\clearpage
\section*{HDCG-56 --- Splines \& Geometric Modeling}
\begin{StoryCard}{HDCG-56}{Splines \& Geometric Modeling}
\storyrow{Epic / Feature}{Applications of Discrete \& Computational Geometry}
\storyrow{Business Value}{Apply geometric concepts to real domains to deliver measurable outcomes and demos. (Splines \& Geometric Modeling).}
\storyrow{Priority / Estimate}{\textbf{Priority:} Must\hfill \textbf{SP:} 3}
\storyrow{Persona}{practitioner building geometry-driven applications}
\storyrow{Dependencies}{prior chapters as referenced}
\storyrow{Assumptions / Risks}{time to internalize proofs vs. breadth}

\StoryBody{As a practitioner building geometry-driven applications, I want to master \textit{Splines \& Geometric Modeling} so that I can apply it to real problems and communicate theoretical insights clearly and reproducibly.}

\NonFunctional{\pill{Reliability} \pill{Reproducibility}}

\BDD{Happy path}{the chapter, examples, and any tooling are available}{I complete the \emph{Hands-on Objective} and validations for this chapter}{the stated outcomes are produced (proof/code/summary) and recorded in the repo with passing checks}

\Meta
\end{StoryCard}

\begin{StoryTasks}
  \item Extract key definitions/lemmas; compile a one-page summary with references to the chapter.
  \item Implement B-spline/NURBS evaluation; verify C1/C2 continuity on examples.
  \item Fit curves to sample points; measure error sensitivity.
  \item Render and annotate control polygon effects.
\end{StoryTasks}

\clearpage
\section*{HDCG-57 --- Solid Modeling (B-Rep \& CSG)}
\begin{StoryCard}{HDCG-57}{Solid Modeling (B-Rep \& CSG)}
\storyrow{Epic / Feature}{Applications of Discrete \& Computational Geometry}
\storyrow{Business Value}{Apply geometric concepts to real domains to deliver measurable outcomes and demos. (Solid Modeling (B-Rep \& CSG)).}
\storyrow{Priority / Estimate}{\textbf{Priority:} Must\hfill \textbf{SP:} 3}
\storyrow{Persona}{practitioner building geometry-driven applications}
\storyrow{Dependencies}{prior chapters as referenced}
\storyrow{Assumptions / Risks}{time to internalize proofs vs. breadth; numerical robustness and degeneracies}

\StoryBody{As a practitioner building geometry-driven applications, I want to master \textit{Solid Modeling (B-Rep \& CSG)} so that I can apply it to real problems and communicate theoretical insights clearly and reproducibly.}

\NonFunctional{\pill{Reliability} \pill{Reproducibility} \pill{Robustness} \pill{Accuracy}}

\BDD{Happy path}{the chapter, examples, and any tooling are available}{I complete the \emph{Hands-on Objective} and validations for this chapter}{the stated outcomes are produced (proof/code/summary) and recorded in the repo with passing checks}

\Meta
\end{StoryCard}

\begin{StoryTasks}
  \item Extract key definitions/lemmas; compile a one-page summary with references to the chapter.
  \item Convert between CSG and B-rep for simple solids; implement Boolean ops.
  \item Detect and repair non-manifold issues; validate watertightness.
  \item Export to a CAD-friendly format and roundtrip.
\end{StoryTasks}

\clearpage
\section*{HDCG-58 --- Robust Statistics: Data Depth \& Medians}
\begin{StoryCard}{HDCG-58}{Robust Statistics: Data Depth \& Medians}
\storyrow{Epic / Feature}{Applications of Discrete \& Computational Geometry}
\storyrow{Business Value}{Apply geometric concepts to real domains to deliver measurable outcomes and demos. (Robust Statistics: Data Depth \& Medians).}
\storyrow{Priority / Estimate}{\textbf{Priority:} Must\hfill \textbf{SP:} 3}
\storyrow{Persona}{data scientist working with geometric methods}
\storyrow{Dependencies}{prior chapters as referenced}
\storyrow{Assumptions / Risks}{time to internalize proofs vs. breadth; numerical robustness and degeneracies}

\StoryBody{As a data scientist working with geometric methods, I want to master \textit{Robust Statistics: Data Depth \& Medians} so that I can apply it to real problems and communicate theoretical insights clearly and reproducibly.}

\NonFunctional{\pill{Reliability} \pill{Reproducibility} \pill{Robustness} \pill{Accuracy}}

\BDD{Happy path}{the chapter, examples, and any tooling are available}{I complete the \emph{Hands-on Objective} and validations for this chapter}{the stated outcomes are produced (proof/code/summary) and recorded in the repo with passing checks}

\Meta
\end{StoryCard}

\begin{StoryTasks}
  \item Extract key definitions/lemmas; compile a one-page summary with references to the chapter.
  \item Compute Tukey depth and halfspace medians on 2D data; visualize contours.
  \item Compare robust vs. least-squares fits under outliers.
  \item Report breakdown points and runtime.
\end{StoryTasks}

\clearpage
\section*{HDCG-59 --- Geographic Information Systems (GIS)}
\begin{StoryCard}{HDCG-59}{Geographic Information Systems (GIS)}
\storyrow{Epic / Feature}{Applications of Discrete \& Computational Geometry}
\storyrow{Business Value}{Apply geometric concepts to real domains to deliver measurable outcomes and demos. (Geographic Information Systems (GIS)).}
\storyrow{Priority / Estimate}{\textbf{Priority:} Must\hfill \textbf{SP:} 3}
\storyrow{Persona}{practitioner building geometry-driven applications}
\storyrow{Dependencies}{prior chapters as referenced, map projections, spherical geometry basics}
\storyrow{Assumptions / Risks}{time to internalize proofs vs. breadth; numerical robustness and degeneracies}

\StoryBody{As a practitioner building geometry-driven applications, I want to master \textit{Geographic Information Systems (GIS)} so that I can apply it to real problems and communicate theoretical insights clearly and reproducibly.}

\NonFunctional{\pill{Reliability} \pill{Reproducibility} \pill{Robustness} \pill{Accuracy}}

\BDD{Happy path}{the chapter, examples, and any tooling are available}{I complete the \emph{Hands-on Objective} and validations for this chapter}{the stated outcomes are produced (proof/code/summary) and recorded in the repo with passing checks}

\Meta
\end{StoryCard}

\begin{StoryTasks}
  \item Extract key definitions/lemmas; compile a one-page summary with references to the chapter.
  \item Implement point-in-polygon on geodesic coordinates; handle holes and winding.
  \item Project between WGS84 and a planar CRS; quantify distortion.
  \item Run spatial joins at scale; profile performance.
\end{StoryTasks}

\clearpage
\section*{HDCG-60 --- Grassmann--Cayley Algebra}
\begin{StoryCard}{HDCG-60}{Grassmann--Cayley Algebra}
\storyrow{Epic / Feature}{Applications of Discrete \& Computational Geometry}
\storyrow{Business Value}{Apply geometric concepts to real domains to deliver measurable outcomes and demos. (Grassmann--Cayley Algebra).}
\storyrow{Priority / Estimate}{\textbf{Priority:} Must\hfill \textbf{SP:} 3}
\storyrow{Persona}{practitioner building geometry-driven applications}
\storyrow{Dependencies}{prior chapters as referenced}
\storyrow{Assumptions / Risks}{time to internalize proofs vs. breadth}

\StoryBody{As a practitioner building geometry-driven applications, I want to master \textit{Grassmann--Cayley Algebra} so that I can apply it to real problems and communicate theoretical insights clearly and reproducibly.}

\NonFunctional{\pill{Reliability} \pill{Reproducibility}}

\BDD{Happy path}{the chapter, examples, and any tooling are available}{I complete the \emph{Hands-on Objective} and validations for this chapter}{the stated outcomes are produced (proof/code/summary) and recorded in the repo with passing checks}

\Meta
\end{StoryCard}

\begin{StoryTasks}
  \item Extract key definitions/lemmas; compile a one-page summary with references to the chapter.
  \item Represent joins/meets using Grassmann--Cayley algebra; solve a projective incidence problem.
  \item Work a small numerical example to verify identities.
  \item Document where this algebra simplifies proofs.
\end{StoryTasks}

\clearpage
\section*{HDCG-61 --- Rigidity \& Scene Analysis}
\begin{StoryCard}{HDCG-61}{Rigidity \& Scene Analysis}
\storyrow{Epic / Feature}{Applications of Discrete \& Computational Geometry}
\storyrow{Business Value}{Apply geometric concepts to real domains to deliver measurable outcomes and demos. (Rigidity \& Scene Analysis).}
\storyrow{Priority / Estimate}{\textbf{Priority:} Must\hfill \textbf{SP:} 3}
\storyrow{Persona}{practitioner building geometry-driven applications}
\storyrow{Dependencies}{prior chapters as referenced}
\storyrow{Assumptions / Risks}{time to internalize proofs vs. breadth}

\StoryBody{As a practitioner building geometry-driven applications, I want to master \textit{Rigidity \& Scene Analysis} so that I can apply it to real problems and communicate theoretical insights clearly and reproducibly.}

\NonFunctional{\pill{Reliability} \pill{Reproducibility}}

\BDD{Happy path}{the chapter, examples, and any tooling are available}{I complete the \emph{Hands-on Objective} and validations for this chapter}{the stated outcomes are produced (proof/code/summary) and recorded in the repo with passing checks}

\Meta
\end{StoryCard}

\begin{StoryTasks}
  \item Extract key definitions/lemmas; compile a one-page summary with references to the chapter.
  \item Test bar-and-joint rigidity via rank conditions; visualize stresses.
  \item Generate minimally rigid (Laman) graphs; perturb to test stability.
  \item Relate rigidity to structure-from-motion intuition.
\end{StoryTasks}

\clearpage
\section*{HDCG-62 --- Rigidity of Symmetric Frameworks}
\begin{StoryCard}{HDCG-62}{Rigidity of Symmetric Frameworks}
\storyrow{Epic / Feature}{Applications of Discrete \& Computational Geometry}
\storyrow{Business Value}{Apply geometric concepts to real domains to deliver measurable outcomes and demos. (Rigidity of Symmetric Frameworks).}
\storyrow{Priority / Estimate}{\textbf{Priority:} Must\hfill \textbf{SP:} 3}
\storyrow{Persona}{practitioner building geometry-driven applications}
\storyrow{Dependencies}{prior chapters as referenced}
\storyrow{Assumptions / Risks}{time to internalize proofs vs. breadth}

\StoryBody{As a practitioner building geometry-driven applications, I want to master \textit{Rigidity of Symmetric Frameworks} so that I can apply it to real problems and communicate theoretical insights clearly and reproducibly.}

\NonFunctional{\pill{Reliability} \pill{Reproducibility}}

\BDD{Happy path}{the chapter, examples, and any tooling are available}{I complete the \emph{Hands-on Objective} and validations for this chapter}{the stated outcomes are produced (proof/code/summary) and recorded in the repo with passing checks}

\Meta
\end{StoryCard}

\begin{StoryTasks}
  \item Extract key definitions/lemmas; compile a one-page summary with references to the chapter.
  \item Use group actions to adapt rigidity counts; build symmetric examples.
  \item Check how symmetry changes generic rigidity; demonstrate a case study.
  \item Summarize computational implications.
\end{StoryTasks}

\clearpage
\section*{HDCG-63 --- Global Rigidity}
\begin{StoryCard}{HDCG-63}{Global Rigidity}
\storyrow{Epic / Feature}{Applications of Discrete \& Computational Geometry}
\storyrow{Business Value}{Apply geometric concepts to real domains to deliver measurable outcomes and demos. (Global Rigidity).}
\storyrow{Priority / Estimate}{\textbf{Priority:} Must\hfill \textbf{SP:} 3}
\storyrow{Persona}{practitioner building geometry-driven applications}
\storyrow{Dependencies}{prior chapters as referenced}
\storyrow{Assumptions / Risks}{time to internalize proofs vs. breadth}

\StoryBody{As a practitioner building geometry-driven applications, I want to master \textit{Global Rigidity} so that I can apply it to real problems and communicate theoretical insights clearly and reproducibly.}

\NonFunctional{\pill{Reliability} \pill{Reproducibility}}

\BDD{Happy path}{the chapter, examples, and any tooling are available}{I complete the \emph{Hands-on Objective} and validations for this chapter}{the stated outcomes are produced (proof/code/summary) and recorded in the repo with passing checks}

\Meta
\end{StoryCard}

\begin{StoryTasks}
  \item Extract key definitions/lemmas; compile a one-page summary with references to the chapter.
  \item Differentiate local vs. global rigidity; check generic conditions on small graphs.
  \item Search for counterexamples; visualize multiple embeddings.
  \item Note algorithmic challenges and complexity hints.
\end{StoryTasks}

\clearpage
\section*{HDCG-64 --- Crystals: Periodic \& Aperiodic Structures}
\begin{StoryCard}{HDCG-64}{Crystals: Periodic \& Aperiodic Structures}
\storyrow{Epic / Feature}{Applications of Discrete \& Computational Geometry}
\storyrow{Business Value}{Apply geometric concepts to real domains to deliver measurable outcomes and demos. (Crystals: Periodic \& Aperiodic Structures).}
\storyrow{Priority / Estimate}{\textbf{Priority:} Must\hfill \textbf{SP:} 3}
\storyrow{Persona}{practitioner building geometry-driven applications}
\storyrow{Dependencies}{prior chapters as referenced}
\storyrow{Assumptions / Risks}{time to internalize proofs vs. breadth}

\StoryBody{As a practitioner building geometry-driven applications, I want to master \textit{Crystals: Periodic \& Aperiodic Structures} so that I can apply it to real problems and communicate theoretical insights clearly and reproducibly.}

\NonFunctional{\pill{Reliability} \pill{Reproducibility}}

\BDD{Happy path}{the chapter, examples, and any tooling are available}{I complete the \emph{Hands-on Objective} and validations for this chapter}{the stated outcomes are produced (proof/code/summary) and recorded in the repo with passing checks}

\Meta
\end{StoryCard}

\begin{StoryTasks}
  \item Extract key definitions/lemmas; compile a one-page summary with references to the chapter.
  \item Generate periodic nets; compute basic invariants and visualize unit cells.
  \item Contrast with an aperiodic example; discuss diffraction-like signatures.
  \item Explore stability under perturbations.
\end{StoryTasks}

\clearpage
\section*{HDCG-65 --- Structural Molecular Biology (Distance Geometry)}
\begin{StoryCard}{HDCG-65}{Structural Molecular Biology (Distance Geometry)}
\storyrow{Epic / Feature}{Applications of Discrete \& Computational Geometry}
\storyrow{Business Value}{Apply geometric concepts to real domains to deliver measurable outcomes and demos. (Structural Molecular Biology (Distance Geometry)).}
\storyrow{Priority / Estimate}{\textbf{Priority:} Must\hfill \textbf{SP:} 3}
\storyrow{Persona}{practitioner building geometry-driven applications}
\storyrow{Dependencies}{prior chapters as referenced}
\storyrow{Assumptions / Risks}{time to internalize proofs vs. breadth; numerical robustness and degeneracies}

\StoryBody{As a practitioner building geometry-driven applications, I want to master \textit{Structural Molecular Biology (Distance Geometry)} so that I can apply it to real problems and communicate theoretical insights clearly and reproducibly.}

\NonFunctional{\pill{Reliability} \pill{Reproducibility} \pill{Robustness} \pill{Accuracy}}

\BDD{Happy path}{the chapter, examples, and any tooling are available}{I complete the \emph{Hands-on Objective} and validations for this chapter}{the stated outcomes are produced (proof/code/summary) and recorded in the repo with passing checks}

\Meta
\end{StoryCard}

\begin{StoryTasks}
  \item Extract key definitions/lemmas; compile a one-page summary with references to the chapter.
  \item Solve a small distance-geometry problem for a peptide backbone; reconstruct coordinates.
  \item Quantify reconstruction error under noise and missing distances.
  \item Discuss ambiguities and constraints (chirality, bond lengths).
\end{StoryTasks}

\clearpage
\section*{HDCG-66 --- Geometry \& Topology of Genomic Data}
\begin{StoryCard}{HDCG-66}{Geometry \& Topology of Genomic Data}
\storyrow{Epic / Feature}{Applications of Discrete \& Computational Geometry}
\storyrow{Business Value}{Apply geometric concepts to real domains to deliver measurable outcomes and demos. (Geometry \& Topology of Genomic Data).}
\storyrow{Priority / Estimate}{\textbf{Priority:} Must\hfill \textbf{SP:} 3}
\storyrow{Persona}{data scientist working with geometric methods}
\storyrow{Dependencies}{prior chapters as referenced}
\storyrow{Assumptions / Risks}{time to internalize proofs vs. breadth}

\StoryBody{As a data scientist working with geometric methods, I want to master \textit{Geometry \& Topology of Genomic Data} so that I can apply it to real problems and communicate theoretical insights clearly and reproducibly.}

\NonFunctional{\pill{Reliability} \pill{Reproducibility}}

\BDD{Happy path}{the chapter, examples, and any tooling are available}{I complete the \emph{Hands-on Objective} and validations for this chapter}{the stated outcomes are produced (proof/code/summary) and recorded in the repo with passing checks}

\Meta
\end{StoryCard}

\begin{StoryTasks}
  \item Extract key definitions/lemmas; compile a one-page summary with references to the chapter.
  \item Embed genomic relationships using geometric or topological summaries.
  \item Compare metrics/embeddings (edit distance, Hamming, phylogenetic).
  \item Interpret structures in terms of recombination/phylogeny signals.
\end{StoryTasks}

\clearpage
\section*{HDCG-67 --- Geometric Software (Survey)}
\begin{StoryCard}{HDCG-67}{Geometric Software (Survey)}
\storyrow{Epic / Feature}{Geometric Software}
\storyrow{Business Value}{Select, compile, and use trusted geometry libraries to accelerate development. (Geometric Software (Survey)).}
\storyrow{Priority / Estimate}{\textbf{Priority:} Must\hfill \textbf{SP:} 3}
\storyrow{Persona}{applied geometry engineer}
\storyrow{Dependencies}{prior chapters as referenced}
\storyrow{Assumptions / Risks}{time to internalize proofs vs. breadth}

\StoryBody{As a applied geometry engineer, I want to master \textit{Geometric Software (Survey)} so that I can apply it to real problems and communicate algorithmic insights clearly and reproducibly.}

\NonFunctional{\pill{Reliability} \pill{Reproducibility}}

\BDD{Happy path}{the chapter, examples, and any tooling are available}{I complete the \emph{Hands-on Objective} and validations for this chapter}{the stated outcomes are produced (proof/code/summary) and recorded in the repo with passing checks}

\Meta
\end{StoryCard}

\begin{StoryTasks}
  \item Extract key definitions/lemmas; compile a one-page summary with references to the chapter.
  \item Inventory CGAL/LEDA/libigl/VTK; record kernel differences and license terms.
  \item Build a minimal 'hello geometry' app creating a hull and a Delaunay triangulation.
  \item Decide on a primary stack for subsequent labs; justify trade-offs.
\end{StoryTasks}

\clearpage
\section*{HDCG-68 --- LEDA \& CGAL Case Studies}
\begin{StoryCard}{HDCG-68}{LEDA \& CGAL Case Studies}
\storyrow{Epic / Feature}{Geometric Software}
\storyrow{Business Value}{Select, compile, and use trusted geometry libraries to accelerate development. (LEDA \& CGAL Case Studies).}
\storyrow{Priority / Estimate}{\textbf{Priority:} Must\hfill \textbf{SP:} 3}
\storyrow{Persona}{applied geometry engineer}
\storyrow{Dependencies}{prior chapters as referenced}
\storyrow{Assumptions / Risks}{time to internalize proofs vs. breadth}

\StoryBody{As a applied geometry engineer, I want to master \textit{LEDA \& CGAL Case Studies} so that I can apply it to real problems and communicate algorithmic insights clearly and reproducibly.}

\NonFunctional{\pill{Reliability} \pill{Reproducibility}}

\BDD{Happy path}{the chapter, examples, and any tooling are available}{I complete the \emph{Hands-on Objective} and validations for this chapter}{the stated outcomes are produced (proof/code/summary) and recorded in the repo with passing checks}

\Meta
\end{StoryCard}

\begin{StoryTasks}
  \item Extract key definitions/lemmas; compile a one-page summary with references to the chapter.
  \item Compile and run sample kernels; implement hull$\to$Delaunay$\to$point-location pipeline.
  \item Add robust predicates and exact constructions where available.
  \item Package as a reusable module with unit tests and CLI demo.
\end{StoryTasks}

\clearpage

\end{document}
