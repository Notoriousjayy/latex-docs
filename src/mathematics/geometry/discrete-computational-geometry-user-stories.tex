
% =====================================================================
% Study Plan — Discrete and Computational Geometry (User Story Template)
% Standalone LaTeX document producing story cards styled like screenshots
% =====================================================================
\documentclass[11pt,a4paper]{article}

% --- Encoding & layout ---
\usepackage[T1]{fontenc}
\usepackage[utf8]{inputenc}
\usepackage{lmodern}
\usepackage{microtype}
\usepackage[a4paper,margin=0.8in]{geometry}
\usepackage{setspace}
\setstretch{1.03}
\usepackage{parskip}

% --- Graphics & color ---
\usepackage[table]{xcolor}
\usepackage{graphicx}
\usepackage{tikz}

% --- Links ---
\usepackage{hyperref}
\hypersetup{colorlinks=true, linkcolor=blue!60!black, urlcolor=blue!60!black, citecolor=blue!60!black}

% --- Lists, tables, boxes ---
\usepackage{enumitem}
\setlist{nosep,leftmargin=*,labelsep=0.5em}
\usepackage{array}
\usepackage{tabularx}
\usepackage{booktabs}
\usepackage[most]{tcolorbox}
\tcbuselibrary{skins,breakable}

% --- Math & symbols ---
\usepackage{amsmath,amssymb}
\usepackage{pifont}

% --- Fonts & smallcaps helpers ---
\newcommand{\scaps}[1]{\textsc{#1}}

% --- Colors (tuned to the screenshot style) ---
\definecolor{CardBorder}{HTML}{D9DEE3}
\definecolor{CardLeft}{HTML}{0B6E8B}   % teal/blue sidebar
\definecolor{CardTitle}{HTML}{0B6E8B}
\definecolor{ChipFG}{HTML}{28323C}
\definecolor{ChipBG}{HTML}{EEF2F5}
\definecolor{Muted}{HTML}{6B7A8C}

% --- Tag/Chip style ---
\newtcbox{\nfchip}{on line,
  colback=ChipBG,
  colframe=CardBorder,
  boxrule=0.3pt,
  arc=3pt, top=2pt, bottom=2pt, left=4pt, right=4pt,
  boxsep=0pt,
  before upper=\footnotesize\color{ChipFG},
}

% --- Helper rows for label/value layout ---
\newcommand{\FieldRow}[2]{%
  \noindent\begin{tabularx}{\linewidth}{@{}>{\bfseries}l X@{}}#1 & #2\\\end{tabularx}%
}

% --- Horizontal divider ---
\newcommand{\CardDivider}{\par\noindent\color{CardBorder}\rule{\linewidth}{0.4pt}\par\noindent\color{black}}

% --- Story Card environment ---
\newtcolorbox{StoryCardBox}[1][]{enhanced, breakable,
  colback=white, colframe=CardBorder, arc=2pt, boxrule=0.5pt,
  left=8pt, right=8pt, top=8pt, bottom=8pt,
  borderline west={3pt}{0pt}{CardLeft},
  #1}

% --- Header macro: CODE — TITLE ---
\newcommand{\StoryHeader}[2]{%
  {\large\bfseries \textcolor{CardTitle}{#1} --- #2}
}

% --- BDD section ---
\newcommand{\BDD}[4]{%
  \FieldRow{Acceptance Criteria (BDD)}{}
  \vspace{-0.75\baselineskip}
  \begin{tabularx}{\linewidth}{@{}>{\bfseries}l X@{}}
  Scenario & #1\\
  Given & #2\\
  When & #3\\
  Then & #4\\
  \end{tabularx}
}

% --- Chips line ---
\newcommand{\NFchips}[1]{%
  \begingroup\def\temp{#1}%
  \noindent
  \foreach \x in {#1}{\nfchip{\x}\hspace{0.3em}}%
  \endgroup
}

% --- Task list ---
\newenvironment{TaskList}{%
  \begin{itemize}[leftmargin=1.25em,label=\(\square\)]
}{\end{itemize}}

% --- Definition lines ---
\newcommand{\DefinitionLine}[2]{%
  {\footnotesize\color{Muted}\textbf{#1:} #2}%
}

% =====================================================================
\begin{document}

\begin{center}
  {\huge \bfseries Study Plan — Discrete and Computational Geometry}\\[0.25em]
  {\large \itshape User Story Template \& Examples}\\[0.5em]
  \normalsize Version \today
\end{center}

\CardDivider

\section*{How to Write Effective User Stories (Quick Guide)}
\begin{StoryCardBox}
\FieldRow{User Story format}{\emph{As a \textbf{[persona]} on the DCG study track, I want to \textbf{[goal]} so that \textbf{[outcome/business value]}.}}
\FieldRow{Good stories are}{\textbf{Independent, Negotiable, Valuable, Estimable, Small, Testable} (INVEST). Keep one verifiable outcome per story.}
\FieldRow{Acceptance Criteria}{Use \textbf{BDD}: \textbf{Scenario, Given, When, Then}. Criteria must be objective and testable.}
\FieldRow{Non-Functional}{Capture qualities that matter to the learning artifact: \nfchip{Rigor} \nfchip{Reproducibility} \nfchip{Clarity} \nfchip{Performance}}
\FieldRow{Definition of Ready}{Persona clear; dependencies listed; estimate set; acceptance criteria drafted; risks understood.}
\FieldRow{Definition of Done}{All ACs pass; tests (or proofs) recorded; code and notes committed; artifacts published.}
\end{StoryCardBox}

\CardDivider

\section*{Story Card Definition}
\begin{StoryCardBox}
\FieldRow{Epic / Feature}{The broader learning objective this story supports (e.g., \emph{Convex Geometry Foundations}).}
\FieldRow{Business Value}{Why this matters (e.g., enables later algorithms, reduces confusion, increases proof fluency).}
\FieldRow{Priority / Estimate}{Priority tag (\emph{Must/Should/Could}) and a small story point estimate (e.g., \emph{SP: 3}).}
\FieldRow{Persona}{Who is executing (e.g., \emph{self-learner, teaching assistant, study group member}).}
\FieldRow{Dependencies}{Prerequisites (e.g., vector geometry, induction, installed toolchain).}
\FieldRow{Assumptions / Risks}{Time-box, scope creep, fragile datasets, numerical robustness.}
\FieldRow{Story}{Write the \emph{As a / I want / so that} sentence.}
\FieldRow{Non-Functional}{Quality tags (\nfchip{Rigor} \nfchip{Reproducibility} \nfchip{Clarity} \nfchip{Accessibility} \nfchip{Privacy}).}
\FieldRow{Acceptance Criteria}{Capture BDD criteria using Scenario/Given/When/Then.}
\FieldRow{Tasks}{Actionable steps with checkboxes. Keep 3--7 items.}
\DefinitionLine{Definition of Ready}{Persona clear; AC drafted; dependencies known; estimate set.}\par
\DefinitionLine{Definition of Done}{All ACs pass; tests green; quality checks; docs updated; artifacts published.}
\end{StoryCardBox}

\CardDivider

\section*{Templates}

% ------------------------------
\subsection*{Blank Story Card Template}
\begin{StoryCardBox}
\StoryHeader{DCG-?}{\emph{Card Title Here}}\\[0.4em]
\FieldRow{Epic / Feature}{\emph{e.g., Chapter Foundations}}
\FieldRow{Business Value}{\emph{e.g., establish core concepts to build advanced topics}}
\FieldRow{Priority / Estimate}{\textbf{Priority:} \emph{Must} \quad \nfchip{SP: 3}}
\FieldRow{Persona}{\emph{self-learner / study group}}
\FieldRow{Dependencies}{\emph{list any math or tooling prerequisites}}
\FieldRow{Assumptions / Risks}{\emph{note uncertainties, time-box, dataset fragility}}
\FieldRow{Story}{\emph{As a \textbf{[persona]}, I want to \textbf{[goal]} so that \textbf{[value]}.}}

\FieldRow{Non-Functional}{\NFchips{Performance,Security,Reliability,Accessibility,Privacy,i18n}}

\BDD{Happy path}{\emph{the chapter text, problems, and tools are available}}{\emph{the hands-on objectives for this card are executed}}{\emph{the stated outcomes are demonstrated and recorded in your repo/notebook}}

\DefinitionLine{Definition of Ready}{Persona clear; AC drafted; Dependencies known; Estimate set.} \,\, \textbullet \,\,
\DefinitionLine{Definition of Done}{All ACs pass; Tests green; Quality checks; Docs updated; Published.}

\CardDivider

\FieldRow{\large Tasks}{}
\begin{TaskList}
  \item \emph{Concrete task 1}
  \item \emph{Concrete task 2}
  \item \emph{Concrete task 3}
  \item \emph{Concrete task 4}
\end{TaskList}
\end{StoryCardBox}

% ------------------------------
\subsection*{Example Cards for This Study Plan}

% === DCG-1 Polygons ===
\begin{StoryCardBox}
\StoryHeader{DCG-1}{Polygons — Predicates \& Basics}\\[0.4em]
\FieldRow{Epic / Feature}{Chapter 1: Polygons}
\FieldRow{Business Value}{Build the robust geometric predicates used by nearly all later chapters.}
\FieldRow{Priority / Estimate}{\textbf{Priority:} Must \quad \nfchip{SP: 3}}
\FieldRow{Persona}{developer on a new geometry repo}
\FieldRow{Dependencies}{Git toolchain; \texttt{numpy}/\texttt{matplotlib} or C++ with CGAL; unit-test framework.}
\FieldRow{Assumptions / Risks}{Robustness of floating-point predicates; handling degenerate cases (collinearity, duplicate points).}
\FieldRow{Story}{As a learner building a geometry toolkit, I want to implement orientation, segment intersection, and point-in-polygon so that downstream algorithms are correct and testable.}

\FieldRow{Non-Functional}{\nfchip{Rigor} \nfchip{Reproducibility} \nfchip{Clarity} \nfchip{Performance}}

\BDD{Happy path}{target repository and test scaffold are available}{all predicate implementations and unit tests are completed}{tests pass on random and adversarial inputs; README includes usage and complexity notes}

\DefinitionLine{Definition of Ready}{Persona clear; AC drafted; Dependencies known; Estimate set.} \,\, \textbullet \,\,
\DefinitionLine{Definition of Done}{All ACs pass; tests green; docs updated; artifacts published.}

\CardDivider
\FieldRow{\large Tasks}{}
\begin{TaskList}
  \item Initialize repo with \texttt{geometry} module and a \texttt{hello\_world} test.
  \item Implement \texttt{orient(p,q,r)}, \texttt{intersects(a,b,c,d)}, and winding vs.\ ray-crossing \texttt{point\_in\_polygon}.
  \item Add unit tests including degenerate and adversarial cases; measure runtime on \(10^5\) ops.
  \item Write a short note: interior-angle sum and a property of monotone polygons.
\end{TaskList}
\end{StoryCardBox}

% === DCG-2 Convex Hulls ===
\begin{StoryCardBox}
\StoryHeader{DCG-2}{Convex Hulls — Implementation \& Analysis}\\[0.4em]
\FieldRow{Epic / Feature}{Chapter 2: Convex Hulls}
\FieldRow{Business Value}{Foundational for Delaunay/Voronoi duality and many optimization problems.}
\FieldRow{Priority / Estimate}{\textbf{Priority:} Must \quad \nfchip{SP: 3}}
\FieldRow{Persona}{developer optimizing geometric kernels}
\FieldRow{Dependencies}{DCG-1 completed; plotting utility for hull edges.}
\FieldRow{Assumptions / Risks}{Handling duplicates/collinearity; performance on large \(n\).}
\FieldRow{Story}{As a learner, I want to implement Graham scan and Andrew's monotone chain so that I can benchmark correctness and \(n\log n\) scaling.}

\FieldRow{Non-Functional}{\nfchip{Performance} \nfchip{Rigor} \nfchip{Reproducibility}}

\BDD{Happy path}{point sets loaded from fixtures}{hulls computed by two algorithms with identical results}{benchmarks and correctness proofs are recorded; README explains edge cases}

\DefinitionLine{Definition of Ready}{Persona clear; AC drafted; Dependencies known; Estimate set.} \,\, \textbullet \,\,
\DefinitionLine{Definition of Done}{All ACs pass; tests green; docs updated; artifacts published.}

\CardDivider
\FieldRow{\large Tasks}{}
\begin{TaskList}
  \item Implement Graham scan and Monotone Chain; add Quickhull via library for comparison.
  \item Create adversarial datasets (points on circle, grid with noise, many collinear).
  \item Benchmark for \(n=10^3,10^4,10^5\); chart time vs.\ \(n\).
  \item Write a brief correctness sketch and discuss the lower bound idea.
\end{TaskList}
\end{StoryCardBox}

% === DCG-3 Triangulations ===
\begin{StoryCardBox}
\StoryHeader{DCG-3}{Triangulations — Polygon \& Delaunay}\\[0.4em]
\FieldRow{Epic / Feature}{Chapter 3: Triangulations}
\FieldRow{Business Value}{Enables meshing, interpolation, and Voronoi duality.}
\FieldRow{Priority / Estimate}{\textbf{Priority:} Must \quad \nfchip{SP: 5}}
\FieldRow{Persona}{student implementing mesh primitives}
\FieldRow{Dependencies}{DCG-1, DCG-2; simple polygon datasets.}
\FieldRow{Assumptions / Risks}{Numerical stability for empty-circle tests; handling holes.}
\FieldRow{Story}{As a learner, I want to triangulate polygons (ear clipping) and generate Delaunay triangulations so that I can reason about flips and mesh quality.}

\FieldRow{Non-Functional}{\nfchip{Rigor} \nfchip{Reproducibility} \nfchip{Clarity}}

\BDD{Happy path}{sample polygons and point sets prepared}{triangulations rendered; flips demonstrated on convex polygons}{visualizations saved; notes explain duality to Voronoi}

\CardDivider
\FieldRow{\large Tasks}{}
\begin{TaskList}
  \item Implement ear clipping for simple polygons; visualize diagonals.
  \item Implement or use a Delaunay routine; verify empty-circle property on random sets.
  \item Add flip operation and show that flips connect all triangulations of a convex \(n\)-gon.
\end{TaskList}
\end{StoryCardBox}

% === DCG-4 Voronoi ===
\begin{StoryCardBox}
\StoryHeader{DCG-4}{Voronoi Diagrams — Duality \& Sweep}\\[0.4em]
\FieldRow{Epic / Feature}{Chapter 4: Voronoi Diagrams}
\FieldRow{Business Value}{Backbone for nearest-neighbor search and spatial partitioning.}
\FieldRow{Priority / Estimate}{\textbf{Priority:} Must \quad \nfchip{SP: 5}}
\FieldRow{Persona}{applied geometry learner}
\FieldRow{Dependencies}{Delaunay from DCG-3; plotting.}
\FieldRow{Assumptions / Risks}{Degeneracies (co-circular points); boundary treatment.}
\FieldRow{Story}{As a learner, I want to construct Voronoi diagrams and relate them to Delaunay triangulations so that I can reason about spatial proximity structures.}

\FieldRow{Non-Functional}{\nfchip{Performance} \nfchip{Clarity} \nfchip{Reproducibility}}

\BDD{Happy path}{clean input point sets}{Voronoi computed and clipped to a bounding box}{Delaunay/Voronoi plotted side-by-side with correctness notes}

\CardDivider
\FieldRow{\large Tasks}{}
\begin{TaskList}
  \item Generate Voronoi with a library or implement a simplified Fortune sweep.
  \item Clip unbounded cells; compare site degrees and Euler relations.
  \item Demonstrate weighted Voronoi variants (additive multiplicative) on toy data.
\end{TaskList}
\end{StoryCardBox}

% === DCG-5 Shape Recovery ===
\begin{StoryCardBox}
\StoryHeader{DCG-5}{Shape Recovery — $\alpha$-Shapes \& Crust}\\[0.4em]
\FieldRow{Epic / Feature}{Chapter 5: Shape Recovery}
\FieldRow{Business Value}{Reconstruction from samples; link to computational topology.}
\FieldRow{Priority / Estimate}{\textbf{Priority:} Should \quad \nfchip{SP: 5}}
\FieldRow{Persona}{research-minded learner}
\FieldRow{Dependencies}{Delaunay; filtration plotting.}
\FieldRow{Assumptions / Risks}{Sampling density assumptions; noise sensitivity.}
\FieldRow{Story}{As a learner, I want to explore $\alpha$-complexes and Crust so that I can reconstruct curves from point samples and analyze stability.}

\FieldRow{Non-Functional}{\nfchip{Rigor} \nfchip{Reproducibility} \nfchip{Clarity}}

\BDD{Happy path}{sampled curves and noise knobs ready}{$\alpha$-sweep performed with snapshots}{report compares precision/recall across $\alpha$; failure cases documented}

\CardDivider
\FieldRow{\large Tasks}{}
\begin{TaskList}
  \item Build $\alpha$-complex via Delaunay filtration; export frames across $\alpha$.
  \item Implement NN-Crust; compare reconstructed boundary to ground truth.
  \item Analyze sensitivity to noise and sparsity with plots.
\end{TaskList}
\end{StoryCardBox}

% === DCG-6 Polygonal Chains ===
\begin{StoryCardBox}
\StoryHeader{DCG-6}{Polygonal Chains — Motion \& Shortest Paths}\\[0.4em]
\FieldRow{Epic / Feature}{Chapter 6: Polygonal Chains}
\FieldRow{Business Value}{Introduces linkages and motion planning; core to robotics/pathfinding.}
\FieldRow{Priority / Estimate}{\textbf{Priority:} Should \quad \nfchip{SP: 5}}
\FieldRow{Persona}{algorithms enthusiast}
\FieldRow{Dependencies}{Predicate library; polygon datasets.}
\FieldRow{Assumptions / Risks}{Collision checks; numerical stability.}
\FieldRow{Story}{As a learner, I want to simulate chain straightening and compute shortest paths in simple polygons so that I understand configuration spaces and the funnel algorithm.}

\FieldRow{Non-Functional}{\nfchip{Clarity} \nfchip{Performance} \nfchip{Reproducibility}}

\BDD{Happy path}{chain editor and polygon inputs available}{straightening attempts and funnel paths computed}{animations exported; complexity of funnel algorithm summarized}

\CardDivider
\FieldRow{\large Tasks}{}
\begin{TaskList}
  \item Build a simple linkage simulator with revolute joints.
  \item Implement the funnel algorithm; compare against visibility graph A*.
  \item Document examples of locked vs.\ unlockable chains.
\end{TaskList}
\end{StoryCardBox}

% === DCG-7 Polyhedra ===
\begin{StoryCardBox}
\StoryHeader{DCG-7}{Polyhedra — Euler, Unfoldings, Rigidity}\\[0.4em]
\FieldRow{Epic / Feature}{Chapter 7: Polyhedra}
\FieldRow{Business Value}{Connects combinatorics and geometry in 3D; foundations for mesh processing.}
\FieldRow{Priority / Estimate}{\textbf{Priority:} Could \quad \nfchip{SP: 5}}
\FieldRow{Persona}{3D geometry explorer}
\FieldRow{Dependencies}{OBJ/PLY loader; plotting.}
\FieldRow{Assumptions / Risks}{Handling non-manifold models; precision of angle sums.}
\FieldRow{Story}{As a learner, I want to verify Euler's formula and produce convex polyhedra unfoldings so that I can reason about curvature and rigidity.}

\FieldRow{Non-Functional}{\nfchip{Rigor} \nfchip{Clarity} \nfchip{Reproducibility}}

\BDD{Happy path}{clean convex models loaded}{V,E,F and angle deficits computed}{unfoldings exported as SVG; notes on Cauchy's rigidity compiled}

\CardDivider
\FieldRow{\large Tasks}{}
\begin{TaskList}
  \item Compute V, E, F; verify Euler for multiple models.
  \item Implement a star or source unfolding; export a gallery of nets.
  \item Summarize scissors congruence/Dehn invariants at a high level.
\end{TaskList}
\end{StoryCardBox}

% === DCG-8 Complexity ===
\begin{StoryCardBox}
\StoryHeader{DCG-8}{Computational Complexity in Geometry}\\[0.4em]
\FieldRow{Epic / Feature}{Chapter 8: Complexity}
\FieldRow{Business Value}{Positions geometric problems within P/NP/\#P; guides algorithm choices.}
\FieldRow{Priority / Estimate}{\textbf{Priority:} Should \quad \nfchip{SP: 3}}
\FieldRow{Persona}{theory-minded learner}
\FieldRow{Dependencies}{Survey papers/books at hand.}
\FieldRow{Assumptions / Risks}{Scope creep; formal reductions time-consuming.}
\FieldRow{Story}{As a learner, I want to survey complexities and implement a small $\varepsilon$-approximation so that I can justify algorithmic trade-offs.}

\FieldRow{Non-Functional}{\nfchip{Rigor} \nfchip{Clarity} \nfchip{Reproducibility}}

\BDD{Happy path}{target problem chosen and datasets prepared}{exact and approximate solutions implemented}{report compares solution quality and runtime; reduction outline included}

\CardDivider
\FieldRow{\large Tasks}{}
\begin{TaskList}
  \item Build a table of complexity classes for canonical DCG problems.
  \item Implement a simple $\varepsilon$-approximation (e.g., 2D $k$-center via grid/coreset).
  \item Write a two-page survey with references.
\end{TaskList}
\end{StoryCardBox}

\end{document}
