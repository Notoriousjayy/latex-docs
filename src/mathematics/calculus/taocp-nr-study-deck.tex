\documentclass[11pt]{article}

\usepackage[margin=1in]{geometry}
\usepackage[T1]{fontenc}
\usepackage[utf8]{inputenc}
\usepackage{lmodern}
\usepackage{amssymb}
\usepackage{xcolor}
\usepackage{hyperref}
\usepackage{enumitem}
\usepackage{tcolorbox}
\tcbuselibrary{skins,breakable}

\hypersetup{
  colorlinks=true,
  linkcolor=blue!60!black,
  urlcolor=blue!60!black,
  citecolor=blue!60!black
}

% Basic card style
\tcbset{
  kanbancard/.style={
    enhanced,
    breakable,
    colback=white,
    colframe=blue!60!black,
    boxrule=0.9pt,
    arc=2mm,
    outer arc=2mm,
    fonttitle=\bfseries,
    coltitle=white,
    attach boxed title to top left={yshift*=-\tcboxedtitleheight/2, xshift=2mm},
    boxed title style={
      colback=blue!70!black,
      colframe=blue!70!black,
      sharp corners
    },
    top=4mm,
    bottom=2mm,
    left=3mm,
    right=3mm
  }
}

\newtcolorbox{StoryCard}[2][]{kanbancard, title={#2}, #1}

\setlist[itemize]{leftmargin=*}

\begin{document}

\begin{center}
  {\LARGE TAOCP Vol.~2 + Numerical Recipes}\\[4pt]
  {\Large Numerical Calculus Study Deck}\\[8pt]
  {\small Kanban-oriented StoryCards grouped by Phases}
\end{center}

\bigskip

% ============================================
% Board Overview
% ============================================

\section*{Kanban Board Overview}

\begin{StoryCard}{Kanban Columns \& Tags}
\textbf{Board Columns:}
\begin{itemize}
  \item \textbf{Backlog} -- All reading and implementation cards not yet started.
  \item \textbf{In Progress} -- Cards currently being read, coded, or experimented on.
  \item \textbf{Review / Experiments} -- Numerical experiments, derivations, error analysis, refactors.
  \item \textbf{Done / Consolidated} -- Code implemented, tested, documented, and integrated into the library.
\end{itemize}

\textbf{Recommended Tags (examples):}
\begin{itemize}
  \item \textbf{Book tags:} \texttt{Book:TAOCP2}, \texttt{Book:NR}.
  \item \textbf{Module tags:} \texttt{Module:core-fp}, \texttt{Module:core-poly}, \texttt{Module:interp}, \texttt{Module:quad}, \texttt{Module:ode}, \texttt{Module:nonlin}, \texttt{Module:linalg}, \texttt{Module:rand}, \texttt{Module:optim}, \texttt{Module:stats}.\@
  \item \textbf{Work type tags:} \texttt{Reading}, \texttt{Implementation}, \texttt{Experiment}, \texttt{Design}, \texttt{Refactor}.\@
\end{itemize}

\textbf{Usage Pattern:}
\begin{itemize}
  \item Each \emph{StoryCard} in this document corresponds to a Kanban card.
  \item Start with Phase~0 cards in \emph{Backlog}. Pull into \emph{In Progress} with a strict WIP limit (e.g., max 2--3 cards).
  \item Move to \emph{Review / Experiments} once code compiles and basic tests run.
  \item Move to \emph{Done / Consolidated} only when:
  \begin{itemize}
    \item There is at least one unit test or numerical experiment.
    \item The API is sketched and documented in the repo.
    \item You have brief notes in your study log.
  \end{itemize}
\end{itemize}
\end{StoryCard}
\clearpage

% ============================================
% Phase 0 – Environment & Test Harness
% ============================================

\section*{Phase 0 -- Environment \& Test Harness}

\begin{StoryCard}{Phase 0 Overview -- Environment \& Test Harness}
\textbf{Goal:} Be able to quickly prototype, run, and visualize numerical methods from TAOCP Vol.~2 and Numerical Recipes.

\medskip
\textbf{Primary Focus:}
\begin{itemize}
  \item Project layout and build system.
  \item Test harness and plotting workflow.
  \item Basic utilities for timing and error measurement.
\end{itemize}

\textbf{Module Tie-ins:}
\begin{itemize}
  \item \texttt{Module:all} (infrastructure for every module).
\end{itemize}

\textbf{Exit Criteria:}
\begin{itemize}
  \item You can write a small numerical routine, compile it, run it, compute error metrics, and plot the results with minimal friction.
  \item The repo layout is stable enough that later phases only add modules, not re-architect the whole tree.
\end{itemize}
\end{StoryCard}
\clearpage

\begin{StoryCard}{Phase 0 Card 1 -- Set Up Numerical Playground Repo}
\textbf{Description:} Create a dedicated repository to host your computational math system and numerical experiments.

\medskip
\textbf{Checklist:}
\begin{itemize}[label=$\square$]
  \item Create a root layout such as:
  \begin{itemize}
    \item \texttt{core/} (floating point, utilities, polynomials).
    \item \texttt{interp/} (interpolation and approximation).
    \item \texttt{integrate/} (quadrature / integration).
    \item \texttt{ode/} (ordinary differential equations).
    \item \texttt{opt/} (optimization).
    \item \texttt{rand/} (random numbers and Monte Carlo).
    \item \texttt{nonlin/} (root finding and nonlinear systems).
    \item \texttt{linalg/} (linear algebra).
    \item \texttt{stats/} (error/statistics tools).
    \item \texttt{examples/}, \texttt{tests/}, \texttt{docs/}.
  \end{itemize}
  \item Add a basic C/C++ build system:
  \begin{itemize}
    \item \texttt{CMakeLists.txt} or equivalent build file.
    \item Configurable build type (Debug/Release).
  \end{itemize}
  \item Integrate a simple test harness:
  \begin{itemize}
    \item Minimal custom test main, or
    \item Catch2 / GoogleTest wired into the build.
  \end{itemize}
  \item Add a plotting hook:
  \begin{itemize}
    \item Helper to dump \texttt{(x, y)} data to CSV.
    \item Document how to plot via Python/Matplotlib or gnuplot.
  \end{itemize}
\end{itemize}

\textbf{Deliverables:}
\begin{itemize}
  \item Repository skeleton committed (e.g., Git).
  \item \texttt{README.md} describing layout, build, and plotting workflow.
\end{itemize}

\textbf{Tags:} \texttt{Infra}, \texttt{Module:all}, \texttt{Implementation}
\end{StoryCard}
\clearpage

\begin{StoryCard}{Phase 0 Card 2 -- Implement Timing \& Error Metrics Utilities}
\textbf{Description:} Provide reusable helpers for timing functions and computing numerical error metrics.

\medskip
\textbf{Checklist:}
\begin{itemize}[label=$\square$]
  \item Implement timing helpers:
  \begin{itemize}
    \item Wall-clock timer based on \texttt{std::chrono} (or platform equivalent).
    \item Scope-based timer to measure function runtimes.
  \end{itemize}
  \item Implement error metric utilities:
  \begin{itemize}
    \item Absolute error: $|x - x_{\mathrm{true}}|$.
    \item Relative error: $\frac{|x - x_{\mathrm{true}}|}{|x_{\mathrm{true}}|}$.
    \item Max norm over a dataset.
    \item RMS error over samples.
  \end{itemize}
  \item Implement convergence-rate estimator:
  \begin{itemize}
    \item Given error values for step sizes $h, h/2, h/4$, estimate observed order $p$.
  \end{itemize}
  \item Add minimal tests:
  \begin{itemize}
    \item Check that zero error is reported correctly.
    \item Check that known error cases behave as expected.
  \end{itemize}
\end{itemize}

\textbf{Deliverables:}
\begin{itemize}
  \item \texttt{core/timing.h}, \texttt{core/timing.cpp}.
  \item \texttt{core/error\_metrics.h}, \texttt{core/error\_metrics.cpp}.
  \item A small example program demonstrating timing and error calculations.
\end{itemize}

\textbf{Tags:} \texttt{Module:core-fp}, \texttt{Implementation}, \texttt{Experiment}
\end{StoryCard}
\clearpage

% ============================================
% Phase 1 – Core Arithmetic, Error & Representation
% ============================================

\section*{Phase 1 -- Core Arithmetic, Error \& Representation}

\begin{StoryCard}{Phase 1 Overview -- Core Arithmetic, Error \& Representation}
\textbf{Goal:} Understand how the machine represents numbers and how that shapes numerical error in calculus algorithms.

\medskip
\textbf{Primary Sources:}
\begin{itemize}
  \item TAOCP Vol.~2, Chapter 4: arithmetic, floating-point, polynomial and power-series arithmetic.
  \item Numerical Recipes:
  \begin{itemize}
    \item Chapter 1: error, accuracy, stability.
    \item Chapter 5: evaluation of functions (polynomials, series).
  \end{itemize}
\end{itemize}

\textbf{Module Tie-ins:}
\begin{itemize}
  \item \texttt{core::fp} (floating-point utilities).
  \item \texttt{core::poly} (polynomial arithmetic and evaluation).
\end{itemize}

\textbf{Exit Criteria:}
\begin{itemize}
  \item You can quantify and visualize rounding error and cancellation.
  \item You have a robust implementation of polynomial evaluation (Horner) with tests.
\end{itemize}
\end{StoryCard}
\clearpage

\begin{StoryCard}{Phase 1 Card 1 -- Read Floating-Point Arithmetic \& Rounding}
\textbf{Description:} Build theoretical understanding of floating-point representation, rounding, and error propagation.

\medskip
\textbf{Reading Tasks:}
\begin{itemize}[label=$\square$]
  \item TAOCP Vol.~2, sections 4.1--4.2:
  \begin{itemize}
    \item Positional number systems.
    \item Floating-point formats and rounding modes.
  \end{itemize}
  \item Numerical Recipes, Section 1.1:
  \begin{itemize}
    \item Error, accuracy, and stability.
    \item Conditioning vs.\ algorithmic stability.
  \end{itemize}
\end{itemize}

\textbf{Notes Deliverable:}
\begin{itemize}
  \item A one-page summary covering:
  \begin{itemize}
    \item Definition of machine epsilon for your platform.
    \item Examples of catastrophic cancellation.
    \item Distinction between well-conditioned problems and stable algorithms.
  \end{itemize}
\end{itemize}

\textbf{Tags:} \texttt{Reading}, \texttt{Book:TAOCP2}, \texttt{Book:NR}, \texttt{Module:core-fp}
\end{StoryCard}
\clearpage

\begin{StoryCard}{Phase 1 Card 2 -- Implement Safe Floating-Point Utilities}
\textbf{Description:} Translate the theory into a \texttt{core::fp} utility module.

\medskip
\textbf{Checklist:}
\begin{itemize}[label=$\square$]
  \item Implement \texttt{double ulp(double x)}:
  \begin{itemize}
    \item Approximate the distance between representable numbers around \texttt{x}.
  \end{itemize}
  \item Implement \texttt{bool nearly\_equal(double a, double b)}:
  \begin{itemize}
    \item Combine absolute and relative tolerance.
  \end{itemize}
  \item Implement robust summation:
  \begin{itemize}
    \item Naive summation for baseline.
    \item Kahan summation or pairwise summation.
  \end{itemize}
  \item Add tests:
  \begin{itemize}
    \item Show difference between naive and Kahan for ill-conditioned sums.
    \item Verify that \texttt{nearly\_equal} behaves correctly for small and large magnitudes.
  \end{itemize}
\end{itemize}

\textbf{Deliverables:}
\begin{itemize}
  \item \texttt{core/fp.h}, \texttt{core/fp.cpp}.
  \item A small example that sums a sequence known to illustrate cancellation, with printed error statistics.
\end{itemize}

\textbf{Tags:} \texttt{Module:core-fp}, \texttt{Implementation}, \texttt{Experiment}
\end{StoryCard}
\clearpage

\begin{StoryCard}{Phase 1 Card 3 -- Read \& Implement Polynomial Evaluation}
\textbf{Description:} Use TAOCP and NR to implement efficient polynomial evaluation and test its numerical behavior.

\medskip
\textbf{Reading Tasks:}
\begin{itemize}[label=$\square$]
  \item TAOCP Vol.~2, section 4.6 (polynomial arithmetic, Horner’s rule).
  \item Numerical Recipes, section 5.1 (polynomials and rational functions).
\end{itemize}

\textbf{Implementation Tasks:}
\begin{itemize}[label=$\square$]
  \item Design a polynomial representation:
  \begin{itemize}
    \item Coefficient array in ascending order (constant term first).
    \item Or a small \texttt{Poly} class with degree and coefficients.
  \end{itemize}
  \item Implement \texttt{poly\_eval} using Horner’s rule.
  \item Optionally implement polynomial addition and multiplication (for later Chebyshev work).
\end{itemize}

\textbf{Experiment Tasks:}
\begin{itemize}[label=$\square$]
  \item Choose test polynomials (e.g., Chebyshev, Taylor expansions).
  \item Evaluate them at many points and compare:
  \begin{itemize}
    \item Direct evaluation vs.\ Horner’s rule.
    \item Error vs.\ degree when approximating known functions.
  \end{itemize}
\end{itemize}

\textbf{Deliverables:}
\begin{itemize}
  \item \texttt{core/poly.h}, \texttt{core/poly.cpp}.
  \item Plots or tables of error vs.\ degree for at least one approximation problem.
\end{itemize}

\textbf{Tags:} \texttt{Module:core-poly}, \texttt{Reading}, \texttt{Implementation}, \texttt{Experiment}
\end{StoryCard}
\clearpage

% ============================================
% Phase 2 – Interpolation & Approximation
% ============================================

\section*{Phase 2 -- Interpolation \& Approximation}

\begin{StoryCard}{Phase 2 Overview -- Interpolation \& Approximation}
\textbf{Goal:} Build interpolation and approximation tools that will feed into differentiation, integration, and ODE/PDE solvers.

\medskip
\textbf{Primary Sources:}
\begin{itemize}
  \item TAOCP Vol.~2, sections 4.6 and 4.7 (polynomials and power series).
  \item Numerical Recipes, Chapter 3 (interpolation and extrapolation), and sections 5.8--5.10 (Chebyshev approximation).
\end{itemize}

\textbf{Module Tie-ins:}
\begin{itemize}
  \item \texttt{interp::} (interpolation).
  \item \texttt{core::poly} (polynomial arithmetic).
\end{itemize}

\textbf{Exit Criteria:}
\begin{itemize}
  \item You can construct interpolation schemes and Chebyshev approximations.
  \item You can compare errors for different node distributions and degrees.
\end{itemize}
\end{StoryCard}
\clearpage

\begin{StoryCard}{Phase 2 Card 1 -- Read Interpolation Basics}
\textbf{Description:} Understand the basic interpolation methods and their properties.

\medskip
\textbf{Reading Tasks:}
\begin{itemize}[label=$\square$]
  \item Numerical Recipes, Chapter 3, sections 3.0--3.3:
  \begin{itemize}
    \item Table lookup, polynomial interpolation, cubic splines.
  \end{itemize}
  \item Skim sections on multidimensional interpolation (3.6--3.7) for later.
\end{itemize}

\textbf{Notes Deliverable:}
\begin{itemize}
  \item Short notes summarizing:
  \begin{itemize}
    \item When polynomial interpolation is appropriate.
    \item Why splines can be more stable.
    \item Dangers of extrapolation.
  \end{itemize}
\end{itemize}

\textbf{Tags:} \texttt{Reading}, \texttt{Book:NR}, \texttt{Module:interp}
\end{StoryCard}
\clearpage

\begin{StoryCard}{Phase 2 Card 2 -- Implement 1D Interpolation Module}
\textbf{Description:} Implement core 1D interpolation routines.

\medskip
\textbf{Implementation Tasks:}
\begin{itemize}[label=$\square$]
  \item Implement polynomial interpolation:
  \begin{itemize}
    \item Given arrays \texttt{x[i]}, \texttt{y[i]}, construct coefficients or direct evaluation routines.
  \end{itemize}
  \item Implement cubic splines:
  \begin{itemize}
    \item Natural or clamped boundary conditions.
    \item Precompute spline coefficients; expose an evaluation function.
  \end{itemize}
  \item Design a clean API:
  \begin{itemize}
    \item \texttt{interp::poly(x, y)} returning an object with an \texttt{operator()}.
    \item \texttt{interp::spline(x, y)} similarly.
  \end{itemize}
\end{itemize}

\textbf{Testing \& Experiments:}
\begin{itemize}[label=$\square$]
  \item Test on smooth functions (e.g., $\sin x$, $\exp x$) and less smooth ones.
  \item Compare interpolation error for different numbers of nodes.
\end{itemize}

\textbf{Deliverables:}
\begin{itemize}
  \item \texttt{interp/interp.h}, \texttt{interp/interp.cpp}.
  \item Example program visualizing interpolated curves vs.\ ground truth.
\end{itemize}

\textbf{Tags:} \texttt{Module:interp}, \texttt{Implementation}, \texttt{Experiment}
\end{StoryCard}
\clearpage

\begin{StoryCard}{Phase 2 Card 3 -- Implement Chebyshev Approximation}
\textbf{Description:} Implement Chebyshev-based function approximation and connect it to polynomial arithmetic.

\medskip
\textbf{Reading Tasks:}
\begin{itemize}[label=$\square$]
  \item Numerical Recipes, sections 5.8--5.10:
  \begin{itemize}
    \item Chebyshev approximation, derivatives, and integrals via coefficients.
  \end{itemize}
  \item Revisit TAOCP Vol.~2, section 4.6 for efficient polynomial evaluation and composition.
\end{itemize}

\textbf{Implementation Tasks:}
\begin{itemize}[label=$\square$]
  \item Implement routines to compute Chebyshev coefficients of a function on $[-1,1]$.
  \item Implement evaluation of the Chebyshev approximation at arbitrary points.
  \item Provide helpers to differentiate/integrate the Chebyshev approximation.
\end{itemize}

\textbf{Experiment Tasks:}
\begin{itemize}[label=$\square$]
  \item Approximate several functions (e.g., $\sin x$, $\exp x$, a Gaussian).
  \item Compare approximation error vs.\ degree for Chebyshev vs.\ naive polynomial interpolation.
\end{itemize}

\textbf{Deliverables:}
\begin{itemize}
  \item \texttt{interp/chebyshev.h}, \texttt{interp/chebyshev.cpp}.
  \item Plots of error vs.\ polynomial degree and node distributions.
\end{itemize}

\textbf{Tags:} \texttt{Module:interp}, \texttt{Module:core-poly}, \texttt{Implementation}, \texttt{Experiment}
\end{StoryCard}
\clearpage

\begin{StoryCard}{Phase 2 Card 4 -- Experiment: Interpolation vs Degree \& Nodes}
\textbf{Description:} Dedicated experiments on the behavior of interpolation as degree and node distribution change.

\medskip
\textbf{Experiment Tasks:}
\begin{itemize}[label=$\square$]
  \item Choose several test functions:
  \begin{itemize}
    \item Smooth (\(\sin x\), \(\exp x\)).
    \item With endpoint features or mild singularities.
  \end{itemize}
  \item Compare:
  \begin{itemize}
    \item Equally spaced nodes vs.\ Chebyshev nodes.
    \item Polynomial interpolation vs.\ splines vs.\ Chebyshev approximation.
  \end{itemize}
  \item Produce plots of:
  \begin{itemize}
    \item Max error vs.\ degree.
    \item Error distribution across the interval.
  \end{itemize}
\end{itemize}

\textbf{Deliverables:}
\begin{itemize}
  \item An \texttt{examples/interp\_study.cpp} (or similar) program.
  \item A short write-up summarizing observations (Runge phenomenon, stability).
\end{itemize}

\textbf{Tags:} \texttt{Module:interp}, \texttt{Experiment}, \texttt{Analysis}
\end{StoryCard}
\clearpage

% ============================================
% Phase 3 – Numerical Differentiation & Integration
% ============================================

\section*{Phase 3 -- Numerical Differentiation \& Integration (Quadrature)}

\begin{StoryCard}{Phase 3 Overview -- Numerical Differentiation \& Integration}
\textbf{Goal:} Directly attack core calculus tasks: derivative estimation and definite integrals.

\medskip
\textbf{Primary Sources:}
\begin{itemize}
  \item Numerical Recipes:
  \begin{itemize}
    \item Section 5.7: numerical derivatives and error behavior.
    \item Chapter 4: integration of functions (classical formulas, Romberg, Gaussian, adaptive, multidimensional).
  \end{itemize}
\end{itemize}

\textbf{Module Tie-ins:}
\begin{itemize}
  \item \texttt{diffint::deriv} (differentiation).
  \item \texttt{diffint::quad} (quadrature).
\end{itemize}

\textbf{Exit Criteria:}
\begin{itemize}
  \item You have derivative estimators and quadrature routines with observed convergence rates matching theory.
  \item You can visualize error vs.\ step size or number of function evaluations.
\end{itemize}
\end{StoryCard}
\clearpage

\begin{StoryCard}{Phase 3 Card 1 -- Read Numerical Differentiation \& Error Behavior}
\textbf{Description:} Build conceptual understanding of finite differences and their errors.

\medskip
\textbf{Reading Tasks:}
\begin{itemize}[label=$\square$]
  \item Numerical Recipes, section 5.7:
  \begin{itemize}
    \item Forward, backward, and central differences.
    \item Step-size tradeoffs between truncation and round-off error.
  \end{itemize}
\end{itemize}

\textbf{Notes Deliverable:}
\begin{itemize}
  \item Short notes or a diagram showing:
  \begin{itemize}
    \item Taylor-expansion-based derivation of finite-difference formulas.
    \item Qualitative shape of error vs.\ step size (U-shaped curve).
  \end{itemize}
\end{itemize}

\textbf{Tags:} \texttt{Reading}, \texttt{Book:NR}, \texttt{Module:diffint-deriv}
\end{StoryCard}
\clearpage

\begin{StoryCard}{Phase 3 Card 2 -- Implement Derivative Estimators}
\textbf{Description:} Implement a small \texttt{diffint::deriv} module.

\medskip
\textbf{Implementation Tasks:}
\begin{itemize}[label=$\square$]
  \item Implement:
  \begin{itemize}
    \item \texttt{deriv\_forward(f, x, h)}.
    \item \texttt{deriv\_backward(f, x, h)}.
    \item \texttt{deriv\_central(f, x, h)}.
  \end{itemize}
  \item Optionally add Richardson extrapolation to boost accuracy.
  \item Use \texttt{core::fp} utilities for error comparisons.
\end{itemize}

\textbf{Experiment Tasks:}
\begin{itemize}[label=$\square$]
  \item For known functions (e.g., \(\sin x\), \(\exp x\), polynomials):
  \begin{itemize}
    \item Sweep over step sizes \(h\).
    \item Plot absolute error vs.\ \(h\) for each scheme.
    \item Show the ``sweet spot'' where truncation and round-off balance.
  \end{itemize}
\end{itemize}

\textbf{Deliverables:}
\begin{itemize}
  \item \texttt{diffint/deriv.h}, \texttt{diffint/deriv.cpp}.
  \item Example program producing error plots.
\end{itemize}

\textbf{Tags:} \texttt{Module:diffint-deriv}, \texttt{Implementation}, \texttt{Experiment}
\end{StoryCard}
\clearpage

\begin{StoryCard}{Phase 3 Card 3 -- Read Integration Methods Overview}
\textbf{Description:} Survey classical quadrature methods.

\medskip
\textbf{Reading Tasks:}
\begin{itemize}[label=$\square$]
  \item Numerical Recipes, sections 4.0--4.4:
  \begin{itemize}
    \item Trapezoidal rule, Simpson’s rule, Romberg integration.
    \item Basics of improper integrals.
  \end{itemize}
\end{itemize}

\textbf{Notes Deliverable:}
\begin{itemize}
  \item A small table summarizing:
  \begin{itemize}
    \item Order of accuracy for each rule.
    \item When each method is appropriate or problematic.
  \end{itemize}
\end{itemize}

\textbf{Tags:} \texttt{Reading}, \texttt{Book:NR}, \texttt{Module:diffint-quad}
\end{StoryCard}
\clearpage

\begin{StoryCard}{Phase 3 Card 4 -- Implement Basic 1D Quadrature}
\textbf{Description:} Implement basic 1D integration routines.

\medskip
\textbf{Implementation Tasks:}
\begin{itemize}[label=$\square$]
  \item Implement composite trapezoidal rule: \texttt{trap(f, a, b, n)}.
  \item Implement composite Simpson’s rule: \texttt{simpson(f, a, b, n)}.
  \item Implement Romberg integration \texttt{romberg(f, a, b)} using trapezoidal estimates.
\end{itemize}

\textbf{Experiment Tasks:}
\begin{itemize}[label=$\square$]
  \item Integrate known functions (polynomials, \(\sin x\), \(\exp x\)):
  \begin{itemize}
    \item Compare numerical results to analytic integrals.
    \item Plot error vs.\ number of function evaluations.
  \end{itemize}
\end{itemize}

\textbf{Deliverables:}
\begin{itemize}
  \item \texttt{diffint/quad.h}, \texttt{diffint/quad.cpp}.
  \item Example program for integration convergence plots.
\end{itemize}

\textbf{Tags:} \texttt{Module:diffint-quad}, \texttt{Implementation}, \texttt{Experiment}
\end{StoryCard}
\clearpage

\begin{StoryCard}{Phase 3 Card 5 -- Implement Advanced Quadrature \& Convergence Plots}
\textbf{Description:} Extend the integration module with more advanced schemes and compare their performance.

\medskip
\textbf{Implementation Tasks:}
\begin{itemize}[label=$\square$]
  \item Implement Gaussian quadrature:
  \begin{itemize}
    \item \texttt{gauss\_legendre(f, a, b, n)}.
  \end{itemize}
  \item Implement an adaptive integrator:
  \begin{itemize}
    \item Subdivide intervals based on local error estimates.
  \end{itemize}
  \item Optionally add multidimensional integration wrappers (tensor product or simple Monte Carlo).
\end{itemize}

\textbf{Experiment Tasks:}
\begin{itemize}[label=$\square$]
  \item For several test integrals:
  \begin{itemize}
    \item Compare error vs.\ function evaluations for trapezoid, Simpson, Romberg, Gaussian, and adaptive methods.
    \item Summarize which method wins for which class of functions.
  \end{itemize}
\end{itemize}

\textbf{Deliverables:}
\begin{itemize}
  \item Extended \texttt{diffint/quad.*} with advanced methods.
  \item Plots or tables of convergence behavior.
\end{itemize}

\textbf{Tags:} \texttt{Module:diffint-quad}, \texttt{Implementation}, \texttt{Experiment}, \texttt{Analysis}
\end{StoryCard}
\clearpage

% ============================================
% Phase 4 – Root Finding & Nonlinear Equations
% ============================================

\section*{Phase 4 -- Root Finding \& Nonlinear Equations}

\begin{StoryCard}{Phase 4 Overview -- Root Finding \& Nonlinear Equations}
\textbf{Goal:} Solve \(f(x) = 0\) and small nonlinear systems using robust algorithms.

\medskip
\textbf{Primary Sources:}
\begin{itemize}
  \item Numerical Recipes, Chapter 9:
  \begin{itemize}
    \item Bracketing methods, secant, Brent, Newton, and nonlinear systems.
  \end{itemize}
\end{itemize}

\textbf{Module Tie-ins:}
\begin{itemize}
  \item \texttt{nonlin::root} (root finding).
  \item \texttt{linalg::} (linear algebra for systems).
  \item \texttt{diffint::deriv} (for Newton’s method derivatives).
\end{itemize}

\textbf{Exit Criteria:}
\begin{itemize}
  \item You have a generic 1D root-finding API with multiple methods.
  \item You can solve small nonlinear systems using a Newton-style method.
\end{itemize}
\end{StoryCard}
\clearpage

\begin{StoryCard}{Phase 4 Card 1 -- Read Root-Finding Landscape}
\textbf{Description:} Understand the tradeoffs between different root-finding methods.

\medskip
\textbf{Reading Tasks:}
\begin{itemize}[label=$\square$]
  \item Numerical Recipes, sections 9.0--9.4:
  \begin{itemize}
    \item Bracketing methods (bisection).
    \item Secant and Brent methods.
    \item Newton’s method and its convergence properties.
  \end{itemize}
\end{itemize}

\textbf{Notes Deliverable:}
\begin{itemize}
  \item A small table summarizing:
  \begin{itemize}
    \item Method type (bracketing vs.\ open).
    \item Convergence rate.
    \item Requirements (derivative? initial bracket?).
  \end{itemize}
\end{itemize}

\textbf{Tags:} \texttt{Reading}, \texttt{Book:NR}, \texttt{Module:nonlin-root}
\end{StoryCard}
\clearpage

\begin{StoryCard}{Phase 4 Card 2 -- Implement 1D Root-Finding API}
\textbf{Description:} Implement a small \texttt{nonlin::root} module with multiple strategies.

\medskip
\textbf{Implementation Tasks:}
\begin{itemize}[label=$\square$]
  \item Implement:
  \begin{itemize}
    \item \texttt{bisect(f, a, b, tol)}.
    \item \texttt{newton(f, df, x0, tol)}.
    \item \texttt{brent(f, a, b, tol)}.
  \end{itemize}
  \item Provide a unified wrapper:
  \begin{itemize}
    \item \texttt{root\_find(f, ...)} that selects method based on configuration.
  \end{itemize}
  \item Use finite differences from \texttt{diffint::deriv} when analytic derivatives are not available.
\end{itemize}

\textbf{Experiment Tasks:}
\begin{itemize}[label=$\square$]
  \item Test on several functions with known roots:
  \begin{itemize}
    \item Polynomials of moderate degree.
    \item Transcendental equations (\(\sin x = x/2\), etc.).
  \end{itemize}
  \item Measure iterations and robustness for different starting guesses.
\end{itemize}

\textbf{Deliverables:}
\begin{itemize}
  \item \texttt{nonlin/root.h}, \texttt{nonlin/root.cpp}.
  \item Example programs illustrating method comparisons.
\end{itemize}

\textbf{Tags:} \texttt{Module:nonlin-root}, \texttt{Implementation}, \texttt{Experiment}
\end{StoryCard}
\clearpage

\begin{StoryCard}{Phase 4 Card 3 -- Implement Nonlinear System Solvers}
\textbf{Description:} Solve small systems \(F(\mathbf{x}) = 0\) using Newton-type methods.

\medskip
\textbf{Implementation Tasks:}
\begin{itemize}[label=$\square$]
  \item Implement Jacobian approximation via finite differences:
  \begin{itemize}
    \item Wrap \texttt{diffint::deriv} for multivariate functions.
  \end{itemize}
  \item Implement a Newton system solver:
  \begin{itemize}
    \item At each iteration, solve \(J(\mathbf{x}_k)\Delta \mathbf{x} = -F(\mathbf{x}_k)\) using \texttt{linalg::}.
  \end{itemize}
  \item Implement simple globalisation strategies:
  \begin{itemize}
    \item Damped steps or basic line search to improve robustness.
  \end{itemize}
\end{itemize}

\textbf{Experiment Tasks:}
\begin{itemize}[label=$\square$]
  \item Test on simple nonlinear systems (e.g., intersections of circles, small mechanical systems).
  \item Track convergence behavior for different initial guesses.
\end{itemize}

\textbf{Deliverables:}
\begin{itemize}
  \item \texttt{nonlin/system.h}, \texttt{nonlin/system.cpp}.
  \item Example programs showing convergence and failure cases.
\end{itemize}

\textbf{Tags:} \texttt{Module:nonlin-root}, \texttt{Module:linalg}, \texttt{Implementation}, \texttt{Experiment}
\end{StoryCard}
\clearpage

% ============================================
% Phase 5 – Linear Algebra & Optimization
% ============================================

\section*{Phase 5 -- Linear Algebra \& Optimization}

\begin{StoryCard}{Phase 5 Overview -- Linear Algebra \& Optimization}
\textbf{Goal:} Build the linear algebra backbone and basic optimization algorithms used by many calculus applications (ODE, PDE, fitting).

\medskip
\textbf{Primary Sources:}
\begin{itemize}
  \item Numerical Recipes:
  \begin{itemize}
    \item Chapter 2: solution of linear algebraic equations.
    \item Chapter 10: minimization or maximization of functions.
  \end{itemize}
\end{itemize}

\textbf{Module Tie-ins:}
\begin{itemize}
  \item \texttt{linalg::} (linear algebra).
  \item \texttt{optim::} (optimization).
\end{itemize}

\textbf{Exit Criteria:}
\begin{itemize}
  \item You can solve small dense systems and simple least-squares problems.
  \item You can perform basic 1D and multidimensional optimization.
\end{itemize}
\end{StoryCard}
\clearpage

\begin{StoryCard}{Phase 5 Card 1 -- Read Linear System Solvers}
\textbf{Description:} Understand practical algorithms for solving \(A\mathbf{x}=\mathbf{b}\).

\medskip
\textbf{Reading Tasks:}
\begin{itemize}[label=$\square$]
  \item Numerical Recipes, sections 2.0--2.4:
  \begin{itemize}
    \item LU decomposition and Gaussian elimination.
    \item Banded systems and conditioning notes.
  \end{itemize}
\end{itemize}

\textbf{Notes Deliverable:}
\begin{itemize}
  \item Short summary of:
  \begin{itemize}
    \item When LU is appropriate vs.\ Cholesky.
    \item The role of pivoting and conditioning.
  \end{itemize}
\end{itemize}

\textbf{Tags:} \texttt{Reading}, \texttt{Book:NR}, \texttt{Module:linalg}
\end{StoryCard}
\clearpage

\begin{StoryCard}{Phase 5 Card 2 -- Implement Linear Algebra Core}
\textbf{Description:} Implement key linear algebra routines.

\medskip
\textbf{Implementation Tasks:}
\begin{itemize}[label=$\square$]
  \item Implement a small dense matrix and vector type, or wrap an existing library with a thin façade.
  \item Implement LU decomposition with partial pivoting.
  \item Implement a solver using LU factors for multiple right-hand sides.
  \item Implement Cholesky decomposition for symmetric positive-definite matrices.
\end{itemize}

\textbf{Experiment Tasks:}
\begin{itemize}[label=$\square$]
  \item Test on small random systems and known test matrices.
  \item Compare residuals for different methods and confirm numerical stability.
\end{itemize}

\textbf{Deliverables:}
\begin{itemize}
  \item \texttt{linalg/linalg.h}, \texttt{linalg/linalg.cpp}.
  \item Unit tests for LU and Cholesky solvers.
\end{itemize}

\textbf{Tags:} \texttt{Module:linalg}, \texttt{Implementation}, \texttt{Experiment}
\end{StoryCard}
\clearpage

\begin{StoryCard}{Phase 5 Card 3 -- Read Scalar \& Multidimensional Optimization}
\textbf{Description:} Review core ideas in 1D and multidimensional optimization.

\medskip
\textbf{Reading Tasks:}
\begin{itemize}[label=$\square$]
  \item Numerical Recipes, sections 10.0--10.4:
  \begin{itemize}
    \item 1D line searches.
    \item Simple multidimensional methods.
  \end{itemize}
\end{itemize}

\textbf{Notes Deliverable:}
\begin{itemize}
  \item Summary of:
  \begin{itemize}
    \item Golden-section search and Brent-like methods.
    \item Basic gradient-based methods and convergence criteria.
  \end{itemize}
\end{itemize}

\textbf{Tags:} \texttt{Reading}, \texttt{Book:NR}, \texttt{Module:optim}
\end{StoryCard}
\clearpage

\begin{StoryCard}{Phase 5 Card 4 -- Implement Optimization Routines}
\textbf{Description:} Implement basic 1D and multidimensional optimization algorithms.

\medskip
\textbf{Implementation Tasks:}
\begin{itemize}[label=$\square$]
  \item 1D minimization:
  \begin{itemize}
    \item Golden-section search.
    \item Brent-type method for robustness.
  \end{itemize}
  \item Multidimensional:
  \begin{itemize}
    \item Gradient descent with line search.
    \item Optionally conjugate gradient or quasi-Newton variant.
  \end{itemize}
\end{itemize}

\textbf{Experiment Tasks:}
\begin{itemize}[label=$\square$]
  \item Run on benchmark functions:
  \begin{itemize}
    \item Quadratic bowls.
    \item Rosenbrock function.
  \end{itemize}
  \item Track iteration counts, convergence behavior, and sensitivity to starting guess.
\end{itemize}

\textbf{Deliverables:}
\begin{itemize}
  \item \texttt{optim/optim.h}, \texttt{optim/optim.cpp}.
  \item Example programs visualizing convergence trajectories (e.g., contour plots with paths).
\end{itemize}

\textbf{Tags:} \texttt{Module:optim}, \texttt{Implementation}, \texttt{Experiment}
\end{StoryCard}
\clearpage

% ============================================
% Phase 6 – ODEs & PDEs
% ============================================

\section*{Phase 6 -- ODEs \& PDEs (Dynamics \& Fields)}

\begin{StoryCard}{Phase 6 Overview -- ODEs \& PDEs}
\textbf{Goal:} Use everything above to solve dynamical systems (ODEs) and simple PDEs.

\medskip
\textbf{Primary Sources:}
\begin{itemize}
  \item Numerical Recipes:
  \begin{itemize}
    \item Chapter 17: integration of ODEs.
    \item Chapters 18 and 20: boundary value problems and PDEs (for stretch goals).
  \end{itemize}
\end{itemize}

\textbf{Module Tie-ins:}
\begin{itemize}
  \item \texttt{ode::ivp}, \texttt{ode::bvp}.
  \item \texttt{pde::} (for simple finite-difference PDE solvers).
\end{itemize}

\textbf{Exit Criteria:}
\begin{itemize}
  \item You can integrate basic ODEs with fixed and adaptive step sizes.
  \item You have at least one simple PDE (e.g., 1D heat equation) implemented and visualized.
\end{itemize}
\end{StoryCard}
\clearpage

\begin{StoryCard}{Phase 6 Card 1 -- Read ODE IVP Basics}
\textbf{Description:} Understand explicit methods and adaptive step-size control.

\medskip
\textbf{Reading Tasks:}
\begin{itemize}[label=$\square$]
  \item Numerical Recipes, sections 17.0--17.2:
  \begin{itemize}
    \item Runge--Kutta methods.
    \item Step-size control strategies.
  \end{itemize}
\end{itemize}

\textbf{Notes Deliverable:}
\begin{itemize}
  \item Short summary of:
  \begin{itemize}
    \item Local vs.\ global truncation error.
    \item Stability considerations for explicit methods.
  \end{itemize}
\end{itemize}

\textbf{Tags:} \texttt{Reading}, \texttt{Book:NR}, \texttt{Module:ode}
\end{StoryCard}
\clearpage

\begin{StoryCard}{Phase 6 Card 2 -- Implement ODE IVP Solvers}
\textbf{Description:} Implement core IVP solvers and test them on classic examples.

\medskip
\textbf{Implementation Tasks:}
\begin{itemize}[label=$\square$]
  \item Implement:
  \begin{itemize}
    \item Explicit Euler step.
    \item RK4 step.
    \item Simple adaptive RK method (e.g., embedded RK pair).
  \end{itemize}
  \item Design an API:
  \begin{itemize}
    \item \texttt{ode::solve\_ivp(f, t0, y0, t\_end, h\_initial, options)} returning sampled trajectory.
  \end{itemize}
\end{itemize}

\textbf{Experiment Tasks:}
\begin{itemize}[label=$\square$]
  \item Test on:
  \begin{itemize}
    \item Exponential decay.
    \item Harmonic oscillator.
    \item Simple nonlinear systems.
  \end{itemize}
  \item Plot solution trajectories and compare against analytic solutions where available.
\end{itemize}

\textbf{Deliverables:}
\begin{itemize}
  \item \texttt{ode/ivp.h}, \texttt{ode/ivp.cpp}.
  \item Example plots of ODE solutions.
\end{itemize}

\textbf{Tags:} \texttt{Module:ode}, \texttt{Implementation}, \texttt{Experiment}
\end{StoryCard}
\clearpage

\begin{StoryCard}{Phase 6 Card 3 -- Read Stiff ODEs \& Multistep Methods}
\textbf{Description:} Understand the issues around stiffness and multistep approaches.

\medskip
\textbf{Reading Tasks:}
\begin{itemize}[label=$\square$]
  \item Numerical Recipes, sections 17.5--17.6:
  \begin{itemize}
    \item Stiff ODEs.
    \item Multistep methods.
  \end{itemize}
\end{itemize}

\textbf{Notes Deliverable:}
\begin{itemize}
  \item Short summary of:
  \begin{itemize}
    \item What stiffness is and why explicit methods fail.
    \item Basic idea of implicit methods and stability regions.
  \end{itemize}
\end{itemize}

\textbf{Tags:} \texttt{Reading}, \texttt{Book:NR}, \texttt{Module:ode}
\end{StoryCard}
\clearpage

\begin{StoryCard}{Phase 6 Card 4 -- Implement Simple Stiff ODE Solver}
\textbf{Description:} Implement a basic implicit scheme using your linear algebra module.

\medskip
\textbf{Implementation Tasks:}
\begin{itemize}[label=$\square$]
  \item Implement implicit Euler step:
  \begin{itemize}
    \item Use Newton iteration to solve the implicit equation at each step.
  \end{itemize}
  \item Reuse \texttt{linalg::} and \texttt{nonlin::} modules.
\end{itemize}

\textbf{Experiment Tasks:}
\begin{itemize}[label=$\square$]
  \item Test on stiff problems (e.g., simple reaction equations).
  \item Compare explicit vs.\ implicit schemes and illustrate instability vs.\ stability.
\end{itemize}

\textbf{Deliverables:}
\begin{itemize}
  \item Extension of \texttt{ode/ivp.*} or a separate \texttt{ode/stiff.*}.
  \item Plots showing behavior on stiff vs.\ non-stiff examples.
\end{itemize}

\textbf{Tags:} \texttt{Module:ode}, \texttt{Module:linalg}, \texttt{Module:nonlin-root}, \texttt{Implementation}, \texttt{Experiment}
\end{StoryCard}
\clearpage

\begin{StoryCard}{Phase 6 Card 5 -- Read PDE Overview \& Implement 1D Heat Equation}
\textbf{Description:} Take a first step into PDEs with a finite-difference 1D heat equation.

\medskip
\textbf{Reading Tasks:}
\begin{itemize}[label=$\square$]
  \item Numerical Recipes, sections 20.0--20.2:
  \begin{itemize}
    \item Flux-conservative and diffusive PDEs.
  \end{itemize}
\end{itemize}

\textbf{Implementation Tasks:}
\begin{itemize}[label=$\square$]
  \item Discretize the 1D heat equation with:
  \begin{itemize}
    \item Explicit scheme (forward Euler in time, central differences in space).
    \item Implicit scheme if time allows.
  \end{itemize}
  \item Use your linear algebra module for the implicit solves.
\end{itemize}

\textbf{Experiment Tasks:}
\begin{itemize}[label=$\square$]
  \item Visualize temperature distribution over time.
  \item Compare stability and accuracy of explicit vs.\ implicit schemes.
\end{itemize}

\textbf{Deliverables:}
\begin{itemize}
  \item \texttt{pde/heat1d.h}, \texttt{pde/heat1d.cpp}.
  \item Plots or animations of heat diffusion.
\end{itemize}

\textbf{Tags:} \texttt{Module:pde}, \texttt{Module:linalg}, \texttt{Implementation}, \texttt{Experiment}
\end{StoryCard}
\clearpage

% ============================================
% Phase 7 – Random Numbers & Monte Carlo
% ============================================

\section*{Phase 7 -- Random Numbers \& Monte Carlo}

\begin{StoryCard}{Phase 7 Overview -- Random Numbers \& Monte Carlo}
\textbf{Goal:} Use random numbers to drive Monte Carlo integration and stochastic simulations.

\medskip
\textbf{Primary Sources:}
\begin{itemize}
  \item TAOCP Vol.~2, Chapter 3: random numbers.
  \item Numerical Recipes, Chapter 7: random numbers and Monte Carlo.
\end{itemize}

\textbf{Module Tie-ins:}
\begin{itemize}
  \item \texttt{rand::core}.
  \item \texttt{diffint::quad} (Monte Carlo as another integration approach).
\end{itemize}

\textbf{Exit Criteria:}
\begin{itemize}
  \item You have a solid uniform RNG and basic distribution helpers.
  \item You can implement and compare Monte Carlo integration with deterministic quadrature.
\end{itemize}
\end{StoryCard}
\clearpage

\begin{StoryCard}{Phase 7 Card 1 -- Read PRNG Theory \& Quality}
\textbf{Description:} Understand the theoretical background for random number generation.

\medskip
\textbf{Reading Tasks:}
\begin{itemize}[label=$\square$]
  \item TAOCP Vol.~2, sections 3.1--3.3:
  \begin{itemize}
    \item Linear congruential generators and other PRNGs.
    \item Statistical tests and randomness criteria.
  \end{itemize}
  \item Numerical Recipes, sections 7.0--7.4:
  \begin{itemize}
    \item Practical PRNGs and basic tests.
  \end{itemize}
\end{itemize}

\textbf{Notes Deliverable:}
\begin{itemize}
  \item Brief notes on:
  \begin{itemize}
    \item Why naive generators fail.
    \item Key properties of good RNGs (period, equidistribution).
  \end{itemize}
\end{itemize}

\textbf{Tags:} \texttt{Reading}, \texttt{Book:TAOCP2}, \texttt{Book:NR}, \texttt{Module:rand}
\end{StoryCard}
\clearpage

\begin{StoryCard}{Phase 7 Card 2 -- Implement Core RNG Module}
\textbf{Description:} Implement uniform PRNGs and common distributions.

\medskip
\textbf{Implementation Tasks:}
\begin{itemize}[label=$\square$]
  \item Implement a high-quality uniform RNG (or wrap the C++ standard library in a controlled way).
  \item Provide:
  \begin{itemize}
    \item \texttt{rand::uniform()}, \texttt{rand::normal()}, \texttt{rand::exponential()}, etc.
  \end{itemize}
  \item Support seeding and independent streams.
\end{itemize}

\textbf{Experiment Tasks:}
\begin{itemize}[label=$\square$]
  \item Perform basic statistical checks:
  \begin{itemize}
    \item Histograms for various distributions.
    \item Simple correlation and autocorrelation checks.
  \end{itemize}
\end{itemize}

\textbf{Deliverables:}
\begin{itemize}
  \item \texttt{rand/rand.h}, \texttt{rand/rand.cpp}.
  \item Small test programs visualizing distribution histograms.
\end{itemize}

\textbf{Tags:} \texttt{Module:rand}, \texttt{Implementation}, \texttt{Experiment}
\end{StoryCard}
\clearpage

\begin{StoryCard}{Phase 7 Card 3 -- Implement Monte Carlo Integrators}
\textbf{Description:} Use your RNG to implement Monte Carlo integration.

\medskip
\textbf{Implementation Tasks:}
\begin{itemize}[label=$\square$]
  \item Implement simple Monte Carlo integrators:
  \begin{itemize}
    \item \texttt{mc\_integrate(f, domain, n\_samples)}.
  \end{itemize}
  \item Optionally add quasi-random sequences (e.g., Sobol or Halton).
\end{itemize}

\textbf{Experiment Tasks:}
\begin{itemize}[label=$\square$]
  \item Compare Monte Carlo vs.\ deterministic quadrature from Phase~3:
  \begin{itemize}
    \item Error vs.\ number of samples.
    \item When Monte Carlo is preferable (e.g., higher-dimensional integrals).
  \end{itemize}
\end{itemize}

\textbf{Deliverables:}
\begin{itemize}
  \item \texttt{diffint/mc\_quad.h}, \texttt{diffint/mc\_quad.cpp} or integrated into \texttt{diffint::quad}.
  \item Plots comparing convergence rates.
\end{itemize}

\textbf{Tags:} \texttt{Module:rand}, \texttt{Module:diffint-quad}, \texttt{Implementation}, \texttt{Experiment}
\end{StoryCard}
\clearpage

% ============================================
% Phase 8 – Capstone Projects
% ============================================

\section*{Phase 8 -- Capstone Projects}

\begin{StoryCard}{Phase 8 Overview -- Capstone Pipelines}
\textbf{Goal:} Execute small end-to-end projects that exercise multiple modules at once.

\medskip
\textbf{Capstone Themes:}
\begin{itemize}
  \item Nonlinear boundary-value ODE problems.
  \item PDE-based models coupled with optimization.
\end{itemize}

\textbf{Exit Criteria:}
\begin{itemize}
  \item At least one nontrivial project is fully implemented, tested, and documented.
  \item You can explain how each module participates in the pipeline.
\end{itemize}
\end{StoryCard}
\clearpage

\begin{StoryCard}{Phase 8 Card 1 -- Capstone: Nonlinear Boundary-Value ODE}
\textbf{Description:} Couple root finding, quadrature, and ODE solvers to handle a nonlinear BVP.

\medskip
\textbf{Pipeline Sketch:}
\begin{itemize}
  \item Model a simple physical or mathematical system (e.g., nonlinear beam, shooting method BVP).
  \item Use:
  \begin{itemize}
    \item \texttt{ode::ivp} for shooting.
    \item \texttt{nonlin::root} for matching boundary conditions.
    \item \texttt{diffint::quad} if necessary for quantities defined by integrals.
  \end{itemize}
\end{itemize}

\textbf{Tasks:}
\begin{itemize}[label=$\square$]
  \item Define the differential equation and boundary conditions.
  \item Implement a shooting method:
  \begin{itemize}
    \item Convert BVP to IVP.
    \item Use root finding to adjust initial conditions.
  \end{itemize}
  \item Validate against known or reference solutions if available.
\end{itemize}

\textbf{Deliverables:}
\begin{itemize}
  \item An \texttt{examples/bvp\_capstone.cpp} (or similar) program.
  \item A short write-up (1--2 pages) explaining model, methods, and results.
\end{itemize}

\textbf{Tags:} \texttt{Module:ode}, \texttt{Module:nonlin-root}, \texttt{Module:diffint-quad}, \texttt{Capstone}
\end{StoryCard}
\clearpage

\begin{StoryCard}{Phase 8 Card 2 -- Capstone: PDE + Optimization}
\textbf{Description:} Combine PDE solving with parameter optimization.

\medskip
\textbf{Pipeline Sketch:}
\begin{itemize}
  \item Choose a simple steady-state PDE model (e.g., Laplace or Poisson).
  \item Introduce a parameter to be fitted (e.g., diffusion coefficient, source term amplitude).
  \item Use:
  \begin{itemize}
    \item \texttt{pde::} for discretization and solution.
    \item \texttt{optim::} to fit the parameter to synthetic or real data.
    \item \texttt{stats::} (if implemented) to evaluate goodness of fit.
  \end{itemize}
\end{itemize}

\textbf{Tasks:}
\begin{itemize}[label=$\square$]
  \item Implement the PDE solver for the chosen model.
  \item Define an objective function measuring mismatch between solution and data.
  \item Optimize the parameter using methods from \texttt{optim::}.
  \item Analyze sensitivity to noise and initial guesses.
\end{itemize}

\textbf{Deliverables:}
\begin{itemize}
  \item An \texttt{examples/pde\_optim\_capstone.cpp} (or similar) program.
  \item A brief report summarizing model, numerical methods, and optimization results.
\end{itemize}

\textbf{Tags:} \texttt{Module:pde}, \texttt{Module:optim}, \texttt{Module:stats}, \texttt{Capstone}
\end{StoryCard}
\clearpage

% ============================================
% Module Reference (Optional Section)
% ============================================

\section*{Module Reference (Quick Mapping)}

\begin{StoryCard}{Module \texttt{core::fp} -- Floating-Point \& Basic Arithmetic}
\textbf{Responsibilities:}
\begin{itemize}
  \item Provide floating-point utilities: \texttt{ulp}, \texttt{nearly\_equal}.
  \item Robust summation and dot-product routines.
  \item Error metrics: absolute, relative, norms.
\end{itemize}

\textbf{Book Mapping:}
\begin{itemize}
  \item TAOCP Vol.~2, Chapter 4 (arithmetic, floating-point).
  \item Numerical Recipes, Chapter 1 (error, accuracy, stability).
\end{itemize}
\end{StoryCard}
\clearpage

\begin{StoryCard}{Module \texttt{core::poly} -- Polynomial \& Power-Series Arithmetic}
\textbf{Responsibilities:}
\begin{itemize}
  \item Represent polynomials and power series.
  \item Implement Horner evaluation, addition, multiplication, composition.
  \item Support Chebyshev and other polynomial-based approximations.
\end{itemize}

\textbf{Book Mapping:}
\begin{itemize}
  \item TAOCP Vol.~2, sections 4.6--4.7 (polynomials, power series).
  \item Numerical Recipes, section 5.1 and 5.8--5.12 (polynomials, Chebyshev methods).
\end{itemize}
\end{StoryCard}
\clearpage

\begin{StoryCard}{Module \texttt{rand::core} -- Random Numbers \& Distributions}
\textbf{Responsibilities:}
\begin{itemize}
  \item High-quality uniform PRNGs.
  \item Transformations to normal, exponential, and other distributions.
  \item Quasi-random sequences for Monte Carlo.
\end{itemize}

\textbf{Book Mapping:}
\begin{itemize}
  \item TAOCP Vol.~2, Chapter 3 (random numbers).
  \item Numerical Recipes, Chapter 7 (random numbers and Monte Carlo).
\end{itemize}
\end{StoryCard}
\clearpage

\begin{StoryCard}{Module \texttt{interp::} -- Interpolation \& Approximation}
\textbf{Responsibilities:}
\begin{itemize}
  \item Table lookup and interpolation in 1D (and later multidim).
  \item Polynomial, rational, spline, and Chebyshev interpolation.
\end{itemize}

\textbf{Book Mapping:}
\begin{itemize}
  \item TAOCP Vol.~2, sections 4.6--4.7 (as algebraic backbone).
  \item Numerical Recipes, Chapter 3 and sections 5.8--5.10 (interpolation, Chebyshev).
\end{itemize}
\end{StoryCard}
\clearpage

\begin{StoryCard}{Module \texttt{diffint::deriv} -- Numerical Differentiation}
\textbf{Responsibilities:}
\begin{itemize}
  \item Finite-difference derivatives (forward, backward, central, higher order).
  \item Step-size selection and error modeling.
  \item Optional Richardson extrapolation.
\end{itemize}

\textbf{Book Mapping:}
\begin{itemize}
  \item Numerical Recipes, section 5.7 (numerical derivatives).
\end{itemize}
\end{StoryCard}
\clearpage

\begin{StoryCard}{Module \texttt{diffint::quad} -- Numerical Integration \& Monte Carlo}
\textbf{Responsibilities:}
\begin{itemize}
  \item 1D quadrature: trapezoid, Simpson, Romberg.
  \item Gaussian and adaptive quadrature.
  \item Multidimensional and Monte Carlo integration (ties to \texttt{rand::core}).
\end{itemize}

\textbf{Book Mapping:}
\begin{itemize}
  \item Numerical Recipes, Chapter 4 (integration of functions).
  \item Numerical Recipes, sections 7.7--7.9 (Monte Carlo integration).
\end{itemize}
\end{StoryCard}
\clearpage

\begin{StoryCard}{Module \texttt{linalg::} -- Linear Algebra Core}
\textbf{Responsibilities:}
\begin{itemize}
  \item Dense linear systems: LU, Cholesky, possibly QR/SVD.
  \item Banded and special matrices.
  \item Supporting routines for eigenproblems as needed.
\end{itemize}

\textbf{Book Mapping:}
\begin{itemize}
  \item Numerical Recipes, Chapter 2 (linear equations).
  \item Numerical Recipes, Chapter 11 (eigensystems), if used.
\end{itemize}
\end{StoryCard}
\clearpage

\begin{StoryCard}{Module \texttt{nonlin::root} -- Root Finding \& Nonlinear Systems}
\textbf{Responsibilities:}
\begin{itemize}
  \item 1D root solvers: bisection, secant, Brent, Newton.
  \item Nonlinear systems via Newton and globalized variants.
\end{itemize}

\textbf{Book Mapping:}
\begin{itemize}
  \item Numerical Recipes, Chapter 9 (root finding and nonlinear systems).
\end{itemize}
\end{StoryCard}
\clearpage

\begin{StoryCard}{Module \texttt{optim::} -- Optimization \& Minimization}
\textbf{Responsibilities:}
\begin{itemize}
  \item 1D minimization: golden-section, Brent.
  \item Multidimensional unconstrained optimization (gradient-based or simplex methods).
\end{itemize}

\textbf{Book Mapping:}
\begin{itemize}
  \item Numerical Recipes, Chapter 10 (minimization or maximization of functions).
\end{itemize}
\end{StoryCard}
\clearpage

\begin{StoryCard}{Module \texttt{spectral::} -- FFT \& Spectral Tools}
\textbf{Responsibilities:}
\begin{itemize}
  \item FFT and inverse FFT.
  \item Spectral differentiation/integration.
  \item Convolution and filtering primitives.
\end{itemize}

\textbf{Book Mapping:}
\begin{itemize}
  \item Numerical Recipes, Chapters 12--13 (FFT and spectral applications).
\end{itemize}
\end{StoryCard}
\clearpage

\begin{StoryCard}{Module \texttt{ode::ivp}, \texttt{ode::bvp} -- Ordinary Differential Equations}
\textbf{Responsibilities:}
\begin{itemize}
  \item IVP solvers: explicit/implicit, adaptive, stiff and non-stiff.
  \item BVP solvers: shooting methods, relaxation methods.
\end{itemize}

\textbf{Book Mapping:}
\begin{itemize}
  \item Numerical Recipes, Chapters 17--18 (ODEs and BVPs).
\end{itemize}
\end{StoryCard}
\clearpage

\begin{StoryCard}{Module \texttt{pde::} -- Partial Differential Equations}
\textbf{Responsibilities:}
\begin{itemize}
  \item Finite-difference solvers for simple PDEs (1D/2D).
  \item Explicit/implicit schemes, stability helpers, and boundary conditions.
\end{itemize}

\textbf{Book Mapping:}
\begin{itemize}
  \item Numerical Recipes, Chapter 20 (PDEs).
\end{itemize}
\end{StoryCard}
\clearpage

\begin{StoryCard}{Module \texttt{stats::} -- Statistical Tools \& Error Modeling}
\textbf{Responsibilities:}
\begin{itemize}
  \item Descriptive statistics and correlation.
  \item Least squares and error/convergence analysis for experiments.
\end{itemize}

\textbf{Book Mapping:}
\begin{itemize}
  \item Numerical Recipes, Chapters 14--15 (statistics and data modeling).
\end{itemize}
\end{StoryCard}
\clearpage

\end{document}
