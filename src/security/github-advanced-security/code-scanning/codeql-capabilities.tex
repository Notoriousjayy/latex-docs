\documentclass[11pt,letterpaper]{article}

% ---------- Encoding, fonts, layout ----------
\usepackage[utf8]{inputenc}
\usepackage[T1]{fontenc}
\usepackage{lmodern}
\usepackage[a4paper,margin=1in]{geometry}
\usepackage{microtype}
\usepackage{parskip}

% ---------- Colors, links, headings ----------
\usepackage[dvipsnames]{xcolor}
\usepackage{hyperref}
\hypersetup{
  colorlinks = true,
  linkcolor  = MidnightBlue,
  urlcolor   = MidnightBlue,
  citecolor  = MidnightBlue
}
\usepackage{titlesec}
\titlespacing*{\section}{0pt}{8pt}{4pt}
\titlespacing*{\subsection}{0pt}{6pt}{3pt}

% ---------- Lists ----------
\usepackage{enumitem}
\setlist{itemsep=2pt, topsep=4pt, leftmargin=1.2em}

% ---------- Code highlighting ----------
\usepackage{minted-config} % compile with -shell-escape
\setminted{
  fontsize=\small,
  breaklines,
  breakanywhere,
  autogobble,
  tabsize=2,
  frame=single,
  framesep=2mm
}
\newminted[yamlcode]{yaml}{}
\newminted[bashcode]{bash}{}

% ---------- Title ----------
\title{\vspace{-1.5em}\textbf{CodeQL Capabilities — Exam-Ready Cheat Sheet}\vspace{-0.5em}}
\date{}
\author{}

\begin{document}
\maketitle
\vspace{-1.25em}
\hrule
\vspace{0.75em}

\section{What is CodeQL?}
\textbf{CodeQL} is GitHub's semantic static analysis engine for code scanning. It builds a \emph{relational database} of your code and runs declarative \textbf{QL} queries over it to detect complex vulnerabilities that simple pattern matching misses.

\section{Why Use It}
\begin{itemize}
  \item Finds data-flow issues (e.g., injection), unsafe crypto, and other vuln classes across many languages.
  \item Integrates with GitHub Actions, runs locally via the \textbf{CodeQL CLI}, and supports authoring in VS Code.
  \item Produces precise results with traces and remediation context.
\end{itemize}

\section{Key Terms}
\begin{description}[style=unboxed,leftmargin=0cm]
  \item[QL Pack] A collection of related CodeQL queries for a \emph{specific language} (e.g., Python).
  \item[Code Query (\texttt{.ql})] A single query to find a specific issue (e.g., SQLi) via CLI/Actions/VS Code.
  \item[Query / Code Suite] A curated \emph{set} of queries grouped by security theme (e.g., web-app security).
\end{description}

\section{Default Query Suites}
GitHub provides prebuilt suites in the public \texttt{github/codeql} repository. Choose suites by \emph{domain} (e.g., security-extended) and pair them with your repo's languages in Actions or the CLI.

\section{End-to-End Flow}
\begin{enumerate}
  \item \textbf{Create database} per language (semantic model of your code).
  \item \textbf{Run queries/suites} over that database.
  \item \textbf{Review results} with locations, data-flow traces, and guidance (great UX in VS Code).
\end{enumerate}
\clearpage

\section{Minimal GitHub Actions Setup (Multi-language)}
\noindent\emph{Drop into \texttt{.github/workflows/codeql.yml}:}
\begin{yamlcode}
name: CodeQL
on:
  push: { branches: [main] }
  pull_request: { branches: [main] }
  schedule:
    - cron: "0 2 * * 1"  # weekly

jobs:
  analyze:
    runs-on: ubuntu-latest
    permissions:
      security-events: write
      contents: read
      actions: read
    strategy:
      fail-fast: false
      matrix:
        language: [javascript-typescript, python]  # add more as needed
    steps:
      - uses: actions/checkout@v4

      - uses: github/codeql-action/init@v3
        with:
          languages: ${{ matrix.language }}
          queries: +security-extended  # or a custom suite

      - uses: github/codeql-action/autobuild@v3

      - uses: github/codeql-action/analyze@v3
        with:
          category: "/language:${{ matrix.language }}"
\end{yamlcode}

\section{Local Workflow (CLI + VS Code)}
\begin{bashcode}
# 1) Create the database (per language)
codeql database create ./db-py \
  --language=python \
  --source-root /path/to/repo

# 2) Analyze with a suite or pack
codeql database analyze ./db-py \
  github/codeql/python/ql/src/codeql-suites/python-security-extended.qls \
  --format=sarifv2.1.0 --output=results.sarif

# Tip: Use the VS Code "CodeQL" extension to open DBs,
# author queries, and inspect results interactively.
\end{bashcode}
\clearpage

\section{Authoring Queries (Mental Model)}
\begin{itemize}
  \item QL is \emph{declarative}: define types/predicates, then finish with \texttt{from ... where ... select ...}.
  \item Data-flow pattern: define \textbf{sources} (e.g., user input) and \textbf{sinks} (e.g., SQL exec), then search for flows without sanitization.
  \item Start from existing queries in \texttt{github/codeql}; customize progressively.
\end{itemize}

\section{Practical Tips}
\begin{itemize}
  \item Begin with GitHub-provided suites; extend with org-specific query packs.
  \item Cache databases in CI and shard by language for large monorepos.
  \item Keep CLI invocations scriptable; capture SARIF for artifacting and dashboards.
  \item Triage in the Code Scanning UI; iterate queries as \emph{guardrails}, not gates, until signal is strong.
\end{itemize}

\vfill
\hrule
\smallskip
{\footnotesize
\textbf{Build Note:} Compile with \texttt{latexmk -pdf -shell-escape <file>} (or enable \emph{--shell-escape} in your LaTeX tool) so \texttt{minted} can call Pygments.
}

\end{document}

