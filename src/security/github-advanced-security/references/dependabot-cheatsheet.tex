
% !TEX TS-program = pdflatex
% Compile with: pdflatex dependabot-cheatsheet.tex
\documentclass[11pt,letterpaper]{article}

% ---------- Encoding & fonts ----------
\usepackage[T1]{fontenc}
\usepackage[utf8]{inputenc}
\usepackage{lmodern}
\usepackage{inconsolata}
\usepackage{microtype}

% ---------- Page & links ----------
\usepackage[margin=1in]{geometry}
\usepackage{parskip}
\usepackage{hyperref}
\hypersetup{
  colorlinks=true,
  linkcolor=black,
  urlcolor=blue,
  citecolor=black
}

% ---------- Colors (must come before titlesec if using colors there) ----------
\usepackage[dvipsnames]{xcolor}
\definecolor{CodeBg}{RGB}{248,248,248}
\definecolor{Accent}{RGB}{0,86,179}
\definecolor{NoteBg}{RGB}{240,248,255}
\definecolor{soft}{HTML}{F3F4F6}

% ---------- Headings ----------
\usepackage{titlesec}
\titleformat{\section}{\large\bfseries\color{Accent}}{\thesection}{0.6em}{}
\titleformat{\subsection}{\bfseries}{\thesubsection}{0.5em}{}
\titlespacing*{\section}{0pt}{8pt}{4pt}
\titlespacing*{\subsection}{0pt}{6pt}{3pt}

% ---------- Lists & tables ----------
\usepackage{enumitem}
\setlist{itemsep=2pt, topsep=4pt, leftmargin=1.1em}
\usepackage{booktabs}
\usepackage{array}
\newcolumntype{L}[1]{>{\raggedright\arraybackslash}p{#1}}
\newcolumntype{C}[1]{>{\centering\arraybackslash}p{#1}}

% ---------- Nice callouts ----------
\usepackage[most]{tcolorbox}
\tcbset{before skip=6pt, after skip=6pt}
\newtcolorbox{noteBox}{
  colback=NoteBg, colframe=Accent!60!black, sharp corners, boxrule=0.5pt
}

% ---------- Code blocks (listings only - CI safe) ----------
\usepackage{listings}
\usepackage{upquote}

\lstdefinelanguage{yaml}{
  keywords={true,false,null},
  sensitive=false,
  comment=[l]{\#},
  morestring=[b]',
  morestring=[b]"
}

\lstset{
  basicstyle=\ttfamily\small,
  backgroundcolor=\color{soft},
  breaklines=true,
  breakatwhitespace=false,
  columns=fullflexible,
  keepspaces=true,
  showstringspaces=false,
  frame=single,
  framerule=0.4pt,
  tabsize=2,
  aboveskip=6pt,
  belowskip=6pt,
  literate=
    {—}{{---}}1
    {–}{{--}}1
    {→}{{$\rightarrow$}}1
}

% ---------- Code block environments ----------
\lstnewenvironment{bashcode}{\lstset{language=bash}}{}
\lstnewenvironment{yamlcode}{\lstset{language=yaml}}{}
\lstnewenvironment{jsoncode}{\lstset{}}{}
\lstnewenvironment{cmakecode}{\lstset{}}{}
\lstnewenvironment{textcode}{\lstset{}}{}
\lstnewenvironment{cppcode}{\lstset{language=C++}}{}
\lstnewenvironment{ccode}{\lstset{language=C}}{}
\lstnewenvironment{inicode}{\lstset{}}{}
\lstnewenvironment{pythoncode}{\lstset{language=Python}}{}
\newcommand{\mintinline}[3][]{\texttt{#3}}

% ---------- Title ----------
\title{\textbf{Dependabot Alerts \& Workflows — Hands-On Cheatsheet}}
\author{}
\date{}

\begin{document}
\maketitle

\begin{noteBox}
\textbf{Heads-up:} This document covers Dependabot alerts, security updates, and best practices for managing dependencies in GitHub repositories.
\end{noteBox}

\section{Core Concepts}
\begin{itemize}
  \item \textbf{Dependency types:} Direct dependencies are packages your code imports/uses. \textit{Transitive} dependencies are packages required by your direct dependencies.
  \item \textbf{Where alerts appear:} Security tab (repository overview), Pull Requests (as comments/checks), and via API/CLI.
  \item \textbf{Security updates:} Dependabot can automatically open PRs to bump vulnerable versions.
\end{itemize}

\section{Enable Security Updates}
Add or adjust \texttt{dependabot.yml} at repo/org level to scan ecosystems and open PRs automatically.
\begin{yamlcode}
version: 2
updates:
  - package-ecosystem: "pip"        # npm, maven, gradle, gomod, etc.
    directory: "/"                  # location of manifest
    schedule:
      interval: "daily"             # daily, weekly, monthly
    open-pull-requests-limit: 5
    rebase-strategy: auto
    reviewers: ["team/security"]
    labels: ["dependencies","security"]
\end{yamlcode}

\section{Triage Workflow}
\begin{enumerate}
  \item \textbf{Review alert severity:} Critical/High alerts should be prioritized.
  \item \textbf{Check if exploitable:} Is the vulnerable code path actually used?
  \item \textbf{Evaluate fix availability:} Is there a patched version?
  \item \textbf{Test upgrade:} Run CI/CD pipeline with the proposed changes.
  \item \textbf{Merge or dismiss:} Document the decision either way.
\end{enumerate}

\section{CLI Commands}
\begin{bashcode}
# List Dependabot alerts for a repository
gh api repos/{owner}/{repo}/dependabot/alerts

# Get specific alert details
gh api repos/{owner}/{repo}/dependabot/alerts/{alert_number}

# Dismiss an alert (with reason)
gh api repos/{owner}/{repo}/dependabot/alerts/{alert_number} \
  -X PATCH -f state=dismissed -f dismissed_reason="tolerable_risk"
\end{bashcode}

\section{Best Practices}
\begin{itemize}
  \item Enable Dependabot for all supported ecosystems in your repository.
  \item Set up branch protection rules requiring security checks to pass.
  \item Review and merge security updates promptly (within 24-48 hours for Critical).
  \item Use \texttt{@dependabot rebase} comment to update stale PRs.
  \item Configure grouped updates to reduce PR noise for related packages.
\end{itemize}

\end{document}
