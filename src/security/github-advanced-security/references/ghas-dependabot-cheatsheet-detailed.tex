% !TEX program = pdflatex
\documentclass[10pt,letterpaper]{article}

% ---------- Packages ----------
\usepackage[utf8]{inputenc}

% ---------- Minted (CI-safe fallback) ----------
\usepackage{xparse}
\usepackage{iftex}
\usepackage{ifthen}

% Some repos run pdflatex without -shell-escape in CI.
% Use minted when shell-escape is enabled; otherwise fall back to listings.
\newif\ifciShellEscape
\ifdefined\pdfshellescape
  \ifnum\pdfshellescape=1\relax
    \ciShellEscapetrue
  \else
    \ciShellEscapefalse
  \fi
\else
  \ciShellEscapefalse
\fi

\ifciShellEscape
  \usepackage[newfloat]{minted}
\else
  \usepackage{listings}
  \usepackage{newfloat}
  \DeclareFloatingEnvironment[fileext=lol, listname={List of Listings}, name=Listing]{listing}

  % Basic listings defaults (safe for UTF-8 text in code blocks)
  \lstset{
    basicstyle=\ttfamily\small,
    breaklines=true,
    breakatwhitespace=false,
    columns=fullflexible,
    keepspaces=true,
    showstringspaces=false,
    tabsize=2,
    frame=single,
    upquote=true,
    % Common Unicode glyphs seen in docs/snippets
    literate=
      {•}{{\textbullet}}1
      {–}{{--}}1
      {—}{{---}}1
      {→}{{$\rightarrow$}}1
      {←}{{$\leftarrow$}}1
  }

  % Define a few common "languages" so listings won't error
  \lstdefinelanguage{yaml}{} 
  \lstdefinelanguage{json}{} 
  \lstdefinelanguage{bash}{} 
  \lstdefinelanguage{console}{} 
  \lstdefinelanguage{powershell}{} 
  \lstdefinelanguage{markdown}{} 
  \lstdefinelanguage{text}{} 

  % minted compatibility shims (ignore minted-style options safely)
  \NewDocumentEnvironment{minted}{ O{} m }{%
    \def\minted@opts{#1}%
    \def\minted@lang{#2}%
    \ifthenelse{\begingroup\edef\temp{\minted@opts}\endgroup\in@{linenos}{\temp}\relax=1\relax}{%
      \lstset{numbers=left, numberstyle=\tiny, stepnumber=1, numbersep=6pt}%
    }{%
      \lstset{numbers=none}%
    }%
    \lstset{language=\minted@lang}%
    \begin{lstlisting}%
  }{%
    \end{lstlisting}%
  }

  \NewDocumentCommand{\mintinline}{ O{} m m }{\texttt{#3}}
  \NewDocumentCommand{\inputminted}{ O{} m m }{\lstinputlisting[language=#2]{#3}}

  \NewDocumentCommand{\setminted}{ O{} m }{}
  \NewDocumentCommand{\setmintedinline}{ O{} m }{}
  \NewDocumentCommand{\usemintedstyle}{ m }{}

  % \newminted{lang}{opts} -> defines an environment <lang>code
  \NewDocumentCommand{\newminted}{ m O{} }{%
    \NewDocumentEnvironment{#1code}{ O{} }{\begin{minted}{#1}}{\end{minted}}%
  }
  % \newmintedfile{lang}{opts} -> defines \input<lang>code{file}
  \NewDocumentCommand{\newmintedfile}{ m O{} }{%
    \expandafter\NewDocumentCommand\csname input#1code\endcsname{ m }{\inputminted{#1}{##1}}%
  }
\fi

\usepackage[T1]{fontenc}
\usepackage{lmodern}
\usepackage[letterpaper,margin=0.8in]{geometry}
\usepackage{microtype}
\usepackage{parskip}
\usepackage{enumitem}
\usepackage{xcolor}
\usepackage{hyperref}
\usepackage{titlesec}
\usepackage{inconsolata}

% ---------- Fix inconsolata/zi4 italic warning (fallback to slanted) ----------
\makeatletter
\@ifundefined{DeclareFontShape}{}{%
  \DeclareFontShape{T1}{zi4}{m}{it}{<->ssub*zi4/m/sl}{}
  \DeclareFontShape{T1}{zi4}{b}{it}{<->ssub*zi4/b/sl}{}
}
\makeatother

\usepackage{fvextra}    % improves verbatim (used by minted)

% ---------- Styling ----------
\definecolor{Ink}{HTML}{1F2937}
\definecolor{Accent}{HTML}{2563EB}
\definecolor{CodeBG}{HTML}{F3F4F6}
\definecolor{Gray}{HTML}{6B7280}
\hypersetup{colorlinks=true, linkcolor=Accent, urlcolor=Accent, citecolor=Accent}
\setlist{nosep, leftmargin=1.1em, topsep=2pt}
\titleformat{\section}{\large\bfseries\color{Ink}}{}{0pt}{}
\titlespacing*{\section}{0pt}{6pt}{3pt}
\titleformat{\subsection}{\bfseries\color{Ink}}{}{0pt}{}
\titlespacing*{\subsection}{0pt}{4pt}{2pt}

% ---------- Minted defaults ----------
\setminted{cache=false,
  fontsize=\small,
  breaklines=true,
  breakanywhere=true,
  autogobble=true,
  frame=single,
  bgcolor=CodeBG,
  tabsize=2,
}

% ---------- Silence first-run missing .toc (prevents strict latexmk aborts) ----------
\makeatletter
\let\ci@orig@starttoc\@starttoc
\def\@starttoc#1{%
  \begingroup
  \IfFileExists{\jobname.#1}{\ci@orig@starttoc{#1}}{}%
  \endgroup
}
\makeatother

\begin{document}
\begin{center}
{\Large \textbf{Dependabot \& Dependency Graph — 1 Page Cheat Sheet}}\\
\smallskip
\textcolor{Gray}{Quick reference for GHAS dependency risk management}
\end{center}

\section{Core Concepts}
\textbf{Vulnerability.} A weakness in software, hardware, or systems that attackers can exploit to gain access, steal data, or disrupt operations. Examples include buffer overflows that allow code injection.\\[2pt]
\textbf{Dependabot Alerts.} Automatic findings when dependencies in your repo match known vulnerable versions. Alerts include metadata, severity, and links to remediation guidance.\\[2pt]
\textbf{Dependabot Security Updates vs Version Updates.}
\begin{itemize}
  \item \textbf{Security Updates:} When a vulnerable version is detected and a safe version exists, Dependabot opens a PR to patch it.
  \item \textbf{Version Updates:} Dependabot proactively opens PRs to keep dependencies current as new versions are released (not just during vulnerability scans).
\end{itemize}
\textbf{Dependency Graph (SBOM).} A dynamic inventory of your repository's dependencies and their relationships; integrates with Dependabot and supports SBOM export.
\medskip

\section{How It Works (At a Glance)}
\begin{enumerate}
  \item Repo manifests (e.g., \texttt{requirements.txt}, \texttt{package-lock.json}, \texttt{pom.xml}) are analyzed.
  \item Dependabot builds the \emph{dependency graph} and tracks versions.
  \item Versions are compared against multiple sources (e.g., NVD, vendor advisories, package registries).
  \item When a match with a known vulnerable version is found, Dependabot issues an alert and, if enabled, opens a PR.
\end{enumerate}

\section{Quick Start: \texttt{.github/dependabot.yml}}
\noindent\textit{Place at repo root in \texttt{.github/}. Compile this doc with \texttt{-shell-escape} to enable syntax highlighting.}
\begin{minted}[linenos]{yaml}
version: 2
updates:
  - package-ecosystem: "pip"          # npm, maven, gradle, cargo, etc.
    directory: "/"                    # location of manifest (e.g., /app)
    schedule:
      interval: "weekly"              # daily | weekly | monthly
      day: "monday"
      time: "09:00"
    open-pull-requests-limit: 5
    reviewers:
      - "org/security-reviewers"
    labels: ["dependabot", "security"]
    ignore:
      - dependency-name: "pytest"
        versions: ["< 5.0.0"]
\end{minted}

\section{Notify Chat Platforms (Example)}
\noindent\textit{Idea: post new alerts or open PRs to Teams via webhook (similar patterns work for Slack).}
\begin{minted}[linenos]{bash}
# In a workflow step, send a message with curl (Teams Incoming Webhook)
curl -X POST "$TEAMS_WEBHOOK_URL" \
  -H 'Content-Type: application/json' \
  -d '{
    "text": "Dependabot found a vulnerable dependency. See Security tab."
  }'
\end{minted}

\section{Operational Tips}
\begin{itemize}
  \item \textbf{Act early.} Treat alerts like code reviews; triage continuously to reduce risk and PR backlog.
  \item \textbf{Tune noise.} Limit open Dependabot PRs per repo; batch schedules to avoid alert floods.
  \item \textbf{Label \& route.} Auto-label Dependabot PRs, assign reviewers, and wire notifications to Teams/Slack.
  \item \textbf{Dismiss responsibly.} Use consistent dismissal reasons (e.g., false positive, already remediated, will not fix with rationale).
\end{itemize}

\section{Exam/Interview Recall}
\begin{itemize}
  \item Dependabot sources include public vulnerability feeds (e.g., NVD), vendor advisories, package registries, partner feeds, community reports, and GitHub research.
  \item The \textbf{dependency graph} powers SBOM export and links findings to manifests.
  \item \textbf{Security updates} vs \textbf{version updates}: both make PRs; security updates react to vulnerabilities, version updates keep you current.
\end{itemize}

\bigskip
{\footnotesize Compile with: \texttt{latexmk -pdf -shell-escape -interaction=nonstopmode ghas-dependabot-cheatsheet-minted.tex}}
\end{document}
