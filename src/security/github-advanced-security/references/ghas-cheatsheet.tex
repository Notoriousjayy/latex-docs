% !TEX TS-program = pdflatex
\documentclass[10pt,a4paper]{article}

% -------------------- Packages --------------------
\usepackage[T1]{fontenc}

% ---------- Minted (CI-safe fallback) ----------
\usepackage{xparse}
\usepackage{iftex}
\usepackage{ifthen}

% Some repos run pdflatex without -shell-escape in CI.
% Use minted when shell-escape is enabled; otherwise fall back to listings.
\newif\ifciShellEscape
\ifdefined\pdfshellescape
  \ifnum\pdfshellescape=1\relax
    \ciShellEscapetrue
  \else
    \ciShellEscapefalse
  \fi
\else
  \ciShellEscapefalse
\fi

\ifciShellEscape
  \usepackage[newfloat,cache=false]{minted}
\else
  \usepackage{listings}
  \usepackage{newfloat}
  \DeclareFloatingEnvironment[fileext=lol, listname={List of Listings}, name=Listing]{listing}

  % Basic listings defaults (safe for UTF-8 text in code blocks)
  \lstset{
    basicstyle=\ttfamily\small,
    breaklines=true,
    breakatwhitespace=false,
    columns=fullflexible,
    keepspaces=true,
    showstringspaces=false,
    tabsize=2,
    frame=single,
    upquote=true,
    % Common Unicode glyphs seen in docs/snippets
    literate=
      {•}{{\textbullet}}1
      {–}{{--}}1
      {—}{{---}}1
      {→}{{$\rightarrow$}}1
      {←}{{$\leftarrow$}}1
  }

  % Define a few common "languages" so listings won't error
  \lstdefinelanguage{yaml}{} 
  \lstdefinelanguage{json}{} 
  \lstdefinelanguage{bash}{} 
  \lstdefinelanguage{console}{} 
  \lstdefinelanguage{powershell}{} 
  \lstdefinelanguage{markdown}{} 
  \lstdefinelanguage{text}{} 

  % minted compatibility shims (ignore minted-style options safely)
  \NewDocumentEnvironment{minted}{ O{} m }{%
    \def\minted@opts{#1}%
    \def\minted@lang{#2}%
    \ifthenelse{\begingroup\edef\temp{\minted@opts}\endgroup\in@{linenos}{\temp}\relax=1\relax}{%
      \lstset{numbers=left, numberstyle=\tiny, stepnumber=1, numbersep=6pt}%
    }{%
      \lstset{numbers=none}%
    }%
    \lstset{language=\minted@lang}%
    \begin{lstlisting}%
  }{%
    \end{lstlisting}%
  }

  \NewDocumentCommand{\mintinline}{ O{} m m }{\texttt{#3}}
  \NewDocumentCommand{\inputminted}{ O{} m m }{\lstinputlisting[language=#2]{#3}}

  \NewDocumentCommand{\setminted}{ O{} m }{}
  \NewDocumentCommand{\setmintedinline}{ O{} m }{}
  \NewDocumentCommand{\usemintedstyle}{ m }{}

  % \newminted{lang}{opts} -> defines an environment <lang>code
  \NewDocumentCommand{\newminted}{ m O{} }{%
    \NewDocumentEnvironment{#1code}{ O{} }{\begin{minted}{#1}}{\end{minted}}%
  }
  % \newmintedfile{lang}{opts} -> defines \input<lang>code{file}
  \NewDocumentCommand{\newmintedfile}{ m O{} }{%
    \expandafter\NewDocumentCommand\csname input#1code\endcsname{ m }{\inputminted{#1}{##1}}%
  }
\fi

\usepackage[utf8]{inputenc}
\usepackage{lmodern}
\usepackage{microtype}
\usepackage[a4paper,margin=0.75in]{geometry}
\usepackage{parskip}
\usepackage[dvipsnames]{xcolor}
\usepackage{hyperref}
\hypersetup{colorlinks=true, linkcolor=MidnightBlue, urlcolor=MidnightBlue, citecolor=MidnightBlue}
\usepackage{enumitem}
\setlist{itemsep=2pt, topsep=3pt, leftmargin=1.2em}
\usepackage{booktabs}
\usepackage{tabularx}
\usepackage{array}
\usepackage[most]{tcolorbox}
\usepackage{pifont}
\usepackage{ragged2e}
\usepackage{graphicx}
\usepackage{multicol}
\usepackage{titlesec}
\titleformat{\section}{\large\bfseries}{\thesection}{0.5em}{}
\titleformat{\subsection}{\normalsize\bfseries}{\thesubsection}{0.5em}{}

% -------------------- Minted Setup --------------------
\usemintedstyle{friendly}
\setminted{
  linenos,
  breaklines,
  fontsize=\footnotesize,
  tabsize=2,
  frame=single,
  framerule=0.3pt,
  rulecolor=\color{Gray},
  bgcolor=black!2,
  numbersep=5pt
}

% -------------------- Tcolorbox Styles --------------------
\tcbset{
  colframe=MidnightBlue,
  colback=blue!2,
  coltitle=black,
  arc=2mm,
  boxrule=0.5pt,
  left=6pt,right=6pt,top=6pt,bottom=6pt
}
\newtcolorbox{callout}[1][]{
  enhanced, sharp corners=south, rounded corners=southwest,
  fonttitle=\bfseries, title=#1
}
\newtcolorbox{pill}[1][]{
  enhanced, colback=gray!15, colframe=gray!40, boxrule=0pt, arc=10pt,
  left=6pt,right=6pt,top=2pt,bottom=2pt
}

% -------------------- Helpers --------------------
\newcommand{\checky}{\textcolor{ForestGreen}{\ding{51}}}
\newcommand{\cross}{\textcolor{BrickRed}{\ding{55}}}
\newcommand{\kbd}[1]{\colorbox{black!10}{\texttt{#1}}}

% -------------------- Document --------------------

% ---------- Silence first-run missing .toc (prevents strict latexmk aborts) ----------
\makeatletter
\let\ci@orig@starttoc\@starttoc
\def\@starttoc#1{%
  \begingroup
  \IfFileExists{\jobname.#1}{\ci@orig@starttoc{#1}}{}%
  \endgroup
}
\makeatother

\begin{document}

\begin{center}
  {\Large\bfseries GitHub Advanced Security (GHAS) --- Quick Reference}\\[2pt]
  {\small Version: \today \quad\textbar\quad Scope: Enterprise \textrightarrow{} Org \textrightarrow{} Repo}
\end{center}

\vspace{0.5em}

\begin{callout}[What GHAS Gives You]
\begin{multicols}{2}
\begin{itemize}
  \item \textbf{Code scanning (CodeQL)} --- SAST with a code-as-data model.
  \item \textbf{Secret scanning} (+ push protection) --- detect and prevent credential leaks.
  \item \textbf{Dependency insights} --- advisories, alerts, automated PRs (Dependabot).
  \item \textbf{Security Overview dashboards} at Enterprise/Org/Repo.
  \item \textbf{Policy inheritance} to auto-enable for new repositories.
\end{itemize}
\columnbreak
\begin{itemize}
  \item \textbf{Shift-left integration} in PR checks and GitHub Actions.
  \item \textbf{Prioritized fix guidance} and audit trail via checks \& reviews.
  \item \textbf{Container/image scanning} surfaced in Security Overview.
  \item \textbf{Compliance support}: enforce status checks before merge.
\end{itemize}
\end{multicols}
\end{callout}

% -------------------- Top-level Concepts --------------------
\section{Key Concepts}

\begin{tabularx}{\linewidth}{@{}>{\raggedright\arraybackslash}p{0.22\linewidth}X@{}}
\toprule
\textbf{Term} & \textbf{Meaning / Notes}\\
\midrule
GHAS & Paid add-on; unlocks code scanning, secret scanning (incl.\ push protection), and advanced dashboards.\\
GHEC & GitHub Enterprise Cloud (github.com).\\
GHES & GitHub Enterprise Server (self-hosted).\\
Security Overview & Aggregated dashboards: view risk \& coverage, drill into alerts, track remediation.\\
\bottomrule
\end{tabularx}

\section{Enablement Path (Top-down)}

\begin{enumerate}
  \item \textbf{Enterprise Settings} $\rightarrow$ \texttt{Policies} $\rightarrow$ \texttt{Code security and analysis}: prefer \emph{inheritance} and default-on for new orgs/repos.
  \item \textbf{Organization Settings} $\rightarrow$ \texttt{Security \& analysis}: enable \emph{Code scanning}, \emph{Secret scanning} (with \emph{Push protection}), and \emph{Dependency alerts}.
  \item \textbf{Repository Settings} $\rightarrow$ \texttt{Security \& analysis}: confirm features \& configure branch protection rules to \emph{require} passing checks.
\end{enumerate}

\begin{callout}[Minimum Safe Defaults]
\begin{itemize}
  \item Secret scanning \textbf{ON} \& \textbf{Push protection ON} for all repos (including private).
  \item Code scanning \textbf{ON} with default CodeQL queries for primary languages.
  \item Dependency alerts \textbf{ON} \& Dependabot security updates \textbf{ON}.
  \item Protected branches require: \emph{All code scanning checks} and \emph{No secret scanning violations} before merge.
\end{itemize}
\end{callout}
\clearpage

% -------------------- CI/CD Integration --------------------
\section{CI/CD Integration}

\subsection*{Code scanning (CodeQL) --- GitHub Actions}
\noindent Create \texttt{.github/workflows/codeql.yml}:

\vspace{0.5em}
\noindent\textit{CodeQL code scanning for a multi-language repo}
\begin{minted}{yaml}
name: CodeQL
on:
  push:
    branches: [ "main" ]
  pull_request:
    branches: [ "main" ]
  schedule:
    - cron:  '0 3 * * 0'   # weekly run (Sun @ 03:00 UTC)

jobs:
  analyze:
    runs-on: ubuntu-latest
    permissions:
      actions: read
      contents: read
      security-events: write
    strategy:
      fail-fast: false
      matrix:
        language: [ 'javascript', 'typescript', 'python', 'java' ]
    steps:
      - uses: actions/checkout@v4
      - uses: github/codeql-action/init@v3
        with:
          languages: ${{ matrix.language }}
      - uses: github/codeql-action/autobuild@v3
      - uses: github/codeql-action/analyze@v3
        with:
          category: "/language:${{ matrix.language }}"
\end{minted}

\subsection*{Dependabot Security Updates}
\noindent Create \texttt{.github/dependabot.yml}:

\vspace{0.5em}
\noindent\textit{Dependabot security updates (weekly)}
\begin{minted}{yaml}
version: 2
updates:
  - package-ecosystem: "npm"
    directory: "/"
    schedule: { interval: "weekly" }
  - package-ecosystem: "maven"
    directory: "/"
    schedule: { interval: "weekly" }
  - package-ecosystem: "pip"
    directory: "/"
    schedule: { interval: "weekly" }
\end{minted}

\subsection*{Secret Scanning \& Push Protection}
\begin{itemize}
  \item Enabled via \texttt{Settings} $\rightarrow$ \texttt{Security \& analysis}. Configure org-wide defaults and \emph{require fixes before merge} using branch protection.
  \item \textbf{Triage tip}: Verify, revoke, rotate, then remediate. Add dismissals only with justification (false positive, test credential, or mitigated).
\end{itemize}

% -------------------- Triage and Policy --------------------
\section{Triage \& Policy}

\subsection*{Alert Handling Flow}
\begin{enumerate}
  \item \textbf{Open alert} $\rightarrow$ assess severity, exploitability, and reachability.
  \item \textbf{Decide}: Fix now (\emph{patch/PR}), mitigate (feature flag, config), or temporarily suppress (with justification \& expiry).
  \item \textbf{Track}: Link to issue/PR, assign owner, set SLA (e.g., Critical 24--48h, High 5d, Medium 10d, Low 30d).
  \item \textbf{Verify}: Ensure status checks pass; close alert with reference to commit or PR.
\end{enumerate}

\subsection*{Dismissal Reasons (Use Sparingly)}
\begin{tabularx}{\linewidth}{@{}>{\raggedright\arraybackslash}p{0.25\linewidth}X@{}}
\toprule
\textbf{Reason} & \textbf{When Appropriate}\\
\midrule
False positive & Tool finding demonstrably incorrect for this code path.\\
Used in tests & Credential/pattern is non-production \& gated to test fixtures.\\
Mitigated & Compensating control prevents exploitation (documented).\\
Won't fix (temporary) & Accepted risk with deadline, owner, and business justification.\\
\bottomrule
\end{tabularx}

% -------------------- Mapping: AppSec to GHAS --------------------
\section{AppSec Core Processes $\leftrightarrow$ GHAS Features}

\begin{tabularx}{\linewidth}{@{}>{\raggedright\arraybackslash}p{0.32\linewidth}X@{}}
\toprule
\textbf{AppSec Process} & \textbf{GHAS Tie-in}\\
\midrule
Secure SDLC Governance & Enforce branch protections; require checks from CodeQL \& secret scanning before merge.\\
Threat Detection/Prevention & Secret scanning with push protection; Dependabot advisories.\\
Static Analysis (SAST) & CodeQL workflows on push, PR, and schedule.\\
Dependency/Container Risk & Dependabot PRs; surfaced in Security Overview.\\
Triage \& Remediation & Use alert pages, issues/PR links, and dashboards to track closure \& SLA.\\
\bottomrule
\end{tabularx}

% -------------------- Operational Tips --------------------
\section{Operational Tips}

\begin{itemize}
  \item \textbf{Auto-on for new repos}: set org policy to inherit security features by default.
  \item \textbf{Language coverage}: verify CodeQL supports your primary languages; supplement with third-party linters where needed.
  \item \textbf{Monorepos}: scope CodeQL with \texttt{paths}/\texttt{paths-ignore} to focus on critical dirs.
  \item \textbf{CI reliability}: pin actions to major versions and use weekly scheduled runs to catch drift.
  \item \textbf{Reporting}: use Security Overview exports and Issues/Projects for tracking KPIs (MTTR, \% fixed by severity, coverage).
\end{itemize}

\begin{callout}[Merge Gates Checklist]
\begin{multicols}{2}
\begin{itemize}
  \item Code scanning checks \checky
  \item Secret scanning (no blocking secrets) \checky
  \item Dependency alerts addressed \checky
  \item Required reviewers approved \checky
  \item Status checks passing \checky
\end{itemize}
\columnbreak
\begin{itemize}
  \item Protected branch rules enforced \checky
  \item PR links to issue/ticket \checky
  \item CI passed (build, test) \checky
  \item Release notes updated \checky
  \item Ownership \& SLA assigned \checky
\end{itemize}
\end{multicols}
\end{callout}

\vfill
\hrule
\vspace{2pt}
{\footnotesize
This one-pager is a concise reference for enabling and operating GHAS across Enterprise $\rightarrow$ Org $\rightarrow$ Repo scopes. Adapt the examples to your repositories and compliance policies.
}

\end{document}