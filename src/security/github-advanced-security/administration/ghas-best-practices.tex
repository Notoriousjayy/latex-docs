\documentclass[11pt]{article}

% ---------- Encoding & layout ----------
\usepackage[T1]{fontenc}
\usepackage[utf8]{inputenc}
\usepackage[a4paper,margin=1in]{geometry}
\usepackage{microtype}
\usepackage{parskip}

% ---------- Colors, links, headings ----------
\usepackage[dvipsnames]{xcolor}
\usepackage{hyperref}
\hypersetup{
  colorlinks=true,
  linkcolor=MidnightBlue,
  urlcolor=MidnightBlue,
  citecolor=MidnightBlue
}
\usepackage{titlesec}
\titleformat{\section}{\large\bfseries\color{MidnightBlue}}{}{0pt}{}
\titleformat{\subsection}{\normalsize\bfseries}{}{0pt}{}

% ---------- Lists ----------
\usepackage{enumitem}
\setlist{itemsep=2pt, topsep=4pt, leftmargin=1.2em}

% ---------- Code (minted) ----------
% --- minted (CI-safe fallback) ----------------------------------------
\newif\ifciwithshellescape
\ifdefined\pdfshellescape
  \ifnum\pdfshellescape=1\relax
    \ciwithshellescapetrue
  \else
    \ciwithshellescapefalse
  \fi
\else
  \ciwithshellescapefalse
\fi

\ifciwithshellescape
  \usepackage[newfloat]{minted}
\else
  % Fallback: compile without -shell-escape / Pygments.
  % This preserves document build determinism in CI (no syntax highlighting).
  \usepackage{listings}
  \usepackage{xcolor}
  \usepackage{textcomp}

  \lstset{
    basicstyle=\ttfamily\small,
    columns=fullflexible,
    keepspaces=true,
    breaklines=true,
    breakatwhitespace=true,
    upquote=true,
    showstringspaces=false,
    literate=
      {•}{{\textbullet}}1
      {–}{{--}}1
      {—}{{---}}1
      {…}{{\ldots}}1
      {“}{{``}}1
      {”}{{''}}1
      {‘}{{`}}1
      {’}{{'}}1
  }

  % minted compatibility shims (options and language are ignored)
  \lstnewenvironment{minted}[2][]{\lstset{}}{}
  \newcommand{\inputminted}[3][]{\lstinputlisting{#3}}
  \newcommand{\mintinline}[2]{\texttt{#2}}
  \providecommand{\setminted}[2][]{}
  \providecommand{\setmintedinline}[2][]{}
  \providecommand{\usemintedstyle}[1]{}
  \providecommand{\newminted}[2][]{}
  \providecommand{\newmintedfile}[2][]{}
  \providecommand{\SetupFloatingEnvironment}[2][]{}

  % Provide a 'listing' float if the document expects it (minted[newfloat]).
  \usepackage{float}
  \makeatletter
  \@ifundefined{c@listing}{%
    \newfloat{listing}{tbp}{lol}
    \floatname{listing}{Listing}
  }{}
  \makeatother
  \providecommand{\listoflistings}{\listof{listing}{List of Listings}}
\fi
% ----------------------------------------------------------------------

\setminted{cache=false,
  fontsize=\footnotesize,
  breaklines=true,
  linenos=false,
  tabsize=2
}
% Define a titled code float environment
\SetupFloatingEnvironment{listing}{name=Code}

% ---------- Title ----------
\title{\textbf{GitHub Advanced Security (GHAS) Best Practices}\\[-2pt]\large Quick Reference for Engineering Teams}
\author{}
\date{}

\begin{document}
\maketitle
\vspace{-1.2em}
\noindent\textit{Purpose.} Practical, auditable practices to get sustained value from GHAS across repos and teams.

\section{Speak the language of vulnerabilities}
\begin{itemize}
  \item \textbf{CVE} = specific known vulnerability; \textbf{CWE} = broader weakness class; \textbf{CVSS} informs urgency.
  \item Link GHAS alerts to CVE/CWE, read remediation notes, act by severity and exploitability.
\end{itemize}

\section{Turn on the full GHAS stack (right-sized)}
\begin{itemize}
  \item Enable and tune \textbf{Dependabot}, \textbf{Secret scanning}, and \textbf{CodeQL}.
  \item Prefer repo-specific config files over defaults to reduce noise.
\end{itemize}

\section{Close vs.\ Dismiss: decision guardrails}
\begin{itemize}
  \item \textbf{Close} when fixed and scanners rerun clean; document ``what changed'' for auditability.
  \item \textbf{Dismiss} (with reason) only for accepted risk, false positives, or unreachable code paths; review periodically.
\end{itemize}

\section{Clear roles and collaboration loop}
\begin{itemize}
  \item \textbf{Developers:} secure coding, unit tests, resolve code scanning alerts, collaborate on threat modeling.
  \item \textbf{Security:} define policy, maintain rules, run audits/pen-tests, coach developers.
\end{itemize}

\section{Cadence that matches risk}
\begin{itemize}
  \item High-risk or fast-moving apps: \textbf{weekly} reviews; lower-risk apps: \textbf{monthly}.
  \item Adapt cadence for new threats, major releases, and exposure changes.
\end{itemize}

\section{Put policy in the repo (and surface it)}
\begin{itemize}
  \item Add \texttt{SECURITY.md} (root or \texttt{.github/}) describing requirements and reporting process.
  \item Reinforce via short enablement sessions or ``lunch and learns.''
\end{itemize}

\begin{listing}[h]
\caption{\texttt{SECURITY.md} starter skeleton}
\begin{minted}{text}
# Security Policy

## Reporting a Vulnerability
Please email security@[yourorg].com with details and steps to reproduce.
Do not create public issues for suspected vulnerabilities.

## Supported Versions
- main: actively maintained
- previous: critical fixes only

## Requirements
- 2FA required for committers
- All PRs: CodeQL + secrets + dependency checks must pass

## Disclosure
We follow coordinated disclosure. We acknowledge within 48h and provide
status updates until resolution.
\end{minted}
\end{listing}

\section{Make scanning policy-driven}
\begin{itemize}
  \item Align \textbf{CodeQL} queries and thresholds to written policy; keep a lean baseline and add targeted custom rules.
  \item Treat alerts as actionable work items with owners and SLAs, not informational noise.
\end{itemize}

\section{Gate merges with branch protection}
\begin{itemize}
  \item Protect \texttt{main} and other critical branches:
  \begin{itemize}
    \item Require PRs and at least one approval.
    \item Require status checks to pass: \texttt{codeql}, \texttt{dependabot}, \texttt{secret-scanning}.
    \item Dismiss stale reviews on new commits.
    \item Restrict who can push; block force-pushes and branch deletion.
  \end{itemize}
\end{itemize}

\section{Notifications that matter}
\begin{itemize}
  \item Route \textbf{critical} security events to the right responders immediately (avoid over-alerting).
\end{itemize}

\section{Stay plugged into the community}
\begin{itemize}
  \item Track advisories and research from sources like the GitHub Security Lab; fold insights into rules and runbooks.
\end{itemize}

\vfill
\noindent\footnotesize\textit{Build note:} This document uses the \texttt{minted} package. Compile with: \texttt{pdflatex -shell-escape ghas-best-practices.tex} (or via \texttt{latexmk -pdf -shell-escape}).

\end{document}

