% !TEX TS-program = pdflatex
\documentclass[11pt,a4paper]{article}

% -------------------- Packages --------------------
\usepackage[T1]{fontenc}
\usepackage{lmodern}
\usepackage{inconsolata}
\usepackage{upquote}
\usepackage{microtype}
\usepackage[margin=1in]{geometry}
\usepackage{parskip}
\usepackage[hyphens]{url}
\usepackage{hyperref}
\usepackage{bookmark}
\usepackage{enumitem}
\usepackage[dvipsnames]{xcolor}
\usepackage{booktabs}
\usepackage{array}
\usepackage{ragged2e}
\usepackage[most]{tcolorbox}
\usepackage{amsmath,amssymb}
\usepackage{titlesec}
\usepackage{graphicx}
\usepackage{listings}

% -------------------- Readability Tweaks --------------------
\linespread{1.03}
\setlength{\parindent}{0pt}
\setlength{\parskip}{0.55em}
\setlength{\emergencystretch}{2em}
\renewcommand{\arraystretch}{1.12}
\raggedbottom
\clubpenalty=10000
\widowpenalty=10000
\displaywidowpenalty=10000
\titlespacing*{\section}{0pt}{0.9em}{0.35em}
\titlespacing*{\subsection}{0pt}{0.75em}{0.25em}
\setlist{leftmargin=*,itemsep=2pt,topsep=4pt}
\setlist[itemize]{itemsep=2pt}
\setlist[enumerate]{itemsep=2pt}

% -------------------- Colors --------------------
\definecolor{Primary}{HTML}{0E7490}
\definecolor{Accent}{HTML}{0EA5E9}
\definecolor{Soft}{HTML}{F1F5F9}
\definecolor{Ink}{HTML}{0F172A}
\definecolor{Meta}{HTML}{475569}
\definecolor{OK}{HTML}{16A34A}
\definecolor{Warn}{HTML}{EA580C}
\definecolor{Bad}{HTML}{DC2626}

\hypersetup{
  colorlinks=true,
  linkcolor=Primary,
  urlcolor=Primary,
  citecolor=Primary,
  breaklinks=true,
  pdfauthor={},
  pdftitle={User Story Template & Guide}
}
\urlstyle{same}
\titleformat{\section}{\large\bfseries\color{Ink}}{\thesection}{0.6em}{}
\titleformat{\subsection}{\normalsize\bfseries\color{Ink}}{\thesubsection}{0.6em}{}

\newcommand{\checkbox}{\(\square\)}
\newcommand{\checkedbox}{\(\blacksquare\)}
\newcommand{\eg}{e.g.\ }
\newcommand{\ie}{i.e.\ }

% -------------------- Gherkin (listings) --------------------
\lstdefinelanguage{Gherkin}{
  morekeywords={Feature,Background,Scenario,Scenario\ Outline,Examples,Given,When,Then,And,But},
  sensitive=true,
}
\lstset{
  language=Gherkin,
  basicstyle=\ttfamily\small,
  keywordstyle=\color{Primary}\bfseries,
  commentstyle=\itshape\color{Meta},
  showstringspaces=false,
  frame=single,
  framerule=0.4pt,
  rulecolor=\color{Soft},
  backgroundcolor=\color{Soft},
  tabsize=2,
  columns=fullflexible,
  keepspaces=true,
  breaklines=true,
  breakatwhitespace=true,
  xleftmargin=1ex,
  framexleftmargin=1ex,
  framesep=0.6ex,
  aboveskip=3pt,
  belowskip=6pt
}

% -------------------- Card Look & Feel (matches screenshot) --------------------
\tcbset{
  colback=gray!2,
  colframe=gray!50,
  colbacktitle=gray!6,
  coltitle=black,
  fonttitle=\bfseries\large,
  arc=2pt,
  boxrule=0.4pt,
  left=8pt,right=8pt,top=8pt,bottom=8pt,
  enhanced,
  breakable,
  borderline west={2pt}{0pt}{MidnightBlue}
}

% Badges/pills
\newtcbox{\pill}{on line, arc=3pt, boxsep=0.8pt, left=4pt,right=4pt,top=1pt,bottom=1pt,
  colframe=gray!50, colback=gray!15, boxrule=0.3pt}
\newcommand{\badge}[1]{\pill{\footnotesize #1}}

% Footer helpers
\newcommand{\DoR}{\textbf{Definition of Ready:} Persona clear; AC drafted; Dependencies known; Estimate set.}
\newcommand{\DoD}{\textbf{Definition of Done:} All ACs pass; Tests green; Security/a11y checks; Docs updated; Deployed/flagged.}
\let\cb\checkbox

% Widths for robust tables (no tabularx needed)
\newlength{\StoryLabelW}
\setlength{\StoryLabelW}{3.2cm}
\newlength{\StoryValueW}
\setlength{\StoryValueW}{\dimexpr\linewidth-\StoryLabelW-2\tabcolsep\relax}

% -------------------- Story Card macro (exact screenshot layout) --------------------
% 1: ID   2: Title   3: Epic/Feature   4: Business Value
% 5: Priority   6: Estimate(SP)   7: Persona   8: Dependencies   9: Assumptions/Risks
\newcommand{\StoryCard}[9]{%
  \newpage
  \begin{tcolorbox}[title={\textbf{#1}\ \textemdash\ #2}]
  \small
  \begin{tabular}{@{}>{\raggedleft\arraybackslash\bfseries}p{\StoryLabelW} >{\RaggedRight\arraybackslash}p{\StoryValueW}@{}}
    Epic / Feature          & #3 \\
    Business Value          & #4 \\
    Priority / Estimate     & \badge{Priority: #5}\ \badge{SP: #6} \\
    Persona                 & #7 \\
    Dependencies            & #8 \\
    Assumptions / Risks     & #9 \\
  \end{tabular}

  \medskip
  \textbf{Story}\quad
  \emph{As a #7, I want to #2 so that #4.}

  \medskip
  \textbf{Non-Functional}\quad
  \badge{Performance}\ \badge{Security}\ \badge{Reliability}\ \badge{Accessibility}\ \badge{Privacy}\ \badge{i18n}

  \medskip
  \textbf{Acceptance Criteria (BDD)}
  \begin{description}[leftmargin=2.4cm, labelwidth=2.3cm, style=nextline, itemsep=2pt, topsep=2pt]
    \item[\textbf{Scenario}] Happy path
    \item[\textbf{Given}] the target repository and pipeline configuration are available
    \item[\textbf{When}] the user completes the \emph{Hands-on Objective}
    \item[\textbf{Then}] the stated \emph{Outcome} is observable and recorded in the pipeline/job summary
  \end{description}

  \vspace{0.2\baselineskip}
  {\footnotesize\color{gray!60}\DoR\ \textbullet\ \DoD}
  \end{tcolorbox}
}

% -------------------- Tasks box (matches screenshot style) --------------------
\newenvironment{TasksBox}[1][Tasks]{%
  \begin{tcolorbox}[
    enhanced,breakable,
    colback=gray!1, colframe=gray!35,
    colbacktitle=gray!6, coltitle=black,
    title={#1}, fonttitle=\bfseries,
    borderline west={1.8pt}{0pt}{MidnightBlue},
    arc=2pt, boxrule=0.4pt,
    left=10pt,right=10pt,top=6pt,bottom=6pt,
    before skip=6pt, after skip=10pt
  ]
  \small
  \begin{itemize}[label=\cb, leftmargin=*, labelsep=0.6em, itemsep=4pt, topsep=2pt, parsep=0pt]
}{%
  \end{itemize}
  \end{tcolorbox}
}

% -------------------- Document --------------------
\begin{document}
\begin{center}
{\huge \textbf{GHAS Application Security User Stories}}\\[2pt]
\textcolor{Meta}{Operational stories for Code Scanning and Secret Scanning implementation and adoption}\\[6pt]
\end{center}
\noindent\textbf{Context}\\
You are implementing application security using GitHub Advanced Security (GHAS). Secret Scanning is already enabled. Code Scanning has recently begun blocking pull requests when critical alerts are present. Your day-to-day work includes clearing Code Scanning and Secret Scanning queues and engaging IT managers with remediation guidance.


\noindent\textbf{Objectives}
\begin{itemize}
  \item Reduce open queues (alerts) to an actionable steady state.
  \item Maintain enforcement on critical Code Scanning findings without disrupting delivery.
  \item Improve remediation throughput through structured outreach and guidance.
  \item Establish governance, SLAs, and reporting to sustain the program.
\end{itemize}

\section{User Stories}
\noindent The following stories are organized by epic and can be loaded into your backlog as-is. Adjust priorities, estimates, and scope to match your organizational constraints.


\subsection*{Epic: Code Scanning Enforcement and Operations}

\StoryCard{GHAS-CS-001}{Standardize critical Code Scanning enforcement on protected branches}{Code Scanning Enforcement}
{Prevent critical vulnerabilities from being merged by requiring Code Scanning to pass on pull requests}
{Must}{5}
{AppSec engineer}
{Branch protection / rulesets; CodeQL workflow; repository admin support}
{Noise from legacy alerts; workflow misconfiguration causing false blocks; remediation capacity constraints}

\begin{TasksBox}[Implementation Tasks]
  \item \cb Inventory target repositories and current Code Scanning workflow state (enabled, languages, default query suites).
  \item \cb Ensure CodeQL analysis runs on pull requests and uploads SARIF successfully (no upload failures).
  \item \cb Configure protected branches to require the Code Scanning check and block merges on “Critical” alerts.
  \item \cb Validate the merge gate with a controlled test PR that introduces and then removes a critical finding.
  \item \cb Publish a short developer guidance note describing what to do when a PR is blocked (triage, fix, request help).
\end{TasksBox}

\textbf{Acceptance Criteria (Gherkin)}
\begin{lstlisting}[language=Gherkin]
Scenario: PR is blocked when a critical Code Scanning alert exists
  Given a repository has Code Scanning enabled for pull requests
  And the protected branch requires the Code Scanning status check
  When a pull request introduces a critical vulnerability
  Then the Code Scanning check fails
  And the pull request cannot be merged until the critical alert is resolved or formally waived

Scenario: PR is mergeable when no critical Code Scanning alerts exist
  Given Code Scanning completes successfully on the pull request
  When there are no critical alerts associated with the pull request diff
  Then the Code Scanning check passes
  And the pull request is eligible to merge (subject to other required checks)
\end{lstlisting}

\clearpage

\StoryCard{GHAS-CS-002}{Baseline and prioritize the Code Scanning backlog for triage}{Code Scanning Operations}
{Create a clear, risk-based order of work so remediation focuses on the highest-impact vulnerabilities first}
{Must}{3}
{AppSec engineer}
{Security dashboard access; repository ownership data; criticality criteria}
{Unknown ownership for legacy repos; duplicated alerts; inconsistent severity mapping}

\begin{TasksBox}[Implementation Tasks]
  \item \cb Export or query open Code Scanning alerts and group by repository, severity, and age.
  \item \cb Define triage buckets (Critical immediate, High within SLA, Medium/Low scheduled) and document SLAs.
  \item \cb Identify top repositories by count of critical/high alerts and by alert age.
  \item \cb Assign an owner for each top repository and open a remediation work item (issue/ticket) per repository.
  \item \cb Set a weekly cadence to refresh backlog metrics and adjust priorities based on new findings.
\end{TasksBox}

\textbf{Acceptance Criteria (Gherkin)}
\begin{lstlisting}[language=Gherkin]
Scenario: Backlog is categorized into actionable triage buckets
  Given open Code Scanning alerts exist across multiple repositories
  When the AppSec engineer runs the backlog baseline process
  Then each alert is assigned a triage bucket based on severity and age
  And each repository has an identified remediation owner for critical and high items
\end{lstlisting}

\clearpage

\StoryCard{GHAS-CS-003}{Reduce Code Scanning queue noise by dismissing false positives with documented rationale}{Code Scanning Operations}
{Improve signal-to-noise so developers focus on actionable findings and PR blocking is reserved for true risk}
{Should}{3}
{AppSec engineer}
{Agreed false-positive policy; access to alert dismissal reasons; developer SMEs for validation}
{Over-dismissal creates blind spots; inconsistent rationale; future regressions if rules are not tuned}

\begin{TasksBox}[Implementation Tasks]
  \item \cb Review recurring alert types that are commonly non-exploitable in your codebase (e.g., test-only paths).
  \item \cb Validate candidates with repo maintainers before dismissal for “False positive” or “Used in tests” reasons.
  \item \cb Dismiss qualifying alerts and include a consistent rationale comment (why not exploitable, scope, references).
  \item \cb Create a short tuning backlog for query suite adjustments or suppression patterns, where appropriate.
  \item \cb Measure improvement by tracking reduction in reopened alerts and decline in repeated false positives.
\end{TasksBox}

\textbf{Acceptance Criteria (Gherkin)}
\begin{lstlisting}[language=Gherkin]
Scenario: False positives are dismissed consistently and auditable
  Given an alert is determined to be non-actionable based on documented criteria
  When the AppSec engineer dismisses the alert
  Then the dismissal includes a standard reason and supporting rationale
  And the alert no longer appears in the open queue metrics
\end{lstlisting}

\clearpage

\StoryCard{GHAS-CS-004}{Provide a rapid-response support lane for PRs blocked by critical Code Scanning alerts}{Code Scanning Enforcement}
{Minimize delivery disruption while maintaining enforcement by offering timely triage and remediation guidance}
{Must}{5}
{Developer}
{On-call or rotation for AppSec support; communication channel (Teams/Slack/email); escalation policy}
{Support capacity may be exceeded; pressure to waive without evidence; inconsistent guidance}

\begin{TasksBox}[Implementation Tasks]
  \item \cb Define an intake path for blocked PRs (template with repo, PR link, alert link, timeline, requester).
  \item \cb Create a standard triage checklist (is it new vs pre-existing, exploitability, reachable code, data flow).
  \item \cb Provide remediation options: code change, safe API, sanitization, guard conditions, or design adjustment.
  \item \cb If a waiver is required, route to the exception process with explicit scope and expiry.
  \item \cb Close the loop by confirming the Code Scanning check passes after remediation or approved waiver.
\end{TasksBox}

\textbf{Acceptance Criteria (Gherkin)}
\begin{lstlisting}[language=Gherkin]
Scenario: A blocked PR receives triage within the support SLA
  Given a pull request is blocked due to a critical Code Scanning alert
  When the developer submits a support request with the required details
  Then AppSec provides an initial triage response within the published SLA
  And the developer receives clear remediation guidance or a documented path to exception handling
\end{lstlisting}

\clearpage


\subsection*{Epic: Governance and Policy}

\StoryCard{GHAS-CS-005}{Establish a time-bounded exception process for critical Code Scanning merge blocks}{Governance and Policy}
{Enable controlled risk acceptance when necessary while preserving auditability and preventing permanent bypasses}
{Must}{3}
{Engineering manager}
{Risk acceptance criteria; approvers list; tracking mechanism (issue/ticket) with expiry}
{Exceptions become routine; missing expiry enforcement; incomplete documentation}

\begin{TasksBox}[Implementation Tasks]
  \item \cb Define eligibility criteria for exceptions (business urgency, compensating controls, mitigation plan).
  \item \cb Create an exception request template capturing scope, justification, affected repos, and expiry date.
  \item \cb Define approval workflow (AppSec + engineering leadership; optionally GRC/compliance when required).
  \item \cb Implement tracking and reminders for expiry and re-validation, including re-scanning after fixes land.
  \item \cb Publish guidance so teams understand exceptions are temporary and require a remediation plan.
\end{TasksBox}

\textbf{Acceptance Criteria (Gherkin)}
\begin{lstlisting}[language=Gherkin]
Scenario: Exception is granted with expiry and remediation plan
  Given an engineering team requests an exception for a blocked PR
  When the request includes justification, scope, and an expiry date
  And the required approvers approve the request
  Then the PR may proceed under the exception
  And the exception is tracked with an expiry and a remediation plan
\end{lstlisting}

\clearpage

\StoryCard{GHAS-GOV-002}{Define and enforce SLAs for Code Scanning and Secret Scanning remediation}{Governance and Policy}
{Ensure consistent risk reduction by setting clear timelines for addressing critical security findings and exposed secrets}
{Must}{3}
{Security program manager}
{Leadership agreement; exception process; tracking and escalation mechanism}
{SLA is not feasible for all teams; lack of enforcement undermines program; excessive exceptions}

\begin{TasksBox}[Implementation Tasks]
  \item \cb Draft SLA targets (e.g., Critical: 7 days; High: 30 days; secrets: rotate within 24 hours) aligned to risk appetite.
  \item \cb Socialize SLAs with IT managers and engineering leadership and refine based on operational realities.
  \item \cb Implement escalation steps for SLA breaches (manager notification, weekly review, leadership escalation).
  \item \cb Tie SLAs to the reporting metrics so breaches are visible and actionable.
\end{TasksBox}

\textbf{Acceptance Criteria (Gherkin)}
\begin{lstlisting}[language=Gherkin]
Scenario: SLA breaches are identified and escalated
  Given SLAs are defined for each severity class and for secrets rotation
  When an alert exceeds its SLA without an approved exception
  Then the owning team and manager are notified
  And the item is escalated according to the published process
\end{lstlisting}

\clearpage


\subsection*{Epic: Secret Scanning Operations and Program}

\StoryCard{GHAS-SS-001}{Triage and clear the Secret Scanning alert queue with a rotation-first workflow}{Secret Scanning Operations}
{Reduce likelihood of credential abuse by ensuring exposed secrets are quickly revoked or rotated and alerts are closed}
{Must}{5}
{Security analyst}
{Credential owner identification; rotation procedures for key providers; incident response guidance}
{Owners are unclear; rotation procedures differ by provider; false positives consume time}

\begin{TasksBox}[Implementation Tasks]
  \item \cb Review incoming Secret Scanning alerts and verify whether each finding is a real secret (not a test token).
  \item \cb Identify the owning team and system for each secret and initiate rotation or revocation immediately.
  \item \cb Validate rotation by confirming the old secret is no longer accepted and the new secret is deployed.
  \item \cb Close the alert with documented remediation steps and links to evidence (PR, ticket, change record).
  \item \cb Track mean time to revoke/rotate and prioritize by exposure risk and repository visibility.
\end{TasksBox}

\textbf{Acceptance Criteria (Gherkin)}
\begin{lstlisting}[language=Gherkin]
Scenario: A valid secret is rotated and the alert is closed with evidence
  Given Secret Scanning detects a valid credential in a repository
  When the credential owner rotates or revokes the exposed secret
  Then the exposed secret is no longer valid for authentication
  And the Secret Scanning alert is closed with linked evidence of remediation
\end{lstlisting}

\clearpage

\StoryCard{GHAS-SS-002}{Improve Secret Scanning signal by standardizing dismissal rules and documenting false positives}{Secret Scanning Operations}
{Maintain an actionable queue by ensuring non-secrets are dismissed consistently and recurring false positives are addressed}
{Should}{3}
{AppSec engineer}
{Dismissal reason taxonomy; agreement on what qualifies as test data; ability to tune patterns}
{Overly aggressive dismissals; continued noisy patterns if not tuned; inconsistent documentation}

\begin{TasksBox}[Implementation Tasks]
  \item \cb Define criteria for “Not a secret” vs “Test credential” vs “Won't fix” for Secret Scanning alerts.
  \item \cb Dismiss non-secrets with consistent rationale comments and reference examples.
  \item \cb Capture recurring patterns that trigger false positives and open a tuning item (custom patterns or allowlists).
  \item \cb Review dismissals periodically to ensure the policy is followed and does not hide real secrets.
\end{TasksBox}

\textbf{Acceptance Criteria (Gherkin)}
\begin{lstlisting}[language=Gherkin]
Scenario: Non-secrets are dismissed consistently
  Given a Secret Scanning alert is reviewed and determined not to be a real secret
  When the alert is dismissed
  Then the dismissal includes a standard reason and rationale
  And recurring false-positive patterns are recorded for tuning
\end{lstlisting}

\clearpage

\StoryCard{GHAS-SS-003}{Ensure Secret Scanning coverage is enabled and monitored across in-scope repositories}{Secret Scanning Program}
{Avoid blind spots by confirming Secret Scanning is enabled everywhere it is required and that alert routing reaches the right owners}
{Must}{2}
{AppSec engineer}
{Repository inventory; scope definition; org/repo admin access; notification routing}
{Scope creep; repos with unique constraints; misrouted notifications delay response}

\begin{TasksBox}[Implementation Tasks]
  \item \cb Validate Secret Scanning enablement status for all in-scope repositories and confirm alert volume baseline.
  \item \cb Confirm alert notifications reach repository owners and the AppSec/SecOps mailbox or channel.
  \item \cb Document the expected response path (owner rotation steps, AppSec escalation, incident criteria).
  \item \cb Create an audit-friendly record of coverage (what is enabled where, and how owners are identified).
\end{TasksBox}

\textbf{Acceptance Criteria (Gherkin)}
\begin{lstlisting}[language=Gherkin]
Scenario: All in-scope repositories have Secret Scanning enabled
  Given an inventory of repositories within program scope
  When the AppSec engineer reviews Secret Scanning settings
  Then Secret Scanning is enabled for each in-scope repository
  And a monitoring view exists to identify newly non-compliant repositories
\end{lstlisting}

\clearpage


\subsection*{Epic: Stakeholder Engagement and Enablement}

\StoryCard{GHAS-ENG-001}{Engage IT managers for high-risk repositories with a structured remediation offer}{Stakeholder Engagement}
{Accelerate remediation by aligning managers on priorities, timelines, and support, reducing friction for engineering teams}
{Must}{3}
{IT manager}
{Accurate ownership mapping; prioritized backlog list; standard outreach materials; meeting scheduling}
{Manager availability; competing priorities; unclear responsibilities between teams}

\begin{TasksBox}[Implementation Tasks]
  \item \cb Identify the top repositories by critical/high Code Scanning alerts and active Secret Scanning findings.
  \item \cb Prepare a repo-specific brief: counts by severity, oldest items, recent PR blocks, and recommended next steps.
  \item \cb Contact the responsible IT manager with an offer for remediation support and a proposed working session.
  \item \cb During the session, agree on an actionable remediation plan (owners, dates, and escalation paths).
  \item \cb Track commitments and follow up weekly until critical items meet the agreed SLA.
\end{TasksBox}

\textbf{Acceptance Criteria (Gherkin)}
\begin{lstlisting}[language=Gherkin]
Scenario: Remediation plan is agreed with an IT manager for a high-risk repository
  Given a repository is identified as high risk based on alert severity and age
  When the AppSec engineer engages the responsible IT manager
  Then a remediation plan is documented with owners and dates
  And progress is tracked until critical risk is reduced to the target threshold
\end{lstlisting}

\clearpage

\StoryCard{GHAS-ENG-002}{Publish developer-facing remediation guidance for Code Scanning and Secret Scanning}{Enablement and Guidance}
{Reduce time-to-fix by giving developers clear, consistent instructions and examples for common findings}
{Must}{3}
{Developer}
{Internal wiki or repo docs; example fixes; SME review; approved policy language}
{Guidance becomes outdated; too generic to be useful; gaps for less common stacks/languages}

\begin{TasksBox}[Implementation Tasks]
  \item \cb Create a concise playbook for Code Scanning: reading findings, mapping to code, common fix patterns, and verification steps.
  \item \cb Create a concise playbook for Secret Scanning: rotation steps, containment, validation, and closure evidence.
  \item \cb Include a “What to do when blocked” section for critical PR gating, including how to request AppSec support.
  \item \cb Publish and socialize guidance with engineering leads and IT managers; update onboarding documentation.
\end{TasksBox}

\textbf{Acceptance Criteria (Gherkin)}
\begin{lstlisting}[language=Gherkin]
Scenario: Developers can follow published guidance to unblock a PR
  Given a pull request is blocked by a critical Code Scanning alert
  When the developer follows the remediation guidance
  Then the developer can identify the affected code path and apply a recommended fix
  And the Code Scanning check passes after the fix is merged into the pull request branch
\end{lstlisting}

\clearpage


\subsection*{Epic: Metrics and Reporting}

\StoryCard{GHAS-GOV-001}{Implement weekly reporting on Code Scanning and Secret Scanning queue health}{Metrics and Reporting}
{Provide transparency and accountability by tracking alert volumes, remediation progress, and SLA adherence}
{Should}{2}
{Security program manager}
{Access to Security Overview or APIs; defined KPIs; distribution list or dashboard location}
{Data inconsistencies across repos; manual reporting does not scale; metric gaming}

\begin{TasksBox}[Implementation Tasks]
  \item \cb Define core KPIs: open critical/high alerts, alert age percentiles, mean time to close, PR blocks count, secrets time-to-rotate.
  \item \cb Generate a weekly report segmented by team/repo with trends and top outliers.
  \item \cb Include a short narrative: what improved, what regressed, and where support is needed.
  \item \cb Share the report with IT managers and engineering leadership; capture action items for outliers.
\end{TasksBox}

\textbf{Acceptance Criteria (Gherkin)}
\begin{lstlisting}[language=Gherkin]
Scenario: Weekly queue health report is produced and shared
  Given Code Scanning and Secret Scanning alerts exist across repositories
  When the reporting job runs on the weekly cadence
  Then a report is generated with the defined KPIs and trends
  And the report is distributed to the agreed stakeholders
\end{lstlisting}

\clearpage

\end{document}
