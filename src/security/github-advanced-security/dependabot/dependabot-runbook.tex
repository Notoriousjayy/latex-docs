\documentclass[11pt,letterpaper]{article}

% ============================================================================
% PACKAGES
% ============================================================================
\usepackage[utf8]{inputenc}
\usepackage[T1]{fontenc}
\usepackage{lmodern}
\usepackage[margin=1in]{geometry}
\usepackage{graphicx}
\usepackage{xcolor}
\usepackage{hyperref}
\usepackage{listings}
\usepackage{booktabs}
\usepackage{longtable}
\usepackage{array}
\usepackage{tabularx}
\usepackage{enumitem}
\usepackage{fancyhdr}
\usepackage{titlesec}
\usepackage{tocloft}
\usepackage{parskip}
\usepackage{microtype}
\usepackage{fontawesome5}
\usepackage{tcolorbox}
\usepackage{mdframed}
\usepackage{amssymb}

% ============================================================================
% COLOR DEFINITIONS
% ============================================================================
\definecolor{anthracite}{RGB}{45, 52, 54}
\definecolor{deepblue}{RGB}{36, 113, 163}
\definecolor{codegreen}{RGB}{40, 167, 69}
\definecolor{codegray}{RGB}{108, 117, 125}
\definecolor{codepurple}{RGB}{136, 84, 208}
\definecolor{backcolour}{RGB}{248, 249, 250}
\definecolor{warningbg}{RGB}{255, 243, 205}
\definecolor{warningborder}{RGB}{255, 193, 7}
\definecolor{infobg}{RGB}{209, 236, 241}
\definecolor{infoborder}{RGB}{23, 162, 184}
\definecolor{tipbg}{RGB}{212, 237, 218}
\definecolor{tipborder}{RGB}{40, 167, 69}
\definecolor{dangerousbg}{RGB}{248, 215, 218}
\definecolor{dangerousborder}{RGB}{220, 53, 69}
\definecolor{linkblue}{RGB}{0, 102, 204}

% ============================================================================
% HYPERREF CONFIGURATION
% ============================================================================
\hypersetup{
    colorlinks=true,
    linkcolor=deepblue,
    filecolor=deepblue,
    urlcolor=linkblue,
    citecolor=deepblue,
    pdftitle={Runbook: Dependabot Security Update PR Customization},
    pdfauthor={Application Security Team},
    pdfsubject={GitHub Security Automation},
    pdfkeywords={Dependabot, GitHub, Security, CODEOWNERS, Microsoft Teams, CI/CD},
    bookmarks=true,
    bookmarksopen=true
}

% ============================================================================
% LISTINGS CONFIGURATION
% ============================================================================
\lstdefinestyle{yamlstyle}{
    backgroundcolor=\color{backcolour},
    commentstyle=\color{codegray}\itshape,
    keywordstyle=\color{deepblue}\bfseries,
    stringstyle=\color{codegreen},
    basicstyle=\ttfamily\small,
    breakatwhitespace=false,
    breaklines=true,
    captionpos=b,
    keepspaces=true,
    numbers=left,
    numbersep=8pt,
    numberstyle=\tiny\color{codegray},
    showspaces=false,
    showstringspaces=false,
    showtabs=false,
    tabsize=2,
    frame=single,
    framerule=0.5pt,
    rulecolor=\color{codegray!50},
    xleftmargin=1.5em,
    framexleftmargin=1.5em,
    aboveskip=1em,
    belowskip=1em
}

\lstdefinelanguage{yaml}{
    keywords={true,false,null,y,n},
    sensitive=false,
    comment=[l]{\#},
    morestring=[b]',
    morestring=[b]"
}

\lstdefinestyle{bashstyle}{
    backgroundcolor=\color{anthracite!5},
    commentstyle=\color{codegray}\itshape,
    keywordstyle=\color{codepurple}\bfseries,
    stringstyle=\color{codegreen},
    basicstyle=\ttfamily\small,
    breaklines=true,
    numbers=none,
    frame=single,
    framerule=0.5pt,
    rulecolor=\color{codegray!50},
    xleftmargin=1em,
    framexleftmargin=0.5em,
    aboveskip=1em,
    belowskip=1em
}

\lstdefinestyle{codestyle}{
    backgroundcolor=\color{backcolour},
    commentstyle=\color{codegray}\itshape,
    keywordstyle=\color{deepblue}\bfseries,
    stringstyle=\color{codegreen},
    basicstyle=\ttfamily\small,
    breaklines=true,
    numbers=left,
    numbersep=8pt,
    numberstyle=\tiny\color{codegray},
    frame=single,
    framerule=0.5pt,
    rulecolor=\color{codegray!50},
    xleftmargin=1.5em,
    framexleftmargin=1.5em,
    aboveskip=1em,
    belowskip=1em
}

\lstset{style=codestyle}

% ============================================================================
% CUSTOM ENVIRONMENTS
% ============================================================================
\tcbuselibrary{skins,breakable}

\newtcolorbox{warningbox}{
    colback=warningbg,
    colframe=warningborder,
    boxrule=1pt,
    arc=3pt,
    left=10pt,
    right=10pt,
    top=8pt,
    bottom=8pt,
    breakable,
    title={\faExclamationTriangle\ \ Warning},
    fonttitle=\bfseries\color{anthracite},
    coltitle=anthracite
}

\newtcolorbox{infobox}{
    colback=infobg,
    colframe=infoborder,
    boxrule=1pt,
    arc=3pt,
    left=10pt,
    right=10pt,
    top=8pt,
    bottom=8pt,
    breakable,
    title={\faInfoCircle\ \ Note},
    fonttitle=\bfseries\color{anthracite},
    coltitle=anthracite
}

\newtcolorbox{tipbox}{
    colback=tipbg,
    colframe=tipborder,
    boxrule=1pt,
    arc=3pt,
    left=10pt,
    right=10pt,
    top=8pt,
    bottom=8pt,
    breakable,
    title={\faLightbulb\ \ Best Practice},
    fonttitle=\bfseries\color{anthracite},
    coltitle=anthracite
}

\newtcolorbox{criticalbox}{
    colback=dangerousbg,
    colframe=dangerousborder,
    boxrule=1pt,
    arc=3pt,
    left=10pt,
    right=10pt,
    top=8pt,
    bottom=8pt,
    breakable,
    title={\faExclamationCircle\ \ Critical},
    fonttitle=\bfseries\color{anthracite},
    coltitle=anthracite
}

% ============================================================================
% HEADER AND FOOTER
% ============================================================================
\pagestyle{fancy}
\fancyhf{}
\fancyhead[L]{\small\textit{Dependabot Security Update PR Customization}}
\fancyhead[R]{\small\textit{AppSec Runbook}}
\fancyfoot[C]{\thepage}
\renewcommand{\headrulewidth}{0.4pt}
\renewcommand{\footrulewidth}{0.4pt}

% ============================================================================
% TITLE FORMATTING
% ============================================================================
\titleformat{\section}
    {\Large\bfseries\color{anthracite}}
    {\thesection}{1em}{}
\titleformat{\subsection}
    {\large\bfseries\color{deepblue}}
    {\thesubsection}{1em}{}
\titleformat{\subsubsection}
    {\normalsize\bfseries\color{anthracite}}
    {\thesubsubsection}{1em}{}

% ============================================================================
% TABLE OF CONTENTS FORMATTING
% ============================================================================
\renewcommand{\cftsecfont}{\bfseries\color{anthracite}}
\renewcommand{\cftsubsecfont}{\color{deepblue}}
\renewcommand{\cftsecpagefont}{\bfseries}
\setlength{\cftbeforesecskip}{6pt}

% ============================================================================
% CUSTOM COMMANDS
% ============================================================================
\newcommand{\code}[1]{\texttt{\small\color{codepurple}#1}}
\newcommand{\filepath}[1]{\texttt{\small\color{codegreen}#1}}
\newcommand{\teamname}[1]{\texttt{@#1}}
\newcommand{\reflink}[2]{\href{#1}{#2}}

% ============================================================================
% DOCUMENT BEGIN
% ============================================================================
\begin{document}

% ============================================================================
% TITLE PAGE
% ============================================================================
\begin{titlepage}
    \centering
    \vspace*{2cm}
    
    {\Huge\bfseries\color{anthracite}Runbook:\\[0.3cm]
    Dependabot Security Update\\[0.1cm]
    PR Customization\par}
    
    \vspace{1.5cm}
    
    {\Large\color{deepblue}Auto-Route Security Update PRs to AppSec\\
    with Review Enforcement and Teams Notifications\par}
    
    \vspace{2cm}
    
    \begin{tcolorbox}[
        colback=backcolour,
        colframe=codegray!50,
        width=0.8\textwidth,
        arc=5pt,
        boxrule=1pt
    ]
        \centering
        \textbf{Scope:} GitHub repositories using Dependabot security updates\\
        and PR-based remediation workflows
    \end{tcolorbox}
    
    \vfill
    
    {\large Application Security Team\par}
    \vspace{0.5cm}
    {\large\today\par}
    
    \vspace{1cm}
    
    {\small Version 1.0}
    
\end{titlepage}

% ============================================================================
% TABLE OF CONTENTS
% ============================================================================
\newpage
\tableofcontents
\newpage

% ============================================================================
% SECTION 1: OBJECTIVE AND OUTCOME
% ============================================================================
\section{Objective and Outcome}
\label{sec:objective}

\subsection{Objective}

This runbook ensures that \textbf{Dependabot security update pull requests} are properly managed through three critical control points:

\begin{enumerate}[leftmargin=2em]
    \item \textbf{Consistent Labeling and Assignment:} All security PRs are automatically tagged and assigned for triage and ownership accountability.
    \item \textbf{Automatic AppSec Review Requests:} The Application Security team is explicitly placed in the approval path via CODEOWNERS integration.
    \item \textbf{Real-Time Microsoft Teams Notifications:} Security updates produce immediate visibility in designated Teams channels, ensuring no critical updates are missed.
\end{enumerate}

\subsection{Acceptance Criteria (``Definition of Done'')}

A Dependabot security PR is considered properly configured when it is opened against the \textbf{default branch} and contains:

\begin{itemize}[leftmargin=2em]
    \item \textbf{Required Labels:} Tags for routing, automation triggers, and dashboard filtering (e.g., \code{security}, \code{dependabot}, \code{appsec-triage})
    \item \textbf{Appropriate Assignees:} A human owner who is accountable for driving the PR to completion
    \item \textbf{Requested Review:} AppSec team explicitly requested via CODEOWNERS patterns
\end{itemize}

Additionally, the repository must enforce:

\begin{itemize}[leftmargin=2em]
    \item \textbf{``Require review from Code Owners''} ruleset/branch protection, ensuring AppSec review is mandatory
    \item \textbf{Teams Channel Notifications:} Active subscription via the official GitHub-Teams integration or custom webhook-based Actions
\end{itemize}

\begin{tipbox}
For comprehensive documentation on ruleset configuration, refer to the \href{https://docs.github.com/en/repositories/configuring-branches-and-merges-in-your-repository/managing-rulesets/available-rules-for-rulesets}{GitHub Docs: Available Rules for Rulesets}.
\end{tipbox}

% ============================================================================
% SECTION 2: KEY CONSTRAINTS AND GOTCHAS
% ============================================================================
\section{Key Constraints and Design Considerations}
\label{sec:constraints}

Understanding these constraints is essential before implementing any configuration. Failure to account for these behaviors is the most common source of automation failures.

\subsection{Dependabot Security Updates Behavior}

\begin{criticalbox}
Security updates behave differently from version updates. Understanding these distinctions is critical for correct configuration.
\end{criticalbox}

\begin{longtable}{@{}p{0.35\textwidth}p{0.6\textwidth}@{}}
\toprule
\textbf{Behavior} & \textbf{Implication} \\
\midrule
\endhead
Security updates are triggered by \textbf{security advisories} & They do not follow your configured \code{schedule}. You cannot predict when they will be created. \\
\midrule
Security update PRs target the \textbf{default branch} & If you configure \code{target-branch}, your customization typically applies to \textbf{version updates} by default, not security updates. \\
\midrule
Security-only customization requires \code{open-pull-requests-limit: 0} & This setting disables version updates for the ecosystem, ensuring all customizations apply exclusively to security updates. \\
\bottomrule
\end{longtable}

\subsection{Reviewers Configuration: CODEOWNERS Mandate}

\begin{warningbox}
GitHub \textbf{removed} the Dependabot \code{reviewers} configuration option on \textbf{May 20, 2025}. All reviewer routing must now be implemented through CODEOWNERS files.
\end{warningbox}

This architectural change means:

\begin{itemize}[leftmargin=2em]
    \item The \code{reviewers:} key in \filepath{dependabot.yml} is deprecated and ignored
    \item Review assignment logic must live in \filepath{.github/CODEOWNERS}
    \item Pattern matching determines which teams/users are requested based on files modified
\end{itemize}

For the official announcement, see the \href{https://github.blog/changelog/2025-04-29-dependabot-reviewers-configuration-option-being-replaced-by-code-owners/}{GitHub Changelog}.

\subsection{CODEOWNERS Requirements That Commonly Break Automation}

CODEOWNERS files have strict requirements. When automation fails, these are the first things to verify:

\begin{longtable}{@{}p{0.35\textwidth}p{0.6\textwidth}@{}}
\toprule
\textbf{Requirement} & \textbf{Details} \\
\midrule
\endhead
\textbf{File Location} & Must be in \filepath{.github/}, repository root, or \filepath{docs/}. GitHub searches in that order and uses the \textbf{first} file found. \\
\midrule
\textbf{Branch Context} & The CODEOWNERS file used is the one on the \textbf{base branch} of the PR, not the head branch. \\
\midrule
\textbf{Permission Requirements} & Teams and users listed must have \textbf{explicit write access} to the repository. Read-only access is insufficient. \\
\midrule
\textbf{Team Visibility} & Teams should be \textbf{visible} (not secret) for consistent behavior across organization members. \\
\bottomrule
\end{longtable}

\begin{infobox}
For complete CODEOWNERS documentation, see \href{https://docs.github.com/articles/about-code-owners}{GitHub Docs: About Code Owners}.
\end{infobox}

% ============================================================================
% SECTION 3: ROLES AND RESPONSIBILITIES
% ============================================================================
\section{Roles and Responsibilities}
\label{sec:roles}

Clear ownership ensures that security PRs are triaged, reviewed, and merged within SLA. The following RACI-style matrix defines accountability:

\begin{longtable}{@{}p{0.28\textwidth}p{0.67\textwidth}@{}}
\toprule
\textbf{Role} & \textbf{Responsibilities} \\
\midrule
\endhead
\textbf{Repository Admin / Maintainer} & 
\begin{itemize}[leftmargin=1em, topsep=0pt, itemsep=2pt]
    \item Enable Dependabot security updates in repository settings
    \item Configure rulesets and branch protections
    \item Create and maintain CODEOWNERS patterns
    \item Install and configure the GitHub-Teams integration
\end{itemize} \\
\midrule
\textbf{AppSec Team} & 
\begin{itemize}[leftmargin=1em, topsep=0pt, itemsep=2pt]
    \item Review security PRs for risk assessment
    \item Define labeling and triage standards
    \item Set and communicate SLA expectations
    \item Configure team review assignment settings (round-robin, load balancing)
\end{itemize} \\
\midrule
\textbf{Service Owners / Dev Leads} & 
\begin{itemize}[leftmargin=1em, topsep=0pt, itemsep=2pt]
    \item Own remediation execution as assignees
    \item Fix build breaks caused by dependency updates
    \item Validate compatibility and integration tests
    \item Escalate when version bumps require code changes
\end{itemize} \\
\midrule
\textbf{Developers} & 
\begin{itemize}[leftmargin=1em, topsep=0pt, itemsep=2pt]
    \item Implement additional remediation when automated version bumps are insufficient
    \item Maintain dependency version constraints
    \item Report false positives or advisory issues
\end{itemize} \\
\bottomrule
\end{longtable}

% ============================================================================
% SECTION 4: ONE-TIME SETUP PROCEDURE
% ============================================================================
\section{One-Time Setup Procedure}
\label{sec:setup}

This section provides step-by-step instructions for initial configuration. Each step includes verification criteria.

\subsection{Step 4.1: Create or Confirm AppSec GitHub Team}
\label{subsec:setup-team}

\subsubsection{Create the Team}

\begin{enumerate}[leftmargin=2em]
    \item Navigate to your organization's Teams settings: \code{github.com/orgs/\{your-org\}/teams}
    \item Create a team named \teamname{your-org/appsec} (or confirm it exists)
    \item Set team visibility to \textbf{Visible} (recommended for transparency)
\end{enumerate}

\subsubsection{Grant Repository Access}

\begin{enumerate}[leftmargin=2em]
    \item For each repository requiring AppSec review:
    \begin{itemize}[leftmargin=2em]
        \item Navigate to Settings $\rightarrow$ Collaborators and teams
        \item Add the AppSec team with \textbf{Write} access minimum
    \end{itemize}
    \item Verify access appears correctly in the team's repository list
\end{enumerate}

\subsubsection{Configure Team Review Assignment (Optional but Recommended)}

To prevent ``notify entire team'' fatigue, configure review assignment routing:

\begin{enumerate}[leftmargin=2em]
    \item Navigate to the team's Settings page
    \item Under ``Code review,'' enable \textbf{Team review assignment}
    \item Select assignment algorithm:
    \begin{itemize}[leftmargin=2em]
        \item \textbf{Round robin:} Distribute reviews evenly
        \item \textbf{Load balance:} Consider existing review load
    \end{itemize}
    \item Set the number of reviewers to request (e.g., 1--2)
\end{enumerate}

\begin{warningbox}
With team auto-assignment enabled, GitHub may replace the team review request with individual members unless you enforce ``Require review from code owners'' in your branch protection rules.
\end{warningbox}

For detailed configuration options, see \href{https://docs.github.com/en/organizations/organizing-members-into-teams/managing-code-review-settings-for-your-team}{GitHub Docs: Managing Code Review Settings}.

\subsection{Step 4.2: Configure dependabot.yml}
\label{subsec:setup-dependabot}

The \filepath{.github/dependabot.yml} file controls Dependabot behavior. For security PR customization, focus on these configuration options:

\subsubsection{Configuration Elements}

\begin{longtable}{@{}p{0.22\textwidth}p{0.73\textwidth}@{}}
\toprule
\textbf{Option} & \textbf{Purpose} \\
\midrule
\endhead
\code{labels} & Drive triage views, dashboards, and Actions automation. Labels are applied to all PRs created for the ecosystem. \\
\midrule
\code{assignees} & Enforce accountability by assigning a human owner responsible for driving the PR to merge. \\
\midrule
\code{groups} & Reduce PR volume by grouping related security updates into single PRs. Use \code{applies-to: security-updates} for security-only grouping. \\
\midrule
\code{open-pull-requests-limit: 0} & Disables version updates, causing all customizations to apply \textbf{only} to security updates. \\
\bottomrule
\end{longtable}

\subsubsection{Example A: Security Updates Only Configuration}

This pattern ensures configuration applies \textbf{exclusively} to security updates:

\begin{lstlisting}[language=yaml, style=yamlstyle, caption={Security-only dependabot.yml configuration}]
# .github/dependabot.yml
version: 2
updates:
  # Node.js / npm ecosystem
  - package-ecosystem: "npm"
    directory: "/"
    schedule:
      interval: "daily"
    # Disable version updates - settings apply to security updates only
    open-pull-requests-limit: 0
    labels:
      - "security"
      - "dependabot"
      - "appsec-triage"
    assignees:
      - "your-service-owner"    # Individual accountable for remediation
      - "your-org/appsec"       # Optional: AppSec team visibility

  # Python / pip ecosystem
  - package-ecosystem: "pip"
    directory: "/"
    schedule:
      interval: "daily"
    open-pull-requests-limit: 0
    labels:
      - "security"
      - "dependabot"
      - "appsec-triage"
    assignees:
      - "your-service-owner"
    # Group all Python security updates into single PRs
    groups:
      python-security:
        applies-to: security-updates
        patterns:
          - "*"
\end{lstlisting}

\subsubsection{Example B: Multi-Ecosystem Configuration with Grouping}

For repositories with multiple dependency ecosystems:

\begin{lstlisting}[language=yaml, style=yamlstyle, caption={Multi-ecosystem dependabot.yml with grouped security updates}]
# .github/dependabot.yml
version: 2
updates:
  # JavaScript (npm)
  - package-ecosystem: "npm"
    directory: "/"
    schedule:
      interval: "weekly"
      day: "monday"
    open-pull-requests-limit: 0
    labels:
      - "security"
      - "dependabot"
      - "javascript"
    assignees:
      - "frontend-lead"
    groups:
      js-security-critical:
        applies-to: security-updates
        patterns:
          - "express*"
          - "lodash*"
          - "axios*"
      js-security-other:
        applies-to: security-updates
        patterns:
          - "*"

  # Go modules
  - package-ecosystem: "gomod"
    directory: "/"
    schedule:
      interval: "weekly"
    open-pull-requests-limit: 0
    labels:
      - "security"
      - "dependabot"
      - "golang"
    assignees:
      - "backend-lead"
    groups:
      go-security:
        applies-to: security-updates
        patterns:
          - "*"

  # Docker base images
  - package-ecosystem: "docker"
    directory: "/"
    schedule:
      interval: "weekly"
    open-pull-requests-limit: 0
    labels:
      - "security"
      - "dependabot"
      - "infrastructure"
    assignees:
      - "platform-lead"

  # GitHub Actions
  - package-ecosystem: "github-actions"
    directory: "/"
    schedule:
      interval: "weekly"
    open-pull-requests-limit: 0
    labels:
      - "security"
      - "dependabot"
      - "ci-cd"
    assignees:
      - "devops-lead"
\end{lstlisting}

\begin{infobox}
For complete configuration options, see \href{https://docs.github.com/en/code-security/how-tos/secure-your-supply-chain/manage-your-dependency-security/customizing-dependabot-security-prs}{GitHub Docs: Customizing Pull Requests for Dependabot Security Updates}.
\end{infobox}

\subsection{Step 4.3: Configure CODEOWNERS for AppSec Review}
\label{subsec:setup-codeowners}

The CODEOWNERS file automatically requests reviewers when PRs modify matching files.

\subsubsection{File Placement}

Create the file at \filepath{.github/CODEOWNERS} (recommended location---searched first by GitHub).

\subsubsection{Files to Protect}

Dependabot security PRs typically modify dependency manifests and lockfiles. Create patterns for all relevant files:

\begin{lstlisting}[language=bash, style=codestyle, caption={Comprehensive CODEOWNERS patterns for dependency files}]
# .github/CODEOWNERS
# ============================================================
# Dependabot Security PR Review Routing
# ============================================================

# =========================
# Policy and Config Files
# =========================
/.github/dependabot.yml         @your-org/appsec
/.github/CODEOWNERS             @your-org/appsec
/.github/workflows/             @your-org/appsec @your-org/devops

# =========================
# Node.js / JavaScript
# =========================
/package.json                   @your-org/appsec
/package-lock.json              @your-org/appsec
/yarn.lock                      @your-org/appsec
/pnpm-lock.yaml                 @your-org/appsec
/**/package.json                @your-org/appsec
/**/package-lock.json           @your-org/appsec

# =========================
# Python
# =========================
/requirements.txt               @your-org/appsec
/requirements*.txt              @your-org/appsec
/poetry.lock                    @your-org/appsec
/pyproject.toml                 @your-org/appsec
/Pipfile                        @your-org/appsec
/Pipfile.lock                   @your-org/appsec
/setup.py                       @your-org/appsec
/**/requirements.txt            @your-org/appsec

# =========================
# Java / JVM
# =========================
/pom.xml                        @your-org/appsec
/**/pom.xml                     @your-org/appsec
/build.gradle                   @your-org/appsec
/build.gradle.kts               @your-org/appsec
/**/build.gradle                @your-org/appsec
/**/build.gradle.kts            @your-org/appsec
/gradle.lockfile                @your-org/appsec
/**/gradle.lockfile             @your-org/appsec
/settings.gradle                @your-org/appsec
/settings.gradle.kts            @your-org/appsec

# =========================
# Go
# =========================
/go.mod                         @your-org/appsec
/go.sum                         @your-org/appsec
/**/go.mod                      @your-org/appsec
/**/go.sum                      @your-org/appsec

# =========================
# .NET / C#
# =========================
/**/*.csproj                    @your-org/appsec
/**/*.fsproj                    @your-org/appsec
/**/*.vbproj                    @your-org/appsec
/**/packages.config             @your-org/appsec
/**/packages.lock.json          @your-org/appsec
/Directory.Build.props          @your-org/appsec
/Directory.Packages.props       @your-org/appsec
/*.sln                          @your-org/appsec

# =========================
# Ruby
# =========================
/Gemfile                        @your-org/appsec
/Gemfile.lock                   @your-org/appsec
/**/Gemfile                     @your-org/appsec
/**/Gemfile.lock                @your-org/appsec

# =========================
# Rust
# =========================
/Cargo.toml                     @your-org/appsec
/Cargo.lock                     @your-org/appsec
/**/Cargo.toml                  @your-org/appsec
/**/Cargo.lock                  @your-org/appsec

# =========================
# PHP
# =========================
/composer.json                  @your-org/appsec
/composer.lock                  @your-org/appsec
/**/composer.json               @your-org/appsec
/**/composer.lock               @your-org/appsec

# =========================
# Docker / Container
# =========================
/Dockerfile                     @your-org/appsec
/**/Dockerfile                  @your-org/appsec
/docker-compose.yml             @your-org/appsec
/docker-compose.yaml            @your-org/appsec
/**/docker-compose*.yml         @your-org/appsec

# =========================
# Infrastructure as Code
# =========================
/terraform.lock.hcl             @your-org/appsec
/**/*.tf                        @your-org/appsec @your-org/platform
\end{lstlisting}

\begin{criticalbox}
\textbf{Operational Requirements:}
\begin{itemize}[leftmargin=1em, topsep=0pt]
    \item CODEOWNERS must be committed to the \textbf{base branch} (usually \code{main})
    \item All teams and users must have \textbf{explicit write access} to the repository
    \item Verify patterns match by checking PR review requests after deployment
\end{itemize}
\end{criticalbox}

\subsection{Step 4.4: Enforce Review Requirements}
\label{subsec:setup-enforcement}

Without enforcement, code owner reviews are \textbf{suggestions}, not requirements. Configure either Rulesets (preferred) or Branch Protection Rules.

\subsubsection{Option A: Repository Rulesets (Recommended)}

Rulesets offer more flexibility and centralized management:

\begin{enumerate}[leftmargin=2em]
    \item Navigate to Settings $\rightarrow$ Rules $\rightarrow$ Rulesets
    \item Create a new ruleset targeting the default branch (e.g., \code{main})
    \item Enable these rules:
    \begin{itemize}[leftmargin=2em]
        \item \textbf{Require a pull request before merging}
        \item \textbf{Required approvals:} Set minimum (e.g., 1)
        \item \textbf{Require review from Code Owners}
        \item \textbf{Require conversation resolution before merging} (optional)
    \end{itemize}
    \item Set enforcement to \textbf{Active}
\end{enumerate}

\subsubsection{Option B: Branch Protection Rules}

For repositories using classic branch protection:

\begin{enumerate}[leftmargin=2em]
    \item Navigate to Settings $\rightarrow$ Branches
    \item Add or edit rule for \code{main} (or default branch pattern)
    \item Enable:
    \begin{itemize}[leftmargin=2em]
        \item Require a pull request before merging
        \item Require approvals (set count)
        \item \textbf{Require review from Code Owners}
    \end{itemize}
    \item Save changes
\end{enumerate}

\begin{tipbox}
Rulesets are preferred over classic branch protection because they support layered rules, organization-level deployment, and better audit visibility.
\end{tipbox}

% ============================================================================
% SECTION 5: MICROSOFT TEAMS NOTIFICATIONS
% ============================================================================
\section{Microsoft Teams Notifications}
\label{sec:teams}

Two patterns are supported for Teams notifications. Choose based on your requirements:

\begin{longtable}{@{}p{0.18\textwidth}p{0.35\textwidth}p{0.40\textwidth}@{}}
\toprule
\textbf{Pattern} & \textbf{Use Case} & \textbf{Characteristics} \\
\midrule
\endhead
Official GitHub App & General PR visibility for all repository activity & Lowest maintenance, feature-rich, GitHub-maintained \\
\midrule
GitHub Actions Webhook & Dependabot-only notifications with custom filtering & Highly customizable, label-based routing, custom payloads \\
\bottomrule
\end{longtable}

\subsection{Pattern 1: Official GitHub App for Microsoft Teams}
\label{subsec:teams-official}

This is the recommended approach for most organizations due to its simplicity and maintenance-free operation.

\subsubsection{Installation Steps}

\begin{enumerate}[leftmargin=2em]
    \item Install the \textbf{Microsoft Teams for GitHub} app:
    \begin{itemize}[leftmargin=2em]
        \item Visit: \url{https://github.com/apps/microsoft-teams-for-github}
        \item Grant access to your organization and repositories
    \end{itemize}
    
    \item In the target Microsoft Teams channel:
    \begin{itemize}[leftmargin=2em]
        \item Sign in using: \code{@github signin}
        \item Complete OAuth authentication
    \end{itemize}
    
    \item Subscribe to repositories using these commands:
\end{enumerate}

\begin{lstlisting}[language=bash, style=bashstyle, caption={Teams channel subscription commands}]
# Sign in (required first time)
@github signin

# Subscribe to all notifications for a repository
@github subscribe your-org/your-repo

# Subscribe to specific features only (recommended)
@GitHub Notifications subscribe your-org/your-repo pulls
@GitHub Notifications subscribe your-org/your-repo reviews

# Unsubscribe from noisy features
@GitHub Notifications unsubscribe your-org/your-repo issues
@GitHub Notifications unsubscribe your-org/your-repo commits

# List current subscriptions
@github subscribe list
\end{lstlisting}

\subsubsection{Recommended Channel Configuration}

\begin{tipbox}
\textbf{Operating Model:}
\begin{itemize}[leftmargin=1em, topsep=0pt]
    \item Create a dedicated channel: \code{\#security-dependabot}
    \item Subscribe only to \code{pulls} and \code{reviews} features
    \item Use the integration's reminder capability for pending PR nudges
    \item Consider per-team channels for large organizations
\end{itemize}
\end{tipbox}

For advanced configuration, see:
\begin{itemize}[leftmargin=2em]
    \item \href{https://github.com/integrations/microsoft-teams}{GitHub Integration Documentation}
    \item \href{https://github.com/marketplace/microsoft-teams-for-github}{GitHub Marketplace Page}
\end{itemize}

\subsection{Pattern 2: GitHub Actions with Teams Webhook}
\label{subsec:teams-webhook}

Use this pattern when you need:
\begin{itemize}[leftmargin=2em]
    \item Notifications \textbf{only} for Dependabot security PRs (not all PRs)
    \item Custom message content (SLA reminders, severity levels, ownership info)
    \item Label-based routing to different channels
    \item Integration with other automation workflows
\end{itemize}

\subsubsection{Teams Webhook Constraints}

\begin{warningbox}
\textbf{Incoming Webhook Limitations:}
\begin{itemize}[leftmargin=1em, topsep=0pt]
    \item Message payload size limit: \textbf{28 KB}
    \item Rate throttling above \textbf{4 requests/second} (implement backoff)
    \item Webhooks don't support interactive elements (buttons, actions)
\end{itemize}
\end{warningbox}

\subsubsection{Example C: Basic Dependabot PR Notification}

\begin{lstlisting}[language=yaml, style=yamlstyle, caption={Basic Teams notification workflow}]
# .github/workflows/teams-notify-dependabot.yml
name: Notify Teams - Dependabot Security PR

on:
  pull_request:
    types: [opened, reopened, ready_for_review]

permissions:
  pull-requests: read

jobs:
  notify:
    runs-on: ubuntu-latest
    # Only trigger for Dependabot PRs that are not drafts
    if: >
      github.event.pull_request.user.login == 'dependabot[bot]' &&
      github.event.pull_request.draft == false
    
    steps:
      - name: Post to Teams Incoming Webhook
        env:
          TEAMS_WEBHOOK_URL: ${{ secrets.TEAMS_INCOMING_WEBHOOK_URL }}
          PR_URL: ${{ github.event.pull_request.html_url }}
          PR_TITLE: ${{ github.event.pull_request.title }}
          REPO: ${{ github.repository }}
          PR_NUMBER: ${{ github.event.pull_request.number }}
        run: |
          payload=$(cat <<EOF
          {
            "@type": "MessageCard",
            "@context": "http://schema.org/extensions",
            "themeColor": "FF6600",
            "summary": "Dependabot Security PR #${PR_NUMBER}",
            "sections": [{
              "activityTitle": "Security Update Required",
              "activitySubtitle": "${REPO}",
              "facts": [{
                "name": "PR Title",
                "value": "${PR_TITLE}"
              }, {
                "name": "Repository",
                "value": "${REPO}"
              }, {
                "name": "Action Required",
                "value": "AppSec review via CODEOWNERS"
              }],
              "markdown": true
            }],
            "potentialAction": [{
              "@type": "OpenUri",
              "name": "View Pull Request",
              "targets": [{
                "os": "default",
                "uri": "${PR_URL}"
              }]
            }]
          }
          EOF
          )
          curl -sS -H "Content-Type: application/json" \
            -d "$payload" "$TEAMS_WEBHOOK_URL"
\end{lstlisting}

\subsubsection{Example D: Enhanced Notification with Label Filtering}

\begin{lstlisting}[language=yaml, style=yamlstyle, caption={Advanced Teams notification with label filtering and severity}]
# .github/workflows/teams-notify-dependabot-enhanced.yml
name: Notify Teams - Dependabot Security PR (Enhanced)

on:
  pull_request:
    types: [opened, reopened, ready_for_review, labeled]

permissions:
  pull-requests: read

jobs:
  notify-appsec:
    runs-on: ubuntu-latest
    # Trigger for Dependabot PRs with appsec-triage label
    if: >
      github.event.pull_request.user.login == 'dependabot[bot]' &&
      github.event.pull_request.draft == false &&
      contains(github.event.pull_request.labels.*.name, 'appsec-triage')
    
    steps:
      - name: Determine Severity Color
        id: severity
        run: |
          LABELS="${{ join(github.event.pull_request.labels.*.name, ',') }}"
          if echo "$LABELS" | grep -qi "critical"; then
            echo "color=FF0000" >> $GITHUB_OUTPUT
            echo "level=CRITICAL" >> $GITHUB_OUTPUT
          elif echo "$LABELS" | grep -qi "high"; then
            echo "color=FF6600" >> $GITHUB_OUTPUT
            echo "level=HIGH" >> $GITHUB_OUTPUT
          else
            echo "color=FFA500" >> $GITHUB_OUTPUT
            echo "level=MODERATE" >> $GITHUB_OUTPUT
          fi

      - name: Post to Teams Incoming Webhook
        env:
          TEAMS_WEBHOOK_URL: ${{ secrets.TEAMS_APPSEC_WEBHOOK_URL }}
          PR_URL: ${{ github.event.pull_request.html_url }}
          PR_TITLE: ${{ github.event.pull_request.title }}
          PR_BODY: ${{ github.event.pull_request.body }}
          REPO: ${{ github.repository }}
          PR_NUMBER: ${{ github.event.pull_request.number }}
          ASSIGNEES: ${{ join(github.event.pull_request.assignees.*.login, ', ') }}
          SEVERITY_COLOR: ${{ steps.severity.outputs.color }}
          SEVERITY_LEVEL: ${{ steps.severity.outputs.level }}
        run: |
          # Truncate body to avoid webhook size limits
          BODY_TRUNCATED=$(echo "$PR_BODY" | head -c 500)
          
          payload=$(cat <<EOF
          {
            "@type": "MessageCard",
            "@context": "http://schema.org/extensions",
            "themeColor": "${SEVERITY_COLOR}",
            "summary": "[${SEVERITY_LEVEL}] Dependabot Security PR #${PR_NUMBER}",
            "sections": [{
              "activityTitle": "Security Update: ${SEVERITY_LEVEL}",
              "activitySubtitle": "${REPO} #${PR_NUMBER}",
              "facts": [{
                "name": "PR Title",
                "value": "${PR_TITLE}"
              }, {
                "name": "Severity",
                "value": "${SEVERITY_LEVEL}"
              }, {
                "name": "Repository",
                "value": "${REPO}"
              }, {
                "name": "Assignees",
                "value": "${ASSIGNEES:-Unassigned}"
              }, {
                "name": "SLA",
                "value": "Critical: 24h | High: 72h | Moderate: 7d"
              }],
              "text": "${BODY_TRUNCATED}",
              "markdown": true
            }],
            "potentialAction": [{
              "@type": "OpenUri",
              "name": "Review PR",
              "targets": [{"os": "default", "uri": "${PR_URL}"}]
            }, {
              "@type": "OpenUri", 
              "name": "View Advisory",
              "targets": [{"os": "default", "uri": "${PR_URL}/files"}]
            }]
          }
          EOF
          )
          curl -sS -H "Content-Type: application/json" \
            -d "$payload" "$TEAMS_WEBHOOK_URL"
\end{lstlisting}

\subsubsection{Storing the Webhook Secret}

\begin{enumerate}[leftmargin=2em]
    \item In Microsoft Teams, create an Incoming Webhook connector for your channel
    \item Copy the webhook URL
    \item In GitHub:
    \begin{itemize}[leftmargin=2em]
        \item Navigate to Settings $\rightarrow$ Secrets and variables $\rightarrow$ Actions
        \item Create a new repository secret: \code{TEAMS\_INCOMING\_WEBHOOK\_URL}
        \item Paste the webhook URL
    \end{itemize}
\end{enumerate}

\begin{infobox}
For Incoming Webhook setup, see \href{https://learn.microsoft.com/en-us/microsoftteams/platform/webhooks-and-connectors/how-to/add-incoming-webhook}{Microsoft Learn: Create an Incoming Webhook}.
\end{infobox}

% ============================================================================
% SECTION 6: OPERATIONAL WORKFLOW
% ============================================================================
\section{Operational Workflow}
\label{sec:operations}

This section describes the day-to-day process for handling Dependabot security PRs.

\subsection{Trigger Event}

A new Dependabot security update PR opens against the default branch when:
\begin{itemize}[leftmargin=2em]
    \item A new security advisory is published for a dependency
    \item An existing advisory is updated with new severity or fix versions
    \item Dependabot detects a vulnerable version during scheduled scans
\end{itemize}

\subsection{AppSec Triage Checklist}

\begin{enumerate}[leftmargin=2em]
    \item \textbf{Verify PR Author:} Confirm the PR author is \code{dependabot[bot]}
    
    \item \textbf{Confirm Reviewer Assignment:}
    \begin{itemize}[leftmargin=2em]
        \item Check that AppSec team is in the ``Reviewers'' section
        \item If missing, verify CODEOWNERS patterns match the modified files
    \end{itemize}
    
    \item \textbf{Validate Labels:}
    \begin{itemize}[leftmargin=2em]
        \item Required: \code{security}, \code{dependabot}
        \item Recommended: \code{appsec-triage}, ecosystem-specific labels
    \end{itemize}
    
    \item \textbf{Confirm Assignee:}
    \begin{itemize}[leftmargin=2em]
        \item Service owner should be assigned
        \item If unassigned, escalate to repository maintainer
    \end{itemize}
    
    \item \textbf{Security Review:}
    \begin{itemize}[leftmargin=2em]
        \item Review the advisory severity (Critical, High, Moderate, Low)
        \item Check affected package and version range
        \item Assess potential for breaking changes
        \item Review changelog/release notes for the updated version
    \end{itemize}
    
    \item \textbf{CI/CD Verification:}
    \begin{itemize}[leftmargin=2em]
        \item Ensure all required checks pass
        \item If tests fail, route to service owner for remediation
        \item Document any manual testing requirements
    \end{itemize}
    
    \item \textbf{Approve and Merge:}
    \begin{itemize}[leftmargin=2em]
        \item Provide approval if all checks pass
        \item Use ``Squash and merge'' to maintain clean history
        \item Add merge comment noting security context if needed
    \end{itemize}
\end{enumerate}

\subsection{Post-Merge Validation}

After merging a security update PR:

\begin{enumerate}[leftmargin=2em]
    \item \textbf{Verify Alert Resolution:}
    \begin{itemize}[leftmargin=2em]
        \item Navigate to Security $\rightarrow$ Dependabot alerts
        \item Confirm the related alert status changes to ``Resolved'' or ``Fixed''
        \item Note: Resolution timing depends on ecosystem and advisory ingestion
    \end{itemize}
    
    \item \textbf{Monitor Deployment:}
    \begin{itemize}[leftmargin=2em]
        \item Verify the change deploys successfully through CI/CD
        \item Check application health metrics post-deployment
    \end{itemize}
    
    \item \textbf{Document Exceptions:}
    \begin{itemize}[leftmargin=2em]
        \item If the update cannot be merged, document the reason
        \item Create a tracking issue for manual remediation
        \item Set a review date for re-evaluation
    \end{itemize}
\end{enumerate}

% ============================================================================
% SECTION 7: TROUBLESHOOTING GUIDE
% ============================================================================
\section{Troubleshooting Guide}
\label{sec:troubleshooting}

This section addresses common issues and their resolutions.

\subsection{Symptom: AppSec Not Auto-Requested as Reviewer}

\begin{longtable}{@{}p{0.35\textwidth}p{0.60\textwidth}@{}}
\toprule
\textbf{Root Cause} & \textbf{Resolution} \\
\midrule
\endhead
CODEOWNERS file in wrong location & Move to \filepath{.github/CODEOWNERS} (searched first). Remove any duplicate files in root or \filepath{docs/}. \\
\midrule
CODEOWNERS not on base branch & Ensure CODEOWNERS is merged to \code{main} (or your default branch). The PR's base branch version is used. \\
\midrule
Team lacks write access & Grant \textbf{Write} permission to the AppSec team in repository settings. \\
\midrule
Pattern doesn't match files & Verify your patterns cover the files Dependabot modifies. Use \code{**/} for recursive matching. \\
\midrule
Secret team visibility & Change team visibility to ``Visible'' for consistent behavior. \\
\bottomrule
\end{longtable}

\subsection{Symptom: Team Requested But Individuals Assigned}

This is expected behavior when team auto-assignment is enabled. GitHub replaces team requests with individuals unless code owner review is enforced.

\textbf{Resolution:} Enable ``Require review from Code Owners'' in your ruleset or branch protection settings.

\subsection{Symptom: Customizations Not Applying to Security PRs}

\begin{longtable}{@{}p{0.35\textwidth}p{0.60\textwidth}@{}}
\toprule
\textbf{Root Cause} & \textbf{Resolution} \\
\midrule
\endhead
Using \code{target-branch} setting & Security PRs target the default branch. \code{target-branch} applies to version updates. Remove or adjust this setting. \\
\midrule
Version updates enabled & Set \code{open-pull-requests-limit: 0} to disable version updates and apply settings to security updates only. \\
\midrule
Configuration syntax error & Validate YAML syntax. Check for indentation issues or invalid keys. \\
\midrule
Configuration on wrong branch & Ensure \filepath{dependabot.yml} is on the default branch. \\
\bottomrule
\end{longtable}

\subsection{Symptom: Teams Channel Not Receiving Notifications}

\subsubsection{For Official GitHub App}

\begin{enumerate}[leftmargin=2em]
    \item Verify the GitHub app is installed and has repository access:
    \begin{itemize}[leftmargin=2em]
        \item Check organization settings $\rightarrow$ Installed GitHub Apps
        \item Verify repository permissions include the affected repository
    \end{itemize}
    
    \item Re-authenticate in Teams:
    \begin{itemize}[leftmargin=2em]
        \item Run \code{@github signout}
        \item Run \code{@github signin}
    \end{itemize}
    
    \item Re-subscribe:
    \begin{itemize}[leftmargin=2em]
        \item Run \code{@github unsubscribe your-org/your-repo}
        \item Run \code{@github subscribe your-org/your-repo pulls reviews}
    \end{itemize}
\end{enumerate}

\subsubsection{For Webhook-Based Actions}

\begin{enumerate}[leftmargin=2em]
    \item \textbf{Verify Secret Exists:}
    \begin{itemize}[leftmargin=2em]
        \item Check Settings $\rightarrow$ Secrets $\rightarrow$ Actions
        \item Confirm \code{TEAMS\_INCOMING\_WEBHOOK\_URL} is present
    \end{itemize}
    
    \item \textbf{Test Webhook URL:}
    \begin{lstlisting}[language=bash, style=bashstyle]
curl -X POST -H "Content-Type: application/json" \
  -d '{"text": "Test message from GitHub Actions"}' \
  "YOUR_WEBHOOK_URL"
    \end{lstlisting}
    
    \item \textbf{Check Workflow Logs:}
    \begin{itemize}[leftmargin=2em]
        \item Navigate to Actions tab
        \item Find the workflow run
        \item Check for curl errors or HTTP response codes
    \end{itemize}
    
    \item \textbf{Verify Payload Size:}
    \begin{itemize}[leftmargin=2em]
        \item Keep total payload under 28 KB
        \item Truncate PR body if necessary
    \end{itemize}
    
    \item \textbf{Check Rate Limits:}
    \begin{itemize}[leftmargin=2em]
        \item Implement retry logic with exponential backoff
        \item Avoid bursts of more than 4 requests/second
    \end{itemize}
\end{enumerate}

\subsection{Symptom: Dependabot PRs Failing CI}

\begin{longtable}{@{}p{0.35\textwidth}p{0.60\textwidth}@{}}
\toprule
\textbf{Root Cause} & \textbf{Resolution} \\
\midrule
\endhead
Breaking API changes & Review changelog for migration guide. May require code changes beyond version bump. \\
\midrule
Incompatible peer dependencies & Use \code{groups} to bundle related updates. May need manual resolution. \\
\midrule
CI secrets unavailable & Dependabot PRs have limited secret access. Use \code{pull\_request\_target} with caution or enable Dependabot secrets. \\
\midrule
Lockfile conflicts & Regenerate lockfile locally and push to the PR branch. \\
\bottomrule
\end{longtable}

% ============================================================================
% SECTION 8: REFERENCE IMPLEMENTATION
% ============================================================================
\section{Reference Implementation}
\label{sec:reference}

This section provides a complete, ready-to-deploy configuration set following the ``golden path'' recommendations.

\subsection{File 1: dependabot.yml}

\begin{lstlisting}[language=yaml, style=yamlstyle, caption={Production-ready dependabot.yml}]
# .github/dependabot.yml
# Golden path configuration for security update PR customization
version: 2

updates:
  # ===================
  # Node.js / npm
  # ===================
  - package-ecosystem: "npm"
    directory: "/"
    schedule:
      interval: "daily"
    open-pull-requests-limit: 0  # Security updates only
    labels:
      - "security"
      - "dependabot"
      - "appsec-triage"
      - "javascript"
    assignees:
      - "your-service-owner"
    groups:
      npm-security:
        applies-to: security-updates
        patterns:
          - "*"

  # ===================
  # Python / pip
  # ===================
  - package-ecosystem: "pip"
    directory: "/"
    schedule:
      interval: "daily"
    open-pull-requests-limit: 0
    labels:
      - "security"
      - "dependabot"
      - "appsec-triage"
      - "python"
    assignees:
      - "your-service-owner"
    groups:
      pip-security:
        applies-to: security-updates
        patterns:
          - "*"

  # ===================
  # Go modules
  # ===================
  - package-ecosystem: "gomod"
    directory: "/"
    schedule:
      interval: "daily"
    open-pull-requests-limit: 0
    labels:
      - "security"
      - "dependabot"
      - "appsec-triage"
      - "golang"
    assignees:
      - "your-service-owner"

  # ===================
  # GitHub Actions
  # ===================
  - package-ecosystem: "github-actions"
    directory: "/"
    schedule:
      interval: "weekly"
    open-pull-requests-limit: 0
    labels:
      - "security"
      - "dependabot"
      - "appsec-triage"
      - "ci-cd"
    assignees:
      - "your-devops-owner"

  # ===================
  # Docker
  # ===================
  - package-ecosystem: "docker"
    directory: "/"
    schedule:
      interval: "weekly"
    open-pull-requests-limit: 0
    labels:
      - "security"
      - "dependabot"
      - "appsec-triage"
      - "infrastructure"
    assignees:
      - "your-platform-owner"
\end{lstlisting}

\subsection{File 2: CODEOWNERS}

\begin{lstlisting}[language=bash, style=codestyle, caption={Production-ready CODEOWNERS}]
# .github/CODEOWNERS
# Golden path configuration for AppSec review routing

# ============================================
# Policy Files (AppSec owns security config)
# ============================================
/.github/dependabot.yml         @your-org/appsec
/.github/CODEOWNERS             @your-org/appsec
/.github/workflows/             @your-org/appsec @your-org/devops

# ============================================
# JavaScript / Node.js Dependencies
# ============================================
/package.json                   @your-org/appsec
/package-lock.json              @your-org/appsec
/yarn.lock                      @your-org/appsec
/pnpm-lock.yaml                 @your-org/appsec

# ============================================
# Python Dependencies
# ============================================
/requirements.txt               @your-org/appsec
/requirements*.txt              @your-org/appsec
/pyproject.toml                 @your-org/appsec
/poetry.lock                    @your-org/appsec
/Pipfile.lock                   @your-org/appsec

# ============================================
# Go Dependencies
# ============================================
/go.mod                         @your-org/appsec
/go.sum                         @your-org/appsec

# ============================================
# Docker / Containers
# ============================================
/Dockerfile                     @your-org/appsec
/docker-compose*.yml            @your-org/appsec
\end{lstlisting}

\subsection{File 3: Teams Notification Workflow}

\begin{lstlisting}[language=yaml, style=yamlstyle, caption={Production-ready Teams notification workflow}]
# .github/workflows/teams-notify-security-pr.yml
name: Notify Teams - Security PR

on:
  pull_request:
    types: [opened, reopened, ready_for_review]

permissions:
  pull-requests: read

jobs:
  notify-teams:
    runs-on: ubuntu-latest
    if: >
      github.event.pull_request.user.login == 'dependabot[bot]' &&
      github.event.pull_request.draft == false
    
    steps:
      - name: Send Teams Notification
        env:
          TEAMS_WEBHOOK_URL: ${{ secrets.TEAMS_INCOMING_WEBHOOK_URL }}
        run: |
          # Build adaptive card payload
          payload=$(cat <<'EOF'
          {
            "@type": "MessageCard",
            "@context": "http://schema.org/extensions",
            "themeColor": "FF6600",
            "summary": "Security PR #${{ github.event.pull_request.number }}",
            "sections": [{
              "activityTitle": "Dependabot Security Update",
              "activitySubtitle": "${{ github.repository }}",
              "facts": [
                {"name": "PR", "value": "#${{ github.event.pull_request.number }}"},
                {"name": "Title", "value": "${{ github.event.pull_request.title }}"},
                {"name": "Action", "value": "AppSec review required"}
              ],
              "markdown": true
            }],
            "potentialAction": [{
              "@type": "OpenUri",
              "name": "View PR",
              "targets": [{"os": "default", "uri": "${{ github.event.pull_request.html_url }}"}]
            }]
          }
          EOF
          )
          
          curl -sS -X POST \
            -H "Content-Type: application/json" \
            -d "$payload" \
            "$TEAMS_WEBHOOK_URL"
\end{lstlisting}

\subsection{Validation Checklist}

After deploying the configuration, run through this checklist:

\begin{enumerate}[leftmargin=2em]
    \item[$\square$] \textbf{dependabot.yml:} Committed to \code{main} branch and validated (no syntax errors)
    \item[$\square$] \textbf{CODEOWNERS:} Committed to \code{main} branch in \filepath{.github/} directory
    \item[$\square$] \textbf{Team Permissions:} AppSec team has Write access to the repository
    \item[$\square$] \textbf{Ruleset/Branch Protection:} ``Require review from Code Owners'' is enabled
    \item[$\square$] \textbf{Teams Integration:} Channel is subscribed to repository PRs
    \item[$\square$] \textbf{Webhook Secret:} \code{TEAMS\_INCOMING\_WEBHOOK\_URL} is set (if using Actions)
    \item[$\square$] \textbf{Test PR:} Create a test security alert to verify end-to-end flow
\end{enumerate}

% ============================================================================
% APPENDICES
% ============================================================================
\appendix

\section{Quick Reference: YAML Configuration Keys}
\label{app:yaml-reference}

\begin{longtable}{@{}p{0.30\textwidth}p{0.65\textwidth}@{}}
\toprule
\textbf{Key} & \textbf{Description} \\
\midrule
\endhead
\code{package-ecosystem} & Dependency ecosystem (npm, pip, maven, gomod, docker, etc.) \\
\midrule
\code{directory} & Location of manifest file relative to repository root \\
\midrule
\code{schedule.interval} & Update check frequency (daily, weekly, monthly) \\
\midrule
\code{open-pull-requests-limit} & Max concurrent version update PRs (0 = disable version updates) \\
\midrule
\code{labels} & Labels to apply to created PRs \\
\midrule
\code{assignees} & Users/teams to assign to PRs \\
\midrule
\code{groups} & Group related updates into single PRs \\
\midrule
\code{groups.*.applies-to} & Scope grouping to \code{version-updates} or \code{security-updates} \\
\midrule
\code{groups.*.patterns} & Package name patterns to include in group \\
\midrule
\code{target-branch} & Branch for version update PRs (not security updates) \\
\bottomrule
\end{longtable}

\section{External References}
\label{app:references}

\begin{enumerate}[leftmargin=2em]
    \item \textbf{GitHub Docs: Customizing Dependabot Security PRs}\\
    \url{https://docs.github.com/en/code-security/how-tos/secure-your-supply-chain/manage-your-dependency-security/customizing-dependabot-security-prs}
    
    \item \textbf{GitHub Docs: About Code Owners}\\
    \url{https://docs.github.com/articles/about-code-owners}
    
    \item \textbf{GitHub Docs: Available Rules for Rulesets}\\
    \url{https://docs.github.com/en/repositories/configuring-branches-and-merges-in-your-repository/managing-rulesets/available-rules-for-rulesets}
    
    \item \textbf{GitHub Docs: Automating Dependabot with GitHub Actions}\\
    \url{https://docs.github.com/enterprise-cloud@latest/code-security/dependabot/working-with-dependabot/automating-dependabot-with-github-actions}
    
    \item \textbf{GitHub Changelog: Dependabot Reviewers Replaced by CODEOWNERS}\\
    \url{https://github.blog/changelog/2025-04-29-dependabot-reviewers-configuration-option-being-replaced-by-code-owners/}
    
    \item \textbf{GitHub Apps: Microsoft Teams for GitHub}\\
    \url{https://github.com/apps/microsoft-teams-for-github}
    
    \item \textbf{GitHub Integration: Microsoft Teams}\\
    \url{https://github.com/integrations/microsoft-teams}
    
    \item \textbf{Microsoft Learn: Create an Incoming Webhook}\\
    \url{https://learn.microsoft.com/en-us/microsoftteams/platform/webhooks-and-connectors/how-to/add-incoming-webhook}
    
    \item \textbf{GitHub Docs: Managing Code Review Settings}\\
    \url{https://docs.github.com/en/organizations/organizing-members-into-teams/managing-code-review-settings-for-your-team}
\end{enumerate}

% ============================================================================
% DOCUMENT END
% ============================================================================
\end{document}
