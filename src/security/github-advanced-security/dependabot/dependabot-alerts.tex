% !TEX program = pdflatex
% Compile with: pdflatex -shell-escape dependabot-cheatsheet.tex
\documentclass[11pt,letterpaper]{article}

% ---------- Packages ----------
\usepackage[T1]{fontenc}

% ---------- Minted (CI-safe fallback) ----------
\usepackage{xparse}
\usepackage{iftex}
\usepackage{ifthen}

% Some repos run pdflatex without -shell-escape in CI.
% Use minted when shell-escape is enabled; otherwise fall back to listings.
\newif\ifciShellEscape
\ifdefined\pdfshellescape
  \ifnum\pdfshellescape=1\relax
    \ciShellEscapetrue
  \else
    \ciShellEscapefalse
  \fi
\else
  \ciShellEscapefalse
\fi

\ifciShellEscape
  \usepackage[newfloat,cache=false]{minted}
\else
  \usepackage{listings}
  \usepackage{newfloat}
  \DeclareFloatingEnvironment[fileext=lol, listname={List of Listings}, name=Listing]{listing}

  % Basic listings defaults (safe for UTF-8 text in code blocks)
  \lstset{
    basicstyle=\ttfamily\small,
    breaklines=true,
    breakatwhitespace=false,
    columns=fullflexible,
    keepspaces=true,
    showstringspaces=false,
    tabsize=2,
    frame=single,
    upquote=true,
    % Common Unicode glyphs seen in docs/snippets
    literate=
      {•}{{\textbullet}}1
      {–}{{--}}1
      {—}{{---}}1
      {→}{{$\rightarrow$}}1
      {←}{{$\leftarrow$}}1
  }

  % Define a few common "languages" so listings won't error
  \lstdefinelanguage{yaml}{} 
  \lstdefinelanguage{json}{} 
  \lstdefinelanguage{bash}{} 
  \lstdefinelanguage{console}{} 
  \lstdefinelanguage{powershell}{} 
  \lstdefinelanguage{markdown}{} 
  \lstdefinelanguage{text}{} 

  % minted compatibility shims (ignore minted-style options safely)
  \NewDocumentEnvironment{minted}{ O{} m }{%
    \def\minted@opts{#1}%
    \def\minted@lang{#2}%
    \ifthenelse{\begingroup\edef\temp{\minted@opts}\endgroup\in@{linenos}{\temp}\relax=1\relax}{%
      \lstset{numbers=left, numberstyle=\tiny, stepnumber=1, numbersep=6pt}%
    }{%
      \lstset{numbers=none}%
    }%
    \lstset{language=\minted@lang}%
    \begin{lstlisting}%
  }{%
    \end{lstlisting}%
  }

  \NewDocumentCommand{\mintinline}{ O{} m m }{\texttt{#3}}
  \NewDocumentCommand{\inputminted}{ O{} m m }{\lstinputlisting[language=#2]{#3}}

  \NewDocumentCommand{\setminted}{ O{} m }{}
  \NewDocumentCommand{\setmintedinline}{ O{} m }{}
  \NewDocumentCommand{\usemintedstyle}{ m }{}

  % \newminted{lang}{opts} -> defines an environment <lang>code
  \NewDocumentCommand{\newminted}{ m O{} }{%
    \NewDocumentEnvironment{#1code}{ O{} }{\begin{minted}{#1}}{\end{minted}}%
  }
  % \newmintedfile{lang}{opts} -> defines \input<lang>code{file}
  \NewDocumentCommand{\newmintedfile}{ m O{} }{%
    \expandafter\NewDocumentCommand\csname input#1code\endcsname{ m }{\inputminted{#1}{##1}}%
  }
\fi

\usepackage[utf8]{inputenc}
\usepackage{lmodern}
\usepackage{microtype}
\usepackage[a4paper,margin=0.8in]{geometry} % change to letter if preferred
\usepackage{parskip}
\usepackage{enumitem}
\usepackage[dvipsnames]{xcolor}
\usepackage{hyperref}
\usepackage{booktabs}
\usepackage{tabularx}
\usepackage{array}
\usepackage{multicol}
\usepackage[most]{tcolorbox}
\usepackage{pifont}

% ---------- Styling ----------
\definecolor{Link}{HTML}{0A66C2}
\definecolor{NoteBG}{HTML}{F5F7FF}
\definecolor{Accent}{HTML}{1F7A8C}
\definecolor{LightGray}{gray}{0.97}

\hypersetup{colorlinks=true, linkcolor=Link, urlcolor=Link, citecolor=Link}
\setlist{itemsep=2pt, topsep=4pt, leftmargin=1.1em}

\newcommand{\cmark}{\textcolor{ForestGreen}{\ding{51}}}
\newcommand{\xmark}{\textcolor{BrickRed}{\ding{55}}}

\newtcolorbox{callout}[1][]{
  enhanced, colback=NoteBG, colframe=Accent!40!black, left=8pt, right=8pt,
  arc=2mm, boxrule=0.4pt, toptitle=2pt, bottomtitle=2pt, #1
}

% A compact heading style
\makeatletter
\def\section{\@startsection{section}{1}{0pt}{6pt}{4pt}{\large\bfseries\color{Accent}}}
\def\subsection{\@startsection{subsection}{2}{0pt}{6pt}{3pt}{\normalsize\bfseries}}
\makeatother

% ---------- Document ----------

% ---------- Silence first-run missing .toc (prevents strict latexmk aborts) ----------
\makeatletter
\let\ci@orig@starttoc\@starttoc
\def\@starttoc#1{%
  \begingroup
  \IfFileExists{\jobname.#1}{\ci@orig@starttoc{#1}}{}%
  \endgroup
}
\makeatother

\begin{document}

\begin{minipage}{0.72\linewidth}
  {\huge\bfseries Dependabot Alerts \& Updates — Quick Cheatsheet}\\[2pt]
  \textit{Enable, configure, and triage GitHub dependency updates with confidence.}
\end{minipage}
\hfill
\begin{minipage}{0.25\linewidth}
  \raggedleft \small
  \textbf{Scope:} GitHub repos \\
  \textbf{Audience:} Dev, Sec, Maintainers
\end{minipage}

\vspace{6pt}
\begin{callout}
\textbf{Key defaults:} \emph{Public} repos have alerts on by default. \emph{Private} repos require enabling in Settings. Viewing alerts typically requires \texttt{read}+ access; enabling/managing may require \texttt{admin} depending on org policy.
\end{callout}

\begin{multicols}{2}
% =================== Column 1 ===================
\section{Enable \& Visibility}
\begin{itemize}
  \item \textbf{Repo} $\rightarrow$ \textbf{Settings} $\rightarrow$ \textbf{Code security \& analysis} $\rightarrow$ toggle \textbf{Dependabot alerts}.
  \item \textbf{Who can view:} anyone with \texttt{read}+ (and everyone for public repos).
  \item \textbf{Who can enable/manage:} usually \texttt{admin}/\texttt{maintain}; org policies may restrict repo-level toggles.
\end{itemize}

\subsection*{Roles (repo level)}
\begin{tabularx}{\linewidth}{@{}l *{5}{>{\centering\arraybackslash}X}@{}}
\toprule
 & \textbf{Read} & \textbf{Triage} & \textbf{Write} & \textbf{Maint.} & \textbf{Admin} \\
\midrule
View alerts & \cmark & \cmark & \cmark & \cmark & \cmark \\
Dismiss alert & \xmark & \cmark & \cmark & \cmark & \cmark \\
Merge PRs & \xmark & \cmark & \cmark & \cmark & \cmark \\
Enable alerts & \xmark & \xmark & \xmark & \cmark & \cmark \\
\bottomrule
\end{tabularx}

\section{Configuration Basics}
\textbf{File:} \texttt{.github/dependabot.yml}

\begin{minted}[fontsize=\footnotesize, breaklines, bgcolor=LightGray, frame=lines]{yaml}
version: 2
updates:
  - package-ecosystem: "npm"     # npm, pip, gomod, maven, gradle, cargo, etc.
    directory: "/"               # path to manifest(s)
    schedule:
      interval: "weekly"         # daily, weekly, monthly
    # groups:                    # optional: group related changes into one PR
    #   minor-deps:
    #     patterns: ["*"]
    #     update-types: ["minor", "patch"]
  - package-ecosystem: "pip"
    directory: "/"
    schedule:
      interval: "weekly"
    ignore:                       # optional: ignore certain deps/versions
      - dependency-name: "example"
        versions: ["<1.2.3"]
\end{minted}

\subsection*{Common Options}
\begin{itemize}
  \item \texttt{schedule.interval}: \texttt{daily} | \texttt{weekly} | \texttt{monthly}.
  \item \texttt{allow}/\texttt{ignore}: control which dependencies update.
  \item \texttt{groups}: consolidate related updates into fewer PRs.
  \item \texttt{open-pull-requests-limit}: cap concurrent update PRs.
\end{itemize}

\section{Notifications \& Automation}
\begin{itemize}
  \item \textbf{Alerts}: GitHub notifications, email, or webhooks (e.g., Slack/Teams via your integration).
  \item \textbf{PRs}: Auto-opened by Dependabot; configure required checks to ensure safe merges.
  \item \textbf{Rules UI}: At org/repo scope you can define rules to auto-dismiss low-risk alerts or auto-open PRs by severity/ecosystem.
\end{itemize}

% =================== Column 2 ===================
\columnbreak

\section{Triage \& Remediation Flow}
\begin{enumerate}
  \item \textbf{Assess} severity, reachability, and exploitability.
  \item \textbf{Decide} on \emph{upgrade}, \emph{pin}, \emph{patch}, or \emph{temporary dismiss}.
  \item \textbf{Validate} with tests, SCA/SAST/DAST, and runtime checks.
  \item \textbf{Merge} when CI passes; monitor post-deploy.
\end{enumerate}

\subsection*{Recommended Dismissal Reasons}
\begin{itemize}
  \item \emph{Not affected} (unreachable code path).
  \item \emph{Legacy/unmaintained} with compensating controls.
  \item \emph{False positive} (erroneous advisory match).
  \item \emph{Will fix} in planned upgrade window with ticket reference.
\end{itemize}

\section{Best Practices}
\begin{itemize}
  \item Keep \texttt{interval} small for critical services; group minor/patch updates.
  \item Set \textbf{required checks} on Dependabot PRs (tests, build, security scans).
  \item Gate merges by \textbf{vuln severity} via branch protection or policy.
  \item Use \textbf{labels} and \textbf{codeowners} to route reviews quickly.
  \item Limit \textbf{concurrent PRs} to avoid queue congestion.
\end{itemize}

\section{Sample Policies \& Patterns}
\subsection*{Limit Concurrent PRs}
\begin{minted}[fontsize=\footnotesize, breaklines, bgcolor=LightGray, frame=lines]{yaml}
version: 2
updates:
  - package-ecosystem: "npm"
    directory: "/"
    open-pull-requests-limit: 5
    schedule: { interval: "daily" }
\end{minted}

\subsection*{Group Minor/Patch Updates}
\begin{minted}[fontsize=\footnotesize, breaklines, bgcolor=LightGray, frame=lines]{yaml}
updates:
  - package-ecosystem: "maven"
    directory: "/"
    schedule: { interval: "weekly" }
    groups:
      routine-minor-patch:
        patterns: ["*"]
        update-types: ["minor", "patch"]
\end{minted}

\subsection*{Ignore Specific Versions}
\begin{minted}[fontsize=\footnotesize, breaklines, bgcolor=LightGray, frame=lines]{yaml}
updates:
  - package-ecosystem: "gomod"
    directory: "/"
    schedule: { interval: "weekly" }
    ignore:
      - dependency-name: "example.com/libfoo"
        versions: ["1.4.x", "<1.3.2"]
\end{minted}

\section{Quick Links}
\begin{itemize}
  \item \textbf{Repo Settings:} Code security \& analysis \textrightarrow{} Dependabot alerts.
  \item \textbf{Config File:} \texttt{.github/dependabot.yml} at repo root.
  \item \textbf{Routing:} Labels + CODEOWNERS for fast PR assignment.
\end{itemize}

\begin{callout}
\textbf{Compile tip:} This document uses \texttt{minted} for syntax highlighting. Compile with \texttt{-shell-escape} enabled (e.g., \texttt{pdflatex -shell-escape}).
\end{callout}

\end{multicols}

\end{document}