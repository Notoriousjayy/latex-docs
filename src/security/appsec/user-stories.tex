\documentclass[11pt,a4paper]{article}

% --- Page + typography ---
\usepackage[a4paper,margin=1in]{geometry}
\usepackage{lmodern}            % Latin Modern fonts
\usepackage[T1]{fontenc}
\usepackage[utf8]{inputenc}
\usepackage{microtype}          % better kerning/justification
\emergencystretch=2em
\usepackage{parskip}            % space between paragraphs, no indents

% --- Structure + lists ---
\usepackage{enumitem}
\setlist{itemsep=2pt, topsep=4pt, leftmargin=1.2em}
% Allow deep nesting so story/task lists don't overflow default limits
\setlistdepth{8}
\renewlist{itemize}{itemize}{8}

\usepackage{titlesec}
\titlespacing*{\section}{0pt}{6pt plus 2pt}{4pt}
\titlespacing*{\subsection}{0pt}{5pt}{3pt}

% --- Color + links ---
\usepackage[dvipsnames]{xcolor}
\usepackage{hyperref}
\hypersetup{
  colorlinks=true,
  linkcolor=black,
  urlcolor=MidnightBlue,
  citecolor=black,
  pdfauthor={Jordan Suber},
  pdftitle={User Stories by Chapter: CI → CD → GitHub Actions}
}
\urlstyle{same}

% --- Math + symbols ---
\usepackage{amsmath,amssymb} % provides \square and math symbols

% --- Layout helpers for story cards ---
\usepackage[skins,breakable]{tcolorbox}
\tcbset{
  colback=gray!2,
  colframe=gray!50,
  arc=2pt,
  boxrule=0.4pt,
  left=8pt,right=8pt,top=8pt,bottom=8pt,
  enhanced jigsaw
}
\usepackage{tabularx}
\usepackage{array}
\usepackage{ragged2e}

% --- Readability helpers for cards ---
\newcolumntype{L}[1]{>{\raggedleft\arraybackslash\bfseries}p{#1}}
\newcolumntype{Y}{>{\RaggedRight\arraybackslash}X}
\newtcbox{\pill}{on line, arc=3pt, boxsep=0.8pt, left=4pt,right=4pt,top=1pt,bottom=1pt,
  colframe=gray!50, colback=gray!15, boxrule=0.3pt}
\newcommand{\badge}[1]{\pill{\footnotesize #1}}

% --- Shortcuts/labels ---
% Make checkbox robust in moving arguments (list labels, bookmarks)
\newcommand{\cb}{\ensuremath{\square}}
\newcommand{\DoR}{\textbf{Definition of Ready:} Persona clear; AC drafted; Dependencies known; Estimate set.}
\newcommand{\DoD}{\textbf{Definition of Done:} All ACs pass; Tests green; Security/a11y checks; Docs updated; Deployed/flagged.}

% Priority styling (optional emphasis if you use Must/Should/Could)
\newcommand{\Priority}[1]{\textbf{Priority:} #1}

% --- Story Card: same 9-arg signature, optimized readability ---
% 1: ID   2: Title   3: Epic/Feature   4: Business Value
% 5: Priority   6: Estimate(SP)   7: Persona   8: Dependencies   9: Assumptions/Risks
\newcommand{\StoryCard}[9]{%
  \newpage
  \begin{tcolorbox}[
    enhanced, breakable,
    colback=gray!2, colframe=gray!50, arc=2pt, boxrule=0.4pt,
    left=8pt,right=8pt,top=8pt,bottom=8pt,
    fonttitle=\bfseries\large,
    title={\textbf{#1}\ \textemdash\ #2},
    colbacktitle=gray!6, coltitle=black,
    borderline west={2pt}{0pt}{MidnightBlue}
  ]
  \small
  \begin{tabularx}{\textwidth}{@{}L{3.2cm}Y@{}}
    Epic / Feature          & #3 \\
    Business Value          & #4 \\
    Priority / Estimate     & \badge{Priority: #5}\ \badge{SP: #6} \\
    Persona                 & #7 \\
    Dependencies            & #8 \\
    Assumptions / Risks     & #9 \\
  \end{tabularx}

  \medskip
  \textbf{Story}\quad
  \emph{As a #7, I want to #2 so that #4.}

  \medskip
  \textbf{Non-Functional}\quad
  \badge{Performance}\ \badge{Security}\ \badge{Reliability}\ \badge{Accessibility}\ \badge{Privacy}\ \badge{i18n}

  \medskip
  \textbf{Acceptance Criteria (BDD)}
  \begin{description}[leftmargin=2.4cm, labelwidth=2.3cm, style=nextline, itemsep=2pt, topsep=2pt]
    \item[\textbf{Scenario}] Happy path
    \item[\textbf{Given}] the target repository and pipeline configuration are available
    \item[\textbf{When}] the user completes the \emph{Hands-on Objective} for this chapter
    \item[\textbf{Then}] the stated \emph{Outcome} for this chapter is observable and recorded in the pipeline/job summary
  \end{description}

  \vspace{0.2\baselineskip}
  {\footnotesize\color{gray!60}\DoR\ \textbullet\ \DoD}
  \end{tcolorbox}
}

% --- Lightweight Tasks box for every story ---
% Usage: \begin{TasksBox} ... \end{TasksBox}
%        \begin{TasksBox}[Pre-Deployment Tasks] ... \end{TasksBox}  % optional custom title
\newenvironment{TasksBox}[1][Tasks]{%
  \begin{tcolorbox}[
    enhanced,breakable,
    colback=gray!1, colframe=gray!35,
    colbacktitle=gray!6, coltitle=black,
    title={#1}, fonttitle=\bfseries,
    borderline west={1.8pt}{0pt}{MidnightBlue},
    arc=2pt, boxrule=0.4pt,
    left=10pt,right=10pt,top=6pt,bottom=6pt,
    before skip=6pt, after skip=10pt
  ]
  \small
  \begin{itemize}[
    label=\cb,           % uses your existing \cb = square checkbox
    leftmargin=*,        % full-width hanging indent for wrapped lines
    labelsep=0.6em,      % space between checkbox and text
    itemsep=4pt,         % vertical breathing room between items
    topsep=2pt, parsep=0pt
  ]
}{%
  \end{itemize}
  \end{tcolorbox}
}


% --- Title ---
\title{User Stories by Chapter:\\ CI \texorpdfstring{$\rightarrow$}{→} CD \texorpdfstring{$\rightarrow$}{→} GitHub Actions}
\author{Compiled for Jordan Suber}
\date{}

\begin{document}
\section{End-to-End Development Workflow — Combined User Stories}

Below are all user stories from the chapters, reorganized into a single, execution-ready workflow.


\subsection{Additional Stories}

% Stories that were not part of the predefined PASTA/Threat-Model flow

\StoryCard{CI-1}{Getting Started}{Continuous Integration}
{Establish shared understanding of CI, fast feedback, and “keep main green” to reduce integration risk}
{Must}{3}
{developer on a new repo}
{Build tooling, unit test framework}
{If build toolchain differs locally vs CI, setup time may expand; risk of flaky initial test}
\begin{TasksBox}
  \item \cb Initialize repo with build tool (\texttt{npm/mvn/gradle}) and a \texttt{hello\_world} unit test.
  \item \cb Create \texttt{ci.yml} with triggers on \texttt{push}/\texttt{pull\_request} to \texttt{main}.
  \item \cb Add steps: checkout, setup toolchain, install deps, build, run tests, upload test report.
  \item \cb Enable required status checks on \texttt{main}; protect branch with fast-forward or merge queue.
  \item \cb Add CI badge to \texttt{README.md}; document “keep main green” policy.
\end{TasksBox}


\StoryCard{CI-2}{Introducing CI}{Continuous Integration}
{Codify team habits (frequent commits, fix red builds) to increase flow efficiency}
{Must}{2}
{team contributor}
{Branch protection supported in VCS}
{Cultural adoption risk; enforce via required checks and stop-the-line policy}
\begin{TasksBox}
  \item \cb Define commit/PR guidelines (small PRs, issue links, naming).
  \item \cb Configure required reviews and required CI checks for \texttt{main}.
  \item \cb Add CODEOWNERS for critical paths; auto-request reviewers.
  \item \cb Create an on-call rotation to “stop the line” on red builds; document SLA.
  \item \cb Add PR template with checklist (tests, docs, security notes).
\end{TasksBox}


\StoryCard{CI-3}{Reducing Risks Using CI}{Continuous Integration}
{Artifacting and visible quality metrics reduce late defects and deployment surprises}
{Must}{3}
{release engineer}
{Artifact storage and coverage tooling}
{Coverage thresholds may fail initially; iterate thresholds upward}
\begin{TasksBox}
  \item \cb Publish build artifacts (packages/bundles) to artifact store; retain for 30 days.
  \item \cb Generate coverage report; upload as artifact and comment summary on PR.
  \item \cb Add static analysis (lint/type-check) and fail on error.
  \item \cb Gate merges with a minimum coverage threshold; start low, ratchet weekly.
  \item \cb Produce a build manifest (commit, version, artifact SHA) and attach to job summary.
\end{TasksBox}


\StoryCard{CI-4}{Build at Every Change}{Continuous Integration}
{Fast, repeatable builds shorten feedback loops and improve developer throughput}
{Must}{5}
{build engineer}
{Cache support, job matrix}
{SLA breach risk if dependencies uncached; add caching and stage split}
\begin{TasksBox}
  \item \cb Add dependency cache with robust keys and restore-keys.
  \item \cb Split jobs (lint, unit, build) to run in parallel.
  \item \cb Add language/runtime matrix (e.g., Node LTS{-}1, LTS, latest).
  \item \cb Set max commit-stage duration target (e.g., \(\leq\) 10 minutes); alert on regressions.
  \item \cb Capture build scan/timing metrics; post to job summary.
\end{TasksBox}


\StoryCard{CI-5}{Continuous Database Integration}{Continuous Integration}
{Versioned migrations prevent schema drift and enable safe evolution}
{Must}{5}
{backend developer}
{DB service in CI, migration tool}
{Migration ordering conflicts; use sandbox DB and rollback scripts}
\begin{TasksBox}
  \item \cb Add migration tool (\texttt{Flyway}/\texttt{Liquibase}/\texttt{Prisma}) to repo.
  \item \cb Provision ephemeral CI database service; apply clean schema on each run.
  \item \cb Implement forward migration and matching rollback script.
  \item \cb Seed minimal test data; run DB tests after migrations.
  \item \cb Upload migration logs and DB schema diff as artifacts.
\end{TasksBox}


\StoryCard{CI-6}{Continuous Testing}{Continuous Integration}
{Layered tests (unit$\rightarrow$component$\rightarrow$system) increase confidence with speed}
{Must}{5}
{QA engineer}
{Test categorization, runners}
{Flaky tests create noise; quarantine and deflake policy}
\begin{TasksBox}
  \item \cb Tag tests by layer (\texttt{@unit}, \texttt{@component}, \texttt{@system}); wire selective runners.
  \item \cb Run unit tests on every push; run slower suites on PR or schedule.
  \item \cb Fail fast on test flakiness; auto-quarantine and create issue.
  \item \cb Collect JUnit/HTML reports and screenshots/videos for failures.
  \item \cb Track pass rate and flake rate trends in job summaries.
\end{TasksBox}


\StoryCard{CI-7}{Continuous Inspection}{Continuous Integration}
{Automated inspection (lint, SAST, coverage) raises baseline quality}
{Should}{3}
{security champion}
{Linters, SAST scanner}
{Initial findings may be high; add waivers and remediation backlog}
\begin{TasksBox}
  \item \cb Add linters/formatters to CI (\texttt{eslint}, \texttt{black}, \texttt{golangci-lint}, etc.).
  \item \cb Enable SAST/SCA scans; upload SARIF to code scanning.
  \item \cb Establish allowlist/waiver mechanism with expiry dates.
  \item \cb Fail on new high/critical issues; summarize in PR.
  \item \cb Schedule weekly full scan job; export trend report.
\end{TasksBox}


\StoryCard{CI-8}{Continuous Deployment (Intro)}{Continuous Integration}
{Versioned packages and rollback scripts de-risk promotions}
{Should}{3}
{release manager}
{Registry access; signing keys}
{Rollback untested; include simulated rollback in staging}
\begin{TasksBox}
  \item \cb Implement semantic versioning and changelog generation.
  \item \cb Sign artifacts/images; push to registry with provenance.
  \item \cb Create staging deploy script; add \texttt{--rollback} path.
  \item \cb Run canary or blue/green simulation in staging; record outcome.
  \item \cb Document release/rollback steps in \texttt{RELEASE.md}.
\end{TasksBox}


\StoryCard{CI-9}{Continuous Feedback}{Continuous Integration}
{Visible signals (badges, PR summaries, alerts) accelerate fixing time}
{Should}{2}
{team lead}
{Chat/webhook integration}
{Alert fatigue risk; tune thresholds and channels}
\begin{TasksBox}
  \item \cb Add CI status/coverage badges to \texttt{README.md}.
  \item \cb Emit concise job summaries (key metrics, links to artifacts).
  \item \cb Integrate notifications to chat with failure-only or noisy-channel rules.
  \item \cb Create “First failure owner” routing; page on red builds during business hours.
  \item \cb Add post-merge dashboard (lead time, pass rate).
\end{TasksBox}

\section{Continuous Delivery (Humble \& Farley) — Part I: Foundations}


\StoryCard{CD-1}{The Problem of Delivering Software}{Continuous Delivery}
{Baseline DORA metrics (lead time, CFR, MTTR) to focus improvement}
{Must}{3}
{product owner}
{Metrics capture in pipeline}
{Metric definitions inconsistent; document glossary and method}
\begin{TasksBox}
  \item \cb Define metric glossary (lead time, CFR, MTTR, deploy frequency).
  \item \cb Capture deployment events and commit timestamps in CI/CD.
  \item \cb Compute metrics in a scheduled job; publish to dashboard.
  \item \cb Add goals and alerts for regressions; include in retro.
  \item \cb Document data sources and caveats in \texttt{docs/metrics.md}.
\end{TasksBox}


\StoryCard{CD-2}{Configuration Management}{Continuous Delivery}
{Immutable builds \& SBOM improve traceability and compliance}
{Must}{5}
{release engineer}
{SBOM tool, artifact store}
{Dependency metadata gaps; pin versions and generate SBOM}
\begin{TasksBox}
  \item \cb Pin base images and dependencies; lockfiles committed.
  \item \cb Generate SBOM (e.g., CycloneDX/SPDX) during build; attach to artifact.
  \item \cb Store artifacts immutably with content address (SHA).
  \item \cb Verify signatures and SBOM presence in release gate.
  \item \cb Record build provenance (builder, inputs) in job summary.
\end{TasksBox}


\StoryCard{CD-3}{Continuous Integration (Bridge)}{Continuous Delivery}
{Align CI policies with CD (small PRs, reviews) to sustain flow}
{Must}{2}
{tech lead}
{Merge rules in VCS}
{Merge queue learning curve; document examples}
\begin{TasksBox}
  \item \cb Enforce PR size checks (warn on \textgreater{} 400 LOC changed).
  \item \cb Require review from codeowners on critical areas.
  \item \cb Enable merge queue/linear history; document usage.
  \item \cb Add examples of good PRs and review checklist.
\end{TasksBox}


\StoryCard{CD-4}{Testing Strategy}{Continuous Delivery}
{Acceptance tests as specification reduce rework}
{Must}{5}
{QA lead}
{AAT framework, test data}
{Environment instability; use ephemeral envs}
\begin{TasksBox}
  \item \cb Stand up acceptance test framework (e.g., Playwright/Cypress/Behave).
  \item \cb Define acceptance criteria as executable specs; tag \texttt{@acceptance}.
  \item \cb Provision ephemeral test env per PR with seeded data.
  \item \cb Capture screenshots/video and HAR on failure.
  \item \cb Publish acceptance report and link from job summary.
\end{TasksBox}

\section{Continuous Delivery — Part II: The Deployment Pipeline}


\StoryCard{CD-5}{Anatomy of the Deployment Pipeline}{Continuous Delivery}
{Explicit stages and gates clarify artifact flow and ownership}
{Must}{3}
{platform engineer}
{Pipeline-as-code}
{Diagram-policy drift; keep diagram in-repo}
\begin{TasksBox}
  \item \cb Model stages (commit \(\rightarrow\) acceptance \(\rightarrow\) NFR \(\rightarrow\) staging).
  \item \cb Define promotion gates and required evidence per stage.
  \item \cb Generate pipeline diagram and commit to \texttt{docs/pipeline.png}.
  \item \cb Add ownership and on-call mapping to each stage.
  \item \cb Validate artifact handoffs with a dry-run.
\end{TasksBox}


\StoryCard{CD-6}{Build \& Deployment Scripting}{Continuous Delivery}
{Idempotent, one-command deploys remove manual error}
{Must}{5}
{devops engineer}
{IaC module, deploy script}
{Hidden manual steps; remove or automate}
\begin{TasksBox}
  \item \cb Create idempotent deploy script (\texttt{deploy.sh}/Make target) with \texttt{--dry-run}.
  \item \cb Externalize config/env via parameters or env files; no secrets in code.
  \item \cb Integrate infra provisioning via IaC module; validate plan before apply.
  \item \cb Add health checks and post-deploy verification step.
  \item \cb Document rollback procedure and test it in staging.
\end{TasksBox}


\StoryCard{CD-7}{The Commit Stage}{Continuous Delivery}
{Sub-10-minute commit stage preserves rapid feedback}
{Must}{3}
{build engineer}
{Caching, parallelization}
{SLA risk; monitor and enforce guardrails}
\begin{TasksBox}
  \item \cb Profile commit stage; identify slowest steps.
  \item \cb Parallelize independent tasks; add caching for deps/builds.
  \item \cb Defer integration/system tests to later stages.
  \item \cb Enforce time budget; fail builds exceeding threshold with guidance.
  \item \cb Track median/p95 duration; surface trend.
\end{TasksBox}


\StoryCard{CD-8}{Automated Acceptance Testing}{Continuous Delivery}
{Reliable AAT improves release confidence}
{Should}{5}
{QA engineer}
{Ephemeral env, seed data}
{Flaky env; capture videos/screenshots for triage}
\begin{TasksBox}
  \item \cb Provision ephemeral env per PR with deterministic seed data.
  \item \cb Run AAT in parallel shards; retry once on failure with quarantine tag.
  \item \cb Collect artifacts (videos, traces); upload for triage.
  \item \cb Block promotion on failing critical AAT scenarios.
  \item \cb Track AAT pass rate by suite/component.
\end{TasksBox}


\StoryCard{CD-9}{Testing Nonfunctional Requirements}{Continuous Delivery}
{Perf/security smoke gates catch regressions early}
{Should}{5}
{SRE / Sec Eng}
{k6, SCA/SAST}
{False positives; thresholds and baselines required}
\begin{TasksBox}
  \item \cb Add quick perf smoke (e.g., k6 1–3 min) against staging; alert on \% change vs baseline.
  \item \cb Run SCA/SAST; fail on new high/critical.
  \item \cb Capture latency/throughput/error-rate; publish trend.
  \item \cb Define thresholds and suppression policy with expiry.
  \item \cb Include dependency vulnerability diff in job summary.
\end{TasksBox}


\StoryCard{CD-10}{Deploying and Releasing Applications}{Continuous Delivery}
{Feature flags and canaries decouple deploy from release}
{Should}{5}
{release manager}
{Flag service}
{Flag debt; add cleanup policy}
\begin{TasksBox}
  \item \cb Integrate feature flag SDK; default flags off on deploy.
  \item \cb Create canary strategy (small \% traffic); monitor key metrics.
  \item \cb Add progressive rollout workflow with manual approval step.
  \item \cb Record flag change events alongside deploy events.
  \item \cb Establish flag cleanup SLA; add linter to detect stale flags.
\end{TasksBox}

\section{Continuous Delivery — Part III: The Delivery Ecosystem}


\StoryCard{CD-11}{Managing Infrastructure and Environments}{Continuous Delivery}
{IaC with parity reduces “works-on-my-machine” failures}
{Must}{5}
{platform engineer}
{IaC repo, drift detection}
{Version drift; pin images and modules}
\begin{TasksBox}
  \item \cb Create environment definitions (dev/stage/prod) via IaC modules.
  \item \cb Pin base images and module versions; document upgrade path.
  \item \cb Enable drift detection; alert on config divergence.
  \item \cb Bake golden images; rotate regularly.
  \item \cb Add smoke checks per environment; publish status page.
\end{TasksBox}


\StoryCard{CD-12}{Managing Data}{Continuous Delivery}
{Expand/contract enables zero-downtime DB changes}
{Must}{5}
{database engineer}
{Migration plan}
{Backward-compat assumptions; test dual-write/dual-read}
\begin{TasksBox}
  \item \cb Design expand/contract plan; add nullable columns/dual paths.
  \item \cb Implement dual-write; monitor divergence.
  \item \cb Backfill data safely (batch/ throttled).
  \item \cb Switch reads; verify; then contract and remove legacy.
  \item \cb Add rollback guardrails and validation queries.
\end{TasksBox}


\StoryCard{CD-13}{Managing Components \& Dependencies}{Continuous Delivery}
{Contracts + pinning stabilize integrations}
{Should}{5}
{service owner}
{CDC tests, dep scan}
{API breakage; run CDC in PRs}
\begin{TasksBox}
  \item \cb Define API contracts (OpenAPI/AsyncAPI) and publish.
  \item \cb Implement consumer-driven contract tests in CI.
  \item \cb Pin external service versions/clients; use compat ranges.
  \item \cb Enable dep update bot; require CI green + contract pass.
  \item \cb Add chaos test for dependency outage fallback.
\end{TasksBox}


\StoryCard{CD-14}{Advanced Version Control}{Continuous Delivery}
{Trunk or short-lived branches scale delivery cadence}
{Should}{3}
{tech lead}
{Policy doc}
{Policy drift; include examples \& bots}
\begin{TasksBox}
  \item \cb Adopt trunk-based policy; enforce short-lived branches (\textless{} 2 days).
  \item \cb Enable pre-commit hooks (format/lint/test).
  \item \cb Configure merge queue and auto-rebase.
  \item \cb Provide exemplars of small PRs and review heuristics.
\end{TasksBox}


\StoryCard{CD-15}{Managing Continuous Delivery}{Continuous Delivery}
{Automated audit trails satisfy governance/compliance}
{Must}{3}
{compliance owner}
{Change log export}
{Gaps in metadata; enrich job summaries}
\begin{TasksBox}
  \item \cb Capture change metadata (who/what/when/why) in job outputs.
  \item \cb Export audit log to central store; retain per policy.
  \item \cb Generate release notes automatically from commits/PRs.
  \item \cb Add compliance checklist to PR template.
  \item \cb Schedule periodic audit report generation.
\end{TasksBox}

\section{Learning GitHub Actions (Brent Laster) — Part I: Foundations}


\StoryCard{GA-1}{The Basics}{GitHub Actions Platform}
{Working CI workflow standardizes contribution checks}
{Must}{2}
{repo maintainer}
{ci.yml}
{Badge visibility; README update needed}
\begin{TasksBox}
  \item \cb Create \texttt{.github/workflows/ci.yml} with checkout, setup, build, test.
  \item \cb Add matrix for OS/runtime if relevant.
  \item \cb Upload test reports and coverage artifacts.
  \item \cb Add build badge to \texttt{README.md}; link to Actions tab.
\end{TasksBox}


\StoryCard{GA-2}{How Actions Work}{GitHub Actions Platform}
{Multi-job workflows with conditionals enable clear stage flow}
{Must}{3}
{automation engineer}
{needs/if usage}
{Over-complex logic; keep expressions simple}
\begin{TasksBox}
  \item \cb Create multi-job workflow (lint \(\rightarrow\) test \(\rightarrow\) build).
  \item \cb Use \texttt{needs:} to express dependencies; add \texttt{if:} conditionals for paths/labels.
  \item \cb Share artifacts between jobs; consume in downstream steps.
  \item \cb Demonstrate manual re-run and \texttt{workflow\_dispatch}.
\end{TasksBox}


\StoryCard{GA-3}{What’s in an Action?}{GitHub Actions Platform}
{Local composite action reduces duplication across repos}
{Should}{3}
{tooling engineer}
{.github/actions/lint}
{Versioning; tag action revisions}
\begin{TasksBox}
  \item \cb Create \texttt{.github/actions/lint/action.yml} composite action.
  \item \cb Parameterize inputs (paths, config file).
  \item \cb Replace duplicated steps in workflows with the composite action.
  \item \cb Tag versions and document usage in \texttt{README}.
\end{TasksBox}


\StoryCard{GA-4}{Working with Workflows}{GitHub Actions Platform}
{Manual dispatch with parameters supports ops tasks}
{Should}{2}
{ops engineer}
{workflow\_dispatch}
{Undocumented inputs; README examples required}
\begin{TasksBox}
  \item \cb Add \texttt{workflow\_dispatch} with typed inputs (env, version).
  \item \cb Validate inputs; fail with helpful messages.
  \item \cb Document runbook and examples in \texttt{README}.
  \item \cb Restrict to maintainers with environment protections.
\end{TasksBox}


\StoryCard{GA-5}{Runners}{GitHub Actions Platform}
{Self-hosted runners unlock custom environments and scale}
{Could}{5}
{platform engineer}
{Runner labels}
{Security posture; isolate and rotate tokens}
\begin{TasksBox}
  \item \cb Provision self-hosted runner VM/container; apply labels.
  \item \cb Configure ephemeral runners; auto-scale pool.
  \item \cb Lock down network and credentials; rotate tokens.
  \item \cb Route heavy jobs to self-hosted via \texttt{runs-on: [self-hosted, label]}.
  \item \cb Monitor utilization and queue times; auto-scale thresholds.
\end{TasksBox}

\section{GitHub Actions — Part II: Building Blocks}


\StoryCard{GA-6}{Managing Workflow Environments}{GitHub Actions Platform}
{Protected environments and OIDC harden deployments}
{Must}{3}
{security champion}
{Env rules, OIDC}
{Reviewer fatigue; configure sensible wait timers}
\begin{TasksBox}
  \item \cb Create environments (staging, prod) with required reviewers and wait timers.
  \item \cb Set \texttt{permissions: read-all} at top; elevate per job minimally.
  \item \cb Configure OIDC trust with cloud provider; remove long-lived secrets.
  \item \cb Use \texttt{environment:} and \texttt{protection\_rules} in deploy jobs.
\end{TasksBox}


\StoryCard{GA-7}{Managing Data Within Workflows}{GitHub Actions Platform}
{Caching \& artifacts speed builds and preserve outputs}
{Should}{3}
{build engineer}
{cache/action, artifacts}
{Stale cache; use keys and restore-keys wisely}
\begin{TasksBox}
  \item \cb Add \texttt{actions/cache} with precise keys and restore-keys.
  \item \cb Upload key build outputs as artifacts; set retention days.
  \item \cb Use \texttt{actions/upload-artifact} for logs/reports.
  \item \cb Periodically bust caches on lockfile change.
\end{TasksBox}


\StoryCard{GA-8}{Managing Workflow Execution}{GitHub Actions Platform}
{Matrices \& concurrency optimize throughput}
{Should}{3}
{automation engineer}
{matrix, concurrency.group}
{Overlapping runs; group naming convention}
\begin{TasksBox}
  \item \cb Add job matrix (os/runtime/db) with fail-fast disabled when appropriate.
  \item \cb Configure \texttt{concurrency: group} with \texttt{cancel-in-progress: true}.
  \item \cb Use \texttt{paths}/\texttt{paths-ignore} to scope triggers.
  \item \cb Surface matrix results in summary table.
\end{TasksBox}

\section{GitHub Actions — Part III: Security and Monitoring}


\StoryCard{GA-9}{Actions and Security}{GitHub Actions Platform}
{Least-privilege and fork hardening reduce supply-chain risk}
{Must}{3}
{security champion}
{permissions: read-all, policy}
{Breakages due to perms; document elevation steps}
\begin{TasksBox}
  \item \cb Set default \texttt{permissions: contents: read}; elevate per job.
  \item \cb Restrict \texttt{pull\_request} from forks from accessing secrets; use \texttt{pull\_request\_target} safely if needed.
  \item \cb Pin third-party actions by commit SHA; audit monthly.
  \item \cb Enable Dependabot security updates for actions and packages.
  \item \cb Add policy checks for disallowed actions and secret scanning.
\end{TasksBox}


\StoryCard{GA-10}{Monitoring, Logging, and Debugging}{GitHub Actions Platform}
{Job summaries improve triage speed and observability}
{Should}{2}
{dev lead}
{summary markdown}
{Noisy logs; link to artifacts and coverage}
\begin{TasksBox}
  \item \cb Write concise \texttt{\{\{ github.step\_summary \}\}} markdown (key metrics, links).
  \item \cb Enable step-level \texttt{if: always()} logs on failure; upload debug bundle.
  \item \cb Add \texttt{ACTIONS\_STEP\_DEBUG} toggle via secrets for deep dives.
  \item \cb Include permalinks to artifacts, coverage, and failing tests.
\end{TasksBox}

\section{GitHub Actions — Part IV: Advanced Topics}


\StoryCard{GA-11}{Creating Custom Actions}{GitHub Actions Platform}
{Versioned public action promotes reuse org-wide}
{Could}{5}
{tooling maintainer}
{JS action repo}
{API changes; semantic versioning policy}
\begin{TasksBox}
  \item \cb Scaffold JS action (\texttt{action.yml}, \texttt{src/}, \texttt{dist/}); commit built assets.
  \item \cb Add inputs/outputs with validation and error handling.
  \item \cb Write unit tests and an example workflow.
  \item \cb Publish \texttt{v1} tag and release notes; maintain \texttt{v1} major tag.
  \item \cb Set up automated release via \texttt{release-please}/semantic-release.
\end{TasksBox}


\StoryCard{GA-12}{Advanced Workflows}{GitHub Actions Platform}
{Reusable workflows enforce governance and DRY}
{Must}{3}
{org admin}
{caller workflows}
{Adoption lag; publish examples and starter kits}
\begin{TasksBox}
  \item \cb Create reusable workflow triggered by \texttt{workflow\_call} with inputs/secrets.
  \item \cb Migrate two repos to use the reusable workflow.
  \item \cb Add org-level examples and starter templates.
  \item \cb Version reusable workflows; document breaking changes.
\end{TasksBox}


\StoryCard{GA-13}{Advanced Workflow Techniques}{GitHub Actions Platform}
{Containerized jobs and services enable realistic CI envs}
{Should}{5}
{integration engineer}
{services:, gh CLI}
{Resource limits; right-size service containers}
\begin{TasksBox}
  \item \cb Run job in a container (\texttt{container:}) with required tools.
  \item \cb Add \texttt{services:} (db/cache); wait-for-health before tests.
  \item \cb Use \texttt{gh} CLI for release/promo tasks; store token via OIDC.
  \item \cb Tune resource limits and cleanup steps to avoid leaks.
\end{TasksBox}


\StoryCard{GA-14}{Migrating to GitHub Actions}{GitHub Actions Platform}
{Parity migration de-risks CI platform change}
{Must}{5}
{platform owner}
{Parity checklist}
{Hidden gaps; track transformers and exceptions}
\begin{TasksBox}
  \item \cb Build a parity checklist (triggers, envs, secrets, artifacts, approvals).
  \item \cb Port one pipeline end-to-end; compare outputs/timings.
  \item \cb Implement transformers for env variables and workspace semantics.
  \item \cb Run dual CI for a sprint; fix gaps; cut over with rollback plan.
  \item \cb Document exceptions and future optimizations.
\end{TasksBox}

\section*{Capstone \& Milestones (Reference)}
CI Milestone: fast/complete tiers, layered tests, inspections, DB migrations, artifacts, manual promote + rollback.\\
CD Milestone: commit $\rightarrow$ acceptance $\rightarrow$ NFR $\rightarrow$ staging, scripted deploys, flags, audit trail.\\
GHA Milestone: reusable workflows, protected envs, least-privilege defaults, custom action, migration playbook.


\newpage
\section*{Appendix — Original Chapters}

\maketitle
\tableofcontents
\newpage

\section*{How This Set Was Built}
Each story maps one chapter’s \emph{Learning Goals} to a concise user story, uses the chapter’s \emph{Hands-on Objective} as the operative behavior, and verifies the published \emph{Outcome}. Stories are intentionally compact (per INVEST) and ready for backlog import. (Source study plan and template referenced externally.)

\section{Continuous Integration (Duvall, Matyas, Glover)}

\StoryCard{CI-1}{Getting Started}{Continuous Integration}
{Establish shared understanding of CI, fast feedback, and “keep main green” to reduce integration risk}
{Must}{3}
{developer on a new repo}
{Build tooling, unit test framework}
{If build toolchain differs locally vs CI, setup time may expand; risk of flaky initial test}
\begin{TasksBox}
  \item \cb Initialize repo with build tool (\texttt{npm/mvn/gradle}) and a \texttt{hello\_world} unit test.
  \item \cb Create \texttt{ci.yml} with triggers on \texttt{push}/\texttt{pull\_request} to \texttt{main}.
  \item \cb Add steps: checkout, setup toolchain, install deps, build, run tests, upload test report.
  \item \cb Enable required status checks on \texttt{main}; protect branch with fast-forward or merge queue.
  \item \cb Add CI badge to \texttt{README.md}; document “keep main green” policy.
\end{TasksBox}

\StoryCard{CI-2}{Introducing CI}{Continuous Integration}
{Codify team habits (frequent commits, fix red builds) to increase flow efficiency}
{Must}{2}
{team contributor}
{Branch protection supported in VCS}
{Cultural adoption risk; enforce via required checks and stop-the-line policy}
\begin{TasksBox}
  \item \cb Define commit/PR guidelines (small PRs, issue links, naming).
  \item \cb Configure required reviews and required CI checks for \texttt{main}.
  \item \cb Add CODEOWNERS for critical paths; auto-request reviewers.
  \item \cb Create an on-call rotation to “stop the line” on red builds; document SLA.
  \item \cb Add PR template with checklist (tests, docs, security notes).
\end{TasksBox}

\StoryCard{CI-3}{Reducing Risks Using CI}{Continuous Integration}
{Artifacting and visible quality metrics reduce late defects and deployment surprises}
{Must}{3}
{release engineer}
{Artifact storage and coverage tooling}
{Coverage thresholds may fail initially; iterate thresholds upward}
\begin{TasksBox}
  \item \cb Publish build artifacts (packages/bundles) to artifact store; retain for 30 days.
  \item \cb Generate coverage report; upload as artifact and comment summary on PR.
  \item \cb Add static analysis (lint/type-check) and fail on error.
  \item \cb Gate merges with a minimum coverage threshold; start low, ratchet weekly.
  \item \cb Produce a build manifest (commit, version, artifact SHA) and attach to job summary.
\end{TasksBox}

\StoryCard{CI-4}{Build at Every Change}{Continuous Integration}
{Fast, repeatable builds shorten feedback loops and improve developer throughput}
{Must}{5}
{build engineer}
{Cache support, job matrix}
{SLA breach risk if dependencies uncached; add caching and stage split}
\begin{TasksBox}
  \item \cb Add dependency cache with robust keys and restore-keys.
  \item \cb Split jobs (lint, unit, build) to run in parallel.
  \item \cb Add language/runtime matrix (e.g., Node LTS{-}1, LTS, latest).
  \item \cb Set max commit-stage duration target (e.g., \(\leq\) 10 minutes); alert on regressions.
  \item \cb Capture build scan/timing metrics; post to job summary.
\end{TasksBox}

\StoryCard{CI-5}{Continuous Database Integration}{Continuous Integration}
{Versioned migrations prevent schema drift and enable safe evolution}
{Must}{5}
{backend developer}
{DB service in CI, migration tool}
{Migration ordering conflicts; use sandbox DB and rollback scripts}
\begin{TasksBox}
  \item \cb Add migration tool (\texttt{Flyway}/\texttt{Liquibase}/\texttt{Prisma}) to repo.
  \item \cb Provision ephemeral CI database service; apply clean schema on each run.
  \item \cb Implement forward migration and matching rollback script.
  \item \cb Seed minimal test data; run DB tests after migrations.
  \item \cb Upload migration logs and DB schema diff as artifacts.
\end{TasksBox}

\StoryCard{CI-6}{Continuous Testing}{Continuous Integration}
{Layered tests (unit$\rightarrow$component$\rightarrow$system) increase confidence with speed}
{Must}{5}
{QA engineer}
{Test categorization, runners}
{Flaky tests create noise; quarantine and deflake policy}
\begin{TasksBox}
  \item \cb Tag tests by layer (\texttt{@unit}, \texttt{@component}, \texttt{@system}); wire selective runners.
  \item \cb Run unit tests on every push; run slower suites on PR or schedule.
  \item \cb Fail fast on test flakiness; auto-quarantine and create issue.
  \item \cb Collect JUnit/HTML reports and screenshots/videos for failures.
  \item \cb Track pass rate and flake rate trends in job summaries.
\end{TasksBox}

\StoryCard{CI-7}{Continuous Inspection}{Continuous Integration}
{Automated inspection (lint, SAST, coverage) raises baseline quality}
{Should}{3}
{security champion}
{Linters, SAST scanner}
{Initial findings may be high; add waivers and remediation backlog}
\begin{TasksBox}
  \item \cb Add linters/formatters to CI (\texttt{eslint}, \texttt{black}, \texttt{golangci-lint}, etc.).
  \item \cb Enable SAST/SCA scans; upload SARIF to code scanning.
  \item \cb Establish allowlist/waiver mechanism with expiry dates.
  \item \cb Fail on new high/critical issues; summarize in PR.
  \item \cb Schedule weekly full scan job; export trend report.
\end{TasksBox}

\StoryCard{CI-8}{Continuous Deployment (Intro)}{Continuous Integration}
{Versioned packages and rollback scripts de-risk promotions}
{Should}{3}
{release manager}
{Registry access; signing keys}
{Rollback untested; include simulated rollback in staging}
\begin{TasksBox}
  \item \cb Implement semantic versioning and changelog generation.
  \item \cb Sign artifacts/images; push to registry with provenance.
  \item \cb Create staging deploy script; add \texttt{--rollback} path.
  \item \cb Run canary or blue/green simulation in staging; record outcome.
  \item \cb Document release/rollback steps in \texttt{RELEASE.md}.
\end{TasksBox}

\StoryCard{CI-9}{Continuous Feedback}{Continuous Integration}
{Visible signals (badges, PR summaries, alerts) accelerate fixing time}
{Should}{2}
{team lead}
{Chat/webhook integration}
{Alert fatigue risk; tune thresholds and channels}
\begin{TasksBox}
  \item \cb Add CI status/coverage badges to \texttt{README.md}.
  \item \cb Emit concise job summaries (key metrics, links to artifacts).
  \item \cb Integrate notifications to chat with failure-only or noisy-channel rules.
  \item \cb Create “First failure owner” routing; page on red builds during business hours.
  \item \cb Add post-merge dashboard (lead time, pass rate).
\end{TasksBox}

\section*{Capstone \& Milestones (Reference)}
CI Milestone: fast/complete tiers, layered tests, inspections, DB migrations, artifacts, manual promote + rollback.\\
CD Milestone: commit $\rightarrow$ acceptance $\rightarrow$ NFR $\rightarrow$ staging, scripted deploys, flags, audit trail.\\
GHA Milestone: reusable workflows, protected envs, least-privilege defaults, custom action, migration playbook.




% ==============================
% Tailored: Emscripten + WebGL2 + WASM  (DROP-IN FIX: uses StoryCard + TasksBox only)
% ==============================
\section*{Tailored User Stories: Emscripten + WebGL2 + WASM}

\StoryCard{A1}{Set up Emscripten SDK \& CMake}{Build Toolchain \& Project Scaffolding}
{Compile C/C++ to WebAssembly reproducibly with one command}
{Must}{3}
{developer}
{emsdk; CMake; shell}
{Version drift across dev/CI; pin SDK}
\begin{TasksBox}
  \item \cb Install and activate \texttt{emsdk}; pin a known-good version in \texttt{README} and CI.
  \item \cb Provide \texttt{toolchain.cmake}; document \texttt{emcmake}/\texttt{emmake} usage.
  \item \cb Verify \texttt{emcc} and \texttt{emar} on \texttt{PATH}; emit to \texttt{build-wasm/}.
  \item \cb \textbf{AC:} Running \texttt{emcmake cmake} then \texttt{emmake make -j} emits \texttt{.html/.js/.wasm} without errors.
\end{TasksBox}

\StoryCard{A2}{Build Profiles (Debug/Release)}{Build Toolchain \& Project Scaffolding}
{Switch quickly between fast iteration and optimized output}
{Must}{2}
{developer}
{CMakePresets.json}
{Wrong flags in Release can hurt perf}
\begin{TasksBox}
  \item \cb Add \texttt{CMakePresets.json}: Debug (\texttt{-O0 -g -sASSERTIONS=1 -sSAFE\_HEAP=1}), Release (\texttt{-O3 -flto -sASSERTIONS=0}).
  \item \cb Document preset usage in \texttt{README}.
  \item \cb \textbf{AC:} \texttt{-DCMAKE\_BUILD\_TYPE=Debug|Release} toggles symbols/optimizations; Release bundle is smaller.
\end{TasksBox}

\StoryCard{B1}{WebGL2 Context Defaults}{WebGL2 Context \& Canvas}
{Consistent rendering across browsers}
{Must}{3}
{player}
{WebGL2-enabled browser}
{Context loss without recovery}
\begin{TasksBox}
  \item \cb Request WebGL2 with sensible attributes (alpha/depth/stencil).
  \item \cb Log capabilities; show friendly error UI on failure.
  \item \cb Handle context lost/restored events.
  \item \cb \textbf{AC:} Context is WebGL2; capabilities logged; fallback path documented.
\end{TasksBox}

\StoryCard{B2}{Hi-DPI Canvas Resizing}{WebGL2 Context \& Canvas}
{Crisp visuals on retina/high-DPR displays}
{Must}{3}
{player}
{Resize listeners}
{Performance drop on resize}
\begin{TasksBox}
  \item \cb Listen for resize; set canvas \texttt{width/height} using device pixel ratio.
  \item \cb Update viewport and projection; preserve aspect ratio.
  \item \cb \textbf{AC:} Resizing updates viewport/projection; no stretching; FPS stable.
\end{TasksBox}

\StoryCard{C1}{Fixed-Step Update, Variable Render}{Game Loop \& Timing}
{Deterministic simulation with smooth rendering}
{Must}{3}
{developer}
{rAF/emscripten\_set\_main\_loop}
{Unbounded delta causes instability}
\begin{TasksBox}
  \item \cb Implement 60\,Hz accumulator loop with clamped delta and interpolation.
  \item \cb Pause on blur; resume on focus; bound time drift.
  \item \cb \textbf{AC:} Fixed tick drives simulation; rendering interpolates; pause/resume behaves.
\end{TasksBox}

\StoryCard{D1}{Unified Input Layer}{Input \& Focus}
{Consistent controls across desktop and mobile}
{Must}{3}
{player}
{DOM events; Emscripten input helpers}
{Key repeat noise; gesture ambiguity}
\begin{TasksBox}
  \item \cb Map keyboard, mouse, and touch to the same action set.
  \item \cb Debounce key repeats; map common gestures; pause input on blur.
  \item \cb \textbf{AC:} Repeats debounced; gestures mapped; escape/menu works across devices.
\end{TasksBox}

\StoryCard{E1}{Preload Assets}{Assets \& FS}
{No “first-frame” misses; deterministic startup}
{Must}{3}
{developer}
{Emscripten \texttt{--preload-file}}
{Silent asset failures}
\begin{TasksBox}
  \item \cb Package textures/shaders/levels with \texttt{--preload-file} (or packager).
  \item \cb Show loading screen until files are ready; log missing assets clearly.
  \item \cb \textbf{AC:} Build emits a \texttt{.data} (or preloaded files); loading screen hides only when ready.
\end{TasksBox}

\StoryCard{E2}{Persist Settings with IDBFS}{Assets \& FS}
{Player settings and progress survive reloads}
{Should}{2}
{player}
{IndexedDB available}
{Quota errors; partial writes}
\begin{TasksBox}
  \item \cb Persist settings/state; \texttt{FS.syncfs} to IndexedDB.
  \item \cb Provide reset-to-defaults; show quota errors.
  \item \cb \textbf{AC:} Data appears in IndexedDB; reload restores state; reset clears storage.
\end{TasksBox}

\StoryCard{F1}{Gesture-Gated Audio}{Audio}
{Low-latency audio that complies with browser policies}
{Must}{2}
{player}
{WebAudio/SDL}
{Autoplay blocked}
\begin{TasksBox}
  \item \cb Resume WebAudio context on first click/tap.
  \item \cb Add mute/unmute; persist volume via IDBFS.
  \item \cb \textbf{AC:} First gesture enables audio; mute/volume persist.
\end{TasksBox}

\StoryCard{G1}{Optimized Release Profile}{Performance}
{Higher FPS and smaller downloads}
{Should}{3}
{developer}
{wasm-opt; LTO; SIMD where safe}
{Aggressive opts reduce compat}
\begin{TasksBox}
  \item \cb Use \texttt{-O3 -flto -sASSERTIONS=0} and apply \texttt{wasm-opt}.
  \item \cb Record FPS and bundle size in CI artifacts.
  \item \cb \textbf{AC:} CI shows size \& FPS metrics; profile build has source maps.
\end{TasksBox}

\StoryCard{G2}{In-Game FPS Overlay}{Performance}
{Quick perf checks during gameplay}
{Could}{2}
{developer}
{HUD toggle}
{Overlay overhead}
\begin{TasksBox}
  \item \cb F3 toggles overlay; show FPS, frame time, draw calls, memory.
  \item \cb Keep overhead minimal; update each frame.
  \item \cb \textbf{AC:} Overlay is accurate and light-weight.
\end{TasksBox}

\StoryCard{H1}{PThreads + OffscreenCanvas (Optional)}{Multithreading}
{Move heavy work off the main thread}
{Could}{5}
{developer}
{Cross-origin isolation; SAB}
{Fallback path often forgotten}
\begin{TasksBox}
  \item \cb Provide \texttt{-sUSE\_PTHREADS=1} build with OffscreenCanvas path.
  \item \cb Detect \texttt{crossOriginIsolated}; fallback to main thread if false.
  \item \cb \textbf{AC:} Worker mode runs under isolation; documented fallback otherwise.
\end{TasksBox}

\StoryCard{H2}{COOP/COEP Headers}{Multithreading}
{Enable SharedArrayBuffer safely}
{Must}{2}
{DevOps engineer}
{Server config}
{Prod headers differ from local}
\begin{TasksBox}
  \item \cb Configure COOP/COEP for local dev and production hosting.
  \item \cb Add runtime isolation check and warning UI.
  \item \cb \textbf{AC:} SAB works under isolation; warning shown if missing.
\end{TasksBox}

\StoryCard{I1}{Scores API (Fetch/WebSockets)}{Networking}
{Leaderboards \& telemetry without UI lockups}
{Should}{3}
{player}
{HTTP endpoint}
{Offline gaps; retry storms}
\begin{TasksBox}
  \item \cb Implement fetch wrapper with timeout/retry/backoff; offline queue.
  \item \cb \textbf{AC:} POST/GET succeed; errors handled; retries/backoff non-blocking.
\end{TasksBox}

\StoryCard{J1}{Headless Smoke Tests}{Testing \& CI/CD}
{Catch regressions automatically}
{Must}{3}
{maintainer}
{Playwright/Puppeteer}
{Flaky CI env}
\begin{TasksBox}
  \item \cb Launch page headlessly; assert WebGL2 context; step frames; assert FPS threshold in Release.
  \item \cb Attach screenshots/videos on failure.
  \item \cb \textbf{AC:} CI run fails on missing WebGL2/low FPS; artifacts uploaded.
\end{TasksBox}

\StoryCard{J2}{Build \& Deploy to GitHub Pages}{Testing \& CI/CD}
{Frictionless publishing from CI}
{Must}{3}
{maintainer}
{GitHub Actions; Pages}
{Cache misses slow builds}
\begin{TasksBox}
  \item \cb Cache emsdk/CMake; build Release; upload artifact; deploy on tag; correct public path.
  \item \cb \textbf{AC:} Tag push publishes Pages; caches used on subsequent builds.
\end{TasksBox}

\StoryCard{K1}{CSP \& Strict MIME}{Security}
{Mitigate injection and loading risks}
{Must}{3}
{security engineer}
{HTTP headers}
{Over-strict CSP blocks assets}
\begin{TasksBox}
  \item \cb Define CSP (report-only first); then enforce; ensure \texttt{application/wasm} and script MIME; add SRI if applicable.
  \item \cb Enable \texttt{crossOriginEmbedderPolicy} when threads are used.
  \item \cb \textbf{AC:} CSP enforced without breaking; correct MIME types served.
\end{TasksBox}

\StoryCard{L1}{Onboarding \& Troubleshooting Guide}{Docs \& DX}
{New contributor builds in \(\le\) 10 minutes}
{Must}{2}
{new contributor}
{Screenshots; links}
{Setup friction stalls adoption}
\begin{TasksBox}
  \item \cb Write step-by-step: emsdk install, \texttt{emcmake} flow, common flags.
  \item \cb Include common errors/fixes; link to Emscripten/WebGL2 references.
  \item \cb \textbf{AC:} Following the guide yields a successful build and run locally.
\end{TasksBox}

\end{document}
