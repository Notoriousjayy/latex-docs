\documentclass[11pt]{article}

% ---------- Page + typography ----------
\usepackage[margin=1in]{geometry}
\usepackage[T1]{fontenc}
\usepackage[utf8]{inputenc}
\usepackage{lmodern}

% ---------- Structure + tables ----------
\usepackage{booktabs}
\usepackage{tabularx}
\usepackage{enumitem}

% ---------- Links ----------
\usepackage[hidelinks]{hyperref}

% ---------- Simple header/footer ----------
\usepackage{fancyhdr}
\pagestyle{fancy}
\fancyhf{}
\lhead{AVIT \texorpdfstring{$\leftrightarrow$}{<->} Seeker Closure Playbook}
\rhead{\thepage}
\renewcommand{\headrulewidth}{0.4pt}

\title{\textbf{AVIT \texorpdfstring{$\leftrightarrow$}{<->} Seeker Closure Playbook}\\
\large Standard Operating Procedure (SOP)}
\author{Prepared for AppSec Operations}
\date{\today}

\begin{document}
\maketitle
\vspace{-0.5em}

\section*{Document intent}
This playbook standardizes how to close vulnerability tickets (``AVIT''; sometimes referenced as ``EBIT'' in legacy tooling) by validating the corresponding finding state in \textbf{Seeker}. It prioritizes repeatability, auditability, and minimizing false-closure risk.

\tableofcontents
\newpage

\section{Scope and non-goals}
\subsection*{In scope}
\begin{itemize}[leftmargin=*]
  \item Closing AVIT/EBIT tickets where the source-of-truth status is validated in Seeker (e.g., \textit{False Positive}, \textit{Resolved}, or \textit{No longer present}).
  \item Capturing minimum evidence so a reviewer can reproduce your conclusion later.
  \item A lightweight daily routine to keep throughput consistent.
\end{itemize}

\subsection*{Out of scope}
\begin{itemize}[leftmargin=*]
  \item Determining whether a finding \emph{should} be marked false positive (that is a triage decision process).
  \item Forensics, exploitability assessment, or remediation design.
  \item Tools not represented in Seeker coverage (unsupported languages/scanners). Maintain a separate ``coverage matrix'' for those cases.
\end{itemize}

\section{Definitions}
\begin{tabularx}{\textwidth}{@{} l X @{}}
\toprule
\textbf{Term} & \textbf{Meaning} \\
\midrule
AVIT / EBIT & Vulnerability ticket item to be closed (naming varies by process/tooling). \\
Seeker & Vulnerability tracking system used as validation source for ticket closure. \\
Application module / Project & The Seeker project context under which vulnerabilities are grouped. \\
Short description & Human-readable vulnerability summary (e.g., ``Sensitive data stored unencrypted''). \\
Finding ID & Numeric identifier used to search within the ticket and within Seeker. \\
\bottomrule
\end{tabularx}

\section{Roles and responsibilities}
\begin{tabularx}{\textwidth}{@{} l X @{}}
\toprule
\textbf{Role} & \textbf{Responsibilities} \\
\midrule
AppSec analyst / ticket owner & Validate Seeker state, apply this playbook, capture evidence, close ticket with correct justification. \\
Developer / repo owner & Implements remediation (if applicable) and confirms functional impact. \\
Seeker admin / SME & Resolves access, project mapping, versioning, and reporting questions; adjudicates ambiguous cases. \\
\bottomrule
\end{tabularx}

\section{Inputs and prerequisites}
\subsection*{Inputs required from the AVIT/EBIT ticket}
Capture these fields \emph{before} you touch Seeker:
\begin{enumerate}[leftmargin=*]
  \item \textbf{Application module / Project name} (the Seeker project context).
  \item \textbf{Vulnerability short description} (the ticket summary label).
  \item \textbf{Finding ID} (numeric ID used for search).
  \item \textbf{Repository / component} and any path/module hints.
\end{enumerate}

\subsection*{Access and sanity checks}
\begin{itemize}[leftmargin=*]
  \item Confirm you can access Seeker and the correct project (application module).
  \item Confirm you can view vulnerabilities and findings (not restricted to ``Open'' only).
\end{itemize}

\section{Closure decision tree (high level)}
Follow this order of operations to reduce false closures:

\begin{enumerate}[leftmargin=*]
  \item \textbf{Confirm the correct Seeker project} using the \textit{Application module}.
  \item \textbf{Search} using Finding ID and/or Short description.
  \item \textbf{Broaden filters} (status + versions) until you have high confidence.
  \item \textbf{Decide} based on what you observe:
  \begin{itemize}[leftmargin=*]
    \item \textbf{Found as False Positive} $\rightarrow$ close ticket with ``Seeker false positive'' justification.
    \item \textbf{Found as Resolved/Fixed} $\rightarrow$ close ticket with ``Resolved in Seeker'' justification.
    \item \textbf{Not found after exhaustive filter checks} $\rightarrow$ close ticket with ``No longer in Seeker'' justification \emph{and} record your evidence checklist.
    \item \textbf{Ambiguous / mismatched context} $\rightarrow$ \emph{do not close}; escalate to Seeker SME and/or project owner.
  \end{itemize}
\end{enumerate}

\section{Step-by-step procedures}

\subsection{Procedure A: Close ticket when Seeker shows \textit{False Positive}}
\textbf{Use this when} the matching finding is explicitly marked \textit{False Positive} in Seeker.

\begin{enumerate}[leftmargin=*]
  \item \textbf{Open the AVIT/EBIT ticket} and note:
  \begin{itemize}[leftmargin=*]
    \item Application module (Seeker project)
    \item Vulnerability short description
    \item Finding ID
  \end{itemize}

  \item \textbf{Open Seeker} and navigate to the \textbf{Application module / Project}.
  \begin{itemize}[leftmargin=*]
    \item \emph{Critical:} validate the project key/name matches what the ticket indicates.
  \end{itemize}

  \item \textbf{Locate the vulnerability record} using the \textbf{Short description}.
  \begin{itemize}[leftmargin=*]
    \item Example short description: \textit{Sensitive data stored unencrypted}.
  \end{itemize}

  \item \textbf{Set Status filter to} \textbf{False Positive}.
  \begin{itemize}[leftmargin=*]
    \item If the UI defaults to \textit{Open}, explicitly change it.
  \end{itemize}

  \item \textbf{Validate the Finding ID}:
  \begin{itemize}[leftmargin=*]
    \item Use Seeker search (e.g., ``contains text'') for the numeric Finding ID.
    \item Confirm the returned record matches the ticket context (repo/component/path, if shown).
  \end{itemize}

  \item \textbf{Close the AVIT/EBIT ticket}:
  \begin{itemize}[leftmargin=*]
    \item Resolution: \textbf{Fixed/Closed} (or your process-equivalent).
    \item Justification: \textbf{Seeker false positive}.
    \item Attach/record evidence per Section~\ref{sec:evidence}.
  \end{itemize}
\end{enumerate}

\subsection{Procedure B: Close ticket when Seeker shows the finding is \textit{No longer present}}
\textbf{Use this when} the finding cannot be located in Seeker after you expand filters and versions.

\begin{enumerate}[leftmargin=*]
  \item \textbf{Confirm correct project context} (Application module / Project).
  \item \textbf{Search by Finding ID} in Seeker.
  \item \textbf{Expand Status filters}:
  \begin{itemize}[leftmargin=*]
    \item Open
    \item Requires approval (or similar workflow states)
    \item Fixed/Resolved (if available)
    \item False Positive (as a final check)
  \end{itemize}
  \item \textbf{Expand Version filters}:
  \begin{itemize}[leftmargin=*]
    \item If Seeker has a ``default version'' selector, iterate through available versions.
    \item Re-run the Finding ID search after changing versions.
  \end{itemize}
  \item \textbf{If still not found}, treat it as \textbf{No longer in Seeker}.
  \begin{itemize}[leftmargin=*]
    \item Close the ticket with justification: \textbf{No longer in Seeker (not found after status/version checks)}.
    \item Record evidence per Section~\ref{sec:evidence}, including which filters/versions you checked.
  \end{itemize}
\end{enumerate}

\subsection{Procedure C: Close ticket when Seeker shows \textit{Resolved/Fixed}}
\textbf{Use this when} the finding is present in Seeker but is explicitly marked resolved/fixed (not necessarily false positive).

\begin{enumerate}[leftmargin=*]
  \item Confirm project context.
  \item Search by Finding ID and validate the match.
  \item Verify resolution state (e.g., ``Fixed'', ``Resolved'', ``Closed'') and confirm that the remediation date/version aligns with expectations.
  \item Close the AVIT/EBIT ticket with justification: \textbf{Resolved in Seeker}.
  \item Record evidence per Section~\ref{sec:evidence}.
\end{enumerate}

\section{Evidence and audit trail}
\label{sec:evidence}
Minimum evidence protects you from rework and prevents accidental false closure.

\subsection*{Evidence checklist (minimum)}
\begin{itemize}[leftmargin=*]
  \item Seeker project (Application module) name and key.
  \item Finding ID searched.
  \item Status filter(s) used (and final observed status).
  \item Version(s) checked (if applicable).
  \item Seeker URL to the finding/vulnerability (or screenshot if URL is not stable).
  \item Timestamp (local time) when validation was performed.
\end{itemize}

\subsection*{Recommended evidence (when ``No longer in Seeker'')}
\begin{itemize}[leftmargin=*]
  \item Note that you checked: Open, Requires approval, Fixed/Resolved, and False Positive statuses.
  \item Note that you checked multiple versions (if Seeker is versioned).
  \item If available, capture a screenshot of the search returning no results with the Finding ID visible.
\end{itemize}

\section{Standard closure text (copy/paste templates)}
Use consistent language so reporting is accurate and reviewers can triage quickly.

\subsection*{Template: Seeker False Positive}
\begin{quote}
Validated Finding \#\underline{\hspace{2cm}} in Seeker under project \underline{\hspace{4cm}}. Status is \textbf{False Positive}. Closing ticket as Fixed/Closed. Evidence: Seeker link/screenshot recorded.
\end{quote}

\subsection*{Template: Resolved in Seeker}
\begin{quote}
Validated Finding \#\underline{\hspace{2cm}} in Seeker under project \underline{\hspace{4cm}}. Finding is \textbf{Resolved/Fixed} in Seeker. Closing ticket as Fixed/Closed. Evidence: Seeker link/screenshot recorded.
\end{quote}

\subsection*{Template: No longer in Seeker}
\begin{quote}
Searched Finding \#\underline{\hspace{2cm}} in Seeker under project \underline{\hspace{4cm}} across relevant statuses and versions. Finding \textbf{not present} (no results). Closing ticket as Fixed/Closed with justification ``No longer in Seeker''. Evidence: filters/versions checked + screenshot recorded.
\end{quote}

\section{Common pitfalls and how to avoid them}
\begin{itemize}[leftmargin=*]
  \item \textbf{Wrong project context:} If you search in the wrong application module, you will conclude ``not found'' incorrectly. Always confirm the project name/key first.
  \item \textbf{Overly narrow status filter:} If Seeker defaults to \textit{Open}, you may miss false positives or resolved items. Expand the status set deliberately.
  \item \textbf{Version mismatch:} Some findings only appear under specific versions. Iterate versions before concluding ``not found.''
  \item \textbf{Using short description only:} Short descriptions can collide. Always validate using Finding ID when available.
\end{itemize}

\section{Operational cadence and prioritization}
\subsection*{Daily routine (15--30 minutes)}
\begin{enumerate}[leftmargin=*]
  \item Pull today’s AVIT/EBIT queue.
  \item Work \textbf{tagged} items first when leadership has indicated tag priority.
  \item For each ticket, run the decision tree and capture minimum evidence.
  \item Close in batches to reduce context switching.
\end{enumerate}

\subsection*{Escalation triggers}
Escalate instead of closing when:
\begin{itemize}[leftmargin=*]
  \item The Seeker project/module is unclear or conflicts with the ticket.
  \item The finding exists but the status is not interpretable (custom workflows, access limitations).
  \item The repo/language appears unsupported by Seeker and you cannot identify the source scanner.
\end{itemize}

\section{Appendix A: Quick reference checklist}
\begin{tabularx}{\textwidth}{@{} l X @{}}
\toprule
\textbf{Step} & \textbf{Check} \\
\midrule
1 & Capture Application module, short description, Finding ID from ticket \\
2 & Open Seeker and confirm correct project/key \\
3 & Search by Finding ID (preferred) \\
4 & Expand status filters (Open, Requires approval, Fixed/Resolved, False Positive) \\
5 & Check versions (if applicable) \\
6 & Decide and close with correct template justification \\
7 & Record evidence (link/screenshot + filters/versions + timestamp) \\
\bottomrule
\end{tabularx}

\end{document}
