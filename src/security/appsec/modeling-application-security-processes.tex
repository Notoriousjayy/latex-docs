\documentclass[11pt]{article}

\usepackage[margin=1in]{geometry}
\usepackage{enumitem}
\usepackage{xcolor}
\usepackage{hyperref}
\usepackage{booktabs}
\usepackage{microtype}

% Readability tweaks
\setlength{\parskip}{0.6em}
\setlength{\parindent}{0pt}
\linespread{1.05}

\setlist[itemize]{leftmargin=1.5em}
\setlist[enumerate]{leftmargin=1.5em}

\hypersetup{
  colorlinks=true,
  linkcolor=blue,
  urlcolor=blue
}

\title{Modeling Application Security Processes with\\
\emph{Work the System}, \emph{Traction}, and \emph{This Is Service Design Doing}}
\author{Jordan Suber}
\date{\today}

\begin{document}

\maketitle

\begin{center}
\textbf{At a Glance}
\end{center}

\begin{itemize}
  \item Section~\ref{sec:purpose} -- Why this document exists and what you can do with it.
  \item Section~\ref{sec:books} -- How each book contributes to modeling AppSec processes.
  \item Section~\ref{sec:reading-order} -- A practical reading order tailored to AppSec work.
  \item Section~\ref{sec:pipeline-sketch} -- A sketch of your current AppSec pipeline as blueprints and SOPs.
  \item Section~\ref{sec:together} -- How to combine these ideas into an iterative improvement loop.
\end{itemize}

\bigskip
\hrule
\tableofcontents
\bigskip
\hrule

\section{Purpose of This Document}
\label{sec:purpose}

This document has two goals:

\begin{enumerate}
  \item To summarize how three books
  \emph{Work the System} (Sam Carpenter),
  \emph{Traction} (Gino Wickman), and
  \emph{This Is Service Design Doing}
  can help with modeling Application Security (AppSec) processes.
  \item To sketch how an existing AppSec pipeline
  ---with CI/CD gates, GitHub Advanced Security (GHAS), secrets scanning,
  and related checks---can be modeled using:
  \begin{itemize}
    \item Service blueprints (from \emph{This Is Service Design Doing}),
    \item Standard operating procedures (SOPs) in the style of
          \emph{Work the System}.
  \end{itemize}
\end{enumerate}

The intent is to give you an actionable way to move from:
\begin{quote}
  ``We have tools and gates''\\
  \emph{to}\\
  ``We have a coherent, documented AppSec \textbf{system} that can be
  improved over time.''
\end{quote}

\medskip

\textbf{Navigation tip.} If you already know these books, you can skim Section~\ref{sec:books} and jump straight to the AppSec-specific material in Sections~\ref{sec:pipeline-sketch} and~\ref{sec:together}.


\clearpage
\section{How Each Book Helps with AppSec Process Modeling}
\label{sec:books}

We will briefly summarize how each of the three books contributes
to AppSec process modeling:

\begin{itemize}
  \item \emph{This Is Service Design Doing} -- Service blueprinting
        and journeys, especially from the developer's point of view.
  \item \emph{Work the System} -- Thinking in terms of systems and SOPs.
  \item \emph{Traction} -- Organizational structure, accountability,
        and execution.
\end{itemize}

% --- 1. This Is Service Design Doing ---------------------------------
\subsection{\emph{This Is Service Design Doing}}
\subsubsection*{Core Idea}

\emph{This Is Service Design Doing} is a practical guide to
service design. It emphasizes:

\begin{itemize}
  \item Understanding users and stakeholders through research,
  \item Mapping \textbf{customer journeys},
  \item Designing services using \textbf{service blueprints},
  \item Prototyping and iterating based on feedback.
\end{itemize}

A \textbf{service blueprint} typically includes:

\begin{itemize}
  \item Customer actions (what the user does),
  \item Frontstage actions (what the user sees),
  \item Backstage actions (what the organization does behind the scenes),
  \item Supporting processes and systems,
  \item Evidence (artifacts, communications, logs).
\end{itemize}

\subsubsection*{For AppSec}

This book is most helpful if your goal is:
\begin{quote}
  \textbf{Design AppSec as a developer-centered service}---with clear
  journeys, touchpoints, and support systems.
\end{quote}

\medskip

\textbf{Navigation tip.} Use Section~\ref{sec:reading-order} as a guide
if you want to see everything applied directly to your AppSec pipeline.

% --- 2. Work the System ----------------------------------------------
\subsection{\emph{Work the System}}
\subsubsection*{Core Idea}

\emph{Work the System} argues that every organization is a collection
of \textbf{systems} and \textbf{subsystems}.
To improve results, you should:
\begin{itemize}
  \item Step outside of the ``whirlwind'' of daily operations,
  \item Identify key systems,
  \item Document them as simple written procedures,
  \item Improve those procedures incrementally.
\end{itemize}

The book emphasizes three core documents:

\begin{enumerate}
  \item A \textbf{Strategic Objective} that defines what success looks like.
  \item A set of \textbf{Operating Principles} that guide decisions.
  \item A collection of \textbf{working procedures} that describe
        how recurring work is done.
\end{enumerate}

Instead of viewing AppSec as a set of tools (GHAS, SAST, DAST, etc.),
you define and document the \textbf{systems} that use those tools:

\begin{itemize}
  \item Inputs (events, triggers, artifacts),
  \item Process (steps, decisions, owners),
  \item Outputs (approvals, tickets, metrics).
\end{itemize}

\subsubsection*{For AppSec}

For AppSec, \emph{Work the System} provides a way to:
\begin{itemize}
  \item Treat each AppSec flow (e.g.\ ``secure PR'', ``secrets handling'',
        ``vulnerability triage'') as a \textbf{system},
  \item Write down clear, step-by-step procedures that reflect how the work
        is actually done today,
  \item Slowly improve these procedures as you learn more, instead of trying
        to design a ``perfect'' process up front.
\end{itemize}

In practice, this means you can:
\begin{itemize}
  \item Create an AppSec \textbf{Strategic Objective} that describes
        what ``good'' looks like (e.g.\ ``We detect and remediate
        critical application vulnerabilities before they are exploitable
        in production, with minimal friction for developers.'').
  \item Define a handful of \textbf{Operating Principles}, such as:
        \begin{itemize}
          \item ``We integrate security checks into existing developer
                workflows wherever possible.''
          \item ``We prioritize fixing the highest-risk issues first.''
          \item ``We document processes in simple language and keep them
                as short as possible.''
        \end{itemize}
  \item Document \textbf{working procedures} for key AppSec systems:
        \begin{itemize}
          \item ``Secure Pull Request'' procedure,
          \item ``Secrets Management'' procedure,
          \item ``Vulnerability Triage'' procedure,
          \item ``Incident Response for AppSec Findings'' procedure.
        \end{itemize}
\end{itemize}

This aligns strongly with AppSec process modeling because it forces you to:

\begin{itemize}
  \item Be explicit about \textbf{who does what, when, and how},
  \item Turn ad-hoc practices into repeatable, improvable systems,
  \item Keep the focus on outcomes (e.g.\ reduced risk, faster remediation)
        rather than on tools alone.
\end{itemize}

% --- 3. Traction ------------------------------------------------------
\subsection{\emph{Traction}}
\subsubsection*{Core Idea}

\emph{Traction} introduces the Entrepreneurial Operating System (EOS),
which focuses on:

\begin{itemize}
  \item Vision (shared understanding of where the organization is going),
  \item People (right people in the right seats),
  \item Data (simple, objective metrics),
  \item Issues (identifying and solving root problems),
  \item Process (documented and followed systems),
  \item Traction (execution and accountability).
\end{itemize}

It is very operational and pragmatic, with tools such as:

\begin{itemize}
  \item Accountability chart (who owns what),
  \item Rocks (90-day priorities),
  \item Scorecards (weekly metrics),
  \item Level 10 meetings (structured weekly meetings),
  \item Documented core processes.
\end{itemize}

\subsubsection*{How It Helps AppSec}

For AppSec, \emph{Traction} helps you:
\begin{itemize}
  \item Clarify \textbf{who owns AppSec processes}:
        \begin{itemize}
          \item Who owns the secure SDLC?
          \item Who owns the CI/CD pipeline gates?
          \item Who owns vulnerability management?
        \end{itemize}
  \item Define AppSec-related \textbf{Rocks}:
        \begin{itemize}
          \item ``Implement secrets scanning across all repos'',
          \item ``Document and roll out SOP for vulnerability triage'',
          \item ``Reduce mean time to remediate critical vulns by 30\%.'' 
        \end{itemize}
  \item Create \textbf{scorecard metrics}:
        \begin{itemize}
          \item Number of open critical vulnerabilities,
          \item Mean time to remediate by severity,
          \item Percentage of services passing security gates,
          \item Percentage of repos with GHAS enabled.
        \end{itemize}
  \item Embed AppSec into the \textbf{leadership cadence}:
        \begin{itemize}
          \item AppSec metrics appear on the weekly scorecard,
          \item AppSec issues are raised and resolved in L10 meetings,
          \item AppSec Rocks are reviewed quarterly.
        \end{itemize}
\end{itemize}

\subsubsection*{For AppSec}

In short, \emph{Traction} gives you an organizational frame so that
AppSec is not ``just a set of tools'' but:
\begin{itemize}
  \item Owned by specific people,
  \item Measured with specific metrics,
  \item Improved through a regular cadence of meetings and Rocks,
  \item Connected to the organization's broader goals.
\end{itemize}

\clearpage
\section{Recommended Reading Order for AppSec Modeling}
\label{sec:reading-order}

If your specific goal is to model Application Security processes,
a pragmatic order that also matches the navigation flow in
Section~\ref{sec:books} is:

\begin{enumerate}
  \item \textbf{\emph{This Is Service Design Doing}}\\
        Start here to learn the tools for mapping journeys and services.
        Your first outcome can be one or two \textbf{service blueprints}
        for key AppSec flows (for example, ``secure PR'' or
        ``secrets handling''). These blueprints give you a developer-centered
        view of how AppSec shows up in day-to-day work.
  \item \textbf{\emph{Work the System}}\\
        Next, use \emph{Work the System} to turn those blueprints into
        \textbf{SOPs}---clear, written procedures that can be followed
        and improved by the team. Here you move from ``this is the journey''
        to ``this is the system we run every time.''
  \item \textbf{\emph{Traction}}\\
        Finally, use \emph{Traction} to plug these SOPs into a broader
        organizational system (EOS):
        \begin{itemize}
          \item clarify who owns each AppSec system (secure PR,
                secrets management, vulnerability triage, etc.),
          \item ensure AppSec KPIs show up on scorecards,
          \item set Rocks and metrics around AppSec improvements,
          \item create a regular cadence for reviewing and refining
                the processes.
        \end{itemize}
\end{enumerate}

This sequence moves from:
\begin{enumerate}
  \item \textbf{Understanding the experience} (service design) --
        see Section~\ref{sec:books}, \emph{This Is Service Design Doing},
  \item \textbf{Documenting the system} (SOPs) --
        see Section~\ref{sec:books}, \emph{Work the System},
  \item \textbf{Embedding it organizationally} (EOS / Traction) --
        see Section~\ref{sec:books}, \emph{Traction}.
\end{enumerate}

\medskip

\textbf{Navigation tip.} Treat this section as your reading roadmap.
As you work through the books, follow the order above and use the
cross-references into Section~\ref{sec:books} to jump straight to the
summaries and AppSec-specific notes. If you just want the integrated
view of how all three books combine into one AppSec modeling approach,
you can also jump directly to Section~\ref{sec:together}.
\clearpage
\section{Sketch: Modeling the Current AppSec Pipeline}
\label{sec:pipeline-sketch}

This section gives a concrete sketch of how to model
an existing AppSec pipeline---with CI/CD gates, GHAS scans,
and secrets scanning---using:

\begin{itemize}
  \item Service blueprints, and
  \item \emph{Work the System}-style SOPs.
\end{itemize}

The goal is not to capture every technical detail, but to
create a model that:
\begin{itemize}
  \item Is easy to understand for both AppSec and engineering leadership,
  \item Makes responsibilities and handoffs visible,
  \item Can be iteratively improved.
\end{itemize}

\subsection{High-Level View of the AppSec Pipeline}

At a very high level, your AppSec pipeline might look like:

\begin{itemize}
  \item Developer writes code and opens a PR,
  \item CI pipeline runs:
        \begin{itemize}
          \item Unit tests,
          \item Static analysis (SAST),
          \item Dependency scanning (SCA),
          \item Secrets scanning,
          \item Other checks as needed.
        \end{itemize}
  \item Results are surfaced on the PR (GHAS, CI status checks),
  \item If all checks pass, the PR can be merged,
  \item If there are issues, the developer must address them
        or request an exception,
  \item Critical findings and exceptions feed into a vulnerability
        management process.
\end{itemize}

This high-level view can be turned into:

\begin{itemize}
  \item One or more \textbf{service blueprints} that show the
        developer journey and the supporting systems,
  \item One or more \textbf{SOPs} that describe the step-by-step
        flows in more procedural detail.
\end{itemize}

\subsection{Service-Blueprint Perspective}

From a service-design perspective, consider the flow
``Secure Pull Request and GHAS Scans''.

\subsubsection*{Blueprint 1: Secure Pull Request and GHAS Scans}

\paragraph{Customer (Developer) Actions}

\begin{itemize}
  \item Create feature branch,
  \item Implement changes,
  \item Commit and push,
  \item Open PR against main branch,
  \item Respond to feedback (including security feedback),
  \item Merge when approvals and checks pass.
\end{itemize}

\paragraph{Frontstage (Visible) Actions}

\begin{itemize}
  \item CI pipeline status on PR,
  \item GHAS alerts and annotations on PR,
  \item Comments from reviewers (including AppSec if applicable),
  \item Dashboard views (e.g.\ security overview in GitHub).
\end{itemize}

\paragraph{Backstage Actions}

\begin{itemize}
  \item CI jobs running tests and scans,
  \item GHAS running SAST and SCA checks,
  \item Secrets scanning jobs,
  \item Notification hooks to Slack/Teams or ticketing systems,
  \item Automation that creates tickets for certain findings.
\end{itemize}

\paragraph{Supporting Processes and Systems}

\begin{itemize}
  \item CI/CD platform configuration,
  \item GitHub repository settings and branch protections,
  \item GHAS configuration (enabled repos, rules),
  \item Secrets management policies,
  \item Vulnerability management tooling.
\end{itemize}

\paragraph{Evidence}

\begin{itemize}
  \item PR history,
  \item CI logs,
  \item GHAS alerts and resolution status,
  \item Tickets created for critical vulns,
  \item Audit trail of approvals.
\end{itemize}

\subsubsection*{Blueprint 2: Vulnerable Dependency Management}

A second blueprint could focus on handling vulnerable dependencies:

\paragraph{Customer (Developer) Actions}

\begin{itemize}
  \item Introduce or update a dependency,
  \item See an alert about a vulnerable dependency,
  \item Decide whether to update, replace, or request an exception,
  \item Implement the chosen mitigation,
  \item Verify that alerts are resolved.
\end{itemize}

\paragraph{Frontstage Actions}

\begin{itemize}
  \item GHAS dependency alerts on PRs and in the repo,
  \item Notifications in dashboards or email,
  \item Ticket(s) for critical dependency issues,
  \item Documentation or guidance surfaced to developers.
\end{itemize}

\paragraph{Backstage Actions}

\begin{itemize}
  \item GHAS scanning dependency manifests,
  \item Jobs that aggregate dependency findings,
  \item Automation that opens tickets for high/critical issues,
  \item Periodic jobs that rescan repos.
\end{itemize}

\paragraph{Supporting Processes and Systems}

\begin{itemize}
  \item Dependency management tooling,
  \item Vulnerability database integrations,
  \item Ticketing system,
  \item Exception and risk-acceptance processes.
\end{itemize}

\paragraph{Evidence}

\begin{itemize}
  \item Dependency alert history,
  \item Ticket history,
  \item Records of exceptions and approvals.
\end{itemize}

\subsection{\emph{Work the System}: SOP for Secure Pull Request Flow}

Once you have a blueprint for the secure PR flow, you can
translate it into a \emph{Work the System}-style SOP.

\subsubsection*{SOP: Secure Pull Request Flow}

\paragraph{Purpose}

Ensure that all changes merged into the main branch have passed
appropriate security checks (GHAS, tests, etc.) and that
exceptions are handled consistently.

\paragraph{Scope}

Applies to all repositories using the standard CI/CD pipeline
and GHAS integration.

\paragraph{Owner}

AppSec lead (or designated pipeline owner).

\paragraph{Procedure (Happy Path)}

\begin{enumerate}
  \item Developer creates a feature branch from the main branch.
  \item Developer commits changes with appropriate tests.
  \item Developer opens a Pull Request to the main branch.
  \item CI pipeline triggers automatically:
        \begin{itemize}
          \item Unit tests,
          \item GHAS SAST and SCA scans,
          \item Secrets scanning,
          \item Other configured checks.
        \end{itemize}
  \item PR shows status checks:
        \begin{itemize}
          \item All required checks must pass,
          \item No blocking GHAS alerts for critical issues.
        \end{itemize}
  \item If all checks pass and required reviewers approve,
        the PR is merged.
\end{enumerate}

\paragraph{Procedure (Exceptions)}

\begin{enumerate}
  \item If GHAS finds a critical issue:
        \begin{itemize}
          \item Developer investigates and attempts to fix,
          \item If not fixable within acceptable time,
                developer or team lead requests an exception.
        \end{itemize}
  \item Exception requests must include:
        \begin{itemize}
          \item Description of the issue,
          \item Justification for the exception,
          \item Proposed mitigation and timeline.
        \end{itemize}
  \item AppSec reviews and either:
        \begin{itemize}
          \item Approves the exception (with conditions),
          \item Rejects the exception and requires a fix.
        \end{itemize}
  \item Approved exceptions are logged in the ticketing system
        and tracked until resolved.
\end{enumerate}

\paragraph{Records and Evidence}

\begin{itemize}
  \item GHAS scan reports attached to PRs,
  \item CI logs,
  \item Tickets for critical findings and exceptions,
  \item Audit trail of approvals and merges.
\end{itemize}

\paragraph{Improvement Loop}

\begin{itemize}
  \item On a monthly or quarterly basis:
        \begin{itemize}
          \item Review aggregate GHAS data,
          \item Identify recurring issues,
          \item Update the SOP and training materials.
        \end{itemize}
\end{itemize}

\medskip

\textbf{Navigation tip.} Treat Section~\ref{sec:together} as your high-level implementation checklist; you can skim it first to get the big picture, then dive back into Sections~\ref{sec:books} and~\ref{sec:pipeline-sketch} as needed.

\clearpage
\section{Putting It All Together}
\label{sec:together}

In practice:

\begin{enumerate}[label=\arabic*.]
  \item Use \emph{This Is Service Design Doing} techniques to:
        \begin{itemize}
          \item Map developer journeys,
          \item Identify key AppSec touchpoints,
          \item Build 1--3 service blueprints
                for your most important AppSec flows.
        \end{itemize}
  \item For each blueprint, write a \emph{Work the System}-style SOP:
        \begin{itemize}
          \item Clear purpose and scope,
          \item Step-by-step procedures,
          \item Owners and escalation paths,
          \item Evidence and improvement loop.
        \end{itemize}
  \item Use \emph{Traction} to:
        \begin{itemize}
          \item Assign owners in an accountability chart,
          \item Make AppSec-related improvements into Rocks,
          \item Create scorecard metrics tied to your blueprints and SOPs,
          \item Ensure AppSec is reviewed regularly at the leadership level.
        \end{itemize}
  \item Iterate:
        \begin{itemize}
          \item Update blueprints as tooling or org structures change,
          \item Refine SOPs based on real usage and incidents,
          \item Adjust metrics and Rocks each quarter as needed.
        \end{itemize}
\end{enumerate}

The combination gives you:

\begin{itemize}
  \item A \textbf{service view} of AppSec (developer experience, touchpoints),
  \item A \textbf{systems view} (SOPs and procedures you can improve),
  \item An \textbf{organizational view} (ownership, metrics, cadence).
\end{itemize}

Done well, this moves AppSec from a loose collection of tools and gates
to a coherent, documented system that is easier to operate, explain,
and improve.

\end{document}
