% !TeX program = pdflatex
% If you prefer minted (Pygments) highlighting, see the optional minted setup
% near the bottom of this preamble.

\documentclass[11pt]{article}

% --------------------------
% Packages
% --------------------------
\usepackage[T1]{fontenc}
\usepackage[utf8]{inputenc}
\usepackage{lmodern}
\usepackage{microtype}
\usepackage{geometry}
\geometry{margin=1in}

\usepackage{booktabs}
\usepackage{longtable}
\usepackage{array}
\usepackage{xcolor}
\usepackage{enumitem}
\usepackage{hyperref}
\usepackage{cleveref}

\usepackage{tcolorbox}
\tcbuselibrary{breakable,skins}

\hypersetup{
  colorlinks=true,
  linkcolor=blue,
  urlcolor=blue,
  citecolor=blue,
  pdftitle={A07:2025 — Authentication Failures},
  pdfauthor={},
  pdfsubject={Application Security},
  pdfkeywords={OWASP, Authentication Failures, MFA, Credential Stuffing, Session Management, SSO, CWE, CVE}
}

% --------------------------
% Styling
% --------------------------
\setlist[itemize]{topsep=4pt, itemsep=2pt, leftmargin=1.25em}
\setlist[enumerate]{topsep=4pt, itemsep=2pt, leftmargin=1.5em}

\newtcolorbox{infobox}[1][]{
  breakable,
  colback=gray!5,
  colframe=gray!35,
  arc=2mm,
  left=3mm,right=3mm,top=2mm,bottom=2mm,
  title=#1,
  fonttitle=\bfseries
}

\newtcolorbox{riskbox}[1][]{
  breakable,
  colback=red!2,
  colframe=red!25,
  arc=2mm,
  left=3mm,right=3mm,top=2mm,bottom=2mm,
  title=#1,
  fonttitle=\bfseries
}

% Optional: minted setup (requires -shell-escape)
% \usepackage[cache=false]{minted}
% \setminted{
%   fontsize=\small,
%   breaklines=true,
%   bgcolor=gray!10,
%   frame=lines
% }

% --------------------------
% Document
% --------------------------
\title{\textbf{A07:2025 — Authentication Failures}}
\date{\today}

\begin{document}
\maketitle

\begin{infobox}[Document Summary]
This document consolidates the provided content for \textit{A07:2025 — Authentication Failures} into a structured, print-ready reference, including background context, scoring metrics, common authentication weakness conditions (credential stuffing, brute force, weak defaults, weak recovery, session/token management flaws, missing MFA), prevention guidance (MFA, password policy alignment with NIST guidance, breached-credential checks, rate limiting, secure session management, and trusted identity platforms), attack scenarios, references, and the mapped CWE list.
\end{infobox}

\tableofcontents
\newpage

\section{Background}
Authentication Failures maintains its position at \#7 with a slight name change to more accurately reflect the 36 CWEs in this category. Despite benefits from standardized frameworks, this category has kept its \#7 rank from 2021.

Notable CWEs include:
\begin{itemize}
  \item CWE-259: Use of Hard-coded Password
  \item CWE-297: Improper Validation of Certificate with Host Mismatch
  \item CWE-287: Improper Authentication
  \item CWE-384: Session Fixation
  \item CWE-798: Use of Hard-coded Credentials
\end{itemize}

\section{Score Table}
\begin{table}[h!]
\centering
\renewcommand{\arraystretch}{1.2}
\begin{tabular}{@{}>{\bfseries}p{4.9cm}p{3.1cm}@{}}
\toprule
Metric & Value \\
\midrule
CWEs Mapped & 36 \\
Max Incidence Rate & 15.80\% \\
Avg Incidence Rate & 2.92\% \\
Max Coverage & 100.00\% \\
Avg Coverage & 37.14\% \\
Avg Weighted Exploit & 7.69 \\
Avg Weighted Impact & 4.44 \\
Total Occurrences & 1{,}120{,}673 \\
Total CVEs & 7{,}147 \\
\bottomrule
\end{tabular}
\caption{Provided scoring summary for Authentication Failures.}
\end{table}

\section{Description}
This vulnerability is present when an attacker can trick a system into recognizing an invalid or incorrect user as legitimate.

\subsection{Common Authentication Weakness Conditions}
There may be authentication weaknesses if the application:

\begin{itemize}
  \item Permits automated attacks such as \textbf{credential stuffing}, where the attacker has a breached list of valid usernames and passwords. More recently this has expanded to \textbf{hybrid credential stuffing} (also called \textbf{password spraying}), where the attacker tries variations or increments of spilled credentials (e.g., \texttt{Password1!}, \texttt{Password2!}, \texttt{Password3!}).
  \item Permits \textbf{brute force} or other automated, scripted attacks that are not quickly blocked.
  \item Permits \textbf{default, weak, or well-known passwords}, such as \texttt{Password1} or an \texttt{admin} username with an \texttt{admin} password.
  \item Allows users to create new accounts with already known-breached credentials.
  \item Allows weak or ineffective credential recovery and forgot-password processes, such as \textbf{knowledge-based answers}, which cannot be made safe.
  \item Uses plain text, encrypted, or weakly hashed password data stores (see A04:2025 --- Cryptographic Failures).
  \item Has missing or ineffective multi-factor authentication (MFA).
  \item Allows weak or ineffective fallbacks if MFA is not available.
  \item Exposes session identifiers in the URL, a hidden field, or another insecure client-accessible location.
  \item Reuses the same session identifier after successful login.
  \item Does not correctly invalidate user sessions or authentication tokens (mainly SSO tokens) during logout or after inactivity.
  \item Does not correctly assert the scope and intended audience of the provided credentials.
\end{itemize}

\section{How to Prevent}

\subsection{Credential and MFA Controls}
\begin{enumerate}
  \item Where possible, implement and enforce use of \textbf{multi-factor authentication} to prevent credential stuffing, brute force, and stolen credential reuse.
  \item Where possible, encourage and enable the use of \textbf{password managers} to help users make stronger choices.
  \item Do not ship or deploy with \textbf{default credentials}, particularly for admin users.
  \item Implement \textbf{weak password checks}, such as testing new or changed passwords against the top 10{,}000 worst passwords list.
  \item During new account creation and password changes, validate against \textbf{known breached credential} lists (e.g., using \url{https://haveibeenpwned.com}).
  \item Align password length, complexity, and rotation policies with NIST 800-63b guidelines in Section 5.1.1 for Memorized Secrets (or other modern, evidence-based password policies).
  \item Do not force human beings to rotate passwords unless you suspect a breach. If you suspect breach, force password resets immediately.
\end{enumerate}

\subsection{Anti-enumeration and Abuse Resistance}
\begin{enumerate}[resume]
  \item Harden registration, credential recovery, and API pathways against \textbf{account enumeration} by using the same messages for all outcomes (e.g., ``Invalid username or password.'').
  \item Limit or increasingly delay failed login attempts, but avoid creating a denial-of-service scenario. Log all failures and alert administrators when credential stuffing, brute force, or other automated attacks are detected or suspected.
\end{enumerate}

\subsection{Session and Token Management}
\begin{enumerate}[resume]
  \item Use a server-side, secure, built-in session manager that generates a new random session ID with high entropy after login.
  \item Ensure session identifiers are not in the URL, are securely stored in a secure cookie, and are invalidated after logout, idle timeouts, and absolute timeouts.
  \item Prefer a premade, well-trusted system for authentication, identity, and session management. Transfer this risk where possible by buying and using a hardened and well-tested system.
  \item Verify the intended use of provided credentials (e.g., for JWTs validate \texttt{aud}, \texttt{iss} claims and scopes).
\end{enumerate}

\section{Example Attack Scenarios}

\subsection{Scenario \#1: Credential Stuffing and Password Spray as a ``Password Oracle''}
Credential stuffing (the use of lists of known username and password combinations) is now a very common attack. More recently, attackers ``increment'' or otherwise adjust passwords based on common human behavior (e.g., changing \texttt{Winter2025} to \texttt{Winter2026}, or \texttt{ILoveMyDog6} to \texttt{ILoveMyDog7}).

This adjusting of password attempts is called a \textbf{hybrid credential stuffing attack} or \textbf{password spray attack}, and it can be even more effective than the traditional version. If an application does not implement defenses against automated threats (brute force, scripts, bots) or credential stuffing, the application can be used as a password oracle to determine whether credentials are valid and to gain unauthorized access.

\subsection{Scenario \#2: Password-only Authentication and Counterproductive Password Policies}
Most successful authentication attacks occur due to continued use of passwords as the sole authentication factor. Password rotation and complexity rules (once considered best practices) can encourage password reuse and weak password choices. Organizations are recommended to stop these practices per NIST 800-63 and to enforce multi-factor authentication on important systems.

\subsection{Scenario \#3: Incorrect Session Timeout and Logout Semantics (Public or Shared Machines)}
Session timeouts are not implemented correctly. A user accesses an application on a public computer and closes the browser tab instead of selecting \textit{Logout}. Another example is when a Single Sign-On (SSO) session cannot be closed via Single Logout (SLO): a single login authenticates the user to multiple systems (mail, documents, chat), but logging out only signs out of the current system.

If an attacker uses the same browser after the victim believes they logged out, but the user remains authenticated to one or more applications, the attacker can access the victim’s accounts. Similar issues arise in offices and enterprises when a sensitive application has not been properly exited and a colleague has temporary access to an unlocked computer.

\section{References}
\begin{itemize}
  \item OWASP Authentication Cheat Sheet
  \item OWASP Secure Coding Practices
\end{itemize}

\newpage
\section{List of Mapped CWEs}
\renewcommand{\arraystretch}{1.1}
\begin{longtable}{@{}p{2.4cm}p{12.0cm}@{}}
\toprule
\textbf{CWE} & \textbf{Title} \\
\midrule
\endfirsthead
\toprule
\textbf{CWE} & \textbf{Title} \\
\midrule
\endhead
\bottomrule
\endfoot

CWE-258  & Empty Password in Configuration File \\
CWE-259  & Use of Hard-coded Password \\
CWE-287  & Improper Authentication \\
CWE-288  & Authentication Bypass Using an Alternate Path or Channel \\
CWE-289  & Authentication Bypass by Alternate Name \\
CWE-290  & Authentication Bypass by Spoofing \\
CWE-291  & Reliance on IP Address for Authentication \\
CWE-293  & Using Referer Field for Authentication \\
CWE-294  & Authentication Bypass by Capture-replay \\
CWE-295  & Improper Certificate Validation \\
CWE-297  & Improper Validation of Certificate with Host Mismatch \\
CWE-298  & Improper Validation of Certificate with Host Mismatch \\
CWE-299  & Improper Validation of Certificate with Host Mismatch \\
CWE-300  & Channel Accessible by Non-Endpoint \\
CWE-302  & Authentication Bypass by Assumed-Immutable Data \\
CWE-303  & Incorrect Implementation of Authentication Algorithm \\
CWE-304  & Missing Critical Step in Authentication \\
CWE-305  & Authentication Bypass by Primary Weakness \\
CWE-306  & Missing Authentication for Critical Function \\
CWE-307  & Improper Restriction of Excessive Authentication Attempts \\
CWE-308  & Use of Single-factor Authentication \\
CWE-309  & Use of Password System for Primary Authentication \\
CWE-346  & Origin Validation Error \\
CWE-350  & Reliance on Reverse DNS Resolution for a Security-Critical Action \\
CWE-384  & Session Fixation \\
CWE-521  & Weak Password Requirements \\
CWE-613  & Insufficient Session Expiration \\
CWE-620  & Unverified Password Change \\
CWE-640  & Weak Password Recovery Mechanism for Forgotten Password \\
CWE-798  & Use of Hard-coded Credentials \\
CWE-940  & Improper Verification of Source of a Communication Channel \\
CWE-941  & Incorrectly Specified Destination in a Communication Channel \\
CWE-1390 & Weak Authentication \\
CWE-1391 & Use of Weak Credentials \\
CWE-1392 & Use of Default Credentials \\
CWE-1393 & Use of Default Password \\

\end{longtable}

\end{document}

