% !TeX program = pdflatex
% If you prefer minted (Pygments) highlighting, see the optional minted setup
% near the bottom of this preamble.

\documentclass[11pt]{article}

% --------------------------
% Packages
% --------------------------
\usepackage[T1]{fontenc}
\usepackage[utf8]{inputenc}
\usepackage{lmodern}
\usepackage{microtype}
\usepackage{geometry}
\geometry{margin=1in}

\usepackage{booktabs}
\usepackage{longtable}
\usepackage{array}
\usepackage{xcolor}
\usepackage{enumitem}
\usepackage{hyperref}
\usepackage{cleveref}

\usepackage{tcolorbox}
\tcbuselibrary{breakable,skins}

\hypersetup{
  colorlinks=true,
  linkcolor=blue,
  urlcolor=blue,
  citecolor=blue,
  pdftitle={A06:2025 — Insecure Design},
  pdfauthor={},
  pdfsubject={Application Security},
  pdfkeywords={OWASP, Insecure Design, Threat Modeling, Secure by Design, Architecture, CWE, CVE}
}

% --------------------------
% Styling
% --------------------------
\setlist[itemize]{topsep=4pt, itemsep=2pt, leftmargin=1.25em}
\setlist[enumerate]{topsep=4pt, itemsep=2pt, leftmargin=1.5em}

\newtcolorbox{infobox}[1][]{
  breakable,
  colback=gray!5,
  colframe=gray!35,
  arc=2mm,
  left=3mm,right=3mm,top=2mm,bottom=2mm,
  title=#1,
  fonttitle=\bfseries
}

\newtcolorbox{keyparts}[1][]{
  breakable,
  colback=blue!3,
  colframe=blue!25,
  arc=2mm,
  left=3mm,right=3mm,top=2mm,bottom=2mm,
  title=#1,
  fonttitle=\bfseries
}

% Optional: minted setup (requires -shell-escape)
% \usepackage{minted-config}
% \setminted{
%   fontsize=\small,
%   breaklines=true,
%   bgcolor=gray!10,
%   frame=lines
% }

% --------------------------
% Document
% --------------------------
\title{\textbf{A06:2025 — Insecure Design}}
\date{\today}

\begin{document}
\maketitle

\begin{infobox}[Document Summary]
This document consolidates the provided content for \textit{A06:2025 — Insecure Design} into a structured, print-ready reference, including background context, scoring metrics, conceptual definition (design vs.\ implementation), the three key parts of secure design (requirements/resource management, secure design methodology, and secure development lifecycle), prevention guidance, attack scenarios, references, and the mapped CWE list.
\end{infobox}

\tableofcontents
\newpage

\section{Background}
Insecure Design slides two spots from \#4 to \#6 in the ranking as A02:2025 --- Security Misconfiguration and A03:2025 --- Software Supply Chain Failures leapfrog it. This category was introduced in 2021, and there have been noticeable improvements in the industry related to threat modeling and a greater emphasis on secure design.

This category focuses on risks related to design and architectural flaws, with a call for greater use of threat modeling, secure design patterns, and reference architectures. It includes flaws in the business logic of an application, such as failing to define unwanted or unexpected state changes inside an application.

As a community, we need to move beyond ``shift-left'' in the coding space to \textbf{pre-code} activities such as requirements writing and application design that are critical for Secure by Design (e.g., see \textit{Establish a Modern AppSec Program: Planning and Design Phase}).

Notable Common Weakness Enumerations (CWEs) include:
\begin{itemize}
  \item CWE-256: Unprotected Storage of Credentials
  \item CWE-269: Improper Privilege Management
  \item CWE-434: Unrestricted Upload of File with Dangerous Type
  \item CWE-501: Trust Boundary Violation
  \item CWE-522: Insufficiently Protected Credentials
\end{itemize}

\section{Score Table}
\begin{table}[h!]
\centering
\renewcommand{\arraystretch}{1.2}
\begin{tabular}{@{}>{\bfseries}p{4.9cm}p{3.1cm}@{}}
\toprule
Metric & Value \\
\midrule
CWEs Mapped & 39 \\
Max Incidence Rate & 22.18\% \\
Avg Incidence Rate & 1.86\% \\
Max Coverage & 88.76\% \\
Avg Coverage & 35.18\% \\
Avg Weighted Exploit & 6.96 \\
Avg Weighted Impact & 4.05 \\
Total Occurrences & 729{,}882 \\
Total CVEs & 7{,}647 \\
\bottomrule
\end{tabular}
\caption{Provided scoring summary for Insecure Design.}
\end{table}

\section{Description}

\subsection{Definition and Scope}
Insecure design is a broad category representing different weaknesses, expressed as \textit{``missing or ineffective control design.''} Insecure design is not the source for all other Top Ten risk categories.

\begin{infobox}[Design vs.\ Implementation]
There is a difference between \textbf{insecure design} and \textbf{insecure implementation}. They have different root causes, occur at different times in the development process, and require different remediations.
\begin{itemize}
  \item A \textbf{secure design} can still have \textbf{implementation defects} leading to exploitable vulnerabilities.
  \item An \textbf{insecure design} cannot be fixed by a perfect implementation because the necessary security controls were never designed to defend against specific attacks.
\end{itemize}
\end{infobox}

One factor contributing to insecure design is the lack of business risk profiling inherent in the software or system being developed, and thus the failure to determine what level of security design is required.

\subsection{Three Key Parts of Having a Secure Design}
\begin{keyparts}[Core Components of Secure Design]
\begin{enumerate}
  \item Gathering Requirements and Resource Management
  \item Creating a Secure Design
  \item Having a Secure Development Lifecycle
\end{enumerate}
\end{keyparts}

\subsubsection{Requirements and Resource Management}
Collect and negotiate the business requirements for an application with the business, including the protection requirements concerning confidentiality, integrity, availability, and authenticity of all data assets and the expected business logic. Take into account how exposed your application will be and whether you need segregation of tenants (beyond those needed for access control).

Compile technical requirements, including functional and non-functional security requirements. Plan and negotiate the budget covering all design, build, testing, and operation, including security activities.

\subsubsection{Secure Design}
Secure design is a culture and methodology that constantly evaluates threats and ensures that code is robustly designed and tested to prevent known attack methods. Threat modeling should be integrated into refinement sessions (or similar activities), looking for changes in data flows, access control, or other security controls.

During user story development, determine the correct flow and failure states, ensure they are well understood and agreed upon by responsible and impacted parties. Analyze assumptions and conditions for expected and failure flows to ensure they remain accurate and desirable. Determine how to validate assumptions and enforce conditions needed for proper behaviors. Ensure results are documented in the user story.

Learn from mistakes and offer positive incentives to promote improvements. Secure design is neither an add-on nor a tool that can simply be applied to software.

\subsubsection{Secure Development Lifecycle}
Secure software requires a secure development lifecycle, a secure design pattern, a paved road methodology, a secure component library, appropriate tooling, threat modeling, and incident post-mortems that are used to improve the process.

Engage security specialists at the beginning of a software project, throughout the project, and for ongoing software maintenance. Consider leveraging the OWASP Software Assurance Maturity Model (SAMM) to help structure secure software development efforts.

\begin{infobox}[Security Culture Emphasis]
Self-responsibility of developers is often underappreciated. Foster a culture of awareness, responsibility, and proactive risk mitigation. Regular security exchanges (e.g., during threat modeling sessions) can generate a mindset for including security in important design decisions.
\end{infobox}

\section{How to Prevent}
\begin{itemize}
  \item Establish and use a secure development lifecycle with AppSec professionals to help evaluate and design security and privacy-related controls.
  \item Establish and use a library of secure design patterns or paved-road components.
  \item Use threat modeling for critical parts of the application such as authentication, access control, business logic, and key flows.
  \item Use threat modeling as an educational tool to generate a security mindset.
  \item Integrate security language and controls into user stories.
  \item Integrate plausibility checks at each tier of the application (from frontend to backend).
  \item Write unit and integration tests to validate that all critical flows are resistant to the threat model. Compile use-cases and misuse-cases for each tier of the application.
  \item Segregate tier layers on system and network layers depending on exposure and protection needs.
  \item Segregate tenants robustly by design throughout all tiers.
\end{itemize}

\section{Example Attack Scenarios}

\subsection{Scenario \#1: Weak Credential Recovery Workflow}
A credential recovery workflow might include ``questions and answers,'' which is prohibited by NIST 800-63b, the OWASP ASVS, and the OWASP Top 10. Questions and answers cannot be trusted as evidence of identity, as more than one person can know the answers. Such functionality should be removed and replaced with a more secure design.

\subsection{Scenario \#2: Business Logic Abuse of Group Booking Discounts}
A cinema chain allows group booking discounts and has a maximum of fifteen attendees before requiring a deposit. Attackers could threat model this flow and test whether an attack vector exists in the business logic, such as booking six hundred seats across all cinemas in a few requests, causing a massive loss of income.

\subsection{Scenario \#3: Missing Anti-bot Design for Scarce Goods}
A retail chain’s e-commerce website does not protect against bots run by scalpers buying high-end video cards to resell on auction websites. This creates negative publicity for the video card makers and retail chain owners, and enduring bad blood with enthusiasts who cannot obtain these cards at any price. Careful anti-bot design and domain logic rules (e.g., suspicious purchases made within seconds of availability) can help identify inauthentic purchases and reject transactions.

\section{References}
\begin{itemize}
  \item OWASP Cheat Sheet: Secure Design Principles
  \item OWASP SAMM: Design \,|\, Secure Architecture
  \item OWASP SAMM: Design \,|\, Threat Assessment
  \item NIST --- Guidelines on Minimum Standards for Developer Verification of Software
  \item The Threat Modeling Manifesto
  \item Awesome Threat Modeling
\end{itemize}

\newpage
\section{List of Mapped CWEs}
\renewcommand{\arraystretch}{1.1}
\begin{longtable}{@{}p{2.4cm}p{12.0cm}@{}}
\toprule
\textbf{CWE} & \textbf{Title} \\
\midrule
\endfirsthead
\toprule
\textbf{CWE} & \textbf{Title} \\
\midrule
\endhead
\bottomrule
\endfoot

CWE-73   & External Control of File Name or Path \\
CWE-183  & Permissive List of Allowed Inputs \\
CWE-256  & Unprotected Storage of Credentials \\
CWE-266  & Incorrect Privilege Assignment \\
CWE-269  & Improper Privilege Management \\
CWE-286  & Incorrect User Management \\
CWE-311  & Missing Encryption of Sensitive Data \\
CWE-312  & Cleartext Storage of Sensitive Information \\
CWE-313  & Cleartext Storage in a File or on Disk \\
CWE-316  & Cleartext Storage of Sensitive Information in Memory \\
CWE-362  & Concurrent Execution using Shared Resource with Improper Synchronization (\textit{Race Condition}) \\
CWE-382  & J2EE Bad Practices: Use of \texttt{System.exit()} \\
CWE-419  & Unprotected Primary Channel \\
CWE-434  & Unrestricted Upload of File with Dangerous Type \\
CWE-436  & Interpretation Conflict \\
CWE-444  & Inconsistent Interpretation of HTTP Requests (\textit{HTTP Request Smuggling}) \\
CWE-451  & User Interface (UI) Misrepresentation of Critical Information \\
CWE-454  & External Initialization of Trusted Variables or Data Stores \\
CWE-472  & External Control of Assumed-Immutable Web Parameter \\
CWE-501  & Trust Boundary Violation \\
CWE-522  & Insufficiently Protected Credentials \\
CWE-525  & Use of Web Browser Cache Containing Sensitive Information \\
CWE-539  & Use of Persistent Cookies Containing Sensitive Information \\
CWE-598  & Use of GET Request Method With Sensitive Query Strings \\
CWE-602  & Client-Side Enforcement of Server-Side Security \\
CWE-628  & Function Call with Incorrectly Specified Arguments \\
CWE-642  & External Control of Critical State Data \\
CWE-646  & Reliance on File Name or Extension of Externally-Supplied File \\
CWE-653  & Insufficient Compartmentalization \\
CWE-656  & Reliance on Security Through Obscurity \\
CWE-657  & Violation of Secure Design Principles \\
CWE-676  & Use of Potentially Dangerous Function \\
CWE-693  & Protection Mechanism Failure \\
CWE-799  & Improper Control of Interaction Frequency \\
CWE-807  & Reliance on Untrusted Inputs in a Security Decision \\
CWE-841  & Improper Enforcement of Behavioral Workflow \\
CWE-1021 & Improper Restriction of Rendered UI Layers or Frames \\
CWE-1022 & Use of Web Link to Untrusted Target with \texttt{window.opener} Access \\
CWE-1125 & Excessive Attack Surface \\

\end{longtable}

\end{document}

