% !TeX program = pdflatex
% If you prefer minted (Pygments) highlighting, see the optional minted setup
% near the bottom of this preamble.

\documentclass[11pt]{article}

% --------------------------
% Packages
% --------------------------
\usepackage[T1]{fontenc}
\usepackage[utf8]{inputenc}
\usepackage{lmodern}
\usepackage{microtype}
\usepackage{geometry}
\geometry{margin=1in}

\usepackage{booktabs}
\usepackage{longtable}
\usepackage{array}
\usepackage{xcolor}
\usepackage{enumitem}
\usepackage{hyperref}
\usepackage{cleveref}

\usepackage{listings}
\usepackage{tcolorbox}
\tcbuselibrary{breakable,skins}

\hypersetup{
  colorlinks=true,
  linkcolor=blue,
  urlcolor=blue,
  citecolor=blue,
  pdftitle={A02:2025 — Security Misconfiguration},
  pdfauthor={},
  pdfsubject={Application Security},
  pdfkeywords={OWASP, Security Misconfiguration, Hardening, Configuration, CWE, CVE}
}

% --------------------------
% Styling
% --------------------------
\setlist[itemize]{topsep=4pt, itemsep=2pt, leftmargin=1.25em}
\setlist[enumerate]{topsep=4pt, itemsep=2pt, leftmargin=1.5em}

\definecolor{CodeBg}{RGB}{247,247,247}
\definecolor{CodeFrame}{RGB}{220,220,220}

\newtcolorbox{infobox}[1][]{
  breakable,
  colback=gray!5,
  colframe=gray!35,
  arc=2mm,
  left=3mm,right=3mm,top=2mm,bottom=2mm,
  title=#1,
  fonttitle=\bfseries
}

% Optional: minted setup (requires -shell-escape)
% \usepackage[cache=false]{minted}
% \setminted{
%   fontsize=\small,
%   breaklines=true,
%   bgcolor=CodeBg,
%   frame=lines
% }

% --------------------------
% Document
% --------------------------
\title{\textbf{A02:2025 — Security Misconfiguration}}
\date{\today}

\begin{document}
\maketitle

\begin{infobox}[Document Summary]
This document consolidates the provided content for \textit{A02:2025 — Security Misconfiguration} into a structured, print-ready reference, including background context, scoring metrics, vulnerability indicators, prevention guidance, practical attack scenarios, references, and the mapped CWE list.
\end{infobox}

\tableofcontents
\newpage

\section{Background}
Moving up from \#5 in the previous edition, 100\% of the applications tested were found to have some form of misconfiguration, with an average incidence rate of 3.00\%, and over 719k occurrences of a Common Weakness Enumeration (CWE) in this risk category. With more shifts into highly configurable software, it is not surprising to see this category moving up. Notable CWEs included are CWE-16 (Configuration) and CWE-611 (Improper Restriction of XML External Entity Reference, XXE).

\section{Score Table}
\begin{table}[h!]
\centering
\renewcommand{\arraystretch}{1.2}
\begin{tabular}{@{}>{\bfseries}p{4.6cm}p{3.4cm}@{}}
\toprule
Metric & Value \\
\midrule
CWEs Mapped & 16 \\
Max Incidence Rate & 27.70\% \\
Avg Incidence Rate & 3.00\% \\
Max Coverage & 100.00\% \\
Avg Coverage & 52.35\% \\
Avg Weighted Exploit & 7.96 \\
Avg Weighted Impact & 3.97 \\
Total Occurrences & 719,084 \\
Total CVEs & 1,375 \\
\bottomrule
\end{tabular}
\caption{Provided scoring summary for Security Misconfiguration.}
\end{table}

\section{Description}
Security misconfiguration is when a system, application, or cloud service is set up incorrectly from a security perspective, creating vulnerabilities.

\subsection{Indicators of Misconfiguration}
The application might be vulnerable if:

\begin{itemize}
  \item It is missing appropriate security hardening across any part of the application stack or has improperly configured permissions on cloud services.
  \item Unnecessary features are enabled or installed (e.g., unnecessary ports, services, pages, accounts, testing frameworks, or privileges).
  \item Default accounts and their passwords are still enabled and unchanged.
  \item There is a lack of central configuration for intercepting excessive error messages; error handling reveals stack traces or other overly informative error messages to users.
  \item For upgraded systems, the latest security features are disabled or not configured securely.
  \item Excessive prioritization of backward compatibility leads to insecure configuration.
  \item Security settings in application servers, application frameworks (e.g., Struts, Spring, ASP.NET), libraries, databases, etc., are not set to secure values.
  \item The server does not send security headers or directives, or they are not set to secure values.
  \item Without a concerted, repeatable application security configuration hardening process, systems are at a higher risk.
\end{itemize}

\section{How to Prevent}
Secure installation processes should be implemented, including:

\begin{enumerate}
  \item \textbf{Repeatable hardening:} Enable fast and easy deployment of additional environments that are appropriately locked down. Development, QA, and production environments should be configured identically, with different credentials used in each environment. Automate this process to minimize the effort required to set up a new secure environment.
  \item \textbf{Minimal platform footprint:} Use a minimal platform without unnecessary features, components, documentation, or samples. Remove or do not install unused features and frameworks.
  \item \textbf{Patch and configuration governance:} Review and update configurations appropriate to all security notes, updates, and patches as part of patch management. Review cloud storage permissions (e.g., S3 bucket permissions).
  \item \textbf{Segmentation and isolation:} Use a segmented application architecture to provide effective separation between components or tenants, with segmentation, containerization, or cloud security groups (ACLs).
  \item \textbf{Client security directives:} Send security directives to clients (e.g., Security Headers).
  \item \textbf{Automated verification:} Implement an automated process to verify the effectiveness of configurations and settings in all environments.
  \item \textbf{Centralized error interception:} Proactively add a central configuration to intercept excessive error messages as a backup.
  \item \textbf{Manual verification cadence:} If verifications are not automated, manually verify annually at a minimum.
  \item \textbf{Prefer federated or ephemeral credentials:} Use identity federation, short-lived credentials, or role-based access mechanisms provided by the underlying platform instead of embedding static keys or secrets in code, configuration files, or pipelines.
\end{enumerate}

\section{Example Attack Scenarios}
\subsection{Scenario \#1: Sample Applications and Default Credentials}
The application server comes with sample applications not removed from the production server. These sample applications have known security flaws that attackers use to compromise the server. Suppose one of these applications is the admin console, and default accounts were not changed; in that case, the attacker logs in with the default password and takes over.

\subsection{Scenario \#2: Directory Listing and Reverse Engineering}
Directory listing is not disabled on the server. An attacker discovers they can list directories, finds and downloads the compiled Java classes, and then decompiles and reverse engineers them to view the code. The attacker then identifies a severe access control flaw in the application.

\subsection{Scenario \#3: Verbose Error Messages}
The application server's configuration allows detailed error messages, such as stack traces, to be returned to users. This potentially exposes sensitive information or underlying flaws, such as component versions that are known to be vulnerable.

\subsection{Scenario \#4: Insecure Cloud Storage Defaults}
A cloud service provider (CSP) defaults to having sharing permissions open to the Internet. This allows sensitive data stored within cloud storage to be accessed.

\section{References}
\begin{itemize}
  \item OWASP Testing Guide: Configuration Management
  \item OWASP Testing Guide: Testing for Error Codes
  \item Application Security Verification Standard V13: Configuration
  \item NIST Guide to General Server Hardening
  \item CIS Security Configuration Guides/Benchmarks
  \item Amazon S3 Bucket Discovery and Enumeration
  \item ScienceDirect: Security Misconfiguration
\end{itemize}

\newpage
\section{List of Mapped CWEs}
\renewcommand{\arraystretch}{1.1}
\begin{longtable}{@{}p{2.2cm}p{12.2cm}@{}}
\toprule
\textbf{CWE} & \textbf{Title} \\
\midrule
\endfirsthead
\toprule
\textbf{CWE} & \textbf{Title} \\
\midrule
\endhead
\bottomrule
\endfoot

CWE-5 & J2EE Misconfiguration: Data Transmission Without Encryption \\
CWE-11 & ASP.NET Misconfiguration: Creating Debug Binary \\
CWE-13 & ASP.NET Misconfiguration: Password in Configuration File \\
CWE-15 & External Control of System or Configuration Setting \\
CWE-16 & Configuration \\
CWE-260 & Password in Configuration File \\
CWE-315 & Cleartext Storage of Sensitive Information in a Cookie \\
CWE-489 & Active Debug Code \\
CWE-526 & Exposure of Sensitive Information Through Environmental Variables \\
CWE-547 & Use of Hard-coded, Security-relevant Constants \\
CWE-611 & Improper Restriction of XML External Entity Reference \\
CWE-614 & Sensitive Cookie in HTTPS Session Without \texttt{Secure} Attribute \\
CWE-776 & Improper Restriction of Recursive Entity References in DTDs (\textit{XML Entity Expansion}) \\
CWE-942 & Permissive Cross-domain Policy with Untrusted Domains \\
CWE-1004 & Sensitive Cookie Without \texttt{HttpOnly} Flag \\
CWE-1174 & ASP.NET Misconfiguration: Improper Model Validation \\

\end{longtable}

\end{document}

