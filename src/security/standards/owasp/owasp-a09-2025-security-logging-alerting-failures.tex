% !TeX program = pdflatex
% If you prefer minted (Pygments) highlighting, see the optional minted setup
% near the bottom of this preamble.

\documentclass[11pt]{article}

% --------------------------
% Packages
% --------------------------
\usepackage[T1]{fontenc}
\usepackage[utf8]{inputenc}
\usepackage{lmodern}
\usepackage{microtype}
\usepackage{geometry}
\geometry{margin=1in}

\usepackage{booktabs}
\usepackage{longtable}
\usepackage{array}
\usepackage{xcolor}
\usepackage{enumitem}
\usepackage{hyperref}
\usepackage{cleveref}

\usepackage{tcolorbox}
\tcbuselibrary{breakable,skins}

\hypersetup{
  colorlinks=true,
  linkcolor=blue,
  urlcolor=blue,
  citecolor=blue,
  pdftitle={A09:2025 — Security Logging \& Alerting Failures},
  pdfauthor={},
  pdfsubject={Application Security},
  pdfkeywords={OWASP, Logging, Alerting, Monitoring, SOC, Incident Response, CWE, CVE}
}

% --------------------------
% Styling
% --------------------------
\setlist[itemize]{topsep=4pt, itemsep=2pt, leftmargin=1.25em}
\setlist[enumerate]{topsep=4pt, itemsep=2pt, leftmargin=1.5em}

\newtcolorbox{infobox}[1][]{
  breakable,
  colback=gray!5,
  colframe=gray!35,
  arc=2mm,
  left=3mm,right=3mm,top=2mm,bottom=2mm,
  title={#1},
  fonttitle=\bfseries
}

\newtcolorbox{controlbox}[1][]{
  breakable,
  colback=blue!3,
  colframe=blue!25,
  arc=2mm,
  left=3mm,right=3mm,top=2mm,bottom=2mm,
  title={#1},
  fonttitle=\bfseries
}

\newtcolorbox{scenario}[1][]{
  breakable,
  colback=green!3,
  colframe=green!25,
  arc=2mm,
  left=3mm,right=3mm,top=2mm,bottom=2mm,
  title={#1},
  fonttitle=\bfseries
}

% Optional: minted setup (requires -shell-escape)
% \usepackage[cache=false]{minted}
% \setminted{cache=false,
%   fontsize=\small,
%   breaklines=true,
%   bgcolor=gray!10,
%   frame=lines
% }

% --------------------------
% Document
% --------------------------
\title{\textbf{A09:2025 — Security Logging \& Alerting Failures}}
\date{\today}

\begin{document}
\maketitle

\begin{infobox}[Document Summary]
This document consolidates the provided content for \textit{A09:2025 — Security Logging \& Alerting Failures} into a structured, print-ready reference, including background context, scoring metrics, the operational impact of insufficient logging/monitoring/alerting, common failure modes (coverage gaps, tampering risk, lack of monitoring, poor escalation, alert fatigue, and missing playbooks), prevention guidance (context-rich logs, integrity controls, encoding, audit trails, SOC use cases and playbooks, honeytokens, and incident response planning), illustrative real-world scenarios, references, and the mapped CWE list.
\end{infobox}

\tableofcontents
\newpage

\section{Background}
Security Logging \& Alerting Failures retains its position at \#9. This category includes a slight name change to emphasize the alerting function needed to induce action on relevant logging events.

This category will always be underrepresented in the data and, for the third time, was voted into its position via the community survey. It is difficult to test for and has minimal representation in CVE/CVSS data (only 723 CVEs), but can be highly impactful for visibility, incident alerting, and forensics.

This category includes issues such as:
\begin{itemize}
  \item CWE-117: Improper output encoding/neutralization for logs
  \item CWE-532: Inserting sensitive data into log files
  \item CWE-778: Insufficient logging
\end{itemize}

\section{Score Table}
\begin{table}[h!]
\centering
\renewcommand{\arraystretch}{1.2}
\begin{tabular}{@{}>{\bfseries}p{4.9cm}p{3.1cm}@{}}
\toprule
Metric & Value \\
\midrule
CWEs Mapped & 5 \\
Max Incidence Rate & 11.33\% \\
Avg Incidence Rate & 3.91\% \\
Max Coverage & 85.96\% \\
Avg Coverage & 46.48\% \\
Avg Weighted Exploit & 7.19 \\
Avg Weighted Impact & 2.65 \\
Total Occurrences & 260{,}288 \\
Total CVEs & 723 \\
\bottomrule
\end{tabular}
\caption{Provided scoring summary for Security Logging \& Alerting Failures.}
\end{table}

\section{Description}
Without logging and monitoring, attacks and breaches cannot be detected. Without alerting, it is difficult to respond quickly and effectively during a security incident.

Insufficient logging, continuous monitoring, detection, and alerting to initiate active responses occur any time:

\begin{itemize}
  \item Auditable events (e.g., logins, failed logins, high-value transactions) are not logged or are logged inconsistently (e.g., logging only successful logins but not failures).
  \item Warnings and errors generate no, inadequate, or unclear log messages.
  \item Log integrity is not properly protected from tampering.
  \item Application and API logs are not monitored for suspicious activity.
  \item Logs are only stored locally and are not properly backed up.
  \item Alerting thresholds and escalation processes are missing or ineffective; alerts are not received or reviewed in a reasonable amount of time.
  \item Penetration testing and dynamic application security testing (DAST) scans (e.g., Burp or ZAP) do not trigger alerts.
  \item The application cannot detect, escalate, or alert for active attacks in real-time or near real-time.
  \item Logging and alerting events are made visible to users or attackers (see A01:2025 --- Broken Access Control), or sensitive information is logged that should not be logged (e.g., PII or PHI).
  \item Log data is not correctly encoded, creating risk of injections or attacks on logging/monitoring systems.
  \item Errors and exceptional conditions are missing or mishandled such that the system is unaware an error occurred and therefore cannot log the problem.
  \item Alert ``use cases'' are missing or outdated, preventing recognition of special situations.
  \item Excessive false positives create alert fatigue, making it impossible to distinguish important alerts from unimportant ones.
  \item Detected alerts cannot be processed correctly because the playbook is incomplete, outdated, or missing.
\end{itemize}

\section{How to Prevent}
Developers should implement some or all of the following controls, depending on application risk:

\begin{controlbox}[Developer and System Controls]
\begin{enumerate}
  \item Ensure all login, access control, and server-side input validation failures can be logged with sufficient user context to identify suspicious or malicious accounts, and retained long enough to support delayed forensic analysis.
  \item Ensure every part of the application containing a security control is logged, whether the control succeeds or fails.
  \item Ensure logs are generated in a format that log management solutions can easily consume.
  \item Ensure log data is encoded correctly to prevent injections or attacks on logging or monitoring systems.
  \item Ensure all transactions have an audit trail with integrity controls to prevent tampering or deletion (e.g., append-only database tables or similar mechanisms).
  \item Ensure transactions that throw an error are rolled back and restarted. Always fail closed.
  \item If application behavior is suspicious, issue an alert. Provide developers guidance on suspicious patterns, or procure a system that supports this capability.
\end{enumerate}
\end{controlbox}

\begin{controlbox}[DevSecOps, SOC, and Operational Controls]
\begin{enumerate}[resume]
  \item Establish effective monitoring and alerting use cases including playbooks so suspicious activity is detected and responded to quickly by the Security Operations Center (SOC).
  \item Add ``honeytokens'' as attacker traps (e.g., in databases, data, or as real/technical user identities). Because they are not used in normal operations, any access produces alertable events with low false positives.
  \item Optionally apply behavior analysis and AI support to reduce false positives and improve alert quality.
  \item Establish or adopt an incident response and recovery plan, such as NIST 800-61r2 (or later). Train developers to recognize application attacks and incidents and to report them appropriately.
  \item Consider commercial and open-source application protection and log correlation products (e.g., OWASP ModSecurity Core Rule Set; Elasticsearch/Logstash/Kibana (ELK) stack), as well as commercial observability tools supporting near real-time response and blocking.
\end{enumerate}
\end{controlbox}

\section{Example Attack Scenarios}

\begin{scenario}[Scenario \#1: Undetected Breach Due to Lack of Logging and Monitoring]
A children's health plan provider could not detect a breach due to lack of monitoring and logging. An external party informed the provider that an attacker had accessed and modified thousands of sensitive health records of more than 3.5 million children. A post-incident review found significant vulnerabilities were not addressed. Without logging or monitoring, the breach may have been ongoing since 2013 (over seven years).
\end{scenario}

\begin{scenario}[Scenario \#2: Third-party Cloud Provider Breach Notification Lag]
A major Indian airline experienced a data breach involving more than ten years of personal data of millions of passengers, including passport and credit card data. The breach occurred at a third-party cloud hosting provider, who notified the airline only after some time.
\end{scenario}

\begin{scenario}[Scenario \#3: GDPR-reportable Breach from Payment App Vulnerabilities]
A major European airline suffered a GDPR-reportable breach reportedly caused by payment application security vulnerabilities exploited by attackers, who harvested more than 400{,}000 customer payment records. The airline was fined 20 million pounds by the privacy regulator.
\end{scenario}

\section{References}
\begin{itemize}
  \item OWASP Proactive Controls: C9: Implement Logging and Monitoring
  \item OWASP Application Security Verification Standard: V16 Security Logging and Error Handling
  \item OWASP Cheat Sheet: Application Logging Vocabulary
  \item OWASP Cheat Sheet: Logging
  \item Data Integrity: Recovering from Ransomware and Other Destructive Events
  \item Data Integrity: Identifying and Protecting Assets Against Ransomware and Other Destructive Events
  \item Data Integrity: Detecting and Responding to Ransomware and Other Destructive Events
  \item Real world example of such failures in Snowflake Breach
\end{itemize}

\newpage
\section{List of Mapped CWEs}
\renewcommand{\arraystretch}{1.1}
\begin{longtable}{@{}p{2.4cm}p{12.0cm}@{}}
\toprule
\textbf{CWE} & \textbf{Title} \\
\midrule
\endfirsthead
\toprule
\textbf{CWE} & \textbf{Title} \\
\midrule
\endhead
\bottomrule
\endfoot

CWE-117 & Improper Output Neutralization for Logs \\
CWE-221 & Information Loss of Omission \\
CWE-223 & Omission of Security-relevant Information \\
CWE-532 & Insertion of Sensitive Information into Log File \\
CWE-778 & Insufficient Logging \\

\end{longtable}

\end{document}

