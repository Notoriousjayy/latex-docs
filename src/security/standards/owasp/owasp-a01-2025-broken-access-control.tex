% !TeX program = pdflatex
% If you prefer minted (Pygments) highlighting, see the optional minted setup
% near the bottom of this preamble.

\documentclass[11pt]{article}

% --------------------------
% Packages
% --------------------------
\usepackage[T1]{fontenc}
\usepackage[utf8]{inputenc}
\usepackage{lmodern}
\usepackage{microtype}
\usepackage{geometry}
\geometry{margin=1in}

\usepackage{booktabs}
\usepackage{longtable}
\usepackage{array}
\usepackage{xcolor}
\usepackage{enumitem}
\usepackage{hyperref}
\usepackage{cleveref}

\usepackage{listings}
\usepackage{tcolorbox}
\tcbuselibrary{listings,breakable,skins}

\hypersetup{
  colorlinks=true,
  linkcolor=blue,
  urlcolor=blue,
  citecolor=blue,
  pdftitle={A01:2025 — Broken Access Control},
  pdfauthor={},
  pdfsubject={Application Security},
  pdfkeywords={OWASP, Broken Access Control, Authorization, CWE, CVE}
}

% --------------------------
% Styling
% --------------------------
\setlist[itemize]{topsep=4pt, itemsep=2pt, leftmargin=1.25em}
\setlist[enumerate]{topsep=4pt, itemsep=2pt, leftmargin=1.5em}

\definecolor{CodeBg}{RGB}{247,247,247}
\definecolor{CodeFrame}{RGB}{220,220,220}

\lstdefinestyle{code}{
  backgroundcolor=\color{CodeBg},
  basicstyle=\ttfamily\small,
  frame=single,
  rulecolor=\color{CodeFrame},
  frameround=ffff,
  breaklines=true,
  breakatwhitespace=true,
  columns=fullflexible,
  showstringspaces=false
}

\newtcolorbox{infobox}[1][]{
  breakable,
  colback=gray!5,
  colframe=gray!35,
  arc=2mm,
  left=3mm,right=3mm,top=2mm,bottom=2mm,
  title=#1,
  fonttitle=\bfseries
}

\newtcolorbox{codebox}[1][]{
  breakable,
  colback=CodeBg,
  colframe=CodeFrame,
  arc=2mm,
  left=2mm,right=2mm,top=1mm,bottom=1mm,
  title=#1,
  fonttitle=\bfseries
}

% Optional: minted setup (requires -shell-escape)
% \usepackage[cache=false]{minted}
% \setminted{cache=false,
%   fontsize=\small,
%   breaklines=true,
%   bgcolor=CodeBg,
%   frame=lines
% }

% --------------------------
% Document
% --------------------------
\title{\textbf{A01:2025 — Broken Access Control}}
\date{\today}

\begin{document}
\maketitle

\begin{infobox}[Document Summary]
This document consolidates the provided content for \textit{A01:2025 — Broken Access Control} into a structured, print-ready reference, including background context, scoring metrics, common vulnerability patterns, prevention guidance, practical attack scenarios, references, and the mapped CWE list.
\end{infobox}

\tableofcontents
\newpage

\section{Background}
Maintaining its position at \#1 in the Top Ten, 100\% of the applications tested were found to have some form of broken access control. Notable CWEs included are CWE-200: Exposure of Sensitive Information to an Unauthorized Actor, CWE-201: Exposure of Sensitive Information Through Sent Data, CWE-918: Server-Side Request Forgery (SSRF), and CWE-352: Cross-Site Request Forgery (CSRF). This category has the highest number of occurrences in the contributed data, and the second highest number of related CVEs.

\section{Score Table}
\begin{table}[h!]
\centering
\renewcommand{\arraystretch}{1.2}
\begin{tabular}{@{}>{\bfseries}p{4.6cm}p{3.4cm}@{}}
\toprule
Metric & Value \\
\midrule
CWEs Mapped & 40 \\
Max Incidence Rate & 20.15\% \\
Avg Incidence Rate & 3.74\% \\
Max Coverage & 100.00\% \\
Avg Coverage & 42.93\% \\
Avg Weighted Exploit & 7.04 \\
Avg Weighted Impact & 3.84 \\
Total Occurrences & 1,839,701 \\
Total CVEs & 32,654 \\
\bottomrule
\end{tabular}
\caption{Provided scoring summary for Broken Access Control.}
\end{table}

\section{Description}
Access control enforces policy such that users cannot act outside of their intended permissions. Failures typically lead to unauthorized information disclosure, modification or destruction of all data, or performing a business function outside the user's limits.

\subsection{Common Access Control Vulnerability Patterns}
\begin{itemize}
  \item Violation of the principle of least privilege (deny by default), where access should only be granted for particular capabilities, roles, or users, but is available to anyone.
  \item Bypassing access control checks by modifying the URL (parameter tampering or force browsing), internal application state, or the HTML page, or by using an attack tool that modifies API requests.
  \item Permitting viewing or editing someone else's account by providing its unique identifier (insecure direct object references).
  \item An accessible API with missing access controls for \texttt{POST}, \texttt{PUT}, and \texttt{DELETE}.
  \item Elevation of privilege: acting as a user without being logged in or gaining privileges beyond those expected of the logged-in user (e.g., admin access).
  \item Metadata manipulation, such as replaying or tampering with a JSON Web Token (JWT) access control token, a cookie, or hidden field manipulated to elevate privileges, or abusing JWT invalidation.
  \item CORS misconfiguration that allows API access from unauthorized or untrusted origins.
  \item Force browsing (guessing URLs) to authenticated pages as an unauthenticated user or to privileged pages as a standard user.
\end{itemize}

\section{How to Prevent}
Access control is only effective when implemented in trusted server-side code or serverless APIs, where the attacker cannot modify the access control check or metadata.

\begin{enumerate}
  \item Except for public resources, deny by default.
  \item Implement access control mechanisms once and reuse them throughout the application, including minimizing Cross-Origin Resource Sharing (CORS) usage.
  \item Model access controls should enforce record ownership rather than allowing users to create, read, update, or delete any record.
  \item Unique application business limit requirements should be enforced by domain models.
  \item Disable web server directory listing and ensure file metadata (e.g., \texttt{.git}) and backup files are not present within web roots.
  \item Log access control failures; alert administrators when appropriate (e.g., repeated failures).
  \item Implement rate limits on API and controller access to minimize the harm from automated attack tooling.
  \item Stateful session identifiers should be invalidated on the server after logout. Stateless JWT tokens should be short-lived to minimize the window of opportunity for an attacker. For longer-lived JWTs, consider using refresh tokens and following OAuth standards to revoke access.
  \item Use well-established toolkits or patterns that provide simple, declarative access controls.
  \item Developers and QA staff should include functional access control in their unit and integration tests.
\end{enumerate}

\section{Example Attack Scenarios}

\subsection{Scenario \#1: Parameter Tampering Against Account Lookups}
The application uses unverified data in an SQL call that is accessing account information:

\begin{codebox}[Example Code (Java)]
\begin{lstlisting}[style=code]
pstmt.setString(1, request.getParameter("acct"));
ResultSet results = pstmt.executeQuery( );
\end{lstlisting}
\end{codebox}

An attacker can simply modify the browser's \texttt{acct} parameter to send any desired account number. If not correctly verified, the attacker can access any user's account.

\begin{infobox}[Illustrative Request]
\url{https://example.com/app/accountInfo?acct=notmyacct}
\end{infobox}

\subsection{Scenario \#2: Forced Browsing to Privileged Endpoints}
Admin rights are required for access to the admin page:

\begin{infobox}[Illustrative Requests]
\url{https://example.com/app/getappInfo}\\
\url{https://example.com/app/admin_getappInfo}
\end{infobox}

If an unauthenticated user can access either page, it's a flaw. If a non-admin can access the admin page, this is a flaw.

\subsection{Scenario \#3: Front-End Only Access Control}
An application places all access control in its front-end. While the attacker cannot get to \url{https://example.com/app/admin_getappInfo} due to JavaScript code running in the browser, they can simply execute:

\begin{codebox}[Illustrative Command]
\begin{lstlisting}[style=code]
$ curl https://example.com/app/admin_getappInfo
\end{lstlisting}
\end{codebox}

\section{References}
\begin{itemize}
  \item OWASP Proactive Controls: C1: Implement Access Control
  \item OWASP Application Security Verification Standard: V8 Authorization
  \item OWASP Testing Guide: Authorization Testing
  \item OWASP Cheat Sheet: Authorization
  \item PortSwigger: Exploiting CORS misconfiguration
  \item OAuth: Revoking Access
\end{itemize}

\newpage
\section{List of Mapped CWEs}
\renewcommand{\arraystretch}{1.1}
\begin{longtable}{@{}p{2.2cm}p{12.2cm}@{}}
\toprule
\textbf{CWE} & \textbf{Title} \\
\midrule
\endfirsthead
\toprule
\textbf{CWE} & \textbf{Title} \\
\midrule
\endhead
\bottomrule
\endfoot

CWE-22 & Improper Limitation of a Pathname to a Restricted Directory (\textit{Path Traversal}) \\
CWE-23 & Relative Path Traversal \\
CWE-36 & Absolute Path Traversal \\
CWE-59 & Improper Link Resolution Before File Access (\textit{Link Following}) \\
CWE-61 & UNIX Symbolic Link (Symlink) Following \\
CWE-65 & Windows Hard Link \\
CWE-200 & Exposure of Sensitive Information to an Unauthorized Actor \\
CWE-201 & Exposure of Sensitive Information Through Sent Data \\
CWE-219 & Storage of File with Sensitive Data Under Web Root \\
CWE-276 & Incorrect Default Permissions \\
CWE-281 & Improper Preservation of Permissions \\
CWE-282 & Improper Ownership Management \\
CWE-283 & Unverified Ownership \\
CWE-284 & Improper Access Control \\
CWE-285 & Improper Authorization \\
CWE-352 & Cross-Site Request Forgery (CSRF) \\
CWE-359 & Exposure of Private Personal Information to an Unauthorized Actor \\
CWE-377 & Insecure Temporary File \\
CWE-379 & Creation of Temporary File in Directory with Insecure Permissions \\
CWE-402 & Transmission of Private Resources into a New Sphere (\textit{Resource Leak}) \\
CWE-424 & Improper Protection of Alternate Path \\
CWE-425 & Direct Request (\textit{Forced Browsing}) \\
CWE-441 & Unintended Proxy or Intermediary (\textit{Confused Deputy}) \\
CWE-497 & Exposure of Sensitive System Information to an Unauthorized Control Sphere \\
CWE-538 & Insertion of Sensitive Information into Externally-Accessible File or Directory \\
CWE-540 & Inclusion of Sensitive Information in Source Code \\
CWE-548 & Exposure of Information Through Directory Listing \\
CWE-552 & Files or Directories Accessible to External Parties \\
CWE-566 & Authorization Bypass Through User-Controlled SQL Primary Key \\
CWE-601 & URL Redirection to Untrusted Site (\textit{Open Redirect}) \\
CWE-615 & Inclusion of Sensitive Information in Source Code Comments \\
CWE-639 & Authorization Bypass Through User-Controlled Key \\
CWE-668 & Exposure of Resource to Wrong Sphere \\
CWE-732 & Incorrect Permission Assignment for Critical Resource \\
CWE-749 & Exposed Dangerous Method or Function \\
CWE-862 & Missing Authorization \\
CWE-863 & Incorrect Authorization \\
CWE-918 & Server-Side Request Forgery (SSRF) \\
CWE-922 & Insecure Storage of Sensitive Information \\
CWE-1275 & Sensitive Cookie with Improper SameSite Attribute \\

\end{longtable}

\end{document}

