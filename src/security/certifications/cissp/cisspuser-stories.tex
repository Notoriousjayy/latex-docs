
\documentclass[11pt,a4paper]{article}

% --- Page & typography ---
\usepackage[a4paper,margin=1in]{geometry}
\usepackage{lmodern}
\usepackage[T1]{fontenc}
\usepackage[utf8]{inputenc}
\usepackage{microtype}
\usepackage{parskip}

% --- Structure & lists ---
\usepackage{titlesec}
\titlespacing*{\section}{0pt}{8pt plus 2pt}{6pt}
\titlespacing*{\subsection}{0pt}{6pt}{4pt}
\usepackage{enumitem}
\setlist{itemsep=2pt, topsep=4pt, leftmargin=1.2em}
\setlistdepth{8}
\renewlist{itemize}{itemize}{8}

% Robust description-style "table" to avoid tabularx
\newlist{metadesc}{description}{1}
\setlist[metadesc]{style=nextline, leftmargin=3.7cm, labelwidth=3.2cm, labelsep=0.5cm, font=\bfseries}

% --- Color & links ---
\usepackage[dvipsnames]{xcolor}
\usepackage{hyperref}
\hypersetup{
  colorlinks=true,
  linkcolor=black,
  urlcolor=MidnightBlue,
  citecolor=black,
  pdftitle={CISSP CBK Study Plan — User Story Template (Robust v3)},
  pdfauthor={},
  pdfsubject={User Story Template},
  pdfcreator={LaTeX}
}

% --- Checkbox symbol (portable) ---
\IfFileExists{amssymb.sty}{\usepackage{amssymb}}{}% load if present
\providecommand{\square}{\fbox{\rule{0pt}{0.85ex}\rule{0.85ex}{0pt}}}% fallback if \square undefined
\newcommand{\citem}{\item[$\square$]}

% --- Boxes ---
\usepackage[most]{tcolorbox}
\definecolor{CardFrame}{RGB}{33,37,41}
\definecolor{CardBack}{RGB}{248,249,250}
\definecolor{PillBack}{RGB}{235,242,252}
\definecolor{PillText}{RGB}{15,76,129}

\newtcolorbox{StoryCardBox}[2][]{%
  enhanced, breakable,
  colback=CardBack, colframe=CardFrame,
  boxrule=0.6pt, arc=2mm, outer arc=2mm,
  title={#2}, fonttitle=\large\bfseries,
  toptitle=2mm, bottomtitle=2mm,
  attach boxed title to top left={yshift=-2mm, xshift=0mm},
  boxed title style={colback=white, colframe=CardFrame, boxrule=0.6pt, arc=2mm},
  left=2mm, right=2mm, top=1.5mm, bottom=1.5mm,
  before skip=8pt, after skip=10pt,
  #1}

\newtcolorbox{TasksBox}[1][]{%
  enhanced, breakable,
  colback=white, colframe=CardFrame!60,
  boxrule=0.5pt, arc=2mm, outer arc=2mm,
  title={Tasks}, fonttitle=\bfseries,
  left=2mm, right=2mm, top=1.5mm, bottom=1.5mm,
  before skip=6pt, after skip=8pt,
  #1}

\newcommand{\Tag}[1]{\tcbox[on line, colback=PillBack, colframe=PillBack,
  coltext=PillText, boxrule=0pt, arc=2mm, left=2pt, right=2pt, top=1pt, bottom=1pt]{\small #1}}
\newcommand{\ThinRule}{\par\vspace{4pt}\hrule height 0.4pt\par\vspace{4pt}}

\title{\textbf{Study Plan — The Official (ISC)${}^2$ CISSP CBK Reference (6th Ed.)}\\[2pt]
\large User Story Card Template (Runaway-proof, portable checkboxes)}
\author{}\date{\today}

\begin{document}
\maketitle
\tableofcontents
\clearpage

\section{Story Card Definition (Required Fields)}
\begin{StoryCardBox}{Story Card Definition}
\begin{metadesc}
\item[Epic / Feature] The capability or chapter grouping this story advances (e.g., ``Domain 1: Security \& Risk Management'').
\item[Business Value] The outcome the story enables (e.g., ``risk-based decision-making'' or ``$\geq$ 80\% on domain practice set'').
\item[Priority / Estimate] Priority (Must/Should/Could) and rough size (story points or hours).
\item[Persona] Who benefits or performs the work (``CISSP candidate'').
\item[Dependencies] Prereqs (prior chapters, tools, accounts, baseline knowledge).
\item[Assumptions] What you believe to be true going in (CBK access, practice bank).
\item[Risks] What could block success (limited time, weak crypto background).
\item[Story] \textbf{As a} \emph{[persona]}, \textbf{I want} \emph{[capability]} \textbf{so that} \emph{[business value]}.
\item[Non-Functional] Tags such as \Tag{Security}, \Tag{Reliability}, \Tag{Privacy}, \Tag{Accessibility}.
\item[Acceptance Criteria (BDD)] Use Given/When/Then. Aim for 3--6 criteria that are objectively verifiable.
\item[Definition of Ready] Persona clear; AC drafted; dependencies known; estimate set; scope $\le$ 1 week.
\item[Definition of Done] All AC pass; notes updated; flashcards created; practice set completed; retrospective logged.
\end{metadesc}
\end{StoryCardBox}

\section{Blank Story Card (Copy Me)}
\begin{StoryCardBox}{<ID> --- <Concise Title>}
\begin{metadesc}
\item[Epic / Feature] <Domain or chapter grouping>
\item[Business Value] <Outcome this story enables>
\item[Priority / Estimate] Priority: <Must/Should/Could> \quad SP: <1--5> (or hours)
\item[Persona] <Who benefits / executes>
\item[Dependencies] <Prereqs>
\item[Assumptions] <Starting assumptions>
\item[Risks] <Potential blockers>
\item[Story] \textit{As a <persona>, I want <capability> so that <business value>.}
\item[Non-Functional] \Tag{Security} \Tag{Reliability} \Tag{Privacy} \Tag{Accessibility}
\end{metadesc}

\ThinRule
\textbf{Acceptance Criteria (BDD)}
\begin{metadesc}
\item[Scenario] Happy path
\item[Given] <preconditions>
\item[When] <action or study work is completed>
\item[Then] <observable outcome / evidence>
\end{metadesc}

\ThinRule
\textbf{Definition of Ready:} persona clear; AC drafted; dependencies known; estimate set.\newline
\textbf{Definition of Done:} all AC pass; notes/flashcards updated; practice set completed; retrospective logged.

\ThinRule
\begin{TasksBox}
\begin{itemize}
  \citem <Task 1 (concrete, 15--60 minutes)>
  \citem <Task 2>
  \citem <Task 3>
  \citem <Create 10--20 flashcards; complete 30--50 domain questions>
  \citem <Summarize key confusions for review>
\end{itemize}
\end{TasksBox}
\end{StoryCardBox}


\clearpage
\section{CISSP CBK Domain User Stories}

\begin{StoryCardBox}{CBK-D1 --- Security \& Risk Management}
\begin{metadesc}
\item[Epic / Feature] Domain 1: Security \& Risk Management
\item[Business Value] Establish a risk-based, policy-driven foundation that improves decision-making across all domains.
\item[Priority / Estimate] Priority: Must \quad SP: 3
\item[Persona] CISSP candidate
\item[Dependencies] Access to CBK ch.~1 materials; practice Q bank; note/flashcard system
\item[Assumptions] 4-hour timebox; basic familiarity with risk terms
\item[Risks] Over-scoping legal topics; skipping retrospective
\item[Story] \textit{As a CISSP candidate, I want to master ethics, governance, and risk analysis so that I can justify control selection and compliance tradeoffs.}
\item[Non-Functional] \Tag{Security} \Tag{Reliability} \Tag{Privacy}
\end{metadesc}

\ThinRule
\textbf{Acceptance Criteria (BDD)}
\begin{metadesc}
\item[Scenario] Governance \& risk application
\item[Given] a domain objective list and a risk register template
\item[When] I complete a 1-page summary and compute SLE/ALE for 3 assets
\item[Then] I can map administrative/technical/physical controls to risks and score $\geq$ 80\% on 40 Domain~1 questions
\end{metadesc}

\ThinRule
\textbf{Definition of Ready:} objectives known; templates ready; time-box defined.\newline
\textbf{Definition of Done:} summary saved; 40Q $\geq$ 80\%; flashcards synced; retrospective logged.

\ThinRule
\begin{TasksBox}
\begin{itemize}
  \citem Capture ethics, due care/diligence, and policy hierarchy on 1 page.
  \citem Build a mini risk register with 3 assets, threats, impacts, and controls.
  \citem Draft 15 flashcards (CIANA, risk appetite, residual risk, e-Discovery, BCP/DR).
  \citem Do 40 mixed Domain~1 questions; tag misses by subtopic.
  \citem Write a 5-sentence takeaway on governance vs management and supply-chain risk.
\end{itemize}
\end{TasksBox}
\end{StoryCardBox}

\clearpage
\begin{StoryCardBox}{CBK-D2 --- Asset Security}
\begin{metadesc}
\item[Epic / Feature] Domain 2: Asset Security
\item[Business Value] Correctly classify and handle data/assets through their lifecycle to meet legal and business needs.
\item[Priority / Estimate] Priority: Must \quad SP: 2
\item[Persona] CISSP candidate
\item[Dependencies] D1 foundations
\item[Assumptions] Access to sample classification policy
\item[Risks] Unclear ownership roles; mishandling retention
\item[Story] \textit{As a CISSP candidate, I want to design a classification and handling scheme so that data is protected appropriately at each lifecycle stage.}
\item[Non-Functional] \Tag{Security} \Tag{Privacy} \Tag{Compliance}
\end{metadesc}

\ThinRule
\textbf{Acceptance Criteria (BDD)}
\begin{metadesc}
\item[Scenario] Classification and handling
\item[Given] a 4-level classification model and role definitions
\item[When] I label owners/custodians and define handling/storage/destruction
\item[Then] I can choose masking/tokenization/DLP strategies and pass 30 Domain~2 questions at $\geq$ 80\%
\end{metadesc}

\ThinRule
\begin{TasksBox}
\begin{itemize}
  \citem Draft a 4-level classification schema with owners/custodians.
  \citem Map lifecycle stages to controls (create, store, use, share, archive, destroy).
  \citem Create 10 flashcards (tokenization vs encryption, media sanitization levels).
  \citem Complete 30 Domain~2 questions; review missed explanations.
  \citem Write retention and disposal rules for each class.
\end{itemize}
\end{TasksBox}
\end{StoryCardBox}

\clearpage
\begin{StoryCardBox}{CBK-D3 --- Security Architecture \& Engineering}
\begin{metadesc}
\item[Epic / Feature] Domain 3: Security Architecture \& Engineering
\item[Business Value] Engineer resilient systems by applying secure design principles, models, and cryptography.
\item[Priority / Estimate] Priority: Must \quad SP: 3
\item[Persona] CISSP candidate
\item[Dependencies] D1 concepts
\item[Assumptions] Time to sketch architecture diagrams
\item[Risks] Confusing model goals (Bell-LaPadula vs Biba); crypto misuse
\item[Story] \textit{As a CISSP candidate, I want to apply secure design and crypto choices so that architectures resist common failure and attack modes.}
\item[Non-Functional] \Tag{Security} \Tag{Reliability}
\end{metadesc}

\ThinRule
\textbf{Acceptance Criteria (BDD)}
\begin{metadesc}
\item[Scenario] Architecture decision
\item[Given] an app with data flows across trust boundaries
\item[When] I annotate controls (TCB, hardware roots, virtualization, key mgmt)
\item[Then] I explain which model enforces confidentiality/integrity and select safe crypto modes for the use case
\end{metadesc}

\ThinRule
\begin{TasksBox}
\begin{itemize}
  \citem List 10 secure design principles; give an example for each.
  \citem Summarize Bell-LaPadula, Biba, Clark-Wilson, Brewer-Nash in 6 lines each.
  \citem Create 12 flashcards (block modes, AEAD, key lifecycles, HSM/TPM).
  \citem Draw a simple architecture and mark control points and trust boundaries.
  \citem Do 35 Domain~3 questions; target weak subtopics.
\end{itemize}
\end{TasksBox}
\end{StoryCardBox}

\clearpage
\begin{StoryCardBox}{CBK-D4 --- Communication \& Network Security}
\begin{metadesc}
\item[Epic / Feature] Domain 4: Communication \& Network Security
\item[Business Value] Design and defend segmented networks and secure communications.
\item[Priority / Estimate] Priority: Must \quad SP: 2
\item[Persona] CISSP candidate
\item[Dependencies] D3 overview
\item[Assumptions] Diagram tool available
\item[Risks] Layer confusion; misplacing controls
\item[Story] \textit{As a CISSP candidate, I want to map threats to layered network controls so that I can justify TLS vs IPsec and wireless protections.}
\item[Non-Functional] \Tag{Security} \Tag{Reliability}
\end{metadesc}

\ThinRule
\textbf{Acceptance Criteria (BDD)}
\begin{metadesc}
\item[Scenario] Segmentation and secure comms
\item[Given] a 3-tier app topology
\item[When] I produce a segmented diagram with FW/IDS/IPS/WAF/NAC annotations
\item[Then] I explain TLS vs IPsec tradeoffs and pass 30 Domain~4 questions at $\geq$ 80\%
\end{metadesc}

\ThinRule
\begin{TasksBox}
\begin{itemize}
  \citem Draw a 3-zone network; mark control points.
  \citem Write 1 paragraph: TLS vs IPsec use cases.
  \citem Create 10 flashcards on Wi-Fi protections and common network attacks.
  \citem Complete 30 Domain~4 questions; review misses.
  \citem Note 5 troubleshooting cues per control (e.g., SSL/TLS versions, cipher suites).
\end{itemize}
\end{TasksBox}
\end{StoryCardBox}

\clearpage
\begin{StoryCardBox}{CBK-D5 --- Identity \& Access Management}
\begin{metadesc}
\item[Epic / Feature] Domain 5: Identity \& Access Management
\item[Business Value] Govern identities and enforce strong authentication and authorization.
\item[Priority / Estimate] Priority: Should \quad SP: 2
\item[Persona] CISSP candidate
\item[Dependencies] D1 policy concepts
\item[Assumptions] Example IdP diagrams
\item[Risks] Confusing federation flows; weak lifecycle governance
\item[Story] \textit{As a CISSP candidate, I want to compare authN/authZ patterns and lifecycle governance so that I can choose RBAC/ABAC and federation appropriately.}
\item[Non-Functional] \Tag{Security} \Tag{Privacy}
\end{metadesc}

\ThinRule
\textbf{Acceptance Criteria (BDD)}
\begin{metadesc}
\item[Scenario] IAM selection
\item[Given] human, device, and service identities
\item[When] I diagram joiner/mover/leaver and federation (SAML/OIDC) flows
\item[Then] I justify MFA/SSO/PAM choices and pass 30 Domain~5 questions at $\geq$ 80\%
\end{metadesc}

\ThinRule
\begin{TasksBox}
\begin{itemize}
  \citem Sketch IdP/SP trust and token flows (SAML/OIDC).
  \citem Define RBAC vs ABAC and give 2 examples each.
  \citem Create 12 flashcards (MFA factors, OAuth scopes, session mgmt).
  \citem Complete 30 Domain~5 questions.
  \citem Write lifecycle governance checklist (JML, reviews, recertification).
\end{itemize}
\end{TasksBox}
\end{StoryCardBox}

\clearpage
\begin{StoryCardBox}{CBK-D6 --- Security Assessment \& Testing}
\begin{metadesc}
\item[Epic / Feature] Domain 6: Security Assessment \& Testing
\item[Business Value] Verify control effectiveness and communicate prioritized remediation.
\item[Priority / Estimate] Priority: Should \quad SP: 2
\item[Persona] CISSP candidate
\item[Dependencies] D1 risk context
\item[Assumptions] Access to sample reports
\item[Risks] Confusing audit independence vs testing
\item[Story] \textit{As a CISSP candidate, I want to plan and execute assessments so that I can select tests, analyze results, and support audits.}
\item[Non-Functional] \Tag{Security} \Tag{Reliability}
\end{metadesc}

\ThinRule
\textbf{Acceptance Criteria (BDD)}
\begin{metadesc}
\item[Scenario] Assessment planning
\item[Given] a scoped system and control list
\item[When] I choose technical/administrative/physical tests and sampling
\item[Then] I produce a short report with prioritized fixes and pass 25 Domain~6 questions at $\geq$ 80\%
\end{metadesc}

\ThinRule
\begin{TasksBox}
\begin{itemize}
  \citem Map control types to test techniques; build a tiny test matrix.
  \citem Draft 8 report bullets: findings, risk, recommendation, owner.
  \citem Create 8 flashcards (sampling, coverage, MTTD/MTTR proxies).
  \citem Complete 25 Domain~6 questions.
  \citem Compare audit vs assessment in 5 bullets.
\end{itemize}
\end{TasksBox}
\end{StoryCardBox}

\clearpage
\begin{StoryCardBox}{CBK-D7 --- Security Operations}
\begin{metadesc}
\item[Epic / Feature] Domain 7: Security Operations
\item[Business Value] Run daily security: monitoring, IR, change/config mgmt, continuity, and safety.
\item[Priority / Estimate] Priority: Must \quad SP: 3
\item[Persona] CISSP candidate
\item[Dependencies] D1/D6
\item[Assumptions] Access to IR checklist template
\item[Risks] Skipping BCP/DR testing steps
\item[Story] \textit{As a CISSP candidate, I want to coordinate incident response and operational controls so that I can minimize impact and recover effectively.}
\item[Non-Functional] \Tag{Security} \Tag{Reliability}
\end{metadesc}

\ThinRule
\textbf{Acceptance Criteria (BDD)}
\begin{metadesc}
\item[Scenario] Incident drill
\item[Given] an IR playbook and logs
\item[When] I walk through prepare/detect/contain/eradicate/recover
\item[Then] I define evidence handling basics and pass 35 Domain~7 questions at $\geq$ 80\%
\end{metadesc}

\ThinRule
\begin{TasksBox}
\begin{itemize}
  \citem Outline IR lifecycle with roles and SLAs.
  \citem List 10 monitoring use cases and associated signals.
  \citem Create 12 flashcards (EDR, NAC, WAF, backups, RTO/RPO).
  \citem Complete 35 Domain~7 questions.
  \citem Draft a DR test plan (tabletop + restore verification).
\end{itemize}
\end{TasksBox}
\end{StoryCardBox}

\clearpage
\begin{StoryCardBox}{CBK-D8 --- Software Development Security}
\begin{metadesc}
\item[Epic / Feature] Domain 8: Software Development Security
\item[Business Value] Integrate security into SDLC and supply chain; enforce secure coding.
\item[Priority / Estimate] Priority: Should \quad SP: 2
\item[Persona] CISSP candidate
\item[Dependencies] D3 architecture basics
\item[Assumptions] Access to a CI/CD checklist
\item[Risks] Overemphasis on tools vs controls
\item[Story] \textit{As a CISSP candidate, I want to embed security in the SDLC so that I can assess software and manage supply-chain risk.}
\item[Non-Functional] \Tag{Security} \Tag{Reliability}
\end{metadesc}

\ThinRule
\textbf{Acceptance Criteria (BDD)}
\begin{metadesc}
\item[Scenario] Secure SDLC checklist
\item[Given] an SDLC stage map (req $\rightarrow$ design $\rightarrow$ build $\rightarrow$ test $\rightarrow$ deploy)
\item[When] I place security activities and artifacts per stage
\item[Then] I review SAST/DAST/IAST/SCA basics and pass 30 Domain~8 questions at $\geq$ 80\%
\end{metadesc}

\ThinRule
\begin{TasksBox}
\begin{itemize}
  \citem Build a secure-SDLC checklist across stages and environments.
  \citem Draft 8 code review checks tied to common CWE classes.
  \citem Create 10 flashcards (SAST vs DAST vs IAST vs SCA; SBOM; secrets mgmt).
  \citem Complete 30 Domain~8 questions.
  \citem Evaluate an acquisition scenario for license \& supply-chain risk.
\end{itemize}
\end{TasksBox}
\end{StoryCardBox}

\end{document}
