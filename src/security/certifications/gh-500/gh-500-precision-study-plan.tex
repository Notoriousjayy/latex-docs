% !TeX program = pdflatex
\documentclass[11pt]{article}

\usepackage[margin=1in]{geometry}

% Fonts and encoding (helps PDF metadata + avoids font substitution noise)
\usepackage[T1]{fontenc}
\usepackage{lmodern}

\usepackage{microtype}
\usepackage{enumitem}
\usepackage{booktabs}
\usepackage{xcolor}
\usepackage{titlesec}
% Hyperlinks (load late)
\usepackage[pdfencoding=auto, psdextra]{hyperref}
\usepackage{xurl} % better URL line breaking (prevents overfull hboxes)

\hypersetup{
  colorlinks=true,
  linkcolor=blue,
  urlcolor=blue,
  citecolor=blue
}


% Reduce the chance of overfull boxes caused by long URLs
\setlength{\emergencystretch}{2em}
\Urlmuskip=0mu plus 1mu

% Consistent, click-safe documentation link item:
% - label on its own line
% - URL on its own line (breakable and clickable)
\newcommand{\DocLink}[2]{%
  \item \textbf{#1}\par
  \noindent\href{#2}{\nolinkurl{#2}}%
}

\setlist[itemize]{leftmargin=*, itemsep=0.35em, topsep=0.35em}
\setlist[enumerate]{leftmargin=*, itemsep=0.35em, topsep=0.35em}

\title{\textbf{GH-500 Precision Study Plan}\\
\large Final 72-Hour Plan (Retake: Tue 27 Jan 2026)}
\author{Jordan Suber}
\date{\today}

\begin{document}
\maketitle

\begin{abstract}
This document is a precision study plan for the Microsoft \emph{Exam GH-500: GitHub Advanced Security}.
It translates the score report into targeted actions for the final 72 hours before the retake, with official documentation links embedded throughout.
The plan prioritizes the highest-return knowledge areas: (1) results interpretation and best practices, (2) dependency vulnerability management tooling, and (3) dependency remediation workflows, while maintaining readiness in Secret Scanning and Code Scanning with CodeQL.
\end{abstract}

\section{Scope Control (Authoritative References)}
Use these pages as the definitive scope and weighting signals for what to study:

\begin{itemize}
  \DocLink{GH-500 Study Guide (skills measured + weighting)}{https://learn.microsoft.com/en-us/credentials/certifications/resources/study-guides/gh-500}
  \DocLink{GitHub Advanced Security certification resources (GitHub Learn)}{https://learn.github.com/certification/GHAS}
  \DocLink{Credential overview (Microsoft Learn)}{https://learn.microsoft.com/en-us/credentials/certifications/github-advanced-security/}
\end{itemize}

\section{Score Report Interpretation (What the Data Means)}
Your section-level chart indicates the following qualitative performance pattern:

\begin{itemize}
  \item \textbf{Strongest}: Secret Scanning; Code Scanning with CodeQL (longer bars).
  \item \textbf{Weakest}: GHAS results, best practices, and corrective measures (shortest bar).
  \item \textbf{Second weakness cluster}: Configure and use Dependabot and Dependency Review.
  \item \textbf{Mid-tier}: Describe the GHAS security features and functionality.
\end{itemize}

\noindent The report also explicitly flags that section bars are not directly additive and do not map cleanly to percent correct due to varying question counts. Treat the chart as a \emph{priority signal}, not a score calculator.

\section{Strategy: What You Are Optimizing For}
The GH-500 exam is heavily scenario-driven. The goal is decision accuracy under constraints.

\begin{enumerate}
  \item \textbf{Correct capability selection:} Given a scenario, choose the correct GHAS feature/tool and where it is configured.
  \item \textbf{Correct next action:} Given an alert or insight, select the best operational next step (triage, remediate, dismiss, enforce).
  \item \textbf{Program-level fluency:} Translate results into corrective measures at scale (coverage gaps, backlog, noise reduction, enforcement).
\end{enumerate}

\section{The Non-Negotiable Review Loop (Use This for Every Practice Set)}
For \textbf{every} missed or guessed question, capture:

\begin{enumerate}
  \item \textbf{Question type} (use the taxonomy in Sections \ref{sec:qtypes-supply} and \ref{sec:qtypes-results}).
  \item \textbf{Decision rule} in one sentence: ``When \emph{X}, choose \emph{Y} because \emph{Z}.'' 
  \item \textbf{Doc anchor} (the specific official page you re-read to lock the rule).
\end{enumerate}

\noindent This forces exam-aligned thinking: tool selection + location + next best action.

\clearpage
\section{Final 72-Hour Precision Plan (Sat--Mon)}
This plan assumes the exam retake is on \textbf{Tue 27 Jan 2026}. Adjust times, not priorities.


\subsection*{Day 1 (T--72 to T--48): Supply Chain Security Mastery (Highest ROI)}
This is the most efficient area for score movement because it is both a weakness and a core GH-500 domain.

\paragraph{Official documentation (read with intent).}
\begin{itemize}
  \DocLink{Dependabot quickstart (feature map)}{https://docs.github.com/en/code-security/tutorials/secure-your-dependencies/dependabot-quickstart-guide}
  \DocLink{Supported ecosystems and repositories}{https://docs.github.com/en/code-security/reference/supply-chain-security/supported-ecosystems-and-repositories}
  \DocLink{Configure Dependabot version updates (dependabot.yml)}{https://docs.github.com/en/code-security/how-tos/secure-your-supply-chain/secure-your-dependencies/configuring-dependabot-version-updates}
  \DocLink{Dependabot options reference}{https://docs.github.com/en/code-security/reference/supply-chain-security/dependabot-options-reference}
  \DocLink{Viewing and updating Dependabot alerts}{https://docs.github.com/en/code-security/how-tos/manage-security-alerts/manage-dependabot-alerts/viewing-and-updating-dependabot-alerts}
  \DocLink{Customizing security update PRs}{https://docs.github.com/en/code-security/how-tos/secure-your-supply-chain/manage-your-dependency-security/customizing-dependabot-security-prs}
  \DocLink{About dependency review}{https://docs.github.com/en/code-security/concepts/supply-chain-security/about-dependency-review}
  \DocLink{Dependency Review Action (configuration)}{https://docs.github.com/en/code-security/how-tos/secure-your-supply-chain/manage-your-dependency-security/configuring-the-dependency-review-action}
  \DocLink{Customize Dependency Review Action config}{https://docs.github.com/en/code-security/supply-chain-security/understanding-your-software-supply-chain/customizing-your-dependency-review-action-configuration}
\end{itemize}

\paragraph{Practice focus.}
\begin{itemize}
  \item Complete \textbf{1 timed} practice exam or section set emphasizing supply chain questions.
  \item Review using the Non-Negotiable Loop. Re-read only the relevant doc anchors above.
\end{itemize}

\paragraph{Deliverable (end of Day 1): One-page Supply Chain Decision Sheet.}
Include these mappings:

\begin{itemize}
  \item \textbf{Detection vs prevention vs remediation automation:}
  \begin{itemize}
    \item Dependency Graph (inventory/context)
    \item Dependabot alerts (vuln detection)
    \item Dependabot security updates (automated remediation PRs)
    \item Dependency review (PR-time visibility)
    \item Dependency Review Action (PR-time enforcement)
  \end{itemize}
  \item \textbf{Where configuration lives:} repository settings vs \texttt{.github/dependabot.yml} vs Actions workflow vs org/repo rules.
\end{itemize}

\clearpage
\subsection*{Day 2 (T--48 to T--24): Results, Best Practices, Corrective Measures (Biggest Gap)}
This is the shortest bar in your section performance chart, and it frequently appears as scenario questions.

\paragraph{Official documentation (read with intent).}
\begin{itemize}
  \DocLink{About security overview}{https://docs.github.com/en/code-security/concepts/security-at-scale/about-security-overview}
  \DocLink{Viewing security insights}{https://docs.github.com/en/code-security/how-tos/view-and-interpret-data/analyze-organization-data/viewing-security-insights}
  \DocLink{Filtering alerts in security overview}{https://docs.github.com/en/code-security/how-tos/manage-security-alerts/remediate-alerts-at-scale/filtering-alerts-in-security-overview}
  \DocLink{Triaging alerts in security overview}{https://docs.github.com/en/code-security/how-tos/manage-security-alerts/remediate-alerts-at-scale/review-alert-dismissal-requests}
  \DocLink{Assessing adoption of security features}{https://docs.github.com/en/code-security/security-overview/assessing-adoption-code-security}
\end{itemize}

\paragraph{Practice focus.}
\begin{itemize}
  \item Complete \textbf{1 targeted} practice set focused on ``interpret results $\rightarrow$ corrective action''.
  \item Force yourself to answer in one sentence: \emph{What should the organization do next?}
\end{itemize}

\paragraph{Deliverable (end of Day 2): One-page Results $\rightarrow$ Actions Map.}
At minimum, include:

\begin{itemize}
  \item Coverage gaps (missing scans/protection) $\rightarrow$ enablement plan.
  \item Backlog/MTTR issues $\rightarrow$ ownership, SLAs, and workflow enforcement.
  \item Excess noise/false positives $\rightarrow$ tuning, rule updates, and policy refinement.
\end{itemize}

\clearpage
\subsection*{Day 3 (T--24 to T--0): Integration + Protect Strength Areas}
\paragraph{Step 1: Full simulation.}
Complete \textbf{1 full timed} practice exam. Review only wrong and guessed answers.

\paragraph{Step 2: Keep-warm blocks.}
Do short, high-yield refreshers to protect points in areas where you are already stronger.

\subsubsection*{Secret Scanning (30--45 minutes)}
\begin{itemize}
  \DocLink{About secret scanning}{https://docs.github.com/en/code-security/concepts/secret-security/about-secret-scanning}
  \DocLink{About push protection}{https://docs.github.com/en/code-security/concepts/secret-security/about-push-protection}
  \DocLink{Managing secret scanning alerts}{https://docs.github.com/en/code-security/how-tos/manage-security-alerts/manage-secret-scanning-alerts}
  \DocLink{Resolving secret scanning alerts}{https://docs.github.com/en/code-security/how-tos/manage-security-alerts/manage-secret-scanning-alerts/resolving-alerts}
  \DocLink{Supported secret scanning patterns}{https://docs.github.com/en/code-security/secret-scanning/introduction/supported-secret-scanning-patterns}
\end{itemize}

\subsubsection*{Code Scanning with CodeQL (45--60 minutes)}
\begin{itemize}
  \DocLink{CodeQL code scanning overview}{https://docs.github.com/en/code-security/concepts/code-scanning/codeql/about-code-scanning-with-codeql}
  \DocLink{Configuring default setup}{https://docs.github.com/en/code-security/how-tos/scan-code-for-vulnerabilities/configure-code-scanning/configuring-default-setup-for-code-scanning}
  \DocLink{Configuring advanced setup}{https://docs.github.com/en/code-security/how-tos/scan-code-for-vulnerabilities/configure-code-scanning/configuring-advanced-setup-for-code-scanning}
  \DocLink{Managing code scanning alerts}{https://docs.github.com/en/code-security/how-tos/manage-security-alerts/manage-code-scanning-alerts}
  \DocLink{Resolving code scanning alerts}{https://docs.github.com/en/code-security/how-tos/manage-security-alerts/manage-code-scanning-alerts/resolving-code-scanning-alerts}
  \DocLink{Uploading SARIF results}{https://docs.github.com/en/code-security/how-tos/scan-code-for-vulnerabilities/integrate-with-existing-tools/uploading-a-sarif-file-to-github}
\end{itemize}

\clearpage
\section{Exam Day (Tue 27 Jan 2026): Warm-Up Without Fatigue}
\begin{itemize}
  \item Do a short warm-up: 15--30 questions total (not a full exam).
  \item Re-read your two one-page sheets: Supply Chain Decision Sheet; Results $\rightarrow$ Actions Map.
  \item Skim the GH-500 skills measured list one last time: \href{https://learn.microsoft.com/en-us/credentials/certifications/resources/study-guides/gh-500}{GH-500 Study Guide}.
\end{itemize}

\section{Question Type Taxonomy (What to Drill)}
\subsection{Supply Chain Security Question Types}\label{sec:qtypes-supply}
Prioritize these until they become automatic:

\begin{enumerate}
  \item \textbf{Tool selection:} Which capability is correct (Dependency Graph vs alerts vs security updates vs review vs review action)?
  \item \textbf{Configuration location:} Where is it configured (repo setting, \texttt{dependabot.yml}, Actions workflow, or policy/ruleset)?
  \item \textbf{Best next action:} Given a vulnerable dependency, what should be done first (create/merge PR, manual bump, policy enforcement, or dismissal with justification)?
  \item \textbf{Ecosystem fit:} Does Dependabot support this ecosystem/repo structure? (\href{https://docs.github.com/en/code-security/reference/supply-chain-security/supported-ecosystems-and-repositories}{Supported ecosystems})
  \item \textbf{PR-time gating:} How to block risky dependency changes in PRs (Dependency Review Action + required checks/rules).
\end{enumerate}

\subsection{Results, Best Practices, and Corrective Measures Question Types}\label{sec:qtypes-results}
\begin{enumerate}
  \item \textbf{Interpretation $\rightarrow$ action:} What is the corrective measure suggested by a dashboard/insight trend?
  \item \textbf{Coverage and adoption:} Which repos lack a control, and how do you standardize enablement?
  \item \textbf{Backlog management:} How to reduce MTTR/queue depth with ownership and workflow enforcement.
  \item \textbf{Noise reduction:} When to tune configuration, apply filters, or adjust processes to reduce false positives.
\end{enumerate}

\section{Practice Exams: How to Use All 9 Efficiently}
Recommended allocation:

\begin{itemize}
  \item \textbf{3 timed full simulations}: one each on Day 1, Day 3, and optionally the morning before the exam.
  \item \textbf{6 targeted drills}: supply chain and results/best practices, using the Non-Negotiable Loop.
\end{itemize}

\noindent \textbf{Rule:} every answer must include (a) feature/tool, (b) where it is configured/viewed, and (c) best next action.

\section{Appendix: Quick Link Index (Official Docs)}
Use this as a fast lookup list for ``what page do I re-open when I forget X?''

\begin{itemize}
  \DocLink{Exam scope}{https://learn.microsoft.com/en-us/credentials/certifications/resources/study-guides/gh-500}
  \DocLink{Security at scale}{https://docs.github.com/en/code-security/concepts/security-at-scale/about-security-overview}
  \DocLink{Supply chain (overview)}{https://docs.github.com/en/code-security/tutorials/secure-your-dependencies/dependabot-quickstart-guide}
  \DocLink{Dependabot config}{https://docs.github.com/en/code-security/how-tos/secure-your-supply-chain/secure-your-dependencies/configuring-dependabot-version-updates}
  \DocLink{Dependabot options}{https://docs.github.com/en/code-security/reference/supply-chain-security/dependabot-options-reference}
  \DocLink{Dependency review}{https://docs.github.com/en/code-security/concepts/supply-chain-security/about-dependency-review}
  \DocLink{Dependency review action}{https://docs.github.com/en/code-security/how-tos/secure-your-supply-chain/manage-your-dependency-security/configuring-the-dependency-review-action}
  \DocLink{Secret scanning}{https://docs.github.com/en/code-security/concepts/secret-security/about-secret-scanning}
  \DocLink{Push protection}{https://docs.github.com/en/code-security/concepts/secret-security/about-push-protection}
  \DocLink{Code scanning}{https://docs.github.com/en/code-security/concepts/code-scanning/codeql/about-code-scanning-with-codeql}
\end{itemize}

\section{Official GitHub Documentation (Direct Links)}
These are canonical GitHub Docs entry points you can rely on throughout GH-500 preparation. Each link is a fully-qualified URL (no relative pathing).

\begin{itemize}
  \DocLink{GitHub Advanced Security (overview)}{https://docs.github.com/en/get-started/learning-about-github/about-github-advanced-security}
  \DocLink{Security and code quality hub}{https://docs.github.com/code-security}
  \DocLink{GitHub security features (index)}{https://docs.github.com/en/code-security/getting-started/github-security-features}
  \DocLink{Quickstart for securing your repository}{https://docs.github.com/en/code-security/getting-started/quickstart-for-securing-your-repository}
\end{itemize}



\end{document}
