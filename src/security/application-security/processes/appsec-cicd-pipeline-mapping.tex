% !TEX TS-program = pdflatex
% Compile with: pdflatex -shell-escape <file>.tex
\documentclass[11pt]{article}

% -------------------- Packages --------------------
\usepackage[T1]{fontenc}
\usepackage[utf8]{inputenc}
\usepackage{lmodern}
\usepackage[a4paper,margin=1in]{geometry}
\usepackage{microtype}
\usepackage{parskip}          % no paragraph indents
\usepackage[dvipsnames]{xcolor}
\usepackage{hyperref}
\hypersetup{
  colorlinks=true,
  linkcolor=MidnightBlue,
  urlcolor=MidnightBlue,
  citecolor=MidnightBlue
}
\usepackage{enumitem}
\setlist{itemsep=2pt,topsep=4pt,leftmargin=1.2em}
\usepackage{booktabs}
\usepackage{tabularx}
\usepackage{longtable}
\usepackage{array}
\usepackage{titlesec}
% --- minted (CI-safe fallback) ----------------------------------------
\newif\ifciwithshellescape
\ifdefined\pdfshellescape
  \ifnum\pdfshellescape=1\relax
    \ciwithshellescapetrue
  \else
    \ciwithshellescapefalse
  \fi
\else
  \ciwithshellescapefalse
\fi

\ifciwithshellescape
  \usepackage[newfloat]{minted}
\else
  % Fallback: compile without -shell-escape / Pygments.
  % This preserves document build determinism in CI (no syntax highlighting).
  \usepackage{listings}
  \usepackage{xcolor}
  \usepackage{textcomp}

  \lstset{
    basicstyle=\ttfamily\small,
    columns=fullflexible,
    keepspaces=true,
    breaklines=true,
    breakatwhitespace=true,
    upquote=true,
    showstringspaces=false,
    literate=
      {•}{{\textbullet}}1
      {–}{{--}}1
      {—}{{---}}1
      {…}{{\ldots}}1
      {“}{{``}}1
      {”}{{''}}1
      {‘}{{`}}1
      {’}{{'}}1
  }

  % minted compatibility shims (options and language are ignored)
  \lstnewenvironment{minted}[2][]{\lstset{}}{}
  \newcommand{\inputminted}[3][]{\lstinputlisting{#3}}
  \newcommand{\mintinline}[2]{\texttt{#2}}
  \providecommand{\setminted}[2][]{}
  \providecommand{\setmintedinline}[2][]{}
  \providecommand{\usemintedstyle}[1]{}
  \providecommand{\newminted}[2][]{}
  \providecommand{\newmintedfile}[2][]{}
  \providecommand{\SetupFloatingEnvironment}[2][]{}

  % Provide a 'listing' float if the document expects it (minted[newfloat]).
  \usepackage{float}
  \makeatletter
  \@ifundefined{c@listing}{%
    \newfloat{listing}{tbp}{lol}
    \floatname{listing}{Listing}
  }{}
  \makeatother
  \providecommand{\listoflistings}{\listof{listing}{List of Listings}}
\fi
% ----------------------------------------------------------------------

\setminted{cache=false,breaklines=true,autogobble=true,fontsize=\small}

% column helpers
\newcolumntype{Y}{>{\raggedright\arraybackslash}p{0.22\textwidth}}
\newcolumntype{S}{>{\raggedright\arraybackslash}p{0.28\textwidth}}
\newcolumntype{E}{>{\raggedright\arraybackslash}p{0.26\textwidth}}

% -------------------- Document --------------------
\begin{document}

\begin{center}
  {\LARGE \textbf{Mapping the Five AppSec Core Processes to a 16-Gate CI/CD Pipeline}}\\[4pt]
  {\small Version 1.1}
\end{center}

\section*{Overview}
This document maps the \textbf{five core AppSec processes} (\textit{Plan/Design, Build, Test, Release, Operate}) to \textbf{16 CI/CD gates}. Each gate lists its primary process, security intent, and example evidence.

\section{Gate \texorpdfstring{$\rightarrow$}{→} AppSec Process Mapping}
\small
\setlength{\LTpre}{6pt}
\setlength{\LTpost}{6pt}
\begin{longtable}{@{}p{0.06\textwidth} Y p{0.13\textwidth} S E@{}}
\toprule
\# & CI/CD Gate & Primary AppSec Process & What this gate enforces (AppSec intent) & Typical evidence / signals \\
\midrule
\endfirsthead
\toprule
\# & CI/CD Gate & Primary AppSec Process & What this gate enforces (AppSec intent) & Typical evidence / signals \\
\midrule
\endhead
01 & Source code version control
   & \textbf{Build}
   & Protected branches; required reviews; signed commits; CODEOWNERS; secret push-protection.
   & Repo settings export; audit log; PR policy status. \\

02 & Optimum branching strategy
   & \textbf{Build}
   & PR-centric flow; short-lived branches; enforced checks before merge.
   & Branch protection rules; PR template; required checks list. \\

12 & Build / deploy / test each commit
   & \textbf{Build}
   & Reproducible builds; pinned actions; secretless OIDC auth; deterministic artifacts.
   & Workflow run logs; build provenance/attestation. \\

04 & $\geq$80\% code coverage
   & \textbf{Build}
   & Minimum unit-test coverage threshold per service.
   & Coverage report artifact; hard fail if below threshold. \\

03 & Static analysis (SAST)
   & \textbf{Build}
   & PR checks for code flaws and secrets; severity thresholds/gating.
   & SAST report; secret-scan report; PR check status. \\

05 & Vulnerability scan
   & \textbf{Build}
   & Dependency/container CVE policy by severity, age, and SLA.
   & SBOM + scan results; allow/deny decision trail. \\

06 & Open-source (SCA / license) scan
   & \textbf{Build}
   & License and component policy compliance.
   & SCA license report; approved/exception record. \\

07 & Artifact version control
   & \textbf{Release}
   & Immutable, signed, provenance-attested artifacts (supply chain).
   & Image digest; signature (e.g., Sigstore); SLSA-like attestations. \\

08 & Auto provision (IaC)
   & \textbf{Operate}
   & Baseline-hardened infrastructure via IaC; policy-as-code on plans.
   & OPA/Conftest results; plan/apply logs. \\

09 & Immutable servers
   & \textbf{Operate}
   & Golden images/immutable containers; drift prevention.
   & Image recipe; container digest pinning; drift alerts. \\

10 & Integration testing
   & \textbf{Test}
   & Security-relevant integration/API tests from misuse cases.
   & Integration test suite results; contract tests; negative tests. \\

11 & Performance / load testing
   & \textbf{Test}
   & Performance/SLO thresholds as DoS guardrails.
   & Load test report vs.\ SLOs; error budgets. \\

14 & Automated change order
   & \textbf{Release}
   & Change governance links risk posture to approvals using objective evidence.
   & Change record referencing scans, SBOM, exceptions. \\

15 & Zero-downtime release
   & \textbf{Release}
   & Progressive rollout (blue/green, canary) with health guardrails.
   & Deployment strategy logs; health-gate status. \\

16 & Feature toggle
   & \textbf{Release}
   & Progressive delivery and kill-switch controls.
   & Toggle audit log; scoped rollout policy. \\

13 & Automated rollback
   & \textbf{Operate}
   & Auto-revert on SLO/SI breach; incident linkage.
   & Rollback trigger tied to SLOs; incident/alert record. \\
\bottomrule
\end{longtable}
\normalsize

\section{Process \texorpdfstring{$\rightarrow$}{→} Gates Index}
\begin{description}[leftmargin=1.4em,labelsep=0.5em,style=nextline]
  \item[\textbf{Plan/Design}] Establishes policies and thresholds used by all gates (especially 01--06 and 08--16).
  \item[\textbf{Build}] \textbf{01, 02, 12, 04, 03, 05, 06}
  \item[\textbf{Test}] \textbf{10, 11}
  \item[\textbf{Release}] \textbf{07, 14, 15, 16}
  \item[\textbf{Operate}] \textbf{08, 09, 13}
\end{description}

\clearpage
\section{Reusable Mapping}
The following block can live in a repo/wiki and be validated by automation.
\begin{minted}{yaml}
appsec_to_cicd_gates:
  - gate: 01
    name: Source code version control
    primary_process: Build
    intent: "Repo protections, reviews, signed commits, secret push-protection"
    evidence: ["branch_protection_export", "audit_log", "required_checks_status"]
  - gate: 02
    name: Optimum branching strategy
    primary_process: Build
    intent: "PR-centric flow; enforce checks before merge"
    evidence: ["PR_template", "branch_rules", "required_checks"]
  - gate: 12
    name: Build/deploy/test each commit
    primary_process: Build
    intent: "Reproducible, pinned, secretless builds"
    evidence: ["workflow_logs", "build_attestation"]
  - gate: 04
    name: ">=80% coverage"
    primary_process: Build
    intent: "Test coverage threshold"
    evidence: ["coverage_report"]
  - gate: 03
    name: Static analysis (SAST)
    primary_process: Build
    intent: "SAST + secrets on PR; severity gating"
    evidence: ["sast_report", "secrets_report", "check_status"]
  - gate: 05
    name: Vulnerability scan
    primary_process: Build
    intent: "Dependency/container vuln policy"
    evidence: ["sbom", "vuln_scan_results"]
  - gate: 06
    name: Open source scan
    primary_process: Build
    intent: "License/composition compliance"
    evidence: ["sca_license_report"]
  - gate: 07
    name: Artifact version control
    primary_process: Release
    intent: "Signed, immutable, provenance-attested artifacts"
    evidence: ["digest", "signature", "provenance_attestation"]
  - gate: 08
    name: Auto provision
    primary_process: Operate
    intent: "IaC security baselines; policy-as-code"
    evidence: ["opa_conftest_results", "plan_apply_logs"]
  - gate: 09
    name: Immutable servers
    primary_process: Operate
    intent: "Golden images; drift prevention"
    evidence: ["image_recipe", "container_digest", "drift_alerts"]
  - gate: 10
    name: Integration testing
    primary_process: Test
    intent: "Security-relevant integration/API checks"
    evidence: ["integration_test_report"]
  - gate: 11
    name: Performance testing
    primary_process: Test
    intent: "Perf/SLO guardrails"
    evidence: ["load_test_report", "error_budget_status"]
  - gate: 14
    name: Automated change order
    primary_process: Release
    intent: "Risk-aware approvals with security evidence"
    evidence: ["change_record_with_scan_links"]
  - gate: 15
    name: Zero downtime release
    primary_process: Release
    intent: "Blue/green or canary with health gates"
    evidence: ["deployment_logs", "health_gate_status"]
  - gate: 16
    name: Feature toggle
    primary_process: Release
    intent: "Progressive delivery and kill-switch controls"
    evidence: ["toggle_audit_log"]
  - gate: 13
    name: Automated rollback
    primary_process: Operate
    intent: "Auto-revert on SLO/SI breach; incident linkage"
    evidence: ["rollback_event", "incident_record"]
\end{minted}

\section*{Notes}
\begin{itemize}
  \item \textbf{Compilation}: This document uses \texttt{minted}. Compile with \texttt{-shell-escape}.
  \item Evidence examples are vendor-agnostic; substitute platform artifacts as needed.
\end{itemize}

\vspace{0.5em}
\noindent\rule{\linewidth}{0.4pt}\\
{\footnotesize Last updated: \today.}

\end{document}
