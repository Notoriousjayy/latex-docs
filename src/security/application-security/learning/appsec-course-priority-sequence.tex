
% !TEX TS-program = pdflatex
\documentclass[11pt]{article}

\usepackage[T1]{fontenc}
\usepackage[utf8]{inputenc}
\usepackage{lmodern}
\usepackage{microtype}
\usepackage[margin=1in]{geometry}
\usepackage{setspace}
\usepackage{enumitem}
\usepackage{booktabs}
\usepackage{tabularx}
\usepackage{xcolor}
\usepackage[hidelinks]{hyperref}
\usepackage{tcolorbox}

\setstretch{1.08}
\setlist[itemize]{leftmargin=*, itemsep=2pt, topsep=4pt}
\setlist[enumerate]{leftmargin=*, itemsep=2pt, topsep=4pt}

\definecolor{Accent}{HTML}{1F4E79}
\definecolor{SoftGray}{HTML}{F4F6F8}

\hypersetup{
  pdftitle={AppSec Course Priority Sequence Aligned to Certification Progression},
  pdfauthor={Jordan Suber},
  colorlinks=false
}

\newtcolorbox{infobox}{
  colback=SoftGray,
  colframe=Accent,
  boxrule=0.6pt,
  arc=3pt,
  left=10pt,
  right=10pt,
  top=8pt,
  bottom=8pt
}

\title{\textbf{AppSec Course Priority Sequence}\\
\large Aligned to a Certification Progression (PortSwigger $\rightarrow$ OSWA $\rightarrow$ BSCP/OSWE)}
\author{Prepared for: \textbf{Jordan Suber}}
\date{\today}

\begin{document}
\maketitle

\begin{infobox}
\textbf{Purpose.} This document provides a practical, priority-ordered sequence for the provided PortSwigger Web Security Academy learning paths, structured to align with a progressive certification-style roadmap:
\begin{itemize}
  \item \textbf{Phase 1 (Foundation):} high-yield core web security concepts and exploitation patterns.
  \item \textbf{Phase 2 (OSWA alignment):} breadth + workflow that strengthens assessment performance.
  \item \textbf{Phase 4 (Depth):} modern surfaces and higher-complexity vulnerability classes, consistent with advanced practitioner expectations.
\end{itemize}

\textbf{Scope.} The sequence uses only the learning paths explicitly provided in the source list. A short section at the end identifies \emph{important gaps} that are commonly required for comprehensive alignment but are not present in the excerpted list.
\end{infobox}

\section{Input Learning Paths Included}
The following learning paths (as provided) are included in this prioritization:
\begin{itemize}
  \item Web cache deception (Practitioner)
  \item WebSockets vulnerabilities (Practitioner)
  \item Authentication vulnerabilities (Practitioner)
  \item Server-side request forgery (SSRF) attacks (Practitioner)
  \item Prototype pollution (Practitioner)
  \item Clickjacking (UI redressing) (Practitioner)
  \item GraphQL API vulnerabilities (Practitioner)
  \item Cross-origin resource sharing (CORS) (Practitioner)
  \item Path traversal (Practitioner)
  \item NoSQL injection (Practitioner)
  \item Race conditions (Practitioner)
  \item Cross-site request forgery (CSRF) (Practitioner)
  \item File upload vulnerabilities (Practitioner)
  \item Web LLM attacks (Practitioner)
  \item API testing (Practitioner)
  \item Server-side vulnerabilities (Apprentice)
  \item SQL injection (Practitioner)
\end{itemize}

\section{Priority Sequence by Phase}

\subsection{Phase 1 (Foundation): High-yield core AppSec fundamentals}
\textbf{Goal:} Build the core vulnerability ``muscle memory'' and web security mental models needed before certification-grade assessments.

\begin{enumerate}
  \item \textbf{Server-side vulnerabilities (Apprentice)}\\
  Broad orientation to common server-side issues and real-world attacker workflows.
  \item \textbf{SQL injection}\\
  Canonical injection class; builds request/response reasoning, data exfiltration patterns, and secure query defense principles.
  \item \textbf{Authentication vulnerabilities}\\
  High-impact and foundational to real application security; improves ability to reason about sessions, credential workflows, and identity controls.
  \item \textbf{Path traversal}\\
  Common, high severity, and excellent for learning systematic input manipulation and server behavior inference.
  \item \textbf{File upload vulnerabilities}\\
  Practical exploitation and defense patterns; complements traversal and authentication by exercising validation, storage, and execution boundaries.
  \item \textbf{Server-side request forgery (SSRF) attacks}\\
  Modern, high-impact issue in cloud and microservice environments; develops internal network and metadata endpoint attack intuition.
  \item \textbf{NoSQL injection}\\
  Extends injection reasoning to modern stacks (operators, schema-less edge cases, query manipulation).
  \item \textbf{Cross-site request forgery (CSRF)}\\
  Core web risk; strengthens understanding of browser security model, state-changing requests, and anti-CSRF defenses.
  \item \textbf{Cross-origin resource sharing (CORS)}\\
  Critical modern SPA/API boundary topic; directly supports robust security review of cross-origin configurations.
  \item \textbf{Clickjacking (UI redressing)}\\
  Efficient coverage item; reinforces UI-layer protections and defense-in-depth.
\end{enumerate}

\begin{infobox}
\textbf{Phase 1 milestone.} After completing items 1--10, you should be equipped with the baseline breadth expected for moving into OSWA-aligned practice and higher-intensity assessment preparation.
\end{infobox}

\subsection{Phase 2 (OSWA alignment): Assessment breadth and testing workflow}
\textbf{Goal:} Expand beyond core vuln classes into attack surface discovery and practical methodology.

\begin{enumerate}[resume]
  \item \textbf{API testing}\\
  Directly supports endpoint discovery, recon, and parameter-handling skill---especially relevant for modern service-oriented architectures.
  \item \textbf{Race conditions}\\
  Increasingly common; strengthens concurrent behavior analysis and tool-driven exploitation workflows (e.g., Repeater/Turbo Intruder patterns).
\end{enumerate}

\subsection{Phase 4 (Depth): Modern surfaces and higher-complexity vulnerability classes}
\textbf{Goal:} Build deeper expertise consistent with advanced practitioner expectations and broader certification pathways.

\begin{enumerate}[resume]
  \item \textbf{GraphQL API vulnerabilities}\\
  Modern API surface area with distinctive failure modes; improves schema/resolver reasoning and bypass identification.
  \item \textbf{WebSockets vulnerabilities}\\
  Real-time application security; extends methodology beyond classic request/response paradigms.
  \item \textbf{Prototype pollution}\\
  Higher complexity; valuable for JavaScript-heavy stacks and for translating findings into secure coding guidance.
  \item \textbf{Web cache deception}\\
  Niche but high-value; sharpens understanding of caching boundaries, routing, and origin-versus-cache discrepancies.
\end{enumerate}

\subsection{Emerging / Specialized (Optional, after strong fundamentals)}
\begin{enumerate}[resume]
  \item \textbf{Web LLM attacks}\\
  Highly relevant for organizations deploying LLM-enabled features; best taken after core web and API foundations are strong, unless your current product surface makes this immediately applicable.
\end{enumerate}

\section{Summary Table}
\begin{table}[h!]
\centering
\renewcommand{\arraystretch}{1.2}
\begin{tabularx}{\textwidth}{@{} l X @{}}
\toprule
\textbf{Phase} & \textbf{Learning Paths (Priority Order)} \\
\midrule
Phase 1 (Foundation) &
Server-side vulnerabilities; SQL injection; Authentication vulnerabilities; Path traversal; File upload vulnerabilities; SSRF attacks; NoSQL injection; CSRF; CORS; Clickjacking \\
\addlinespace
Phase 2 (OSWA alignment) &
API testing; Race conditions \\
\addlinespace
Phase 4 (Depth) &
GraphQL API vulnerabilities; WebSockets vulnerabilities; Prototype pollution; Web cache deception \\
\addlinespace
Emerging / Specialized &
Web LLM attacks \\
\bottomrule
\end{tabularx}
\end{table}

\section{Important Gaps (Not Included in the Provided Excerpt)}
If the objective is strict alignment to a comprehensive web application security preparation track, ensure you also cover high-frequency topics that are commonly required but were \emph{not present} in the supplied list. Examples typically include:
\begin{itemize}
  \item Cross-site scripting (XSS)
  \item Access control / IDOR
  \item Request smuggling
  \item XML external entity (XXE)
  \item Deserialization / object injection
\end{itemize}

\begin{infobox}
\textbf{Recommendation.} Treat the gaps above as a parallel ``coverage checklist.'' If your certification roadmap explicitly includes them, integrate them into Phase 1--2 based on the guide's ordering and your organization's stack (e.g., SPA-heavy, GraphQL-heavy, microservices-heavy).
\end{infobox}

\section{Suggested Usage}
\begin{itemize}
  \item Use \textbf{Phase 1} as your baseline competency build.
  \item Use \textbf{Phase 2} to raise assessment performance through methodology and coverage expansion.
  \item Use \textbf{Phase 4} for depth and modern surface expertise, emphasizing realistic attack paths and defense strategies.
\end{itemize}

\vfill
\noindent\textit{Document version: v1.0}

\end{document}
