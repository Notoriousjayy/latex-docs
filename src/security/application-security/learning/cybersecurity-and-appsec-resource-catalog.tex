
% !TEX program = pdflatex
\documentclass[11pt]{article}

\usepackage[margin=1in]{geometry}
\usepackage{booktabs}
\usepackage{longtable}
\usepackage{array}
\usepackage{xcolor}
\usepackage{hyperref}
\usepackage{enumitem}
\usepackage{parskip}
\usepackage{titlesec}

\hypersetup{
  colorlinks=true,
  linkcolor=blue,
  urlcolor=blue,
  citecolor=blue
}

\titleformat{\section}{\Large\bfseries}{\thesection}{0.75em}{}
\titleformat{\subsection}{\large\bfseries}{\thesubsection}{0.75em}{}

\newcolumntype{L}[1]{>{\raggedright\arraybackslash}p{#1}}

\title{Cybersecurity \& AppSec Resource Catalog}
\author{Jordan Suber}
\date{February 2, 2026}

\begin{document}
\maketitle

\textbf{Purpose.} This document consolidates a curated set of high-quality resources (free and paid) aligned to:
\begin{itemize}[leftmargin=1.5em]
  \item an OMS Cybersecurity / CISO-track learning plan (policy, governance, risk, privacy, and program leadership), and
  \item an AppSec Engineer + AppSec Operations practice model (secure SDLC, verification, API security, and hands-on skills).
\end{itemize}

\textbf{How to use.} Start with \emph{Standards \& Frameworks} to anchor vocabulary and expectations, then pick a focused track:
(1) OMS/CISO program depth, and (2) AppSec engineering depth + labs.

\tableofcontents
\newpage

\section{Resource Catalog}

\small
\subsection{Standards \& Frameworks (Mostly Free)}

% Column widths tuned to fit within \textwidth under 1-inch margins.
\begin{longtable}{L{4.5cm} L{2.0cm} L{1.5cm} L{1.2cm} L{5.3cm}}
\toprule
\textbf{Resource} & \textbf{Type} & \textbf{Access} & \textbf{Link} & \textbf{Why it matters / How to use} \\
\midrule
\endfirsthead
\toprule
\textbf{Resource} & \textbf{Type} & \textbf{Access} & \textbf{Link} & \textbf{Why it matters / How to use} \\
\midrule
\endhead
\midrule
\multicolumn{5}{r}{\emph{Continued on next page}} \\
\bottomrule
\endfoot
\bottomrule
\endlastfoot
Secure Software Development Framework (SSDF) v1.1 — NIST SP 800-218 & Standard & Free & \href{https://csrc.nist.gov/pubs/sp/800/218/final}{Open} & Foundational secure SDLC practice set; useful for policy-to-engineering mapping. \\
Secure Software Development Framework (SSDF) v1.1 — PDF & Standard & Free & \href{https://nvlpubs.nist.gov/nistpubs/specialpublications/nist.sp.800-218.pdf}{Open} & Direct PDF for offline reference and citation. \\
NIST Cybersecurity Framework (CSF) 2.0 — NIST CSWP 29 (Final) & Framework & Free & \href{https://csrc.nist.gov/pubs/cswp/29/the-nist-cybersecurity-framework-csf-20/final}{Open} & Outcome-based framework for communicating and managing cyber risk enterprise-wide. \\
NIST Cybersecurity Framework (CSF) 2.0 — PDF & Framework & Free & \href{https://nvlpubs.nist.gov/nistpubs/CSWP/NIST.CSWP.29.pdf}{Open} & Direct PDF for executive-facing readouts and program alignment. \\
Security and Privacy Controls — NIST SP 800-53 Rev. 5 (Update 1 page) & Standard & Free & \href{https://csrc.nist.gov/pubs/sp/800/53/r5/upd1/final}{Open} & Canonical control catalog for governance/compliance control mapping. \\
OWASP Application Security Verification Standard (ASVS) & Standard & Free & \href{https://owasp.org/www-project-application-security-verification-standard/}{Open} & Security requirements baseline for web apps and services; great for AppSec acceptance criteria. \\
OWASP Top 10 API Security Risks – 2023 & Standard & Free & \href{https://owasp.org/API-Security/editions/2023/en/0x11-t10/}{Open} & API risk taxonomy + mitigations; aligns well with modern service architectures. \\
MITRE ATT\&CK® & Knowledge Base & Free & \href{https://attack.mitre.org/}{Open} & Threat-informed defense reference for adversary tactics/techniques and mapping detections. \\
ISO/IEC 27001:2022 (ISMS Requirements) & Standard & Paid & \href{https://www.iso.org/standard/27001}{Open} & Best-known ISMS standard; governance backbone for security programs. \\
\end{longtable}

\subsection{OMS Cybersecurity \& CISO-Track Anchors}

\begin{longtable}{L{4.5cm} L{2.0cm} L{1.5cm} L{1.2cm} L{5.3cm}}
\toprule
\textbf{Resource} & \textbf{Type} & \textbf{Access} & \textbf{Link} & \textbf{Why it matters / How to use} \\
\midrule
\endfirsthead
\toprule
\textbf{Resource} & \textbf{Type} & \textbf{Access} & \textbf{Link} & \textbf{Why it matters / How to use} \\
\midrule
\endhead
\midrule
\multicolumn{5}{r}{\emph{Continued on next page}} \\
\bottomrule
\endfoot
\bottomrule
\endlastfoot
Georgia Tech OMS Cybersecurity — Curriculum & Program & Paid (tuition) & \href{https://pe.gatech.edu/degrees/cybersecurity/curriculum}{Open} & Use as the sequencing backbone for your degree plan and elective alignment. \\
Georgia Tech OMS Cybersecurity — Curriculum Grid (PDF) & Program & Free & \href{https://pe.gatech.edu/sites/default/files/degrees/cybersecurity/oms-cybersecurity-curriculum-grid.pdf}{Open} & Quick reference for core/flexible core and specialization requirements. \\
Georgia Tech MS in Cybersecurity — Catalog Entry & Program & Free & \href{https://catalog.gatech.edu/programs/cybersecurity-ms/}{Open} & Authoritative credit-hour requirements and program description. \\
NIST OLIR: ISO/IEC 27001:2022 $\leftrightarrow$ CSF 2.0 Informative Reference & Mapping & Free & \href{https://csrc.nist.gov/projects/olir/informative-reference-catalog/details?referenceId=154}{Open} & Useful when building crosswalks between ISMS programs and CSF outcomes. \\
\end{longtable}

\subsection{AppSec Engineer \& AppSec Operations (High-Value Training)}

\begin{longtable}{L{4.5cm} L{2.0cm} L{1.5cm} L{1.2cm} L{5.3cm}}
\toprule
\textbf{Resource} & \textbf{Type} & \textbf{Access} & \textbf{Link} & \textbf{Why it matters / How to use} \\
\midrule
\endfirsthead
\toprule
\textbf{Resource} & \textbf{Type} & \textbf{Access} & \textbf{Link} & \textbf{Why it matters / How to use} \\
\midrule
\endhead
\midrule
\multicolumn{5}{r}{\emph{Continued on next page}} \\
\bottomrule
\endfoot
\bottomrule
\endlastfoot
PortSwigger Web Security Academy & Course/Labs & Free & \href{https://portswigger.net/web-security}{Open} & Best-in-class hands-on web vulnerability labs; pairs well with Burp Suite usage. \\
Stanford CS253 — Web Security & Course & Free & \href{https://cs253.stanford.edu/}{Open} & Structured web security survey course with practical countermeasure focus. \\
MIT OpenCourseWare 6.858 — Computer Systems Security (Fall 2014) & Course & Free & \href{https://ocw.mit.edu/courses/6-858-computer-systems-security-fall-2014/}{Open} & Deep systems security fundamentals: threat models, OS security, protocols, research grounding. \\
OpenSecurityTraining2 (OST2) & Course Hub & Free & \href{https://p.ost2.fyi/}{Open} & Free/open-access security training with rigorous reverse engineering and vulnerability content. \\
OpenSecurityTraining2 — Courses Index & Course Hub & Free & \href{https://p.ost2.fyi/courses}{Open} & Browse and select tracks (RE, vulns, hardware) aligned to your gaps. \\
TryHackMe & Labs Platform & Free + Paid & \href{https://tryhackme.com/}{Open} & Guided paths with browser-based labs; good for consistent daily reps. \\
Hack The Box & Labs Platform & Free + Paid & \href{https://www.hackthebox.com/}{Open} & Challenging labs and tracks; strong for skill benchmarking and scenario variety. \\
SANS Institute & Training Provider & Paid & \href{https://www.sans.org/}{Open} & Premium hands-on courses and research; good for deep specialization and leadership tracks. \\
OffSec (Offensive Security) & Training Provider & Paid & \href{https://www.offsec.com/}{Open} & Hands-on training + certifications; rigorous offensive fundamentals that translate to better AppSec. \\
\end{longtable}

\subsection{Books — Security, AppSec, Governance, and Systems}

\begin{longtable}{L{4.5cm} L{2.0cm} L{1.5cm} L{1.2cm} L{5.3cm}}
\toprule
\textbf{Resource} & \textbf{Type} & \textbf{Access} & \textbf{Link} & \textbf{Why it matters / How to use} \\
\midrule
\endfirsthead
\toprule
\textbf{Resource} & \textbf{Type} & \textbf{Access} & \textbf{Link} & \textbf{Why it matters / How to use} \\
\midrule
\endhead
\midrule
\multicolumn{5}{r}{\emph{Continued on next page}} \\
\bottomrule
\endfoot
\bottomrule
\endlastfoot
Security Engineering — Ross Anderson & Book & Paid & \href{https://books.google.com/books?q=Security+Engineering+Ross+Anderson}{Open} & Classic, broad security engineering reference; excellent conceptual depth. \\
Computer Security: Principles and Practice — Stallings \& Brown & Book & Paid & \href{https://books.google.com/books?q=Computer+Security+Principles+and+Practice+Stallings+Brown}{Open} & Strong survey-style text spanning crypto, access control, app security, and ops. \\
Computer Security: Art and Science — Matt Bishop & Book & Paid & \href{https://books.google.com/books?q=Computer+Security+Art+and+Science+Matt+Bishop}{Open} & Foundational security principles with rigorous treatment. \\
Computer Networking: A Top-Down Approach — Kurose \& Ross & Book & Paid & \href{https://books.google.com/books?q=Computer+Networking+A+Top-Down+Approach+Kurose+Ross}{Open} & Networking fundamentals for security engineers; practical protocol understanding. \\
Computer Networks — Tanenbaum \& Wetherall & Book & Paid & \href{https://books.google.com/books?q=Computer+Networks+Tanenbaum+Wetherall}{Open} & Deeper protocol and architecture coverage; excellent reference. \\
Information Security Management Handbook & Book & Paid & \href{https://books.google.com/books?q=Information+Security+Management+Handbook}{Open} & Management and program reference, useful for governance and program design. \\
How to Measure Anything in Cybersecurity Risk — Hubbard \& Seiersen & Book & Paid & \href{https://books.google.com/books?q=How+to+Measure+Anything+in+Cybersecurity+Risk+Hubbard+Seiersen}{Open} & Quantitative risk thinking for prioritization and board-facing communication. \\
IT Governance — Weill \& Ross & Book & Paid & \href{https://books.google.com/books?q=IT+Governance+Weill+Ross}{Open} & How governance structures drive outcomes; useful for CISO-track operating models. \\
The CISO Desk Reference Guide & Book & Paid & \href{https://books.google.com/books?q=CISO+Desk+Reference+Guide}{Open} & Practical CISO playbook and reference framing. \\
Information Privacy Law & Book & Paid & \href{https://books.google.com/books?q=Information+Privacy+Law}{Open} & Legal foundations and privacy governance context. \\
The Privacy Engineer's Manifesto — Dennedy, Fox, Finneran & Book & Paid & \href{https://books.google.com/books?q=The+Privacy+Engineer%27s+Manifesto+Dennedy+Fox+Finneran}{Open} & Privacy engineering and operationalization guidance. \\
Serious Cryptography — Jean-Philippe Aumasson & Book & Paid & \href{https://books.google.com/books?q=Serious+Cryptography+Aumasson}{Open} & Modern crypto concepts with practical guidance. \\
Introduction to Modern Cryptography — Katz \& Lindell & Book & Paid & \href{https://books.google.com/books?q=Introduction+to+Modern+Cryptography+Katz+Lindell}{Open} & Academic-grade rigor for cryptographic foundations. \\
Cryptography Engineering — Ferguson, Schneier, Kohno & Book & Paid & \href{https://books.google.com/books?q=Cryptography+Engineering+Ferguson+Schneier+Kohno}{Open} & Applied cryptography and engineering perspective. \\
Software Engineering — Ian Sommerville & Book & Paid & \href{https://books.google.com/books?q=Software+Engineering+Sommerville}{Open} & Core SE principles—useful for SDLC structure and quality systems. \\
Software Engineering at Google & Book & Paid & \href{https://books.google.com/books?q=Software+Engineering+at+Google}{Open} & Engineering practices for scale; culture/process and technical foundations. \\
Designing Data-Intensive Applications — Kleppmann & Book & Paid & \href{https://books.google.com/books?q=Designing+Data-Intensive+Applications+Kleppmann}{Open} & Distributed systems realities: reliability, consistency, operations. \\
Operating Systems: Three Easy Pieces (OSTEP) & Book & Free & \href{https://books.google.com/books?q=Operating+Systems+Three+Easy+Pieces}{Open} & Widely used OS text (also freely available online); supports low-level security understanding. \\
Modern Operating Systems — Tanenbaum \& Bos & Book & Paid & \href{https://books.google.com/books?q=Modern+Operating+Systems+Tanenbaum+Bos}{Open} & OS architecture depth; great for systems-level security intuition. \\
Enterprise Integration Patterns — Hohpe \& Woolf & Book & Paid & \href{https://books.google.com/books?q=Enterprise+Integration+Patterns+Hohpe+Woolf}{Open} & Messaging/integration patterns; helpful for API/event-driven security thinking. \\
Site Reliability Engineering (SRE) — Google & Book & Free & \href{https://sre.google/books/}{Open} & Free online; production reliability principles that intersect strongly with security. \\
Building Secure and Reliable Systems — Google & Book & Free & \href{https://google.github.io/building-secure-and-reliable-systems/raw/toc.html}{Open} & Free online; security-reliability joint design and operational best practices. \\
Threat Modeling: Designing for Security — Adam Shostack & Book & Paid & \href{https://www.wiley.com/en-us/Threat%2BModeling%3A%2BDesigning%2Bfor%2BSecurity-p-9781118809990}{Open} & Definitive threat modeling process guide; strong for design reviews. \\
The Web Application Hacker's Handbook — Stuttard \& Pinto & Book & Paid & \href{https://books.google.com/books?q=The+Web+Application+Hacker%27s+Handbook+Stuttard+Pinto}{Open} & Classic web app testing reference; pairs with Burp-based practice. \\
Practical Binary Analysis — Dang, Gazet, Bachaalany & Book & Paid & \href{https://books.google.com/books?q=Practical+Binary+Analysis+Dang+Gazet+Bachaalany}{Open} & Binary analysis fundamentals for vulnerability research. \\
Practical Malware Analysis — Sikorski \& Honig & Book & Paid & \href{https://books.google.com/books?q=Practical+Malware+Analysis+Sikorski+Honig}{Open} & Solid malware RE methodology; valuable for incident and research context. \\
The Art of Memory Forensics — Ligh et al. & Book & Paid & \href{https://books.google.com/books?q=The+Art+of+Memory+Forensics+Ligh}{Open} & Memory forensics at depth; strengthens detection/IR fluency. \\
\end{longtable}

\subsection{Certifications (Management + Leadership Signaling)}

\begin{longtable}{L{4.5cm} L{2.0cm} L{1.5cm} L{1.2cm} L{5.3cm}}
\toprule
\textbf{Resource} & \textbf{Type} & \textbf{Access} & \textbf{Link} & \textbf{Why it matters / How to use} \\
\midrule
\endfirsthead
\toprule
\textbf{Resource} & \textbf{Type} & \textbf{Access} & \textbf{Link} & \textbf{Why it matters / How to use} \\
\midrule
\endhead
\midrule
\multicolumn{5}{r}{\emph{Continued on next page}} \\
\bottomrule
\endfoot
\bottomrule
\endlastfoot
ISACA Certified Information Security Manager (CISM) & Certification & Paid & \href{https://www.isaca.org/credentialing/cism}{Open} & Program management, governance, risk, and incident management domains. \\
ISC2 CISSP & Certification & Paid & \href{https://www.isc2.org/certifications/cissp}{Open} & Broad security leadership credential covering eight domains. \\
\end{longtable}


\normalsize
\section{Notes on Access \& Quality}
\begin{itemize}[leftmargin=1.5em]
  \item \textbf{Free vs Paid:} Many platforms offer a free tier with optional paid labs or certificates (e.g., TryHackMe, Hack The Box). Books are typically paid unless explicitly noted as free online.
  \item \textbf{Priority order:} (1) SSDF + ASVS/API Top 10 for AppSec execution, (2) CSF/800-53/ISO 27001 for governance alignment, (3) hands-on labs for skills reinforcement.
  \item \textbf{CISO-track practice:} Use CSF 2.0 and ISO 27001 to structure outcomes and operating rhythms; use 800-53 for control depth and audit-ready mapping.
\end{itemize}

\end{document}
