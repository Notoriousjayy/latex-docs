\documentclass[11pt,letterpaper]{article}

% ============================================================================
% PACKAGES
% ============================================================================
\usepackage[utf8]{inputenc}
\usepackage[T1]{fontenc}
\usepackage[margin=1in]{geometry}
\usepackage{titlesec}
\usepackage{titletoc}
\usepackage{enumitem}
\usepackage{booktabs}
\usepackage{tabularx}
\usepackage{longtable}
\usepackage{array}
\usepackage{xcolor}
\usepackage{hyperref}
\usepackage{fancyhdr}
\usepackage{graphicx}
\usepackage{tcolorbox}
\usepackage{parskip}
\usepackage{etoolbox}
\usepackage{multicol}
\usepackage{calc}

% ============================================================================
% COLOR DEFINITIONS
% ============================================================================
\definecolor{primaryblue}{RGB}{24, 78, 119}
\definecolor{accentblue}{RGB}{44, 130, 201}
\definecolor{lightgray}{RGB}{245, 245, 245}
\definecolor{darkgray}{RGB}{64, 64, 64}
\definecolor{accentgreen}{RGB}{46, 125, 50}
\definecolor{accentorange}{RGB}{230, 126, 34}
\definecolor{accentpurple}{RGB}{142, 68, 173}
\definecolor{lightblue}{RGB}{232, 245, 253}

% ============================================================================
% HYPERREF SETUP
% ============================================================================
\hypersetup{
    colorlinks=true,
    linkcolor=primaryblue,
    urlcolor=accentblue,
    pdftitle={Diagram-Driven Reading Sequence for AppSec Teams},
    pdfauthor={Application Security Program},
    pdfsubject={Strategic Reading Guide for Architecture Documentation}
}

% ============================================================================
% HEADER/FOOTER
% ============================================================================
\pagestyle{fancy}
\fancyhf{}
\fancyhead[L]{\small\textcolor{darkgray}{AppSec Reading Sequence}}
\fancyhead[R]{\small\textcolor{darkgray}{Diagram-Driven Learning}}
\fancyfoot[C]{\thepage}
\renewcommand{\headrulewidth}{0.4pt}
\renewcommand{\footrulewidth}{0pt}

% ============================================================================
% SECTION FORMATTING
% ============================================================================
\titleformat{\section}
    {\Large\bfseries\color{primaryblue}}
    {\thesection}{1em}{}[\titlerule]

\titleformat{\subsection}
    {\large\bfseries\color{accentblue}}
    {\thesubsection}{1em}{}

\titleformat{\subsubsection}
    {\normalsize\bfseries\color{darkgray}}
    {\thesubsubsection}{1em}{}

% ============================================================================
% TCOLORBOX STYLES
% ============================================================================
\newtcolorbox{bookbox}[2][]{
    colback=lightgray,
    colframe=primaryblue,
    fonttitle=\bfseries\large,
    title={\textcolor{accentorange}{#2}},
    boxrule=1pt,
    arc=3pt,
    left=10pt,
    right=10pt,
    top=8pt,
    bottom=8pt,
    #1
}

\newtcolorbox{diagrambox}{
    colback=lightblue,
    colframe=accentblue,
    boxrule=0.5pt,
    arc=2pt,
    left=8pt,
    right=8pt,
    top=6pt,
    bottom=6pt,
    before skip=10pt,
    after skip=10pt
}

\newtcolorbox{stakeholderbox}{
    colback=white,
    colframe=accentgreen,
    boxrule=0.5pt,
    arc=2pt,
    left=8pt,
    right=8pt,
    top=6pt,
    bottom=6pt
}

\newtcolorbox{infobox}[1][]{
    colback=lightgray,
    colframe=darkgray,
    fonttitle=\bfseries,
    title=#1,
    boxrule=0.5pt,
    arc=2pt,
    left=8pt,
    right=8pt,
    top=6pt,
    bottom=6pt
}

\newtcolorbox{highlightbox}[1][]{
    colback=lightgray,
    colframe=accentpurple,
    fonttitle=\bfseries,
    title=#1,
    boxrule=1pt,
    arc=3pt,
    left=10pt,
    right=10pt,
    top=8pt,
    bottom=8pt
}

% ============================================================================
% CUSTOM COMMANDS
% ============================================================================
\newcommand{\bookfield}[2]{\textbf{#1:} #2}
\newcommand{\booktitle}[1]{\textit{#1}}
\newcommand{\phase}[1]{\textcolor{accentorange}{\textbf{Phase #1}}}

% ============================================================================
% DOCUMENT
% ============================================================================
\begin{document}

% ============================================================================
% TITLE PAGE
% ============================================================================
\begin{titlepage}
    \centering
    \vspace*{1.5cm}
    
    {\Huge\bfseries\color{primaryblue} Diagram-Driven\\[0.3cm] Reading Sequence\par}
    
    \vspace{0.8cm}
    
    {\Large\color{accentblue} A Strategic Guide for AppSec Teams\\[0.2cm]
    Documenting and Improving Security Processes\par}
    
    \vspace{1.5cm}
    
    \begin{tcolorbox}[
        colback=lightgray,
        colframe=primaryblue,
        width=0.9\textwidth,
        arc=5pt,
        boxrule=1.5pt
    ]
    \centering
    \large
    \textbf{Objective:} Maximize business value per stakeholder through\\
    systematic reading and diagram production
    
    \vspace{0.5cm}
    
    \textbf{Reading List:}\\[0.3cm]
    \begin{tabular}{ll}
    \booktitle{Thinking in Systems} & Donella Meadows \\
    \booktitle{The E-Myth Revisited} & Michael Gerber \\
    \booktitle{Work the System} & Sam Carpenter \\
    \booktitle{Service Design Doing} & Stickdorn et al. \\
    \booktitle{Traction} & Gino Wickman \\
    \booktitle{Scaling Up} & Verne Harnish \\
    \booktitle{Modern Software Engineering} & Dave Farley \\
    \booktitle{The Mythical Man-Month} & Fred Brooks \\
    \booktitle{Practical Object-Oriented Design} & Sandi Metz \\
    \end{tabular}
    \end{tcolorbox}
    
    \vspace{1.5cm}
    
    \begin{tabular}{ll}
        \textbf{Document Type:} & Strategic Reading Guide \\[0.3cm]
        \textbf{Target Audience:} & AppSec Teams \& Leadership \\[0.3cm]
        \textbf{Focus Areas:} & Intake, Threat Modeling, Vulnerability Management, \\
        & Exceptions, CI/CD Gates, Reporting \\[0.3cm]
        \textbf{Methodology:} & Diagram-Driven Learning \\
    \end{tabular}
    
    \vfill
    
    {\large Application Security Program\par}
    
    \vspace{0.5cm}
    
    {\small\textcolor{darkgray}{Version 1.0}\par}
    
\end{titlepage}

% ============================================================================
% TABLE OF CONTENTS
% ============================================================================
\tableofcontents
\newpage

% ============================================================================
% EXECUTIVE SUMMARY
% ============================================================================
\section{Executive Summary}

This document presents a strategically sequenced reading program designed to help Application Security (AppSec) teams systematically document and improve their processes. The sequence is optimized to maximize business value per stakeholder by pairing each book with specific diagram deliverables that address real organizational needs.

\subsection{Core Philosophy}

The reading sequence follows a deliberate progression:

\begin{center}
\begin{tcolorbox}[
    colback=lightblue,
    colframe=accentblue,
    width=0.95\textwidth,
    arc=3pt
]
\centering
\large
\textbf{Why} $\rightarrow$ \textbf{What} $\rightarrow$ \textbf{How} $\rightarrow$ \textbf{Scale} $\rightarrow$ \textbf{Sustain}
\end{tcolorbox}
\end{center}

Each book builds upon the previous, creating a coherent body of knowledge that translates directly into actionable architecture documentation.

\subsection{Value Proposition}

This approach delivers three key benefits:

\begin{enumerate}[nosep]
    \item \textbf{Purposeful Reading:} Every book connects to specific diagram outputs, ensuring knowledge translates to artifacts.
    \item \textbf{Stakeholder Alignment:} Diagrams are mapped to stakeholder concerns, ensuring relevance and adoption.
    \item \textbf{Progressive Complexity:} The sequence builds from foundational concepts to tactical implementation.
\end{enumerate}

\newpage

% ============================================================================
% READING SEQUENCE OVERVIEW
% ============================================================================
\section{Reading Sequence Overview}

The nine-book sequence is organized into three phases, each serving distinct organizational needs:

\begin{center}
\renewcommand{\arraystretch}{1.5}
\begin{tabular}{clll}
\toprule
\textbf{\#} & \textbf{Book} & \textbf{Phase} & \textbf{Core Outcome} \\
\midrule
1 & \booktitle{Thinking in Systems} & Foundation & System behavior understanding \\
2 & \booktitle{The E-Myth Revisited} & Foundation & Standardization mindset \\
3 & \booktitle{Work the System} & Foundation & Process documentation \\
\midrule
4 & \booktitle{Service Design Doing} & Design & Service optimization \\
5 & \booktitle{Traction} & Governance & Operating cadence \\
6 & \booktitle{Scaling Up} & Governance & Scalable execution \\
\midrule
7 & \booktitle{Modern Software Engineering} & Implementation & DevSecOps integration \\
8 & \booktitle{The Mythical Man-Month} & Implementation & Coordination management \\
9 & \booktitle{Practical OO Design} & Implementation & Technical architecture \\
\bottomrule
\end{tabular}
\end{center}

\vspace{1cm}

\begin{highlightbox}[Reading Time Investment]
\begin{itemize}[nosep]
    \item \textbf{Phase 1 (Foundation):} 4--6 weeks $\rightarrow$ Establishes ``why'' and ``what''
    \item \textbf{Phase 2 (Design \& Governance):} 4--6 weeks $\rightarrow$ Establishes ``how'' and ``who''
    \item \textbf{Phase 3 (Implementation):} 4--6 weeks $\rightarrow$ Establishes technical execution
\end{itemize}

\textbf{Total Program Duration:} 12--18 weeks for full sequence with diagram production
\end{highlightbox}

\newpage

% ============================================================================
% PHASE 1: FOUNDATION
% ============================================================================
\section{Phase 1: Foundation}

The foundation phase establishes the conceptual framework for understanding AppSec as a system, the importance of standardization, and the mechanics of process documentation.

% ----------------------------------------------------------------------------
% BOOK 1: THINKING IN SYSTEMS
% ----------------------------------------------------------------------------
\subsection{Book 1: Thinking in Systems (Meadows)}

\begin{bookbox}{``How the AppSec System Behaves''}
Use this book first to avoid diagramming symptoms instead of causes. It provides the framing for feedback loops, delays, unintended consequences, and system resilience.

\vspace{0.5cm}

\textbf{Key Concepts to Extract:}
\begin{itemize}[nosep]
    \item Stocks, flows, and feedback loops
    \item System boundaries and interfaces
    \item Leverage points for intervention
    \item Delays and their effects on behavior
    \item Resilience vs.\ brittleness in systems
\end{itemize}
\end{bookbox}

\subsubsection{Diagrams to Produce}

\begin{diagrambox}
\textbf{High Business Value --- Produce Early}

\begin{description}[leftmargin=1.5cm, style=nextline]
    \item[System Context + Boundary Diagram]
    Shows AppSec as a service within the broader Engineering ecosystem. Defines what is inside vs.\ outside the AppSec system boundary, identifying key interfaces with CI/CD, ticketing, CMDB, IAM, and GRC systems.
    
    \item[Causal Loop Diagrams]
    Visualize recurring pain points and their systemic causes. Example: ``Scanner noise $\rightarrow$ developer frustration $\rightarrow$ ignored findings $\rightarrow$ increased risk $\rightarrow$ more scanners deployed $\rightarrow$ more noise.'' These diagrams reveal reinforcing and balancing loops that drive organizational behavior.
    
    \item[Capability Map]
    A high-level view of AppSec capabilities: Secure SDLC, Vulnerability Management, Security Reviews, Exception Management, Metrics \& Reporting. This establishes the ``what'' before diving into the ``how.''
\end{description}
\end{diagrambox}

\subsubsection{Primary Stakeholders Served}

\begin{stakeholderbox}
\begin{description}[nosep, leftmargin=2cm]
    \item[Executives/BOD] Understand why change is needed and where investment yields returns
    \item[Eng Leadership] See systemic causes of friction, not just symptoms
    \item[GRC/Audit] Appreciate control effectiveness within system dynamics
\end{description}
\end{stakeholderbox}

\subsubsection{Application to AppSec Processes}

\begin{center}
\renewcommand{\arraystretch}{1.3}
\begin{tabular}{p{4cm}p{9cm}}
\toprule
\textbf{AppSec Process} & \textbf{Systems Thinking Application} \\
\midrule
Intake & Model request queue as a stock; identify what increases/decreases it \\
Vulnerability Mgmt & Map feedback loops between finding volume and remediation capacity \\
Exceptions & Understand accumulation of risk debt over time \\
CI/CD Gates & Analyze delays between detection and developer feedback \\
Reporting & Connect leading indicators to lagging outcomes \\
\bottomrule
\end{tabular}
\end{center}

\newpage

% ----------------------------------------------------------------------------
% BOOK 2: THE E-MYTH REVISITED
% ----------------------------------------------------------------------------
\subsection{Book 2: The E-Myth Revisited (Gerber)}

\begin{bookbox}{``Why Standardization and Repeatable Systems Matter''}
This book provides the foundation for treating AppSec work as a set of services with standardized operating procedures, not ad-hoc heroics. It emphasizes working \textit{on} the system rather than \textit{in} it.

\vspace{0.5cm}

\textbf{Key Concepts to Extract:}
\begin{itemize}[nosep]
    \item The franchise prototype model
    \item Working on the business vs.\ in the business
    \item Systemization of every repeatable task
    \item The importance of documented procedures
    \item Predictability through standardization
\end{itemize}
\end{bookbox}

\subsubsection{Diagrams to Produce}

\begin{diagrambox}
\textbf{Core Deliverables}

\begin{description}[leftmargin=1.5cm, style=nextline]
    \item[AppSec Service Catalog]
    A comprehensive inventory of AppSec services including: service name, description, entry criteria (what triggers the service), required inputs, outputs/deliverables, SLAs (response time, completion time), escalation paths, and service owner. This becomes the ``menu'' that stakeholders use to engage AppSec.
    
    \item[High-Level Operating Model]
    A visual representation of who does what, when, and why. Shows the relationship between AppSec roles (security engineers, architects, managers) and their responsibilities across different service types. Establishes clear ownership and accountability.
\end{description}
\end{diagrambox}

\subsubsection{Primary Stakeholders Served}

\begin{stakeholderbox}
\begin{description}[nosep, leftmargin=2.5cm]
    \item[Executives] See AppSec as a predictable, manageable function
    \item[AppSec Leadership] Gain tools for capacity planning and resource allocation
    \item[Eng Managers] Understand what to expect and how to engage AppSec
\end{description}
\end{stakeholderbox}

\subsubsection{Service Catalog Structure}

\begin{center}
\renewcommand{\arraystretch}{1.3}
\begin{tabular}{p{3.5cm}p{3cm}p{3cm}p{3cm}}
\toprule
\textbf{Service} & \textbf{Entry Criteria} & \textbf{Output} & \textbf{SLA} \\
\midrule
Security Review & Design doc ready & Findings report & 5 business days \\
Threat Model & Architecture defined & Threat register & 10 business days \\
Vuln Triage & Finding submitted & Priority assigned & 24 hours \\
Exception Request & Justification provided & Approval/denial & 48 hours \\
Pen Test & Scope defined & Test report & 2--4 weeks \\
\bottomrule
\end{tabular}
\end{center}

\newpage

% ----------------------------------------------------------------------------
% BOOK 3: WORK THE SYSTEM
% ----------------------------------------------------------------------------
\subsection{Book 3: Work the System (Carpenter)}

\begin{bookbox}{``Document the Way Work Actually Happens''}
With the mindset established, this book provides execution mechanics for capturing procedures and improving them. It emphasizes separating yourself from the work to see processes objectively.

\vspace{0.5cm}

\textbf{Key Concepts to Extract:}
\begin{itemize}[nosep]
    \item Strategic objective, operating principles, and working procedures
    \item The ``outside and slightly elevated'' perspective
    \item Procedure documentation as a management tool
    \item Continuous refinement of processes
    \item Measuring process effectiveness
\end{itemize}
\end{bookbox}

\subsubsection{Diagrams to Produce}

\begin{diagrambox}
\textbf{Core AppSec Process Set --- Immediate Friction Reduction}

\begin{description}[leftmargin=1.5cm, style=nextline]
    \item[BPMN Swimlane: AppSec Intake]
    Request submission $\rightarrow$ triage $\rightarrow$ categorization $\rightarrow$ routing $\rightarrow$ assignment $\rightarrow$ closure. Shows handoffs between requesters, AppSec triage, and assigned engineers.
    
    \item[BPMN Swimlane: Secure Design Review / Threat Modeling]
    Engagement request $\rightarrow$ scoping $\rightarrow$ context gathering $\rightarrow$ threat enumeration $\rightarrow$ mitigation identification $\rightarrow$ documentation $\rightarrow$ sign-off $\rightarrow$ tracking.
    
    \item[BPMN Swimlane: Vulnerability Management]
    Discovery $\rightarrow$ deduplication $\rightarrow$ triage $\rightarrow$ assignment $\rightarrow$ remediation $\rightarrow$ verification $\rightarrow$ closure. Includes false positive handling and escalation paths.
    
    \item[BPMN Swimlane: Exception/Waiver Process]
    Request $\rightarrow$ justification review $\rightarrow$ compensating controls assessment $\rightarrow$ risk acceptance decision $\rightarrow$ documentation $\rightarrow$ expiry management.
    
    \item[RACI Overlays]
    For each BPMN diagram, add RACI (Responsible, Accountable, Consulted, Informed) designations to every step. This provides immediate clarity on ownership.
\end{description}
\end{diagrambox}

\subsubsection{Primary Stakeholders Served}

\begin{stakeholderbox}
\begin{description}[nosep, leftmargin=2.5cm]
    \item[AppSec Team] Clear procedures reduce ambiguity and enable onboarding
    \item[Developers] Know exactly what to expect at each step
    \item[SRE/Ops] Understand their role in vulnerability remediation
    \item[Eng Managers] Can predict timelines and plan capacity
\end{description}
\end{stakeholderbox}

\subsubsection{RACI Matrix Example: Vulnerability Management}

\begin{center}
\renewcommand{\arraystretch}{1.2}
\footnotesize
\begin{tabular}{lccccc}
\toprule
\textbf{Activity} & \textbf{AppSec} & \textbf{Dev Team} & \textbf{Eng Mgr} & \textbf{SRE} & \textbf{GRC} \\
\midrule
Discovery/Scanning & R/A & I & I & C & I \\
Triage \& Prioritization & R/A & C & I & C & I \\
Assignment & R & A & R & I & I \\
Remediation & C & R/A & I & C & I \\
Verification & R/A & I & I & C & I \\
Closure & R/A & I & I & I & I \\
Exception Approval & R & C & C & I & A \\
\bottomrule
\end{tabular}
\end{center}

\newpage

% ============================================================================
% PHASE 2: DESIGN & GOVERNANCE
% ============================================================================
\section{Phase 2: Design \& Governance}

The second phase focuses on optimizing AppSec as a consumable internal service and establishing governance structures that scale.

% ----------------------------------------------------------------------------
% BOOK 4: SERVICE DESIGN DOING
% ----------------------------------------------------------------------------
\subsection{Book 4: Service Design Doing (Stickdorn et al.)}

\begin{bookbox}{``Design AppSec as a Consumable Internal Service''}
Once you can document current state, use service design to improve it end-to-end. This book is explicitly process/design oriented and diagram-heavy, focusing on handoffs, touchpoints, and wait states.

\vspace{0.5cm}

\textbf{Key Concepts to Extract:}
\begin{itemize}[nosep]
    \item Service blueprinting methodology
    \item Customer journey mapping
    \item Touchpoint analysis
    \item Frontstage vs.\ backstage activities
    \item Moments of truth identification
\end{itemize}
\end{bookbox}

\subsubsection{Diagrams to Produce}

\begin{diagrambox}
\textbf{Service Optimization Artifacts}

\begin{description}[leftmargin=1.5cm, style=nextline]
    \item[Service Blueprints for AppSec Services]
    Multi-layer diagrams showing: (1) customer actions (developer touchpoints), (2) frontstage interactions (visible AppSec activities), (3) backstage activities (invisible AppSec work), (4) support processes (tooling, systems), and (5) physical evidence (artifacts produced). Create one blueprint per major service.
    
    \item[Customer Journey Maps]
    Visualize the developer experience when engaging AppSec from initial need through resolution. Identify pain points, emotional states, and opportunities for improvement. Focus on: awareness, engagement, waiting, resolution, and follow-up phases.
    
    \item[Value Stream Maps]
    Map the flow of work from vulnerability discovery to remediation completion. Identify: process time (actual work), wait time (queues, delays), and cycle time (total elapsed time). Highlight where time is lost and improvement opportunities.
\end{description}
\end{diagrambox}

\subsubsection{Primary Stakeholders Served}

\begin{stakeholderbox}
\begin{description}[nosep, leftmargin=3cm]
    \item[Developers] Experience less friction, clearer expectations
    \item[Eng Productivity] Reduced toil and faster throughput
    \item[AppSec Leadership] Data-driven service improvement
\end{description}
\end{stakeholderbox}

\subsubsection{Service Blueprint Layers}

\begin{center}
\renewcommand{\arraystretch}{1.4}
\begin{tabular}{p{3.5cm}p{9.5cm}}
\toprule
\textbf{Layer} & \textbf{Description} \\
\midrule
Customer Actions & What the developer does (submit request, provide info, implement fix) \\
Frontstage & Visible AppSec activities (respond to request, provide guidance) \\
Line of Visibility & --- \\
Backstage & Invisible AppSec work (triage, research, tooling configuration) \\
Support Processes & Systems and tools (scanners, ticketing, reporting dashboards) \\
\bottomrule
\end{tabular}
\end{center}

\newpage

% ----------------------------------------------------------------------------
% BOOK 5: TRACTION
% ----------------------------------------------------------------------------
\subsection{Book 5: Traction (Wickman)}

\begin{bookbox}{``Turn Diagrams into an Operating Cadence''}
With services and processes defined, Traction helps institutionalize accountability and metrics. It provides a practical operating system for running the AppSec function.

\vspace{0.5cm}

\textbf{Key Concepts to Extract:}
\begin{itemize}[nosep]
    \item The Entrepreneurial Operating System (EOS)
    \item Accountability charts (not org charts)
    \item Rocks (quarterly priorities)
    \item Scorecards with leading/lagging indicators
    \item Level 10 meeting structure
    \item Issues, Discuss, Solve (IDS) methodology
\end{itemize}
\end{bookbox}

\subsubsection{Diagrams to Produce}

\begin{diagrambox}
\textbf{Governance That Doesn't Feel Bureaucratic}

\begin{description}[leftmargin=1.5cm, style=nextline]
    \item[Accountability Chart]
    A function-based view of security ownership. Unlike org charts (which show reporting relationships), accountability charts show who owns what outcomes. Map: AppSec leadership, vulnerability management, security architecture, tooling/automation, metrics/reporting, and exception governance.
    
    \item[Scorecard Diagram]
    Visual representation of key metrics: leading indicators (scan coverage, review requests, training completion), lagging indicators (MTTR, open critical vulns, exception count), targets, and owners. Design for weekly review cadence.
    
    \item[Meeting/Decision Cadence Map]
    Document the rhythm of governance: daily standups, weekly tactical meetings, monthly leadership reviews, quarterly planning sessions. Map decision rights: who can approve exceptions, who escalates blockers, who reviews metrics.
\end{description}
\end{diagrambox}

\subsubsection{Primary Stakeholders Served}

\begin{stakeholderbox}
\begin{description}[nosep, leftmargin=2.5cm]
    \item[Executives] Clear accountability and measurable progress
    \item[Eng Leadership] Predictable cadence for engagement
    \item[AppSec Manager] Structured operating rhythm
\end{description}
\end{stakeholderbox}

\subsubsection{Sample Scorecard Structure}

\begin{center}
\renewcommand{\arraystretch}{1.3}
\footnotesize
\begin{tabular}{p{4cm}cccl}
\toprule
\textbf{Metric} & \textbf{Target} & \textbf{This Week} & \textbf{Trend} & \textbf{Owner} \\
\midrule
Critical vulns open $>$30d & $<$5 & 3 & $\downarrow$ & Vuln Mgmt Lead \\
MTTR (Critical) & $<$7d & 5.2d & $\downarrow$ & Vuln Mgmt Lead \\
Security reviews completed & $>$10/wk & 12 & $\rightarrow$ & Security Architect \\
Exception requests pending & $<$3 & 2 & $\downarrow$ & AppSec Manager \\
Scan coverage (\%) & $>$95\% & 97\% & $\uparrow$ & Tooling Lead \\
\bottomrule
\end{tabular}
\end{center}

\newpage

% ----------------------------------------------------------------------------
% BOOK 6: SCALING UP
% ----------------------------------------------------------------------------
\subsection{Book 6: Scaling Up (Harnish)}

\begin{bookbox}{``Make It Scale Without Breaking''}
Read after Traction to expand the operating system into scalable execution and cross-team alignment. Focuses on the four decisions: People, Strategy, Execution, and Cash.

\vspace{0.5cm}

\textbf{Key Concepts to Extract:}
\begin{itemize}[nosep]
    \item One-page strategic plan
    \item Meeting rhythm (daily, weekly, monthly, quarterly, annually)
    \item Rockefeller Habits checklist
    \item KPI cascading and alignment
    \item Cross-functional process ownership
\end{itemize}
\end{bookbox}

\subsubsection{Diagrams to Produce}

\begin{diagrambox}
\textbf{Scalable Execution Artifacts}

\begin{description}[leftmargin=1.5cm, style=nextline]
    \item[Operating Rhythm Map]
    Visual calendar showing the hierarchy of planning and review cycles: annual strategic planning $\rightarrow$ quarterly rock-setting $\rightarrow$ monthly metric reviews $\rightarrow$ weekly tactical meetings $\rightarrow$ daily standups. Shows how information flows up and decisions flow down.
    
    \item[KPI Tree]
    Hierarchical diagram connecting AppSec operational metrics to engineering outcomes to business results. Example: Scan coverage $\rightarrow$ Finding detection rate $\rightarrow$ Vulnerability exposure $\rightarrow$ Security incidents $\rightarrow$ Business risk posture.
    
    \item[Organizational Interface Map]
    Shows how AppSec connects with adjacent functions: Platform Engineering (tooling integration), Product Engineering (feature security), GRC (compliance alignment), IT/Ops (infrastructure security). Defines integration points and shared responsibilities.
\end{description}
\end{diagrambox}

\subsubsection{Primary Stakeholders Served}

\begin{stakeholderbox}
\begin{description}[nosep, leftmargin=3cm]
    \item[Executives] See how AppSec connects to business outcomes
    \item[Portfolio/Program Mgmt] Understand cross-functional dependencies
    \item[Eng Leadership] Clear integration points and expectations
\end{description}
\end{stakeholderbox}

\subsubsection{Operating Rhythm Example}

\begin{center}
\renewcommand{\arraystretch}{1.3}
\begin{tabular}{llp{7cm}}
\toprule
\textbf{Cadence} & \textbf{Duration} & \textbf{Focus} \\
\midrule
Daily & 15 min & Blockers, priorities, quick wins \\
Weekly & 60--90 min & Scorecard review, IDS on issues, rock progress \\
Monthly & 2--3 hours & Metric deep-dives, process improvements, staffing \\
Quarterly & Full day & Rock-setting, strategic alignment, retrospective \\
Annually & 2 days & Strategy refresh, annual planning, team development \\
\bottomrule
\end{tabular}
\end{center}

\newpage

% ============================================================================
% PHASE 3: IMPLEMENTATION
% ============================================================================
\section{Phase 3: Implementation}

The final phase translates business-process thinking into technical architecture and execution, ensuring AppSec integrates seamlessly with engineering practices.

% ----------------------------------------------------------------------------
% BOOK 7: MODERN SOFTWARE ENGINEERING
% ----------------------------------------------------------------------------
\subsection{Book 7: Modern Software Engineering (Farley)}

\begin{bookbox}{``Align AppSec to Fast, Reliable Delivery''}
This book translates business-process diagrams into SDLC/DevSecOps system diagrams that engineering teams respect. It emphasizes empiricism, feedback, and continuous improvement.

\vspace{0.5cm}

\textbf{Key Concepts to Extract:}
\begin{itemize}[nosep]
    \item Continuous delivery as an engineering discipline
    \item Feedback loops and cycle time optimization
    \item Deployment pipelines and quality gates
    \item Testability and observability
    \item Managing complexity through modularity
\end{itemize}
\end{bookbox}

\subsubsection{Diagrams to Produce}

\begin{diagrambox}
\textbf{DevSecOps Integration Artifacts}

\begin{description}[leftmargin=1.5cm, style=nextline]
    \item[Secure SDLC Value Stream]
    End-to-end visualization: Idea $\rightarrow$ Design $\rightarrow$ Code $\rightarrow$ Build $\rightarrow$ Test $\rightarrow$ Deploy $\rightarrow$ Operate. Overlay security activities at each stage: threat modeling (design), SAST/SCA (code/build), DAST (deploy), runtime protection (operate).
    
    \item[CI/CD Security Gates Diagram]
    Pipeline architecture showing: where security checks execute, what triggers them, pass/fail criteria, blocking vs.\ warning behaviors, SLAs for feedback, and override mechanisms. Include SAST, SCA, secrets scanning, IaC scanning, and container scanning.
    
    \item[Toolchain Data-Flow Diagram]
    Signal flow from source to action: Scanners $\rightarrow$ Findings aggregation $\rightarrow$ Deduplication $\rightarrow$ Enrichment $\rightarrow$ Ticket creation $\rightarrow$ Assignment $\rightarrow$ Remediation tracking $\rightarrow$ Verification $\rightarrow$ Reporting. Shows integration points between tools.
\end{description}
\end{diagrambox}

\subsubsection{Primary Stakeholders Served}

\begin{stakeholderbox}
\begin{description}[nosep, leftmargin=2.5cm]
    \item[Engineering Org] See security as part of delivery, not a blocker
    \item[Platform/DevOps] Clear integration requirements
    \item[AppSec Engineers] Technical architecture for tooling work
\end{description}
\end{stakeholderbox}

\subsubsection{CI/CD Gate Configuration Example}

\begin{center}
\renewcommand{\arraystretch}{1.3}
\footnotesize
\begin{tabular}{p{2.5cm}p{2cm}p{2.5cm}p{2.5cm}p{2.5cm}}
\toprule
\textbf{Check} & \textbf{Stage} & \textbf{Block Criteria} & \textbf{Warn Criteria} & \textbf{SLA} \\
\midrule
SAST & Build & Critical findings & High findings & $<$5 min \\
SCA & Build & Critical CVE & High CVE & $<$3 min \\
Secrets & Pre-commit & Any detection & --- & $<$30 sec \\
IaC Scan & Build & Critical misconfig & High misconfig & $<$2 min \\
Container & Build & Critical CVE & High CVE & $<$5 min \\
\bottomrule
\end{tabular}
\end{center}

\newpage

% ----------------------------------------------------------------------------
% BOOK 8: THE MYTHICAL MAN-MONTH
% ----------------------------------------------------------------------------
\subsection{Book 8: The Mythical Man-Month (Brooks)}

\begin{bookbox}{``Avoid Coordination Failures in Security Initiatives''}
This classic helps prevent predictable failure modes when rolling out process improvements across teams. Its insights on communication overhead and project management remain highly relevant.

\vspace{0.5cm}

\textbf{Key Concepts to Extract:}
\begin{itemize}[nosep]
    \item Brooks's Law: Adding people to a late project makes it later
    \item Communication overhead grows quadratically
    \item The surgical team model
    \item Conceptual integrity in design
    \item The second-system effect
    \item Plan to throw one away
\end{itemize}
\end{bookbox}

\subsubsection{Diagrams to Produce}

\begin{diagrambox}
\textbf{Coordination and Rollout Artifacts}

\begin{description}[leftmargin=1.5cm, style=nextline]
    \item[Communication and Handoff Map]
    Visualize where coordination costs appear in AppSec processes. Identify high-overhead touchpoints: cross-team handoffs, approval chains, information requests. Use this to simplify communication paths and reduce coordination burden.
    
    \item[Program Rollout Plan Diagram]
    Phased adoption plan for new processes or tools: pilot teams $\rightarrow$ early adopters $\rightarrow$ majority rollout $\rightarrow$ laggards. Include feedback loops at each phase, success criteria for progression, and rollback triggers. Map by repository criticality or team readiness.
\end{description}
\end{diagrambox}

\subsubsection{Primary Stakeholders Served}

\begin{stakeholderbox}
\begin{description}[nosep, leftmargin=3cm]
    \item[Program/Project Leadership] Realistic rollout planning
    \item[Eng Leadership] Understanding of coordination costs
    \item[AppSec Lead] Change management strategy
\end{description}
\end{stakeholderbox}

\subsubsection{Rollout Phase Model}

\begin{center}
\renewcommand{\arraystretch}{1.3}
\begin{tabular}{lcp{6.5cm}}
\toprule
\textbf{Phase} & \textbf{Coverage} & \textbf{Activities} \\
\midrule
Pilot & 2--3 teams & Validate process, gather feedback, refine tooling \\
Early Adopters & 10--15\% & Expand to willing teams, document patterns \\
Majority & 50--80\% & Standardized rollout, training at scale \\
Laggards & Remaining & Targeted support, exception handling \\
\bottomrule
\end{tabular}
\end{center}

\newpage

% ----------------------------------------------------------------------------
% BOOK 9: PRACTICAL OBJECT-ORIENTED DESIGN
% ----------------------------------------------------------------------------
\subsection{Book 9: Practical Object-Oriented Design (Metz)}

\begin{bookbox}{``Make Technical Diagrams Match Maintainable Systems''}
This tactical book improves the quality of architecture and module diagrams underlying AppSec tooling and integrations. It ensures technical implementations remain maintainable.

\vspace{0.5cm}

\textbf{Key Concepts to Extract:}
\begin{itemize}[nosep]
    \item Single Responsibility Principle
    \item Dependency management and injection
    \item Interface design and duck typing
    \item Composition over inheritance
    \item Managing change through good design
\end{itemize}
\end{bookbox}

\subsubsection{Diagrams to Produce}

\begin{diagrambox}
\textbf{Technical Architecture Artifacts}

\begin{description}[leftmargin=1.5cm, style=nextline]
    \item[Module Decomposition for AppSec Services/Tooling]
    Break down AppSec technical components into well-defined modules with clear responsibilities: Scanner integration modules, Finding normalization, Deduplication engine, Ticket sync adapters, Reporting aggregators, API interfaces. Each module should have a single responsibility.
    
    \item[Integration/Adapter Diagrams]
    Document boundaries between AppSec systems and external tools: Scanner adapters (abstract different SAST/SCA tools), CI adapters (GitHub Actions, GitLab CI, Jenkins), Ticketing adapters (Jira, ServiceNow), Reporting adapters (dashboards, exports). Show how adapters isolate change.
\end{description}
\end{diagrambox}

\subsubsection{Primary Stakeholders Served}

\begin{stakeholderbox}
\begin{description}[nosep, leftmargin=2.5cm]
    \item[AppSec Engineers] Clear technical architecture for implementation
    \item[Platform Teams] Integration patterns and expectations
\end{description}
\end{stakeholderbox}

\subsubsection{Module Responsibility Matrix}

\begin{center}
\renewcommand{\arraystretch}{1.3}
\footnotesize
\begin{tabular}{p{3.5cm}p{5cm}p{4cm}}
\toprule
\textbf{Module} & \textbf{Responsibility} & \textbf{Dependencies} \\
\midrule
Scanner Adapter & Normalize findings from various tools & Finding Schema \\
Deduplication Engine & Identify and merge duplicate findings & Finding Store \\
Enrichment Service & Add context (asset, owner, criticality) & CMDB Adapter \\
Ticket Sync & Bidirectional sync with ticketing systems & Ticketing Adapter \\
Reporting Aggregator & Compile metrics and generate reports & Finding Store, Config \\
\bottomrule
\end{tabular}
\end{center}

\newpage

% ============================================================================
% STAKEHOLDER VALUE ALIGNMENT
% ============================================================================
\section{Stakeholder-to-Book Value Alignment}

This section provides a quick reference for mapping stakeholder needs to the reading sequence.

\subsection{Alignment Matrix}

\begin{center}
\renewcommand{\arraystretch}{1.5}
\begin{tabular}{p{3.5cm}p{9cm}}
\toprule
\textbf{Stakeholder Group} & \textbf{Priority Reading Path} \\
\midrule
\textbf{Executives / BOD} & 
\booktitle{Thinking in Systems} $\rightarrow$ \booktitle{Traction} $\rightarrow$ \booktitle{Scaling Up}

\textit{Focus: Risk narrative, governance structures, measurable outcomes} \\
\midrule
\textbf{Engineering Leadership} & 
\booktitle{Work the System} $\rightarrow$ \booktitle{Modern Software Engineering} $\rightarrow$ \booktitle{Service Design Doing}

\textit{Focus: Flow efficiency, bottleneck reduction, clear ownership} \\
\midrule
\textbf{Developers} & 
\booktitle{Service Design Doing} $\rightarrow$ \booktitle{Work the System}

\textit{Focus: Clear intake process, predictable expectations, reduced toil} \\
\midrule
\textbf{GRC / Audit} & 
\booktitle{Thinking in Systems} $\rightarrow$ \booktitle{Work the System}

\textit{Focus: Controls as process, evidence flows, defensible repeatability} \\
\midrule
\textbf{AppSec Team} & 
Full sequence in order

\textit{Focus: Complete progression from ``why'' to ``how'' to ``scale''} \\
\bottomrule
\end{tabular}
\end{center}

\subsection{Diagram-to-Stakeholder Mapping}

\begin{center}
\renewcommand{\arraystretch}{1.3}
\footnotesize
\begin{tabular}{p{4cm}ccccc}
\toprule
\textbf{Diagram Type} & \textbf{Exec} & \textbf{Eng Lead} & \textbf{Dev} & \textbf{GRC} & \textbf{AppSec} \\
\midrule
System Context & $\bullet$ & $\bullet$ & & $\bullet$ & $\bullet$ \\
Causal Loop & $\bullet$ & $\bullet$ & & $\bullet$ & $\bullet$ \\
Service Catalog & $\bullet$ & $\bullet$ & $\bullet$ & $\bullet$ & $\bullet$ \\
BPMN Workflows & & $\bullet$ & $\bullet$ & $\bullet$ & $\bullet$ \\
Service Blueprints & & $\bullet$ & $\bullet$ & & $\bullet$ \\
Value Stream Maps & & $\bullet$ & & $\bullet$ & $\bullet$ \\
Accountability Chart & $\bullet$ & $\bullet$ & & $\bullet$ & $\bullet$ \\
Scorecard & $\bullet$ & $\bullet$ & & $\bullet$ & $\bullet$ \\
KPI Tree & $\bullet$ & $\bullet$ & & $\bullet$ & $\bullet$ \\
CI/CD Gates & & $\bullet$ & $\bullet$ & & $\bullet$ \\
Toolchain Data-Flow & & $\bullet$ & & & $\bullet$ \\
Module Decomposition & & & & & $\bullet$ \\
\bottomrule
\end{tabular}
\end{center}

\newpage

% ============================================================================
% IMPLEMENTATION GUIDANCE
% ============================================================================
\section{Implementation Guidance}

\subsection{Execution Timeline}

\begin{infobox}[Recommended Pacing]
\begin{description}[leftmargin=2cm]
    \item[Weeks 1--2] Read \booktitle{Thinking in Systems}; produce system context and causal loop diagrams
    \item[Weeks 3--4] Read \booktitle{The E-Myth Revisited}; produce service catalog draft
    \item[Weeks 5--6] Read \booktitle{Work the System}; produce core BPMN workflows with RACI
    \item[Weeks 7--8] Read \booktitle{Service Design Doing}; produce service blueprints and journey maps
    \item[Weeks 9--10] Read \booktitle{Traction}; produce accountability chart and scorecard
    \item[Weeks 11--12] Read \booktitle{Scaling Up}; produce KPI tree and operating rhythm
    \item[Weeks 13--14] Read \booktitle{Modern Software Engineering}; produce CI/CD gate diagrams
    \item[Weeks 15--16] Read \booktitle{The Mythical Man-Month}; produce rollout plan
    \item[Weeks 17--18] Read \booktitle{Practical OO Design}; produce module decomposition
\end{description}
\end{infobox}

\subsection{Success Criteria}

Each phase should produce tangible, reviewed artifacts:

\begin{center}
\renewcommand{\arraystretch}{1.3}
\begin{tabular}{lp{8cm}}
\toprule
\textbf{Phase} & \textbf{Exit Criteria} \\
\midrule
Foundation & System context approved by leadership; service catalog published; core BPMN workflows documented \\
Design \& Governance & Service blueprints validated with developers; scorecard in weekly use; operating rhythm established \\
Implementation & CI/CD gates deployed to pilot; rollout plan approved; technical architecture documented \\
\bottomrule
\end{tabular}
\end{center}

\subsection{Common Pitfalls to Avoid}

\begin{itemize}[nosep]
    \item \textbf{Diagram without purpose:} Every diagram should serve a specific stakeholder concern
    \item \textbf{Perfect before publish:} Start with draft diagrams and iterate based on feedback
    \item \textbf{Tools before process:} Document the process before automating it
    \item \textbf{Big bang rollout:} Phase implementations to gather feedback and adjust
    \item \textbf{Ignoring tribal knowledge:} Interview practitioners to capture actual workflows
\end{itemize}

\newpage

% ============================================================================
% APPENDIX: COMPLETE DIAGRAM INVENTORY
% ============================================================================
\section*{Appendix A: Complete Diagram Inventory}
\addcontentsline{toc}{section}{Appendix A: Complete Diagram Inventory}

\begin{center}
\renewcommand{\arraystretch}{1.3}
\footnotesize
\begin{longtable}{clll}
\toprule
\textbf{\#} & \textbf{Diagram} & \textbf{Source Book} & \textbf{Type} \\
\midrule
\endfirsthead
\toprule
\textbf{\#} & \textbf{Diagram} & \textbf{Source Book} & \textbf{Type} \\
\midrule
\endhead
1 & System Context + Boundary & Thinking in Systems & C4 L1 \\
2 & Causal Loop Diagrams & Thinking in Systems & Systems \\
3 & Capability Map & Thinking in Systems & Capability \\
4 & AppSec Service Catalog & E-Myth Revisited & Catalog \\
5 & High-Level Operating Model & E-Myth Revisited & Operating Model \\
6 & Intake BPMN & Work the System & BPMN \\
7 & Threat Modeling BPMN & Work the System & BPMN \\
8 & Vulnerability Mgmt BPMN & Work the System & BPMN \\
9 & Exception Process BPMN & Work the System & BPMN \\
10 & RACI Overlays & Work the System & RACI \\
11 & Service Blueprints & Service Design Doing & Blueprint \\
12 & Customer Journey Maps & Service Design Doing & Journey Map \\
13 & Value Stream Maps & Service Design Doing & VSM \\
14 & Accountability Chart & Traction & Org \\
15 & Scorecard Diagram & Traction & Metrics \\
16 & Meeting/Decision Cadence & Traction & Cadence \\
17 & Operating Rhythm Map & Scaling Up & Cadence \\
18 & KPI Tree & Scaling Up & Metrics \\
19 & Org Interface Map & Scaling Up & Integration \\
20 & Secure SDLC Value Stream & Modern Software Eng & VSM \\
21 & CI/CD Security Gates & Modern Software Eng & Pipeline \\
22 & Toolchain Data-Flow & Modern Software Eng & DFD \\
23 & Communication/Handoff Map & Mythical Man-Month & Communication \\
24 & Program Rollout Plan & Mythical Man-Month & Rollout \\
25 & Module Decomposition & Practical OO Design & Architecture \\
26 & Integration/Adapter Diagrams & Practical OO Design & Architecture \\
\bottomrule
\end{longtable}
\end{center}

\newpage

% ============================================================================
% APPENDIX: BOOK REFERENCE DETAILS
% ============================================================================
\section*{Appendix B: Book Reference Details}
\addcontentsline{toc}{section}{Appendix B: Book Reference Details}

\begin{center}
\renewcommand{\arraystretch}{1.4}
\begin{tabular}{p{5cm}p{3.5cm}p{4.5cm}}
\toprule
\textbf{Title} & \textbf{Author} & \textbf{Key Application} \\
\midrule
Thinking in Systems: A Primer & Donella H. Meadows & System dynamics, feedback loops \\
The E-Myth Revisited & Michael E. Gerber & Standardization, service mindset \\
Work the System & Sam Carpenter & Process documentation \\
Service Design Doing & Stickdorn et al. & Service optimization \\
Traction & Gino Wickman & Operating cadence, EOS \\
Scaling Up & Verne Harnish & Scalable execution \\
Modern Software Engineering & Dave Farley & DevSecOps integration \\
The Mythical Man-Month & Frederick P. Brooks Jr. & Coordination management \\
Practical Object-Oriented Design & Sandi Metz & Technical architecture \\
\bottomrule
\end{tabular}
\end{center}

\vspace{2cm}

\begin{highlightbox}[Next Steps]
This reading sequence and diagram backlog can be converted into a \textbf{Views-and-Beyond style architecture documentation package} with:
\begin{itemize}[nosep]
    \item Epics organized by process area (intake, threat modeling, vulnerability management, exceptions, CI/CD gates, reporting)
    \item Diagram deliverables as user stories with acceptance criteria
    \item Stakeholder concerns mapped to each artifact
    \item Dependencies and execution order
\end{itemize}

See the companion document: \textit{AppSec Architecture Documentation Package --- Views-and-Beyond Style Diagram Backlog} for the complete ticketable backlog.
\end{highlightbox}

\end{document}
